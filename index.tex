% Options for packages loaded elsewhere
% Options for packages loaded elsewhere
\PassOptionsToPackage{unicode}{hyperref}
\PassOptionsToPackage{hyphens}{url}
\PassOptionsToPackage{dvipsnames,svgnames,x11names}{xcolor}
%
\documentclass[
  letterpaper,
  DIV=11,
  numbers=noendperiod]{scrreprt}
\usepackage{xcolor}
\usepackage{amsmath,amssymb}
\setcounter{secnumdepth}{5}
\usepackage{iftex}
\ifPDFTeX
  \usepackage[T1]{fontenc}
  \usepackage[utf8]{inputenc}
  \usepackage{textcomp} % provide euro and other symbols
\else % if luatex or xetex
  \usepackage{unicode-math} % this also loads fontspec
  \defaultfontfeatures{Scale=MatchLowercase}
  \defaultfontfeatures[\rmfamily]{Ligatures=TeX,Scale=1}
\fi
\usepackage{lmodern}
\ifPDFTeX\else
  % xetex/luatex font selection
\fi
% Use upquote if available, for straight quotes in verbatim environments
\IfFileExists{upquote.sty}{\usepackage{upquote}}{}
\IfFileExists{microtype.sty}{% use microtype if available
  \usepackage[]{microtype}
  \UseMicrotypeSet[protrusion]{basicmath} % disable protrusion for tt fonts
}{}
\makeatletter
\@ifundefined{KOMAClassName}{% if non-KOMA class
  \IfFileExists{parskip.sty}{%
    \usepackage{parskip}
  }{% else
    \setlength{\parindent}{0pt}
    \setlength{\parskip}{6pt plus 2pt minus 1pt}}
}{% if KOMA class
  \KOMAoptions{parskip=half}}
\makeatother
% Make \paragraph and \subparagraph free-standing
\makeatletter
\ifx\paragraph\undefined\else
  \let\oldparagraph\paragraph
  \renewcommand{\paragraph}{
    \@ifstar
      \xxxParagraphStar
      \xxxParagraphNoStar
  }
  \newcommand{\xxxParagraphStar}[1]{\oldparagraph*{#1}\mbox{}}
  \newcommand{\xxxParagraphNoStar}[1]{\oldparagraph{#1}\mbox{}}
\fi
\ifx\subparagraph\undefined\else
  \let\oldsubparagraph\subparagraph
  \renewcommand{\subparagraph}{
    \@ifstar
      \xxxSubParagraphStar
      \xxxSubParagraphNoStar
  }
  \newcommand{\xxxSubParagraphStar}[1]{\oldsubparagraph*{#1}\mbox{}}
  \newcommand{\xxxSubParagraphNoStar}[1]{\oldsubparagraph{#1}\mbox{}}
\fi
\makeatother

\usepackage{color}
\usepackage{fancyvrb}
\newcommand{\VerbBar}{|}
\newcommand{\VERB}{\Verb[commandchars=\\\{\}]}
\DefineVerbatimEnvironment{Highlighting}{Verbatim}{commandchars=\\\{\}}
% Add ',fontsize=\small' for more characters per line
\usepackage{framed}
\definecolor{shadecolor}{RGB}{241,243,245}
\newenvironment{Shaded}{\begin{snugshade}}{\end{snugshade}}
\newcommand{\AlertTok}[1]{\textcolor[rgb]{0.68,0.00,0.00}{#1}}
\newcommand{\AnnotationTok}[1]{\textcolor[rgb]{0.37,0.37,0.37}{#1}}
\newcommand{\AttributeTok}[1]{\textcolor[rgb]{0.40,0.45,0.13}{#1}}
\newcommand{\BaseNTok}[1]{\textcolor[rgb]{0.68,0.00,0.00}{#1}}
\newcommand{\BuiltInTok}[1]{\textcolor[rgb]{0.00,0.23,0.31}{#1}}
\newcommand{\CharTok}[1]{\textcolor[rgb]{0.13,0.47,0.30}{#1}}
\newcommand{\CommentTok}[1]{\textcolor[rgb]{0.37,0.37,0.37}{#1}}
\newcommand{\CommentVarTok}[1]{\textcolor[rgb]{0.37,0.37,0.37}{\textit{#1}}}
\newcommand{\ConstantTok}[1]{\textcolor[rgb]{0.56,0.35,0.01}{#1}}
\newcommand{\ControlFlowTok}[1]{\textcolor[rgb]{0.00,0.23,0.31}{\textbf{#1}}}
\newcommand{\DataTypeTok}[1]{\textcolor[rgb]{0.68,0.00,0.00}{#1}}
\newcommand{\DecValTok}[1]{\textcolor[rgb]{0.68,0.00,0.00}{#1}}
\newcommand{\DocumentationTok}[1]{\textcolor[rgb]{0.37,0.37,0.37}{\textit{#1}}}
\newcommand{\ErrorTok}[1]{\textcolor[rgb]{0.68,0.00,0.00}{#1}}
\newcommand{\ExtensionTok}[1]{\textcolor[rgb]{0.00,0.23,0.31}{#1}}
\newcommand{\FloatTok}[1]{\textcolor[rgb]{0.68,0.00,0.00}{#1}}
\newcommand{\FunctionTok}[1]{\textcolor[rgb]{0.28,0.35,0.67}{#1}}
\newcommand{\ImportTok}[1]{\textcolor[rgb]{0.00,0.46,0.62}{#1}}
\newcommand{\InformationTok}[1]{\textcolor[rgb]{0.37,0.37,0.37}{#1}}
\newcommand{\KeywordTok}[1]{\textcolor[rgb]{0.00,0.23,0.31}{\textbf{#1}}}
\newcommand{\NormalTok}[1]{\textcolor[rgb]{0.00,0.23,0.31}{#1}}
\newcommand{\OperatorTok}[1]{\textcolor[rgb]{0.37,0.37,0.37}{#1}}
\newcommand{\OtherTok}[1]{\textcolor[rgb]{0.00,0.23,0.31}{#1}}
\newcommand{\PreprocessorTok}[1]{\textcolor[rgb]{0.68,0.00,0.00}{#1}}
\newcommand{\RegionMarkerTok}[1]{\textcolor[rgb]{0.00,0.23,0.31}{#1}}
\newcommand{\SpecialCharTok}[1]{\textcolor[rgb]{0.37,0.37,0.37}{#1}}
\newcommand{\SpecialStringTok}[1]{\textcolor[rgb]{0.13,0.47,0.30}{#1}}
\newcommand{\StringTok}[1]{\textcolor[rgb]{0.13,0.47,0.30}{#1}}
\newcommand{\VariableTok}[1]{\textcolor[rgb]{0.07,0.07,0.07}{#1}}
\newcommand{\VerbatimStringTok}[1]{\textcolor[rgb]{0.13,0.47,0.30}{#1}}
\newcommand{\WarningTok}[1]{\textcolor[rgb]{0.37,0.37,0.37}{\textit{#1}}}

\usepackage{longtable,booktabs,array}
\usepackage{calc} % for calculating minipage widths
% Correct order of tables after \paragraph or \subparagraph
\usepackage{etoolbox}
\makeatletter
\patchcmd\longtable{\par}{\if@noskipsec\mbox{}\fi\par}{}{}
\makeatother
% Allow footnotes in longtable head/foot
\IfFileExists{footnotehyper.sty}{\usepackage{footnotehyper}}{\usepackage{footnote}}
\makesavenoteenv{longtable}
\usepackage{graphicx}
\makeatletter
\newsavebox\pandoc@box
\newcommand*\pandocbounded[1]{% scales image to fit in text height/width
  \sbox\pandoc@box{#1}%
  \Gscale@div\@tempa{\textheight}{\dimexpr\ht\pandoc@box+\dp\pandoc@box\relax}%
  \Gscale@div\@tempb{\linewidth}{\wd\pandoc@box}%
  \ifdim\@tempb\p@<\@tempa\p@\let\@tempa\@tempb\fi% select the smaller of both
  \ifdim\@tempa\p@<\p@\scalebox{\@tempa}{\usebox\pandoc@box}%
  \else\usebox{\pandoc@box}%
  \fi%
}
% Set default figure placement to htbp
\def\fps@figure{htbp}
\makeatother





\setlength{\emergencystretch}{3em} % prevent overfull lines

\providecommand{\tightlist}{%
  \setlength{\itemsep}{0pt}\setlength{\parskip}{0pt}}



 


\KOMAoption{captions}{tableheading}
\makeatletter
\@ifpackageloaded{bookmark}{}{\usepackage{bookmark}}
\makeatother
\makeatletter
\@ifpackageloaded{caption}{}{\usepackage{caption}}
\AtBeginDocument{%
\ifdefined\contentsname
  \renewcommand*\contentsname{Table of contents}
\else
  \newcommand\contentsname{Table of contents}
\fi
\ifdefined\listfigurename
  \renewcommand*\listfigurename{List of Figures}
\else
  \newcommand\listfigurename{List of Figures}
\fi
\ifdefined\listtablename
  \renewcommand*\listtablename{List of Tables}
\else
  \newcommand\listtablename{List of Tables}
\fi
\ifdefined\figurename
  \renewcommand*\figurename{Figure}
\else
  \newcommand\figurename{Figure}
\fi
\ifdefined\tablename
  \renewcommand*\tablename{Table}
\else
  \newcommand\tablename{Table}
\fi
}
\@ifpackageloaded{float}{}{\usepackage{float}}
\floatstyle{ruled}
\@ifundefined{c@chapter}{\newfloat{codelisting}{h}{lop}}{\newfloat{codelisting}{h}{lop}[chapter]}
\floatname{codelisting}{Listing}
\newcommand*\listoflistings{\listof{codelisting}{List of Listings}}
\makeatother
\makeatletter
\makeatother
\makeatletter
\@ifpackageloaded{caption}{}{\usepackage{caption}}
\@ifpackageloaded{subcaption}{}{\usepackage{subcaption}}
\makeatother
\usepackage{bookmark}
\IfFileExists{xurl.sty}{\usepackage{xurl}}{} % add URL line breaks if available
\urlstyle{same}
\hypersetup{
  pdftitle={Оценка водных биоресурсов при недостатке данных в среде R (для начинающих)},
  pdfauthor={Сергей Баканёв},
  colorlinks=true,
  linkcolor={blue},
  filecolor={Maroon},
  citecolor={Blue},
  urlcolor={Blue},
  pdfcreator={LaTeX via pandoc}}


\title{Оценка водных биоресурсов при недостатке данных в среде R (для
начинающих)}
\author{Сергей Баканёв}
\date{2025-04-30}
\begin{document}
\maketitle

\renewcommand*\contentsname{Table of contents}
{
\hypersetup{linkcolor=}
\setcounter{tocdepth}{2}
\tableofcontents
}

\bookmarksetup{startatroot}

\chapter*{Аннотация}\label{ux430ux43dux43dux43eux442ux430ux446ux438ux44f}
\addcontentsline{toc}{chapter}{Аннотация}

\markboth{Аннотация}{Аннотация}

\textbf{Баканев С. В. (2025) Оценка водных биоресурсов при недостатке
данных в среде R (для начинающих). --- Курс лекций и практических
занятий, адрес доступа:~https://mombus.github.io/cRab}

Здесь будет краткая аннотация.

\bookmarksetup{startatroot}

\chapter{Введение}\label{ux432ux432ux435ux434ux435ux43dux438ux435}

Здесь будет введение

\bookmarksetup{startatroot}

\chapter{Анализ и визуализация данных
улова}\label{ux430ux43dux430ux43bux438ux437-ux438-ux432ux438ux437ux443ux430ux43bux438ux437ux430ux446ux438ux44f-ux434ux430ux43dux43dux44bux445-ux443ux43bux43eux432ux430}

\section{Введение (в
R)}\label{ux432ux432ux435ux434ux435ux43dux438ux435-ux432-r}

\section{Загрузка данных и первичный
осмотр}\label{ux437ux430ux433ux440ux443ux437ux43aux430-ux434ux430ux43dux43dux44bux445-ux438-ux43fux435ux440ux432ux438ux447ux43dux44bux439-ux43eux441ux43cux43eux442ux440}

ссылка на файл:
\href{https://mombus.github.io/cRab/data/shrimp_catch.csv}{shrimp\_catch.csv}

\begin{Shaded}
\begin{Highlighting}[]
\CommentTok{\# Установка рабочей директории}
\FunctionTok{setwd}\NormalTok{(}\StringTok{"C:/TEXTBOOK/"}\NormalTok{)}
\CommentTok{\# Загрузка библиотек}
\FunctionTok{library}\NormalTok{(tidyverse)}
\CommentTok{\# Загрузка данных}
\NormalTok{data }\OtherTok{\textless{}{-}} \FunctionTok{read\_csv}\NormalTok{(}\StringTok{"shrimp\_catch.csv"}\NormalTok{)}
\end{Highlighting}
\end{Shaded}

Команда \texttt{glimpse} знакомит со структурой данных:

\begin{Shaded}
\begin{Highlighting}[]
\CommentTok{\# Просмотр структуры и первых строк загруженных данных}
\FunctionTok{glimpse}\NormalTok{(data)}
\end{Highlighting}
\end{Shaded}

\begin{Shaded}
\begin{Highlighting}[]
\NormalTok{Rows}\SpecialCharTok{:} \DecValTok{230}
\NormalTok{Columns}\SpecialCharTok{:} \DecValTok{5}
\SpecialCharTok{$}\NormalTok{ id     }\SpecialCharTok{\textless{}}\NormalTok{int}\SpecialCharTok{\textgreater{}} \DecValTok{1}\NormalTok{, }\DecValTok{2}\NormalTok{, }\DecValTok{3}\NormalTok{, }\DecValTok{4}\NormalTok{, }\DecValTok{5}\NormalTok{, }\DecValTok{6}\NormalTok{, }\DecValTok{7}\NormalTok{, }\DecValTok{8}\NormalTok{, }\DecValTok{9}\NormalTok{, }\DecValTok{10}\NormalTok{, }\DecValTok{11}\NormalTok{, }\DecValTok{12}\NormalTok{, }\DecValTok{13}\NormalTok{, }\DecValTok{14}\NormalTok{, }\DecValTok{15}\NormalTok{, }\DecValTok{16}\NormalTok{, }\DecValTok{17}\NormalTok{, }\DecValTok{18}\NormalTok{, }\SpecialCharTok{\textasciitilde{}}
\ErrorTok{$}\NormalTok{ age    }\SpecialCharTok{\textless{}}\NormalTok{int}\SpecialCharTok{\textgreater{}} \DecValTok{2}\NormalTok{, }\DecValTok{4}\NormalTok{, }\DecValTok{4}\NormalTok{, }\DecValTok{4}\NormalTok{, }\DecValTok{1}\NormalTok{, }\DecValTok{4}\NormalTok{, }\DecValTok{2}\NormalTok{, }\DecValTok{2}\NormalTok{, }\DecValTok{4}\NormalTok{, }\DecValTok{3}\NormalTok{, }\DecValTok{4}\NormalTok{, }\DecValTok{3}\NormalTok{, }\DecValTok{2}\NormalTok{, }\DecValTok{1}\NormalTok{, }\DecValTok{2}\NormalTok{, }\DecValTok{1}\NormalTok{, }\DecValTok{2}\NormalTok{, }\DecValTok{2}\NormalTok{, }\DecValTok{2}\NormalTok{, }\DecValTok{2}\NormalTok{, }\DecValTok{3}\NormalTok{, }\SpecialCharTok{\textasciitilde{}}
\ErrorTok{$}\NormalTok{ length }\SpecialCharTok{\textless{}}\NormalTok{dbl}\SpecialCharTok{\textgreater{}} \FloatTok{20.45450}\NormalTok{, }\FloatTok{25.88928}\NormalTok{, }\FloatTok{29.42257}\NormalTok{, }\FloatTok{30.68292}\NormalTok{, }\FloatTok{12.46059}\NormalTok{, }\FloatTok{28.52152}\NormalTok{, }\FloatTok{17.}\SpecialCharTok{\textasciitilde{}}
\ErrorTok{$}\NormalTok{ weight }\SpecialCharTok{\textless{}}\NormalTok{dbl}\SpecialCharTok{\textgreater{}} \FloatTok{1.28221748}\NormalTok{, }\FloatTok{1.97476899}\NormalTok{, }\FloatTok{2.65412595}\NormalTok{, }\FloatTok{3.44746476}\NormalTok{, }\FloatTok{0.13404801}\NormalTok{, }\FloatTok{2.3}\SpecialCharTok{\textasciitilde{}}
\ErrorTok{$}\NormalTok{ sex    }\SpecialCharTok{\textless{}}\NormalTok{chr}\SpecialCharTok{\textgreater{}} \StringTok{"M"}\NormalTok{, }\StringTok{"F"}\NormalTok{, }\StringTok{"F"}\NormalTok{, }\StringTok{"F"}\NormalTok{, }\StringTok{"M"}\NormalTok{, }\StringTok{"F"}\NormalTok{, }\StringTok{"M"}\NormalTok{, }\StringTok{"M"}\NormalTok{, }\StringTok{"F"}\NormalTok{, }\StringTok{"F"}\NormalTok{, }\StringTok{"F"}\NormalTok{, }\StringTok{"F"}\NormalTok{, }\StringTok{"M"}\SpecialCharTok{\textasciitilde{}}
\ErrorTok{\textgreater{}} 
\end{Highlighting}
\end{Shaded}

Можно использовать команду \texttt{str} --- показывает внутреннюю
\textbf{структуру} объекта , включая количество строк, столбцов,
названия переменных, их типы (\texttt{chr}, \texttt{num}, \texttt{int} и
др.), а также несколько первых значений.

\begin{Shaded}
\begin{Highlighting}[]
\FunctionTok{str}\NormalTok{(data)}
\end{Highlighting}
\end{Shaded}

\begin{Shaded}
\begin{Highlighting}[]
\StringTok{\textquotesingle{}data.frame\textquotesingle{}}\SpecialCharTok{:}   \DecValTok{230}\NormalTok{ obs. of  }\DecValTok{5}\NormalTok{ variables}\SpecialCharTok{:}
 \ErrorTok{$}\NormalTok{ id    }\SpecialCharTok{:}\NormalTok{ int  }\DecValTok{1} \DecValTok{2} \DecValTok{3} \DecValTok{4} \DecValTok{5} \DecValTok{6} \DecValTok{7} \DecValTok{8} \DecValTok{9} \DecValTok{10}\NormalTok{ ...}
 \SpecialCharTok{$}\NormalTok{ age   }\SpecialCharTok{:}\NormalTok{ int  }\DecValTok{2} \DecValTok{4} \DecValTok{4} \DecValTok{4} \DecValTok{1} \DecValTok{4} \DecValTok{2} \DecValTok{2} \DecValTok{4} \DecValTok{3}\NormalTok{ ...}
 \SpecialCharTok{$}\NormalTok{ length}\SpecialCharTok{:}\NormalTok{ num  }\FloatTok{20.5} \FloatTok{25.9} \FloatTok{29.4} \FloatTok{30.7} \FloatTok{12.5}\NormalTok{ ...}
 \SpecialCharTok{$}\NormalTok{ weight}\SpecialCharTok{:}\NormalTok{ num  }\FloatTok{1.282} \FloatTok{1.975} \FloatTok{2.654} \FloatTok{3.447} \FloatTok{0.134}\NormalTok{ ...}
 \SpecialCharTok{$}\NormalTok{ sex   }\SpecialCharTok{:}\NormalTok{ chr  }\StringTok{"M"} \StringTok{"F"} \StringTok{"F"} \StringTok{"F"}\NormalTok{ ...}
\SpecialCharTok{\textgreater{}}
\end{Highlighting}
\end{Shaded}

\section{Описательная статистика и
визуализация}\label{ux43eux43fux438ux441ux430ux442ux435ux43bux44cux43dux430ux44f-ux441ux442ux430ux442ux438ux441ux442ux438ux43aux430-ux438-ux432ux438ux437ux443ux430ux43bux438ux437ux430ux446ux438ux44f}

Команда \texttt{summary} выводит \textbf{описательную статистику} для
каждой числовой переменной: минимум, 1-й квартиль, медиана, среднее, 3-й
квартиль, максимум; для категориальных переменных --- частоты.

\begin{Shaded}
\begin{Highlighting}[]
\CommentTok{\# Общая статистика}
\FunctionTok{summary}\NormalTok{(data)}
\end{Highlighting}
\end{Shaded}

\begin{Shaded}
\begin{Highlighting}[]
\NormalTok{       id              age            length          weight       }
\NormalTok{ Min.   }\SpecialCharTok{:}  \FloatTok{1.00}\NormalTok{   Min.   }\SpecialCharTok{:}\FloatTok{1.000}\NormalTok{   Min.   }\SpecialCharTok{:} \FloatTok{7.65}\NormalTok{   Min.   }\SpecialCharTok{:{-}}\FloatTok{0.3334}  
 \DecValTok{1}\NormalTok{st Qu.}\SpecialCharTok{:} \FloatTok{58.25}   \DecValTok{1}\NormalTok{st Qu.}\SpecialCharTok{:}\FloatTok{2.000}   \DecValTok{1}\NormalTok{st Qu.}\SpecialCharTok{:}\FloatTok{17.62}   \DecValTok{1}\NormalTok{st Qu.}\SpecialCharTok{:} \FloatTok{0.6320}  
\NormalTok{ Median }\SpecialCharTok{:}\FloatTok{115.50}\NormalTok{   Median }\SpecialCharTok{:}\FloatTok{3.000}\NormalTok{   Median }\SpecialCharTok{:}\FloatTok{22.49}\NormalTok{   Median }\SpecialCharTok{:} \FloatTok{1.3660}  
\NormalTok{ Mean   }\SpecialCharTok{:}\FloatTok{115.50}\NormalTok{   Mean   }\SpecialCharTok{:}\FloatTok{2.509}\NormalTok{   Mean   }\SpecialCharTok{:}\FloatTok{21.68}\NormalTok{   Mean   }\SpecialCharTok{:} \FloatTok{1.4933}  
 \DecValTok{3}\NormalTok{rd Qu.}\SpecialCharTok{:}\FloatTok{172.75}   \DecValTok{3}\NormalTok{rd Qu.}\SpecialCharTok{:}\FloatTok{3.000}   \DecValTok{3}\NormalTok{rd Qu.}\SpecialCharTok{:}\FloatTok{26.03}   \DecValTok{3}\NormalTok{rd Qu.}\SpecialCharTok{:} \FloatTok{2.1148}  
\NormalTok{ Max.   }\SpecialCharTok{:}\FloatTok{230.00}\NormalTok{   Max.   }\SpecialCharTok{:}\FloatTok{4.000}\NormalTok{   Max.   }\SpecialCharTok{:}\FloatTok{36.02}\NormalTok{   Max.   }\SpecialCharTok{:} \FloatTok{5.1316}  
\NormalTok{     sex           }
\NormalTok{ Length}\SpecialCharTok{:}\DecValTok{230}        
\NormalTok{ Class }\SpecialCharTok{:}\NormalTok{character  }
\NormalTok{ Mode  }\SpecialCharTok{:}\NormalTok{character  }
\end{Highlighting}
\end{Shaded}

Простейшими командами можно вычислить, например, соотоношение полов или
корреляцию длина-вес.

\begin{Shaded}
\begin{Highlighting}[]
\CommentTok{\# Соотношение полов}
\FunctionTok{prop.table}\NormalTok{(}\FunctionTok{table}\NormalTok{(data}\SpecialCharTok{$}\NormalTok{sex)) }\SpecialCharTok{\%\textgreater{}\%} \FunctionTok{round}\NormalTok{(}\DecValTok{2}\NormalTok{)}
\end{Highlighting}
\end{Shaded}

\begin{Shaded}
\begin{Highlighting}[]
\NormalTok{   F    M }
\FloatTok{0.35} \FloatTok{0.65} 
\end{Highlighting}
\end{Shaded}

\begin{Shaded}
\begin{Highlighting}[]
\CommentTok{\# Корреляция длина{-}вес с p{-}value}
\NormalTok{cor\_test }\OtherTok{\textless{}{-}} \FunctionTok{cor.test}\NormalTok{(data}\SpecialCharTok{$}\NormalTok{length, data}\SpecialCharTok{$}\NormalTok{weight, }
                     \AttributeTok{method =} \StringTok{"pearson"}\NormalTok{, }
                     \AttributeTok{exact =} \ConstantTok{FALSE}\NormalTok{,}
                     \AttributeTok{na.action =}\NormalTok{ na.omit)}
 
\NormalTok{cor\_coef }\OtherTok{\textless{}{-}} \FunctionTok{round}\NormalTok{(cor\_test}\SpecialCharTok{$}\NormalTok{estimate, }\DecValTok{2}\NormalTok{)}
\NormalTok{p\_value }\OtherTok{\textless{}{-}}\NormalTok{ scales}\SpecialCharTok{::}\FunctionTok{pvalue}\NormalTok{(cor\_test}\SpecialCharTok{$}\NormalTok{p.value, }\AttributeTok{accuracy =}\NormalTok{ .}\DecValTok{001}\NormalTok{)}
 
\FunctionTok{cat}\NormalTok{(}\StringTok{"Корреляция Пирсона: r ="}\NormalTok{, cor\_coef, }\StringTok{", p ="}\NormalTok{, p\_value, }\StringTok{"}\SpecialCharTok{\textbackslash{}n}\StringTok{"}\NormalTok{)}
\end{Highlighting}
\end{Shaded}

\begin{Shaded}
\begin{Highlighting}[]
\NormalTok{Корреляция Пирсона}\SpecialCharTok{:}\NormalTok{ r }\OtherTok{=} \FloatTok{0.95}\NormalTok{ , p }\OtherTok{=} \ErrorTok{\textless{}}\FloatTok{0.001} 
\end{Highlighting}
\end{Shaded}

\begin{Shaded}
\begin{Highlighting}[]
\CommentTok{\# Распределение возраста}
\FunctionTok{table}\NormalTok{(data}\SpecialCharTok{$}\NormalTok{age)}
\end{Highlighting}
\end{Shaded}

\begin{Shaded}
\begin{Highlighting}[]
\DecValTok{1}  \DecValTok{2}  \DecValTok{3}  \DecValTok{4} 
\DecValTok{43} \DecValTok{68} \DecValTok{77} \DecValTok{40} 
\end{Highlighting}
\end{Shaded}

\begin{Shaded}
\begin{Highlighting}[]
\CommentTok{\# Средние значения длины и веса по группам}
\NormalTok{data }\SpecialCharTok{\%\textgreater{}\%}
   \FunctionTok{group\_by}\NormalTok{(age) }\SpecialCharTok{\%\textgreater{}\%}
   \FunctionTok{summarise}\NormalTok{(}
     \AttributeTok{mean\_length =} \FunctionTok{mean}\NormalTok{(length),}
     \AttributeTok{sd\_length =} \FunctionTok{sd}\NormalTok{(length),}
     \AttributeTok{mean\_weight =} \FunctionTok{mean}\NormalTok{(weight),}
     \AttributeTok{sd\_weight =} \FunctionTok{sd}\NormalTok{(weight))}
\end{Highlighting}
\end{Shaded}

\begin{Shaded}
\begin{Highlighting}[]
\CommentTok{\# A tibble: 4 x 5}
\NormalTok{    age mean\_length sd\_length mean\_weight sd\_weight}
  \SpecialCharTok{\textless{}}\NormalTok{dbl}\SpecialCharTok{\textgreater{}}       \ErrorTok{\textless{}}\NormalTok{dbl}\SpecialCharTok{\textgreater{}}     \ErrorTok{\textless{}}\NormalTok{dbl}\SpecialCharTok{\textgreater{}}       \ErrorTok{\textless{}}\NormalTok{dbl}\SpecialCharTok{\textgreater{}}     \ErrorTok{\textless{}}\NormalTok{dbl}\SpecialCharTok{\textgreater{}}
\DecValTok{1}     \DecValTok{1}        \FloatTok{12.7}      \FloatTok{1.37}       \FloatTok{0.249}     \FloatTok{0.234}
\DecValTok{2}     \DecValTok{2}        \FloatTok{19.2}      \FloatTok{1.88}       \FloatTok{0.919}     \FloatTok{0.341}
\DecValTok{3}     \DecValTok{3}        \FloatTok{24.8}      \FloatTok{1.72}       \FloatTok{1.88}      \FloatTok{0.424}
\DecValTok{4}     \DecValTok{4}        \FloatTok{29.1}      \FloatTok{2.28}       \FloatTok{2.96}      \FloatTok{0.804}
\SpecialCharTok{\textgreater{}} 
\end{Highlighting}
\end{Shaded}

\subsection{Построение гистограммы для переменной `length' (длина
креветок)}\label{ux43fux43eux441ux442ux440ux43eux435ux43dux438ux435-ux433ux438ux441ux442ux43eux433ux440ux430ux43cux43cux44b-ux434ux43bux44f-ux43fux435ux440ux435ux43cux435ux43dux43dux43eux439-length-ux434ux43bux438ux43dux430-ux43aux440ux435ux432ux435ux442ux43eux43a}

Для первого визуального знакомства команда \texttt{hist} строит
гистограмму --- простой график, который показывает, как распределены
значения числовой переменной. В данном случае отображается распределение
длин креветок из набора данных.

\begin{Shaded}
\begin{Highlighting}[]
\FunctionTok{hist}\NormalTok{(data}\SpecialCharTok{$}\NormalTok{length, }
     \AttributeTok{main =} \StringTok{"Гистограмма длины креветок"}\NormalTok{,          }\CommentTok{\# Заголовок графика}
     \AttributeTok{xlab =} \StringTok{"Длина (см)"}\NormalTok{,                          }\CommentTok{\# Подпись оси X}
     \AttributeTok{ylab =} \StringTok{"Частота"}\NormalTok{,                             }\CommentTok{\# Подпись оси Y}
     \AttributeTok{col =} \StringTok{"lightblue"}\NormalTok{,                            }\CommentTok{\# Цвет столбцов}
     \AttributeTok{border =} \StringTok{"black"}\NormalTok{,                             }\CommentTok{\# Цвет границ столбцов}
     \AttributeTok{breaks =} \DecValTok{15}\NormalTok{)                                   }\CommentTok{\# Количество интервалов}
\end{Highlighting}
\end{Shaded}

\begin{figure}[H]

{\centering \includegraphics[width=0.6\linewidth,height=\textheight,keepaspectratio]{images/hist_shrimp.PNG}

}

\caption{Рис. 1.1: Гистограмма длины креветок}

\end{figure}%

\subsection{\texorpdfstring{Визуализация в
\texttt{ggridges}}{Визуализация в ggridges}}\label{ux432ux438ux437ux443ux430ux43bux438ux437ux430ux446ux438ux44f-ux432-ggridges}

Для элегантных и компактных графиков подходит библиотека
\texttt{ggridges}. Посторим распределение длины креветки в зависимости
от пола и возраста.

\begin{Shaded}
\begin{Highlighting}[]
\FunctionTok{library}\NormalTok{(ggplot2)}
\FunctionTok{library}\NormalTok{(ggridges)}

\FunctionTok{ggplot}\NormalTok{(data, }\FunctionTok{aes}\NormalTok{(}\AttributeTok{x =}\NormalTok{ length, }
                 \AttributeTok{y =}\NormalTok{ sex, }
                 \AttributeTok{group =}\NormalTok{ sex, }
                 \AttributeTok{fill =}\NormalTok{ sex)) }\SpecialCharTok{+}
  \FunctionTok{geom\_density\_ridges}\NormalTok{(}\AttributeTok{scale =} \DecValTok{2}\NormalTok{, }\AttributeTok{alpha =} \FloatTok{0.7}\NormalTok{) }\SpecialCharTok{+}
  \FunctionTok{scale\_y\_discrete}\NormalTok{(}\AttributeTok{expand =} \FunctionTok{c}\NormalTok{(}\DecValTok{0}\NormalTok{, }\DecValTok{0}\NormalTok{)) }\SpecialCharTok{+}
  \FunctionTok{scale\_x\_continuous}\NormalTok{(}\AttributeTok{expand =} \FunctionTok{c}\NormalTok{(}\DecValTok{0}\NormalTok{, }\DecValTok{0}\NormalTok{)) }\SpecialCharTok{+}
  \FunctionTok{labs}\NormalTok{(}
    \AttributeTok{title =} \StringTok{"Распределение длины карапакса по полу"}\NormalTok{,}
    \AttributeTok{x =} \StringTok{"Длина карапакса (мм)"}\NormalTok{,}
    \AttributeTok{y =} \StringTok{"Пол"}
\NormalTok{  ) }\SpecialCharTok{+}
  \FunctionTok{theme}\NormalTok{(}
    \AttributeTok{panel.border =} \FunctionTok{element\_blank}\NormalTok{(),  }\CommentTok{\# Убирает рамку вокруг графика}
    \AttributeTok{axis.line =} \FunctionTok{element\_line}\NormalTok{(}\AttributeTok{color =} \StringTok{"black"}\NormalTok{)  }\CommentTok{\# Сохраняет осевые линии (опционально)}
\NormalTok{  )}
\end{Highlighting}
\end{Shaded}

\begin{figure}[H]

{\centering \includegraphics[width=0.6\linewidth,height=\textheight,keepaspectratio]{images/ggridges_shrimp.PNG}

}

\caption{Рис. 1.2: Пол-длина креветок с использованием
\texttt{ggridges}}

\end{figure}%

\section{Выявление аутлайеров
(выбросов)}\label{ux432ux44bux44fux432ux43bux435ux43dux438ux435-ux430ux443ux442ux43bux430ux439ux435ux440ux43eux432-ux432ux44bux431ux440ux43eux441ux43eux432}

Аутлаеры (выбросы) --- наблюдения, значительно отклоняющиеся от общего
распределения данных. Их идентификация критически важна, так как они
могут искажать результаты анализа. Один из надёжных методов обнаружения
выбросов --- \textbf{метод межквартильного размаха (IQR)}.

\subsection{\texorpdfstring{\textbf{Теория
метода}}{Теория метода}}\label{ux442ux435ux43eux440ux438ux44f-ux43cux435ux442ux43eux434ux430}

\begin{enumerate}
\def\labelenumi{\arabic{enumi}.}
\item
  \textbf{Расчёт квартилей}:

  \begin{itemize}
  \item
    \textbf{Q1} (25-й перцентиль): значение, ниже которого находится
    25\% данных.
  \item
    \textbf{Q3} (75-й перцентиль): значение, ниже которого находится
    75\% данных.
  \item
    \textbf{IQR = Q3 - Q1}: мера разброса средней половины данных.
  \end{itemize}
\item
  \textbf{Границы аутлаеров}:

  \begin{itemize}
  \item
    \textbf{Нижняя граница}: Q1−1.5×IQRQ1−1.5×IQR
  \item
    \textbf{Верхняя граница}: Q3+1.5×IQRQ3+1.5×IQR\\
    Наблюдения за этими пределами считаются выбросами.
  \end{itemize}
\end{enumerate}

\subsection{\texorpdfstring{\textbf{Преимущества
метода}}{Преимущества метода}}\label{ux43fux440ux435ux438ux43cux443ux449ux435ux441ux442ux432ux430-ux43cux435ux442ux43eux434ux430}

\begin{itemize}
\item
  Устойчивость к асимметрии распределения.
\item
  Не требует предположения о нормальности данных.
\end{itemize}

\begin{Shaded}
\begin{Highlighting}[]
\CommentTok{\# Метод межквартильного размаха}
\NormalTok{outliers }\OtherTok{\textless{}{-}}\NormalTok{ data }\SpecialCharTok{\%\textgreater{}\%}
  \FunctionTok{mutate}\NormalTok{(}
    \AttributeTok{length\_z =} \FunctionTok{scale}\NormalTok{(length),}
    \AttributeTok{weight\_z =} \FunctionTok{scale}\NormalTok{(weight)}
\NormalTok{  ) }\SpecialCharTok{\%\textgreater{}\%} 
  \FunctionTok{filter}\NormalTok{(}\FunctionTok{abs}\NormalTok{(length\_z) }\SpecialCharTok{\textgreater{}} \DecValTok{3} \SpecialCharTok{|} \FunctionTok{abs}\NormalTok{(weight\_z) }\SpecialCharTok{\textgreater{}} \DecValTok{3}\NormalTok{)}

\CommentTok{\# Визуализация}
\FunctionTok{ggplot}\NormalTok{(data, }\FunctionTok{aes}\NormalTok{(}\AttributeTok{x =}\NormalTok{ length, }\AttributeTok{y =}\NormalTok{ weight)) }\SpecialCharTok{+}
  \FunctionTok{geom\_point}\NormalTok{(}\FunctionTok{aes}\NormalTok{(}\AttributeTok{color =} \StringTok{"Обычные"}\NormalTok{), }\AttributeTok{alpha =} \FloatTok{0.5}\NormalTok{) }\SpecialCharTok{+}
  \FunctionTok{geom\_point}\NormalTok{(}\AttributeTok{data =}\NormalTok{ outliers, }\FunctionTok{aes}\NormalTok{(}\AttributeTok{color =} \StringTok{"Аутлаеры"}\NormalTok{), }\AttributeTok{size =} \DecValTok{3}\NormalTok{) }\SpecialCharTok{+}
  \FunctionTok{scale\_color\_manual}\NormalTok{(}\AttributeTok{values =} \FunctionTok{c}\NormalTok{(}\StringTok{"Обычные"} \OtherTok{=} \StringTok{"grey50"}\NormalTok{, }\StringTok{"Аутлаеры"} \OtherTok{=} \StringTok{"red"}\NormalTok{)) }\SpecialCharTok{+}
  \FunctionTok{labs}\NormalTok{(}\AttributeTok{title =} \StringTok{"Выявление аномальных наблюдений"}\NormalTok{, }\AttributeTok{color =} \StringTok{"Тип"}\NormalTok{)}
\end{Highlighting}
\end{Shaded}

\begin{figure}[H]

{\centering \includegraphics[width=0.6\linewidth,height=\textheight,keepaspectratio]{images/outliers_shrimp.PNG}

}

\caption{Рис. 1.3: Распределение длины карапакса}

\end{figure}%

\section{Определение возрастной структуры: статистические методы анализа
размерных
данных}\label{ux43eux43fux440ux435ux434ux435ux43bux435ux43dux438ux435-ux432ux43eux437ux440ux430ux441ux442ux43dux43eux439-ux441ux442ux440ux443ux43aux442ux443ux440ux44b-ux441ux442ux430ux442ux438ux441ux442ux438ux447ux435ux441ux43aux438ux435-ux43cux435ux442ux43eux434ux44b-ux430ux43dux430ux43bux438ux437ux430-ux440ux430ux437ux43cux435ux440ux43dux44bux445-ux434ux430ux43dux43dux44bux445}

Возрастная структура популяции --- часто важна для расчёта промысловой
смертности, оценки репродуктивного потенциала и прогнозирования динамики
запасов. Поскольку прямое измерение возраста часто невозможно (например,
у беспозвоночных или рыб без четких возрастных меток), используются
статистические методы, выделяющие группы в смешанных распределениях
размеров.

\textbf{Основные подходы:}

\begin{enumerate}
\def\labelenumi{\arabic{enumi}.}
\item
  \textbf{Метод k-средних (k-means)} --- алгоритм кластеризации,
  группирующий особи в заданное число кластеров (возрастных групп) на
  основе их размеров.
\item
  \textbf{Метод Бхаттачарии} --- статистический подход для разделения
  смешанных нормальных распределений, часто применяемый для
  идентификации мод в гистограммах.
\item
  \textbf{EM-алгоритм} --- оценка параметров смеси распределений,
  подходящая для данных с перекрывающимися возрастными группами.
\item
  \textbf{Гауссовы смеси (GMM)} --- расширение метода Бхаттачарии для
  многомерного анализа.
\item
  \textbf{Ядерное сглаживание} --- непараметрический метод визуализации
  плотности, помогающий выявить скрытые моды.
\end{enumerate}

Рассмотрим метод k-средних (k-means) и метод Бхаттачарии, предварительно
построив гистограмму.

\begin{Shaded}
\begin{Highlighting}[]
\CommentTok{\# Загрузка библиотек}
\FunctionTok{library}\NormalTok{(tidyverse)}
\FunctionTok{library}\NormalTok{(mixtools)}
\CommentTok{\# Гистограмма длины с наложением плотности}
\FunctionTok{ggplot}\NormalTok{(data, }\FunctionTok{aes}\NormalTok{(}\AttributeTok{x =}\NormalTok{ length)) }\SpecialCharTok{+}
  \FunctionTok{geom\_histogram}\NormalTok{(}\FunctionTok{aes}\NormalTok{(}\AttributeTok{y =} \FunctionTok{after\_stat}\NormalTok{(density)), }\AttributeTok{fill =} \StringTok{"steelblue"}\NormalTok{, }\AttributeTok{bins =} \DecValTok{20}\NormalTok{, }\AttributeTok{alpha =} \FloatTok{0.7}\NormalTok{) }\SpecialCharTok{+}
  \FunctionTok{geom\_density}\NormalTok{(}\AttributeTok{color =} \StringTok{"\#FC4E07"}\NormalTok{, }\AttributeTok{linewidth =} \DecValTok{1}\NormalTok{) }\SpecialCharTok{+}
  \FunctionTok{labs}\NormalTok{(}\AttributeTok{title =} \StringTok{"Распределение длины карапакса"}\NormalTok{, }
       \AttributeTok{subtitle =} \StringTok{"Пики могут соответствовать возрастным группам"}\NormalTok{,}
       \AttributeTok{x =} \StringTok{"Длина (мм)"}\NormalTok{)}
\end{Highlighting}
\end{Shaded}

\begin{figure}[H]

{\centering \includegraphics[width=0.6\linewidth,height=\textheight,keepaspectratio]{images/hist_dens_shrimp.PNG}

}

\caption{Рис. 1.3: Распределение длины карапакса}

\end{figure}%

\begin{Shaded}
\begin{Highlighting}[]
\CommentTok{\# Кластеризация по длине (K{-}means как пример)}
\FunctionTok{set.seed}\NormalTok{(}\DecValTok{123}\NormalTok{)}
\NormalTok{clusters }\OtherTok{\textless{}{-}} \FunctionTok{kmeans}\NormalTok{(data}\SpecialCharTok{$}\NormalTok{length, }\AttributeTok{centers =} \DecValTok{4}\NormalTok{)  }\CommentTok{\# Предполагаем 4 возрастные группы}
\NormalTok{data}\SpecialCharTok{$}\NormalTok{cluster }\OtherTok{\textless{}{-}} \FunctionTok{factor}\NormalTok{(clusters}\SpecialCharTok{$}\NormalTok{cluster)}

\CommentTok{\# Визуализация кластеров}
\FunctionTok{ggplot}\NormalTok{(data, }\FunctionTok{aes}\NormalTok{(}\AttributeTok{x =}\NormalTok{ length, }\AttributeTok{fill =}\NormalTok{ cluster)) }\SpecialCharTok{+}
  \FunctionTok{geom\_histogram}\NormalTok{(}\AttributeTok{bins =} \DecValTok{25}\NormalTok{, }\AttributeTok{alpha =} \FloatTok{0.7}\NormalTok{) }\SpecialCharTok{+}
  \FunctionTok{labs}\NormalTok{(}\AttributeTok{title =} \StringTok{"Кластеризация по длине)"}\NormalTok{, }
       \AttributeTok{x =} \StringTok{"Длина (мм)"}\NormalTok{)}
\end{Highlighting}
\end{Shaded}

\begin{figure}[H]

{\centering \includegraphics[width=0.6\linewidth,height=\textheight,keepaspectratio]{images/cluster_shrimp.PNG}

}

\caption{Рис. 1.4: Кластеризация по длине}

\end{figure}%

\begin{Shaded}
\begin{Highlighting}[]
\CommentTok{\# Установка рабочей директории}
\FunctionTok{setwd}\NormalTok{(}\StringTok{"C:/TEXTBOOK/"}\NormalTok{)}

\CommentTok{\# Загрузка библиотек}
\FunctionTok{library}\NormalTok{(tidyverse)}
\FunctionTok{library}\NormalTok{(mixtools)}

\CommentTok{\# Загрузка данных}
\NormalTok{data }\OtherTok{\textless{}{-}} \FunctionTok{read.csv}\NormalTok{(}\StringTok{"shrimp\_catch.csv"}\NormalTok{)}

\CommentTok{\# 1. Построение и отображение гистограммы}
\FunctionTok{hist}\NormalTok{(data}\SpecialCharTok{$}\NormalTok{length, }\AttributeTok{breaks =} \DecValTok{20}\NormalTok{, }\AttributeTok{main =} \StringTok{"Гистограмма распределения длин карапаксов"}\NormalTok{,}
     \AttributeTok{xlab =} \StringTok{"Длина карапакса (мм)"}\NormalTok{, }\AttributeTok{ylab =} \StringTok{"Частота"}\NormalTok{)}

\CommentTok{\# 2. Инициализация параметров (предположим 4 возрастные группы)}
\NormalTok{init\_params }\OtherTok{\textless{}{-}} \FunctionTok{list}\NormalTok{(}
  \AttributeTok{lambda =} \FunctionTok{rep}\NormalTok{(}\DecValTok{1}\SpecialCharTok{/}\DecValTok{4}\NormalTok{, }\DecValTok{4}\NormalTok{),}
  \AttributeTok{mu =} \FunctionTok{c}\NormalTok{(}\DecValTok{13}\NormalTok{, }\DecValTok{19}\NormalTok{, }\DecValTok{25}\NormalTok{, }\DecValTok{32}\NormalTok{),}
  \AttributeTok{sigma =} \FunctionTok{c}\NormalTok{(}\FloatTok{1.5}\NormalTok{, }\FloatTok{1.75}\NormalTok{, }\FloatTok{1.75}\NormalTok{, }\FloatTok{2.5}\NormalTok{)}
\NormalTok{)}

\CommentTok{\# 3. Разделение смеси распределений методом EM}
\NormalTok{fit }\OtherTok{\textless{}{-}} \FunctionTok{normalmixEM}\NormalTok{(data}\SpecialCharTok{$}\NormalTok{length, }\AttributeTok{k =} \DecValTok{4}\NormalTok{, }\AttributeTok{maxit =} \DecValTok{1000}\NormalTok{, }\AttributeTok{epsilon =} \FloatTok{1e{-}3}\NormalTok{,}
                   \AttributeTok{lambda =}\NormalTok{ init\_params}\SpecialCharTok{$}\NormalTok{lambda,}
                   \AttributeTok{mu =}\NormalTok{ init\_params}\SpecialCharTok{$}\NormalTok{mu,}
                   \AttributeTok{sigma =}\NormalTok{ init\_params}\SpecialCharTok{$}\NormalTok{sigma)}

\CommentTok{\# 4. Визуализация результатов с ggplot2}
\CommentTok{\# Генерация сетки для построения кривых}
\NormalTok{x\_grid }\OtherTok{\textless{}{-}} \FunctionTok{seq}\NormalTok{(}\FunctionTok{min}\NormalTok{(data}\SpecialCharTok{$}\NormalTok{length), }\FunctionTok{max}\NormalTok{(data}\SpecialCharTok{$}\NormalTok{length), }\AttributeTok{length.out =} \DecValTok{500}\NormalTok{)}

\CommentTok{\# Функция смеси}
\NormalTok{mixture\_density }\OtherTok{\textless{}{-}} \ControlFlowTok{function}\NormalTok{(x) \{}
\NormalTok{  fit}\SpecialCharTok{$}\NormalTok{lambda[}\DecValTok{1}\NormalTok{] }\SpecialCharTok{*} \FunctionTok{dnorm}\NormalTok{(x, fit}\SpecialCharTok{$}\NormalTok{mu[}\DecValTok{1}\NormalTok{], fit}\SpecialCharTok{$}\NormalTok{sigma[}\DecValTok{1}\NormalTok{]) }\SpecialCharTok{+}
\NormalTok{  fit}\SpecialCharTok{$}\NormalTok{lambda[}\DecValTok{2}\NormalTok{] }\SpecialCharTok{*} \FunctionTok{dnorm}\NormalTok{(x, fit}\SpecialCharTok{$}\NormalTok{mu[}\DecValTok{2}\NormalTok{], fit}\SpecialCharTok{$}\NormalTok{sigma[}\DecValTok{2}\NormalTok{]) }\SpecialCharTok{+}
\NormalTok{  fit}\SpecialCharTok{$}\NormalTok{lambda[}\DecValTok{3}\NormalTok{] }\SpecialCharTok{*} \FunctionTok{dnorm}\NormalTok{(x, fit}\SpecialCharTok{$}\NormalTok{mu[}\DecValTok{3}\NormalTok{], fit}\SpecialCharTok{$}\NormalTok{sigma[}\DecValTok{3}\NormalTok{]) }\SpecialCharTok{+}
\NormalTok{  fit}\SpecialCharTok{$}\NormalTok{lambda[}\DecValTok{4}\NormalTok{] }\SpecialCharTok{*} \FunctionTok{dnorm}\NormalTok{(x, fit}\SpecialCharTok{$}\NormalTok{mu[}\DecValTok{4}\NormalTok{], fit}\SpecialCharTok{$}\NormalTok{sigma[}\DecValTok{4}\NormalTok{])}
\NormalTok{\}}

\CommentTok{\# График}
\FunctionTok{ggplot}\NormalTok{(data, }\FunctionTok{aes}\NormalTok{(}\AttributeTok{x =}\NormalTok{ length)) }\SpecialCharTok{+}
  \CommentTok{\# Гистограмма}
  \FunctionTok{geom\_histogram}\NormalTok{(}\FunctionTok{aes}\NormalTok{(}\AttributeTok{y =} \FunctionTok{after\_stat}\NormalTok{(density)), }\AttributeTok{bins =} \DecValTok{20}\NormalTok{, }\AttributeTok{fill =} \StringTok{"white"}\NormalTok{, }\AttributeTok{color =} \StringTok{"black"}\NormalTok{, }\AttributeTok{alpha =} \FloatTok{0.7}\NormalTok{) }\SpecialCharTok{+}
  \CommentTok{\# Исходное распределение (гладкая линия)}
  \FunctionTok{geom\_density}\NormalTok{(}\AttributeTok{color =} \StringTok{"red"}\NormalTok{, }\AttributeTok{lwd =} \FloatTok{1.2}\NormalTok{) }\SpecialCharTok{+}
  \CommentTok{\# Смесь распределений}
  \FunctionTok{stat\_function}\NormalTok{(}\AttributeTok{fun =}\NormalTok{ mixture\_density, }\AttributeTok{color =} \StringTok{"black"}\NormalTok{, }\AttributeTok{lwd =} \FloatTok{1.5}\NormalTok{) }\SpecialCharTok{+}
  \CommentTok{\# Компоненты смеси}
  \FunctionTok{stat\_function}\NormalTok{(}\AttributeTok{fun =} \ControlFlowTok{function}\NormalTok{(x) fit}\SpecialCharTok{$}\NormalTok{lambda[}\DecValTok{1}\NormalTok{] }\SpecialCharTok{*} \FunctionTok{dnorm}\NormalTok{(x, fit}\SpecialCharTok{$}\NormalTok{mu[}\DecValTok{1}\NormalTok{], fit}\SpecialCharTok{$}\NormalTok{sigma[}\DecValTok{1}\NormalTok{]), }\AttributeTok{color =} \StringTok{"blue"}\NormalTok{, }\AttributeTok{lwd =} \DecValTok{1}\NormalTok{) }\SpecialCharTok{+}
  \FunctionTok{stat\_function}\NormalTok{(}\AttributeTok{fun =} \ControlFlowTok{function}\NormalTok{(x) fit}\SpecialCharTok{$}\NormalTok{lambda[}\DecValTok{2}\NormalTok{] }\SpecialCharTok{*} \FunctionTok{dnorm}\NormalTok{(x, fit}\SpecialCharTok{$}\NormalTok{mu[}\DecValTok{2}\NormalTok{], fit}\SpecialCharTok{$}\NormalTok{sigma[}\DecValTok{2}\NormalTok{]), }\AttributeTok{color =} \StringTok{"green"}\NormalTok{, }\AttributeTok{lwd =} \DecValTok{1}\NormalTok{) }\SpecialCharTok{+}
  \FunctionTok{stat\_function}\NormalTok{(}\AttributeTok{fun =} \ControlFlowTok{function}\NormalTok{(x) fit}\SpecialCharTok{$}\NormalTok{lambda[}\DecValTok{3}\NormalTok{] }\SpecialCharTok{*} \FunctionTok{dnorm}\NormalTok{(x, fit}\SpecialCharTok{$}\NormalTok{mu[}\DecValTok{3}\NormalTok{], fit}\SpecialCharTok{$}\NormalTok{sigma[}\DecValTok{3}\NormalTok{]), }\AttributeTok{color =} \StringTok{"orange"}\NormalTok{, }\AttributeTok{lwd =} \DecValTok{1}\NormalTok{) }\SpecialCharTok{+}
  \FunctionTok{stat\_function}\NormalTok{(}\AttributeTok{fun =} \ControlFlowTok{function}\NormalTok{(x) fit}\SpecialCharTok{$}\NormalTok{lambda[}\DecValTok{4}\NormalTok{] }\SpecialCharTok{*} \FunctionTok{dnorm}\NormalTok{(x, fit}\SpecialCharTok{$}\NormalTok{mu[}\DecValTok{4}\NormalTok{], fit}\SpecialCharTok{$}\NormalTok{sigma[}\DecValTok{4}\NormalTok{]), }\AttributeTok{color =} \StringTok{"purple"}\NormalTok{, }\AttributeTok{lwd =} \DecValTok{1}\NormalTok{) }\SpecialCharTok{+}
  
  \CommentTok{\# Настройка темы и легенды}
  \FunctionTok{theme\_minimal}\NormalTok{() }\SpecialCharTok{+}
  \FunctionTok{labs}\NormalTok{(}
    \AttributeTok{x =} \StringTok{"Длина карапакса (мм)"}\NormalTok{,}
    \AttributeTok{y =} \StringTok{"Плотность"}\NormalTok{,}
    \AttributeTok{title =} \StringTok{"Разделение возрастных групп методом EM"}
\NormalTok{  )}
\end{Highlighting}
\end{Shaded}

\begin{figure}[H]

{\centering \includegraphics[width=0.6\linewidth,height=\textheight,keepaspectratio]{images/bhattacharya_shrimp.PNG}

}

\caption{Рис. 1.5: Метод Бхаттачарии}

\end{figure}%

\section{Уравнение
Берталанфи}\label{ux443ux440ux430ux432ux43dux435ux43dux438ux435-ux431ux435ux440ux442ux430ux43bux430ux43dux444ux438}

Уравнение Берталанфи --- фундаментальная модель в рыбохозяйственной
науке, описывающая асимптотический рост организмов. Оно имеет вид: \[
L(t) = L_{\infty} \cdot \left(1 - e^{-k \cdot (t - t_0)}\right)
\] где \emph{L\textsubscript{∞}}--- теоретическая максимальная длина
особи, \emph{k}--- коэффициент скорости роста,
\emph{t\textsubscript{0}}--- гипотетический возраст при нулевой длине.

В приведённом коде модель применяется для анализа роста северной
креветки :

\begin{enumerate}
\def\labelenumi{\arabic{enumi}.}
\item
  \textbf{Подготовка данных}: Удаление аутлаеров (например, строк 10 и
  50) повышает точность оценки параметров.
\item
  \textbf{Инициализация параметров}:

  \begin{itemize}
  \item
    \emph{L\textsubscript{∞}} задаётся как максимальная наблюдаемая
    длина в данных.
  \item
    \emph{k} и \emph{t\textsubscript{0}} подбираются итеративно методом
    нелинейных наименьших квадратов (\textbf{\texttt{nls}}).
  \end{itemize}
\item
  \textbf{Визуализация}: График сопоставляет эмпирические данные (точки)
  с предсказаниями модели (красная линия), демонстрируя, как рост
  замедляется с приближением к \emph{L∞}.
\end{enumerate}

\textbf{Интерпретация параметров}:

\begin{itemize}
\item
  Высокое значение \emph{k} (\textgreater0.3) указывает на быстрый рост
  молоди.
\item
  \emph{t\textsubscript{0}}\textless0 может отражать ранний метаморфоз
  личинок.
\end{itemize}

\begin{Shaded}
\begin{Highlighting}[]
\CommentTok{\# Загрузка библиотек}
\FunctionTok{library}\NormalTok{(ggplot2)}
\FunctionTok{library}\NormalTok{(dplyr)}
\FunctionTok{library}\NormalTok{(nlme)}

\CommentTok{\# Загрузка данных}
\NormalTok{data }\OtherTok{\textless{}{-}} \FunctionTok{read.csv}\NormalTok{(}\StringTok{"shrimp\_catch.csv"}\NormalTok{)}

\CommentTok{\# Преобразование возраста в числовой формат}
\NormalTok{data}\SpecialCharTok{$}\NormalTok{age\_num }\OtherTok{\textless{}{-}} \FunctionTok{as.numeric}\NormalTok{(data}\SpecialCharTok{$}\NormalTok{age)}

\CommentTok{\# Удаление аутлайеров (если необходимо)}
\NormalTok{data\_clean }\OtherTok{\textless{}{-}}\NormalTok{ data }\SpecialCharTok{\%\textgreater{}\%}
  \FunctionTok{filter}\NormalTok{(}\SpecialCharTok{!}\NormalTok{id }\SpecialCharTok{\%in\%} \FunctionTok{c}\NormalTok{(}\DecValTok{10}\NormalTok{, }\DecValTok{50}\NormalTok{))  }\CommentTok{\# Пример удаления строк с аномалиями}

\CommentTok{\# Начальные параметры на основе данных}
\NormalTok{L\_inf\_start }\OtherTok{\textless{}{-}} \FunctionTok{max}\NormalTok{(data\_clean}\SpecialCharTok{$}\NormalTok{length, }\AttributeTok{na.rm =} \ConstantTok{TRUE}\NormalTok{)  }\CommentTok{\# Максимальная длина}
\NormalTok{k\_start }\OtherTok{\textless{}{-}} \FloatTok{0.3}                                        \CommentTok{\# Средняя скорость роста}
\NormalTok{t0\_start }\OtherTok{\textless{}{-}} \SpecialCharTok{{-}}\FloatTok{0.5}                                      \CommentTok{\# Гипотетический возраст}

\CommentTok{\# Подгонка модели с увеличенным числом итераций}
\NormalTok{model }\OtherTok{\textless{}{-}} \FunctionTok{nls}\NormalTok{(}
\NormalTok{  length }\SpecialCharTok{\textasciitilde{}}\NormalTok{ L\_inf }\SpecialCharTok{*}\NormalTok{ (}\DecValTok{1} \SpecialCharTok{{-}} \FunctionTok{exp}\NormalTok{(}\SpecialCharTok{{-}}\NormalTok{k }\SpecialCharTok{*}\NormalTok{ (age\_num }\SpecialCharTok{{-}}\NormalTok{ t0))),}
  \AttributeTok{data =}\NormalTok{ data\_clean,}
  \AttributeTok{start =} \FunctionTok{list}\NormalTok{(}\AttributeTok{L\_inf =}\NormalTok{ L\_inf\_start, }\AttributeTok{k =}\NormalTok{ k\_start, }\AttributeTok{t0 =}\NormalTok{ t0\_start),}
  \AttributeTok{control =} \FunctionTok{nls.control}\NormalTok{(}\AttributeTok{maxiter =} \DecValTok{200}\NormalTok{, }\AttributeTok{warnOnly =} \ConstantTok{TRUE}\NormalTok{)  }\CommentTok{\# Увеличиваем лимит итераций}
\NormalTok{)}

\CommentTok{\# Вывод результатов}
\FunctionTok{summary}\NormalTok{(model)}

\CommentTok{\# Создание последовательности возрастов для предсказания}
\NormalTok{age\_seq }\OtherTok{\textless{}{-}} \FunctionTok{seq}\NormalTok{(}\FunctionTok{min}\NormalTok{(data\_clean}\SpecialCharTok{$}\NormalTok{age\_num), }\FunctionTok{max}\NormalTok{(data\_clean}\SpecialCharTok{$}\NormalTok{age\_num), }\AttributeTok{by =} \FloatTok{0.1}\NormalTok{)}

\CommentTok{\# Предсказание значений длины}
\NormalTok{length\_pred }\OtherTok{\textless{}{-}} \FunctionTok{predict}\NormalTok{(model, }\AttributeTok{newdata =} \FunctionTok{data.frame}\NormalTok{(}\AttributeTok{age\_num =}\NormalTok{ age\_seq))}

\CommentTok{\# Построение графика}
\FunctionTok{ggplot}\NormalTok{(data\_clean, }\FunctionTok{aes}\NormalTok{(}\AttributeTok{x =}\NormalTok{ age\_num, }\AttributeTok{y =}\NormalTok{ length)) }\SpecialCharTok{+}
  \FunctionTok{geom\_point}\NormalTok{(}\FunctionTok{aes}\NormalTok{(}\AttributeTok{color =}\NormalTok{ age), }\AttributeTok{alpha =} \FloatTok{0.7}\NormalTok{) }\SpecialCharTok{+}
  \FunctionTok{geom\_line}\NormalTok{(}\AttributeTok{data =} \FunctionTok{data.frame}\NormalTok{(}\AttributeTok{age\_num =}\NormalTok{ age\_seq, }\AttributeTok{length =}\NormalTok{ length\_pred), }
            \FunctionTok{aes}\NormalTok{(}\AttributeTok{x =}\NormalTok{ age\_num, }\AttributeTok{y =}\NormalTok{ length), }\AttributeTok{color =} \StringTok{"red"}\NormalTok{, }\AttributeTok{linewidth =} \FloatTok{1.2}\NormalTok{) }\SpecialCharTok{+}
  \FunctionTok{labs}\NormalTok{(}
    \AttributeTok{title =} \StringTok{"Рост креветок по уравнению Берталанфи"}\NormalTok{,}
    \AttributeTok{x =} \StringTok{"Возраст (годы)"}\NormalTok{,}
    \AttributeTok{y =} \StringTok{"Длина карапакса (мм)"}\NormalTok{,}
    \AttributeTok{color =} \StringTok{"Возрастная группа"}
\NormalTok{  ) }\SpecialCharTok{+}
  \FunctionTok{theme\_minimal}\NormalTok{()}

\CommentTok{\# Сохранение графика}
\FunctionTok{ggsave}\NormalTok{(}\StringTok{"bertalanffy\_model.png"}\NormalTok{, }\AttributeTok{width =} \DecValTok{8}\NormalTok{, }\AttributeTok{height =} \DecValTok{6}\NormalTok{)}
\end{Highlighting}
\end{Shaded}

\begin{figure}[H]

{\centering \includegraphics[width=0.6\linewidth,height=\textheight,keepaspectratio]{images/bertalanffy_model.PNG}

}

\caption{Рис. 1.6: Рост креветок по уравнению Берталанфи}

\end{figure}%

\section{Огива, логистическая кривая и 50\%-ное
созревание}\label{ux43eux433ux438ux432ux430-ux43bux43eux433ux438ux441ux442ux438ux447ux435ux441ux43aux430ux44f-ux43aux440ux438ux432ux430ux44f-ux438-50-ux43dux43eux435-ux441ux43eux437ux440ux435ux432ux430ux43dux438ux435}

Логистическая кривая --- ключевой инструмент для моделирования бинарных
процессов, таких как созревание или смена пола у организмов. В случае
протоандрических креветок (\emph{Pandalus borealis}), которые меняют пол
с возрастом, зависимость вероятности быть самкой от длины карапакса
можно описать логистической функцией:

\[
P(F) = \frac{1}{1 + e^{-(\beta_0 + \beta_1 \cdot длина)}}
\]

где \emph{P(F)} --- вероятность принадлежности к женскому полу,
\emph{β\textsubscript{0}} --- интерсепт, \emph{β\textsubscript{1}} ---
коэффициент влияния длины.

Точка перегиба логистической кривой соответствует длине, при которой
вероятность быть самкой равна 50\%: \[
L_{50} = -\frac{\beta_0}{\beta_1}
\]

\begin{figure}[H]

{\centering \includegraphics[width=0.6\linewidth,height=\textheight,keepaspectratio]{images/logistic_model_shrimp.PNG}

}

\caption{Рис. 1.7: Логистическая кривая}

\end{figure}%

Огива (кумулятивная кривая) показывает накопление вероятности с
увеличением длины. Для анализа созревания её можно построить через
интеграл логистической функции. Визуально она демонстрирует, как доля
самок возрастает с размером.

\begin{figure}[H]

{\centering \includegraphics[width=0.6\linewidth,height=\textheight,keepaspectratio]{images/ogive_shrimp.PNG}

}

\caption{Рис. 1.8: Огива}

\end{figure}%

\subsection{\texorpdfstring{\textbf{Оценка
модели}}{Оценка модели}}\label{ux43eux446ux435ux43dux43aux430-ux43cux43eux434ux435ux43bux438}

\begin{enumerate}
\def\labelenumi{\arabic{enumi}.}
\item
  \textbf{ROC-кривая и AUC}:

  \begin{itemize}
  \item
    Площадь под ROC-кривой (AUC) \textgreater0.7 указывает на хорошую
    предсказательную способность модели.
  \item
    Значение AUC = 0.94(пример из кода) подтверждает сильную связь длины
    и пола.
  \end{itemize}
\end{enumerate}

\begin{figure}[H]

{\centering \includegraphics[width=0.6\linewidth,height=\textheight,keepaspectratio]{images/ROC_shrimp.PNG}

}

\caption{Рис. 1.9: ROC-кривая и AUC}

\end{figure}%

\begin{enumerate}
\def\labelenumi{\arabic{enumi}.}
\setcounter{enumi}{1}
\item
  \textbf{Интерпретация коэффициентов}:

  \begin{itemize}
  \item
    Положительный \emph{β\textsubscript{1}} означает: с ростом длины
    вероятность быть самкой увеличивается.
  \item
    Например, \emph{β\textsubscript{1}}=0.25 → увеличение длины на 1 мм
    повышает шансы в e\textsuperscript{0.25}≈1.28 раза.
  \end{itemize}
\end{enumerate}

\subsection{\texorpdfstring{\textbf{Биологический
контекст}}{Биологический контекст}}\label{ux431ux438ux43eux43bux43eux433ux438ux447ux435ux441ux43aux438ux439-ux43aux43eux43dux442ux435ux43aux441ux442}

\begin{itemize}
\item
  \textbf{Протоандрический гермафродитизм}: У креветок смена пола с
  самцов на самок происходит при достижении критического размера
  (\textasciitilde25-28 мм).
\item
  \textbf{L50 как индикатор}: Снижение \emph{L\textsubscript{50}} в
  популяции может сигнализировать о стрессовых условиях (перелов,
  изменение среды), ускоряющих созревание.
\end{itemize}

\begin{Shaded}
\begin{Highlighting}[]
\CommentTok{\# Установка рабочей директории}
\FunctionTok{setwd}\NormalTok{(}\StringTok{"C:/TEXTBOOK/"}\NormalTok{)}

\CommentTok{\# Загрузка библиотек}
\FunctionTok{library}\NormalTok{(tidyverse)}
\FunctionTok{library}\NormalTok{(pROC)}
\FunctionTok{library}\NormalTok{(ggplot2)}

\CommentTok{\# Загрузка данных}
\NormalTok{data }\OtherTok{\textless{}{-}} \FunctionTok{read\_csv}\NormalTok{(}\StringTok{"shrimp\_catch.csv"}\NormalTok{)}

\CommentTok{\# 1. Предобработка данных {-}{-}{-}{-}{-}{-}{-}{-}{-}{-}{-}{-}{-}{-}{-}{-}{-}{-}{-}{-}{-}{-}{-}{-}{-}{-}{-}{-}{-}{-}{-}{-}{-}{-}{-}{-}{-}{-}{-}{-}{-}{-}{-}{-}{-}{-}{-}{-}{-}{-}{-}{-}{-}}
\CommentTok{\# Удаление аутлаеров методом IQR}
\NormalTok{Q1 }\OtherTok{\textless{}{-}} \FunctionTok{quantile}\NormalTok{(data}\SpecialCharTok{$}\NormalTok{length, }\FloatTok{0.25}\NormalTok{)}
\NormalTok{Q3 }\OtherTok{\textless{}{-}} \FunctionTok{quantile}\NormalTok{(data}\SpecialCharTok{$}\NormalTok{length, }\FloatTok{0.75}\NormalTok{)}
\NormalTok{IQR }\OtherTok{\textless{}{-}}\NormalTok{ Q3 }\SpecialCharTok{{-}}\NormalTok{ Q1}
\NormalTok{data\_clean }\OtherTok{\textless{}{-}}\NormalTok{ data }\SpecialCharTok{\%\textgreater{}\%}
  \FunctionTok{filter}\NormalTok{(length }\SpecialCharTok{\textgreater{}=}\NormalTok{ Q1 }\SpecialCharTok{{-}} \FloatTok{1.5}\SpecialCharTok{*}\NormalTok{IQR }\SpecialCharTok{\&}\NormalTok{ length }\SpecialCharTok{\textless{}=}\NormalTok{ Q3 }\SpecialCharTok{+} \FloatTok{1.5}\SpecialCharTok{*}\NormalTok{IQR)}

\CommentTok{\# 2. Логистическая регрессия {-}{-}{-}{-}{-}{-}{-}{-}{-}{-}{-}{-}{-}{-}{-}{-}{-}{-}{-}{-}{-}{-}{-}{-}{-}{-}{-}{-}{-}{-}{-}{-}{-}{-}{-}{-}{-}{-}{-}{-}{-}{-}{-}{-}{-}{-}{-}{-}{-}{-}}
\CommentTok{\# Преобразование пола в бинарную переменную}
\NormalTok{data\_clean}\SpecialCharTok{$}\NormalTok{sex\_binary }\OtherTok{\textless{}{-}} \FunctionTok{ifelse}\NormalTok{(data\_clean}\SpecialCharTok{$}\NormalTok{sex }\SpecialCharTok{==} \StringTok{"F"}\NormalTok{, }\DecValTok{1}\NormalTok{, }\DecValTok{0}\NormalTok{)}

\CommentTok{\# Подгонка модели}
\NormalTok{model\_logit }\OtherTok{\textless{}{-}} \FunctionTok{glm}\NormalTok{(sex\_binary }\SpecialCharTok{\textasciitilde{}}\NormalTok{ length, }
                   \AttributeTok{data =}\NormalTok{ data\_clean, }
                   \AttributeTok{family =} \FunctionTok{binomial}\NormalTok{(}\AttributeTok{link =} \StringTok{"logit"}\NormalTok{))}

\CommentTok{\# Расчет коэффициентов}
\NormalTok{beta0 }\OtherTok{\textless{}{-}} \FunctionTok{coef}\NormalTok{(model\_logit)[}\DecValTok{1}\NormalTok{]}
\NormalTok{beta1 }\OtherTok{\textless{}{-}} \FunctionTok{coef}\NormalTok{(model\_logit)[}\DecValTok{2}\NormalTok{]}

\CommentTok{\# Вычисление L50 (длина 50\% созревания)}
\NormalTok{L50 }\OtherTok{\textless{}{-}} \FunctionTok{round}\NormalTok{(}\SpecialCharTok{{-}}\NormalTok{beta0}\SpecialCharTok{/}\NormalTok{beta1, }\DecValTok{1}\NormalTok{)}

\CommentTok{\# 3. Визуализация {-}{-}{-}{-}{-}{-}{-}{-}{-}{-}{-}{-}{-}{-}{-}{-}{-}{-}{-}{-}{-}{-}{-}{-}{-}{-}{-}{-}{-}{-}{-}{-}{-}{-}{-}{-}{-}{-}{-}{-}{-}{-}{-}{-}{-}{-}{-}{-}{-}{-}{-}{-}{-}{-}{-}{-}{-}{-}{-}{-}}
\CommentTok{\# Логистическая кривая}
\FunctionTok{ggplot}\NormalTok{(data\_clean, }\FunctionTok{aes}\NormalTok{(}\AttributeTok{x =}\NormalTok{ length, }\AttributeTok{y =}\NormalTok{ sex\_binary)) }\SpecialCharTok{+}
  \FunctionTok{geom\_point}\NormalTok{(}\FunctionTok{aes}\NormalTok{(}\AttributeTok{color =}\NormalTok{ sex), }\AttributeTok{alpha =} \FloatTok{0.6}\NormalTok{, }\AttributeTok{size =} \DecValTok{2}\NormalTok{) }\SpecialCharTok{+}
  \FunctionTok{geom\_line}\NormalTok{(}\FunctionTok{aes}\NormalTok{(}\AttributeTok{y =} \FunctionTok{predict}\NormalTok{(model\_logit, }\AttributeTok{type =} \StringTok{"response"}\NormalTok{)), }
            \AttributeTok{color =} \StringTok{"\#D81B60"}\NormalTok{, }\AttributeTok{linewidth =} \FloatTok{1.5}\NormalTok{) }\SpecialCharTok{+}
  \FunctionTok{geom\_vline}\NormalTok{(}\AttributeTok{xintercept =}\NormalTok{ L50, }\AttributeTok{linetype =} \StringTok{"dashed"}\NormalTok{, }\AttributeTok{color =} \StringTok{"\#1E88E5"}\NormalTok{) }\SpecialCharTok{+}
  \FunctionTok{annotate}\NormalTok{(}\StringTok{"text"}\NormalTok{, }\AttributeTok{x =}\NormalTok{ L50 }\SpecialCharTok{+} \DecValTok{2}\NormalTok{, }\AttributeTok{y =} \FloatTok{0.2}\NormalTok{, }
           \AttributeTok{label =} \FunctionTok{paste}\NormalTok{(}\StringTok{"L50 ="}\NormalTok{, L50, }\StringTok{"мм"}\NormalTok{), }\AttributeTok{color =} \StringTok{"\#1E88E5"}\NormalTok{) }\SpecialCharTok{+}
  \FunctionTok{scale\_color\_manual}\NormalTok{(}\AttributeTok{values =} \FunctionTok{c}\NormalTok{(}\StringTok{"\#FFC107"}\NormalTok{, }\StringTok{"\#1976D2"}\NormalTok{)) }\SpecialCharTok{+}
  \FunctionTok{labs}\NormalTok{(}
    \AttributeTok{title =} \StringTok{"Зависимость пола от длины карапакса"}\NormalTok{,}
    \AttributeTok{subtitle =} \StringTok{"Логистическая регрессия с 50\%{-}ной точкой созревания"}\NormalTok{,}
    \AttributeTok{x =} \StringTok{"Длина карапакса (мм)"}\NormalTok{,}
    \AttributeTok{y =} \StringTok{"Вероятность быть самкой (P(F))"}
\NormalTok{  ) }\SpecialCharTok{+}
  \FunctionTok{theme\_minimal}\NormalTok{(}\AttributeTok{base\_size =} \DecValTok{12}\NormalTok{)}

\CommentTok{\# Огива (кумулятивное распределение)}
\NormalTok{data\_ogive }\OtherTok{\textless{}{-}}\NormalTok{ data\_clean }\SpecialCharTok{\%\textgreater{}\%}
  \FunctionTok{arrange}\NormalTok{(length) }\SpecialCharTok{\%\textgreater{}\%}
  \FunctionTok{mutate}\NormalTok{(}
    \AttributeTok{cum\_females =} \FunctionTok{cumsum}\NormalTok{(sex\_binary),}
    \AttributeTok{cum\_prob =}\NormalTok{ cum\_females }\SpecialCharTok{/} \FunctionTok{max}\NormalTok{(cum\_females)}
\NormalTok{  )}

\FunctionTok{ggplot}\NormalTok{(data\_ogive, }\FunctionTok{aes}\NormalTok{(}\AttributeTok{x =}\NormalTok{ length, }\AttributeTok{y =}\NormalTok{ cum\_prob)) }\SpecialCharTok{+}
  \FunctionTok{geom\_line}\NormalTok{(}\AttributeTok{color =} \StringTok{"\#4CAF50"}\NormalTok{, }\AttributeTok{linewidth =} \FloatTok{1.5}\NormalTok{) }\SpecialCharTok{+}
  \FunctionTok{geom\_vline}\NormalTok{(}\AttributeTok{xintercept =}\NormalTok{ L50, }\AttributeTok{linetype =} \StringTok{"dashed"}\NormalTok{, }\AttributeTok{color =} \StringTok{"\#1E88E5"}\NormalTok{) }\SpecialCharTok{+}
  \FunctionTok{geom\_hline}\NormalTok{(}\AttributeTok{yintercept =} \FloatTok{0.5}\NormalTok{, }\AttributeTok{linetype =} \StringTok{"dotted"}\NormalTok{, }\AttributeTok{color =} \StringTok{"\#757575"}\NormalTok{) }\SpecialCharTok{+}
  \FunctionTok{annotate}\NormalTok{(}\StringTok{"text"}\NormalTok{, }\AttributeTok{x =}\NormalTok{ L50 }\SpecialCharTok{+} \DecValTok{2}\NormalTok{, }\AttributeTok{y =} \FloatTok{0.55}\NormalTok{, }
           \AttributeTok{label =} \FunctionTok{paste}\NormalTok{(}\StringTok{"50\% созревание при"}\NormalTok{, L50, }\StringTok{"мм"}\NormalTok{), }\AttributeTok{color =} \StringTok{"\#1E88E5"}\NormalTok{) }\SpecialCharTok{+}
  \FunctionTok{scale\_y\_continuous}\NormalTok{(}\AttributeTok{labels =}\NormalTok{ scales}\SpecialCharTok{::}\NormalTok{percent) }\SpecialCharTok{+}
  \FunctionTok{labs}\NormalTok{(}
    \AttributeTok{title =} \StringTok{"Огива: Кумулятивное распределение самок"}\NormalTok{,}
    \AttributeTok{x =} \StringTok{"Длина карапакса (мм)"}\NormalTok{,}
    \AttributeTok{y =} \StringTok{"Накопленная доля самок"}
\NormalTok{  ) }\SpecialCharTok{+}
  \FunctionTok{theme\_minimal}\NormalTok{(}\AttributeTok{base\_size =} \DecValTok{12}\NormalTok{)}

\CommentTok{\# 4. Оценка модели {-}{-}{-}{-}{-}{-}{-}{-}{-}{-}{-}{-}{-}{-}{-}{-}{-}{-}{-}{-}{-}{-}{-}{-}{-}{-}{-}{-}{-}{-}{-}{-}{-}{-}{-}{-}{-}{-}{-}{-}{-}{-}{-}{-}{-}{-}{-}{-}{-}{-}{-}{-}{-}{-}{-}{-}{-}{-}{-}}
\CommentTok{\# ROC{-}анализ}
\NormalTok{roc\_obj }\OtherTok{\textless{}{-}} \FunctionTok{roc}\NormalTok{(data\_clean}\SpecialCharTok{$}\NormalTok{sex\_binary, }\FunctionTok{predict}\NormalTok{(model\_logit, }\AttributeTok{type =} \StringTok{"response"}\NormalTok{))}
\NormalTok{auc\_value }\OtherTok{\textless{}{-}} \FunctionTok{round}\NormalTok{(}\FunctionTok{auc}\NormalTok{(roc\_obj), }\DecValTok{2}\NormalTok{)}

\CommentTok{\# График ROC{-}кривой}
\FunctionTok{plot}\NormalTok{(roc\_obj, }\AttributeTok{col =} \StringTok{"\#E53935"}\NormalTok{, }\AttributeTok{main =} \FunctionTok{paste}\NormalTok{(}\StringTok{"ROC{-}кривая (AUC ="}\NormalTok{, auc\_value, }\StringTok{")"}\NormalTok{))}

\CommentTok{\# 5. Сохранение результатов {-}{-}{-}{-}{-}{-}{-}{-}{-}{-}{-}{-}{-}{-}{-}{-}{-}{-}{-}{-}{-}{-}{-}{-}{-}{-}{-}{-}{-}{-}{-}{-}{-}{-}{-}{-}{-}{-}{-}{-}{-}{-}{-}{-}{-}{-}{-}{-}{-}{-}}
\FunctionTok{ggsave}\NormalTok{(}\StringTok{"logistic\_curve.png"}\NormalTok{, }\AttributeTok{width =} \DecValTok{8}\NormalTok{, }\AttributeTok{height =} \DecValTok{6}\NormalTok{, }\AttributeTok{dpi =} \DecValTok{300}\NormalTok{)}
\FunctionTok{ggsave}\NormalTok{(}\StringTok{"ogive\_curve.png"}\NormalTok{, }\AttributeTok{width =} \DecValTok{8}\NormalTok{, }\AttributeTok{height =} \DecValTok{6}\NormalTok{, }\AttributeTok{dpi =} \DecValTok{300}\NormalTok{)}

\CommentTok{\# Вывод ключевых метрик}
\FunctionTok{cat}\NormalTok{(}\StringTok{"Результаты анализа:}\SpecialCharTok{\textbackslash{}n}\StringTok{"}\NormalTok{)}
\FunctionTok{cat}\NormalTok{(}\StringTok{"{-} Длина 50\%{-}ного созревания (L50):"}\NormalTok{, L50, }\StringTok{"мм}\SpecialCharTok{\textbackslash{}n}\StringTok{"}\NormalTok{)}
\FunctionTok{cat}\NormalTok{(}\StringTok{"{-} AUC модели:"}\NormalTok{, auc\_value, }\StringTok{"}\SpecialCharTok{\textbackslash{}n}\StringTok{"}\NormalTok{)}
\FunctionTok{cat}\NormalTok{(}\StringTok{"{-} Коэффициенты модели:}\SpecialCharTok{\textbackslash{}n}\StringTok{"}\NormalTok{)}
\FunctionTok{cat}\NormalTok{(}\StringTok{"  Intercept (β0):"}\NormalTok{, }\FunctionTok{round}\NormalTok{(beta0, }\DecValTok{2}\NormalTok{), }\StringTok{"}\SpecialCharTok{\textbackslash{}n}\StringTok{"}\NormalTok{)}
\FunctionTok{cat}\NormalTok{(}\StringTok{"  Slope (β1):"}\NormalTok{, }\FunctionTok{round}\NormalTok{(beta1, }\DecValTok{2}\NormalTok{), }\StringTok{"}\SpecialCharTok{\textbackslash{}n}\StringTok{"}\NormalTok{)}
\end{Highlighting}
\end{Shaded}

\section{Сравнение групп, параметров,
моделей}\label{ux441ux440ux430ux432ux43dux435ux43dux438ux435-ux433ux440ux443ux43fux43f-ux43fux430ux440ux430ux43cux435ux442ux440ux43eux432-ux43cux43eux434ux435ux43bux435ux439}

\subsection{Сравнение групп (на примере самцов и
самок)}\label{ux441ux440ux430ux432ux43dux435ux43dux438ux435-ux433ux440ux443ux43fux43f-ux43dux430-ux43fux440ux438ux43cux435ux440ux435-ux441ux430ux43cux446ux43eux432-ux438-ux441ux430ux43cux43eux43a}

Рассмотрим методы сравнения количественных характеристик (длина, вес)
между самцами и самками северной креветки. Анализ включает проверку
нормальности распределения, выбор подходящего статистического теста и
визуализацию различий.

\subsubsection{Подготовка
данных}\label{ux43fux43eux434ux433ux43eux442ux43eux432ux43aux430-ux434ux430ux43dux43dux44bux445}

Загрузим данные и выделим подвыборки для самцов и самок:

\begin{Shaded}
\begin{Highlighting}[]
\CommentTok{\# Установка рабочей директории}
\FunctionTok{setwd}\NormalTok{(}\StringTok{"C:/TEXTBOOK/"}\NormalTok{)}

\CommentTok{\# Загрузка библиотек  }
\FunctionTok{library}\NormalTok{(tidyverse)  }
\FunctionTok{library}\NormalTok{(ggplot2)  }
\FunctionTok{library}\NormalTok{(rstatix)}
\FunctionTok{library}\NormalTok{(ggpubr)}

\CommentTok{\# Загрузка данных  }
\NormalTok{data }\OtherTok{\textless{}{-}} \FunctionTok{read\_csv}\NormalTok{(}\StringTok{"shrimp\_catch.csv"}\NormalTok{) }\SpecialCharTok{\%\textgreater{}\%}
  \FunctionTok{filter}\NormalTok{(}\SpecialCharTok{!}\NormalTok{id }\SpecialCharTok{\%in\%} \FunctionTok{c}\NormalTok{(}\DecValTok{10}\NormalTok{, }\DecValTok{50}\NormalTok{))  }\CommentTok{\# Удаление аномальных наблюдений }

\CommentTok{\# Фильтрация данных по полу  }
\NormalTok{males }\OtherTok{\textless{}{-}}\NormalTok{ data }\SpecialCharTok{\%\textgreater{}\%} \FunctionTok{filter}\NormalTok{(sex }\SpecialCharTok{==} \StringTok{"M"}\NormalTok{)  }
\NormalTok{females }\OtherTok{\textless{}{-}}\NormalTok{ data }\SpecialCharTok{\%\textgreater{}\%} \FunctionTok{filter}\NormalTok{(sex }\SpecialCharTok{==} \StringTok{"F"}\NormalTok{) }
\end{Highlighting}
\end{Shaded}

\subsubsection{Проверка нормальности
распределения}\label{ux43fux440ux43eux432ux435ux440ux43aux430-ux43dux43eux440ux43cux430ux43bux44cux43dux43eux441ux442ux438-ux440ux430ux441ux43fux440ux435ux434ux435ux43bux435ux43dux438ux44f}

Перед сравнением групп проверим, соответствуют ли данные нормальному
распределению (тест Шапиро-Уилка):

\begin{Shaded}
\begin{Highlighting}[]
\CommentTok{\# Проверка нормальности для длины самцов  }
\FunctionTok{shapiro\_test}\NormalTok{(males}\SpecialCharTok{$}\NormalTok{length)  }
\CommentTok{\# Проверка нормальности для длины самок  }
\FunctionTok{shapiro\_test}\NormalTok{(females}\SpecialCharTok{$}\NormalTok{length) }
\end{Highlighting}
\end{Shaded}

Если p-value \textgreater{} 0.05, распределение считается нормальным. В
противном случае используем непараметрические методы.

\subsubsection{Сравнение средних
значений}\label{ux441ux440ux430ux432ux43dux435ux43dux438ux435-ux441ux440ux435ux434ux43dux438ux445-ux437ux43dux430ux447ux435ux43dux438ux439}

Если данные нормальны: t-тест

\begin{Shaded}
\begin{Highlighting}[]
\CommentTok{\# T{-}тест для сравнения длин самцов и самок  }
\NormalTok{t\_test\_result }\OtherTok{\textless{}{-}} \FunctionTok{t\_test}\NormalTok{(length }\SpecialCharTok{\textasciitilde{}}\NormalTok{ sex, }\AttributeTok{data =}\NormalTok{ data)  }
\NormalTok{t\_test\_result }
\end{Highlighting}
\end{Shaded}

Если данные не нормальны: U-тест Манна-Уитни

\begin{Shaded}
\begin{Highlighting}[]
\CommentTok{\# U{-}тест для сравнения длин самцов и самок  }
\NormalTok{mannwhitney\_result }\OtherTok{\textless{}{-}} \FunctionTok{wilcox\_test}\NormalTok{(length }\SpecialCharTok{\textasciitilde{}}\NormalTok{ sex, }\AttributeTok{data =}\NormalTok{ data)  }
\NormalTok{mannwhitney\_result }
\end{Highlighting}
\end{Shaded}

\subsubsection{Эффект размера (коэффициент
Коэна)}\label{ux44dux444ux444ux435ux43aux442-ux440ux430ux437ux43cux435ux440ux430-ux43aux43eux44dux444ux444ux438ux446ux438ux435ux43dux442-ux43aux43eux44dux43dux430}

Для оценки практической значимости различий рассчитаем коэффициент
Коэна:

\begin{Shaded}
\begin{Highlighting}[]
\CommentTok{\# Расчет коэффициента Коэна  }
\NormalTok{cohens\_d\_result }\OtherTok{\textless{}{-}} \FunctionTok{cohens\_d}\NormalTok{(length }\SpecialCharTok{\textasciitilde{}}\NormalTok{ sex, }\AttributeTok{data =}\NormalTok{ data)  }
\NormalTok{cohens\_d\_result  }
\end{Highlighting}
\end{Shaded}

\begin{itemize}
\item
  \textbf{d \textless{} 0.2} : малый эффект,
\item
  \textbf{d ≈ 0.5} : средний эффект,
\item
  \textbf{d \textgreater{} 0.8} : большой эффект.
\end{itemize}

\subsubsection{\texorpdfstring{\textbf{Визуализация
различий}}{Визуализация различий}}\label{ux432ux438ux437ux443ux430ux43bux438ux437ux430ux446ux438ux44f-ux440ux430ux437ux43bux438ux447ux438ux439}

Построим boxplot для визуального сравнения длин самцов и самок:

\begin{Shaded}
\begin{Highlighting}[]
\FunctionTok{ggplot}\NormalTok{(data, }\FunctionTok{aes}\NormalTok{(}\AttributeTok{x =}\NormalTok{ sex, }\AttributeTok{y =}\NormalTok{ length, }\AttributeTok{fill =}\NormalTok{ sex)) }\SpecialCharTok{+}  
  \FunctionTok{geom\_boxplot}\NormalTok{(}\AttributeTok{color =} \StringTok{"black"}\NormalTok{, }\AttributeTok{alpha =} \FloatTok{0.7}\NormalTok{) }\SpecialCharTok{+}  
  \FunctionTok{stat\_compare\_means}\NormalTok{(}\AttributeTok{method =} \StringTok{"t.test"}\NormalTok{) }\SpecialCharTok{+}  \CommentTok{\# Добавление p{-}value  }
  \FunctionTok{labs}\NormalTok{(}\AttributeTok{title =} \StringTok{"Сравнение длин самцов и самок"}\NormalTok{,  }
       \AttributeTok{x =} \StringTok{"Пол"}\NormalTok{, }\AttributeTok{y =} \StringTok{"Длина карапакса (мм)"}\NormalTok{) }\SpecialCharTok{+}  
  \FunctionTok{theme\_minimal}\NormalTok{() }
\end{Highlighting}
\end{Shaded}

\begin{figure}[H]

{\centering \includegraphics[width=0.6\linewidth,height=\textheight,keepaspectratio]{images/ttest_shrimp.PNG}

}

\caption{Рис. 1.10: Boxplot сравнения длин самцов и самок}

\end{figure}%

\subsubsection{\texorpdfstring{\textbf{Интерпретация
результатов}}{Интерпретация результатов}}\label{ux438ux43dux442ux435ux440ux43fux440ux435ux442ux430ux446ux438ux44f-ux440ux435ux437ux443ux43bux44cux442ux430ux442ux43eux432}

\begin{enumerate}
\def\labelenumi{\arabic{enumi}.}
\item
  Если p-value \textless{} 0.05, различия между группами статистически
  значимы.
\item
  Эффект размера помогает оценить биологическую важность различий.
  Например, если самки значительно крупнее самцов (d = 1.2), это может
  указывать на половой диморфизм, связанный с репродуктивной стратегией.

  \subsubsection{\texorpdfstring{\textbf{Пример полного анализа для
  веса}}{Пример полного анализа для веса}}\label{ux43fux440ux438ux43cux435ux440-ux43fux43eux43bux43dux43eux433ux43e-ux430ux43dux430ux43bux438ux437ux430-ux434ux43bux44f-ux432ux435ux441ux430}
\end{enumerate}

\begin{Shaded}
\begin{Highlighting}[]
\CommentTok{\# Полный анализ для веса  }
\NormalTok{weight\_analysis }\OtherTok{\textless{}{-}}\NormalTok{ data }\SpecialCharTok{\%\textgreater{}\%}  
  \FunctionTok{group\_by}\NormalTok{(sex) }\SpecialCharTok{\%\textgreater{}\%}  
  \FunctionTok{summarise}\NormalTok{(  }
    \AttributeTok{mean\_weight =} \FunctionTok{mean}\NormalTok{(weight),  }
    \AttributeTok{sd\_weight =} \FunctionTok{sd}\NormalTok{(weight),  }
    \AttributeTok{n =} \FunctionTok{n}\NormalTok{()  }
\NormalTok{  ) }\SpecialCharTok{\%\textgreater{}\%}  
  \FunctionTok{mutate}\NormalTok{(  }
    \AttributeTok{t\_test =} \FunctionTok{list}\NormalTok{(}\FunctionTok{t\_test}\NormalTok{(weight }\SpecialCharTok{\textasciitilde{}}\NormalTok{ sex, }\AttributeTok{data =}\NormalTok{ data)),  }
    \AttributeTok{cohens\_d =} \FunctionTok{list}\NormalTok{(}\FunctionTok{cohens\_d}\NormalTok{(weight }\SpecialCharTok{\textasciitilde{}}\NormalTok{ sex, }\AttributeTok{data =}\NormalTok{ data))  }
\NormalTok{  )  }

\CommentTok{\# Вывод результатов  }
\FunctionTok{print}\NormalTok{(weight\_analysis) }

\CommentTok{\# Распределение веса по полу}
\FunctionTok{ggplot}\NormalTok{(data, }\FunctionTok{aes}\NormalTok{(}\AttributeTok{x =} \FunctionTok{factor}\NormalTok{(sex), }\AttributeTok{y =}\NormalTok{ weight, }\AttributeTok{fill =} \FunctionTok{factor}\NormalTok{(sex))) }\SpecialCharTok{+}
  \FunctionTok{geom\_violin}\NormalTok{(}\AttributeTok{trim =} \ConstantTok{FALSE}\NormalTok{, }\AttributeTok{alpha =} \FloatTok{0.7}\NormalTok{) }\SpecialCharTok{+}
  \FunctionTok{geom\_boxplot}\NormalTok{(}\AttributeTok{width =} \FloatTok{0.2}\NormalTok{, }\AttributeTok{outlier.shape =} \ConstantTok{NA}\NormalTok{, }\AttributeTok{fill =} \StringTok{"white"}\NormalTok{) }\SpecialCharTok{+}
  \FunctionTok{labs}\NormalTok{(}\AttributeTok{title =} \StringTok{"Распределение веса по полу"}\NormalTok{, }\AttributeTok{x =} \StringTok{"Пол"}\NormalTok{, }\AttributeTok{y =} \StringTok{"Вес (г)"}\NormalTok{) }\SpecialCharTok{+}
  \FunctionTok{theme\_minimal}\NormalTok{()}
\end{Highlighting}
\end{Shaded}

\begin{figure}[H]

{\centering \includegraphics[width=0.6\linewidth,height=\textheight,keepaspectratio]{images/violin_shrimp.PNG}

}

\caption{Рис. 1.12: Violin plot для визуализации распределения веса}

\end{figure}%

\subsubsection{\texorpdfstring{\textbf{Выводы}}{Выводы}}\label{ux432ux44bux432ux43eux434ux44b}

\begin{enumerate}
\def\labelenumi{\arabic{enumi}.}
\item
  Используйте t-тест для нормальных данных и U-тест для ненормальных.
\item
  Дополните анализ оценкой эффекта размера для биологической
  интерпретации.
\item
  Визуализируйте различия с помощью boxplot или violin plot.
\end{enumerate}

\textbf{Рекомендации} :

\begin{itemize}
\item
  Для многомерных данных (например, одновременное сравнение длины, веса
  и возраста) применяйте MANOVA.
\item
  Если группы неоднородны (например, разный возрастной состав),
  используйте ковариационный анализ (ANCOVA).

  \subsection{\texorpdfstring{\textbf{Что делать, если тест на
  нормальность не пройден для одной из
  групп?}}{Что делать, если тест на нормальность не пройден для одной из групп?}}\label{ux447ux442ux43e-ux434ux435ux43bux430ux442ux44c-ux435ux441ux43bux438-ux442ux435ux441ux442-ux43dux430-ux43dux43eux440ux43cux430ux43bux44cux43dux43eux441ux442ux44c-ux43dux435-ux43fux440ux43eux439ux434ux435ux43d-ux434ux43bux44f-ux43eux434ux43dux43eux439-ux438ux437-ux433ux440ux443ux43fux43f}

  При сравнении количественных характеристик (например, длины карапакса
  у самцов и самок) важно учитывать, соответствуют ли данные нормальному
  распределению. Если тест на нормальность (например, Шапиро-Уилка)
  показывает значимое отклонение от нормальности для одной из групп, это
  влияет на выбор статистического теста и интерпретацию результатов.

  \subsubsection{\texorpdfstring{\textbf{Пример из нашего
  анализа}}{Пример из нашего анализа}}\label{ux43fux440ux438ux43cux435ux440-ux438ux437-ux43dux430ux448ux435ux433ux43e-ux430ux43dux430ux43bux438ux437ux430}

  Мы провели сравнение длины карапакса между самцами и самками:

  \begin{itemize}
  \item
    Для самцов: \textbf{\texttt{shapiro\_test(males\$length)}} → p-value
    = \textbf{0.000574} (нормальность отвергнута).
  \item
    Для самок: \textbf{\texttt{shapiro\_test(females\$length)}} →
    p-value = \textbf{0.891} (нормальность подтверждена).
  \end{itemize}

  Несмотря на это, мы применили как \textbf{t-тест} , так и
  \textbf{U-тест Манна-Уитни} :

  \begin{itemize}
  \item
    \textbf{t-тест} : p-value = 1.46e-40 (значимо).
  \item
    \textbf{U-тест} : p-value = 1.97e-27 (значимо).
  \item
    Коэффициент Коэна: d = 2.14 (большой эффект).
  \end{itemize}

  \subsubsection{\texorpdfstring{\textbf{Почему это
  работает?}}{Почему это работает?}}\label{ux43fux43eux447ux435ux43cux443-ux44dux442ux43e-ux440ux430ux431ux43eux442ux430ux435ux442}

  \begin{enumerate}
  \def\labelenumi{\arabic{enumi}.}
  \item
    \textbf{t-тест устойчив к умеренным отклонениям от нормальности} :

    \begin{itemize}
    \item
      При больших выборках (n \textgreater{} 30) центральная предельная
      теорема позволяет использовать t-тест даже при слабо выраженной
      асимметрии.
    \item
      В вашем случае выборка самцов (n = 149) достаточно велика, чтобы
      компенсировать отклонение от нормальности.
    \end{itemize}
  \item
    \textbf{U-тест Манна-Уитни --- непараметрическая альтернатива} :

    \begin{itemize}
    \item
      Этот тест не требует нормальности и сравнивает ранги, а не средние
      значения.
    \item
      Он подтверждает значимость различий, что усиливает доверие к
      выводу.
    \end{itemize}
  \item
    \textbf{Эффект размера (коэффициент Кобена)} :

    \begin{itemize}
    \tightlist
    \item
      d = 2.14 указывает на \textbf{большой эффект} , что важно для
      биологической интерпретации, даже если p-values значимы.
    \end{itemize}
  \end{enumerate}
\end{itemize}

\subsection{Сравнение параметров (линейные модели для оценки
межгрупповых
различий)}\label{ux441ux440ux430ux432ux43dux435ux43dux438ux435-ux43fux430ux440ux430ux43cux435ux442ux440ux43eux432-ux43bux438ux43dux435ux439ux43dux44bux435-ux43cux43eux434ux435ux43bux438-ux434ux43bux44f-ux43eux446ux435ux43dux43aux438-ux43cux435ux436ux433ux440ux443ux43fux43fux43eux432ux44bux445-ux440ux430ux437ux43bux438ux447ux438ux439}

Для сравнения параметров двух линейных моделей (например, скорости роста
самцов и самок) используем следующий подход.

\begin{Shaded}
\begin{Highlighting}[]
\CommentTok{\# Установка рабочей директории}
\FunctionTok{setwd}\NormalTok{(}\StringTok{"C:/TEXTBOOK/"}\NormalTok{)}

\CommentTok{\# Загрузка библиотек}
\FunctionTok{library}\NormalTok{(tidyverse)}
\FunctionTok{library}\NormalTok{(ggplot2)}
\FunctionTok{library}\NormalTok{(broom)}
\FunctionTok{library}\NormalTok{(knitr)}

\CommentTok{\# Загрузка данных}
\NormalTok{data }\OtherTok{\textless{}{-}} \FunctionTok{read\_csv}\NormalTok{(}\StringTok{"shrimp\_catch.csv"}\NormalTok{) }\SpecialCharTok{\%\textgreater{}\%}
  \FunctionTok{filter}\NormalTok{(}\SpecialCharTok{!}\NormalTok{id }\SpecialCharTok{\%in\%} \FunctionTok{c}\NormalTok{(}\DecValTok{10}\NormalTok{, }\DecValTok{50}\NormalTok{))  }\CommentTok{\# Удаление аномальных наблюдений}

\CommentTok{\# Фильтрация данных по полу}
\NormalTok{data\_male }\OtherTok{\textless{}{-}}\NormalTok{ data }\SpecialCharTok{\%\textgreater{}\%} \FunctionTok{filter}\NormalTok{(sex }\SpecialCharTok{==} \StringTok{"M"}\NormalTok{)}
\NormalTok{data\_female }\OtherTok{\textless{}{-}}\NormalTok{ data }\SpecialCharTok{\%\textgreater{}\%} \FunctionTok{filter}\NormalTok{(sex }\SpecialCharTok{==} \StringTok{"F"}\NormalTok{)}

\CommentTok{\# Построение моделей}
\NormalTok{model\_male }\OtherTok{\textless{}{-}} \FunctionTok{lm}\NormalTok{(length }\SpecialCharTok{\textasciitilde{}}\NormalTok{ age, }\AttributeTok{data =}\NormalTok{ data\_male)}
\NormalTok{model\_female }\OtherTok{\textless{}{-}} \FunctionTok{lm}\NormalTok{(length }\SpecialCharTok{\textasciitilde{}}\NormalTok{ age, }\AttributeTok{data =}\NormalTok{ data\_female)}

\FunctionTok{ggplot}\NormalTok{(data, }\FunctionTok{aes}\NormalTok{(age, length, }\AttributeTok{color =}\NormalTok{ sex)) }\SpecialCharTok{+}
  \FunctionTok{geom\_point}\NormalTok{(}\AttributeTok{alpha =} \FloatTok{0.5}\NormalTok{) }\SpecialCharTok{+}
  \FunctionTok{geom\_smooth}\NormalTok{(}\AttributeTok{method =} \StringTok{"lm"}\NormalTok{, }\AttributeTok{formula =}\NormalTok{ y }\SpecialCharTok{\textasciitilde{}}\NormalTok{ x) }\SpecialCharTok{+}
  \FunctionTok{scale\_color\_manual}\NormalTok{(}\AttributeTok{values =} \FunctionTok{c}\NormalTok{(}\StringTok{"\#E7B800"}\NormalTok{, }\StringTok{"\#00AFBB"}\NormalTok{)) }\SpecialCharTok{+}
  \FunctionTok{labs}\NormalTok{(}\AttributeTok{x =} \StringTok{"Возраст"}\NormalTok{, }\AttributeTok{y =} \StringTok{"Длина (мм)"}\NormalTok{) }\SpecialCharTok{+}
  \FunctionTok{theme\_minimal}\NormalTok{()}
\end{Highlighting}
\end{Shaded}

\begin{figure}[H]

{\centering \includegraphics[width=0.6\linewidth,height=\textheight,keepaspectratio]{images/comparison_shrimp.PNG}

}

\caption{Рис. 1.15: Визуализация моделей}

\end{figure}%

\textbf{Метод 1: Объединенная модель с взаимодействиями}

\begin{Shaded}
\begin{Highlighting}[]
\CommentTok{\# Установка рабочей директории}
\NormalTok{joint\_model }\OtherTok{\textless{}{-}} \FunctionTok{lm}\NormalTok{(length }\SpecialCharTok{\textasciitilde{}}\NormalTok{ age }\SpecialCharTok{*}\NormalTok{ sex, }\AttributeTok{data =}\NormalTok{ data)}
\FunctionTok{summary}\NormalTok{(joint\_model) }\SpecialCharTok{\%\textgreater{}\%} 
\NormalTok{  broom}\SpecialCharTok{::}\FunctionTok{tidy}\NormalTok{() }\SpecialCharTok{\%\textgreater{}\%} 
  \FunctionTok{filter}\NormalTok{(term }\SpecialCharTok{==} \StringTok{"age:sexM"}\NormalTok{) }\SpecialCharTok{\%\textgreater{}\%} 
  \FunctionTok{kable}\NormalTok{(}\AttributeTok{caption =} \StringTok{"Проверка различия наклонов"}\NormalTok{, }\AttributeTok{digits =} \DecValTok{3}\NormalTok{)}
\end{Highlighting}
\end{Shaded}

\begin{Shaded}
\begin{Highlighting}[]
\NormalTok{Table}\SpecialCharTok{:}\NormalTok{ Проверка различия наклонов}

\SpecialCharTok{|}\NormalTok{term     }\SpecialCharTok{|}\NormalTok{ estimate}\SpecialCharTok{|}\NormalTok{ std.error}\SpecialCharTok{|}\NormalTok{ statistic}\SpecialCharTok{|}\NormalTok{ p.value}\SpecialCharTok{|}
\ErrorTok{|:}\SpecialCharTok{{-}{-}{-}{-}{-}{-}{-}{-}}\ErrorTok{|}\SpecialCharTok{{-}{-}{-}{-}{-}{-}{-}{-}}\ErrorTok{:|}\SpecialCharTok{{-}{-}{-}{-}{-}{-}{-}{-}{-}}\ErrorTok{:|}\SpecialCharTok{{-}{-}{-}{-}{-}{-}{-}{-}{-}}\ErrorTok{:|}\SpecialCharTok{{-}{-}{-}{-}{-}{-}{-}}\ErrorTok{:|}
\ErrorTok{|}\NormalTok{age}\SpecialCharTok{:}\NormalTok{sexM }\SpecialCharTok{|}     \FloatTok{1.86}\SpecialCharTok{|}     \FloatTok{0.459}\SpecialCharTok{|}     \FloatTok{4.053}\SpecialCharTok{|}       \DecValTok{0}\SpecialCharTok{|}
\ErrorTok{\textgreater{}} 
\end{Highlighting}
\end{Shaded}

\textbf{Интерпретация:}\\
Значимый коэффициент взаимодействия \textbf{\texttt{age:sexM}} (p
\textless{} 0.05) указывает на статистически значимые различия в
скорости роста между полами.

\textbf{Метод 2: Тест Вальда}

\begin{Shaded}
\begin{Highlighting}[]
\FunctionTok{library}\NormalTok{(car)}
\NormalTok{delta\_beta }\OtherTok{\textless{}{-}} \FunctionTok{coef}\NormalTok{(model\_male)[}\StringTok{"age"}\NormalTok{] }\SpecialCharTok{{-}} \FunctionTok{coef}\NormalTok{(model\_female)[}\StringTok{"age"}\NormalTok{]}
\NormalTok{se\_diff }\OtherTok{\textless{}{-}} \FunctionTok{sqrt}\NormalTok{(}\FunctionTok{vcov}\NormalTok{(model\_male)[}\StringTok{"age"}\NormalTok{,}\StringTok{"age"}\NormalTok{] }\SpecialCharTok{+} \FunctionTok{vcov}\NormalTok{(model\_female)[}\StringTok{"age"}\NormalTok{,}\StringTok{"age"}\NormalTok{])}
\NormalTok{z\_score }\OtherTok{\textless{}{-}}\NormalTok{ delta\_beta }\SpecialCharTok{/}\NormalTok{ se\_diff}
\NormalTok{p\_value }\OtherTok{\textless{}{-}} \DecValTok{2} \SpecialCharTok{*} \FunctionTok{pnorm}\NormalTok{(}\SpecialCharTok{{-}}\FunctionTok{abs}\NormalTok{(z\_score))}

\FunctionTok{cat}\NormalTok{(}\StringTok{"Разница коэффициентов:"}\NormalTok{, }\FunctionTok{round}\NormalTok{(delta\_beta, }\DecValTok{3}\NormalTok{), }
    \StringTok{"}\SpecialCharTok{\textbackslash{}n}\StringTok{Z{-}статистика:"}\NormalTok{, }\FunctionTok{round}\NormalTok{(z\_score, }\DecValTok{3}\NormalTok{),}
    \StringTok{"}\SpecialCharTok{\textbackslash{}n}\StringTok{p{-}value:"}\NormalTok{, }\FunctionTok{format.pval}\NormalTok{(p\_value, }\AttributeTok{digits =} \DecValTok{2}\NormalTok{))}


\NormalTok{comparison\_table }\OtherTok{\textless{}{-}} \FunctionTok{data.frame}\NormalTok{(}
\NormalTok{  Параметр }\OtherTok{=} \FunctionTok{c}\NormalTok{(}\StringTok{"Скорость роста самцов"}\NormalTok{, }\StringTok{"Скорость роста самок"}\NormalTok{, }\StringTok{"Разница"}\NormalTok{),}
\NormalTok{  Значение }\OtherTok{=} \FunctionTok{c}\NormalTok{(}
    \FunctionTok{round}\NormalTok{(}\FunctionTok{coef}\NormalTok{(model\_male)[}\StringTok{"age"}\NormalTok{], }\DecValTok{2}\NormalTok{),}
    \FunctionTok{round}\NormalTok{(}\FunctionTok{coef}\NormalTok{(model\_female)[}\StringTok{"age"}\NormalTok{], }\DecValTok{2}\NormalTok{),}
    \FunctionTok{round}\NormalTok{(delta\_beta, }\DecValTok{2}\NormalTok{)}
\NormalTok{  ),}
  \StringTok{\textasciigrave{}}\AttributeTok{p{-}value}\StringTok{\textasciigrave{}} \OtherTok{=} \FunctionTok{c}\NormalTok{(}
    \FunctionTok{format.pval}\NormalTok{(}\FunctionTok{summary}\NormalTok{(model\_male)}\SpecialCharTok{$}\NormalTok{coefficients[}\StringTok{"age"}\NormalTok{,}\DecValTok{4}\NormalTok{], }\AttributeTok{digits =} \DecValTok{2}\NormalTok{),}
    \FunctionTok{format.pval}\NormalTok{(}\FunctionTok{summary}\NormalTok{(model\_female)}\SpecialCharTok{$}\NormalTok{coefficients[}\StringTok{"age"}\NormalTok{,}\DecValTok{4}\NormalTok{], }\AttributeTok{digits =} \DecValTok{2}\NormalTok{),}
    \FunctionTok{format.pval}\NormalTok{(p\_value, }\AttributeTok{digits =} \DecValTok{2}\NormalTok{)}
\NormalTok{  )}
\NormalTok{)}
\FunctionTok{kable}\NormalTok{(comparison\_table, }\AttributeTok{caption =} \StringTok{"Сравнение коэффициентов роста"}\NormalTok{)}
\end{Highlighting}
\end{Shaded}

Вывод

\begin{Shaded}
\begin{Highlighting}[]
\SpecialCharTok{:}\NormalTok{ Сравнение коэффициентов роста}

\SpecialCharTok{|}\NormalTok{Параметр              }\SpecialCharTok{|}\NormalTok{ Значение}\SpecialCharTok{|}\NormalTok{p.value }\SpecialCharTok{|}
\ErrorTok{|:}\SpecialCharTok{{-}{-}{-}{-}{-}{-}{-}{-}{-}{-}{-}{-}{-}{-}{-}{-}{-}{-}{-}{-}{-}}\ErrorTok{|}\SpecialCharTok{{-}{-}{-}{-}{-}{-}{-}{-}}\ErrorTok{:|:}\SpecialCharTok{{-}{-}{-}{-}{-}{-}{-}}\ErrorTok{|}
\ErrorTok{|}\NormalTok{Скорость роста самцов }\SpecialCharTok{|}     \FloatTok{5.95}\SpecialCharTok{|}\ErrorTok{\textless{}}\FloatTok{2e{-}16}  \SpecialCharTok{|}
\ErrorTok{|}\NormalTok{Скорость роста самок  }\SpecialCharTok{|}     \FloatTok{4.09}\SpecialCharTok{|}\FloatTok{5.2e{-}13} \SpecialCharTok{|}
\ErrorTok{|}\NormalTok{Разница               }\SpecialCharTok{|}     \FloatTok{1.86}\SpecialCharTok{|}\FloatTok{0.00024} \SpecialCharTok{|}
\ErrorTok{\textgreater{}} 
\end{Highlighting}
\end{Shaded}

\textbf{Интерпретация:}\\
Значимая \emph{разница} (p \textless{} 0.05) указывает на статистически
значимые различия в скорости роста между полами.

\subsection{Сравнение
моделей}\label{ux441ux440ux430ux432ux43dux435ux43dux438ux435-ux43cux43eux434ux435ux43bux435ux439}

Одним из ключевых аспектов анализа биологических данных является
определение формы зависимости между переменными. В данном разделе мы
рассмотрим основы подбора модели зависимости между длиной и весом
креветок. Начиная с простой линейной модели, мы постепенно перейдем к
более сложным нелинейным моделям, чтобы продемонстрировать методику
выбора наилучшей модели. Cравним три модели --- линейную, полиномиальную
и степенную --- чтобы определить, какая из них наилучшим образом
описывает данные. Цель анализа --- найти математическую зависимость,
которая:

\begin{enumerate}
\def\labelenumi{\arabic{enumi}.}
\item
  Точно предсказывает вес креветки по её длине.
\item
  Имеет биологическую интерпретацию.
\item
  Минимизирует ошибку предсказания.
\end{enumerate}

\subsubsection{Модели и их
параметры}\label{ux43cux43eux434ux435ux43bux438-ux438-ux438ux445-ux43fux430ux440ux430ux43cux435ux442ux440ux44b}

\begin{enumerate}
\def\labelenumi{\arabic{enumi}.}
\tightlist
\item
  \textbf{Линейная}:
  \(\text{weight} = \beta_0 + \beta_1\cdot\text{length}\)
\item
  \textbf{Полиномиальная 3-й степени}:
  \(\text{weight} = \beta_0 + \beta_1\cdot\text{length} + \beta_2\cdot\text{length}^2 + \beta_3\cdot\text{length}^3\)
\item
  \textbf{Степенная}: \(\text{weight} = a\cdot\text{length}^b\)
\end{enumerate}

\subsubsection{Метрики}\label{ux43cux435ux442ux440ux438ux43aux438}

\begin{itemize}
\tightlist
\item
  \textbf{R²} - (коэффициент детерминации): чем ближе к 1, тем лучше
  модель объясняет данные.
\item
  \textbf{AIC} -(информационный критерий Акаике): чем меньше значение,
  тем лучше модель с учётом её сложности.
\end{itemize}

\subsubsection{\texorpdfstring{\textbf{Результаты}}{Результаты}}\label{ux440ux435ux437ux443ux43bux44cux442ux430ux442ux44b}

\paragraph{\texorpdfstring{\textbf{1. Линейная
модель}}{1. Линейная модель}}\label{ux43bux438ux43dux435ux439ux43dux430ux44f-ux43cux43eux434ux435ux43bux44c}

\begin{Shaded}
\begin{Highlighting}[]
\NormalTok{Coefficients}\SpecialCharTok{:}
\NormalTok{             Estimate Std. Error t value }\FunctionTok{Pr}\NormalTok{(}\SpecialCharTok{\textgreater{}}\ErrorTok{|}\NormalTok{t}\SpecialCharTok{|}\NormalTok{)    }
\NormalTok{(Intercept) }\SpecialCharTok{{-}}\FloatTok{2.115}      \FloatTok{0.085}     \SpecialCharTok{{-}}\FloatTok{24.86}   \SpecialCharTok{\textless{}}\FloatTok{2e{-}16} \SpecialCharTok{**}\ErrorTok{*}
\NormalTok{length       }\FloatTok{0.1665}     \FloatTok{0.0038}    \FloatTok{43.71}    \SpecialCharTok{\textless{}}\FloatTok{2e{-}16} \SpecialCharTok{**}\ErrorTok{*}
\end{Highlighting}
\end{Shaded}

\begin{itemize}
\item
  \textbf{R² = 0.894}
\item
  \textbf{AIC = 148.02}
\end{itemize}

\begin{figure}[H]

{\centering \includegraphics[width=0.6\linewidth,height=\textheight,keepaspectratio]{images/linear_shrimp.PNG}

}

\caption{Рис. 1.5: Линейная модель}

\end{figure}%

\paragraph{\texorpdfstring{\textbf{2. Полиномиальная
модель}}{2. Полиномиальная модель}}\label{ux43fux43eux43bux438ux43dux43eux43cux438ux430ux43bux44cux43dux430ux44f-ux43cux43eux434ux435ux43bux44c}

\begin{Shaded}
\begin{Highlighting}[]
\NormalTok{Coefficients}\SpecialCharTok{:}
\NormalTok{                 Estimate Std. Error t value }\FunctionTok{Pr}\NormalTok{(}\SpecialCharTok{\textgreater{}}\ErrorTok{|}\NormalTok{t}\SpecialCharTok{|}\NormalTok{)    }
\FunctionTok{poly}\NormalTok{(length,}\DecValTok{3}\NormalTok{)}\DecValTok{1}  \FloatTok{14.5038}    \FloatTok{0.2127}    \FloatTok{68.18}   \SpecialCharTok{\textless{}}\FloatTok{2e{-}16} \SpecialCharTok{**}\ErrorTok{*}
\FunctionTok{poly}\NormalTok{(length,}\DecValTok{3}\NormalTok{)}\DecValTok{2}   \FloatTok{3.7209}    \FloatTok{0.2127}    \FloatTok{17.49}   \SpecialCharTok{\textless{}}\FloatTok{2e{-}16} \SpecialCharTok{**}\ErrorTok{*}
\FunctionTok{poly}\NormalTok{(length,}\DecValTok{3}\NormalTok{)}\DecValTok{3}   \FloatTok{0.9526}    \FloatTok{0.2127}     \FloatTok{4.48}  \FloatTok{1.2e{-}05} \SpecialCharTok{**}\ErrorTok{*}
\end{Highlighting}
\end{Shaded}

\begin{itemize}
\item
  \textbf{R² = 0.957}
\item
  \textbf{AIC = -52.80}
\end{itemize}

\begin{figure}[H]

{\centering \includegraphics[width=0.6\linewidth,height=\textheight,keepaspectratio]{images/poly_shrimp.PNG}

}

\caption{Рис. 1.5: Полиномиальная модель}

\end{figure}%

\paragraph{\texorpdfstring{\textbf{3. Степенная
модель}}{3. Степенная модель}}\label{ux441ux442ux435ux43fux435ux43dux43dux430ux44f-ux43cux43eux434ux435ux43bux44c}

\begin{Shaded}
\begin{Highlighting}[]
\NormalTok{Parameters}\SpecialCharTok{:}
\NormalTok{   Estimate Std. Error t value }\FunctionTok{Pr}\NormalTok{(}\SpecialCharTok{\textgreater{}}\ErrorTok{|}\NormalTok{t}\SpecialCharTok{|}\NormalTok{)    }
\NormalTok{a }\FloatTok{0.000157}   \FloatTok{0.000028}    \FloatTok{5.60}  \FloatTok{6.3e{-}08} \SpecialCharTok{**}\ErrorTok{*}
\NormalTok{b }\FloatTok{2.920160}   \FloatTok{0.054102}   \FloatTok{53.98}   \SpecialCharTok{\textless{}}\FloatTok{2e{-}16} \SpecialCharTok{**}\ErrorTok{*}
\end{Highlighting}
\end{Shaded}

\begin{itemize}
\item
  \textbf{R² = 0.955}
\item
  \textbf{AIC = -48.43} \begin{center}
  \includegraphics[width=0.6\linewidth,height=\textheight,keepaspectratio]{images/power_shrimp.PNG}
  \end{center}
\end{itemize}

\subsubsection{\texorpdfstring{\textbf{3. Сравнение
моделей}}{3. Сравнение моделей}}\label{ux441ux440ux430ux432ux43dux435ux43dux438ux435-ux43cux43eux434ux435ux43bux435ux439-1}

\begin{longtable}[]{@{}lll@{}}
\toprule\noalign{}
\textbf{Модель} & \textbf{R²} & \textbf{AIC} \\
\midrule\noalign{}
\endhead
\bottomrule\noalign{}
\endlastfoot
Линейная & 0.894 & 148.02 \\
Полиномиальная & 0.957 & -52.80 \\
Степенная & 0.955 & -48.43 \\
\end{longtable}

\textbf{Выводы:}

\begin{enumerate}
\def\labelenumi{\arabic{enumi}.}
\item
  \textbf{Полиномиальная модель} демонстрирует наилучшие показатели
  (максимальный R² и минимальный AIC).
\item
  \textbf{Степенная модель} близка по качеству, но её параметр
  \emph{b}≈2.92 близок к биологически ожидаемому значению 3 (вес
  пропорционален объёму).
\item
  \textbf{Линейная модель} существенно уступает по точности.
\end{enumerate}

\subsubsection{\texorpdfstring{\textbf{4.
Рекомендации}}{4. Рекомендации}}\label{ux440ux435ux43aux43eux43cux435ux43dux434ux430ux446ux438ux438}

\begin{itemize}
\item
  \textbf{Для прогнозирования} используйте полиномиальную модель, так
  как она минимизирует ошибку.
\item
  \textbf{Для биологической интерпретации} предпочтительна степенная
  модель: weight∝length\textsuperscript{2.92}.
\item
  \textbf{Избегайте переобучения:} Полиномиальные модели высокой степени
  могут терять интерпретируемость.
\end{itemize}

\subsubsection{\texorpdfstring{\textbf{5. Визуализация
остатков}}{5. Визуализация остатков}}\label{ux432ux438ux437ux443ux430ux43bux438ux437ux430ux446ux438ux44f-ux43eux441ux442ux430ux442ux43aux43eux432}

Остатки степенной модели распределены равномерно, что подтверждает её
адекватность: \begin{center}
\includegraphics[width=0.6\linewidth,height=\textheight,keepaspectratio]{images/residuals_shrimp.PNG}
\end{center}

\subsubsection{\texorpdfstring{\textbf{Заключение}}{Заключение}}\label{ux437ux430ux43aux43bux44eux447ux435ux43dux438ux435}

Для анализа зависимости веса от длины северной креветки
\textbf{рекомендуется}:

\begin{enumerate}
\def\labelenumi{\arabic{enumi}.}
\item
  \textbf{Полиномиальная модель} --- для задач, требующих максимальной
  точности.
\item
  \textbf{Степенная модель} --- для интерпретации биологических
  закономерностей.
\end{enumerate}

Скрипт вышеописанных событий:

\begin{Shaded}
\begin{Highlighting}[]
\CommentTok{\# Установка рабочей директории}
\FunctionTok{setwd}\NormalTok{(}\StringTok{"C:/TEXTBOOK/"}\NormalTok{)}

\CommentTok{\# Загрузка библиотек}
\FunctionTok{library}\NormalTok{(tidyverse)}
\FunctionTok{library}\NormalTok{(ggplot2)}

\CommentTok{\# Загрузка данных}
\NormalTok{data }\OtherTok{\textless{}{-}} \FunctionTok{read\_csv}\NormalTok{(}\StringTok{"shrimp\_catch.csv"}\NormalTok{) }\SpecialCharTok{\%\textgreater{}\%}
  \FunctionTok{filter}\NormalTok{(}\SpecialCharTok{!}\NormalTok{id }\SpecialCharTok{\%in\%} \FunctionTok{c}\NormalTok{(}\DecValTok{10}\NormalTok{, }\DecValTok{50}\NormalTok{))  }\CommentTok{\# Удаление аномальных наблюдений}

\CommentTok{\# Проверка структуры}
\FunctionTok{glimpse}\NormalTok{(data)}

\CommentTok{\# Линейная модель: вес \textasciitilde{} длина}
\NormalTok{model\_linear }\OtherTok{\textless{}{-}} \FunctionTok{lm}\NormalTok{(weight }\SpecialCharTok{\textasciitilde{}}\NormalTok{ length, }\AttributeTok{data =}\NormalTok{ data)}
\FunctionTok{summary}\NormalTok{(model\_linear)}

\CommentTok{\# Визуализация}
\FunctionTok{ggplot}\NormalTok{(data, }\FunctionTok{aes}\NormalTok{(}\AttributeTok{x =}\NormalTok{ length, }\AttributeTok{y =}\NormalTok{ weight)) }\SpecialCharTok{+}
  \FunctionTok{geom\_point}\NormalTok{(}\AttributeTok{color =} \StringTok{"steelblue"}\NormalTok{, }\AttributeTok{alpha =} \FloatTok{0.7}\NormalTok{) }\SpecialCharTok{+}
  \FunctionTok{geom\_smooth}\NormalTok{(}\AttributeTok{method =} \StringTok{"lm"}\NormalTok{, }\AttributeTok{color =} \StringTok{"\#FC4E07"}\NormalTok{) }\SpecialCharTok{+}
  \FunctionTok{labs}\NormalTok{(}\AttributeTok{title =} \StringTok{"Линейная модель"}\NormalTok{, }\AttributeTok{x =} \StringTok{"Длина (мм)"}\NormalTok{, }\AttributeTok{y =} \StringTok{"Вес (г)"}\NormalTok{)}


\CommentTok{\# Полиномиальная модель: вес \textasciitilde{} длина + длина? + длина?}
\NormalTok{model\_poly }\OtherTok{\textless{}{-}} \FunctionTok{lm}\NormalTok{(weight }\SpecialCharTok{\textasciitilde{}} \FunctionTok{poly}\NormalTok{(length, }\DecValTok{3}\NormalTok{), }\AttributeTok{data =}\NormalTok{ data)}
\FunctionTok{summary}\NormalTok{(model\_poly)}

\CommentTok{\# Визуализация}
\FunctionTok{ggplot}\NormalTok{(data, }\FunctionTok{aes}\NormalTok{(}\AttributeTok{x =}\NormalTok{ length, }\AttributeTok{y =}\NormalTok{ weight)) }\SpecialCharTok{+}
  \FunctionTok{geom\_point}\NormalTok{(}\AttributeTok{color =} \StringTok{"steelblue"}\NormalTok{, }\AttributeTok{alpha =} \FloatTok{0.7}\NormalTok{) }\SpecialCharTok{+}
  \FunctionTok{geom\_smooth}\NormalTok{(}\AttributeTok{method =} \StringTok{"lm"}\NormalTok{, }\AttributeTok{formula =}\NormalTok{ y }\SpecialCharTok{\textasciitilde{}} \FunctionTok{poly}\NormalTok{(x, }\DecValTok{3}\NormalTok{), }\AttributeTok{color =} \StringTok{"\#E7B800"}\NormalTok{) }\SpecialCharTok{+}
  \FunctionTok{labs}\NormalTok{(}\AttributeTok{title =} \StringTok{"Полиномиальная модель"}\NormalTok{, }\AttributeTok{x =} \StringTok{"Длина (мм)"}\NormalTok{, }\AttributeTok{y =} \StringTok{"Вес (г)"}\NormalTok{)}


\CommentTok{\# Степенная модель: вес \textasciitilde{} длина\^{}k (k подбирается)}
\NormalTok{model\_power }\OtherTok{\textless{}{-}} \FunctionTok{nls}\NormalTok{(weight }\SpecialCharTok{\textasciitilde{}}\NormalTok{ a }\SpecialCharTok{*}\NormalTok{ length}\SpecialCharTok{\^{}}\NormalTok{b, }
                   \AttributeTok{data =}\NormalTok{ data, }
                   \AttributeTok{start =} \FunctionTok{list}\NormalTok{(}\AttributeTok{a =} \FloatTok{0.001}\NormalTok{, }\AttributeTok{b =} \DecValTok{3}\NormalTok{))  }\CommentTok{\# Начальные значения}
\FunctionTok{summary}\NormalTok{(model\_power)}

\CommentTok{\# Визуализация}
\NormalTok{data}\SpecialCharTok{$}\NormalTok{pred\_power }\OtherTok{\textless{}{-}} \FunctionTok{predict}\NormalTok{(model\_power)}
\FunctionTok{ggplot}\NormalTok{(data, }\FunctionTok{aes}\NormalTok{(}\AttributeTok{x =}\NormalTok{ length, }\AttributeTok{y =}\NormalTok{ weight)) }\SpecialCharTok{+}
  \FunctionTok{geom\_point}\NormalTok{(}\AttributeTok{color =} \StringTok{"steelblue"}\NormalTok{, }\AttributeTok{alpha =} \FloatTok{0.7}\NormalTok{) }\SpecialCharTok{+}
  \FunctionTok{geom\_line}\NormalTok{(}\FunctionTok{aes}\NormalTok{(}\AttributeTok{y =}\NormalTok{ pred\_power), }\AttributeTok{color =} \StringTok{"\#00BA38"}\NormalTok{, }\AttributeTok{linewidth =} \FloatTok{1.2}\NormalTok{) }\SpecialCharTok{+}
  \FunctionTok{labs}\NormalTok{(}\AttributeTok{title =} \StringTok{"Степенная модель"}\NormalTok{, }\AttributeTok{x =} \StringTok{"Длина (мм)"}\NormalTok{, }\AttributeTok{y =} \StringTok{"Вес (г)"}\NormalTok{)}

\CommentTok{\# Расчет AIC}
\FunctionTok{AIC}\NormalTok{(model\_linear, model\_poly, model\_power)}

\CommentTok{\# Расчет R?}
\NormalTok{r2\_linear }\OtherTok{\textless{}{-}} \FunctionTok{summary}\NormalTok{(model\_linear)}\SpecialCharTok{$}\NormalTok{r.squared}
\NormalTok{r2\_poly }\OtherTok{\textless{}{-}} \FunctionTok{summary}\NormalTok{(model\_poly)}\SpecialCharTok{$}\NormalTok{r.squared}
\NormalTok{r2\_power }\OtherTok{\textless{}{-}} \DecValTok{1} \SpecialCharTok{{-}} \FunctionTok{sum}\NormalTok{(}\FunctionTok{residuals}\NormalTok{(model\_power)}\SpecialCharTok{\^{}}\DecValTok{2}\NormalTok{) }\SpecialCharTok{/} \FunctionTok{sum}\NormalTok{((data}\SpecialCharTok{$}\NormalTok{weight }\SpecialCharTok{{-}} \FunctionTok{mean}\NormalTok{(data}\SpecialCharTok{$}\NormalTok{weight))}\SpecialCharTok{\^{}}\DecValTok{2}\NormalTok{)}

\CommentTok{\# Создание таблицы сравнения моделей}
\NormalTok{comparison\_table }\OtherTok{\textless{}{-}} \FunctionTok{data.frame}\NormalTok{(}
\NormalTok{  Модель }\OtherTok{=} \FunctionTok{c}\NormalTok{(}\StringTok{"Линейная"}\NormalTok{, }\StringTok{"Полиномиальная"}\NormalTok{, }\StringTok{"Степенная"}\NormalTok{),}
  \AttributeTok{R\_square =} \FunctionTok{c}\NormalTok{(r2\_linear, r2\_poly, r2\_power),}
  \AttributeTok{AIC =} \FunctionTok{c}\NormalTok{(}\FunctionTok{AIC}\NormalTok{(model\_linear), }\FunctionTok{AIC}\NormalTok{(model\_poly), }\FunctionTok{AIC}\NormalTok{(model\_power))}
\NormalTok{)}

\CommentTok{\# Вывод таблицы}
\FunctionTok{print}\NormalTok{(comparison\_table)}

\CommentTok{\# Остатки для степенной модели}
\NormalTok{data}\SpecialCharTok{$}\NormalTok{residuals }\OtherTok{\textless{}{-}} \FunctionTok{residuals}\NormalTok{(model\_power)}

\FunctionTok{ggplot}\NormalTok{(data, }\FunctionTok{aes}\NormalTok{(}\AttributeTok{x =}\NormalTok{ length, }\AttributeTok{y =}\NormalTok{ residuals)) }\SpecialCharTok{+}
  \FunctionTok{geom\_point}\NormalTok{(}\AttributeTok{color =} \StringTok{"\#FC4E07"}\NormalTok{, }\AttributeTok{alpha =} \FloatTok{0.7}\NormalTok{) }\SpecialCharTok{+}
  \FunctionTok{geom\_hline}\NormalTok{(}\AttributeTok{yintercept =} \DecValTok{0}\NormalTok{, }\AttributeTok{linetype =} \StringTok{"dashed"}\NormalTok{) }\SpecialCharTok{+}
  \FunctionTok{labs}\NormalTok{(}\AttributeTok{title =} \StringTok{"Остатки степенной модели"}\NormalTok{, }\AttributeTok{x =} \StringTok{"Длина (мм)"}\NormalTok{, }\AttributeTok{y =} \StringTok{"Ошибка"}\NormalTok{)}
\end{Highlighting}
\end{Shaded}

\bookmarksetup{startatroot}

\chapter{Нейронные сети в экологии: практическое
введение}\label{ux43dux435ux439ux440ux43eux43dux43dux44bux435-ux441ux435ux442ux438-ux432-ux44dux43aux43eux43bux43eux433ux438ux438-ux43fux440ux430ux43aux442ux438ux447ux435ux441ux43aux43eux435-ux432ux432ux435ux434ux435ux43dux438ux435}

\section{Введение}\label{ux432ux432ux435ux434ux435ux43dux438ux435-1}

Это практическое занятие можно рассматривать не только как введение в
нейронные сети, но и как введение в экологическое моделирование в общем
с помошью R. Занятие основано на статье Андрея Викторовича Коросова
``\href{https://ecopri.ru/journal/article.php?id=14002}{Нейронные сети в
экологии: введение}'', опубликованной в журнале Принципы экологии, №3,
2023, стр. 76-96. В статье рассмотрены основы нейросетевого
моделирования в экологии, начиная с классических регрессионных методов и
заканчивая искусственными нейронными сетями. Представленные примеры
демонстрируют эволюционный переход от простых линейных моделей к сложным
нейросетевым конструкциям, что позволяет решать задачи классификации и
прогнозирования в экологии. ``Теоретическая'' лекция, основанная на этой
статье, находиться по \href{KOROSOV.ppt}{ссылке}.

Можно скачать скрипт целиком в трех версиях:
\href{https://mombus.github.io/cRab/data/KOROSOV.R}{KOROSOV.R} -
максимально приближен к оригинальной работе;
\href{https://mombus.github.io/cRab/data/KOROSOV_updated.R}{KOROSOV\_updated.R}
- тотже скрипт, но с комментариями и пояснениями (содержание этого
занятия);
\href{https://mombus.github.io/cRab/data/KOROSOV_visual.R}{KOROSOV\_visual.R}
- почти такой же с дополнительным продвинутым визуалом и аналитикой.

\subsubsection{\texorpdfstring{\textbf{Для работы
скрипта:}}{Для работы скрипта:}}\label{ux434ux43bux44f-ux440ux430ux431ux43eux442ux44b-ux441ux43aux440ux438ux43fux442ux430}

\begin{enumerate}
\def\labelenumi{\arabic{enumi}.}
\item
  Скачайте файлы данных
  (\href{https://mombus.github.io/cRab/data/vipkar.csv}{vipkar.csv} и
  \href{https://mombus.github.io/cRab/data/kihzsdat.csv}{kihzsdat.csv})
\item
  Установите рабочую директорию в setwd()
\item
  Установите необходимые пакеты :
  \textbf{\texttt{install.packages(c("neuralnet",\ "ggplot2"))}}
\end{enumerate}

\begin{Shaded}
\begin{Highlighting}[]
\CommentTok{\# ЗАГРУЗКА БИБЛИОТЕК И НАСТРОЙКА СРЕДЫ ================================}
\FunctionTok{library}\NormalTok{(neuralnet)   }\CommentTok{\# Для построения нейронных сетей}
\FunctionTok{library}\NormalTok{(ggplot2)     }\CommentTok{\# Для продвинутой визуализации (в данном скрипте не используется напрямую)}

\CommentTok{\# Установите свою рабочую директорию (где лежат файлы данных)}
\CommentTok{\# setwd("C:/ВАША\_ДИРЕКТОРИЯ/")}
\end{Highlighting}
\end{Shaded}

\section{ЛИНЕЙНАЯ
РЕГРЕССИЯ}\label{ux43bux438ux43dux435ux439ux43dux430ux44f-ux440ux435ux433ux440ux435ux441ux441ux438ux44f}

В этом разделе мы изучим основы экологического моделирования на примере
зависимости массы тела гадюки от ее длины. Вы построите простую линейную
регрессионную модель, визуализируете данные и линию регрессии, а также
интерпретируете результаты с помощью функции \texttt{summary()}.

Загружаем данные

\begin{Shaded}
\begin{Highlighting}[]
\CommentTok{\# Данные: масса (w) и длина тела (lt) гадюк (в см и граммах)}
\NormalTok{w }\OtherTok{\textless{}{-}} \FunctionTok{c}\NormalTok{(}\DecValTok{85}\NormalTok{, }\DecValTok{90}\NormalTok{, }\DecValTok{85}\NormalTok{, }\DecValTok{95}\NormalTok{, }\DecValTok{95}\NormalTok{, }\DecValTok{135}\NormalTok{, }\DecValTok{165}\NormalTok{, }\DecValTok{135}\NormalTok{, }\DecValTok{140}\NormalTok{)}
\NormalTok{lt }\OtherTok{\textless{}{-}} \FunctionTok{c}\NormalTok{(}\DecValTok{51}\NormalTok{, }\DecValTok{51}\NormalTok{, }\DecValTok{52}\NormalTok{, }\DecValTok{54}\NormalTok{, }\DecValTok{54}\NormalTok{, }\DecValTok{59}\NormalTok{, }\DecValTok{59}\NormalTok{, }\DecValTok{60}\NormalTok{, }\DecValTok{62}\NormalTok{)}
\end{Highlighting}
\end{Shaded}

Строим и запускаем модель \[
w_t = a_0 + a_1 \cdot l_t
\]

где: - \(w_t\) --- зависимая переменная, - \(a_0\) --- свободный член, -
\(a_1\) --- коэффициент регрессии, - \(l_t\) --- независимая переменная.

\begin{Shaded}
\begin{Highlighting}[]
\CommentTok{\# Построение линейной модели: w = a0 + a1*lt}
\NormalTok{lreg }\OtherTok{\textless{}{-}} \FunctionTok{lm}\NormalTok{(w }\SpecialCharTok{\textasciitilde{}}\NormalTok{ lt)}
\end{Highlighting}
\end{Shaded}

Выведем результаты модели

\begin{Shaded}
\begin{Highlighting}[]
\CommentTok{\# Просмотр результатов модели:}
\FunctionTok{summary}\NormalTok{(lreg)  }\CommentTok{\# Обратите внимание на коэффициенты и p{-}значения}
\end{Highlighting}
\end{Shaded}

На экране появится:

\begin{Shaded}
\begin{Highlighting}[]
\NormalTok{Call}\SpecialCharTok{:}
\FunctionTok{lm}\NormalTok{(}\AttributeTok{formula =}\NormalTok{ w }\SpecialCharTok{\textasciitilde{}}\NormalTok{ lt)}

\NormalTok{Residuals}\SpecialCharTok{:}
\NormalTok{    Min      }\DecValTok{1}\NormalTok{Q  Median      }\DecValTok{3}\NormalTok{Q     Max }
\SpecialCharTok{{-}}\FloatTok{13.452}  \SpecialCharTok{{-}}\FloatTok{7.585}  \SpecialCharTok{{-}}\FloatTok{4.868}   \FloatTok{1.490}  \FloatTok{30.623} 

\NormalTok{Coefficients}\SpecialCharTok{:}
\NormalTok{            Estimate Std. Error t value }\FunctionTok{Pr}\NormalTok{(}\SpecialCharTok{\textgreater{}}\ErrorTok{|}\NormalTok{t}\SpecialCharTok{|}\NormalTok{)    }
\NormalTok{(Intercept) }\SpecialCharTok{{-}}\FloatTok{240.766}     \FloatTok{64.457}  \SpecialCharTok{{-}}\FloatTok{3.735} \FloatTok{0.007308} \SpecialCharTok{**} 
\NormalTok{lt             }\FloatTok{6.358}      \FloatTok{1.153}   \FloatTok{5.516} \FloatTok{0.000891} \SpecialCharTok{**}\ErrorTok{*}
\SpecialCharTok{{-}{-}{-}}
\NormalTok{Signif. codes}\SpecialCharTok{:}  \DecValTok{0}\NormalTok{ ‘}\SpecialCharTok{**}\ErrorTok{*}\NormalTok{’ }\FloatTok{0.001}\NormalTok{ ‘}\SpecialCharTok{**}\NormalTok{’ }\FloatTok{0.01}\NormalTok{ ‘}\SpecialCharTok{*}\NormalTok{’ }\FloatTok{0.05}\NormalTok{ ‘.’ }\FloatTok{0.1}\NormalTok{ ‘ ’ }\DecValTok{1}

\NormalTok{Residual standard error}\SpecialCharTok{:} \FloatTok{13.81}\NormalTok{ on }\DecValTok{7}\NormalTok{ degrees of freedom}
\NormalTok{Multiple R}\SpecialCharTok{{-}}\NormalTok{squared}\SpecialCharTok{:}  \FloatTok{0.813}\NormalTok{,     Adjusted R}\SpecialCharTok{{-}}\NormalTok{squared}\SpecialCharTok{:}  \FloatTok{0.7863} 
\NormalTok{F}\SpecialCharTok{{-}}\NormalTok{statistic}\SpecialCharTok{:} \FloatTok{30.43}\NormalTok{ on }\DecValTok{1}\NormalTok{ and }\DecValTok{7}\NormalTok{ DF,  p}\SpecialCharTok{{-}}\NormalTok{value}\SpecialCharTok{:} \FloatTok{0.0008911}
\end{Highlighting}
\end{Shaded}

Мы получили результаты линейной регрессии, где зависимая переменная ---
масса тела гадюки (w), а независимая переменная --- длина тела (lt).
Разберем каждый параметр:

1. **Call (Вызов модели):**

`lm(formula = w \textasciitilde{} lt)`

Это просто напоминание, какая модель была построена. Здесь указано, что
мы моделировали зависимость массы (w) от длины тела (lt) с помощью
линейной регрессии.

2. **Residuals (Остатки):**

Остатки --- это разница между наблюдаемыми значениями массы и
предсказанными моделью значениями. Они показывают, насколько хорошо
модель описывает данные.

\begin{itemize}
\item
  `Min`: минимальный остаток = -13.452 (наибольшее недооцененное
  значение)
\item
  `1Q`: первый квартиль = -7.585 (25\% остатков меньше этого значения)
\item
  `Median`: медиана остатков = -4.868 (середина распределения остатков)
\item
  `3Q`: третий квартиль = 1.490 (75\% остатков меньше этого значения)
\item
  `Max`: максимальный остаток = 30.623 (наибольшее переоцененное
  значение)
\end{itemize}

Распределение остатков: медиана немного смещена влево (отрицательное
значение), а размах между 1Q и 3Q составляет примерно 9 единиц. Это
может указывать на легкую асимметрию, но выборка мала.

3. **Coefficients (Коэффициенты):**

\begin{itemize}
\item
  `(Intercept)`: свободный член (a0) = -240.766. Это предсказанное
  значение массы при длине тела, равной нулю. Биологически это не имеет
  смысла (длина не может быть нулевой), но это математическая
  особенность модели.
\item
  `lt`: коэффициент регрессии (a1) = 6.358. Это означает, что при
  увеличении длины тела на 1 см масса тела увеличивается в среднем на
  6.358 г.
\end{itemize}

Для каждого коэффициента приведены:

\begin{itemize}
\item
  `Estimate`: точечная оценка коэффициента.
\item
  `Std. Error`: стандартная ошибка оценки коэффициента. Для intercept =
  64.457, для lt = 1.153. Это мера изменчивости оценки.
\item
  `t value`: t-статистика. Рассчитывается как Estimate / Std.Error. Для
  intercept: -240.766 / 64.457 ≈ -3.735; для lt: 6.358 / 1.153 ≈ 5.516.
\item
  `Pr(\textgreater\textbar t\textbar)`: p-значение для проверки гипотезы
  о равенстве коэффициента нулю.
\item
  Для intercept: p=0.007308 (значим на уровне α=0.01, т.е. intercept
  статистически значимо отличается от нуля).
\item
  Для lt: p=0.000891 (значим на уровне α=0.001). Это означает, что длина
  тела значимо влияет на массу.
\end{itemize}

Значимость кодов: три звездочки (`***`) означают, что коэффициент значим
на уровне 0.001.

4. **Residual standard error (Стандартная ошибка остатков):** 13.81 на 7
степенях свободы. Это мера разброса остатков. В среднем, предсказания
модели отклоняются от реальных значений на ±13.81 г. Степени свободы
(df) = n - 2 = 9 - 2 = 7 (n --- количество наблюдений).

5. **Multiple R-squared (Коэффициент детерминации R²):** 0.813. Это
означает, что 81.3\% вариации массы тела объясняется длиной тела.
Остальные 18.7\% --- это неучтенные факторы и случайная изменчивость.

6. **Adjusted R-squared (Скорректированный R²):** 0.7863. Этот
показатель корректирует R² с учетом числа предикторов. Он полезен при
сравнении моделей с разным числом предикторов. Здесь он немного меньше
R², так как учитывает, что в модели один предиктор.

7. **F-statistic (F-статистика):** 30.43 на 1 и 7 степенях свободы.
Проверяет гипотезу о том, что все коэффициенты (кроме intercept) равны
нулю (т.е. модель не лучше, чем модель только с константой).

\begin{itemize}
\tightlist
\item
  p-value: 0.0008911 (крайне значимый), что означает, что модель в целом
  адекватна.
\end{itemize}

**Выводы:**

- Уравнение модели: `w = -240.77 + 6.36 * lt`

- Длина тела значимо влияет на массу (p\textless0.001).

- Модель объясняет 81.3\% вариации массы.

- На каждый сантиметр длины тела масса увеличивается примерно на 6.36 г.

- Остатки модели показывают, что есть несколько точек, которые модель
предсказывает с заметной ошибкой (особенно максимальный остаток в 30.6
г). Возможно, для более точного прогноза нужна нелинейная модель или
учет дополнительных факторов.

**Рекомендации:**

- Проверить допущения линейной регрессии (нормальность остатков,
гомоскедастичность) с помощью диагностических графиков.

- Рассмотреть возможность включения других переменных (например,
возраста, пола) в модель.

- Убедиться, что в данных нет выбросов, которые могут влиять на
коэффициенты.

\begin{Shaded}
\begin{Highlighting}[]
\CommentTok{\# Визуализация зависимости}
\FunctionTok{plot}\NormalTok{(lt, w, }
     \AttributeTok{main =} \StringTok{"Зависимость массы от длины тела гадюки"}\NormalTok{, }
     \AttributeTok{xlab =} \StringTok{"Длина тела (см)"}\NormalTok{, }
     \AttributeTok{ylab =} \StringTok{"Масса (г)"}\NormalTok{, }
     \AttributeTok{pch =} \DecValTok{19}\NormalTok{,        }\CommentTok{\# Кружки вместо стандартных точек}
     \AttributeTok{col =} \StringTok{"darkgreen"}\NormalTok{)}
\FunctionTok{abline}\NormalTok{(lreg, }\AttributeTok{col =} \StringTok{"red"}\NormalTok{, }\AttributeTok{lwd =} \DecValTok{2}\NormalTok{)  }\CommentTok{\# Добавляем линию регрессии}
\end{Highlighting}
\end{Shaded}

\begin{figure}[H]

{\centering \includegraphics[width=0.6\linewidth,height=\textheight,keepaspectratio]{images/KOROSOV1.PNG}

}

\caption{Рис. 1.: Пример линейной регрессии}

\end{figure}%

\section{ЧИСЛЕННАЯ
ОПТИМИЗАЦИЯ}\label{ux447ux438ux441ux43bux435ux43dux43dux430ux44f-ux43eux43fux442ux438ux43cux438ux437ux430ux446ux438ux44f}

Здесь вы познакомитесь с численными методами оптимизации параметров
моделей, которые применяются, когда аналитическое решение невозможно. На
примере той же зависимости массы от длины вы подгоните параметры модели
с помощью функции \texttt{nls()} и сравните результаты с аналитическим
решением.

Аналитические методы дают точное решение в виде математической формулы,
используя алгебраические преобразования и теоремы математического
анализа. Они идеальны для простых моделей, где существуют явные решения,
обеспечивая прозрачную интерпретацию параметров. В экологии такие методы
применимы для базовых зависимостей типа линейной регрессии. Численные
методы используются, когда аналитическое решение невозможно, и работают
через последовательные приближения, начиная со стартовых значений и
итеративно улучшая параметры модели. Они незаменимы для сложных
экологических моделей с нелинейными зависимостями, взаимодействиями
факторов и ``шумными'' полевыми данными, позволяя решать задачи,
недоступные для аналитических подходов.

\begin{Shaded}
\begin{Highlighting}[]
\CommentTok{\# Подгонка параметров через оптимизацию}
\NormalTok{nls\_model }\OtherTok{\textless{}{-}} \FunctionTok{nls}\NormalTok{(w }\SpecialCharTok{\textasciitilde{}}\NormalTok{ a0 }\SpecialCharTok{+}\NormalTok{ a1 }\SpecialCharTok{*}\NormalTok{ lt, }\AttributeTok{start =} \FunctionTok{list}\NormalTok{(}\AttributeTok{a0 =} \DecValTok{1}\NormalTok{, }\AttributeTok{a1 =} \DecValTok{1}\NormalTok{))}
\FunctionTok{summary}\NormalTok{(nls\_model)}
\end{Highlighting}
\end{Shaded}

На экране появится:

\begin{Shaded}
\begin{Highlighting}[]
\NormalTok{Formula}\SpecialCharTok{:}\NormalTok{ w }\SpecialCharTok{\textasciitilde{}}\NormalTok{ a0 }\SpecialCharTok{+}\NormalTok{ a1 }\SpecialCharTok{*}\NormalTok{ lt}

\NormalTok{Parameters}\SpecialCharTok{:}
\NormalTok{   Estimate Std. Error t value }\FunctionTok{Pr}\NormalTok{(}\SpecialCharTok{\textgreater{}}\ErrorTok{|}\NormalTok{t}\SpecialCharTok{|}\NormalTok{)    }
\NormalTok{a0 }\SpecialCharTok{{-}}\FloatTok{240.766}     \FloatTok{64.457}  \SpecialCharTok{{-}}\FloatTok{3.735} \FloatTok{0.007308} \SpecialCharTok{**} 
\NormalTok{a1    }\FloatTok{6.358}      \FloatTok{1.153}   \FloatTok{5.516} \FloatTok{0.000891} \SpecialCharTok{**}\ErrorTok{*}
\SpecialCharTok{{-}{-}{-}}
\NormalTok{Signif. codes}\SpecialCharTok{:}  \DecValTok{0}\NormalTok{ ‘}\SpecialCharTok{**}\ErrorTok{*}\NormalTok{’ }\FloatTok{0.001}\NormalTok{ ‘}\SpecialCharTok{**}\NormalTok{’ }\FloatTok{0.01}\NormalTok{ ‘}\SpecialCharTok{*}\NormalTok{’ }\FloatTok{0.05}\NormalTok{ ‘.’ }\FloatTok{0.1}\NormalTok{ ‘ ’ }\DecValTok{1}

\NormalTok{Residual standard error}\SpecialCharTok{:} \FloatTok{13.81}\NormalTok{ on }\DecValTok{7}\NormalTok{ degrees of freedom}

\NormalTok{Number of iterations to convergence}\SpecialCharTok{:} \DecValTok{1} 
\NormalTok{Achieved convergence tolerance}\SpecialCharTok{:} \FloatTok{3.247e{-}08}
\end{Highlighting}
\end{Shaded}

\subsection{\texorpdfstring{\textbf{Интерпретация результатов
модели}}{Интерпретация результатов модели}}\label{ux438ux43dux442ux435ux440ux43fux440ux435ux442ux430ux446ux438ux44f-ux440ux435ux437ux443ux43bux44cux442ux430ux442ux43eux432-ux43cux43eux434ux435ux43bux438}

Мы построили линейную модель зависимости массы гадюки (w) от длины её
тела (lt) по формуле:\\
\textbf{\texttt{w\ =\ a0\ +\ a1\ *\ lt}}

\textbf{Ключевые параметры модели:}

\begin{itemize}
\item
  \textbf{a0 (свободный член)}: -240.8 г\\
  Это теоретическая масса при нулевой длине тела. Отрицательное значение
  указывает, что модель не подходит для очень молодых особей.
\item
  \textbf{a1 (коэффициент при lt)}: 6.36 г/см\\
  Каждый дополнительный сантиметр длины тела увеличивает массу в среднем
  на 6.36 г.
\end{itemize}

\textbf{Точность и значимость:}

\begin{itemize}
\item
  Оба коэффициента \textbf{высоко значимы} (p \textless{} 0.01), что
  подтверждает реальность зависимости.
\item
  Стандартная ошибка для a1 составляет 1.15 г/см - это значит, что
  реальное значение, вероятно, находится между 5.2 и 7.5 г/см.
\item
  Модель хорошо сошлась за 1 шаг (итерацию), что говорит об удачном
  подборе параметров.
\end{itemize}

\textbf{Ошибка прогноза:}\\
Среднее отклонение предсказаний от реальных значений - 13.8 г
(стандартная ошибка остатков). Для особи массой 100 г это означает
возможную ошибку прогноза около 14\%.

\begin{quote}
\textbf{Биологический смысл:} Модель подтверждает сильную аллометрию -
крупные гадюки имеют относительно большую массу тела. Каждый сантиметр
длины добавляет около 6.4 г массы. Для особи длиной 55 см прогнозируемая
масса составит: -240.8 + 6.36*55 ≈ 109 г.
\end{quote}

\#\#МНОЖЕСТВЕННАЯ РЕГРЕССИЯ

В этом разделе мы расширим модель, включив несколько факторов. Вы
построите множественную регрессию, учитывающую одновременно длину тела и
длину хвоста гадюки, и научитесь интерпретировать влияние нескольких
предикторов на зависимую переменную.

\begin{Shaded}
\begin{Highlighting}[]
\CommentTok{\# Добавляем новый признак {-} длину хвоста (lc)}
\NormalTok{w }\OtherTok{\textless{}{-}} \FunctionTok{c}\NormalTok{(}\DecValTok{40}\NormalTok{, }\DecValTok{156}\NormalTok{, }\DecValTok{105}\NormalTok{, }\DecValTok{85}\NormalTok{, }\DecValTok{80}\NormalTok{, }\DecValTok{50}\NormalTok{, }\DecValTok{75}\NormalTok{, }\DecValTok{48}\NormalTok{, }\DecValTok{75}\NormalTok{, }\DecValTok{67}\NormalTok{)}
\NormalTok{lt }\OtherTok{\textless{}{-}} \FunctionTok{c}\NormalTok{(}\DecValTok{44}\NormalTok{, }\DecValTok{59}\NormalTok{, }\DecValTok{49}\NormalTok{, }\DecValTok{50}\NormalTok{, }\DecValTok{54}\NormalTok{, }\DecValTok{43}\NormalTok{, }\DecValTok{49}\NormalTok{, }\DecValTok{42}\NormalTok{, }\DecValTok{47}\NormalTok{, }\DecValTok{47}\NormalTok{)}
\NormalTok{lc }\OtherTok{\textless{}{-}} \FunctionTok{c}\NormalTok{(}\DecValTok{70}\NormalTok{, }\DecValTok{78}\NormalTok{, }\DecValTok{66}\NormalTok{, }\DecValTok{90}\NormalTok{, }\DecValTok{83}\NormalTok{, }\DecValTok{70}\NormalTok{, }\DecValTok{62}\NormalTok{, }\DecValTok{75}\NormalTok{, }\DecValTok{40}\NormalTok{, }\DecValTok{80}\NormalTok{)}
\end{Highlighting}
\end{Shaded}

Используя glm-функцию, построим модель с двумя предикторами: \[
w = a_0 + a_1 \cdot l_t + a_2 \cdot l_c
\]

где: - \(w\) --- масса гадюки, - \(l_t\) --- длина тела гадюки, -
\(l_c\) --- длина хвоста гадюки, - \(a_0\) --- свободный член
(константа), - \(a_1\) --- коэффициент регрессии при длине тела, -
\(a_2\) --- коэффициент регрессии при длине хвоста.

\begin{Shaded}
\begin{Highlighting}[]
\CommentTok{\# Множественная регрессия: w = a0 + a1*lt + a2*lc}
\NormalTok{multi\_reg }\OtherTok{\textless{}{-}} \FunctionTok{glm}\NormalTok{(w }\SpecialCharTok{\textasciitilde{}}\NormalTok{ lt }\SpecialCharTok{+}\NormalTok{ lc)}
\FunctionTok{summary}\NormalTok{(multi\_reg)}
\end{Highlighting}
\end{Shaded}

На экране появится:

\begin{Shaded}
\begin{Highlighting}[]
\NormalTok{Call}\SpecialCharTok{:}
\FunctionTok{glm}\NormalTok{(}\AttributeTok{formula =}\NormalTok{ w }\SpecialCharTok{\textasciitilde{}}\NormalTok{ lt }\SpecialCharTok{+}\NormalTok{ lc)}

\NormalTok{Coefficients}\SpecialCharTok{:}
\NormalTok{             Estimate Std. Error t value }\FunctionTok{Pr}\NormalTok{(}\SpecialCharTok{\textgreater{}}\ErrorTok{|}\NormalTok{t}\SpecialCharTok{|}\NormalTok{)    }
\NormalTok{(Intercept) }\SpecialCharTok{{-}}\FloatTok{191.2982}    \FloatTok{53.6908}  \SpecialCharTok{{-}}\FloatTok{3.563} \FloatTok{0.009183} \SpecialCharTok{**} 
\NormalTok{lt             }\FloatTok{6.0308}     \FloatTok{1.1051}   \FloatTok{5.457} \FloatTok{0.000949} \SpecialCharTok{**}\ErrorTok{*}
\NormalTok{lc            }\SpecialCharTok{{-}}\FloatTok{0.3150}     \FloatTok{0.4133}  \SpecialCharTok{{-}}\FloatTok{0.762} \FloatTok{0.470913}    
\SpecialCharTok{{-}{-}{-}}
\NormalTok{Signif. codes}\SpecialCharTok{:}  \DecValTok{0}\NormalTok{ ‘}\SpecialCharTok{**}\ErrorTok{*}\NormalTok{’ }\FloatTok{0.001}\NormalTok{ ‘}\SpecialCharTok{**}\NormalTok{’ }\FloatTok{0.01}\NormalTok{ ‘}\SpecialCharTok{*}\NormalTok{’ }\FloatTok{0.05}\NormalTok{ ‘.’ }\FloatTok{0.1}\NormalTok{ ‘ ’ }\DecValTok{1}

\NormalTok{(Dispersion parameter }\ControlFlowTok{for}\NormalTok{ gaussian family taken to be }\FloatTok{270.9752}\NormalTok{)}

\NormalTok{    Null deviance}\SpecialCharTok{:} \FloatTok{10132.9}\NormalTok{  on }\DecValTok{9}\NormalTok{  degrees of freedom}
\NormalTok{Residual deviance}\SpecialCharTok{:}  \FloatTok{1896.8}\NormalTok{  on }\DecValTok{7}\NormalTok{  degrees of freedom}
\NormalTok{AIC}\SpecialCharTok{:} \FloatTok{88.832}

\NormalTok{Number of Fisher Scoring iterations}\SpecialCharTok{:} \DecValTok{2}
\end{Highlighting}
\end{Shaded}

\subsection{\texorpdfstring{\textbf{Интерпретация результатов
множественной
регрессии}}{Интерпретация результатов множественной регрессии}}\label{ux438ux43dux442ux435ux440ux43fux440ux435ux442ux430ux446ux438ux44f-ux440ux435ux437ux443ux43bux44cux442ux430ux442ux43eux432-ux43cux43dux43eux436ux435ux441ux442ux432ux435ux43dux43dux43eux439-ux440ux435ux433ux440ux435ux441ux441ux438ux438}

Мы исследовали зависимость массы гадюки (w) от длины тела (lt) и длины
хвоста (lc) с помощью модели:\\
\textbf{\texttt{w\ =\ b0\ +\ b1*lt\ +\ b2*lc}}

\textbf{Ключевые выводы модели:}

\begin{enumerate}
\def\labelenumi{\arabic{enumi}.}
\item
  \textbf{Длина тела (lt) сильно влияет на массу}:

  \begin{itemize}
  \item
    Коэффициент: +6.03 г/см
  \item
    Каждый сантиметр длины тела увеличивает массу на \textasciitilde6 г
  \item
    Высокая значимость (p = 0.00095)
  \end{itemize}
\item
  \textbf{Длина хвоста (lc) не влияет значимо на массу}:

  \begin{itemize}
  \item
    Коэффициент: -0.315 г/см (незначимый)
  \item
    p-значение 0.47 \textgreater{} 0.05 - статистически недостоверно
  \item
    После учета длины тела, длина хвоста не добавляет информации
  \end{itemize}
\item
  \textbf{Свободный член (b0)}: -191.3 г\\
  Отрицательное значение подтверждает нелинейность роста у молодых
  особей
\end{enumerate}

\textbf{Качество модели:}

\begin{itemize}
\item
  Модель объясняет значительную часть вариации:\\
  Общая вариация (Null deviance) = 10132.9\\
  Остаточная вариация (Residual deviance) = 1896.8 → \textbf{Объяснено
  81\% вариации}
\item
  AIC = 88.8 (ниже, чем у модели без lc - 92.1, что указывает на лучшее
  качество)
\item
  Модель быстро сошлась за 2 итерации
\end{itemize}

\textbf{Биологическая интерпретация:}

\begin{enumerate}
\def\labelenumi{\arabic{enumi}.}
\item
  Масса тела определяется в основном длиной туловища, а не хвоста
\item
  Для прогноза массы достаточно учитывать только длину тела
\item
  Пример прогноза для особи (lt=50 см, lc=70 см):\\
  \textbf{\texttt{-191.3\ +\ 6.03*50\ -\ 0.315*70\ ≈\ 111\ г}}
\end{enumerate}

\begin{quote}
\textbf{Рекомендация}: При изучении массы гадюк можно исключить длину
хвоста из модели, так как она не вносит значимого вклада в предсказание.
Основным морфометрическим показателем остается длина тела.
\end{quote}

\section{НЕЛИНЕЙНЫЕ
ЗАВИСИМОСТИ}\label{ux43dux435ux43bux438ux43dux435ux439ux43dux44bux435-ux437ux430ux432ux438ux441ux438ux43cux43eux441ux442ux438}

Экологические данные часто имеют нелинейный характер. Здесь вы
смоделируете степенную зависимость (аллометрию) между массой и длиной
тела, используя линеаризацию через логарифмирование, а затем
визуализируете кривую модели.

\begin{Shaded}
\begin{Highlighting}[]
\CommentTok{\# Часто в экологии связи имеют степенной характер: w = a0 * lt\^{}a1}
\CommentTok{\# Линеаризация через логарифмирование}
\NormalTok{log\_model }\OtherTok{\textless{}{-}} \FunctionTok{lm}\NormalTok{(}\FunctionTok{log}\NormalTok{(w) }\SpecialCharTok{\textasciitilde{}} \FunctionTok{log}\NormalTok{(lt))}

\CommentTok{\# Преобразование коэффициентов обратно}
\NormalTok{a0 }\OtherTok{\textless{}{-}} \FunctionTok{exp}\NormalTok{(}\FunctionTok{coef}\NormalTok{(log\_model)[}\DecValTok{1}\NormalTok{])  }\CommentTok{\# Переход от логарифмов}
\NormalTok{a1 }\OtherTok{\textless{}{-}} \FunctionTok{coef}\NormalTok{(log\_model)[}\DecValTok{2}\NormalTok{]       }\CommentTok{\# Показатель степени}

\CommentTok{\# Визуализация степенной зависимости}
\FunctionTok{plot}\NormalTok{(lt, w, }
     \AttributeTok{main =} \StringTok{"Степенная зависимость массы от длины"}\NormalTok{, }
     \AttributeTok{xlab =} \StringTok{"Длина тела (см)"}\NormalTok{, }
     \AttributeTok{ylab =} \StringTok{"Масса (г)"}\NormalTok{,}
     \AttributeTok{pch =} \DecValTok{17}\NormalTok{,}
     \AttributeTok{col =} \StringTok{"blue"}\NormalTok{)}
\FunctionTok{curve}\NormalTok{(a0 }\SpecialCharTok{*}\NormalTok{ x}\SpecialCharTok{\^{}}\NormalTok{a1, }\AttributeTok{add =} \ConstantTok{TRUE}\NormalTok{, }\AttributeTok{col =} \StringTok{"red"}\NormalTok{, }\AttributeTok{lwd =} \DecValTok{2}\NormalTok{)  }\CommentTok{\# Кривая модели}
\end{Highlighting}
\end{Shaded}

\begin{figure}[H]

{\centering \includegraphics[width=0.6\linewidth,height=\textheight,keepaspectratio]{images/KOROSOV2.PNG}

}

\caption{Рис. 2.: Расчет степенной функции}

\end{figure}%

\section{ЛОГИСТИЧЕСКАЯ
РЕГРЕССИЯ}\label{ux43bux43eux433ux438ux441ux442ux438ux447ux435ux441ux43aux430ux44f-ux440ux435ux433ux440ux435ux441ux441ux438ux44f}

Вы изучите моделирование пороговых эффектов в экологии на примере
смертности дафний в зависимости от концентрации токсиканта. Построив
логистическую регрессию, вы получите S-образную кривую, характерную для
таких процессов.

\begin{Shaded}
\begin{Highlighting}[]
\CommentTok{\# Пример: смертность дафний при разных концентрациях токсиканта}
\CommentTok{\# Данные:}
\NormalTok{K }\OtherTok{\textless{}{-}} \FunctionTok{c}\NormalTok{(}\DecValTok{100}\NormalTok{, }\DecValTok{126}\NormalTok{, }\DecValTok{158}\NormalTok{, }\DecValTok{200}\NormalTok{, }\DecValTok{251}\NormalTok{, }\DecValTok{316}\NormalTok{, }\DecValTok{398}\NormalTok{, }\DecValTok{501}\NormalTok{, }\DecValTok{631}\NormalTok{, }\DecValTok{794}\NormalTok{, }\DecValTok{1000}\NormalTok{)}
\NormalTok{p }\OtherTok{\textless{}{-}} \FunctionTok{c}\NormalTok{(}\DecValTok{0}\NormalTok{, }\DecValTok{0}\NormalTok{, }\DecValTok{0}\NormalTok{, }\DecValTok{0}\NormalTok{, }\DecValTok{0}\NormalTok{, }\FloatTok{0.5}\NormalTok{, }\FloatTok{0.5}\NormalTok{, }\DecValTok{1}\NormalTok{, }\DecValTok{1}\NormalTok{, }\DecValTok{1}\NormalTok{, }\DecValTok{1}\NormalTok{)  }\CommentTok{\# Доля погибших}
\NormalTok{d }\OtherTok{\textless{}{-}} \FunctionTok{data.frame}\NormalTok{(K, p)}

\CommentTok{\# Построение логистической модели}
\NormalTok{logit\_model }\OtherTok{\textless{}{-}} \FunctionTok{glm}\NormalTok{(p }\SpecialCharTok{\textasciitilde{}}\NormalTok{ K, }\AttributeTok{family =} \FunctionTok{binomial}\NormalTok{(), }\AttributeTok{data =}\NormalTok{ d)}

\CommentTok{\# Визуализация S{-}образной кривой}
\FunctionTok{plot}\NormalTok{(d}\SpecialCharTok{$}\NormalTok{K, d}\SpecialCharTok{$}\NormalTok{p, }
     \AttributeTok{xlab =} \StringTok{"Концентрация токсиканта (мг/л)"}\NormalTok{, }
     \AttributeTok{ylab =} \StringTok{"Доля погибших"}\NormalTok{, }
     \AttributeTok{main =} \StringTok{"Токсическое воздействие на дафний"}\NormalTok{,}
     \AttributeTok{pch =} \DecValTok{19}\NormalTok{,}
     \AttributeTok{col =} \StringTok{"red"}\NormalTok{)}
\FunctionTok{lines}\NormalTok{(d}\SpecialCharTok{$}\NormalTok{K, }\FunctionTok{predict}\NormalTok{(logit\_model, }\AttributeTok{type =} \StringTok{"response"}\NormalTok{), }
      \AttributeTok{col =} \StringTok{"blue"}\NormalTok{, }\AttributeTok{lwd =} \DecValTok{2}\NormalTok{, }\AttributeTok{lty =} \DecValTok{1}\NormalTok{)}
\end{Highlighting}
\end{Shaded}

\begin{figure}[H]

{\centering \includegraphics[width=0.6\linewidth,height=\textheight,keepaspectratio]{images/KOROSOV3.PNG}

}

\caption{Рис. 3.: Расчет логистической регрессии гибели дафний в
токсиканте}

\end{figure}%

\section{ПЕРЕХОД К
СЕТЯМ}\label{ux43fux435ux440ux435ux445ux43eux434-ux43a-ux441ux435ux442ux44fux43c}

Сделаем первый шаг к нейронным сетям, построив простейшую сеть без
скрытых слоев (аналог линейной регрессии) для модели токсичности. Вы
визуализируете структуру сети и убедитесь, что она дает результат,
аналогичный линейной модели.

\begin{Shaded}
\begin{Highlighting}[]
\CommentTok{\# Простейшая нейросеть (аналог линейной регрессии)}
\NormalTok{nn\_simple }\OtherTok{\textless{}{-}} \FunctionTok{neuralnet}\NormalTok{(p }\SpecialCharTok{\textasciitilde{}}\NormalTok{ K, }\AttributeTok{data =}\NormalTok{ d, }\AttributeTok{hidden =} \DecValTok{0}\NormalTok{)}

\CommentTok{\# Визуализация структуры сети}
\FunctionTok{plot}\NormalTok{(nn\_simple, }\AttributeTok{rep =} \StringTok{"best"}\NormalTok{)}
\end{Highlighting}
\end{Shaded}

\begin{figure}[H]

{\centering \includegraphics[width=0.4\linewidth,height=\textheight,keepaspectratio]{images/KOROSOV4.PNG}

}

\caption{Рис. 4.: Схема нейрона}

\end{figure}%

\section{НЕЙРОНЫ КАК НЕЛИНЕЙНЫЕ
ПРЕОБРАЗОВАТЕЛИ}\label{ux43dux435ux439ux440ux43eux43dux44b-ux43aux430ux43a-ux43dux435ux43bux438ux43dux435ux439ux43dux44bux435-ux43fux440ux435ux43eux431ux440ux430ux437ux43eux432ux430ux442ux435ux43bux438}

Здесь вы добавите в нейронную сеть скрытый слой с одним нейроном, что
позволит моделировать нелинейные зависимости. Вы сравните результат
работы такой сети с логистической регрессией и увидите, как нейронная
сеть имитирует пороговый эффект.

\begin{Shaded}
\begin{Highlighting}[]
\CommentTok{\# Сеть с одним скрытым нейроном (имитирует логистическую регрессию)}
\NormalTok{nn\_1hidden }\OtherTok{\textless{}{-}} \FunctionTok{neuralnet}\NormalTok{(p }\SpecialCharTok{\textasciitilde{}}\NormalTok{ K, }\AttributeTok{data =}\NormalTok{ d, }\AttributeTok{hidden =} \DecValTok{1}\NormalTok{)}

\CommentTok{\# Сравнение с логистической регрессией}
\FunctionTok{plot}\NormalTok{(d}\SpecialCharTok{$}\NormalTok{K, }\FunctionTok{predict}\NormalTok{(logit\_model, }\AttributeTok{type =} \StringTok{"response"}\NormalTok{), }
     \AttributeTok{type =} \StringTok{"l"}\NormalTok{, }
     \AttributeTok{col =} \StringTok{"darkgreen"}\NormalTok{, }
     \AttributeTok{lwd =} \DecValTok{2}\NormalTok{,}
     \AttributeTok{xlab =} \StringTok{"Концентрация"}\NormalTok{, }
     \AttributeTok{ylab =} \StringTok{"Смертность"}\NormalTok{,}
     \AttributeTok{main =} \StringTok{"Сравнение моделей"}\NormalTok{)}
\FunctionTok{lines}\NormalTok{(d}\SpecialCharTok{$}\NormalTok{K, }\FunctionTok{predict}\NormalTok{(nn\_1hidden, d), }\AttributeTok{col =} \StringTok{"blue"}\NormalTok{, }\AttributeTok{lty =} \DecValTok{2}\NormalTok{, }\AttributeTok{lwd =} \DecValTok{2}\NormalTok{)}
\FunctionTok{legend}\NormalTok{(}\StringTok{"bottomright"}\NormalTok{, }
       \AttributeTok{legend =} \FunctionTok{c}\NormalTok{(}\StringTok{"Логистическая регрессия"}\NormalTok{, }\StringTok{"Нейронная сеть (1 нейрон)"}\NormalTok{),}
       \AttributeTok{col =} \FunctionTok{c}\NormalTok{(}\StringTok{"darkgreen"}\NormalTok{, }\StringTok{"blue"}\NormalTok{), }
       \AttributeTok{lty =} \DecValTok{1}\SpecialCharTok{:}\DecValTok{2}\NormalTok{,}
       \AttributeTok{lwd =} \DecValTok{2}\NormalTok{)}
\end{Highlighting}
\end{Shaded}

\begin{figure}[H]

{\centering \includegraphics[width=0.4\linewidth,height=\textheight,keepaspectratio]{images/KOROSOV5.PNG}

}

\caption{Рис. 5.: Сравнение работы}

\end{figure}%

\section{КЛАССИФИКАЦИЯ В
ЭКОЛОГИИ}\label{ux43aux43bux430ux441ux441ux438ux444ux438ux43aux430ux446ux438ux44f-ux432-ux44dux43aux43eux43bux43eux433ux438ux438}

Вы примените нейронные сети для решения задачи классификации -
определения пола гадюк по морфометрическим признакам. Построив и сравнив
несколько архитектур сетей (без скрытых нейронов, с одним и тремя
нейронами), вы оцените их точность.

\begin{Shaded}
\begin{Highlighting}[]
\CommentTok{\# Загрузка данных по гадюкам (пол, длина тела, длина хвоста, масса)}
\NormalTok{v }\OtherTok{\textless{}{-}} \FunctionTok{read.csv}\NormalTok{(}\StringTok{"vipkar.csv"}\NormalTok{)}
\FunctionTok{head}\NormalTok{(v, }\DecValTok{3}\NormalTok{)  }\CommentTok{\# Просмотр первых строк данных}
\end{Highlighting}
\end{Shaded}

Модель без скрытых нейронов (аналог линейной регрессии)

\begin{Shaded}
\begin{Highlighting}[]
\NormalTok{nv0 }\OtherTok{\textless{}{-}} \FunctionTok{neuralnet}\NormalTok{(ns }\SpecialCharTok{\textasciitilde{}}\NormalTok{ lc, }\AttributeTok{data =}\NormalTok{ v, }\AttributeTok{hidden =} \DecValTok{0}\NormalTok{)}
\FunctionTok{plot}\NormalTok{(nv0)  }\CommentTok{\# Визуализация простейшей сети}
\end{Highlighting}
\end{Shaded}

\begin{figure}[H]

{\centering \includegraphics[width=0.4\linewidth,height=\textheight,keepaspectratio]{images/KOROSOV6.PNG}

}

\caption{Рис. 6.: Визуализация простейшей сети}

\end{figure}%

Модель с одним скрытым нейроном

\begin{Shaded}
\begin{Highlighting}[]
\NormalTok{nv1 }\OtherTok{\textless{}{-}} \FunctionTok{neuralnet}\NormalTok{(ns }\SpecialCharTok{\textasciitilde{}}\NormalTok{ lc, }\AttributeTok{data =}\NormalTok{ v, }\AttributeTok{hidden =} \DecValTok{1}\NormalTok{)}
\FunctionTok{plot}\NormalTok{(nv1)  }\CommentTok{\# Схема сети с одним нейроном}
\end{Highlighting}
\end{Shaded}

\begin{figure}[H]

{\centering \includegraphics[width=0.4\linewidth,height=\textheight,keepaspectratio]{images/KOROSOV7.PNG}

}

\caption{Рис. 7.: Схема сети с одним нейроном}

\end{figure}%

Модель с тремя скрытыми нейронами (полноценная нейросеть)

\begin{Shaded}
\begin{Highlighting}[]
\NormalTok{nv3 }\OtherTok{\textless{}{-}} \FunctionTok{neuralnet}\NormalTok{(ns }\SpecialCharTok{\textasciitilde{}}\NormalTok{ lc }\SpecialCharTok{+}\NormalTok{ lt }\SpecialCharTok{+}\NormalTok{ w, }\AttributeTok{data =}\NormalTok{ v, }\AttributeTok{hidden =} \DecValTok{3}\NormalTok{)}
\FunctionTok{plot}\NormalTok{(nv3)  }\CommentTok{\# Визуализация сложной сети}
\end{Highlighting}
\end{Shaded}

\begin{figure}[H]

{\centering \includegraphics[width=0.4\linewidth,height=\textheight,keepaspectratio]{images/KOROSOV8.PNG}

}

\caption{Рис. 8.: Модель с тремя скрытыми нейронами}

\end{figure}%

Оценка точности классификации

\begin{Shaded}
\begin{Highlighting}[]
\NormalTok{predictions }\OtherTok{\textless{}{-}} \FunctionTok{predict}\NormalTok{(nv3, v)}
\NormalTok{predicted\_sex }\OtherTok{\textless{}{-}} \FunctionTok{round}\NormalTok{(predictions)}
\NormalTok{accuracy }\OtherTok{\textless{}{-}} \FunctionTok{mean}\NormalTok{(v}\SpecialCharTok{$}\NormalTok{ns }\SpecialCharTok{==}\NormalTok{ predicted\_sex)}
\FunctionTok{cat}\NormalTok{(}\StringTok{"Точность классификации:"}\NormalTok{, }\FunctionTok{round}\NormalTok{(accuracy}\SpecialCharTok{*}\DecValTok{100}\NormalTok{, }\DecValTok{1}\NormalTok{), }\StringTok{"\%}\SpecialCharTok{\textbackslash{}n}\StringTok{"}\NormalTok{)}
\end{Highlighting}
\end{Shaded}

Сравнение разных архитектур нейронных сетей (см. срипт
\href{https://mombus.github.io/cRab/data/KOROSOV_visual.R}{KOROSOV\_visual.R})

\begin{figure}[H]

{\centering \includegraphics[width=0.6\linewidth,height=\textheight,keepaspectratio]{images/KOROSOV9.PNG}

}

\caption{Рис. 9.: Точность определения пола гадюк}

\end{figure}%

\section{ПРОСТРАНСТВЕННОЕ
МОДЕЛИРОВАНИЕ}\label{ux43fux440ux43eux441ux442ux440ux430ux43dux441ux442ux432ux435ux43dux43dux43eux435-ux43cux43eux434ux435ux43bux438ux440ux43eux432ux430ux43dux438ux435}

В завершение вы построите нейронную сеть для прогнозирования численности
гадюк на островах по характеристикам биотопов. Вы разделите данные на
обучающую и тестовую выборки, оцените точность модели и используете ее
для прогноза в новых условиях.

\begin{Shaded}
\begin{Highlighting}[]
\CommentTok{\# Данные по островам Кижского архипелага}
\NormalTok{v }\OtherTok{\textless{}{-}} \FunctionTok{read.csv}\NormalTok{(}\StringTok{"kihzsdat.csv"}\NormalTok{)}
\FunctionTok{head}\NormalTok{(v, }\DecValTok{3}\NormalTok{)  }\CommentTok{\# Структура данных: площадь, биотопы, численность видов}

\CommentTok{\# Случайное разделение данных на обучающую и тестовую выборки}
\FunctionTok{set.seed}\NormalTok{(}\DecValTok{123}\NormalTok{)  }\CommentTok{\# Для воспроизводимости}
\NormalTok{train\_indices }\OtherTok{\textless{}{-}} \FunctionTok{sample}\NormalTok{(}\DecValTok{1}\SpecialCharTok{:}\FunctionTok{nrow}\NormalTok{(v), }\DecValTok{12}\NormalTok{)}
\NormalTok{train\_data }\OtherTok{\textless{}{-}}\NormalTok{ v[train\_indices, ]}
\NormalTok{test\_data }\OtherTok{\textless{}{-}}\NormalTok{ v[}\SpecialCharTok{{-}}\NormalTok{train\_indices, ]}

\CommentTok{\# Построение нейросети с 5 нейронами в скрытом слое}
\NormalTok{model }\OtherTok{\textless{}{-}} \FunctionTok{neuralnet}\NormalTok{(vb }\SpecialCharTok{\textasciitilde{}}\NormalTok{ fo }\SpecialCharTok{+}\NormalTok{ me }\SpecialCharTok{+}\NormalTok{ bo, }\AttributeTok{data =}\NormalTok{ train\_data, }\AttributeTok{hidden =} \DecValTok{5}\NormalTok{)}

\CommentTok{\# Прогнозирование на обучающей выборке}
\NormalTok{train\_pred }\OtherTok{\textless{}{-}} \FunctionTok{predict}\NormalTok{(model, train\_data)}
\NormalTok{train\_accuracy }\OtherTok{\textless{}{-}} \FunctionTok{mean}\NormalTok{(}\FunctionTok{round}\NormalTok{(train\_pred) }\SpecialCharTok{==}\NormalTok{ train\_data}\SpecialCharTok{$}\NormalTok{vb)}
\FunctionTok{cat}\NormalTok{(}\StringTok{"Точность на обучающей выборке:"}\NormalTok{, }\FunctionTok{round}\NormalTok{(train\_accuracy}\SpecialCharTok{*}\DecValTok{100}\NormalTok{, }\DecValTok{1}\NormalTok{), }\StringTok{"\%}\SpecialCharTok{\textbackslash{}n}\StringTok{"}\NormalTok{)}

\CommentTok{\# Прогнозирование на тестовой выборке}
\NormalTok{test\_pred }\OtherTok{\textless{}{-}} \FunctionTok{predict}\NormalTok{(model, test\_data)}
\NormalTok{test\_accuracy }\OtherTok{\textless{}{-}} \FunctionTok{mean}\NormalTok{(}\FunctionTok{round}\NormalTok{(test\_pred) }\SpecialCharTok{==}\NormalTok{ test\_data}\SpecialCharTok{$}\NormalTok{vb)}
\FunctionTok{cat}\NormalTok{(}\StringTok{"Точность на тестовой выборке:"}\NormalTok{, }\FunctionTok{round}\NormalTok{(test\_accuracy}\SpecialCharTok{*}\DecValTok{100}\NormalTok{, }\DecValTok{1}\NormalTok{), }\StringTok{"\%}\SpecialCharTok{\textbackslash{}n}\StringTok{"}\NormalTok{)}

\CommentTok{\# Прогноз для новых условий (пример)}
\NormalTok{new\_conditions }\OtherTok{\textless{}{-}} \FunctionTok{data.frame}\NormalTok{(}
  \AttributeTok{fo =} \FunctionTok{c}\NormalTok{(}\FloatTok{57.9}\NormalTok{, }\FloatTok{35.3}\NormalTok{, }\FloatTok{83.0}\NormalTok{),  }\CommentTok{\# Площадь лесов (\%)}
  \AttributeTok{me =} \FunctionTok{c}\NormalTok{(}\FloatTok{4.1}\NormalTok{, }\FloatTok{0.0}\NormalTok{, }\FloatTok{7.3}\NormalTok{),     }\CommentTok{\# Площадь лугов (\%)}
  \AttributeTok{bo =} \FunctionTok{c}\NormalTok{(}\FloatTok{3.4}\NormalTok{, }\FloatTok{7.9}\NormalTok{, }\FloatTok{11.5}\NormalTok{)     }\CommentTok{\# Площадь болот (\%)}
\NormalTok{)}

\NormalTok{future\_pred }\OtherTok{\textless{}{-}} \FunctionTok{predict}\NormalTok{(model, new\_conditions)}
\FunctionTok{cat}\NormalTok{(}\StringTok{"Прогнозируемая численность гадюк:"}\NormalTok{, }\FunctionTok{round}\NormalTok{(future\_pred), }\StringTok{"}\SpecialCharTok{\textbackslash{}n}\StringTok{"}\NormalTok{)}
\end{Highlighting}
\end{Shaded}

\bookmarksetup{startatroot}

\chapter{Основы
картографии}\label{ux43eux441ux43dux43eux432ux44b-ux43aux430ux440ux442ux43eux433ux440ux430ux444ux438ux438}

\section{Введение}\label{ux432ux432ux435ux434ux435ux43dux438ux435-2}

Примеры карт и скрипты, которые могут быть полезны в рыбохозяйственных,
гидробиологическских и пр. океанологических исследованиях. На этом
занятии мы рассмотрим создание карт для визуализации пространственных
данных в исследованиях водных биоресурсов.

Вы научитесь создавать различные типы карт, используя современные пакеты
R, такие как \texttt{ggplot2}, \texttt{sf}, \texttt{rnaturalearth} и
другие. Мы начнем с простых карт распределения уловов по данным съемок,
затем перейдем к более сложным: картам с береговой линией, картам,
включающим нулевые уловы, распределению по квартилям. Далее рассмотрим
фасеточные карты для сравнения лет, карты с пространственной
автокорреляцией (LISA), а также промысловые карты (включая картограммы)
и гибридные карты, объединяющие данные съемок и промысла. В заключение
мы покажем, как создавать карты для раздела ``Материал и методы'' и
карты с врезками.

Обратите внимание, что примеры карт, представленные в интернете
(например, в HTML-версии этого документа), могут быть в уменьшенном
качестве. Однако при работе в R вы можете экспортировать полученные
графики в векторные (PDF, SVG) или растровые (PNG, TIFF) форматы с
высоким разрешением (до 600 dpi), что подходит для публикаций в научных
журналах.

\textbf{Для работы скрипта:}

\begin{enumerate}
\def\labelenumi{\arabic{enumi}.}
\item
  Скачайте файл данных
  (\href{https://mombus.github.io/cRab/data/KARTOGRAPHIC.xlsx}{KARTOGRAPHIC.xlsx})
\item
  Установите рабочую директорию в setwd()
\item
  Установите необходимые пакеты :
  \textbf{\texttt{install.packages(c("readxl",\ "tidyverse,\ "rnaturalearth",\ "sf",\ "viridis"\ ))}}
  \texttt{и\ др.}
\end{enumerate}

\section{Карта распределения уловов в
съемке}\label{ux43aux430ux440ux442ux430-ux440ux430ux441ux43fux440ux435ux434ux435ux43bux435ux43dux438ux44f-ux443ux43bux43eux432ux43eux432-ux432-ux441ux44aux435ux43cux43aux435}

Данная карта демонстрирует распределение уловов краба в ходе
исследовательской съемки. На ней отображены точки наблюдений, где размер
и цвет точек соответствуют величине улова.

\begin{figure}[H]

{\centering \includegraphics[width=0.7\linewidth,height=\textheight,keepaspectratio]{images/KARTOGRAPH1.jpg}

}

\caption{Рис. 1.: Пример карты распределения уловов в съемке}

\end{figure}%

В скрипте границы карты (лимиты) определяются автоматически с буфером,
но чаще их просто устанавливают вручную, например:

\begin{Shaded}
\begin{Highlighting}[]
\NormalTok{xmin }\OtherTok{\textless{}{-}} \DecValTok{37}
\NormalTok{xmax }\OtherTok{\textless{}{-}} \DecValTok{49}
\NormalTok{ymin }\OtherTok{\textless{}{-}} \FloatTok{68.5}
\NormalTok{ymax }\OtherTok{\textless{}{-}} \FloatTok{70.5}
\end{Highlighting}
\end{Shaded}

Скрипт карты целиком:

\begin{Shaded}
\begin{Highlighting}[]
\CommentTok{\# Очистка памяти и установка рабочей папки}
\FunctionTok{rm}\NormalTok{(}\AttributeTok{list =} \FunctionTok{ls}\NormalTok{())}
\FunctionTok{setwd}\NormalTok{(}\StringTok{"C:/COURSES/KARTOGRAPH/"}\NormalTok{)}

\CommentTok{\# Загрузка необходимых пакетов}
\FunctionTok{library}\NormalTok{(tidyverse)  }\CommentTok{\# Обработка данных и визуализация}
\FunctionTok{library}\NormalTok{(readxl)     }\CommentTok{\# Чтение Excel{-}файлов}

\CommentTok{\# 1. ЗАГРУЗКА ДАННЫХ}
\NormalTok{DATA }\OtherTok{\textless{}{-}} \FunctionTok{read\_excel}\NormalTok{(}\StringTok{"KARTOGRAPHIC.xlsx"}\NormalTok{, }\AttributeTok{sheet =} \StringTok{"SURVEY"}\NormalTok{) }\SpecialCharTok{\%\textgreater{}\%} 
  \FunctionTok{filter}\NormalTok{(YEAR }\SpecialCharTok{==} \DecValTok{2023}\NormalTok{, SURV }\SpecialCharTok{==} \StringTok{"CRAB"}\NormalTok{)  }\CommentTok{\# Фильтр для 2023 года и съемки CRAB}

\CommentTok{\# 2. АВТОМАТИЧЕСКИЙ РАСЧЕТ ГРАНИЦ С БУФЕРОМ 5\%}
\CommentTok{\# Расчет диапазонов координат}
\NormalTok{x\_range }\OtherTok{\textless{}{-}} \FunctionTok{range}\NormalTok{(DATA}\SpecialCharTok{$}\NormalTok{X, }\AttributeTok{na.rm =} \ConstantTok{TRUE}\NormalTok{)}
\NormalTok{y\_range }\OtherTok{\textless{}{-}} \FunctionTok{range}\NormalTok{(DATA}\SpecialCharTok{$}\NormalTok{Y, }\AttributeTok{na.rm =} \ConstantTok{TRUE}\NormalTok{)}

\CommentTok{\# Расчет 5\% буфера}
\NormalTok{x\_buffer }\OtherTok{\textless{}{-}} \FloatTok{0.05} \SpecialCharTok{*} \FunctionTok{diff}\NormalTok{(x\_range)}
\NormalTok{y\_buffer }\OtherTok{\textless{}{-}} \FloatTok{0.05} \SpecialCharTok{*} \FunctionTok{diff}\NormalTok{(y\_range)}

\CommentTok{\# Установка границ с буфером}
\NormalTok{xmin }\OtherTok{\textless{}{-}}\NormalTok{ x\_range[}\DecValTok{1}\NormalTok{] }\SpecialCharTok{{-}}\NormalTok{ x\_buffer}
\NormalTok{xmax }\OtherTok{\textless{}{-}}\NormalTok{ x\_range[}\DecValTok{2}\NormalTok{] }\SpecialCharTok{+}\NormalTok{ x\_buffer}
\NormalTok{ymin }\OtherTok{\textless{}{-}}\NormalTok{ y\_range[}\DecValTok{1}\NormalTok{] }\SpecialCharTok{{-}}\NormalTok{ y\_buffer}
\NormalTok{ymax }\OtherTok{\textless{}{-}}\NormalTok{ y\_range[}\DecValTok{2}\NormalTok{] }\SpecialCharTok{+}\NormalTok{ y\_buffer}

\CommentTok{\# 3. ВИЗУАЛИЗАЦИЯ ТОЧЕК}
\FunctionTok{ggplot}\NormalTok{(DATA) }\SpecialCharTok{+}
  \CommentTok{\# Точки наблюдений с размером и цветом по величине улова}
  \FunctionTok{geom\_point}\NormalTok{(}\FunctionTok{aes}\NormalTok{(}\AttributeTok{x =}\NormalTok{ X, }\AttributeTok{y =}\NormalTok{ Y, }\AttributeTok{size =}\NormalTok{ PROM, }\AttributeTok{color =}\NormalTok{ PROM), }\AttributeTok{alpha =} \FloatTok{0.7}\NormalTok{) }\SpecialCharTok{+}
  
  \CommentTok{\# Цветовая шкала (виридисная палитра)}
  \FunctionTok{scale\_color\_viridis\_c}\NormalTok{(}\AttributeTok{option =} \StringTok{"H"}\NormalTok{, }\AttributeTok{name =} \StringTok{"Улов"}\NormalTok{) }\SpecialCharTok{+}
  
  \CommentTok{\# Шкала размеров точек}
  \FunctionTok{scale\_size\_continuous}\NormalTok{(}\AttributeTok{name =} \StringTok{"Улов"}\NormalTok{) }\SpecialCharTok{+}
  
  \CommentTok{\# Настройка границ с автоматически рассчитанными значениями}
  \FunctionTok{coord\_cartesian}\NormalTok{(}\AttributeTok{xlim =} \FunctionTok{c}\NormalTok{(xmin, xmax), }\AttributeTok{ylim =} \FunctionTok{c}\NormalTok{(ymin, ymax)) }\SpecialCharTok{+}
  
  \CommentTok{\# Подписи осей}
  \FunctionTok{labs}\NormalTok{(}\AttributeTok{x =} \StringTok{"Долгота"}\NormalTok{, }\AttributeTok{y =} \StringTok{"Широта"}\NormalTok{, }
       \AttributeTok{title =} \StringTok{"Распределение уловов краба"}\NormalTok{, }
       \AttributeTok{subtitle =} \StringTok{"2023 год, тип съемки: CRAB"}\NormalTok{) }\SpecialCharTok{+}
  
  \CommentTok{\# Оформление графика}
  \FunctionTok{theme\_bw}\NormalTok{() }\SpecialCharTok{+}
  \FunctionTok{theme}\NormalTok{(}
    \AttributeTok{panel.grid =} \FunctionTok{element\_line}\NormalTok{(}\AttributeTok{color =} \StringTok{"grey90"}\NormalTok{),}
    \AttributeTok{legend.position =} \StringTok{"bottom"}
\NormalTok{  )}
\end{Highlighting}
\end{Shaded}

\section{Карта распределения уловов в съемке с береговой
линией}\label{ux43aux430ux440ux442ux430-ux440ux430ux441ux43fux440ux435ux434ux435ux43bux435ux43dux438ux44f-ux443ux43bux43eux432ux43eux432-ux432-ux441ux44aux435ux43cux43aux435-ux441-ux431ux435ux440ux435ux433ux43eux432ux43eux439-ux43bux438ux43dux438ux435ux439}

\begin{figure}[H]

{\centering \includegraphics[width=0.7\linewidth,height=\textheight,keepaspectratio]{images/KARTOGRAPH2.jpg}

}

\caption{Рис. 2.: Пример карты распределения уловов в съемке с береговой
линией}

\end{figure}%

\begin{Shaded}
\begin{Highlighting}[]
\CommentTok{\# Очистка окружения и установка рабочей директории}
\FunctionTok{rm}\NormalTok{(}\AttributeTok{list =} \FunctionTok{ls}\NormalTok{())  }\CommentTok{\# Удаление всех объектов из глобального окружения}
\FunctionTok{setwd}\NormalTok{(}\StringTok{"C:/COURSES/KARTOGRAPH/"}\NormalTok{)  }\CommentTok{\# Установка рабочей директории}

\CommentTok{\# Загрузка необходимых библиотек}
\FunctionTok{library}\NormalTok{(rnaturalearth)  }\CommentTok{\# Для получения векторных карт мира}
\FunctionTok{library}\NormalTok{(tidyverse)      }\CommentTok{\# Коллекция пакетов для работы с данными}
\FunctionTok{library}\NormalTok{(sf)             }\CommentTok{\# Пространственный анализ}

\DocumentationTok{\#\#\#\#\#\#\# ЗАГРУЗКА ДАННЫХ И ПОДГОТОВКА ПРОСТРАНСТВЕННЫХ ОБЪЕКТОВ \#\#\#\#\#\#\#\#\#\#\#\#\#\#\#\#}

\CommentTok{\# Чтение и фильтрация данных}
\NormalTok{DATA }\OtherTok{\textless{}{-}}\NormalTok{ readxl}\SpecialCharTok{::}\FunctionTok{read\_excel}\NormalTok{(}\StringTok{"KARTOGRAPHIC.xlsx"}\NormalTok{, }\AttributeTok{sheet =} \StringTok{"SURVEY"}\NormalTok{) }\SpecialCharTok{\%\textgreater{}\%} 
  \FunctionTok{filter}\NormalTok{(YEAR }\SpecialCharTok{==} \DecValTok{2023}\NormalTok{, SURV }\SpecialCharTok{==} \StringTok{"CRAB"}\NormalTok{)  }\CommentTok{\# Фильтр данных за 2023 год по типу съемки}

\CommentTok{\# Получение границ России}
\NormalTok{russia }\OtherTok{\textless{}{-}} \FunctionTok{ne\_countries}\NormalTok{(}\AttributeTok{scale =} \DecValTok{10}\NormalTok{, }\AttributeTok{country =} \StringTok{"Russia"}\NormalTok{)  }\CommentTok{\# Загрузка векторных границ (масштаб 1:10м)}

\CommentTok{\# Установка границ отображаемой области (долгота/широта)}
\NormalTok{xmin}\OtherTok{=}\DecValTok{37}  \CommentTok{\# Западная граница}
\NormalTok{xmax}\OtherTok{=}\DecValTok{49}  \CommentTok{\# Восточная граница}
\NormalTok{ymin}\OtherTok{=}\FloatTok{68.5} \CommentTok{\# Южная граница}
\NormalTok{ymax}\OtherTok{=}\FloatTok{70.5} \CommentTok{\# Северная граница}

\CommentTok{\# Построение карты}
\FunctionTok{ggplot}\NormalTok{() }\SpecialCharTok{+}
  \CommentTok{\# Базовая карта России}
  \FunctionTok{geom\_sf}\NormalTok{(}\AttributeTok{data =}\NormalTok{ russia, }\AttributeTok{fill =} \StringTok{"lightblue"}\NormalTok{) }\SpecialCharTok{+} 
  \CommentTok{\# Ограничение области отображения}
  \FunctionTok{coord\_sf}\NormalTok{(}\AttributeTok{xlim =} \FunctionTok{c}\NormalTok{(xmin, xmax), }\AttributeTok{ylim =} \FunctionTok{c}\NormalTok{(ymin, ymax)) }\SpecialCharTok{+}
  \CommentTok{\# Точки наблюдений с размером и цветом по переменной PROM}
  \FunctionTok{geom\_point}\NormalTok{(}\FunctionTok{aes}\NormalTok{(}\AttributeTok{x =}\NormalTok{ X, }\AttributeTok{y =}\NormalTok{ Y, }\AttributeTok{size =}\NormalTok{ PROM, }\AttributeTok{color =}\NormalTok{ PROM),}
             \AttributeTok{data =}\NormalTok{ DATA, }\AttributeTok{alpha =} \FloatTok{0.6}\NormalTok{) }\SpecialCharTok{+}
  \CommentTok{\# Цветовая шкала (viridis, вариант H)}
  \FunctionTok{scale\_color\_viridis\_c}\NormalTok{(}\AttributeTok{option =} \StringTok{"H"}\NormalTok{)}
\end{Highlighting}
\end{Shaded}

\section{Карта распределения уловов, включая
нулевые}\label{ux43aux430ux440ux442ux430-ux440ux430ux441ux43fux440ux435ux434ux435ux43bux435ux43dux438ux44f-ux443ux43bux43eux432ux43eux432-ux432ux43aux43bux44eux447ux430ux44f-ux43dux443ux43bux435ux432ux44bux435}

\begin{figure}[H]

{\centering \includegraphics[width=0.7\linewidth,height=\textheight,keepaspectratio]{images/KARTOGRAPH3.jpg}

}

\caption{Рис. 3.: Карта распределения уловов, включая нулевые}

\end{figure}%

\begin{Shaded}
\begin{Highlighting}[]
\CommentTok{\# Очистка окружения и установка рабочей директории}
\FunctionTok{rm}\NormalTok{(}\AttributeTok{list =} \FunctionTok{ls}\NormalTok{())  }\CommentTok{\# Удаление всех объектов из глобального окружения}
\FunctionTok{setwd}\NormalTok{(}\StringTok{"C:/COURSES/KARTOGRAPH/"}\NormalTok{)  }\CommentTok{\# Установка рабочей директории}

\CommentTok{\# Загрузка необходимых библиотек}
\FunctionTok{library}\NormalTok{(rnaturalearth)  }\CommentTok{\# Для получения векторных карт мира}
\FunctionTok{library}\NormalTok{(tidyverse)      }\CommentTok{\# Коллекция пакетов для работы с данными}
\FunctionTok{library}\NormalTok{(sf)             }\CommentTok{\# Пространственный анализ}

\DocumentationTok{\#\#\#\#\#\#\# ЗАГРУЗКА ДАННЫХ И ПОДГОТОВКА ПРОСТРАНСТВЕННЫХ ОБЪЕКТОВ \#\#\#\#\#\#\#\#\#\#\#\#\#\#\#\#}

\CommentTok{\# Чтение и фильтрация данных}
\NormalTok{DATA }\OtherTok{\textless{}{-}}\NormalTok{ readxl}\SpecialCharTok{::}\FunctionTok{read\_excel}\NormalTok{(}\StringTok{"KARTOGRAPHIC.xlsx"}\NormalTok{, }\AttributeTok{sheet =} \StringTok{"SURVEY"}\NormalTok{) }\SpecialCharTok{\%\textgreater{}\%} 
  \FunctionTok{filter}\NormalTok{(YEAR }\SpecialCharTok{==} \DecValTok{2023}\NormalTok{, SURV }\SpecialCharTok{==} \StringTok{"CRAB"}\NormalTok{)  }\CommentTok{\# Фильтр данных за 2023 год по типу съемки}

\CommentTok{\# Получение границ России}
\NormalTok{russia }\OtherTok{\textless{}{-}} \FunctionTok{ne\_countries}\NormalTok{(}\AttributeTok{scale =} \DecValTok{10}\NormalTok{, }\AttributeTok{country =} \StringTok{"Russia"}\NormalTok{)  }\CommentTok{\# Загрузка векторных границ (масштаб 1:10м)}

\CommentTok{\# Установка границ отображаемой области (долгота/широта)}
\NormalTok{xmin}\OtherTok{=}\DecValTok{37}  \CommentTok{\# Западная граница}
\NormalTok{xmax}\OtherTok{=}\DecValTok{49}  \CommentTok{\# Восточная граница}
\NormalTok{ymin}\OtherTok{=}\FloatTok{68.5} \CommentTok{\# Южная граница}
\NormalTok{ymax}\OtherTok{=}\FloatTok{70.5} \CommentTok{\# Северная граница}

\CommentTok{\# Построение карты}
\FunctionTok{ggplot}\NormalTok{() }\SpecialCharTok{+}
  \CommentTok{\# Базовая карта России}
  \FunctionTok{geom\_sf}\NormalTok{(}\AttributeTok{data =}\NormalTok{ russia, }\AttributeTok{fill =} \StringTok{"lightblue"}\NormalTok{) }\SpecialCharTok{+} 
  \CommentTok{\# Ограничение области отображения}
  \FunctionTok{coord\_sf}\NormalTok{(}\AttributeTok{xlim =} \FunctionTok{c}\NormalTok{(xmin, xmax), }\AttributeTok{ylim =} \FunctionTok{c}\NormalTok{(ymin, ymax)) }\SpecialCharTok{+}
  \CommentTok{\# Точки наблюдений с размером и цветом по переменной PROM (ненулевые уловы)}
  \FunctionTok{geom\_point}\NormalTok{(}\FunctionTok{aes}\NormalTok{(}\AttributeTok{x =}\NormalTok{ X, }\AttributeTok{y =}\NormalTok{ Y, }\AttributeTok{size =}\NormalTok{ PROM, }\AttributeTok{color =}\NormalTok{ PROM),}
             \AttributeTok{data =} \FunctionTok{filter}\NormalTok{(DATA, PROM }\SpecialCharTok{\textgreater{}} \DecValTok{0}\NormalTok{), }\AttributeTok{alpha =} \FloatTok{0.6}\NormalTok{) }\SpecialCharTok{+}
  \CommentTok{\# Точки для нулевых уловов (крестики)}
  \FunctionTok{geom\_point}\NormalTok{(}\FunctionTok{aes}\NormalTok{(}\AttributeTok{x =}\NormalTok{ X, }\AttributeTok{y =}\NormalTok{ Y),}
             \AttributeTok{data =} \FunctionTok{filter}\NormalTok{(DATA, PROM }\SpecialCharTok{==} \DecValTok{0}\NormalTok{),}
             \AttributeTok{shape =} \DecValTok{4}\NormalTok{, }\AttributeTok{size =} \DecValTok{1}\NormalTok{, }\AttributeTok{stroke =} \DecValTok{1}\NormalTok{, }\AttributeTok{color =} \StringTok{"black"}\NormalTok{) }\SpecialCharTok{+}
  \CommentTok{\# Цветовая шкала (viridis, вариант H)}
  \FunctionTok{scale\_color\_viridis\_c}\NormalTok{(}\AttributeTok{option =} \StringTok{"H"}\NormalTok{)}
\end{Highlighting}
\end{Shaded}

\section{Карта распределения уловов, распределенных по
квартилям}\label{ux43aux430ux440ux442ux430-ux440ux430ux441ux43fux440ux435ux434ux435ux43bux435ux43dux438ux44f-ux443ux43bux43eux432ux43eux432-ux440ux430ux441ux43fux440ux435ux434ux435ux43bux435ux43dux43dux44bux445-ux43fux43e-ux43aux432ux430ux440ux442ux438ux43bux44fux43c}

\begin{figure}[H]

{\centering \includegraphics[width=0.7\linewidth,height=\textheight,keepaspectratio]{images/KARTOGRAPH4.jpg}

}

\caption{Рис. 4.: Карта распределения уловов, распределенных по
квартилям}

\end{figure}%

\begin{Shaded}
\begin{Highlighting}[]
\CommentTok{\# Очистка окружения и установка рабочей директории}
\FunctionTok{rm}\NormalTok{(}\AttributeTok{list =} \FunctionTok{ls}\NormalTok{())}
\FunctionTok{setwd}\NormalTok{(}\StringTok{"C:/COURSES/KARTOGRAPH/"}\NormalTok{)}

\CommentTok{\# Загрузка необходимых библиотек}
\FunctionTok{library}\NormalTok{(rnaturalearth)}
\FunctionTok{library}\NormalTok{(tidyverse)}
\FunctionTok{library}\NormalTok{(sf)}

\DocumentationTok{\#\#\#\#\#\#\# ЗАГРУЗКА ДАННЫХ И ПОДГОТОВКА ПРОСТРАНСТВЕННЫХ ОБЪЕКТОВ \#\#\#\#\#\#\#\#\#\#\#\#\#\#\#\#}

\CommentTok{\# Чтение и фильтрация данных}
\NormalTok{DATA }\OtherTok{\textless{}{-}}\NormalTok{ readxl}\SpecialCharTok{::}\FunctionTok{read\_excel}\NormalTok{(}\StringTok{"KARTOGRAPHIC.xlsx"}\NormalTok{, }\AttributeTok{sheet =} \StringTok{"SURVEY"}\NormalTok{) }\SpecialCharTok{\%\textgreater{}\%} 
  \FunctionTok{filter}\NormalTok{(YEAR }\SpecialCharTok{==} \DecValTok{2023}\NormalTok{, SURV }\SpecialCharTok{==} \StringTok{"CRAB"}\NormalTok{)}

\CommentTok{\# Получение границ России}
\NormalTok{russia }\OtherTok{\textless{}{-}} \FunctionTok{ne\_countries}\NormalTok{(}\AttributeTok{scale =} \DecValTok{10}\NormalTok{, }\AttributeTok{country =} \StringTok{"Russia"}\NormalTok{) }\SpecialCharTok{\%\textgreater{}\%} 
  \FunctionTok{st\_as\_sf}\NormalTok{()}

\CommentTok{\# Установка границ отображаемой области}
\NormalTok{xmin}\OtherTok{=}\DecValTok{37}\NormalTok{; xmax}\OtherTok{=}\DecValTok{49}\NormalTok{; ymin}\OtherTok{=}\FloatTok{68.5}\NormalTok{; ymax}\OtherTok{=}\FloatTok{70.5}

\DocumentationTok{\#\#\#\#\#\#\# ПОДГОТОВКА ДАННЫХ ДЛЯ ВИЗУАЛИЗАЦИИ \#\#\#\#\#\#\#\#\#\#\#\#\#\#\#\#}
\CommentTok{\# Вычисляем квартили отдельно}
\NormalTok{quantiles }\OtherTok{\textless{}{-}} \FunctionTok{quantile}\NormalTok{(DATA}\SpecialCharTok{$}\NormalTok{PROM[DATA}\SpecialCharTok{$}\NormalTok{PROM }\SpecialCharTok{\textgreater{}} \DecValTok{0}\NormalTok{], }\AttributeTok{probs =} \FunctionTok{seq}\NormalTok{(}\DecValTok{0}\NormalTok{, }\DecValTok{1}\NormalTok{, }\FloatTok{0.25}\NormalTok{))}

\CommentTok{\# Создаем 4 категории с реальными диапазонами значений}
\NormalTok{nonzero\_data }\OtherTok{\textless{}{-}}\NormalTok{ DATA }\SpecialCharTok{\%\textgreater{}\%} 
  \FunctionTok{filter}\NormalTok{(PROM }\SpecialCharTok{\textgreater{}} \DecValTok{0}\NormalTok{) }\SpecialCharTok{\%\textgreater{}\%}
  \FunctionTok{mutate}\NormalTok{(}
    \AttributeTok{PROM\_cat =} \FunctionTok{cut}\NormalTok{(}
\NormalTok{      PROM,}
      \AttributeTok{breaks =}\NormalTok{ quantiles,}
      \AttributeTok{include.lowest =} \ConstantTok{TRUE}\NormalTok{,}
      \AttributeTok{labels =} \FunctionTok{c}\NormalTok{(}
        \FunctionTok{sprintf}\NormalTok{(}\StringTok{"\%.1f {-} \%.1f"}\NormalTok{, quantiles[}\DecValTok{1}\NormalTok{], quantiles[}\DecValTok{2}\NormalTok{]),}
        \FunctionTok{sprintf}\NormalTok{(}\StringTok{"\%.1f {-} \%.1f"}\NormalTok{, quantiles[}\DecValTok{2}\NormalTok{], quantiles[}\DecValTok{3}\NormalTok{]),}
        \FunctionTok{sprintf}\NormalTok{(}\StringTok{"\%.1f {-} \%.1f"}\NormalTok{, quantiles[}\DecValTok{3}\NormalTok{], quantiles[}\DecValTok{4}\NormalTok{]),}
        \FunctionTok{sprintf}\NormalTok{(}\StringTok{"\%.1f {-} \%.1f"}\NormalTok{, quantiles[}\DecValTok{4}\NormalTok{], quantiles[}\DecValTok{5}\NormalTok{])}
\NormalTok{      )}
\NormalTok{    )}
\NormalTok{  )}

\CommentTok{\# Построение карты}
\FunctionTok{ggplot}\NormalTok{() }\SpecialCharTok{+}
  \CommentTok{\# Базовая карта России}
  \FunctionTok{geom\_sf}\NormalTok{(}\AttributeTok{data =}\NormalTok{ russia, }\AttributeTok{fill =} \StringTok{"lightblue"}\NormalTok{, }\AttributeTok{color =} \StringTok{"gray40"}\NormalTok{) }\SpecialCharTok{+} 
  \CommentTok{\# Ограничение области отображения}
  \FunctionTok{coord\_sf}\NormalTok{(}\AttributeTok{xlim =} \FunctionTok{c}\NormalTok{(xmin, xmax), }\AttributeTok{ylim =} \FunctionTok{c}\NormalTok{(ymin, ymax)) }\SpecialCharTok{+}
  \CommentTok{\# Точки наблюдений с категориальным размером}
  \FunctionTok{geom\_point}\NormalTok{(}
    \AttributeTok{data =}\NormalTok{ nonzero\_data,}
    \FunctionTok{aes}\NormalTok{(}\AttributeTok{x =}\NormalTok{ X, }\AttributeTok{y =}\NormalTok{ Y, }\AttributeTok{size =}\NormalTok{ PROM\_cat, }\AttributeTok{color =}\NormalTok{ PROM),}
    \AttributeTok{alpha =} \FloatTok{0.7}
\NormalTok{  ) }\SpecialCharTok{+}
  \CommentTok{\# Точки для нулевых уловов (крестики)}
  \FunctionTok{geom\_point}\NormalTok{(}
    \AttributeTok{data =} \FunctionTok{filter}\NormalTok{(DATA, PROM }\SpecialCharTok{==} \DecValTok{0}\NormalTok{),}
    \FunctionTok{aes}\NormalTok{(}\AttributeTok{x =}\NormalTok{ X, }\AttributeTok{y =}\NormalTok{ Y),}
    \AttributeTok{shape =} \DecValTok{4}\NormalTok{, }\AttributeTok{size =} \FloatTok{1.2}\NormalTok{, }\AttributeTok{stroke =} \DecValTok{1}\NormalTok{, }\AttributeTok{color =} \StringTok{"black"}
\NormalTok{  ) }\SpecialCharTok{+}
  \CommentTok{\# Цветовая шкала (непрерывная)}
  \FunctionTok{scale\_color\_viridis\_c}\NormalTok{(}\AttributeTok{option =} \StringTok{"H"}\NormalTok{, }\AttributeTok{name =} \ConstantTok{NULL}\NormalTok{) }\SpecialCharTok{+}
  \CommentTok{\# Ручная настройка размеров для категорий}
  \FunctionTok{scale\_size\_manual}\NormalTok{(}
    \AttributeTok{name =} \StringTok{"Улов (экз./ч)"}\NormalTok{,}
    \AttributeTok{values =} \FunctionTok{c}\NormalTok{(}\DecValTok{2}\NormalTok{, }\DecValTok{4}\NormalTok{, }\DecValTok{6}\NormalTok{, }\DecValTok{8}\NormalTok{),  }\CommentTok{\# Размеры точек для категорий}
    \AttributeTok{drop =} \ConstantTok{FALSE}
\NormalTok{  ) }\SpecialCharTok{+}
  \CommentTok{\# Настройки темы}
  \FunctionTok{theme\_bw}\NormalTok{() }\SpecialCharTok{+}
  \FunctionTok{labs}\NormalTok{(}
    \AttributeTok{title =} \StringTok{"Распределение уловов краба (2023)"}\NormalTok{,}
    \AttributeTok{subtitle =} \StringTok{"Черные крестики {-} нулевые уловы"}\NormalTok{,}
    \AttributeTok{x =} \StringTok{"Долгота"}\NormalTok{, }
    \AttributeTok{y =} \StringTok{"Широта"}
\NormalTok{  ) }\SpecialCharTok{+}
  \FunctionTok{theme}\NormalTok{(}
    \AttributeTok{panel.grid =} \FunctionTok{element\_line}\NormalTok{(}\AttributeTok{color =} \StringTok{"gray90"}\NormalTok{),}
    \AttributeTok{legend.position =} \StringTok{"bottom"}
\NormalTok{  )}
\end{Highlighting}
\end{Shaded}

\section{Карта распределения уловов по
фасеткам}\label{ux43aux430ux440ux442ux430-ux440ux430ux441ux43fux440ux435ux434ux435ux43bux435ux43dux438ux44f-ux443ux43bux43eux432ux43eux432-ux43fux43e-ux444ux430ux441ux435ux442ux43aux430ux43c}

\begin{figure}[H]

{\centering \includegraphics[width=0.8\linewidth,height=\textheight,keepaspectratio]{images/KARTOGRAPH5.jpg}

}

\caption{Рис. 5.: Карта распределения уловов по фасеткам}

\end{figure}%

\begin{Shaded}
\begin{Highlighting}[]
\CommentTok{\# Очистка окружения и установка рабочей директории}
\FunctionTok{rm}\NormalTok{(}\AttributeTok{list =} \FunctionTok{ls}\NormalTok{())}
\FunctionTok{setwd}\NormalTok{(}\StringTok{"C:/COURSES/KARTOGRAPH/"}\NormalTok{)}

\CommentTok{\# Установка и подключение библиотек (если не установлено — раскомментируй)}
\CommentTok{\# install.packages(c("rnaturalearth", "tidyverse", "sf", "readxl", "viridis"))}
\FunctionTok{library}\NormalTok{(rnaturalearth)}
\FunctionTok{library}\NormalTok{(tidyverse)}
\FunctionTok{library}\NormalTok{(sf)}
\FunctionTok{library}\NormalTok{(readxl)}
\FunctionTok{library}\NormalTok{(viridis)}

\DocumentationTok{\#\#\#\#\#\#\# ЗАГРУЗКА ДАННЫХ И ПОДГОТОВКА ПРОСТРАНСТВЕННЫХ ОБЪЕКТОВ \#\#\#\#\#\#\#\#\#\#\#\#\#\#\#\#}

\CommentTok{\# Чтение и фильтрация данных (убираем фильтр по году, чтобы работать со всеми годами)}
\NormalTok{DATA }\OtherTok{\textless{}{-}}\NormalTok{ readxl}\SpecialCharTok{::}\FunctionTok{read\_excel}\NormalTok{(}\StringTok{"KARTOGRAPHIC.xlsx"}\NormalTok{, }\AttributeTok{sheet =} \StringTok{"SURVEY"}\NormalTok{) }\SpecialCharTok{\%\textgreater{}\%} 
  \FunctionTok{filter}\NormalTok{(SURV }\SpecialCharTok{==} \StringTok{"CRAB"}\NormalTok{)}

\CommentTok{\# Получение границ России}
\NormalTok{russia }\OtherTok{\textless{}{-}} \FunctionTok{ne\_countries}\NormalTok{(}\AttributeTok{scale =} \DecValTok{10}\NormalTok{, }\AttributeTok{country =} \StringTok{"Russia"}\NormalTok{) }\SpecialCharTok{\%\textgreater{}\%} 
  \FunctionTok{st\_as\_sf}\NormalTok{()}

\CommentTok{\# Установка границ отображаемой области}
\NormalTok{xmin }\OtherTok{\textless{}{-}} \DecValTok{37}\NormalTok{; xmax }\OtherTok{\textless{}{-}} \DecValTok{49}
\NormalTok{ymin }\OtherTok{\textless{}{-}} \FloatTok{68.5}\NormalTok{; ymax }\OtherTok{\textless{}{-}} \FloatTok{70.5}

\CommentTok{\# Вычисляем общие квартили для всех лет (чтобы категории были сопоставимыми)}
\NormalTok{quantiles }\OtherTok{\textless{}{-}} \FunctionTok{quantile}\NormalTok{(DATA}\SpecialCharTok{$}\NormalTok{PROM[DATA}\SpecialCharTok{$}\NormalTok{PROM }\SpecialCharTok{\textgreater{}} \DecValTok{0}\NormalTok{], }\AttributeTok{probs =} \FunctionTok{seq}\NormalTok{(}\DecValTok{0}\NormalTok{, }\DecValTok{1}\NormalTok{, }\FloatTok{0.25}\NormalTok{))}

\CommentTok{\# Создаем данные с ненулевыми уловами и категориями}
\NormalTok{nonzero\_data }\OtherTok{\textless{}{-}}\NormalTok{ DATA }\SpecialCharTok{\%\textgreater{}\%}
  \FunctionTok{filter}\NormalTok{(PROM }\SpecialCharTok{\textgreater{}} \DecValTok{0}\NormalTok{) }\SpecialCharTok{\%\textgreater{}\%}
  \FunctionTok{mutate}\NormalTok{(}
\AttributeTok{PROM\_cat =} \FunctionTok{cut}\NormalTok{(}
\NormalTok{  PROM,}
  \AttributeTok{breaks =} \FunctionTok{c}\NormalTok{(}\SpecialCharTok{{-}}\ConstantTok{Inf}\NormalTok{, quantiles[}\DecValTok{2}\SpecialCharTok{:}\DecValTok{4}\NormalTok{], }\ConstantTok{Inf}\NormalTok{),}
  \AttributeTok{include.lowest =} \ConstantTok{TRUE}\NormalTok{,}
  \AttributeTok{labels =} \FunctionTok{c}\NormalTok{(}
    \FunctionTok{sprintf}\NormalTok{(}\StringTok{"\%d {-} \%d"}\NormalTok{, }\FunctionTok{floor}\NormalTok{(quantiles[}\DecValTok{1}\NormalTok{]), }\FunctionTok{floor}\NormalTok{(quantiles[}\DecValTok{2}\NormalTok{])),}
    \FunctionTok{sprintf}\NormalTok{(}\StringTok{"\%d {-} \%d"}\NormalTok{, }\FunctionTok{floor}\NormalTok{(quantiles[}\DecValTok{2}\NormalTok{]), }\FunctionTok{floor}\NormalTok{(quantiles[}\DecValTok{3}\NormalTok{])),}
    \FunctionTok{sprintf}\NormalTok{(}\StringTok{"\%d {-} \%d"}\NormalTok{, }\FunctionTok{floor}\NormalTok{(quantiles[}\DecValTok{3}\NormalTok{]), }\FunctionTok{floor}\NormalTok{(quantiles[}\DecValTok{4}\NormalTok{])),}
    \FunctionTok{sprintf}\NormalTok{(}\StringTok{"\%d {-} \%d"}\NormalTok{, }\FunctionTok{floor}\NormalTok{(quantiles[}\DecValTok{4}\NormalTok{]), }\FunctionTok{floor}\NormalTok{(}\FunctionTok{max}\NormalTok{(DATA}\SpecialCharTok{$}\NormalTok{PROM)))}
\NormalTok{  )}
\NormalTok{)}
\NormalTok{  )}

\CommentTok{\# Отдельно выделяем точки с нулевым уловом}
\NormalTok{zero\_data }\OtherTok{\textless{}{-}}\NormalTok{ DATA }\SpecialCharTok{\%\textgreater{}\%} \FunctionTok{filter}\NormalTok{(PROM }\SpecialCharTok{==} \DecValTok{0}\NormalTok{)}

\DocumentationTok{\#\#\#\#\#\#\# ВИЗУАЛИЗАЦИЯ \#\#\#\#\#\#\#\#\#\#\#\#\#\#\#\#}

\CommentTok{\# Фасеточная карта по годам}
\FunctionTok{ggplot}\NormalTok{() }\SpecialCharTok{+}
  \CommentTok{\# Граница России}
  \FunctionTok{geom\_sf}\NormalTok{(}\AttributeTok{data =}\NormalTok{ russia, }\AttributeTok{fill =} \StringTok{"lightblue"}\NormalTok{, }\AttributeTok{color =} \StringTok{"gray40"}\NormalTok{) }\SpecialCharTok{+}
  
  \CommentTok{\# Ограничение области отображения}
  \FunctionTok{coord\_sf}\NormalTok{(}\AttributeTok{xlim =} \FunctionTok{c}\NormalTok{(xmin, xmax), }\AttributeTok{ylim =} \FunctionTok{c}\NormalTok{(ymin, ymax)) }\SpecialCharTok{+}
  
  \CommentTok{\# Точки с уловом}
  \FunctionTok{geom\_point}\NormalTok{(}
    \AttributeTok{data =}\NormalTok{ nonzero\_data,}
    \FunctionTok{aes}\NormalTok{(}\AttributeTok{x =}\NormalTok{ X, }\AttributeTok{y =}\NormalTok{ Y, }\AttributeTok{size =}\NormalTok{ PROM\_cat, }\AttributeTok{color =}\NormalTok{ PROM),}
    \AttributeTok{alpha =} \FloatTok{0.7}
\NormalTok{  ) }\SpecialCharTok{+}
  
  \CommentTok{\# Нулевые уловы — крестики}
  \FunctionTok{geom\_point}\NormalTok{(}
    \AttributeTok{data =}\NormalTok{ zero\_data,}
    \FunctionTok{aes}\NormalTok{(}\AttributeTok{x =}\NormalTok{ X, }\AttributeTok{y =}\NormalTok{ Y),}
    \AttributeTok{shape =} \DecValTok{4}\NormalTok{, }\AttributeTok{size =} \FloatTok{1.2}\NormalTok{, }\AttributeTok{stroke =} \DecValTok{1}\NormalTok{, }\AttributeTok{color =} \StringTok{"black"}
\NormalTok{  ) }\SpecialCharTok{+}
  
  \CommentTok{\# Цветовая шкала}
  \FunctionTok{scale\_color\_viridis\_c}\NormalTok{(}\AttributeTok{option =} \StringTok{"H"}\NormalTok{, }\AttributeTok{name =} \ConstantTok{NULL}\NormalTok{) }\SpecialCharTok{+}
  
  \CommentTok{\# Настройка размеров точек по категориям}
  \FunctionTok{scale\_size\_manual}\NormalTok{(}
    \AttributeTok{name =} \StringTok{"Улов (экз./ч)"}\NormalTok{,}
    \AttributeTok{values =} \FunctionTok{c}\NormalTok{(}\DecValTok{1}\NormalTok{, }\DecValTok{2}\NormalTok{,}\DecValTok{4}\NormalTok{, }\DecValTok{6}\NormalTok{),}
    \AttributeTok{drop =} \ConstantTok{FALSE}
\NormalTok{  ) }\SpecialCharTok{+}
  
  \CommentTok{\# Фасет по годам}
  \FunctionTok{facet\_wrap}\NormalTok{(}\SpecialCharTok{\textasciitilde{}}\NormalTok{ YEAR, }\AttributeTok{ncol =} \DecValTok{2}\NormalTok{, }\AttributeTok{labeller =}\NormalTok{ label\_value) }\SpecialCharTok{+}
  
  \CommentTok{\# Тема и заголовок}
  \FunctionTok{theme\_bw}\NormalTok{() }\SpecialCharTok{+}
  \FunctionTok{labs}\NormalTok{(}
    \AttributeTok{title =} \StringTok{"Распределение уловов краба по годам"}\NormalTok{,}
    \AttributeTok{subtitle =} \ConstantTok{NULL}\NormalTok{,}
    \AttributeTok{x =} \StringTok{"Долгота"}\NormalTok{, }
    \AttributeTok{y =} \StringTok{"Широта"}
\NormalTok{  ) }\SpecialCharTok{+}
  \FunctionTok{theme}\NormalTok{(}
    \AttributeTok{panel.grid =} \FunctionTok{element\_line}\NormalTok{(}\AttributeTok{color =} \StringTok{"gray90"}\NormalTok{),}
    \AttributeTok{legend.position =} \StringTok{"bottom"}
\NormalTok{  )}
\end{Highlighting}
\end{Shaded}

\section{Карта распределения уловов с автокорреляцией
LISA}\label{ux43aux430ux440ux442ux430-ux440ux430ux441ux43fux440ux435ux434ux435ux43bux435ux43dux438ux44f-ux443ux43bux43eux432ux43eux432-ux441-ux430ux432ux442ux43eux43aux43eux440ux440ux435ux43bux44fux446ux438ux435ux439-lisa}

Алгоритм LISA (Local Indicators of Spatial Association) -- это набор
статистических мер, которые позволяют выявить пространственную
кластеризацию или выбросы на уровне отдельных объектов или
местоположений в географических данных. В отличие от глобальных
показателей пространственной автокорреляции, которые дают общую
характеристику пространственной структуры данных, LISA позволяет
определить, какие конкретные объекты или области вносят наибольший вклад
в пространственную структуру.

\begin{figure}[H]

{\centering \includegraphics[width=0.8\linewidth,height=\textheight,keepaspectratio]{images/KARTOGRAPH6.jpg}

}

\caption{Рис. 6.: Карта распределения уловов с автокорреляцией LISA}

\end{figure}%

\begin{Shaded}
\begin{Highlighting}[]
\CommentTok{\# Очистка окружения и установка рабочей директории}
\FunctionTok{rm}\NormalTok{(}\AttributeTok{list =} \FunctionTok{ls}\NormalTok{())}
\FunctionTok{setwd}\NormalTok{(}\StringTok{"C:/COURSES/KARTOGRAPH/"}\NormalTok{)}

\CommentTok{\# Загрузка библиотек}
\FunctionTok{library}\NormalTok{(rnaturalearth)}
\FunctionTok{library}\NormalTok{(tidyverse)}
\FunctionTok{library}\NormalTok{(sf)}
\FunctionTok{library}\NormalTok{(spdep)}
\FunctionTok{library}\NormalTok{(ggspatial)}
\FunctionTok{library}\NormalTok{(readxl)}

\CommentTok{\# 1. ЗАГРУЗКА И ПОДГОТОВКА ДАННЫХ}
\NormalTok{DATA }\OtherTok{\textless{}{-}} \FunctionTok{read\_excel}\NormalTok{(}\StringTok{"KARTOGRAPHIC.xlsx"}\NormalTok{, }\AttributeTok{sheet =} \StringTok{"SURVEY"}\NormalTok{) }\SpecialCharTok{\%\textgreater{}\%} 
  \FunctionTok{filter}\NormalTok{(YEAR }\SpecialCharTok{==} \DecValTok{2023}\NormalTok{, SURV }\SpecialCharTok{==} \StringTok{"CRAB"}\NormalTok{)}

\CommentTok{\# Проверка названий колонок}
\FunctionTok{print}\NormalTok{(}\FunctionTok{names}\NormalTok{(DATA))  }\CommentTok{\# Убедитесь, что координаты названы правильно}

\CommentTok{\# Преобразование в пространственные данные (замените X/Y на ваши названия)}
\NormalTok{points\_sf }\OtherTok{\textless{}{-}} \FunctionTok{st\_as\_sf}\NormalTok{(DATA, }\AttributeTok{coords =} \FunctionTok{c}\NormalTok{(}\StringTok{"X"}\NormalTok{, }\StringTok{"Y"}\NormalTok{), }\AttributeTok{crs =} \DecValTok{4326}\NormalTok{)}

\CommentTok{\# 2. ПОЛУЧЕНИЕ КАРТЫ РОССИИ}
\CommentTok{\# Задаем границы области}
\NormalTok{xmin }\OtherTok{\textless{}{-}} \DecValTok{37}
\NormalTok{xmax }\OtherTok{\textless{}{-}} \DecValTok{49}
\NormalTok{ymin }\OtherTok{\textless{}{-}} \FloatTok{68.5}
\NormalTok{ymax }\OtherTok{\textless{}{-}} \FloatTok{70.5}

\CommentTok{\# Создаём ограничивающий прямоугольник}
\NormalTok{bbox }\OtherTok{\textless{}{-}} \FunctionTok{st\_bbox}\NormalTok{(}\FunctionTok{c}\NormalTok{(}\AttributeTok{xmin =}\NormalTok{ xmin, }\AttributeTok{xmax =}\NormalTok{ xmax, }\AttributeTok{ymin =}\NormalTok{ ymin, }\AttributeTok{ymax =}\NormalTok{ ymax), }\AttributeTok{crs =} \DecValTok{4326}\NormalTok{)}
\NormalTok{bbox\_poly }\OtherTok{\textless{}{-}} \FunctionTok{st\_as\_sfc}\NormalTok{(bbox)}

\CommentTok{\# Карта России}
\NormalTok{russia }\OtherTok{\textless{}{-}} \FunctionTok{ne\_countries}\NormalTok{(}\AttributeTok{country =} \StringTok{"Russia"}\NormalTok{, }\AttributeTok{scale =} \DecValTok{10}\NormalTok{) }\SpecialCharTok{\%\textgreater{}\%} 
  \FunctionTok{st\_as\_sf}\NormalTok{() }\SpecialCharTok{\%\textgreater{}\%} 
  \FunctionTok{st\_crop}\NormalTok{(bbox)  }\CommentTok{\# Обрезка без st\_intersection}

\CommentTok{\# 3. ПОДГОТОВКА ТОЧЕК}
\CommentTok{\# Удаление дубликатов по координатам}
\NormalTok{coords }\OtherTok{\textless{}{-}} \FunctionTok{st\_coordinates}\NormalTok{(points\_sf)}
\NormalTok{points\_sf }\OtherTok{\textless{}{-}}\NormalTok{ points\_sf[}\SpecialCharTok{!}\FunctionTok{duplicated}\NormalTok{(coords), , drop }\OtherTok{=} \ConstantTok{FALSE}\NormalTok{]}

\CommentTok{\# Перевод в UTM}
\NormalTok{points\_utm }\OtherTok{\textless{}{-}} \FunctionTok{st\_transform}\NormalTok{(points\_sf, }\AttributeTok{crs =} \DecValTok{32638}\NormalTok{)}

\CommentTok{\# 4. АНАЛИЗ LISA}
\CommentTok{\# Матрица весов}
\NormalTok{knn }\OtherTok{\textless{}{-}} \FunctionTok{knearneigh}\NormalTok{(points\_utm, }\AttributeTok{k =} \DecValTok{4}\NormalTok{)}
\NormalTok{nb }\OtherTok{\textless{}{-}} \FunctionTok{knn2nb}\NormalTok{(knn)}
\NormalTok{listw }\OtherTok{\textless{}{-}} \FunctionTok{nb2listw}\NormalTok{(nb, }\AttributeTok{style =} \StringTok{"W"}\NormalTok{)}

\CommentTok{\# Локальный Моран}
\NormalTok{local\_moran }\OtherTok{\textless{}{-}} \FunctionTok{localmoran}\NormalTok{(points\_utm}\SpecialCharTok{$}\NormalTok{PROM, listw)}

\CommentTok{\# Добавляем кластеры}
\NormalTok{points\_utm }\OtherTok{\textless{}{-}}\NormalTok{ points\_utm }\SpecialCharTok{\%\textgreater{}\%}
  \FunctionTok{mutate}\NormalTok{(}
    \AttributeTok{Local\_I =}\NormalTok{ local\_moran[, }\StringTok{"Ii"}\NormalTok{],}
    \AttributeTok{P\_value =}\NormalTok{ local\_moran[, }\StringTok{"Pr(z != E(Ii))"}\NormalTok{],}
    \AttributeTok{Mean\_PROM =} \FunctionTok{mean}\NormalTok{(PROM, }\AttributeTok{na.rm =} \ConstantTok{TRUE}\NormalTok{),  }\CommentTok{\# Добавляем среднее значение}
    \AttributeTok{Cluster =} \FunctionTok{case\_when}\NormalTok{(}
\NormalTok{      Local\_I }\SpecialCharTok{\textgreater{}} \DecValTok{0} \SpecialCharTok{\&}\NormalTok{ PROM }\SpecialCharTok{\textgreater{}}\NormalTok{ Mean\_PROM }\SpecialCharTok{\textasciitilde{}} \StringTok{"High{-}High"}\NormalTok{,}
\NormalTok{      Local\_I }\SpecialCharTok{\textgreater{}} \DecValTok{0} \SpecialCharTok{\&}\NormalTok{ PROM }\SpecialCharTok{\textless{}=}\NormalTok{ Mean\_PROM }\SpecialCharTok{\textasciitilde{}} \StringTok{"Low{-}Low"}\NormalTok{,  }\CommentTok{\# Включаем PROM == 0}
\NormalTok{      Local\_I }\SpecialCharTok{\textless{}} \DecValTok{0} \SpecialCharTok{\&}\NormalTok{ PROM }\SpecialCharTok{\textgreater{}}\NormalTok{ Mean\_PROM }\SpecialCharTok{\textasciitilde{}} \StringTok{"High{-}Low"}\NormalTok{,}
\NormalTok{      Local\_I }\SpecialCharTok{\textless{}} \DecValTok{0} \SpecialCharTok{\&}\NormalTok{ PROM }\SpecialCharTok{\textless{}=}\NormalTok{ Mean\_PROM }\SpecialCharTok{\textasciitilde{}} \StringTok{"Low{-}High"}\NormalTok{,  }\CommentTok{\# PROM == 0 попадает сюда}
      \ConstantTok{TRUE} \SpecialCharTok{\textasciitilde{}} \StringTok{"Not significant"}
\NormalTok{    )}
\NormalTok{  )}

\CommentTok{\# Обратно в WGS84}
\NormalTok{points\_result }\OtherTok{\textless{}{-}} \FunctionTok{st\_transform}\NormalTok{(points\_utm, }\AttributeTok{crs =} \DecValTok{4326}\NormalTok{)}

\CommentTok{\# 5. ВИЗУАЛИЗАЦИЯ}
\NormalTok{cluster\_colors }\OtherTok{\textless{}{-}} \FunctionTok{c}\NormalTok{(}
  \StringTok{"High{-}High"} \OtherTok{=} \StringTok{"red"}\NormalTok{,}
  \StringTok{"Low{-}Low"} \OtherTok{=} \StringTok{"blue"}\NormalTok{,}
  \StringTok{"High{-}Low"} \OtherTok{=} \StringTok{"pink"}\NormalTok{,}
  \StringTok{"Low{-}High"} \OtherTok{=} \StringTok{"lightblue"}\NormalTok{,}
  \StringTok{"Not significant"} \OtherTok{=} \StringTok{"gray"}
\NormalTok{)}

\FunctionTok{ggplot}\NormalTok{() }\SpecialCharTok{+}
  \CommentTok{\# Карта России}
  \FunctionTok{geom\_sf}\NormalTok{(}\AttributeTok{data =}\NormalTok{ russia, }\AttributeTok{fill =} \StringTok{"lightblue"}\NormalTok{, }\AttributeTok{color =} \StringTok{"black"}\NormalTok{) }\SpecialCharTok{+}
  
  \CommentTok{\# Все точки (включая PROM == 0) — в одном слое}
  \FunctionTok{geom\_sf}\NormalTok{(}
    \AttributeTok{data =}\NormalTok{ points\_result,}
    \FunctionTok{aes}\NormalTok{(}\AttributeTok{color =}\NormalTok{ Cluster, }\AttributeTok{size =}\NormalTok{ PROM),}
    \AttributeTok{alpha =} \FloatTok{0.8}
\NormalTok{  ) }\SpecialCharTok{+}
  
  \CommentTok{\# Настройки координат и масштаба}
  \FunctionTok{coord\_sf}\NormalTok{(}\AttributeTok{xlim =} \FunctionTok{c}\NormalTok{(xmin, xmax), }\AttributeTok{ylim =} \FunctionTok{c}\NormalTok{(ymin, ymax), }\AttributeTok{expand =} \ConstantTok{FALSE}\NormalTok{) }\SpecialCharTok{+}
  \FunctionTok{annotation\_scale}\NormalTok{(}\AttributeTok{location =} \StringTok{"tl"}\NormalTok{, }\AttributeTok{width\_hint =} \FloatTok{0.3}\NormalTok{) }\SpecialCharTok{+}
  
  \CommentTok{\# Цвет и размер}
  \FunctionTok{scale\_color\_manual}\NormalTok{(}\AttributeTok{values =}\NormalTok{ cluster\_colors) }\SpecialCharTok{+}
  \FunctionTok{scale\_size\_continuous}\NormalTok{(}\AttributeTok{range =} \FunctionTok{c}\NormalTok{(}\DecValTok{1}\NormalTok{, }\DecValTok{8}\NormalTok{), }\AttributeTok{name =} \StringTok{"Величина улова"}\NormalTok{) }\SpecialCharTok{+}
  
  \CommentTok{\# Заголовки и тема}
  \FunctionTok{labs}\NormalTok{(}
    \AttributeTok{title =} \StringTok{"Пространственная автокорреляция уловов краба (LISA)"}\NormalTok{,}
    \AttributeTok{subtitle =} \StringTok{"2023 год, тип съемки: CRAB"}\NormalTok{,}
    \AttributeTok{color =} \StringTok{"Тип кластера"}
\NormalTok{  ) }\SpecialCharTok{+}
  
\FunctionTok{theme\_minimal}\NormalTok{() }\SpecialCharTok{+}
  \FunctionTok{theme}\NormalTok{(}
    \AttributeTok{plot.title =} \FunctionTok{element\_text}\NormalTok{(}\AttributeTok{hjust =} \FloatTok{0.5}\NormalTok{, }\AttributeTok{face =} \StringTok{"bold"}\NormalTok{),}
    \AttributeTok{plot.subtitle =} \FunctionTok{element\_text}\NormalTok{(}\AttributeTok{hjust =} \FloatTok{0.5}\NormalTok{),}
    \AttributeTok{legend.position =} \StringTok{"right"}\NormalTok{,}
    \AttributeTok{panel.border =} \FunctionTok{element\_rect}\NormalTok{(}\AttributeTok{colour =} \StringTok{"black"}\NormalTok{, }\AttributeTok{size =} \DecValTok{1}\NormalTok{, }\AttributeTok{fill =} \ConstantTok{NA}\NormalTok{)  }\CommentTok{\# Рамка вокруг карты}
\NormalTok{  )}
\end{Highlighting}
\end{Shaded}

\section{Карта распределения уловов с автокорреляцией LISA по
фасеткам}\label{ux43aux430ux440ux442ux430-ux440ux430ux441ux43fux440ux435ux434ux435ux43bux435ux43dux438ux44f-ux443ux43bux43eux432ux43eux432-ux441-ux430ux432ux442ux43eux43aux43eux440ux440ux435ux43bux44fux446ux438ux435ux439-lisa-ux43fux43e-ux444ux430ux441ux435ux442ux43aux430ux43c}

\begin{figure}[H]

{\centering \includegraphics[width=0.8\linewidth,height=\textheight,keepaspectratio]{images/KARTOGRAPH7.jpg}

}

\caption{Рис. 7.: Карта распределения уловов с автокорреляцией LISA по
фасеткам}

\end{figure}%

\begin{Shaded}
\begin{Highlighting}[]
\CommentTok{\# Очистка памяти и установка рабочей папки}
\FunctionTok{rm}\NormalTok{(}\AttributeTok{list =} \FunctionTok{ls}\NormalTok{())}
\FunctionTok{setwd}\NormalTok{(}\StringTok{"C:/COURSES/KARTOGRAPH/"}\NormalTok{)}

\CommentTok{\# Загрузка необходимых пакетов}
\FunctionTok{library}\NormalTok{(rnaturalearth)  }\CommentTok{\# Географические карты}
\FunctionTok{library}\NormalTok{(tidyverse)      }\CommentTok{\# Обработка данных и визуализация}
\FunctionTok{library}\NormalTok{(sf)             }\CommentTok{\# Пространственные данные}
\FunctionTok{library}\NormalTok{(spdep)          }\CommentTok{\# Пространственная статистика}
\FunctionTok{library}\NormalTok{(ggspatial)      }\CommentTok{\# Дополнения для карт в ggplot}
\FunctionTok{library}\NormalTok{(readxl)         }\CommentTok{\# Чтение Excel{-}файлов}

\CommentTok{\# 1. ЗАГРУЗКА И ПРЕОБРАЗОВАНИЕ ДАННЫХ}
\CommentTok{\# {-} Чтение данных из Excel}
\CommentTok{\# {-} Фильтрация только данных по крабу}
\NormalTok{DATA }\OtherTok{\textless{}{-}} \FunctionTok{read\_excel}\NormalTok{(}\StringTok{"KARTOGRAPHIC.xlsx"}\NormalTok{, }\AttributeTok{sheet =} \StringTok{"SURVEY"}\NormalTok{) }\SpecialCharTok{\%\textgreater{}\%} 
  \FunctionTok{filter}\NormalTok{(SURV }\SpecialCharTok{==} \StringTok{"CRAB"}\NormalTok{)}

\CommentTok{\# Преобразование в пространственный объект с координатами}
\NormalTok{points\_sf }\OtherTok{\textless{}{-}} \FunctionTok{st\_as\_sf}\NormalTok{(DATA, }\AttributeTok{coords =} \FunctionTok{c}\NormalTok{(}\StringTok{"X"}\NormalTok{, }\StringTok{"Y"}\NormalTok{), }\AttributeTok{crs =} \DecValTok{4326}\NormalTok{)}

\CommentTok{\# 2. ПОДГОТОВКА КАРТОГРАФИЧЕСКОЙ ОСНОВЫ}
\CommentTok{\# {-} Определение границ области исследования}
\NormalTok{xmin }\OtherTok{\textless{}{-}} \DecValTok{37}\NormalTok{; xmax }\OtherTok{\textless{}{-}} \DecValTok{49}\NormalTok{; ymin }\OtherTok{\textless{}{-}} \FloatTok{68.5}\NormalTok{; ymax }\OtherTok{\textless{}{-}} \FloatTok{70.7}

\CommentTok{\# {-} Создание ограничивающего прямоугольника}
\NormalTok{bbox }\OtherTok{\textless{}{-}} \FunctionTok{st\_bbox}\NormalTok{(}\FunctionTok{c}\NormalTok{(}\AttributeTok{xmin =}\NormalTok{ xmin, }\AttributeTok{xmax =}\NormalTok{ xmax, }\AttributeTok{ymin =}\NormalTok{ ymin, }\AttributeTok{ymax =}\NormalTok{ ymax), }\AttributeTok{crs =} \DecValTok{4326}\NormalTok{)}

\CommentTok{\# {-} Загрузка и обрезка карты России по заданным границам}
\NormalTok{russia }\OtherTok{\textless{}{-}} \FunctionTok{ne\_countries}\NormalTok{(}\AttributeTok{country =} \StringTok{"Russia"}\NormalTok{, }\AttributeTok{scale =} \DecValTok{10}\NormalTok{) }\SpecialCharTok{\%\textgreater{}\%} 
  \FunctionTok{st\_as\_sf}\NormalTok{() }\SpecialCharTok{\%\textgreater{}\%} 
  \FunctionTok{st\_crop}\NormalTok{(bbox)}

\CommentTok{\# 3. ФУНКЦИЯ ДЛЯ ПРОСТРАНСТВЕННОГО АНАЛИЗА ПО ГОДАМ}
\NormalTok{analyze\_year }\OtherTok{\textless{}{-}} \ControlFlowTok{function}\NormalTok{(data\_year) \{}
  \CommentTok{\# Удаление дубликатов координат}
\NormalTok{  coords }\OtherTok{\textless{}{-}} \FunctionTok{st\_coordinates}\NormalTok{(data\_year)}
\NormalTok{  data\_year }\OtherTok{\textless{}{-}}\NormalTok{ data\_year[}\SpecialCharTok{!}\FunctionTok{duplicated}\NormalTok{(coords), , drop }\OtherTok{=} \ConstantTok{FALSE}\NormalTok{]}
  
  \CommentTok{\# Перепроецирование в UTM для точных расчетов}
\NormalTok{  points\_utm }\OtherTok{\textless{}{-}} \FunctionTok{st\_transform}\NormalTok{(data\_year, }\AttributeTok{crs =} \DecValTok{32638}\NormalTok{)}
  
  \CommentTok{\# Построение матрицы пространственных весов (4 ближайших соседа)}
\NormalTok{  knn }\OtherTok{\textless{}{-}} \FunctionTok{knearneigh}\NormalTok{(points\_utm, }\AttributeTok{k =} \DecValTok{4}\NormalTok{)}
\NormalTok{  nb }\OtherTok{\textless{}{-}} \FunctionTok{knn2nb}\NormalTok{(knn)}
\NormalTok{  listw }\OtherTok{\textless{}{-}} \FunctionTok{nb2listw}\NormalTok{(nb, }\AttributeTok{style =} \StringTok{"W"}\NormalTok{)  }\CommentTok{\# Стандартизованная матрица}
  
  \CommentTok{\# Расчет локальной пространственной автокорреляции (LISA)}
\NormalTok{  local\_moran }\OtherTok{\textless{}{-}} \FunctionTok{localmoran}\NormalTok{(points\_utm}\SpecialCharTok{$}\NormalTok{PROM, listw)}
  
  \CommentTok{\# Классификация кластеров на основе результатов}
\NormalTok{  points\_utm }\OtherTok{\textless{}{-}}\NormalTok{ points\_utm }\SpecialCharTok{\%\textgreater{}\%}
    \FunctionTok{mutate}\NormalTok{(}
      \AttributeTok{Local\_I =}\NormalTok{ local\_moran[, }\StringTok{"Ii"}\NormalTok{],}
      \AttributeTok{P\_value =}\NormalTok{ local\_moran[, }\StringTok{"Pr(z != E(Ii))"}\NormalTok{],}
      \AttributeTok{Mean\_PROM =} \FunctionTok{mean}\NormalTok{(PROM, }\AttributeTok{na.rm =} \ConstantTok{TRUE}\NormalTok{),}
      \AttributeTok{Cluster =} \FunctionTok{case\_when}\NormalTok{(}
\NormalTok{        Local\_I }\SpecialCharTok{\textgreater{}} \DecValTok{0} \SpecialCharTok{\&}\NormalTok{ PROM }\SpecialCharTok{\textgreater{}}\NormalTok{ Mean\_PROM }\SpecialCharTok{\textasciitilde{}} \StringTok{"High{-}High"}\NormalTok{,     }\CommentTok{\# Горячая точка}
\NormalTok{        Local\_I }\SpecialCharTok{\textgreater{}} \DecValTok{0} \SpecialCharTok{\&}\NormalTok{ PROM }\SpecialCharTok{\textless{}=}\NormalTok{ Mean\_PROM }\SpecialCharTok{\textasciitilde{}} \StringTok{"Low{-}Low"}\NormalTok{,      }\CommentTok{\# Холодная точка}
\NormalTok{        Local\_I }\SpecialCharTok{\textless{}} \DecValTok{0} \SpecialCharTok{\&}\NormalTok{ PROM }\SpecialCharTok{\textgreater{}}\NormalTok{ Mean\_PROM }\SpecialCharTok{\textasciitilde{}} \StringTok{"High{-}Low"}\NormalTok{,      }\CommentTok{\# Выброс (высокий среди низких)}
\NormalTok{        Local\_I }\SpecialCharTok{\textless{}} \DecValTok{0} \SpecialCharTok{\&}\NormalTok{ PROM }\SpecialCharTok{\textless{}=}\NormalTok{ Mean\_PROM }\SpecialCharTok{\textasciitilde{}} \StringTok{"Low{-}High"}\NormalTok{,     }\CommentTok{\# Выброс (низкий среди высоких)}
        \ConstantTok{TRUE} \SpecialCharTok{\textasciitilde{}} \StringTok{"Not significant"}                          \CommentTok{\# Незначимые}
\NormalTok{      )}
\NormalTok{    )}
  
  \CommentTok{\# Возврат в географические координаты}
  \FunctionTok{st\_transform}\NormalTok{(points\_utm, }\AttributeTok{crs =} \DecValTok{4326}\NormalTok{)}
\NormalTok{\}}

\CommentTok{\# 4. ОБРАБОТКА ДАННЫХ ПО ГОДАМ}
\CommentTok{\# {-} Разделение данных по годам}
\CommentTok{\# {-} Применение анализа для каждого года}
\CommentTok{\# {-} Объединение результатов}
\NormalTok{results\_list }\OtherTok{\textless{}{-}}\NormalTok{ DATA }\SpecialCharTok{\%\textgreater{}\%}
  \FunctionTok{group\_split}\NormalTok{(YEAR) }\SpecialCharTok{\%\textgreater{}\%} 
  \FunctionTok{lapply}\NormalTok{(}\ControlFlowTok{function}\NormalTok{(group) \{}
    \FunctionTok{analyze\_year}\NormalTok{(}\FunctionTok{st\_as\_sf}\NormalTok{(group, }\AttributeTok{coords =} \FunctionTok{c}\NormalTok{(}\StringTok{"X"}\NormalTok{, }\StringTok{"Y"}\NormalTok{), }\AttributeTok{crs =} \DecValTok{4326}\NormalTok{))}
\NormalTok{  \}) }\SpecialCharTok{\%\textgreater{}\%}
  \FunctionTok{bind\_rows}\NormalTok{()}

\CommentTok{\# 5. КАТЕГОРИЗАЦИЯ УЛОВОВ}
\CommentTok{\# {-} Расчет квантилей для всего набора данных}
\NormalTok{PROM\_breaks }\OtherTok{\textless{}{-}} \FunctionTok{quantile}\NormalTok{(results\_list}\SpecialCharTok{$}\NormalTok{PROM, }
                         \AttributeTok{probs =} \FunctionTok{c}\NormalTok{(}\DecValTok{0}\NormalTok{, }\FloatTok{0.25}\NormalTok{, }\FloatTok{0.5}\NormalTok{, }\FloatTok{0.75}\NormalTok{, }\DecValTok{1}\NormalTok{), }
                         \AttributeTok{na.rm =} \ConstantTok{TRUE}\NormalTok{) }\SpecialCharTok{\%\textgreater{}\%} 
  \FunctionTok{round}\NormalTok{(}\DecValTok{1}\NormalTok{)  }\CommentTok{\# Округление значений}

\CommentTok{\# {-} Создание меток с реальными диапазонами}
\NormalTok{PROM\_labels }\OtherTok{\textless{}{-}} \FunctionTok{sprintf}\NormalTok{(}\StringTok{"\%.1f {-} \%.1f"}\NormalTok{, PROM\_breaks[}\DecValTok{1}\SpecialCharTok{:}\DecValTok{4}\NormalTok{], PROM\_breaks[}\DecValTok{2}\SpecialCharTok{:}\DecValTok{5}\NormalTok{])}

\CommentTok{\# {-} Добавление категорий уловов в данные}
\NormalTok{results\_list }\OtherTok{\textless{}{-}}\NormalTok{ results\_list }\SpecialCharTok{\%\textgreater{}\%}
  \FunctionTok{mutate}\NormalTok{(}
    \AttributeTok{PROM\_category =} \FunctionTok{cut}\NormalTok{(}
\NormalTok{      PROM, }
      \AttributeTok{breaks =}\NormalTok{ PROM\_breaks, }
      \AttributeTok{labels =}\NormalTok{ PROM\_labels,}
      \AttributeTok{include.lowest =} \ConstantTok{TRUE}
\NormalTok{    )}
\NormalTok{  )}

\CommentTok{\# 6. ВИЗУАЛИЗАЦИЯ РЕЗУЛЬТАТОВ}
\CommentTok{\# Цветовая схема для типов кластеров}
\NormalTok{cluster\_colors }\OtherTok{\textless{}{-}} \FunctionTok{c}\NormalTok{(}
  \StringTok{"High{-}High"} \OtherTok{=} \StringTok{"red"}\NormalTok{,       }\CommentTok{\# Горячие точки}
  \StringTok{"Low{-}Low"} \OtherTok{=} \StringTok{"blue"}\NormalTok{,        }\CommentTok{\# Холодные точки}
  \StringTok{"High{-}Low"} \OtherTok{=} \StringTok{"pink"}\NormalTok{,       }\CommentTok{\# Выбросы высокие}
  \StringTok{"Low{-}High"} \OtherTok{=} \StringTok{"lightblue"}\NormalTok{,  }\CommentTok{\# Выбросы низкие}
  \StringTok{"Not significant"} \OtherTok{=} \StringTok{"gray"} \CommentTok{\# Незначимые}
\NormalTok{)}

\CommentTok{\# Построение карты}
\FunctionTok{ggplot}\NormalTok{(}\AttributeTok{data =}\NormalTok{ results\_list) }\SpecialCharTok{+}
  \CommentTok{\# Базовая карта России}
  \FunctionTok{geom\_sf}\NormalTok{(}\AttributeTok{data =}\NormalTok{ russia, }\AttributeTok{fill =} \StringTok{"\#E8E5D6"}\NormalTok{, }\AttributeTok{color =} \StringTok{"black"}\NormalTok{, }\AttributeTok{inherit.aes =} \ConstantTok{FALSE}\NormalTok{) }\SpecialCharTok{+}
  
  \CommentTok{\# Точки наблюдений с цветом по кластерам и размером по уловам}
  \FunctionTok{geom\_sf}\NormalTok{(}\FunctionTok{aes}\NormalTok{(}\AttributeTok{color =}\NormalTok{ Cluster, }\AttributeTok{size =}\NormalTok{ PROM\_category), }\AttributeTok{alpha =} \FloatTok{0.8}\NormalTok{) }\SpecialCharTok{+}
  
  \CommentTok{\# Разделение на панели по годам}
  \FunctionTok{facet\_wrap}\NormalTok{(}\SpecialCharTok{\textasciitilde{}}\NormalTok{ YEAR, }\AttributeTok{ncol =} \DecValTok{2}\NormalTok{) }\SpecialCharTok{+}
  
  \CommentTok{\# Установка границ карты}
  \FunctionTok{coord\_sf}\NormalTok{(}\AttributeTok{xlim =} \FunctionTok{c}\NormalTok{(xmin, xmax), }\AttributeTok{ylim =} \FunctionTok{c}\NormalTok{(ymin, ymax), }\AttributeTok{expand =} \ConstantTok{FALSE}\NormalTok{) }\SpecialCharTok{+}
  
  \CommentTok{\# Настройка легенды для кластеров}
  \FunctionTok{scale\_color\_manual}\NormalTok{(}
    \AttributeTok{values =}\NormalTok{ cluster\_colors,}
    \AttributeTok{name =} \StringTok{"Тип кластера"}\NormalTok{,}
    \AttributeTok{guide =} \FunctionTok{guide\_legend}\NormalTok{(}\AttributeTok{nrow =} \DecValTok{2}\NormalTok{)}
\NormalTok{  ) }\SpecialCharTok{+}
  
  \CommentTok{\# Настройка легенды для уловов (реальные диапазоны)}
  \FunctionTok{scale\_size\_manual}\NormalTok{(}
    \AttributeTok{name =} \StringTok{"Величина улова"}\NormalTok{,}
    \AttributeTok{values =} \FunctionTok{c}\NormalTok{(}\DecValTok{1}\NormalTok{, }\DecValTok{2}\NormalTok{, }\DecValTok{3}\NormalTok{, }\DecValTok{5}\NormalTok{),  }\CommentTok{\# Размеры точек для 4{-}х категорий}
    \AttributeTok{breaks =} \FunctionTok{levels}\NormalTok{(results\_list}\SpecialCharTok{$}\NormalTok{PROM\_category),}
    \AttributeTok{guide =} \FunctionTok{guide\_legend}\NormalTok{(}\AttributeTok{nrow =} \DecValTok{2}\NormalTok{)}
\NormalTok{  ) }\SpecialCharTok{+}
  
  \CommentTok{\# Заголовки и подписи}
  \FunctionTok{labs}\NormalTok{(}
    \AttributeTok{title =} \StringTok{"Пространственная автокорреляция уловов краба (LISA)"}\NormalTok{,}
    \AttributeTok{subtitle =} \StringTok{"Тип съемки: CRAB"}
\NormalTok{  ) }\SpecialCharTok{+}
  
  \CommentTok{\# Оформление графика}
  \FunctionTok{theme\_minimal}\NormalTok{() }\SpecialCharTok{+}
  \FunctionTok{theme}\NormalTok{(}
    \AttributeTok{axis.text.x =} \FunctionTok{element\_text}\NormalTok{(}\AttributeTok{size =} \DecValTok{9}\NormalTok{, }\AttributeTok{margin =} \FunctionTok{margin}\NormalTok{(}\AttributeTok{t =} \DecValTok{5}\NormalTok{)),}
    \AttributeTok{axis.text.y =} \FunctionTok{element\_text}\NormalTok{(}\AttributeTok{size =} \DecValTok{9}\NormalTok{, }\AttributeTok{angle =} \DecValTok{90}\NormalTok{, }\AttributeTok{hjust =} \FloatTok{0.5}\NormalTok{, }\AttributeTok{margin =} \FunctionTok{margin}\NormalTok{(}\AttributeTok{r =} \DecValTok{5}\NormalTok{)),}
    \AttributeTok{panel.background =} \FunctionTok{element\_rect}\NormalTok{(}\AttributeTok{fill =} \StringTok{"\#F0F8FF"}\NormalTok{, }\AttributeTok{color =} \ConstantTok{NA}\NormalTok{),  }\CommentTok{\# Фон океана}
    \AttributeTok{panel.grid.major =} \FunctionTok{element\_line}\NormalTok{(}\AttributeTok{color =} \StringTok{"grey90"}\NormalTok{, }\AttributeTok{linetype =} \StringTok{"dotted"}\NormalTok{),}
    \AttributeTok{legend.position =} \StringTok{"bottom"}\NormalTok{,           }\CommentTok{\# Легенда внизу}
    \AttributeTok{legend.box =} \StringTok{"horizontal"}\NormalTok{,            }\CommentTok{\# Горизонтальное расположение}
    \AttributeTok{panel.border =} \FunctionTok{element\_rect}\NormalTok{(}\AttributeTok{fill =} \ConstantTok{NA}\NormalTok{, }\AttributeTok{color =} \StringTok{"black"}\NormalTok{, }\AttributeTok{size =} \FloatTok{0.7}\NormalTok{),}
    \AttributeTok{strip.background =} \FunctionTok{element\_rect}\NormalTok{(}\AttributeTok{fill =} \StringTok{"white"}\NormalTok{, }\AttributeTok{color =} \StringTok{"black"}\NormalTok{, }\AttributeTok{size =} \FloatTok{0.7}\NormalTok{),  }\CommentTok{\# Заголовки панелей}
    \AttributeTok{strip.text =} \FunctionTok{element\_text}\NormalTok{(}\AttributeTok{size =} \DecValTok{11}\NormalTok{, }\AttributeTok{face =} \StringTok{"bold"}\NormalTok{)}
\NormalTok{  ) }\SpecialCharTok{+}
  
  \CommentTok{\# Разметка осей (долгота с шагом 2°, широта с шагом 1°)}
  \FunctionTok{scale\_x\_continuous}\NormalTok{(}
    \AttributeTok{breaks =} \FunctionTok{seq}\NormalTok{(}\FunctionTok{floor}\NormalTok{(xmin), }\FunctionTok{ceiling}\NormalTok{(xmax), }\AttributeTok{by =} \DecValTok{2}\NormalTok{),}
    \AttributeTok{labels =} \ControlFlowTok{function}\NormalTok{(x) }\FunctionTok{paste0}\NormalTok{(x, }\StringTok{"°E"}\NormalTok{)}
\NormalTok{  ) }\SpecialCharTok{+}
  \FunctionTok{scale\_y\_continuous}\NormalTok{(}
    \AttributeTok{breaks =} \FunctionTok{seq}\NormalTok{(}\FunctionTok{floor}\NormalTok{(ymin), }\FunctionTok{ceiling}\NormalTok{(ymax), }\AttributeTok{by =} \DecValTok{1}\NormalTok{),  }
    \AttributeTok{labels =} \ControlFlowTok{function}\NormalTok{(y) }\FunctionTok{paste0}\NormalTok{(y, }\StringTok{"°N"}\NormalTok{)}
\NormalTok{  )}
\end{Highlighting}
\end{Shaded}

\section{Промысловые карты с квартильным распределением
уловов}\label{ux43fux440ux43eux43cux44bux441ux43bux43eux432ux44bux435-ux43aux430ux440ux442ux44b-ux441-ux43aux432ux430ux440ux442ux438ux43bux44cux43dux44bux43c-ux440ux430ux441ux43fux440ux435ux434ux435ux43bux435ux43dux438ux435ux43c-ux443ux43bux43eux432ux43eux432}

\begin{figure}[H]

{\centering \includegraphics[width=0.8\linewidth,height=\textheight,keepaspectratio]{images/KARTOGRAPH8.jpg}

}

\caption{Рис. 8.: Промысловые карты с квартильным распределением уловов}

\end{figure}%

\begin{Shaded}
\begin{Highlighting}[]
\CommentTok{\# Очистка окружения и установка рабочей директории}
\FunctionTok{rm}\NormalTok{(}\AttributeTok{list =} \FunctionTok{ls}\NormalTok{())}
\FunctionTok{setwd}\NormalTok{(}\StringTok{"C:/COURSES/KARTOGRAPH/"}\NormalTok{)}

\CommentTok{\# Загрузка необходимых библиотек}
\FunctionTok{library}\NormalTok{(rnaturalearth)}
\FunctionTok{library}\NormalTok{(tidyverse)}
\FunctionTok{library}\NormalTok{(sf)}

\DocumentationTok{\#\#\#\#\#\#\# ЗАГРУЗКА ДАННЫХ И ПОДГОТОВКА ПРОСТРАНСТВЕННЫХ ОБЪЕКТОВ \#\#\#\#\#\#\#\#\#\#\#\#\#\#\#\#}

\CommentTok{\# Чтение и фильтрация данных}
\NormalTok{DATA }\OtherTok{\textless{}{-}}\NormalTok{ readxl}\SpecialCharTok{::}\FunctionTok{read\_excel}\NormalTok{(}\StringTok{"KARTOGRAPHIC.xlsx"}\NormalTok{, }\AttributeTok{sheet =} \StringTok{"FISHERY"}\NormalTok{) }\SpecialCharTok{\%\textgreater{}\%} 
  \FunctionTok{filter}\NormalTok{(YEAR }\SpecialCharTok{==} \DecValTok{2023}\NormalTok{)}

\CommentTok{\# Получение границ России}
\NormalTok{russia }\OtherTok{\textless{}{-}} \FunctionTok{ne\_countries}\NormalTok{(}\AttributeTok{scale =} \DecValTok{10}\NormalTok{, }\AttributeTok{country =} \StringTok{"Russia"}\NormalTok{) }\SpecialCharTok{\%\textgreater{}\%} 
  \FunctionTok{st\_as\_sf}\NormalTok{()}

\CommentTok{\# Установка границ отображаемой области}
\NormalTok{xmin}\OtherTok{=}\DecValTok{37}\NormalTok{; xmax}\OtherTok{=}\DecValTok{48}\NormalTok{; ymin}\OtherTok{=}\FloatTok{68.6}\NormalTok{; ymax}\OtherTok{=}\DecValTok{71}

\DocumentationTok{\#\#\#\#\#\#\# ПОДГОТОВКА ДАННЫХ ДЛЯ ВИЗУАЛИЗАЦИИ \#\#\#\#\#\#\#\#\#\#\#\#\#\#\#\#}
\CommentTok{\# Вычисляем квартили отдельно}
\NormalTok{quantiles }\OtherTok{\textless{}{-}} \FunctionTok{quantile}\NormalTok{(DATA}\SpecialCharTok{$}\NormalTok{CPUE[DATA}\SpecialCharTok{$}\NormalTok{CPUE }\SpecialCharTok{\textgreater{}} \DecValTok{0}\NormalTok{], }\AttributeTok{probs =} \FunctionTok{seq}\NormalTok{(}\DecValTok{0}\NormalTok{, }\DecValTok{1}\NormalTok{, }\FloatTok{0.25}\NormalTok{))}

\CommentTok{\# Создаем 4 категории с реальными диапазонами значений}
\NormalTok{nonzero\_data }\OtherTok{\textless{}{-}}\NormalTok{ DATA }\SpecialCharTok{\%\textgreater{}\%} 
  \FunctionTok{filter}\NormalTok{(CPUE }\SpecialCharTok{\textgreater{}} \DecValTok{0}\NormalTok{) }\SpecialCharTok{\%\textgreater{}\%}
  \FunctionTok{mutate}\NormalTok{(}
    \AttributeTok{CPUE\_cat =} \FunctionTok{cut}\NormalTok{(}
\NormalTok{      CPUE,}
      \AttributeTok{breaks =}\NormalTok{ quantiles,}
      \AttributeTok{include.lowest =} \ConstantTok{TRUE}\NormalTok{,}
      \AttributeTok{labels =} \FunctionTok{c}\NormalTok{(}
        \FunctionTok{sprintf}\NormalTok{(}\StringTok{"\%.1f {-} \%.1f"}\NormalTok{, quantiles[}\DecValTok{1}\NormalTok{], quantiles[}\DecValTok{2}\NormalTok{]),}
        \FunctionTok{sprintf}\NormalTok{(}\StringTok{"\%.1f {-} \%.1f"}\NormalTok{, quantiles[}\DecValTok{2}\NormalTok{], quantiles[}\DecValTok{3}\NormalTok{]),}
        \FunctionTok{sprintf}\NormalTok{(}\StringTok{"\%.1f {-} \%.1f"}\NormalTok{, quantiles[}\DecValTok{3}\NormalTok{], quantiles[}\DecValTok{4}\NormalTok{]),}
        \FunctionTok{sprintf}\NormalTok{(}\StringTok{"\%.1f {-} \%.1f"}\NormalTok{, quantiles[}\DecValTok{4}\NormalTok{], quantiles[}\DecValTok{5}\NormalTok{])}
\NormalTok{      )}
\NormalTok{    )}
\NormalTok{  )}

\CommentTok{\# Построение карты}
\FunctionTok{ggplot}\NormalTok{() }\SpecialCharTok{+}
  \CommentTok{\# Базовая карта России}
  \FunctionTok{geom\_sf}\NormalTok{(}\AttributeTok{data =}\NormalTok{ russia, }\AttributeTok{fill =} \StringTok{"lightblue"}\NormalTok{, }\AttributeTok{color =} \StringTok{"gray40"}\NormalTok{) }\SpecialCharTok{+} 
  \CommentTok{\# Ограничение области отображения}
  \FunctionTok{coord\_sf}\NormalTok{(}\AttributeTok{xlim =} \FunctionTok{c}\NormalTok{(xmin, xmax), }\AttributeTok{ylim =} \FunctionTok{c}\NormalTok{(ymin, ymax)) }\SpecialCharTok{+}
  \CommentTok{\# Точки наблюдений с категориальным размером}
  \FunctionTok{geom\_point}\NormalTok{(}
    \AttributeTok{data =}\NormalTok{ nonzero\_data,}
    \FunctionTok{aes}\NormalTok{(}\AttributeTok{x =}\NormalTok{ X, }\AttributeTok{y =}\NormalTok{ Y, }\AttributeTok{size =}\NormalTok{ CPUE\_cat, }\AttributeTok{color =}\NormalTok{ CPUE),}
    \AttributeTok{alpha =} \FloatTok{0.7}
\NormalTok{  ) }\SpecialCharTok{+}
  \CommentTok{\# Точки для нулевых уловов (крестики)}
  \FunctionTok{geom\_point}\NormalTok{(}
    \AttributeTok{data =} \FunctionTok{filter}\NormalTok{(DATA, CPUE }\SpecialCharTok{==} \DecValTok{0}\NormalTok{),}
    \FunctionTok{aes}\NormalTok{(}\AttributeTok{x =}\NormalTok{ X, }\AttributeTok{y =}\NormalTok{ Y),}
    \AttributeTok{shape =} \DecValTok{4}\NormalTok{, }\AttributeTok{size =} \FloatTok{1.2}\NormalTok{, }\AttributeTok{stroke =} \DecValTok{1}\NormalTok{, }\AttributeTok{color =} \StringTok{"black"}
\NormalTok{  ) }\SpecialCharTok{+}
  \CommentTok{\# Цветовая шкала (непрерывная)}
  \FunctionTok{scale\_color\_viridis\_c}\NormalTok{(}\AttributeTok{option =} \StringTok{"H"}\NormalTok{, }\AttributeTok{name =} \ConstantTok{NULL}\NormalTok{) }\SpecialCharTok{+}
  \CommentTok{\# Ручная настройка размеров для категорий}
  \FunctionTok{scale\_size\_manual}\NormalTok{(}
    \AttributeTok{name =} \StringTok{"CPUE"}\NormalTok{,}
    \AttributeTok{values =} \FunctionTok{c}\NormalTok{(}\DecValTok{2}\NormalTok{, }\DecValTok{4}\NormalTok{, }\DecValTok{6}\NormalTok{, }\DecValTok{8}\NormalTok{),  }\CommentTok{\# Размеры точек для категорий}
    \AttributeTok{drop =} \ConstantTok{FALSE}
\NormalTok{  ) }\SpecialCharTok{+}
  \CommentTok{\# Настройки темы}
  \FunctionTok{theme\_bw}\NormalTok{() }\SpecialCharTok{+}
  \FunctionTok{labs}\NormalTok{(}
    \AttributeTok{title =} \StringTok{"Распределение CPUE краба (2023)"}\NormalTok{,}
    \AttributeTok{subtitle =} \ConstantTok{NULL}\NormalTok{,}
    \AttributeTok{x =} \StringTok{"Долгота"}\NormalTok{, }
    \AttributeTok{y =} \StringTok{"Широта"}
\NormalTok{  ) }\SpecialCharTok{+}
  \FunctionTok{theme}\NormalTok{(}
    \AttributeTok{panel.grid =} \FunctionTok{element\_line}\NormalTok{(}\AttributeTok{color =} \StringTok{"gray90"}\NormalTok{),}
    \AttributeTok{legend.position =} \StringTok{"bottom"}
\NormalTok{  )}
\end{Highlighting}
\end{Shaded}

\section{Промысловые карты с агрегацией в центрах полигонов
(промквадратов)}\label{ux43fux440ux43eux43cux44bux441ux43bux43eux432ux44bux435-ux43aux430ux440ux442ux44b-ux441-ux430ux433ux440ux435ux433ux430ux446ux438ux435ux439-ux432-ux446ux435ux43dux442ux440ux430ux445-ux43fux43eux43bux438ux433ux43eux43dux43eux432-ux43fux440ux43eux43cux43aux432ux430ux434ux440ux430ux442ux43eux432}

\begin{figure}[H]

{\centering \includegraphics[width=0.8\linewidth,height=\textheight,keepaspectratio]{images/KARTOGRAPH9.jpg}

}

\caption{Рис. 9.: Промысловые карты с агрегацией в центрах полигонов
(промквадратов)}

\end{figure}%

\begin{Shaded}
\begin{Highlighting}[]
\CommentTok{\# Очистка окружения и установка рабочей директории}
\FunctionTok{rm}\NormalTok{(}\AttributeTok{list =} \FunctionTok{ls}\NormalTok{())}
\FunctionTok{setwd}\NormalTok{(}\StringTok{"C:/COURSES/KARTOGRAPH/"}\NormalTok{)}

\CommentTok{\# Загрузка необходимых библиотек}
\FunctionTok{library}\NormalTok{(rnaturalearth)}
\FunctionTok{library}\NormalTok{(tidyverse)}
\FunctionTok{library}\NormalTok{(sf)}

\DocumentationTok{\#\#\#\#\#\#\# ЗАГРУЗКА ДАННЫХ И ПОДГОТОВКА ПРОСТРАНСТВЕННЫХ ОБЪЕКТОВ \#\#\#\#\#\#\#\#\#\#\#\#\#\#\#\#}

\CommentTok{\# Чтение и фильтрация данных}
\NormalTok{DATA }\OtherTok{\textless{}{-}}\NormalTok{ readxl}\SpecialCharTok{::}\FunctionTok{read\_excel}\NormalTok{(}\StringTok{"KARTOGRAPHIC.xlsx"}\NormalTok{, }\AttributeTok{sheet =} \StringTok{"FISHERY"}\NormalTok{) }\SpecialCharTok{\%\textgreater{}\%} 
  \FunctionTok{filter}\NormalTok{(YEAR }\SpecialCharTok{==} \DecValTok{2023}\NormalTok{)}

\CommentTok{\# Преобразуем CPUE в пространственные точки}
\NormalTok{spec\_points }\OtherTok{\textless{}{-}} \FunctionTok{st\_as\_sf}\NormalTok{(DATA, }\AttributeTok{coords =} \FunctionTok{c}\NormalTok{(}\StringTok{"X"}\NormalTok{, }\StringTok{"Y"}\NormalTok{), }\AttributeTok{crs =} \DecValTok{4326}\NormalTok{)}

\CommentTok{\# Карта России}
\NormalTok{russia }\OtherTok{\textless{}{-}} \FunctionTok{ne\_countries}\NormalTok{(}\AttributeTok{scale =} \DecValTok{10}\NormalTok{, }\AttributeTok{country =} \StringTok{"Russia"}\NormalTok{) }

\CommentTok{\# Параметры карты и сетки}
\NormalTok{xmin }\OtherTok{\textless{}{-}} \DecValTok{32}\NormalTok{; xmax }\OtherTok{\textless{}{-}} \DecValTok{48}\NormalTok{; ymin }\OtherTok{\textless{}{-}} \DecValTok{68}\NormalTok{; ymax }\OtherTok{\textless{}{-}} \DecValTok{72}
\NormalTok{xcs }\OtherTok{\textless{}{-}} \DecValTok{1}\NormalTok{; ycs }\OtherTok{\textless{}{-}} \FloatTok{0.25}


\CommentTok{\# Создание основного датафрейма и пространственных объектов}
\NormalTok{points\_sf }\OtherTok{\textless{}{-}} \FunctionTok{st\_as\_sf}\NormalTok{(DATA, }\AttributeTok{coords =} \FunctionTok{c}\NormalTok{(}\StringTok{"X"}\NormalTok{, }\StringTok{"Y"}\NormalTok{), }\AttributeTok{crs =} \DecValTok{4326}\NormalTok{)}

\CommentTok{\# Создание сетки}
\NormalTok{grid\_sf }\OtherTok{\textless{}{-}} \FunctionTok{st\_make\_grid}\NormalTok{(points\_sf, }
                        \AttributeTok{cellsize =} \FunctionTok{c}\NormalTok{(xcs, ycs),}
                        \AttributeTok{n =} \FunctionTok{c}\NormalTok{(}\DecValTok{2} \SpecialCharTok{+}\NormalTok{ (xmax }\SpecialCharTok{{-}}\NormalTok{ xmin)}\SpecialCharTok{/}\NormalTok{xcs, }\DecValTok{2} \SpecialCharTok{+}\NormalTok{ (ymax }\SpecialCharTok{{-}}\NormalTok{ ymin)}\SpecialCharTok{/}\NormalTok{ycs),}
                        \AttributeTok{offset =} \FunctionTok{c}\NormalTok{(xmin }\SpecialCharTok{{-}}\NormalTok{ xcs, ymin }\SpecialCharTok{{-}}\NormalTok{ ycs)) }\SpecialCharTok{\%\textgreater{}\%} 
  \FunctionTok{st\_sf}\NormalTok{() }\SpecialCharTok{\%\textgreater{}\%} 
  \FunctionTok{mutate}\NormalTok{(}\AttributeTok{cell\_id =} \FunctionTok{row\_number}\NormalTok{())}

\CommentTok{\# Присоединяем точки Catch к сетке и агрегируем по ячейкам и годам}
\NormalTok{shares\_df\_catch }\OtherTok{\textless{}{-}} \FunctionTok{st\_join}\NormalTok{(points\_sf, grid\_sf) }\SpecialCharTok{\%\textgreater{}\%} 
  \FunctionTok{st\_drop\_geometry}\NormalTok{() }\SpecialCharTok{\%\textgreater{}\%} 
  \FunctionTok{group\_by}\NormalTok{(cell\_id, YEAR) }\SpecialCharTok{\%\textgreater{}\%} 
  \FunctionTok{summarise}\NormalTok{(}
    \AttributeTok{Count =} \FunctionTok{n}\NormalTok{(),}
    \AttributeTok{CATCH =} \FunctionTok{mean}\NormalTok{(CPUE, }\AttributeTok{na.rm =} \ConstantTok{TRUE}\NormalTok{)}
\NormalTok{  ) }\SpecialCharTok{\%\textgreater{}\%} 
  \FunctionTok{ungroup}\NormalTok{()}

\CommentTok{\# Присоединяем статистику Catch к сетке}
\NormalTok{gird\_shares\_catch }\OtherTok{\textless{}{-}} \FunctionTok{right\_join}\NormalTok{(grid\_sf, shares\_df\_catch, }\AttributeTok{by =} \StringTok{"cell\_id"}\NormalTok{)}



\CommentTok{\# Центроиды сетки по W}
\NormalTok{CENTROIDS\_W }\OtherTok{\textless{}{-}}\NormalTok{ gird\_shares\_catch }\SpecialCharTok{\%\textgreater{}\%} 
  \FunctionTok{st\_centroid}\NormalTok{()}

\DocumentationTok{\#\#\#\#\#\#\#\#\#\#\#\#\#\#\#\#\#\#\#\# ВИЗУАЛИЗАЦИЯ \#\#\#\#\#\#\#\#\#\#\#\#\#\#\#\#\#\#\#\#\#\#\#\#\#\#\#\#\#\#\#\#\#\#\#\#\#\#\#\#\#}

\FunctionTok{ggplot}\NormalTok{() }\SpecialCharTok{+}
  \CommentTok{\# 1. Сетка без заливки}
  \FunctionTok{geom\_sf}\NormalTok{(}\AttributeTok{data =}\NormalTok{ grid\_sf, }\AttributeTok{fill =} \ConstantTok{NA}\NormalTok{, }\AttributeTok{color =} \StringTok{"grey80"}\NormalTok{, }\AttributeTok{linewidth =} \FloatTok{0.3}\NormalTok{) }\SpecialCharTok{+}
  
  \CommentTok{\# 2. Границы России}
  \FunctionTok{geom\_sf}\NormalTok{(}\AttributeTok{data =}\NormalTok{ russia, }\AttributeTok{fill =} \StringTok{"grey95"}\NormalTok{) }\SpecialCharTok{+}
  
  \CommentTok{\# 3. Центроиды ячеек с CATCH (цвет и размер по значению)}
  \FunctionTok{geom\_sf}\NormalTok{(}
    \AttributeTok{data =}\NormalTok{ CENTROIDS\_W, }
    \FunctionTok{aes}\NormalTok{(}\AttributeTok{size =}\NormalTok{ CATCH, }\AttributeTok{color =}\NormalTok{ CATCH),}
    \AttributeTok{shape =} \DecValTok{16}\NormalTok{, }
    \AttributeTok{alpha =} \FloatTok{0.7}
\NormalTok{  ) }\SpecialCharTok{+}
  
  \CommentTok{\# 4. Цветовая шкала (viridis как в первом скрипте)}
  \FunctionTok{scale\_color\_viridis\_c}\NormalTok{(}
    \AttributeTok{option =} \StringTok{"H"}\NormalTok{, }
    \AttributeTok{name =} \ConstantTok{NULL}\NormalTok{,}
    \AttributeTok{limits =} \FunctionTok{c}\NormalTok{(}\DecValTok{0}\NormalTok{, }\FunctionTok{max}\NormalTok{(gird\_shares\_catch}\SpecialCharTok{$}\NormalTok{CATCH, }\AttributeTok{na.rm =} \ConstantTok{TRUE}\NormalTok{))}
\NormalTok{  ) }\SpecialCharTok{+}
  
  \CommentTok{\# 5. Шкала размера центроидов}
  \FunctionTok{scale\_size\_continuous}\NormalTok{(}
    \AttributeTok{range =} \FunctionTok{c}\NormalTok{(}\DecValTok{1}\NormalTok{, }\DecValTok{10}\NormalTok{), }
    \AttributeTok{name =} \StringTok{"CPUE"}
\NormalTok{  ) }\SpecialCharTok{+}
  
  \CommentTok{\# 6. Обрезаем область отображения}
  \FunctionTok{coord\_sf}\NormalTok{(}
    \AttributeTok{xlim =} \FunctionTok{c}\NormalTok{(xmin, xmax), }
    \AttributeTok{ylim =} \FunctionTok{c}\NormalTok{(ymin, ymax),}
    \AttributeTok{expand =} \ConstantTok{FALSE}  \CommentTok{\# Точное соответствие границ}
\NormalTok{  ) }\SpecialCharTok{+}
  
  \CommentTok{\# 7. Шкалы для осей координат}
  \FunctionTok{scale\_x\_continuous}\NormalTok{(}
    \AttributeTok{breaks =} \FunctionTok{seq}\NormalTok{(xmin, xmax, }\AttributeTok{by =} \DecValTok{2}\NormalTok{),  }\CommentTok{\# Метки каждые 2 градуса}
    \AttributeTok{name =} \StringTok{"Долгота"}
\NormalTok{  ) }\SpecialCharTok{+}
  \FunctionTok{scale\_y\_continuous}\NormalTok{(}
    \AttributeTok{breaks =} \FunctionTok{seq}\NormalTok{(ymin, ymax, }\AttributeTok{by =} \FloatTok{0.5}\NormalTok{),  }\CommentTok{\# Метки каждые 0.5 градуса}
    \AttributeTok{name =} \StringTok{"Широта"}
\NormalTok{  ) }\SpecialCharTok{+}
  
  \CommentTok{\# 8. Тема оформления}
  \FunctionTok{theme\_minimal}\NormalTok{() }\SpecialCharTok{+}
  \FunctionTok{theme}\NormalTok{(}
    \AttributeTok{panel.grid =} \FunctionTok{element\_blank}\NormalTok{(),}
    \AttributeTok{legend.position =} \StringTok{"bottom"}\NormalTok{,}
    \AttributeTok{panel.border =} \FunctionTok{element\_rect}\NormalTok{(}\AttributeTok{fill =} \ConstantTok{NA}\NormalTok{, }\AttributeTok{color =} \StringTok{"black"}\NormalTok{, }\AttributeTok{size =} \FloatTok{0.5}\NormalTok{),}
    \CommentTok{\# Добавляем сетку для осей координат}
    \AttributeTok{panel.grid.major =} \FunctionTok{element\_line}\NormalTok{(}\AttributeTok{color =} \StringTok{"gray90"}\NormalTok{, }\AttributeTok{linewidth =} \FloatTok{0.2}\NormalTok{)}
\NormalTok{  ) }\SpecialCharTok{+}
  
  \CommentTok{\# 9. Явное указание названий осей (дублируем для надежности)}
  \FunctionTok{labs}\NormalTok{(}\AttributeTok{x =} \StringTok{"Долгота"}\NormalTok{, }\AttributeTok{y =} \StringTok{"Широта"}\NormalTok{)}
\end{Highlighting}
\end{Shaded}

\section{Промысловые карты -
картограммы}\label{ux43fux440ux43eux43cux44bux441ux43bux43eux432ux44bux435-ux43aux430ux440ux442ux44b---ux43aux430ux440ux442ux43eux433ux440ux430ux43cux43cux44b}

\begin{figure}[H]

{\centering \includegraphics[width=0.8\linewidth,height=\textheight,keepaspectratio]{images/KARTOGRAPH10.jpg}

}

\caption{Рис. 10.: Промысловые карты - картограммы}

\end{figure}%

\begin{Shaded}
\begin{Highlighting}[]
\CommentTok{\# Очистка окружения и установка рабочей директории}
\FunctionTok{rm}\NormalTok{(}\AttributeTok{list =} \FunctionTok{ls}\NormalTok{())}
\FunctionTok{setwd}\NormalTok{(}\StringTok{"C:/COURSES/KARTOGRAPH/"}\NormalTok{)}

\CommentTok{\# Загрузка необходимых библиотек}
\FunctionTok{library}\NormalTok{(rnaturalearth)}
\FunctionTok{library}\NormalTok{(tidyverse)}
\FunctionTok{library}\NormalTok{(sf)}

\DocumentationTok{\#\#\#\#\#\#\# ЗАГРУЗКА ДАННЫХ И ПОДГОТОВКА ПРОСТРАНСТВЕННЫХ ОБЪЕКТОВ \#\#\#\#\#\#\#\#\#\#\#\#\#\#\#\#}

\CommentTok{\# Чтение и фильтрация данных}
\NormalTok{DATA }\OtherTok{\textless{}{-}}\NormalTok{ readxl}\SpecialCharTok{::}\FunctionTok{read\_excel}\NormalTok{(}\StringTok{"KARTOGRAPHIC.xlsx"}\NormalTok{, }\AttributeTok{sheet =} \StringTok{"FISHERY"}\NormalTok{) }\SpecialCharTok{\%\textgreater{}\%} 
  \FunctionTok{filter}\NormalTok{(YEAR }\SpecialCharTok{==} \DecValTok{2023}\NormalTok{)}

\CommentTok{\# Преобразуем CPUE в пространственные точки}
\NormalTok{spec\_points }\OtherTok{\textless{}{-}} \FunctionTok{st\_as\_sf}\NormalTok{(DATA, }\AttributeTok{coords =} \FunctionTok{c}\NormalTok{(}\StringTok{"X"}\NormalTok{, }\StringTok{"Y"}\NormalTok{), }\AttributeTok{crs =} \DecValTok{4326}\NormalTok{)}

\CommentTok{\# Карта России}
\NormalTok{russia }\OtherTok{\textless{}{-}} \FunctionTok{ne\_countries}\NormalTok{(}\AttributeTok{scale =} \DecValTok{10}\NormalTok{, }\AttributeTok{country =} \StringTok{"Russia"}\NormalTok{) }

\CommentTok{\# Параметры карты и сетки}
\NormalTok{xmin }\OtherTok{\textless{}{-}} \DecValTok{32}\NormalTok{; xmax }\OtherTok{\textless{}{-}} \DecValTok{48}\NormalTok{; ymin }\OtherTok{\textless{}{-}} \DecValTok{68}\NormalTok{; ymax }\OtherTok{\textless{}{-}} \DecValTok{72}
\NormalTok{xcs }\OtherTok{\textless{}{-}} \DecValTok{1}\NormalTok{; ycs }\OtherTok{\textless{}{-}} \FloatTok{0.25}


\CommentTok{\# Создание основного датафрейма и пространственных объектов}
\NormalTok{points\_sf }\OtherTok{\textless{}{-}} \FunctionTok{st\_as\_sf}\NormalTok{(DATA, }\AttributeTok{coords =} \FunctionTok{c}\NormalTok{(}\StringTok{"X"}\NormalTok{, }\StringTok{"Y"}\NormalTok{), }\AttributeTok{crs =} \DecValTok{4326}\NormalTok{)}

\CommentTok{\# Создание сетки}
\NormalTok{grid\_sf }\OtherTok{\textless{}{-}} \FunctionTok{st\_make\_grid}\NormalTok{(points\_sf, }
                        \AttributeTok{cellsize =} \FunctionTok{c}\NormalTok{(xcs, ycs),}
                        \AttributeTok{n =} \FunctionTok{c}\NormalTok{(}\DecValTok{2} \SpecialCharTok{+}\NormalTok{ (xmax }\SpecialCharTok{{-}}\NormalTok{ xmin)}\SpecialCharTok{/}\NormalTok{xcs, }\DecValTok{2} \SpecialCharTok{+}\NormalTok{ (ymax }\SpecialCharTok{{-}}\NormalTok{ ymin)}\SpecialCharTok{/}\NormalTok{ycs),}
                        \AttributeTok{offset =} \FunctionTok{c}\NormalTok{(xmin }\SpecialCharTok{{-}}\NormalTok{ xcs, ymin }\SpecialCharTok{{-}}\NormalTok{ ycs)) }\SpecialCharTok{\%\textgreater{}\%} 
  \FunctionTok{st\_sf}\NormalTok{() }\SpecialCharTok{\%\textgreater{}\%} 
  \FunctionTok{mutate}\NormalTok{(}\AttributeTok{cell\_id =} \FunctionTok{row\_number}\NormalTok{())}

\CommentTok{\# Присоединяем точки Catch к сетке и агрегируем по ячейкам и годам}
\NormalTok{shares\_df\_catch }\OtherTok{\textless{}{-}} \FunctionTok{st\_join}\NormalTok{(points\_sf, grid\_sf) }\SpecialCharTok{\%\textgreater{}\%} 
  \FunctionTok{st\_drop\_geometry}\NormalTok{() }\SpecialCharTok{\%\textgreater{}\%} 
  \FunctionTok{group\_by}\NormalTok{(cell\_id, YEAR) }\SpecialCharTok{\%\textgreater{}\%} 
  \FunctionTok{summarise}\NormalTok{(}
    \AttributeTok{Count =} \FunctionTok{n}\NormalTok{(),}
    \AttributeTok{CATCH =} \FunctionTok{mean}\NormalTok{(CPUE, }\AttributeTok{na.rm =} \ConstantTok{TRUE}\NormalTok{)}
\NormalTok{  ) }\SpecialCharTok{\%\textgreater{}\%} 
  \FunctionTok{ungroup}\NormalTok{()}

\CommentTok{\# Присоединяем статистику Catch к сетке}
\NormalTok{gird\_shares\_catch }\OtherTok{\textless{}{-}} \FunctionTok{right\_join}\NormalTok{(grid\_sf, shares\_df\_catch, }\AttributeTok{by =} \StringTok{"cell\_id"}\NormalTok{)}



\CommentTok{\# Центроиды сетки по W}
\NormalTok{CENTROIDS\_W }\OtherTok{\textless{}{-}}\NormalTok{ gird\_shares\_catch }\SpecialCharTok{\%\textgreater{}\%} 
  \FunctionTok{st\_centroid}\NormalTok{()}

\DocumentationTok{\#\#\#\#\#\#\#\#\#\#\#\#\#\#\#\#\#\#\#\# ВИЗУАЛИЗАЦИЯ \#\#\#\#\#\#\#\#\#\#\#\#\#\#\#\#\#\#\#\#\#\#\#\#\#\#\#\#\#\#\#\#\#\#\#\#\#\#\#\#\#}

\FunctionTok{ggplot}\NormalTok{() }\SpecialCharTok{+}
  \CommentTok{\# 1. Сетка без заливки}
  \FunctionTok{geom\_sf}\NormalTok{(}\AttributeTok{data =}\NormalTok{ grid\_sf, }\AttributeTok{fill =} \ConstantTok{NA}\NormalTok{, }\AttributeTok{color =} \StringTok{"grey80"}\NormalTok{, }\AttributeTok{linewidth =} \FloatTok{0.3}\NormalTok{) }\SpecialCharTok{+}
  
  \CommentTok{\# 2. Границы России}
  \FunctionTok{geom\_sf}\NormalTok{(}\AttributeTok{data =}\NormalTok{ russia, }\AttributeTok{fill =} \StringTok{"grey95"}\NormalTok{) }\SpecialCharTok{+}
  
  \CommentTok{\# 3. Заливка по улову с палитрой viridis option "H"}
  \FunctionTok{geom\_sf}\NormalTok{(}\AttributeTok{data =}\NormalTok{ gird\_shares\_catch, }\FunctionTok{aes}\NormalTok{(}\AttributeTok{fill =}\NormalTok{ CATCH), }\AttributeTok{color =} \ConstantTok{NA}\NormalTok{) }\SpecialCharTok{+}
  
  \CommentTok{\# 4. Цветовая шкала viridis option "H" для заливки}
  \FunctionTok{scale\_fill\_viridis\_c}\NormalTok{(}
    \AttributeTok{option =} \StringTok{"H"}\NormalTok{, }
    \AttributeTok{name =} \StringTok{"CPUE"}\NormalTok{,}
    \AttributeTok{limits =} \FunctionTok{c}\NormalTok{(}\DecValTok{0}\NormalTok{, }\FunctionTok{max}\NormalTok{(gird\_shares\_catch}\SpecialCharTok{$}\NormalTok{CATCH, }\AttributeTok{na.rm =} \ConstantTok{TRUE}\NormalTok{)),}
    \AttributeTok{na.value =} \StringTok{"transparent"}
\NormalTok{  ) }\SpecialCharTok{+}
  
  \CommentTok{\# 5. Обрезаем область отображения}
  \FunctionTok{coord\_sf}\NormalTok{(}
    \AttributeTok{xlim =} \FunctionTok{c}\NormalTok{(xmin, xmax), }
    \AttributeTok{ylim =} \FunctionTok{c}\NormalTok{(ymin, ymax),}
    \AttributeTok{expand =} \ConstantTok{FALSE}
\NormalTok{  ) }\SpecialCharTok{+}
  
  \CommentTok{\# 6. Шкалы для осей координат}
  \FunctionTok{scale\_x\_continuous}\NormalTok{(}
    \AttributeTok{breaks =} \FunctionTok{seq}\NormalTok{(xmin, xmax, }\AttributeTok{by =} \DecValTok{2}\NormalTok{),}
    \AttributeTok{name =} \StringTok{"Долгота"}
\NormalTok{  ) }\SpecialCharTok{+}
  \FunctionTok{scale\_y\_continuous}\NormalTok{(}
    \AttributeTok{breaks =} \FunctionTok{seq}\NormalTok{(ymin, ymax, }\AttributeTok{by =} \FloatTok{0.5}\NormalTok{),}
    \AttributeTok{name =} \StringTok{"Широта"}
\NormalTok{  ) }\SpecialCharTok{+}
  
  \CommentTok{\# 7. Тема оформления}
  \FunctionTok{theme\_minimal}\NormalTok{() }\SpecialCharTok{+}
  \FunctionTok{theme}\NormalTok{(}
    \AttributeTok{panel.grid =} \FunctionTok{element\_blank}\NormalTok{(),}
    \AttributeTok{legend.position =} \StringTok{"bottom"}\NormalTok{,}
    \AttributeTok{panel.border =} \FunctionTok{element\_rect}\NormalTok{(}\AttributeTok{fill =} \ConstantTok{NA}\NormalTok{, }\AttributeTok{color =} \StringTok{"black"}\NormalTok{, }\AttributeTok{size =} \FloatTok{0.5}\NormalTok{),}
    \AttributeTok{panel.grid.major =} \FunctionTok{element\_line}\NormalTok{(}\AttributeTok{color =} \StringTok{"gray90"}\NormalTok{, }\AttributeTok{size =} \FloatTok{0.2}\NormalTok{)}
\NormalTok{  ) }\SpecialCharTok{+}
  \FunctionTok{labs}\NormalTok{(}\AttributeTok{x =} \StringTok{"Долгота"}\NormalTok{, }\AttributeTok{y =} \StringTok{"Широта"}\NormalTok{)}
\end{Highlighting}
\end{Shaded}

\section{Промысловые карты - картограммы по
фасеткам}\label{ux43fux440ux43eux43cux44bux441ux43bux43eux432ux44bux435-ux43aux430ux440ux442ux44b---ux43aux430ux440ux442ux43eux433ux440ux430ux43cux43cux44b-ux43fux43e-ux444ux430ux441ux435ux442ux43aux430ux43c}

\begin{figure}[H]

{\centering \includegraphics[width=0.8\linewidth,height=\textheight,keepaspectratio]{images/KARTOGRAPH11.jpg}

}

\caption{Рис. 11.: Промысловые карты - картограммы по фасеткам}

\end{figure}%

\begin{Shaded}
\begin{Highlighting}[]
\CommentTok{\# Очистка окружения и установка рабочей директории}
\FunctionTok{rm}\NormalTok{(}\AttributeTok{list =} \FunctionTok{ls}\NormalTok{())}
\FunctionTok{setwd}\NormalTok{(}\StringTok{"C:/COURSES/KARTOGRAPH/"}\NormalTok{)}

\FunctionTok{library}\NormalTok{(rnaturalearth)}
\FunctionTok{library}\NormalTok{(tidyverse)}
\FunctionTok{library}\NormalTok{(ggspatial)}
\FunctionTok{library}\NormalTok{(sf)}

\DocumentationTok{\#\#\#\#\#\#\# READ DATA AND PREPARE SPATIAL OBJECTS \#\#\#\#\#\#\#\#\#\#\#\#\#\#\#\#\#\#\#\#\#\#\#\#\#\#\#\#}

\CommentTok{\# Чтение и фильтрация данных}
\NormalTok{DATA }\OtherTok{\textless{}{-}}\NormalTok{ readxl}\SpecialCharTok{::}\FunctionTok{read\_excel}\NormalTok{(}\StringTok{"KARTOGRAPHIC.xlsx"}\NormalTok{, }\AttributeTok{sheet =} \StringTok{"FISHERY"}\NormalTok{) }\SpecialCharTok{\%\textgreater{}\%} 
  \FunctionTok{filter}\NormalTok{(YEAR }\SpecialCharTok{\textgreater{}} \DecValTok{2020} \SpecialCharTok{\&}\NormalTok{ YEAR }\SpecialCharTok{\textless{}} \DecValTok{2025}\NormalTok{)}

\CommentTok{\# Карта России}
\NormalTok{russia }\OtherTok{\textless{}{-}} \FunctionTok{ne\_countries}\NormalTok{(}\AttributeTok{scale =} \DecValTok{10}\NormalTok{, }\AttributeTok{country =} \StringTok{"Russia"}\NormalTok{) }

\CommentTok{\# Параметры карты и сетки}
\NormalTok{xmin }\OtherTok{\textless{}{-}} \DecValTok{32}\NormalTok{; xmax }\OtherTok{\textless{}{-}} \DecValTok{48}\NormalTok{; ymin }\OtherTok{\textless{}{-}} \DecValTok{68}\NormalTok{; ymax }\OtherTok{\textless{}{-}} \DecValTok{72}
\NormalTok{xcs }\OtherTok{\textless{}{-}} \DecValTok{2}\NormalTok{; ycs }\OtherTok{\textless{}{-}} \FloatTok{0.5}

\CommentTok{\# Преобразование в пространственные объекты}
\NormalTok{points\_sf }\OtherTok{\textless{}{-}} \FunctionTok{st\_as\_sf}\NormalTok{(DATA, }\AttributeTok{coords =} \FunctionTok{c}\NormalTok{(}\StringTok{"X"}\NormalTok{, }\StringTok{"Y"}\NormalTok{), }\AttributeTok{crs =} \DecValTok{4326}\NormalTok{)}

\CommentTok{\# Создание сетки}
\NormalTok{grid\_sf }\OtherTok{\textless{}{-}} \FunctionTok{st\_make\_grid}\NormalTok{(}
\NormalTok{  points\_sf,}
  \AttributeTok{cellsize =} \FunctionTok{c}\NormalTok{(xcs, ycs),}
  \AttributeTok{n =} \FunctionTok{c}\NormalTok{(}\DecValTok{2} \SpecialCharTok{+}\NormalTok{ (xmax }\SpecialCharTok{{-}}\NormalTok{ xmin)}\SpecialCharTok{/}\NormalTok{xcs, }\DecValTok{2} \SpecialCharTok{+}\NormalTok{ (ymax }\SpecialCharTok{{-}}\NormalTok{ ymin)}\SpecialCharTok{/}\NormalTok{ycs),}
  \AttributeTok{offset =} \FunctionTok{c}\NormalTok{(xmin }\SpecialCharTok{{-}}\NormalTok{ xcs, ymin }\SpecialCharTok{{-}}\NormalTok{ ycs)}
\NormalTok{) }\SpecialCharTok{\%\textgreater{}\%} 
  \FunctionTok{st\_sf}\NormalTok{() }\SpecialCharTok{\%\textgreater{}\%} 
  \FunctionTok{mutate}\NormalTok{(}\AttributeTok{cell\_id =} \FunctionTok{row\_number}\NormalTok{())}

\CommentTok{\# Агрегация данных по сетке и годам}
\NormalTok{shares\_df\_catch }\OtherTok{\textless{}{-}}\NormalTok{ points\_sf }\SpecialCharTok{\%\textgreater{}\%} 
  \FunctionTok{st\_join}\NormalTok{(grid\_sf) }\SpecialCharTok{\%\textgreater{}\%} 
  \FunctionTok{st\_drop\_geometry}\NormalTok{() }\SpecialCharTok{\%\textgreater{}\%} 
  \FunctionTok{group\_by}\NormalTok{(cell\_id, YEAR) }\SpecialCharTok{\%\textgreater{}\%} 
  \FunctionTok{summarise}\NormalTok{(}\AttributeTok{CATCH =} \FunctionTok{mean}\NormalTok{(CPUE, }\AttributeTok{na.rm =} \ConstantTok{TRUE}\NormalTok{), }\AttributeTok{.groups =} \StringTok{\textquotesingle{}drop\textquotesingle{}}\NormalTok{)}

\CommentTok{\# Присоединение статистики к сетке}
\NormalTok{gird\_shares\_catch }\OtherTok{\textless{}{-}}\NormalTok{ grid\_sf }\SpecialCharTok{\%\textgreater{}\%} 
  \FunctionTok{right\_join}\NormalTok{(shares\_df\_catch, }\AttributeTok{by =} \StringTok{"cell\_id"}\NormalTok{)}

\DocumentationTok{\#\#\#\#\#\#\#\#\#\#\#\#\#\#\#\#\#\#\#\# ВИЗУАЛИЗАЦИЯ \#\#\#\#\#\#\#\#\#\#\#\#\#\#\#\#\#\#\#\#\#\#\#\#\#\#\#\#\#\#\#\#\#\#\#\#\#\#\#\#\#}

\CommentTok{\# Определяем общий максимум CPUE для единой шкалы цветов}
\NormalTok{catch\_max }\OtherTok{\textless{}{-}} \FunctionTok{max}\NormalTok{(gird\_shares\_catch}\SpecialCharTok{$}\NormalTok{CATCH, }\AttributeTok{na.rm =} \ConstantTok{TRUE}\NormalTok{)}

\CommentTok{\# Рассчитываем шаг для подписей (в 2 раза реже исходной сетки)}
\NormalTok{x\_breaks }\OtherTok{\textless{}{-}} \FunctionTok{seq}\NormalTok{(xmin, xmax, }\AttributeTok{by =}\NormalTok{ xcs }\SpecialCharTok{*} \DecValTok{2}\NormalTok{)  }\CommentTok{\# 4 градуса}
\NormalTok{y\_breaks }\OtherTok{\textless{}{-}} \FunctionTok{seq}\NormalTok{(ymin, ymax, }\AttributeTok{by =}\NormalTok{ ycs }\SpecialCharTok{*} \DecValTok{2}\NormalTok{)  }\CommentTok{\# 1 градус}

\CommentTok{\# Функция для форматирования подписей: пропускаем первую подпись}
\NormalTok{format\_labels }\OtherTok{\textless{}{-}} \ControlFlowTok{function}\NormalTok{(breaks) \{}
\NormalTok{  labels }\OtherTok{\textless{}{-}} \FunctionTok{paste0}\NormalTok{(breaks, }\StringTok{"°"}\NormalTok{)}
\NormalTok{  labels[}\DecValTok{1}\NormalTok{] }\OtherTok{\textless{}{-}} \StringTok{""}  \CommentTok{\# Пропускаем первую подпись}
  \FunctionTok{return}\NormalTok{(labels)}
\NormalTok{\}}

\FunctionTok{ggplot}\NormalTok{() }\SpecialCharTok{+}
  \CommentTok{\# Контуры сетки}
  \FunctionTok{geom\_sf}\NormalTok{(}\AttributeTok{data =}\NormalTok{ grid\_sf, }\AttributeTok{fill =} \ConstantTok{NA}\NormalTok{, }\AttributeTok{color =} \StringTok{"grey80"}\NormalTok{, }\AttributeTok{linewidth =} \FloatTok{0.3}\NormalTok{) }\SpecialCharTok{+}
  
  \CommentTok{\# Заливка по улову с цветовой схемой viridis}
  \FunctionTok{geom\_sf}\NormalTok{(}\AttributeTok{data =}\NormalTok{ gird\_shares\_catch, }\FunctionTok{aes}\NormalTok{(}\AttributeTok{fill =}\NormalTok{ CATCH), }\AttributeTok{color =} \ConstantTok{NA}\NormalTok{) }\SpecialCharTok{+}
  
  \CommentTok{\# Границы России}
  \FunctionTok{geom\_sf}\NormalTok{(}\AttributeTok{data =}\NormalTok{ russia, }\AttributeTok{fill =} \StringTok{"\#E8E5D6"}\NormalTok{) }\SpecialCharTok{+}
  
  \CommentTok{\# Фасетирование по годам}
  \FunctionTok{facet\_wrap}\NormalTok{(}\SpecialCharTok{\textasciitilde{}}\NormalTok{ YEAR, }\AttributeTok{nrow =} \DecValTok{2}\NormalTok{) }\SpecialCharTok{+}
  
  \CommentTok{\# Цветовая шкала}
  \FunctionTok{scale\_fill\_viridis\_c}\NormalTok{(}
    \AttributeTok{option =} \StringTok{"H"}\NormalTok{, }
    \AttributeTok{name =} \StringTok{"CPUE"}\NormalTok{,}
    \AttributeTok{limits =} \FunctionTok{c}\NormalTok{(}\DecValTok{0}\NormalTok{, catch\_max),}
    \AttributeTok{na.value =} \StringTok{"transparent"}
\NormalTok{  ) }\SpecialCharTok{+}
  
  \CommentTok{\# Область отображения}
  \FunctionTok{coord\_sf}\NormalTok{(}
    \AttributeTok{xlim =} \FunctionTok{c}\NormalTok{(xmin, xmax), }
    \AttributeTok{ylim =} \FunctionTok{c}\NormalTok{(ymin, ymax),}
    \AttributeTok{expand =} \ConstantTok{FALSE}
\NormalTok{  ) }\SpecialCharTok{+}
  
  \CommentTok{\# Управление подписями осей с символом градуса (пропускаем первую подпись)}
  \FunctionTok{scale\_x\_continuous}\NormalTok{(}
    \AttributeTok{breaks =}\NormalTok{ x\_breaks,}
    \AttributeTok{labels =}\NormalTok{ format\_labels}
\NormalTok{  ) }\SpecialCharTok{+}
  \FunctionTok{scale\_y\_continuous}\NormalTok{(}
    \AttributeTok{breaks =}\NormalTok{ y\_breaks,}
    \AttributeTok{labels =}\NormalTok{ format\_labels}
\NormalTok{  ) }\SpecialCharTok{+}
  
  \CommentTok{\# Оформление с тиками на осях}
  \FunctionTok{theme\_minimal}\NormalTok{() }\SpecialCharTok{+}
  \FunctionTok{theme}\NormalTok{(}
    \AttributeTok{panel.grid =} \FunctionTok{element\_blank}\NormalTok{(),}
    \AttributeTok{legend.position =} \StringTok{"bottom"}\NormalTok{,}
    \AttributeTok{legend.key.width =} \FunctionTok{unit}\NormalTok{(}\FloatTok{2.5}\NormalTok{, }\StringTok{"cm"}\NormalTok{),}
    \AttributeTok{legend.title =} \FunctionTok{element\_text}\NormalTok{(}\AttributeTok{vjust =} \FloatTok{0.8}\NormalTok{, }\AttributeTok{size =} \DecValTok{12}\NormalTok{),}
    \AttributeTok{panel.border =} \FunctionTok{element\_rect}\NormalTok{(}\AttributeTok{fill =} \ConstantTok{NA}\NormalTok{, }\AttributeTok{color =} \StringTok{"black"}\NormalTok{, }\AttributeTok{size =} \FloatTok{0.7}\NormalTok{),}
    \AttributeTok{panel.grid.major =} \FunctionTok{element\_line}\NormalTok{(}\AttributeTok{color =} \StringTok{"grey90"}\NormalTok{, }\AttributeTok{size =} \FloatTok{0.2}\NormalTok{),}
    \AttributeTok{strip.background =} \FunctionTok{element\_rect}\NormalTok{(}\AttributeTok{fill =} \StringTok{"\#E8E5D6"}\NormalTok{, }\AttributeTok{color =} \StringTok{"black"}\NormalTok{),}
    \AttributeTok{strip.text =} \FunctionTok{element\_text}\NormalTok{(}\AttributeTok{face =} \StringTok{"bold"}\NormalTok{, }\AttributeTok{size =} \DecValTok{12}\NormalTok{),}
    \AttributeTok{axis.text.x =} \FunctionTok{element\_text}\NormalTok{(}\AttributeTok{size =} \DecValTok{9}\NormalTok{, }\AttributeTok{angle =} \DecValTok{0}\NormalTok{, }\AttributeTok{margin =} \FunctionTok{margin}\NormalTok{(}\AttributeTok{t =} \DecValTok{5}\NormalTok{)),}
    \AttributeTok{axis.text.y =} \FunctionTok{element\_text}\NormalTok{(}\AttributeTok{size =} \DecValTok{9}\NormalTok{, }\AttributeTok{angle =} \DecValTok{90}\NormalTok{, }\AttributeTok{hjust =} \FloatTok{0.5}\NormalTok{, }\AttributeTok{margin =} \FunctionTok{margin}\NormalTok{(}\AttributeTok{r =} \DecValTok{5}\NormalTok{)),}
    \AttributeTok{axis.title.x =} \FunctionTok{element\_blank}\NormalTok{(),}
    \AttributeTok{axis.title.y =} \FunctionTok{element\_blank}\NormalTok{(),}
    
    \CommentTok{\# Тики (засечки) на оси}
    \AttributeTok{axis.ticks =} \FunctionTok{element\_line}\NormalTok{(}\AttributeTok{color =} \StringTok{"black"}\NormalTok{, }\AttributeTok{size =} \FloatTok{0.5}\NormalTok{),}
    \AttributeTok{axis.ticks.length =} \FunctionTok{unit}\NormalTok{(}\FloatTok{0.2}\NormalTok{, }\StringTok{"cm"}\NormalTok{),}
    \AttributeTok{axis.ticks.x =} \FunctionTok{element\_line}\NormalTok{(}\AttributeTok{color =} \StringTok{"black"}\NormalTok{, }\AttributeTok{size =} \FloatTok{0.5}\NormalTok{),}
    \AttributeTok{axis.ticks.y =} \FunctionTok{element\_line}\NormalTok{(}\AttributeTok{color =} \StringTok{"black"}\NormalTok{, }\AttributeTok{size =} \FloatTok{0.5}\NormalTok{)}
\NormalTok{  ) }\SpecialCharTok{+}
  
  \CommentTok{\# Настройка легенды}
  \FunctionTok{guides}\NormalTok{(}\AttributeTok{fill =} \FunctionTok{guide\_colorbar}\NormalTok{(}
    \AttributeTok{title.position =} \StringTok{"top"}\NormalTok{,}
    \AttributeTok{title.hjust =} \FloatTok{0.5}\NormalTok{,}
    \AttributeTok{barwidth =} \DecValTok{15}\NormalTok{,}
    \AttributeTok{frame.colour =} \StringTok{"black"}\NormalTok{,}
    \AttributeTok{ticks.colour =} \StringTok{"black"}
\NormalTok{  ))}

\CommentTok{\# Сохранение результата}
\FunctionTok{ggsave}\NormalTok{(}\StringTok{"KARTOGRAPH11.jpg"}\NormalTok{, }
       \AttributeTok{device =} \StringTok{"jpeg"}\NormalTok{, }
       \AttributeTok{dpi =} \DecValTok{300}\NormalTok{,}
       \AttributeTok{width =} \DecValTok{7}\NormalTok{,}
       \AttributeTok{height =} \DecValTok{5}\NormalTok{,}
       \AttributeTok{units =} \StringTok{"in"}\NormalTok{)}
\end{Highlighting}
\end{Shaded}

\section{Гибридные карты - картограммы и точки (съемка и промысловые
данные)}\label{ux433ux438ux431ux440ux438ux434ux43dux44bux435-ux43aux430ux440ux442ux44b---ux43aux430ux440ux442ux43eux433ux440ux430ux43cux43cux44b-ux438-ux442ux43eux447ux43aux438-ux441ux44aux435ux43cux43aux430-ux438-ux43fux440ux43eux43cux44bux441ux43bux43eux432ux44bux435-ux434ux430ux43dux43dux44bux435}

\begin{figure}[H]

{\centering \includegraphics[width=0.8\linewidth,height=\textheight,keepaspectratio]{images/KARTOGRAPH12.jpg}

}

\caption{Рис. 12.: Гибридные карты - картограммы и точки (съемка и
промысловые данные)}

\end{figure}%

\begin{Shaded}
\begin{Highlighting}[]
\CommentTok{\# Очистка окружения и установка рабочей директории}
\FunctionTok{rm}\NormalTok{(}\AttributeTok{list =} \FunctionTok{ls}\NormalTok{())}
\FunctionTok{setwd}\NormalTok{(}\StringTok{"C:/COURSES/KARTOGRAPH/"}\NormalTok{)}

\FunctionTok{library}\NormalTok{(rnaturalearth)}
\FunctionTok{library}\NormalTok{(tidyverse)}
\FunctionTok{library}\NormalTok{(ggspatial)}
\FunctionTok{library}\NormalTok{(sf)}

\DocumentationTok{\#\#\#\#\#\#\# READ DATA AND PREPARE SPATIAL OBJECTS \#\#\#\#\#\#\#\#\#\#\#\#\#\#\#\#\#\#\#\#\#\#\#\#\#\#\#\#}

\CommentTok{\# Чтение и фильтрация данных}
\NormalTok{DATA }\OtherTok{\textless{}{-}}\NormalTok{ readxl}\SpecialCharTok{::}\FunctionTok{read\_excel}\NormalTok{(}\StringTok{"KARTOGRAPHIC.xlsx"}\NormalTok{, }\AttributeTok{sheet =} \StringTok{"FISHERY"}\NormalTok{) }\SpecialCharTok{\%\textgreater{}\%} 
  \FunctionTok{filter}\NormalTok{(YEAR }\SpecialCharTok{\textgreater{}} \DecValTok{2020} \SpecialCharTok{\&}\NormalTok{ YEAR }\SpecialCharTok{\textless{}} \DecValTok{2025}\NormalTok{)}

\NormalTok{SURVEY }\OtherTok{\textless{}{-}}\NormalTok{ readxl}\SpecialCharTok{::}\FunctionTok{read\_excel}\NormalTok{(}\StringTok{"KARTOGRAPHIC.xlsx"}\NormalTok{, }\AttributeTok{sheet =} \StringTok{"SURVEY"}\NormalTok{) }\SpecialCharTok{\%\textgreater{}\%} 
  \FunctionTok{filter}\NormalTok{(YEAR }\SpecialCharTok{\textgreater{}} \DecValTok{2020} \SpecialCharTok{\&}\NormalTok{ YEAR }\SpecialCharTok{\textless{}} \DecValTok{2025}\NormalTok{, SURV }\SpecialCharTok{==} \StringTok{"CRAB"}\NormalTok{)}

\CommentTok{\# Создание 4 категорий для переменной PROM}
\NormalTok{breaks }\OtherTok{\textless{}{-}} \FunctionTok{quantile}\NormalTok{(SURVEY}\SpecialCharTok{$}\NormalTok{PROM, }
                   \AttributeTok{probs =} \FunctionTok{c}\NormalTok{(}\DecValTok{0}\NormalTok{, }\FloatTok{0.25}\NormalTok{, }\FloatTok{0.5}\NormalTok{, }\FloatTok{0.75}\NormalTok{, }\DecValTok{1}\NormalTok{), }
                   \AttributeTok{na.rm =} \ConstantTok{TRUE}\NormalTok{)}
\NormalTok{SURVEY}\SpecialCharTok{$}\NormalTok{PROM\_cat }\OtherTok{\textless{}{-}} \FunctionTok{cut}\NormalTok{(SURVEY}\SpecialCharTok{$}\NormalTok{PROM,}
                       \AttributeTok{breaks =}\NormalTok{ breaks,}
                       \AttributeTok{include.lowest =} \ConstantTok{TRUE}\NormalTok{,}
                       \AttributeTok{labels =} \FunctionTok{c}\NormalTok{(}\StringTok{"Q1 (Low)"}\NormalTok{, }\StringTok{"Q2"}\NormalTok{, }\StringTok{"Q3"}\NormalTok{, }\StringTok{"Q4 (High)"}\NormalTok{))}

\CommentTok{\# Карта России}
\NormalTok{russia }\OtherTok{\textless{}{-}} \FunctionTok{ne\_countries}\NormalTok{(}\AttributeTok{scale =} \DecValTok{10}\NormalTok{, }\AttributeTok{country =} \StringTok{"Russia"}\NormalTok{) }

\CommentTok{\# Параметры карты и сетки}
\NormalTok{xmin }\OtherTok{\textless{}{-}} \DecValTok{32}\NormalTok{; xmax }\OtherTok{\textless{}{-}} \DecValTok{48}\NormalTok{; ymin }\OtherTok{\textless{}{-}} \DecValTok{68}\NormalTok{; ymax }\OtherTok{\textless{}{-}} \DecValTok{72}
\NormalTok{xcs }\OtherTok{\textless{}{-}} \DecValTok{2}\NormalTok{; ycs }\OtherTok{\textless{}{-}} \FloatTok{0.5}

\CommentTok{\# Преобразование в пространственные объекты}
\NormalTok{points\_sf }\OtherTok{\textless{}{-}} \FunctionTok{st\_as\_sf}\NormalTok{(DATA, }\AttributeTok{coords =} \FunctionTok{c}\NormalTok{(}\StringTok{"X"}\NormalTok{, }\StringTok{"Y"}\NormalTok{), }\AttributeTok{crs =} \DecValTok{4326}\NormalTok{)}
\NormalTok{survey\_sf }\OtherTok{\textless{}{-}} \FunctionTok{st\_as\_sf}\NormalTok{(SURVEY, }\AttributeTok{coords =} \FunctionTok{c}\NormalTok{(}\StringTok{"X"}\NormalTok{, }\StringTok{"Y"}\NormalTok{), }\AttributeTok{crs =} \DecValTok{4326}\NormalTok{)}

\CommentTok{\# Создание сетки}
\NormalTok{grid\_sf }\OtherTok{\textless{}{-}} \FunctionTok{st\_make\_grid}\NormalTok{(}
\NormalTok{  points\_sf,}
  \AttributeTok{cellsize =} \FunctionTok{c}\NormalTok{(xcs, ycs),}
  \AttributeTok{n =} \FunctionTok{c}\NormalTok{(}\DecValTok{2} \SpecialCharTok{+}\NormalTok{ (xmax }\SpecialCharTok{{-}}\NormalTok{ xmin)}\SpecialCharTok{/}\NormalTok{xcs, }\DecValTok{2} \SpecialCharTok{+}\NormalTok{ (ymax }\SpecialCharTok{{-}}\NormalTok{ ymin)}\SpecialCharTok{/}\NormalTok{ycs),}
  \AttributeTok{offset =} \FunctionTok{c}\NormalTok{(xmin }\SpecialCharTok{{-}}\NormalTok{ xcs, ymin }\SpecialCharTok{{-}}\NormalTok{ ycs)}
\NormalTok{) }\SpecialCharTok{\%\textgreater{}\%} 
  \FunctionTok{st\_sf}\NormalTok{() }\SpecialCharTok{\%\textgreater{}\%} 
  \FunctionTok{mutate}\NormalTok{(}\AttributeTok{cell\_id =} \FunctionTok{row\_number}\NormalTok{())}

\CommentTok{\# Агрегация данных по сетке и годам}
\NormalTok{shares\_df\_catch }\OtherTok{\textless{}{-}}\NormalTok{ points\_sf }\SpecialCharTok{\%\textgreater{}\%} 
  \FunctionTok{st\_join}\NormalTok{(grid\_sf) }\SpecialCharTok{\%\textgreater{}\%} 
  \FunctionTok{st\_drop\_geometry}\NormalTok{() }\SpecialCharTok{\%\textgreater{}\%} 
  \FunctionTok{group\_by}\NormalTok{(cell\_id, YEAR) }\SpecialCharTok{\%\textgreater{}\%} 
  \FunctionTok{summarise}\NormalTok{(}\AttributeTok{CATCH =} \FunctionTok{mean}\NormalTok{(CPUE, }\AttributeTok{na.rm =} \ConstantTok{TRUE}\NormalTok{), }\AttributeTok{.groups =} \StringTok{\textquotesingle{}drop\textquotesingle{}}\NormalTok{)}

\CommentTok{\# Присоединение статистики к сетке}
\NormalTok{gird\_shares\_catch }\OtherTok{\textless{}{-}}\NormalTok{ grid\_sf }\SpecialCharTok{\%\textgreater{}\%} 
  \FunctionTok{right\_join}\NormalTok{(shares\_df\_catch, }\AttributeTok{by =} \StringTok{"cell\_id"}\NormalTok{)}

\DocumentationTok{\#\#\#\#\#\#\#\#\#\#\#\#\#\#\#\#\#\#\# ВИЗУАЛИЗАЦИЯ \#\#\#\#\#\#\#\#\#\#\#\#\#\#\#\#\#\#\#\#\#\#\#\#\#\#\#\#\#\#\#\#\#\#\#\#\#\#\#\#\#}
\FunctionTok{ggplot}\NormalTok{() }\SpecialCharTok{+}
  \CommentTok{\# Контуры сетки}
  \FunctionTok{geom\_sf}\NormalTok{(}\AttributeTok{data =}\NormalTok{ grid\_sf, }\AttributeTok{fill =} \ConstantTok{NA}\NormalTok{, }\AttributeTok{color =} \StringTok{"grey80"}\NormalTok{, }\AttributeTok{linewidth =} \FloatTok{0.3}\NormalTok{) }\SpecialCharTok{+}
  
  \CommentTok{\# Заливка по улову (средний CPUE)}
  \FunctionTok{geom\_sf}\NormalTok{(}\AttributeTok{data =}\NormalTok{ gird\_shares\_catch, }\FunctionTok{aes}\NormalTok{(}\AttributeTok{fill =}\NormalTok{ CATCH)) }\SpecialCharTok{+}
  
  \CommentTok{\# Границы России}
  \FunctionTok{geom\_sf}\NormalTok{(}\AttributeTok{data =}\NormalTok{ russia, }\AttributeTok{fill =} \StringTok{"\#E8E5D6"}\NormalTok{) }\SpecialCharTok{+}
  
  \CommentTok{\# Точки SURVEY: голубые с черной окантовкой}
  \FunctionTok{geom\_sf}\NormalTok{(}\AttributeTok{data =}\NormalTok{ survey\_sf, }
          \FunctionTok{aes}\NormalTok{(}\AttributeTok{size =}\NormalTok{ PROM\_cat),}
          \AttributeTok{fill =} \StringTok{"lightblue"}\NormalTok{,    }\CommentTok{\# Голубая заливка}
          \AttributeTok{color =} \StringTok{"black"}\NormalTok{,        }\CommentTok{\# Черная окантовка}
          \AttributeTok{alpha =} \FloatTok{0.7}\NormalTok{,}
          \AttributeTok{shape =} \DecValTok{21}\NormalTok{,             }\CommentTok{\# Круг с обводкой}
          \AttributeTok{stroke =} \FloatTok{0.5}\NormalTok{,           }\CommentTok{\# Толщина окантовки}
          \AttributeTok{show.legend =} \StringTok{"point"}\NormalTok{) }\SpecialCharTok{+}
  
  \CommentTok{\# Фасетирование по годам}
  \FunctionTok{facet\_wrap}\NormalTok{(}\SpecialCharTok{\textasciitilde{}}\NormalTok{ YEAR, }\AttributeTok{nrow =} \DecValTok{2}\NormalTok{) }\SpecialCharTok{+}
  
  \CommentTok{\# Цветовая шкала для заливки}
  \FunctionTok{scale\_fill\_gradient}\NormalTok{(}
    \AttributeTok{low =} \StringTok{"white"}\NormalTok{, }
    \AttributeTok{high =} \StringTok{"red"}\NormalTok{,}
    \AttributeTok{na.value =} \ConstantTok{NA}\NormalTok{,}
    \AttributeTok{limits =} \FunctionTok{c}\NormalTok{(}\DecValTok{0}\NormalTok{, }\FunctionTok{max}\NormalTok{(gird\_shares\_catch}\SpecialCharTok{$}\NormalTok{CATCH, }\AttributeTok{na.rm =} \ConstantTok{TRUE}\NormalTok{)),}
    \AttributeTok{name =} \StringTok{"Catch (CPUE)"}
\NormalTok{  ) }\SpecialCharTok{+}
  
  \CommentTok{\# Шкала размеров для точек}
  \FunctionTok{scale\_size\_manual}\NormalTok{(}
    \AttributeTok{name =} \StringTok{"PROM Category"}\NormalTok{,}
    \AttributeTok{values =} \FunctionTok{c}\NormalTok{(}\FloatTok{1.5}\NormalTok{, }\FloatTok{2.5}\NormalTok{, }\FloatTok{3.5}\NormalTok{, }\FloatTok{4.5}\NormalTok{)  }\CommentTok{\# Размеры точек для 4 категорий}
\NormalTok{  ) }\SpecialCharTok{+}
  
  \DocumentationTok{\#\#\# ОСИ С ГЕОГРАФИЧЕСКИМИ КООРДИНАТАМИ }\AlertTok{\#\#\#}
  \FunctionTok{scale\_x\_continuous}\NormalTok{(}
    \AttributeTok{breaks =} \FunctionTok{c}\NormalTok{(}\DecValTok{32}\NormalTok{, }\DecValTok{38}\NormalTok{, }\DecValTok{44}\NormalTok{, }\DecValTok{48}\NormalTok{),                    }
    \AttributeTok{labels =} \FunctionTok{c}\NormalTok{(}\StringTok{"32°E"}\NormalTok{, }\StringTok{"38°E"}\NormalTok{, }\StringTok{"44°E"}\NormalTok{, }\StringTok{"48°E"}\NormalTok{),    }
    \AttributeTok{name =} \ConstantTok{NULL}
\NormalTok{  ) }\SpecialCharTok{+}
  \FunctionTok{scale\_y\_continuous}\NormalTok{(}
    \AttributeTok{breaks =} \FunctionTok{c}\NormalTok{(}\FloatTok{68.5}\NormalTok{, }\FloatTok{69.5}\NormalTok{, }\FloatTok{70.5}\NormalTok{, }\FloatTok{71.5}\NormalTok{),          }
    \AttributeTok{labels =} \FunctionTok{c}\NormalTok{(}\StringTok{"68.5°N"}\NormalTok{, }\StringTok{"69.5°N"}\NormalTok{, }\StringTok{"70.5°N"}\NormalTok{, }\StringTok{"71.5°N"}\NormalTok{),}
    \AttributeTok{name =} \ConstantTok{NULL}
\NormalTok{  ) }\SpecialCharTok{+}
  
  \CommentTok{\# Область отображения}
  \FunctionTok{coord\_sf}\NormalTok{(}\AttributeTok{xlim =} \FunctionTok{c}\NormalTok{(xmin, xmax), }\AttributeTok{ylim =} \FunctionTok{c}\NormalTok{(ymin, ymax)) }\SpecialCharTok{+}
  
  \CommentTok{\# Оформление}
  \FunctionTok{theme\_minimal}\NormalTok{() }\SpecialCharTok{+}
  \FunctionTok{theme}\NormalTok{(}
    \AttributeTok{axis.text.x =} \FunctionTok{element\_text}\NormalTok{(}\AttributeTok{size =} \DecValTok{8}\NormalTok{),}
    \AttributeTok{axis.text.y =} \FunctionTok{element\_text}\NormalTok{(}\AttributeTok{size =} \DecValTok{8}\NormalTok{, }\AttributeTok{angle =} \DecValTok{90}\NormalTok{, }\AttributeTok{hjust =} \FloatTok{0.5}\NormalTok{),}
    \AttributeTok{panel.grid =} \FunctionTok{element\_line}\NormalTok{(}\AttributeTok{color =} \StringTok{"grey90"}\NormalTok{),}
    \AttributeTok{legend.position =} \StringTok{"bottom"}\NormalTok{,}
    \AttributeTok{legend.box =} \StringTok{"horizontal"}\NormalTok{,  }\CommentTok{\# Размещение легенд в одну строку}
    \AttributeTok{panel.border =} \FunctionTok{element\_rect}\NormalTok{(}\AttributeTok{fill =} \ConstantTok{NA}\NormalTok{, }\AttributeTok{color =} \StringTok{"black"}\NormalTok{, }\AttributeTok{size =} \FloatTok{0.5}\NormalTok{),}
    \AttributeTok{strip.background =} \FunctionTok{element\_rect}\NormalTok{(}\AttributeTok{fill =} \StringTok{"white"}\NormalTok{, }\AttributeTok{color =} \StringTok{"black"}\NormalTok{)}
\NormalTok{  ) }\SpecialCharTok{+}
  \CommentTok{\# Управление легендами}
  \FunctionTok{guides}\NormalTok{(}
    \AttributeTok{fill =} \FunctionTok{guide\_colorbar}\NormalTok{(}\AttributeTok{title.position =} \StringTok{"top"}\NormalTok{, }\AttributeTok{title.hjust =} \FloatTok{0.5}\NormalTok{),}
    \AttributeTok{size =} \FunctionTok{guide\_legend}\NormalTok{(}\AttributeTok{title.position =} \StringTok{"top"}\NormalTok{, }\AttributeTok{title.hjust =} \FloatTok{0.5}\NormalTok{)}
\NormalTok{  )}
\end{Highlighting}
\end{Shaded}

\section{Карты для ``главы Материал и
методы''}\label{ux43aux430ux440ux442ux44b-ux434ux43bux44f-ux433ux43bux430ux432ux44b-ux43cux430ux442ux435ux440ux438ux430ux43b-ux438-ux43cux435ux442ux43eux434ux44b}

\begin{figure}[H]

{\centering \includegraphics[width=0.8\linewidth,height=\textheight,keepaspectratio]{images/KARTOGRAPH13.jpg}

}

\caption{Рис. 13.: Карты для ``главы Материал и методы''}

\end{figure}%

\begin{Shaded}
\begin{Highlighting}[]
\CommentTok{\# Очистка окружения и установка рабочей директории}
\FunctionTok{rm}\NormalTok{(}\AttributeTok{list =} \FunctionTok{ls}\NormalTok{()) }\CommentTok{\# Удаление всех объектов из глобального окружения}
\FunctionTok{setwd}\NormalTok{(}\StringTok{"C:/COURSES/KARTOGRAPH/"}\NormalTok{) }\CommentTok{\# Установка рабочей директории}

\CommentTok{\# {-}{-}{-}{-}{-}{-}{-}{-}{-}{-}{-}{-}{-}{-}{-}{-}{-}}
\CommentTok{\# ЗАГРУЗКА ПАКЕТОВ}
\CommentTok{\# {-}{-}{-}{-}{-}{-}{-}{-}{-}{-}{-}{-}{-}{-}{-}{-}{-}}
\FunctionTok{library}\NormalTok{(sf)          }\CommentTok{\# Пространственные операции с векторными данными}
\FunctionTok{library}\NormalTok{(marmap)      }\CommentTok{\# Работа с батиметрическими данными (карты глубин)}
\FunctionTok{library}\NormalTok{(tidyverse)   }\CommentTok{\# Коллекция пакетов для обработки данных (dplyr, ggplot2 и др.)}
\FunctionTok{library}\NormalTok{(rnaturalearth) }\CommentTok{\# Векторные картографические данные (границы, береговые линии)}
\FunctionTok{library}\NormalTok{(ggspatial)   }\CommentTok{\# Инструменты для пространственной визуализации в ggplot}
\FunctionTok{library}\NormalTok{(readxl)      }\CommentTok{\# Импорт данных из Excel{-}файлов}

\CommentTok{\# {-}{-}{-}{-}{-}{-}{-}{-}{-}{-}{-}{-}{-}{-}{-}{-}{-}}
\CommentTok{\# ЗАГРУЗКА ДАННЫХ}
\CommentTok{\# {-}{-}{-}{-}{-}{-}{-}{-}{-}{-}{-}{-}{-}{-}{-}{-}{-}}
\CommentTok{\# Чтение данных из Excel{-}листа}
\NormalTok{DATA }\OtherTok{\textless{}{-}}\NormalTok{ readxl}\SpecialCharTok{::}\FunctionTok{read\_excel}\NormalTok{(}\StringTok{"KARTOGRAPHIC.xlsx"}\NormalTok{, }\AttributeTok{sheet =} \StringTok{"SURVEY"}\NormalTok{)}

\CommentTok{\# Фильтрация данных:}
\NormalTok{SUMMER }\OtherTok{\textless{}{-}}\NormalTok{ DATA[DATA}\SpecialCharTok{$}\NormalTok{SURV }\SpecialCharTok{==} \StringTok{"SUM"} \SpecialCharTok{\&}\NormalTok{ DATA}\SpecialCharTok{$}\NormalTok{YEAR }\SpecialCharTok{==} \DecValTok{2024}\NormalTok{, ] }\CommentTok{\# Летние исследования 2024}
\NormalTok{CRAB }\OtherTok{\textless{}{-}}\NormalTok{ DATA[DATA}\SpecialCharTok{$}\NormalTok{SURV }\SpecialCharTok{==} \StringTok{"CRAB"} \SpecialCharTok{\&}\NormalTok{ DATA}\SpecialCharTok{$}\NormalTok{YEAR }\SpecialCharTok{==} \DecValTok{2024}\NormalTok{, ]   }\CommentTok{\# Крабовые исследования 2024}

\CommentTok{\# {-}{-}{-}{-}{-}{-}{-}{-}{-}{-}{-}{-}{-}{-}{-}{-}{-}}
\CommentTok{\# ПОДГОТОВКА КАРТОГРАФИЧЕСКОЙ ОСНОВЫ}
\CommentTok{\# {-}{-}{-}{-}{-}{-}{-}{-}{-}{-}{-}{-}{-}{-}{-}{-}{-}}
\CommentTok{\# Загрузка векторных границ России}
\NormalTok{russia\_map }\OtherTok{\textless{}{-}} \FunctionTok{ne\_states}\NormalTok{(}\AttributeTok{country =} \StringTok{"russia"}\NormalTok{, }\AttributeTok{returnclass =} \StringTok{"sf"}\NormalTok{)}

\CommentTok{\# Загрузка береговой линии мирового океана}
\NormalTok{coast }\OtherTok{\textless{}{-}} \FunctionTok{ne\_coastline}\NormalTok{(}\AttributeTok{scale =} \DecValTok{10}\NormalTok{, }\AttributeTok{returnclass =} \StringTok{"sf"}\NormalTok{)}

\CommentTok{\# Создание сетки для навигации (5° по долготе, 1° по широте)}
\NormalTok{ga\_grid }\OtherTok{\textless{}{-}}\NormalTok{ russia\_map }\SpecialCharTok{\%\textgreater{}\%} 
  \FunctionTok{st\_make\_grid}\NormalTok{(}\AttributeTok{cellsize =} \FunctionTok{c}\NormalTok{(}\DecValTok{5}\NormalTok{, }\DecValTok{1}\NormalTok{), }\AttributeTok{offset =} \FunctionTok{c}\NormalTok{(}\DecValTok{30}\NormalTok{, }\DecValTok{67}\NormalTok{))}

\CommentTok{\# Установка границ региона интереса}
\NormalTok{xmin }\OtherTok{\textless{}{-}} \DecValTok{30}\NormalTok{; xmax }\OtherTok{\textless{}{-}} \DecValTok{58}
\NormalTok{ymin }\OtherTok{\textless{}{-}} \DecValTok{67}\NormalTok{; ymax }\OtherTok{\textless{}{-}} \FloatTok{72.5}

\CommentTok{\# {-}{-}{-}{-}{-}{-}{-}{-}{-}{-}{-}{-}{-}{-}{-}{-}{-}}
\CommentTok{\# БАТИМЕТРИЧЕСКИЕ ДАННЫЕ}
\CommentTok{\# {-}{-}{-}{-}{-}{-}{-}{-}{-}{-}{-}{-}{-}{-}{-}{-}{-}}
\CommentTok{\# Загрузка данных о глубинах из базы NOAA}
\NormalTok{bat }\OtherTok{\textless{}{-}} \FunctionTok{getNOAA.bathy}\NormalTok{(}
  \AttributeTok{lon1 =}\NormalTok{ xmin, }\AttributeTok{lon2 =}\NormalTok{ xmax,}
  \AttributeTok{lat1 =}\NormalTok{ ymin, }\AttributeTok{lat2 =}\NormalTok{ ymax,}
  \AttributeTok{resolution =} \DecValTok{1}\NormalTok{,   }\CommentTok{\# Разрешение данных (1 минута дуги)}
  \AttributeTok{keep =} \ConstantTok{TRUE}       \CommentTok{\# Сохранить кэш на диске}
\NormalTok{)}

\CommentTok{\# Преобразование в таблицу XYZ (долгота, широта, глубина)}
\NormalTok{bat\_xyz }\OtherTok{\textless{}{-}} \FunctionTok{as.xyz}\NormalTok{(bat)}

\CommentTok{\# Создание цветовой схемы для глубин:}
\NormalTok{breaks }\OtherTok{\textless{}{-}} \FunctionTok{c}\NormalTok{(}\SpecialCharTok{{-}}\DecValTok{10000}\NormalTok{, }\SpecialCharTok{{-}}\DecValTok{7000}\NormalTok{, }\SpecialCharTok{{-}}\DecValTok{6000}\NormalTok{, }\SpecialCharTok{{-}}\DecValTok{5000}\NormalTok{, }\SpecialCharTok{{-}}\DecValTok{4000}\NormalTok{, }\SpecialCharTok{{-}}\DecValTok{3000}\NormalTok{, }\SpecialCharTok{{-}}\DecValTok{2000}\NormalTok{, }\SpecialCharTok{{-}}\DecValTok{1000}\NormalTok{, }
            \SpecialCharTok{{-}}\DecValTok{500}\NormalTok{, }\SpecialCharTok{{-}}\DecValTok{200}\NormalTok{, }\SpecialCharTok{{-}}\DecValTok{50}\NormalTok{, }\SpecialCharTok{{-}}\DecValTok{1}\NormalTok{, }\DecValTok{5}\NormalTok{, }\DecValTok{50}\NormalTok{, }\DecValTok{100}\NormalTok{, }\DecValTok{150}\NormalTok{, }\DecValTok{200}\NormalTok{, }\DecValTok{300}\NormalTok{, }\DecValTok{400}\NormalTok{, }\DecValTok{500}\NormalTok{, }\DecValTok{1000}\NormalTok{, }\DecValTok{3000}\NormalTok{)}
\NormalTok{cols }\OtherTok{\textless{}{-}} \FunctionTok{c}\NormalTok{(}
  \StringTok{"\#5e99d6"}\NormalTok{, }\StringTok{"\#669cd4"}\NormalTok{, }\StringTok{"\#6c9fd4"}\NormalTok{, }\StringTok{"\#96bce3"}\NormalTok{, }\StringTok{"\#AEC8E3"}\NormalTok{, }\StringTok{"\#a6c4e3"}\NormalTok{,}
  \StringTok{"\#AEC8E3"}\NormalTok{, }\StringTok{"\#BBD0EB"}\NormalTok{, }\StringTok{"\#C7DCF1"}\NormalTok{, }\StringTok{"\#DAECFA"}\NormalTok{, }\StringTok{"\#D2E5F6"}\NormalTok{, }\StringTok{"\#e1f2d8"}\NormalTok{,}
  \StringTok{"\#B8D3AA"}\NormalTok{, }\StringTok{"\#b3b387"}\NormalTok{, }\StringTok{"\#9EC187"}\NormalTok{, }\StringTok{"\#C7D097"}\NormalTok{, }\StringTok{"\#DADBAF"}\NormalTok{, }\StringTok{"\#F3F0C7"}\NormalTok{,}
  \StringTok{"\#E6DBA8"}\NormalTok{, }\StringTok{"\#DACFA1"}\NormalTok{, }\StringTok{"\#D1BF81"}\NormalTok{, }\StringTok{"\#C69D45"}
\NormalTok{)}

\CommentTok{\# Категоризация глубин для визуализации}
\NormalTok{bat\_xyz}\SpecialCharTok{$}\NormalTok{V4 }\OtherTok{\textless{}{-}} \FunctionTok{cut}\NormalTok{(bat\_xyz}\SpecialCharTok{$}\NormalTok{V3, }\AttributeTok{breaks =}\NormalTok{ breaks)}
\NormalTok{niveles }\OtherTok{\textless{}{-}} \FunctionTok{levels}\NormalTok{(bat\_xyz}\SpecialCharTok{$}\NormalTok{V4)  }\CommentTok{\# Сохранение уровней для легенды}

\CommentTok{\# {-}{-}{-}{-}{-}{-}{-}{-}{-}{-}{-}{-}{-}{-}{-}{-}{-}}
\CommentTok{\# ПОСТРОЕНИЕ БАЗОВОЙ КАРТЫ}
\CommentTok{\# {-}{-}{-}{-}{-}{-}{-}{-}{-}{-}{-}{-}{-}{-}{-}{-}{-}}
\NormalTok{map }\OtherTok{\textless{}{-}} \FunctionTok{ggplot}\NormalTok{() }\SpecialCharTok{+}
  \CommentTok{\# Векторные границы России}
  \FunctionTok{geom\_sf}\NormalTok{(}\AttributeTok{data =}\NormalTok{ russia\_map) }\SpecialCharTok{+}
  \CommentTok{\# Батиметрическая подложка (цвет = глубина)}
  \FunctionTok{geom\_tile}\NormalTok{(}\AttributeTok{data =}\NormalTok{ bat\_xyz, }\FunctionTok{aes}\NormalTok{(}\AttributeTok{x =}\NormalTok{ V1, }\AttributeTok{y =}\NormalTok{ V2, }\AttributeTok{fill =}\NormalTok{ V4), }\AttributeTok{show.legend =} \ConstantTok{FALSE}\NormalTok{) }\SpecialCharTok{+}
  \CommentTok{\# Цветовая схема для глубин}
  \FunctionTok{scale\_fill\_manual}\NormalTok{(}\AttributeTok{name =} \StringTok{"Глубина"}\NormalTok{, }\AttributeTok{values =}\NormalTok{ cols, }\AttributeTok{breaks =}\NormalTok{ niveles) }\SpecialCharTok{+}
  \CommentTok{\# Наложение сетки}
  \FunctionTok{geom\_sf}\NormalTok{(}\AttributeTok{data =}\NormalTok{ ga\_grid, }\AttributeTok{alpha =} \FloatTok{0.01}\NormalTok{, }\AttributeTok{linetype =} \DecValTok{3}\NormalTok{) }\SpecialCharTok{+}
  \CommentTok{\# Береговая линия}
  \FunctionTok{geom\_sf}\NormalTok{(}\AttributeTok{data =}\NormalTok{ coast, }\AttributeTok{linewidth =} \FloatTok{0.2}\NormalTok{, }\AttributeTok{fill =} \ConstantTok{NA}\NormalTok{) }\SpecialCharTok{+}
  \CommentTok{\# Ограничение области карты}
  \FunctionTok{coord\_sf}\NormalTok{(}\AttributeTok{xlim =} \FunctionTok{c}\NormalTok{(}\DecValTok{32}\NormalTok{, }\DecValTok{56}\NormalTok{), }\AttributeTok{ylim =} \FunctionTok{c}\NormalTok{(}\FloatTok{68.5}\NormalTok{, }\FloatTok{72.3}\NormalTok{)) }\SpecialCharTok{+}
  \CommentTok{\# Масштабная линейка (top{-}left)}
  \FunctionTok{annotation\_scale}\NormalTok{(}\AttributeTok{location =} \StringTok{"tl"}\NormalTok{, }\AttributeTok{width\_hint =} \FloatTok{0.2}\NormalTok{) }\SpecialCharTok{+}
  \CommentTok{\# Оформление}
  \FunctionTok{labs}\NormalTok{(}\AttributeTok{x =} \ConstantTok{NULL}\NormalTok{, }\AttributeTok{y =} \ConstantTok{NULL}\NormalTok{, }\AttributeTok{fill =} \StringTok{"Глубина (м)"}\NormalTok{) }\SpecialCharTok{+}
  \FunctionTok{theme}\NormalTok{(}\AttributeTok{panel.border =} \FunctionTok{element\_rect}\NormalTok{(}\AttributeTok{colour =} \StringTok{"black"}\NormalTok{, }\AttributeTok{fill =} \ConstantTok{NA}\NormalTok{, }\AttributeTok{linewidth =} \DecValTok{1}\NormalTok{))}

\CommentTok{\# {-}{-}{-}{-}{-}{-}{-}{-}{-}{-}{-}{-}{-}{-}{-}{-}{-}}
\CommentTok{\# ДОБАВЛЕНИЕ АННОТАЦИЙ}
\CommentTok{\# {-}{-}{-}{-}{-}{-}{-}{-}{-}{-}{-}{-}{-}{-}{-}{-}{-}}
\NormalTok{map }\OtherTok{\textless{}{-}}\NormalTok{ map }\SpecialCharTok{+}
  \FunctionTok{annotate}\NormalTok{(}\StringTok{"text"}\NormalTok{, }\AttributeTok{x =} \DecValTok{40}\NormalTok{, }\AttributeTok{y =} \FloatTok{72.1}\NormalTok{, }\AttributeTok{size =} \DecValTok{5}\NormalTok{, }
           \AttributeTok{label =} \StringTok{"Баренцево море"}\NormalTok{, }\AttributeTok{fontface =} \StringTok{"bold"}\NormalTok{) }\SpecialCharTok{+}
  \FunctionTok{annotate}\NormalTok{(}\StringTok{"text"}\NormalTok{, }\AttributeTok{x =} \FloatTok{52.2}\NormalTok{, }\AttributeTok{y =} \FloatTok{69.1}\NormalTok{, }\AttributeTok{size =} \DecValTok{4}\NormalTok{, }
           \AttributeTok{label =} \StringTok{"о. Колгуев"}\NormalTok{, }\AttributeTok{fontface =} \StringTok{"bold"}\NormalTok{) }\SpecialCharTok{+}
  \FunctionTok{annotate}\NormalTok{(}\StringTok{"text"}\NormalTok{, }\AttributeTok{x =} \DecValTok{33}\NormalTok{, }\AttributeTok{y =} \FloatTok{68.9}\NormalTok{, }\AttributeTok{size =} \DecValTok{4}\NormalTok{, }
           \AttributeTok{label =} \StringTok{"Кольский"}\NormalTok{, }\AttributeTok{fontface =} \StringTok{"bold"}\NormalTok{) }\SpecialCharTok{+}
  \FunctionTok{annotate}\NormalTok{(}\StringTok{"text"}\NormalTok{, }\AttributeTok{x =} \DecValTok{33}\NormalTok{, }\AttributeTok{y =} \FloatTok{68.6}\NormalTok{, }\AttributeTok{size =} \DecValTok{4}\NormalTok{, }
           \AttributeTok{label =} \StringTok{"п{-}ов"}\NormalTok{, }\AttributeTok{fontface =} \StringTok{"bold"}\NormalTok{)}

\CommentTok{\# {-}{-}{-}{-}{-}{-}{-}{-}{-}{-}{-}{-}{-}{-}{-}{-}{-}}
\CommentTok{\# ДОБАВЛЕНИЕ ТОЧЕК НАБЛЮДЕНИЙ}
\CommentTok{\# {-}{-}{-}{-}{-}{-}{-}{-}{-}{-}{-}{-}{-}{-}{-}{-}{-}}
\NormalTok{map }\OtherTok{\textless{}{-}}\NormalTok{ map }\SpecialCharTok{+}
  \CommentTok{\# Точки исследований краба (синие)}
  \FunctionTok{geom\_point}\NormalTok{(}
    \AttributeTok{data =}\NormalTok{ CRAB, }
    \FunctionTok{aes}\NormalTok{(}\AttributeTok{x =}\NormalTok{ X }\SpecialCharTok{+} \FloatTok{0.2}\NormalTok{, }\AttributeTok{y =}\NormalTok{ Y), }\CommentTok{\# Смещение для визуального разделения}
    \AttributeTok{size =} \DecValTok{3}\NormalTok{, }\AttributeTok{color =} \StringTok{"black"}\NormalTok{, }\AttributeTok{fill =} \StringTok{"\#1E90FF"}\NormalTok{, }
    \AttributeTok{shape =} \DecValTok{21}\NormalTok{, }\AttributeTok{alpha =} \DecValTok{1}
\NormalTok{  ) }\SpecialCharTok{+}
  \CommentTok{\# Точки летних исследований (оранжевые)}
  \FunctionTok{geom\_point}\NormalTok{(}
    \AttributeTok{data =}\NormalTok{ SUMMER, }
    \FunctionTok{aes}\NormalTok{(}\AttributeTok{x =}\NormalTok{ X, }\AttributeTok{y =}\NormalTok{ Y), }
    \AttributeTok{size =} \DecValTok{3}\NormalTok{, }\AttributeTok{color =} \StringTok{"black"}\NormalTok{, }\AttributeTok{fill =} \StringTok{"\#FFA500"}\NormalTok{, }
    \AttributeTok{shape =} \DecValTok{21}\NormalTok{, }\AttributeTok{alpha =} \DecValTok{1}
\NormalTok{  )}

\CommentTok{\# Вывод финальной карты}
\FunctionTok{print}\NormalTok{(map)}

\CommentTok{\# {-}{-}{-}{-}{-}{-}{-}{-}{-}{-}{-}{-}{-}{-}{-}{-}{-}}
\CommentTok{\# СОХРАНЕНИЕ РЕЗУЛЬТАТА}
\CommentTok{\# {-}{-}{-}{-}{-}{-}{-}{-}{-}{-}{-}{-}{-}{-}{-}{-}{-}}
\FunctionTok{ggsave}\NormalTok{(}\StringTok{"DATA\_MAP.jpg"}\NormalTok{, }
       \AttributeTok{plot =}\NormalTok{ map,          }\CommentTok{\# Используем явное указание объекта}
       \AttributeTok{device =} \StringTok{"jpeg"}\NormalTok{, }
       \AttributeTok{dpi =} \DecValTok{600}\NormalTok{,           }\CommentTok{\# Высокое разрешение}
       \AttributeTok{width =} \DecValTok{7}\NormalTok{,           }\CommentTok{\# Ширина в дюймах}
       \AttributeTok{height =} \DecValTok{5}\NormalTok{,          }\CommentTok{\# Высота в дюймах}
       \AttributeTok{units =} \StringTok{"in"}\NormalTok{)}
\end{Highlighting}
\end{Shaded}

\section{Карты с картой-врезкой и
маршрутом}\label{ux43aux430ux440ux442ux44b-ux441-ux43aux430ux440ux442ux43eux439-ux432ux440ux435ux437ux43aux43eux439-ux438-ux43cux430ux440ux448ux440ux443ux442ux43eux43c}

\begin{figure}[H]

{\centering \includegraphics[width=0.8\linewidth,height=\textheight,keepaspectratio]{images/KARTOGRAPH14.jpg}

}

\caption{Рис. 14.: Карты с картой-врезкой и маршрутом}

\end{figure}%

\begin{Shaded}
\begin{Highlighting}[]
\CommentTok{\# Очистка окружения и установка рабочей директории}
\FunctionTok{rm}\NormalTok{(}\AttributeTok{list =} \FunctionTok{ls}\NormalTok{()) }\CommentTok{\# Удаление всех объектов из глобального окружения}
\FunctionTok{setwd}\NormalTok{(}\StringTok{"C:/COURSES/KARTOGRAPH/"}\NormalTok{) }\CommentTok{\# Установка рабочей директории}

\CommentTok{\# {-}{-}{-}{-}{-}{-}{-}{-}{-}{-}{-}{-}{-}{-}{-}{-}{-}}
\CommentTok{\# ЗАГРУЗКА ПАКЕТОВ}
\CommentTok{\# {-}{-}{-}{-}{-}{-}{-}{-}{-}{-}{-}{-}{-}{-}{-}{-}{-}}
\FunctionTok{library}\NormalTok{(sf)          }\CommentTok{\# Пространственные операции с векторными данными}
\FunctionTok{library}\NormalTok{(marmap)      }\CommentTok{\# Работа с батиметрическими данными (карты глубин)}
\FunctionTok{library}\NormalTok{(tidyverse)   }\CommentTok{\# Коллекция пакетов для обработки данных}
\FunctionTok{library}\NormalTok{(rnaturalearth) }\CommentTok{\# Векторные картографические данные}
\FunctionTok{library}\NormalTok{(ggspatial)   }\CommentTok{\# Инструменты для пространственной визуализации}
\FunctionTok{library}\NormalTok{(readxl)      }\CommentTok{\# Импорт данных из Excel}
\FunctionTok{library}\NormalTok{(ggOceanMaps) }\CommentTok{\# Специализированные карты океанов}
\FunctionTok{library}\NormalTok{(cowplot)     }\CommentTok{\# Компоновка графиков и добавление элементов}

\CommentTok{\# {-}{-}{-}{-}{-}{-}{-}{-}{-}{-}{-}{-}{-}{-}{-}{-}{-}}
\CommentTok{\# ЗАГРУЗКА ДАННЫХ}
\CommentTok{\# {-}{-}{-}{-}{-}{-}{-}{-}{-}{-}{-}{-}{-}{-}{-}{-}{-}}
\CommentTok{\# Чтение данных из Excel}
\NormalTok{DATA }\OtherTok{\textless{}{-}}\NormalTok{ readxl}\SpecialCharTok{::}\FunctionTok{read\_excel}\NormalTok{(}\StringTok{"KARTOGRAPHIC.xlsx"}\NormalTok{, }\AttributeTok{sheet =} \StringTok{"SURVEY"}\NormalTok{)}

\CommentTok{\# Фильтрация данных (крабовые исследования 2022)}
\NormalTok{DATA }\OtherTok{\textless{}{-}}\NormalTok{ DATA[DATA}\SpecialCharTok{$}\NormalTok{SURV }\SpecialCharTok{==} \StringTok{"CRAB"} \SpecialCharTok{\&}\NormalTok{ DATA}\SpecialCharTok{$}\NormalTok{YEAR }\SpecialCharTok{==} \DecValTok{2022}\NormalTok{, ]}

\CommentTok{\# Загрузка векторных границ России}
\NormalTok{russia\_map }\OtherTok{\textless{}{-}} \FunctionTok{ne\_states}\NormalTok{(}\AttributeTok{country =} \StringTok{"russia"}\NormalTok{, }\AttributeTok{returnclass =} \StringTok{"sf"}\NormalTok{)}

\CommentTok{\# Установка границ региона интереса}
\NormalTok{xmin }\OtherTok{\textless{}{-}} \DecValTok{35}\NormalTok{; xmax }\OtherTok{\textless{}{-}} \DecValTok{50}
\NormalTok{ymin }\OtherTok{\textless{}{-}} \FloatTok{67.2}\NormalTok{; ymax }\OtherTok{\textless{}{-}} \DecValTok{71}

\CommentTok{\# {-}{-}{-}{-}{-}{-}{-}{-}{-}{-}{-}{-}{-}{-}{-}{-}{-}}
\CommentTok{\# БАТИМЕТРИЧЕСКИЕ ДАННЫЕ}
\CommentTok{\# {-}{-}{-}{-}{-}{-}{-}{-}{-}{-}{-}{-}{-}{-}{-}{-}{-}}
\CommentTok{\# Загрузка данных о глубинах}
\NormalTok{bat }\OtherTok{\textless{}{-}} \FunctionTok{getNOAA.bathy}\NormalTok{(xmin, xmax, ymin, ymax, }\AttributeTok{resolution =} \DecValTok{1}\NormalTok{, }\AttributeTok{keep =} \ConstantTok{TRUE}\NormalTok{)}
\NormalTok{bat\_xyz }\OtherTok{\textless{}{-}} \FunctionTok{as.xyz}\NormalTok{(bat)}

\CommentTok{\# Определение цветовых уровней для глубин}
\NormalTok{breaks }\OtherTok{\textless{}{-}} \FunctionTok{c}\NormalTok{(}\SpecialCharTok{{-}}\DecValTok{10000}\NormalTok{, }\SpecialCharTok{{-}}\DecValTok{7000}\NormalTok{, }\SpecialCharTok{{-}}\DecValTok{6000}\NormalTok{, }\SpecialCharTok{{-}}\DecValTok{5000}\NormalTok{, }\SpecialCharTok{{-}}\DecValTok{4000}\NormalTok{, }\SpecialCharTok{{-}}\DecValTok{3000}\NormalTok{, }\SpecialCharTok{{-}}\DecValTok{2000}\NormalTok{, }\SpecialCharTok{{-}}\DecValTok{1000}\NormalTok{, }
            \SpecialCharTok{{-}}\DecValTok{500}\NormalTok{, }\SpecialCharTok{{-}}\DecValTok{200}\NormalTok{, }\SpecialCharTok{{-}}\DecValTok{50}\NormalTok{, }\SpecialCharTok{{-}}\DecValTok{1}\NormalTok{, }\DecValTok{5}\NormalTok{, }\DecValTok{50}\NormalTok{, }\DecValTok{100}\NormalTok{, }\DecValTok{150}\NormalTok{, }\DecValTok{200}\NormalTok{, }\DecValTok{300}\NormalTok{, }\DecValTok{400}\NormalTok{, }\DecValTok{500}\NormalTok{, }\DecValTok{1000}\NormalTok{, }\DecValTok{3000}\NormalTok{)}
\NormalTok{cols }\OtherTok{\textless{}{-}} \FunctionTok{c}\NormalTok{(}
  \StringTok{"\#5e99d6"}\NormalTok{, }\StringTok{"\#669cd4"}\NormalTok{, }\StringTok{"\#6c9fd4"}\NormalTok{, }\StringTok{"\#96bce3"}\NormalTok{, }\StringTok{"\#AEC8E3"}\NormalTok{, }\StringTok{"\#a6c4e3"}\NormalTok{,}
  \StringTok{"\#AEC8E3"}\NormalTok{, }\StringTok{"\#BBD0EB"}\NormalTok{, }\StringTok{"\#C7DCF1"}\NormalTok{, }\StringTok{"\#DAECFA"}\NormalTok{, }\StringTok{"\#D2E5F6"}\NormalTok{, }\StringTok{"\#e1f2d8"}\NormalTok{,}
  \StringTok{"\#B8D3AA"}\NormalTok{, }\StringTok{"\#b3b387"}\NormalTok{, }\StringTok{"\#9EC187"}\NormalTok{, }\StringTok{"\#C7D097"}\NormalTok{, }\StringTok{"\#DADBAF"}\NormalTok{, }\StringTok{"\#F3F0C7"}\NormalTok{,}
  \StringTok{"\#E6DBA8"}\NormalTok{, }\StringTok{"\#DACFA1"}\NormalTok{, }\StringTok{"\#D1BF81"}\NormalTok{, }\StringTok{"\#C69D45"}
\NormalTok{)}

\CommentTok{\# Категоризация глубин}
\NormalTok{bat\_xyz}\SpecialCharTok{$}\NormalTok{V4 }\OtherTok{\textless{}{-}} \FunctionTok{cut}\NormalTok{(bat\_xyz}\SpecialCharTok{$}\NormalTok{V3, }\AttributeTok{breaks =}\NormalTok{ breaks)}
\NormalTok{niveles }\OtherTok{\textless{}{-}} \FunctionTok{levels}\NormalTok{(bat\_xyz}\SpecialCharTok{$}\NormalTok{V4)}

\CommentTok{\# Создание координатной сетки}
\NormalTok{ga\_grid }\OtherTok{\textless{}{-}}\NormalTok{ russia\_map }\SpecialCharTok{\%\textgreater{}\%} 
  \FunctionTok{st\_make\_grid}\NormalTok{(}\AttributeTok{cellsize =} \FunctionTok{c}\NormalTok{(}\DecValTok{2}\NormalTok{, }\FloatTok{0.5}\NormalTok{), }\AttributeTok{offset =} \FunctionTok{c}\NormalTok{(}\DecValTok{34}\NormalTok{, }\DecValTok{67}\NormalTok{))}

\CommentTok{\# {-}{-}{-}{-}{-}{-}{-}{-}{-}{-}{-}{-}{-}{-}{-}{-}{-}}
\CommentTok{\# ПОСТРОЕНИЕ ОСНОВНОЙ КАРТЫ}
\CommentTok{\# {-}{-}{-}{-}{-}{-}{-}{-}{-}{-}{-}{-}{-}{-}{-}{-}{-}}
\NormalTok{map }\OtherTok{\textless{}{-}} \FunctionTok{ggplot}\NormalTok{() }\SpecialCharTok{+}
  \CommentTok{\# Векторные границы России}
  \FunctionTok{geom\_sf}\NormalTok{(}\AttributeTok{data =}\NormalTok{ russia\_map) }\SpecialCharTok{+}
  \CommentTok{\# Батиметрическая подложка}
  \FunctionTok{geom\_tile}\NormalTok{(}\AttributeTok{data =}\NormalTok{ bat\_xyz, }\FunctionTok{aes}\NormalTok{(}\AttributeTok{x =}\NormalTok{ V1, }\AttributeTok{y =}\NormalTok{ V2, }\AttributeTok{fill =}\NormalTok{ V4), }\AttributeTok{show.legend =} \ConstantTok{FALSE}\NormalTok{) }\SpecialCharTok{+}
  \FunctionTok{scale\_fill\_manual}\NormalTok{(}\AttributeTok{values =}\NormalTok{ cols, }\AttributeTok{breaks =}\NormalTok{ niveles) }\SpecialCharTok{+}
  \CommentTok{\# Контур нулевой глубины (береговая линия)}
  \FunctionTok{geom\_contour}\NormalTok{(}\AttributeTok{data =}\NormalTok{ bat\_xyz, }\FunctionTok{aes}\NormalTok{(}\AttributeTok{x =}\NormalTok{ V1, }\AttributeTok{y =}\NormalTok{ V2, }\AttributeTok{z =}\NormalTok{ V3), }
               \AttributeTok{breaks =} \DecValTok{0}\NormalTok{, }\AttributeTok{color =} \StringTok{"black"}\NormalTok{, }\AttributeTok{linewidth =} \FloatTok{0.5}\NormalTok{) }\SpecialCharTok{+} 
  \CommentTok{\# Координатная сетка}
  \FunctionTok{geom\_sf}\NormalTok{(}\AttributeTok{data =}\NormalTok{ ga\_grid, }\AttributeTok{alpha =} \FloatTok{0.01}\NormalTok{, }\AttributeTok{linetype =} \DecValTok{3}\NormalTok{) }\SpecialCharTok{+}
  \CommentTok{\# Ограничение области карты}
  \FunctionTok{coord\_sf}\NormalTok{(}\AttributeTok{xlim =} \FunctionTok{c}\NormalTok{(}\DecValTok{36}\NormalTok{, }\DecValTok{49}\NormalTok{), }\AttributeTok{ylim =} \FunctionTok{c}\NormalTok{(}\FloatTok{67.4}\NormalTok{, }\FloatTok{70.8}\NormalTok{)) }\SpecialCharTok{+} 
  \CommentTok{\# Масштабная линейка}
  \FunctionTok{annotation\_scale}\NormalTok{(}\AttributeTok{location =} \StringTok{"tr"}\NormalTok{, }\AttributeTok{width\_hint =} \FloatTok{0.5}\NormalTok{) }\SpecialCharTok{+}
  \FunctionTok{labs}\NormalTok{(}\AttributeTok{x =} \ConstantTok{NULL}\NormalTok{, }\AttributeTok{y =} \ConstantTok{NULL}\NormalTok{) }\SpecialCharTok{+}
  \CommentTok{\# Географические подписи}
  \FunctionTok{annotate}\NormalTok{(}\StringTok{"text"}\NormalTok{, }\AttributeTok{x =} \DecValTok{47}\NormalTok{, }\AttributeTok{y =} \FloatTok{70.7}\NormalTok{, }\AttributeTok{size =} \DecValTok{5}\NormalTok{, }
           \AttributeTok{label =} \StringTok{"Баренцево море"}\NormalTok{, }\AttributeTok{fontface =} \StringTok{"bold"}\NormalTok{) }\SpecialCharTok{+}
  \FunctionTok{annotate}\NormalTok{(}\StringTok{"text"}\NormalTok{, }\AttributeTok{x =} \FloatTok{48.4}\NormalTok{, }\AttributeTok{y =} \FloatTok{68.62}\NormalTok{, }\AttributeTok{size =} \DecValTok{4}\NormalTok{,}
           \AttributeTok{label =} \StringTok{"о. Колгуев"}\NormalTok{, }\AttributeTok{fontface =} \StringTok{"bold"}\NormalTok{) }\SpecialCharTok{+}
  \FunctionTok{annotate}\NormalTok{(}\StringTok{"text"}\NormalTok{, }\AttributeTok{x =} \FloatTok{37.5}\NormalTok{, }\AttributeTok{y =} \FloatTok{67.7}\NormalTok{, }\AttributeTok{size =} \DecValTok{5}\NormalTok{,}
           \AttributeTok{label =} \StringTok{"Кольский п{-}ов"}\NormalTok{, }\AttributeTok{fontface =} \StringTok{"bold"}\NormalTok{) }\SpecialCharTok{+}
  \CommentTok{\# Маршрут и точки исследований}
  \FunctionTok{geom\_path}\NormalTok{(}\AttributeTok{data =}\NormalTok{ DATA, }\FunctionTok{aes}\NormalTok{(}\AttributeTok{x =}\NormalTok{ X, }\AttributeTok{y =}\NormalTok{ Y), }\AttributeTok{color =} \StringTok{"black"}\NormalTok{) }\SpecialCharTok{+}
  \FunctionTok{geom\_point}\NormalTok{(}\AttributeTok{data =}\NormalTok{ DATA, }\FunctionTok{aes}\NormalTok{(}\AttributeTok{x =}\NormalTok{ X, }\AttributeTok{y =}\NormalTok{ Y), }
             \AttributeTok{size =} \DecValTok{3}\NormalTok{, }\AttributeTok{color =} \StringTok{"black"}\NormalTok{, }\AttributeTok{fill =} \StringTok{"white"}\NormalTok{, }
             \AttributeTok{shape =} \DecValTok{21}\NormalTok{, }\AttributeTok{alpha =} \FloatTok{0.8}\NormalTok{) }\SpecialCharTok{+}
  \CommentTok{\# ДОБАВЛЕНИЕ РАМКИ {-} ключевое изменение}
  \FunctionTok{theme}\NormalTok{(}\AttributeTok{panel.border =} \FunctionTok{element\_rect}\NormalTok{(}\AttributeTok{colour =} \StringTok{"black"}\NormalTok{, }\AttributeTok{fill =} \ConstantTok{NA}\NormalTok{, }\AttributeTok{linewidth =} \FloatTok{1.5}\NormalTok{))}

\CommentTok{\# {-}{-}{-}{-}{-}{-}{-}{-}{-}{-}{-}{-}{-}{-}{-}{-}{-}}
\CommentTok{\# СОЗДАНИЕ ВСТАВКИ{-}ЛОКАЦИИ}
\CommentTok{\# {-}{-}{-}{-}{-}{-}{-}{-}{-}{-}{-}{-}{-}{-}{-}{-}{-}}
\CommentTok{\# Область для вставки}
\NormalTok{ins }\OtherTok{\textless{}{-}} \FunctionTok{data.frame}\NormalTok{(}\AttributeTok{lon =} \FunctionTok{c}\NormalTok{(}\DecValTok{10}\NormalTok{, }\DecValTok{10}\NormalTok{, }\DecValTok{70}\NormalTok{, }\DecValTok{70}\NormalTok{), }\AttributeTok{lat =} \FunctionTok{c}\NormalTok{(}\DecValTok{67}\NormalTok{, }\DecValTok{80}\NormalTok{, }\DecValTok{80}\NormalTok{, }\DecValTok{67}\NormalTok{))}

\CommentTok{\# Получение данных для вставки}
\NormalTok{mar\_bathy }\OtherTok{\textless{}{-}} \FunctionTok{getNOAA.bathy}\NormalTok{(}\DecValTok{9}\NormalTok{, }\DecValTok{71}\NormalTok{, }\FloatTok{66.5}\NormalTok{, }\DecValTok{83}\NormalTok{, }\AttributeTok{res =} \DecValTok{4}\NormalTok{, }\AttributeTok{keep =} \ConstantTok{TRUE}\NormalTok{)}
\NormalTok{bathy }\OtherTok{\textless{}{-}} \FunctionTok{raster\_bathymetry}\NormalTok{(stars}\SpecialCharTok{::}\FunctionTok{st\_as\_stars}\NormalTok{(marmap}\SpecialCharTok{::}\FunctionTok{as.raster}\NormalTok{(mar\_bathy)), }
                           \AttributeTok{depths =} \ConstantTok{NULL}\NormalTok{, }\AttributeTok{verbose =} \ConstantTok{FALSE}\NormalTok{)}

\CommentTok{\# Построение вставки}
\NormalTok{insetmap }\OtherTok{\textless{}{-}} \FunctionTok{basemap}\NormalTok{(ins, }\AttributeTok{shapefiles =} \FunctionTok{list}\NormalTok{(}\AttributeTok{land =}\NormalTok{ dd\_land, }\AttributeTok{bathy =}\NormalTok{ bathy), }
                   \AttributeTok{bathy.style =} \StringTok{"rub"}\NormalTok{, }\AttributeTok{legends =} \ConstantTok{FALSE}\NormalTok{) }\SpecialCharTok{+}
  \CommentTok{\# Прямоугольник, обозначающий область основной карты}
  \FunctionTok{geom\_rect}\NormalTok{(}\FunctionTok{aes}\NormalTok{(}\AttributeTok{xmin =} \DecValTok{35}\NormalTok{, }\AttributeTok{xmax =} \DecValTok{51}\NormalTok{, }\AttributeTok{ymin =} \FloatTok{67.5}\NormalTok{, }\AttributeTok{ymax =} \DecValTok{71}\NormalTok{), }
            \AttributeTok{fill =} \StringTok{"black"}\NormalTok{, }\AttributeTok{color =} \StringTok{"black"}\NormalTok{, }\AttributeTok{alpha =} \FloatTok{0.2}\NormalTok{) }\SpecialCharTok{+}
  \FunctionTok{labs}\NormalTok{(}\AttributeTok{y =} \ConstantTok{NULL}\NormalTok{, }\AttributeTok{x =} \ConstantTok{NULL}\NormalTok{) }\SpecialCharTok{+}
  \CommentTok{\# Упрощение оформления}
  \FunctionTok{theme}\NormalTok{(}\AttributeTok{axis.text.x =} \FunctionTok{element\_blank}\NormalTok{(), }
        \AttributeTok{axis.text.y =} \FunctionTok{element\_blank}\NormalTok{(),}
        \CommentTok{\# Рамка для вставки}
        \AttributeTok{panel.border =} \FunctionTok{element\_rect}\NormalTok{(}\AttributeTok{colour =} \StringTok{"black"}\NormalTok{, }\AttributeTok{fill =} \ConstantTok{NA}\NormalTok{, }\AttributeTok{linewidth =} \DecValTok{1}\NormalTok{))}

\CommentTok{\# {-}{-}{-}{-}{-}{-}{-}{-}{-}{-}{-}{-}{-}{-}{-}{-}{-}}
\CommentTok{\# ФИНАЛЬНАЯ КОМПОНОВКА С РАМКОЙ}
\CommentTok{\# {-}{-}{-}{-}{-}{-}{-}{-}{-}{-}{-}{-}{-}{-}{-}{-}{-}}
\NormalTok{MAP }\OtherTok{\textless{}{-}} \FunctionTok{ggdraw}\NormalTok{() }\SpecialCharTok{+}
  \CommentTok{\# Основная карта}
  \FunctionTok{draw\_plot}\NormalTok{(map) }\SpecialCharTok{+}
  \CommentTok{\# Вставка с позиционированием}
  \FunctionTok{draw\_plot}\NormalTok{(insetmap,}
            \AttributeTok{height =} \FloatTok{0.3}\NormalTok{,}
            \AttributeTok{x =} \SpecialCharTok{{-}}\FloatTok{0.26}\NormalTok{,}
            \AttributeTok{y =} \FloatTok{0.55}\NormalTok{) }

\CommentTok{\# Вывод финальной карты}
\FunctionTok{print}\NormalTok{(MAP)}

\CommentTok{\# {-}{-}{-}{-}{-}{-}{-}{-}{-}{-}{-}{-}{-}{-}{-}{-}{-}}
\CommentTok{\# СОХРАНЕНИЕ РЕЗУЛЬТАТА}
\CommentTok{\# {-}{-}{-}{-}{-}{-}{-}{-}{-}{-}{-}{-}{-}{-}{-}{-}{-}}
\FunctionTok{ggsave}\NormalTok{(}\StringTok{"DATA\_MAP\_FRAMED.jpg"}\NormalTok{, }
       \AttributeTok{plot =}\NormalTok{ MAP,}
       \AttributeTok{device =} \StringTok{"jpeg"}\NormalTok{, }
       \AttributeTok{dpi =} \DecValTok{600}\NormalTok{,}
       \AttributeTok{width =} \DecValTok{7}\NormalTok{,}
       \AttributeTok{height =} \DecValTok{6}\NormalTok{,}
       \AttributeTok{units =} \StringTok{"in"}\NormalTok{)}
\end{Highlighting}
\end{Shaded}

\bookmarksetup{startatroot}

\chapter{sdmTMB - оценка и визуализация индекса обилия по
съемке}\label{sdmtmb---ux43eux446ux435ux43dux43aux430-ux438-ux432ux438ux437ux443ux430ux43bux438ux437ux430ux446ux438ux44f-ux438ux43dux434ux435ux43aux441ux430-ux43eux431ux438ux43bux438ux44f-ux43fux43e-ux441ux44aux435ux43cux43aux435}

\section{Введение}\label{ux432ux432ux435ux434ux435ux43dux438ux435-3}

Оценка индекса промыслового запаса камчатского краба в Баренцевом море с
использованием пространственно-временного моделирования (библиотека R:
sdmTMB)

\textbf{Цель:}\\
Продемонстрировать применение современных методов SDM (Species
Distribution Modeling) и GAMM (Generalized Additive Mixed Models) для
стандартизации оценки запасов промысловых видов на примере камчатского
краба.

\textbf{Ключевые аспекты:}

\begin{enumerate}
\def\labelenumi{\arabic{enumi}.}
\item
  \textbf{Подготовка данных}:

  \begin{itemize}
  \item
    Преобразование координат в проекцию UTM (км)
  \item
    Фильтрация данных через выпуклую оболочку (convex hull)
  \item
    Создание прогнозной сетки с шагом 10 км (2 км)
  \end{itemize}
\item
  \textbf{Моделирование}:

  \begin{itemize}
  \item
    Построение треугольной сетки (mesh) для учета пространственной
    автокорреляции
  \item
    Подбор модели sdmTMB с пространственно-временными случайными
    эффектами
  \item
    Учет ключевых факторов: температура, глубина, тип съемки
  \end{itemize}
\item
  \textbf{Визуализация}:

  \begin{itemize}
  \item
    Карты распределения плотности с наложением данных съемок
  \item
    Динамика индекса обилия с 50\% и 95\% доверительными интервалами
  \end{itemize}
\end{enumerate}

\textbf{Для работы скрипта:}

\begin{enumerate}
\def\labelenumi{\arabic{enumi}.}
\item
  Скачайте файл данных
  (\href{https://mombus.github.io/cRab/data/KARTOGRAPHIC.xlsx}{KARTOGRAPHIC.xlsx})
\item
  Установите рабочую директорию в setwd()
\item
  Установите необходимые пакеты (см. начало скрипта).
\end{enumerate}

\section{Базовая
оценка}\label{ux431ux430ux437ux43eux432ux430ux44f-ux43eux446ux435ux43dux43aux430}

\begin{Shaded}
\begin{Highlighting}[]
\CommentTok{\# {-}{-}{-}{-}{-}{-}{-}{-}{-}{-}{-}{-}{-}{-}{-}{-}{-}{-}{-}{-}{-}{-}{-}{-}{-}{-}{-}}
\CommentTok{\# 1. ПОДГОТОВКА СРЕДЫ И ДАННЫХ}
\CommentTok{\# {-}{-}{-}{-}{-}{-}{-}{-}{-}{-}{-}{-}{-}{-}{-}{-}{-}{-}{-}{-}{-}{-}{-}{-}{-}{-}{-}}

\CommentTok{\# Очистка рабочей среды}
\FunctionTok{rm}\NormalTok{(}\AttributeTok{list =} \FunctionTok{ls}\NormalTok{())}

\CommentTok{\# Установка рабочей директории (замените на свою)}
\FunctionTok{setwd}\NormalTok{(}\StringTok{"C:/COMBINE/"}\NormalTok{)}

\CommentTok{\# Загрузка необходимых пакетов}
\FunctionTok{library}\NormalTok{(readxl)       }\CommentTok{\# Для чтения Excel{-}файлов}
\FunctionTok{library}\NormalTok{(ggplot2)      }\CommentTok{\# Визуализация данных}
\FunctionTok{library}\NormalTok{(dplyr)        }\CommentTok{\# Обработка данных}
\FunctionTok{library}\NormalTok{(PBSmapping)   }\CommentTok{\# Для работы с пространственными данными}
\FunctionTok{library}\NormalTok{(sdmTMB)       }\CommentTok{\# Пространственно{-}временное моделирование}
\FunctionTok{library}\NormalTok{(INLA)         }\CommentTok{\# Продвинутые пространственные модели}
\FunctionTok{library}\NormalTok{(sp)           }\CommentTok{\# Классы для пространственных данных}
\FunctionTok{library}\NormalTok{(sf)           }\CommentTok{\# Пространственные данные (современный формат)}
\FunctionTok{library}\NormalTok{(rnaturalearth) }\CommentTok{\# Загрузка картографических данных}

\CommentTok{\# Загрузка данных из Excel{-}файла}
\NormalTok{data }\OtherTok{\textless{}{-}}\NormalTok{ readxl}\SpecialCharTok{::}\FunctionTok{read\_excel}\NormalTok{(}\StringTok{"KARTOGRAPHIC.xlsx"}\NormalTok{, }\AttributeTok{sheet =} \StringTok{"SURVEY"}\NormalTok{)}

\CommentTok{\# Просмотр структуры данных}
\FunctionTok{str}\NormalTok{(data)}
\end{Highlighting}
\end{Shaded}

Должно выглядеть так:

\begin{Shaded}
\begin{Highlighting}[]
\SpecialCharTok{\textgreater{}} \FunctionTok{str}\NormalTok{(data)}
\NormalTok{tibble [}\DecValTok{1}\NormalTok{,}\DecValTok{126}\NormalTok{ x }\DecValTok{20}\NormalTok{] (S3}\SpecialCharTok{:}\NormalTok{ tbl\_df}\SpecialCharTok{/}\NormalTok{tbl}\SpecialCharTok{/}\NormalTok{data.frame)}
 \SpecialCharTok{$}\NormalTok{ NUM    }\SpecialCharTok{:}\NormalTok{ num [}\DecValTok{1}\SpecialCharTok{:}\DecValTok{1126}\NormalTok{] }\DecValTok{1} \DecValTok{2} \DecValTok{3} \DecValTok{4} \DecValTok{5} \DecValTok{6} \DecValTok{7} \DecValTok{8} \DecValTok{9} \DecValTok{10}\NormalTok{ ...}
 \SpecialCharTok{$}\NormalTok{ CALL   }\SpecialCharTok{:}\NormalTok{ chr [}\DecValTok{1}\SpecialCharTok{:}\DecValTok{1126}\NormalTok{] }\StringTok{"UFJN"} \StringTok{"UFJN"} \StringTok{"UFJN"} \StringTok{"UFJN"}\NormalTok{ ...}
 \SpecialCharTok{$}\NormalTok{ CRUSE  }\SpecialCharTok{:}\NormalTok{ num [}\DecValTok{1}\SpecialCharTok{:}\DecValTok{1126}\NormalTok{] }\DecValTok{112} \DecValTok{112} \DecValTok{112} \DecValTok{112} \DecValTok{112} \DecValTok{112} \DecValTok{112} \DecValTok{112} \DecValTok{112} \DecValTok{112}\NormalTok{ ...}
 \SpecialCharTok{$}\NormalTok{ SURV   }\SpecialCharTok{:}\NormalTok{ chr [}\DecValTok{1}\SpecialCharTok{:}\DecValTok{1126}\NormalTok{] }\StringTok{"SUM"} \StringTok{"SUM"} \StringTok{"SUM"} \StringTok{"SUM"}\NormalTok{ ...}
 \SpecialCharTok{$}\NormalTok{ TRAL   }\SpecialCharTok{:}\NormalTok{ num [}\DecValTok{1}\SpecialCharTok{:}\DecValTok{1126}\NormalTok{] }\DecValTok{2} \DecValTok{3} \DecValTok{5} \DecValTok{7} \DecValTok{9} \DecValTok{11} \DecValTok{13} \DecValTok{15} \DecValTok{17} \DecValTok{19}\NormalTok{ ...}
 \SpecialCharTok{$}\NormalTok{ DATE   }\SpecialCharTok{:}\NormalTok{ POSIXct[}\DecValTok{1}\SpecialCharTok{:}\DecValTok{1126}\NormalTok{], format}\SpecialCharTok{:} \StringTok{"2019{-}08{-}16"} \StringTok{"2019{-}08{-}16"}\NormalTok{ ...}
 \SpecialCharTok{$}\NormalTok{ MONTH  }\SpecialCharTok{:}\NormalTok{ num [}\DecValTok{1}\SpecialCharTok{:}\DecValTok{1126}\NormalTok{] }\DecValTok{8} \DecValTok{8} \DecValTok{8} \DecValTok{8} \DecValTok{8} \DecValTok{8} \DecValTok{8} \DecValTok{8} \DecValTok{8} \DecValTok{8}\NormalTok{ ...}
 \SpecialCharTok{$}\NormalTok{ YEAR   }\SpecialCharTok{:}\NormalTok{ num [}\DecValTok{1}\SpecialCharTok{:}\DecValTok{1126}\NormalTok{] }\DecValTok{2019} \DecValTok{2019} \DecValTok{2019} \DecValTok{2019} \DecValTok{2019}\NormalTok{ ...}
 \SpecialCharTok{$}\NormalTok{ TIME   }\SpecialCharTok{:}\NormalTok{ chr [}\DecValTok{1}\SpecialCharTok{:}\DecValTok{1126}\NormalTok{] }\StringTok{"9:43"} \StringTok{"14:19"} \StringTok{"19:33"} \StringTok{"2:47"}\NormalTok{ ...}
 \SpecialCharTok{$}\NormalTok{ DECMIN }\SpecialCharTok{:}\NormalTok{ num [}\DecValTok{1}\SpecialCharTok{:}\DecValTok{1126}\NormalTok{] }\FloatTok{1.04} \FloatTok{0.15} \FloatTok{0.15} \FloatTok{0.15} \FloatTok{0.15} \FloatTok{0.15} \FloatTok{0.15} \FloatTok{0.15} \FloatTok{0.15} \FloatTok{0.15}\NormalTok{ ...}
 \SpecialCharTok{$}\NormalTok{ DUR    }\SpecialCharTok{:}\NormalTok{ num [}\DecValTok{1}\SpecialCharTok{:}\DecValTok{1126}\NormalTok{] }\FloatTok{1.73} \FloatTok{0.25} \FloatTok{0.25} \FloatTok{0.25} \FloatTok{0.25}\NormalTok{ ...}
 \SpecialCharTok{$}\NormalTok{ DEPTH  }\SpecialCharTok{:}\NormalTok{ num [}\DecValTok{1}\SpecialCharTok{:}\DecValTok{1126}\NormalTok{] }\DecValTok{200} \DecValTok{198} \DecValTok{196} \DecValTok{132} \DecValTok{128} \DecValTok{131} \DecValTok{64} \DecValTok{73} \DecValTok{91} \DecValTok{62}\NormalTok{ ...}
 \SpecialCharTok{$}\NormalTok{ SPEED  }\SpecialCharTok{:}\NormalTok{ num [}\DecValTok{1}\SpecialCharTok{:}\DecValTok{1126}\NormalTok{] }\DecValTok{3} \DecValTok{3} \DecValTok{3} \DecValTok{3} \DecValTok{3} \DecValTok{3} \DecValTok{3} \DecValTok{3} \DecValTok{3} \DecValTok{3}\NormalTok{ ...}
 \SpecialCharTok{$}\NormalTok{ CATCH  }\SpecialCharTok{:}\NormalTok{ num [}\DecValTok{1}\SpecialCharTok{:}\DecValTok{1126}\NormalTok{] }\FloatTok{12.9} \FloatTok{365.3} \DecValTok{253} \FloatTok{163.9} \FloatTok{55.7}\NormalTok{ ...}
 \SpecialCharTok{$}\NormalTok{ Y      }\SpecialCharTok{:}\NormalTok{ num [}\DecValTok{1}\SpecialCharTok{:}\DecValTok{1126}\NormalTok{] }\FloatTok{69.5} \FloatTok{69.5} \FloatTok{69.4} \FloatTok{68.8} \FloatTok{69.4}\NormalTok{ ...}
 \SpecialCharTok{$}\NormalTok{ X      }\SpecialCharTok{:}\NormalTok{ num [}\DecValTok{1}\SpecialCharTok{:}\DecValTok{1126}\NormalTok{] }\FloatTok{35.8} \FloatTok{35.9} \FloatTok{37.4} \FloatTok{38.6} \DecValTok{39}\NormalTok{ ...}
 \SpecialCharTok{$}\NormalTok{ PROM   }\SpecialCharTok{:}\NormalTok{ num [}\DecValTok{1}\SpecialCharTok{:}\DecValTok{1126}\NormalTok{] }\DecValTok{2} \DecValTok{0} \DecValTok{0} \DecValTok{3} \DecValTok{0} \DecValTok{6} \DecValTok{6} \DecValTok{34} \DecValTok{22} \DecValTok{9}\NormalTok{ ...}
 \SpecialCharTok{$}\NormalTok{ Density}\SpecialCharTok{:}\NormalTok{ num [}\DecValTok{1}\SpecialCharTok{:}\DecValTok{1126}\NormalTok{] }\DecValTok{30} \DecValTok{0} \DecValTok{0} \DecValTok{45} \DecValTok{0}\NormalTok{ ...}
 \SpecialCharTok{$}\NormalTok{ DIST   }\SpecialCharTok{:}\NormalTok{ num [}\DecValTok{1}\SpecialCharTok{:}\DecValTok{1126}\NormalTok{] }\FloatTok{28.7} \FloatTok{28.7} \FloatTok{49.9} \FloatTok{37.3} \FloatTok{90.8}\NormalTok{ ...}
 \SpecialCharTok{$}\NormalTok{ TEMP   }\SpecialCharTok{:}\NormalTok{ num [}\DecValTok{1}\SpecialCharTok{:}\DecValTok{1126}\NormalTok{] }\FloatTok{5.57} \FloatTok{5.49} \FloatTok{4.99} \FloatTok{4.8} \FloatTok{4.4}\NormalTok{ ...}
\SpecialCharTok{\textgreater{}} 
\end{Highlighting}
\end{Shaded}

Далее:

\begin{Shaded}
\begin{Highlighting}[]
\CommentTok{\# {-}{-}{-}{-}{-}{-}{-}{-}{-}{-}{-}{-}{-}{-}{-}{-}{-}{-}{-}{-}{-}{-}{-}{-}{-}{-}{-}{-}{-}{-}{-}{-}{-}{-}{-}{-}{-}{-}{-}{-}{-}{-}{-}{-}{-}{-}{-}{-}{-}{-}}
\CommentTok{\# 2. ПРЕОБРАЗОВАНИЕ КООРДИНАТ В ПРОЕКЦИЮ UTM (в км)}
\CommentTok{\# {-}{-}{-}{-}{-}{-}{-}{-}{-}{-}{-}{-}{-}{-}{-}{-}{-}{-}{-}{-}{-}{-}{-}{-}{-}{-}{-}{-}{-}{-}{-}{-}{-}{-}{-}{-}{-}{-}{-}{-}{-}{-}{-}{-}{-}{-}{-}{-}{-}{-}}

\CommentTok{\# Создание пространственного объекта из данных}
\NormalTok{data\_sf }\OtherTok{\textless{}{-}} \FunctionTok{st\_as\_sf}\NormalTok{(}
\NormalTok{  data, }
  \AttributeTok{coords =} \FunctionTok{c}\NormalTok{(}\StringTok{"X"}\NormalTok{, }\StringTok{"Y"}\NormalTok{), }\CommentTok{\# Указание столбцов с координатами}
  \AttributeTok{crs =} \DecValTok{4326}            \CommentTok{\# Система координат WGS84 (широта/долгота)}
\NormalTok{) }

\CommentTok{\# Преобразование в UTM зону 37N (метры)}
\NormalTok{data\_utm }\OtherTok{\textless{}{-}} \FunctionTok{st\_transform}\NormalTok{(data\_sf, }\AttributeTok{crs =} \DecValTok{32637}\NormalTok{) }

\CommentTok{\# Извлечение координат и перевод в километры}
\NormalTok{utm\_coords }\OtherTok{\textless{}{-}} \FunctionTok{st\_coordinates}\NormalTok{(data\_utm)}
\NormalTok{data}\SpecialCharTok{$}\NormalTok{xkm }\OtherTok{\textless{}{-}}\NormalTok{ utm\_coords[, }\DecValTok{1}\NormalTok{] }\SpecialCharTok{/} \DecValTok{1000}  \CommentTok{\# X в км}
\NormalTok{data}\SpecialCharTok{$}\NormalTok{ykm }\OtherTok{\textless{}{-}}\NormalTok{ utm\_coords[, }\DecValTok{2}\NormalTok{] }\SpecialCharTok{/} \DecValTok{1000}  \CommentTok{\# Y в км}

\CommentTok{\# Очистка временных объектов}
\FunctionTok{rm}\NormalTok{(data\_sf, data\_utm, utm\_coords)}

\CommentTok{\# {-}{-}{-}{-}{-}{-}{-}{-}{-}{-}{-}{-}{-}{-}{-}{-}{-}{-}{-}{-}{-}{-}{-}{-}{-}{-}{-}{-}{-}{-}{-}{-}{-}{-}{-}{-}{-}{-}{-}{-}{-}}
\CommentTok{\# 3. ОПРЕДЕЛЕНИЕ ГРАНИЦ ИССЛЕДОВАНИЯ}
\CommentTok{\# {-}{-}{-}{-}{-}{-}{-}{-}{-}{-}{-}{-}{-}{-}{-}{-}{-}{-}{-}{-}{-}{-}{-}{-}{-}{-}{-}{-}{-}{-}{-}{-}{-}{-}{-}{-}{-}{-}{-}{-}{-}}

\CommentTok{\# Вычисление границ исследовательского полигона}
\NormalTok{xl }\OtherTok{\textless{}{-}} \FunctionTok{c}\NormalTok{(}\FunctionTok{min}\NormalTok{(data}\SpecialCharTok{$}\NormalTok{xkm), }\FunctionTok{max}\NormalTok{(data}\SpecialCharTok{$}\NormalTok{xkm))  }\CommentTok{\# Границы по X}
\NormalTok{yl }\OtherTok{\textless{}{-}} \FunctionTok{c}\NormalTok{(}\FunctionTok{min}\NormalTok{(data}\SpecialCharTok{$}\NormalTok{ykm), }\FunctionTok{max}\NormalTok{(data}\SpecialCharTok{$}\NormalTok{ykm))  }\CommentTok{\# Границы по Y}

\CommentTok{\# {-}{-}{-}{-}{-}{-}{-}{-}{-}{-}{-}{-}{-}{-}{-}{-}{-}{-}{-}{-}{-}{-}{-}{-}{-}{-}{-}{-}{-}{-}{-}{-}{-}{-}{-}{-}{-}{-}{-}{-}}
\CommentTok{\# 4. СОЗДАНИЕ РАСТРОВОЙ СЕТКИ ДЛЯ МОДЕЛИ}
\CommentTok{\# {-}{-}{-}{-}{-}{-}{-}{-}{-}{-}{-}{-}{-}{-}{-}{-}{-}{-}{-}{-}{-}{-}{-}{-}{-}{-}{-}{-}{-}{-}{-}{-}{-}{-}{-}{-}{-}{-}{-}{-}}

\CommentTok{\# Создание равномерной сетки с шагом 10 км }
\NormalTok{GRID }\OtherTok{\textless{}{-}} \FunctionTok{makeGrid}\NormalTok{(}
  \AttributeTok{x =} \FunctionTok{seq}\NormalTok{(xl[}\DecValTok{1}\NormalTok{], xl[}\DecValTok{2}\NormalTok{], }\DecValTok{10}\NormalTok{), }
  \AttributeTok{y =} \FunctionTok{seq}\NormalTok{(yl[}\DecValTok{1}\NormalTok{], yl[}\DecValTok{2}\NormalTok{], }\DecValTok{10}\NormalTok{),}
  \AttributeTok{byrow =} \ConstantTok{FALSE}\NormalTok{,}
  \AttributeTok{projection =} \StringTok{"UTM"}\NormalTok{, }
  \AttributeTok{zone =} \DecValTok{37}
\NormalTok{)}

\CommentTok{\# Расчет центроидов ячеек сетки}
\NormalTok{GRID }\OtherTok{\textless{}{-}} \FunctionTok{calcCentroid}\NormalTok{(GRID, }\AttributeTok{rollup =} \DecValTok{3}\NormalTok{)}

\CommentTok{\# {-}{-}{-}{-}{-}{-}{-}{-}{-}{-}{-}{-}{-}{-}{-}{-}{-}{-}{-}{-}{-}{-}{-}{-}{-}{-}{-}{-}{-}{-}{-}{-}{-}{-}{-}{-}{-}{-}{-}{-}{-}{-}{-}{-}{-}{-}{-}{-}{-}{-}{-}{-}{-}{-}{-}{-}{-}{-}{-}}
\CommentTok{\# 5. ПОСТРОЕНИЕ ВЫПУКЛОЙ ОБОЛОЧКИ (CONVEX HULL) ДЛЯ ДАННЫХ}
\CommentTok{\# {-}{-}{-}{-}{-}{-}{-}{-}{-}{-}{-}{-}{-}{-}{-}{-}{-}{-}{-}{-}{-}{-}{-}{-}{-}{-}{-}{-}{-}{-}{-}{-}{-}{-}{-}{-}{-}{-}{-}{-}{-}{-}{-}{-}{-}{-}{-}{-}{-}{-}{-}{-}{-}{-}{-}{-}{-}{-}{-}}

\CommentTok{\# Создание выпуклой оболочки вокруг точек данных}
\NormalTok{Hull }\OtherTok{\textless{}{-}} \FunctionTok{inla.nonconvex.hull}\NormalTok{(}\FunctionTok{cbind}\NormalTok{(data}\SpecialCharTok{$}\NormalTok{xkm, data}\SpecialCharTok{$}\NormalTok{ykm), }\AttributeTok{convex =} \SpecialCharTok{{-}}\FloatTok{0.03}\NormalTok{)}

\CommentTok{\# Визуализация оболочки }
 \FunctionTok{plot}\NormalTok{(Hull)}
\end{Highlighting}
\end{Shaded}

\begin{figure}[H]

{\centering \includegraphics[width=0.6\linewidth,height=\textheight,keepaspectratio]{images/sdmTMB1.PNG}

}

\caption{Рис. 1.: Визуализация оболочки съемок}

\end{figure}%

\begin{Shaded}
\begin{Highlighting}[]
\CommentTok{\# Визуализация оболочки и точек съемок 2019{-}2024}
\FunctionTok{points}\NormalTok{(data}\SpecialCharTok{$}\NormalTok{xkm, data}\SpecialCharTok{$}\NormalTok{ykm, }\AttributeTok{pch=}\DecValTok{1}\NormalTok{, }\AttributeTok{cex=}\FloatTok{0.55}\NormalTok{,}\AttributeTok{col=}\StringTok{"black"}\NormalTok{)}
\end{Highlighting}
\end{Shaded}

\begin{figure}[H]

{\centering \includegraphics[width=0.6\linewidth,height=\textheight,keepaspectratio]{images/sdmTMB2.PNG}

}

\caption{Рис. 2.: Визуализация оболочки и точек съемок}

\end{figure}%

\begin{Shaded}
\begin{Highlighting}[]
\CommentTok{\# Фильтрация сетки: оставляем только точки внутри оболочки}
\NormalTok{line }\OtherTok{\textless{}{-}}\NormalTok{ Hull}\SpecialCharTok{$}\NormalTok{loc[, }\DecValTok{1}\SpecialCharTok{:}\DecValTok{2}\NormalTok{] }\SpecialCharTok{\%\textgreater{}\%} \FunctionTok{as.data.frame}\NormalTok{()}
\FunctionTok{colnames}\NormalTok{(line) }\OtherTok{\textless{}{-}} \FunctionTok{c}\NormalTok{(}\StringTok{"X"}\NormalTok{, }\StringTok{"Y"}\NormalTok{)}
\NormalTok{GRID}\SpecialCharTok{$}\NormalTok{AREA }\OtherTok{\textless{}{-}} \FunctionTok{point.in.polygon}\NormalTok{(GRID}\SpecialCharTok{$}\NormalTok{X, GRID}\SpecialCharTok{$}\NormalTok{Y, line}\SpecialCharTok{$}\NormalTok{X, line}\SpecialCharTok{$}\NormalTok{Y)}
\NormalTok{GRID }\OtherTok{\textless{}{-}}\NormalTok{ GRID[GRID}\SpecialCharTok{$}\NormalTok{AREA }\SpecialCharTok{\textgreater{}} \FloatTok{0.1}\NormalTok{, }\FunctionTok{c}\NormalTok{(}\StringTok{"X"}\NormalTok{, }\StringTok{"Y"}\NormalTok{)]  }\CommentTok{\# Только внутренние точки}

\CommentTok{\# {-}{-}{-}{-}{-}{-}{-}{-}{-}{-}{-}{-}{-}{-}{-}{-}{-}{-}{-}{-}{-}{-}{-}{-}{-}{-}{-}{-}{-}{-}{-}{-}{-}{-}{-}{-}{-}{-}{-}{-}{-}{-}{-}{-}{-}{-}{-}{-}{-}}
\CommentTok{\# 6. ПОДГОТОВКА СЕТКИ ДЛЯ ПРОГНОЗИРОВАНИЯ}
\CommentTok{\# {-}{-}{-}{-}{-}{-}{-}{-}{-}{-}{-}{-}{-}{-}{-}{-}{-}{-}{-}{-}{-}{-}{-}{-}{-}{-}{-}{-}{-}{-}{-}{-}{-}{-}{-}{-}{-}{-}{-}{-}{-}{-}{-}{-}{-}{-}{-}{-}{-}}

\CommentTok{\# Создание временной сетки (для каждого года)}
\NormalTok{grid }\OtherTok{\textless{}{-}} \FunctionTok{replicate\_df}\NormalTok{(GRID, }\StringTok{"YEAR"}\NormalTok{, }\FunctionTok{unique}\NormalTok{(data}\SpecialCharTok{$}\NormalTok{YEAR))}
\FunctionTok{colnames}\NormalTok{(grid) }\OtherTok{\textless{}{-}} \FunctionTok{c}\NormalTok{(}\StringTok{"xkm"}\NormalTok{, }\StringTok{"ykm"}\NormalTok{, }\StringTok{"YEAR"}\NormalTok{)}
\NormalTok{grid}\SpecialCharTok{$}\NormalTok{SURV }\OtherTok{\textless{}{-}} \StringTok{"CRAB"}  \CommentTok{\# Добавляем информацию о типе съемки}

\CommentTok{\# Визуализация оболочки и сетки для прогнозирования (grid)}
 \FunctionTok{plot}\NormalTok{(Hull)}
 \FunctionTok{points}\NormalTok{(grid}\SpecialCharTok{$}\NormalTok{xkm, grid}\SpecialCharTok{$}\NormalTok{ykm, }\AttributeTok{pch=}\DecValTok{1}\NormalTok{, }\AttributeTok{cex=}\FloatTok{0.55}\NormalTok{,}\AttributeTok{col=}\StringTok{"black"}\NormalTok{)}
\end{Highlighting}
\end{Shaded}

\begin{figure}[H]

{\centering \includegraphics[width=0.6\linewidth,height=\textheight,keepaspectratio]{images/sdmTMB3.PNG}

}

\caption{Рис. 3.: Визуализация оболочки и сетки для прогнозирования
(grid)}

\end{figure}%

\begin{Shaded}
\begin{Highlighting}[]
\CommentTok{\# {-}{-}{-}{-}{-}{-}{-}{-}{-}{-}{-}{-}{-}{-}{-}{-}{-}{-}{-}{-}{-}{-}{-}{-}{-}{-}{-}{-}{-}{-}{-}{-}{-}{-}{-}{-}{-}{-}{-}{-}{-}{-}{-}{-}{-}{-}{-}{-}{-}{-}{-}}
\CommentTok{\# 7. ПОСТРОЕНИЕ ПРОСТРАНСТВЕННОЙ СЕТКИ (MESH)}
\CommentTok{\# {-}{-}{-}{-}{-}{-}{-}{-}{-}{-}{-}{-}{-}{-}{-}{-}{-}{-}{-}{-}{-}{-}{-}{-}{-}{-}{-}{-}{-}{-}{-}{-}{-}{-}{-}{-}{-}{-}{-}{-}{-}{-}{-}{-}{-}{-}{-}{-}{-}{-}{-}}

\CommentTok{\# Создание треугольной сетки для пространственного моделирования}
\NormalTok{mesh\_sdm }\OtherTok{\textless{}{-}} \FunctionTok{make\_mesh}\NormalTok{(}
\NormalTok{  data, }
  \FunctionTok{c}\NormalTok{(}\StringTok{"xkm"}\NormalTok{, }\StringTok{"ykm"}\NormalTok{),  }\CommentTok{\# Координаты}
  \AttributeTok{cutoff =} \DecValTok{10}        \CommentTok{\# Минимальное расстояние между узлами (км)}
\NormalTok{)}

\CommentTok{\# Визуализация сетки }
 \FunctionTok{plot}\NormalTok{(mesh\_sdm)}
\end{Highlighting}
\end{Shaded}

\begin{figure}[H]

{\centering \includegraphics[width=0.6\linewidth,height=\textheight,keepaspectratio]{images/sdmTMB4.PNG}

}

\caption{Рис. 4.: Визуализация сетки (mesh)}

\end{figure}%

\begin{Shaded}
\begin{Highlighting}[]
\CommentTok{\# {-}{-}{-}{-}{-}{-}{-}{-}{-}{-}{-}{-}{-}{-}{-}{-}{-}{-}{-}{-}{-}{-}{-}{-}{-}{-}{-}{-}{-}{-}{-}{-}{-}{-}{-}{-}{-}{-}{-}{-}{-}{-}{-}{-}{-}{-}{-}{-}{-}{-}{-}}
\CommentTok{\# 8. ПОСТРОЕНИЕ ПРОСТРАНСТВЕННО{-}ВРЕМЕННОЙ МОДЕЛИ}
\CommentTok{\# {-}{-}{-}{-}{-}{-}{-}{-}{-}{-}{-}{-}{-}{-}{-}{-}{-}{-}{-}{-}{-}{-}{-}{-}{-}{-}{-}{-}{-}{-}{-}{-}{-}{-}{-}{-}{-}{-}{-}{-}{-}{-}{-}{-}{-}{-}{-}{-}{-}{-}{-}}

\NormalTok{m }\OtherTok{\textless{}{-}} \FunctionTok{sdmTMB}\NormalTok{(}
  \AttributeTok{data =}\NormalTok{ data, }
  \AttributeTok{formula =}\NormalTok{ Density }\SpecialCharTok{\textasciitilde{}} \DecValTok{0} \SpecialCharTok{+} \FunctionTok{as.factor}\NormalTok{(YEAR),  }\CommentTok{\# Формула: плотность зависит от года}
  \AttributeTok{time =} \StringTok{"YEAR"}\NormalTok{,         }\CommentTok{\# Временная переменная}
  \AttributeTok{mesh =}\NormalTok{ mesh\_sdm,       }\CommentTok{\# Пространственная сетка}
  \AttributeTok{family =} \FunctionTok{tweedie}\NormalTok{(}\AttributeTok{link =} \StringTok{"log"}\NormalTok{),  }\CommentTok{\# Статистическое распределение}
  \AttributeTok{spatial =} \StringTok{"on"}\NormalTok{,        }\CommentTok{\# Включение пространственных эффектов}
  \AttributeTok{spatiotemporal =} \StringTok{"iid"} \CommentTok{\# Пространственно{-}временные эффекты}
\NormalTok{)}

\CommentTok{\# Вывод результатов модели}
\FunctionTok{summary}\NormalTok{(m)}
\FunctionTok{AIC}\NormalTok{(m)  }\CommentTok{\# Критерий Акаике}
\FunctionTok{sanity}\NormalTok{(m)  }\CommentTok{\# Проверка корректности модели}
\end{Highlighting}
\end{Shaded}

Получили результаты:

\begin{Shaded}
\begin{Highlighting}[]
\SpecialCharTok{\textgreater{}} \CommentTok{\# Вывод результатов модели}
\ErrorTok{\textgreater{}} \FunctionTok{summary}\NormalTok{(m)}
\NormalTok{Spatiotemporal model fit by ML [}\StringTok{\textquotesingle{}sdmTMB\textquotesingle{}}\NormalTok{]}
\NormalTok{Formula}\SpecialCharTok{:}\NormalTok{ Density }\SpecialCharTok{\textasciitilde{}} \DecValTok{0} \SpecialCharTok{+} \FunctionTok{as.factor}\NormalTok{(YEAR)}
\NormalTok{Mesh}\SpecialCharTok{:} \FunctionTok{mesh\_sdm}\NormalTok{ (isotropic covariance)}
\NormalTok{Time column}\SpecialCharTok{:}\NormalTok{ YEAR}
\NormalTok{Data}\SpecialCharTok{:}\NormalTok{ data}
\NormalTok{Family}\SpecialCharTok{:} \FunctionTok{tweedie}\NormalTok{(}\AttributeTok{link =} \StringTok{\textquotesingle{}log\textquotesingle{}}\NormalTok{)}
 
\NormalTok{Conditional model}\SpecialCharTok{:}
\NormalTok{                    coef.est coef.se}
\FunctionTok{as.factor}\NormalTok{(YEAR)}\DecValTok{2019}     \FloatTok{2.15}    \FloatTok{0.72}
\FunctionTok{as.factor}\NormalTok{(YEAR)}\DecValTok{2020}     \FloatTok{1.49}    \FloatTok{0.76}
\FunctionTok{as.factor}\NormalTok{(YEAR)}\DecValTok{2021}     \FloatTok{1.74}    \FloatTok{0.76}
\FunctionTok{as.factor}\NormalTok{(YEAR)}\DecValTok{2022}     \FloatTok{1.62}    \FloatTok{0.73}
\FunctionTok{as.factor}\NormalTok{(YEAR)}\DecValTok{2023}     \FloatTok{1.50}    \FloatTok{0.74}
\FunctionTok{as.factor}\NormalTok{(YEAR)}\DecValTok{2024}     \FloatTok{1.56}    \FloatTok{0.72}

\NormalTok{Dispersion parameter}\SpecialCharTok{:} \FloatTok{19.71}
\NormalTok{Tweedie p}\SpecialCharTok{:} \FloatTok{1.50}
\NormalTok{Matern range}\SpecialCharTok{:} \FloatTok{142.65}
\NormalTok{Spatial SD}\SpecialCharTok{:} \FloatTok{2.01}
\NormalTok{Spatiotemporal IID SD}\SpecialCharTok{:} \FloatTok{0.95}
\NormalTok{ML criterion at convergence}\SpecialCharTok{:} \FloatTok{5984.224}

\NormalTok{See ?tidy.sdmTMB to extract these values as a data frame.}
\SpecialCharTok{\textgreater{}} \FunctionTok{AIC}\NormalTok{(m)  }\CommentTok{\# Критерий Акаике}
\NormalTok{[}\DecValTok{1}\NormalTok{] }\FloatTok{11990.45}
\SpecialCharTok{\textgreater{}} \FunctionTok{sanity}\NormalTok{(m)  }\CommentTok{\# Проверка корректности модели}
\NormalTok{v Non}\SpecialCharTok{{-}}\NormalTok{linear minimizer suggests successful convergence}
\NormalTok{v Hessian matrix is positive definite}
\NormalTok{v No extreme or very small eigenvalues detected}
\NormalTok{v No gradients with respect to fixed effects are }\SpecialCharTok{\textgreater{}=} \FloatTok{0.001}
\NormalTok{v No fixed}\SpecialCharTok{{-}}\NormalTok{effect standard errors are }\ConstantTok{NA}
\NormalTok{v No standard errors look unreasonably large}
\NormalTok{v No sigma parameters are }\SpecialCharTok{\textless{}} \FloatTok{0.01}
\NormalTok{v No sigma parameters are }\SpecialCharTok{\textgreater{}} \DecValTok{100}
\NormalTok{v Range parameter doesn}\StringTok{\textquotesingle{}t look unreasonably large}
\end{Highlighting}
\end{Shaded}

\subsubsection{\texorpdfstring{\textbf{Годовые
эффекты:}}{Годовые эффекты:}}\label{ux433ux43eux434ux43eux432ux44bux435-ux44dux444ux444ux435ux43aux442ux44b}

\begin{verbatim}
2019: 2.15 ± 0.72 → exp(2.15) ≈ 8.58 экз./км²
2020: 1.49 ± 0.76 → exp(1.49) ≈ 4.44 экз./км²
2024: 1.56 ± 0.72 → exp(1.56) ≈ 4.76 экз./км²
\end{verbatim}

\begin{itemize}
\item
  \textbf{2019 год} - пик запаса (8.58 экз./км²)
\item
  \textbf{2020 год} - резкое снижение (-52\% к 2019)
\item
  \textbf{2021-2024} - стабилизация на уровне \textasciitilde4.5-5.0
  экз./км²
\item
  \textbf{Стандартные ошибки} \textasciitilde0.75:

  \begin{itemize}
  \item
    Приемлемая точность для данных такого объема
  \item
    Все годовые оценки статистически значимы
  \end{itemize}

  Модель пространственно-временного распределения плотности камчатского
  краба успешно прошла все диагностические проверки, демонстрируя
  отличную сходимость и статистическую надежность. Параметр
  распределения Твиди (p=1.50) оптимально соответствует данным траловых
  съемок, учитывая характерную для уловов передисперсию и избыток
  нулевых значений.

  Годовые оценки показывают выраженную динамику запаса: в 2019 году
  зафиксирован пик плотности (8.58 экз./км²), после чего в 2020 году
  произошло резкое снижение до 4.44 экз./км². В последующие годы
  (2021-2024) плотность стабилизировалась на уровне 4.5-5.0 экз./км²,
  что составляет примерно 55\% от максимальных значений 2019 года.
  Стандартные ошибки годовых коэффициентов (0.72-0.76) свидетельствуют о
  хорошей точности оценок при текущем объеме данных.

  Пространственная структура распределения характеризуется
  крупномасштабными скоплениями с диапазоном корреляции 143 км (Matern
  range: 142.65), что согласуется с известными особенностями миграций
  камчатского краба. Высокое значение пространственной изменчивости
  (SD=2.01) отражает типичную для вида мозаичность распределения, где
  участки высокой плотности соседствуют с зонами отсутствия особей.
  Умеренная пространственно-временная изменчивость (IID SD=0.95)
  указывает на относительную стабильность пространственной структуры
  запаса между годами.

  Параметр дисперсии (19.71) подтверждает ожидаемо высокую
  вариабельность данных, характерную для траловых съемок морских
  гидробионтов. Полученные результаты надежно фиксируют значительное
  сокращение запаса после 2019 года с последующей стабилизацией на
  пониженном уровне.

  \subsection{\texorpdfstring{\textbf{Пояснение результатов
  \texttt{sanity(m)} для начинающих гидробиологов (от
  DeepSeek):}}{Пояснение результатов sanity(m) для начинающих гидробиологов (от DeepSeek):}}\label{ux43fux43eux44fux441ux43dux435ux43dux438ux435-ux440ux435ux437ux443ux43bux44cux442ux430ux442ux43eux432-sanitym-ux434ux43bux44f-ux43dux430ux447ux438ux43dux430ux44eux449ux438ux445-ux433ux438ux434ux440ux43eux431ux438ux43eux43bux43eux433ux43eux432-ux43eux442-deepseek}

  \textbf{1.
  \texttt{v\ Non-linear\ minimizer\ suggests\ successful\ convergence}}\\
  (Нелинейный оптимизатор успешно сошелся)\\
  \emph{Пояснение:} Алгоритм поиска параметров модели корректно завершил
  работу. Это значит, что модель ``научилась'' описывать ваши данные и
  не застряла в промежуточных вычислениях. Как если бы вы успешно
  завершили лабораторный анализ без технических сбоев.

  \textbf{2. \texttt{v\ Hessian\ matrix\ is\ positive\ definite}}\\
  (Матрица Гессе положительно определена)\\
  \emph{Пояснение:} Математическое подтверждение, что найденные
  параметры модели действительно оптимальны. Аналогично тому, как в
  микроскопии вы видите четкий фокус - здесь модель ``четко видит''
  закономерности в данных.

  \textbf{3.
  \texttt{v\ No\ extreme\ or\ very\ small\ eigenvalues\ detected}}\\
  (Не обнаружено экстремальных или очень маленьких собственных
  значений)\\
  \emph{Пояснение:} Модель статистически стабильна. Представьте, что вы
  измеряете длину рыб - если бы ваш штангенциркуль иногда показывал 0
  или 1000 мм, это было бы проблемой. Здесь аналогично - вычисления
  надежны.

  \textbf{4.
  \texttt{v\ No\ gradients\ with\ respect\ to\ fixed\ effects\ are\ \textgreater{}=\ 0.001}}\\
  (Градиенты для фиксированных эффектов \textless{} 0.001)\\
  \emph{Пояснение:} Все ключевые параметры модели (например, влияние
  года на плотность) рассчитаны точно. Это как убедиться, что все
  измерения в вашем эксперименте выполнены с требуемой точностью (±0.1
  мг, ±1 см и т.д.).

  \textbf{5. \texttt{v\ No\ fixed-effect\ standard\ errors\ are\ NA}}\\
  (Стандартные ошибки для фиксированных эффектов не отсутствуют)\\
  \emph{Пояснение:} Для каждого рассчитанного параметра (например,
  годовых оценок) указана погрешность. Важно как в химическом анализе -
  если для концентрации вещества нет погрешности, результат ненадежен.

  \textbf{6.
  \texttt{v\ No\ standard\ errors\ look\ unreasonably\ large}}\\
  (Стандартные ошибки выглядят разумными)\\
  \emph{Пояснение:} Погрешности оценок адекватны. Например, если
  плотность краба 5±1 экз./км² - это нормально, но 5±100 экз./км² было
  бы бессмысленным.

  \textbf{7.
  \texttt{v\ No\ sigma\ parameters\ are\ \textless{}\ 0.01}}\\
  (Параметры сигма не меньше 0.01)\\
  \emph{Пояснение:} Модель не игнорирует важные источники изменчивости.
  Аналогично тому, что в пробе воды вы не упустили бы важный показатель,
  сказав ``он слишком мал''.

  \textbf{8.
  \texttt{v\ No\ sigma\ parameters\ are\ \textgreater{}\ 100}}\\
  (Параметры сигма не превышают 100)\\
  \emph{Пояснение:} Модель не преувеличивает случайные вариации. Как
  если бы вы не приписали естественные колебания температуры воды
  катастрофическому изменению климата.

  \textbf{9.
  \texttt{v\ Range\ parameter\ doesn\textquotesingle{}t\ look\ unreasonably\ large}}\\
  (Параметр диапазона не выглядит чрезмерно большим)\\
  \emph{Пояснение:} Пространственная автокорреляция имеет биологически
  осмысленный масштаб. Например, если модель показала бы, что скопления
  краба одинаковы на расстоянии 1000 км - это было бы нереалистично.
\end{itemize}

\begin{Shaded}
\begin{Highlighting}[]
\CommentTok{\# {-}{-}{-}{-}{-}{-}{-}{-}{-}{-}{-}{-}{-}{-}{-}{-}{-}{-}{-}{-}{-}{-}{-}{-}{-}{-}{-}{-}{-}{-}{-}{-}{-}{-}{-}{-}{-}{-}{-}{-}{-}{-}{-}{-}{-}{-}{-}{-}{-}{-}{-}}
\CommentTok{\# 9. ДИАГНОСТИКА МОДЕЛИ}
\CommentTok{\# {-}{-}{-}{-}{-}{-}{-}{-}{-}{-}{-}{-}{-}{-}{-}{-}{-}{-}{-}{-}{-}{-}{-}{-}{-}{-}{-}{-}{-}{-}{-}{-}{-}{-}{-}{-}{-}{-}{-}{-}{-}{-}{-}{-}{-}{-}{-}{-}{-}{-}{-}}

\CommentTok{\# Расчет остатков модели}
\NormalTok{data}\SpecialCharTok{$}\NormalTok{resids }\OtherTok{\textless{}{-}} \FunctionTok{residuals}\NormalTok{(m) }

\CommentTok{\# Гистограмма остатков}
\FunctionTok{hist}\NormalTok{(data}\SpecialCharTok{$}\NormalTok{resids)}

\CommentTok{\# График квантиль{-}квантиль}
\FunctionTok{qqnorm}\NormalTok{(data}\SpecialCharTok{$}\NormalTok{resids)}
\FunctionTok{abline}\NormalTok{(}\AttributeTok{a =} \DecValTok{0}\NormalTok{, }\AttributeTok{b =} \DecValTok{1}\NormalTok{)}
\end{Highlighting}
\end{Shaded}

\begin{figure}[H]

{\centering \includegraphics[width=0.6\linewidth,height=\textheight,keepaspectratio]{images/sdmTMB5.PNG}

}

\caption{Рис. 5.: Гистограмма остатков}

\end{figure}%

\begin{figure}[H]

{\centering \includegraphics[width=0.6\linewidth,height=\textheight,keepaspectratio]{images/sdmTMB6.PNG}

}

\caption{Рис. 6.: График квантиль-квантиль}

\end{figure}%

\begin{Shaded}
\begin{Highlighting}[]
\CommentTok{\# {-}{-}{-}{-}{-}{-}{-}{-}{-}{-}{-}{-}{-}{-}{-}{-}{-}{-}{-}{-}{-}{-}{-}{-}{-}{-}{-}{-}{-}{-}{-}{-}{-}{-}{-}{-}{-}{-}{-}{-}{-}{-}{-}{-}{-}{-}{-}{-}{-}{-}{-}}
\CommentTok{\# 10. ПРОГНОЗИРОВАНИЕ НА СЕТКЕ}
\CommentTok{\# {-}{-}{-}{-}{-}{-}{-}{-}{-}{-}{-}{-}{-}{-}{-}{-}{-}{-}{-}{-}{-}{-}{-}{-}{-}{-}{-}{-}{-}{-}{-}{-}{-}{-}{-}{-}{-}{-}{-}{-}{-}{-}{-}{-}{-}{-}{-}{-}{-}{-}{-}}

\CommentTok{\# Прогноз значений плотности на сетке}
\NormalTok{predictions }\OtherTok{\textless{}{-}} \FunctionTok{predict}\NormalTok{(m, }\AttributeTok{newdata =}\NormalTok{ grid, }\AttributeTok{return\_tmb\_object =} \ConstantTok{TRUE}\NormalTok{)}
\NormalTok{RASP }\OtherTok{\textless{}{-}}\NormalTok{ predictions}\SpecialCharTok{$}\NormalTok{data}

\CommentTok{\# Преобразование координат обратно в широту/долготу}
\NormalTok{RASP}\SpecialCharTok{$}\NormalTok{xkm\_m }\OtherTok{\textless{}{-}}\NormalTok{ RASP}\SpecialCharTok{$}\NormalTok{xkm }\SpecialCharTok{*} \DecValTok{1000}  \CommentTok{\# Обратно в метры}
\NormalTok{RASP}\SpecialCharTok{$}\NormalTok{ykm\_m }\OtherTok{\textless{}{-}}\NormalTok{ RASP}\SpecialCharTok{$}\NormalTok{ykm }\SpecialCharTok{*} \DecValTok{1000}

\CommentTok{\# Создание пространственного объекта в UTM}
\NormalTok{utm\_proj }\OtherTok{\textless{}{-}} \FunctionTok{CRS}\NormalTok{(}\StringTok{"+proj=utm +zone=37 +datum=WGS84 +units=m +no\_defs"}\NormalTok{)}
\NormalTok{coords }\OtherTok{\textless{}{-}} \FunctionTok{cbind}\NormalTok{(RASP}\SpecialCharTok{$}\NormalTok{xkm\_m, RASP}\SpecialCharTok{$}\NormalTok{ykm\_m)}
\NormalTok{sp\_points }\OtherTok{\textless{}{-}} \FunctionTok{SpatialPoints}\NormalTok{(coords, }\AttributeTok{proj4string =}\NormalTok{ utm\_proj)}

\CommentTok{\# Преобразование в WGS84 (широта/долгота)}
\NormalTok{wgs84\_proj }\OtherTok{\textless{}{-}} \FunctionTok{CRS}\NormalTok{(}\StringTok{"+proj=longlat +datum=WGS84"}\NormalTok{)}
\NormalTok{sp\_points\_latlon }\OtherTok{\textless{}{-}} \FunctionTok{spTransform}\NormalTok{(sp\_points, wgs84\_proj)}

\CommentTok{\# Добавление координат в основной датафрейм}
\NormalTok{RASP}\SpecialCharTok{$}\NormalTok{X }\OtherTok{\textless{}{-}} \FunctionTok{coordinates}\NormalTok{(sp\_points\_latlon)[, }\DecValTok{1}\NormalTok{]  }\CommentTok{\# Долгота}
\NormalTok{RASP}\SpecialCharTok{$}\NormalTok{Y }\OtherTok{\textless{}{-}} \FunctionTok{coordinates}\NormalTok{(sp\_points\_latlon)[, }\DecValTok{2}\NormalTok{]  }\CommentTok{\# Широта}

\CommentTok{\# Удаление временных столбцов}
\NormalTok{RASP}\SpecialCharTok{$}\NormalTok{xkm\_m }\OtherTok{\textless{}{-}} \ConstantTok{NULL}
\NormalTok{RASP}\SpecialCharTok{$}\NormalTok{ykm\_m }\OtherTok{\textless{}{-}} \ConstantTok{NULL}

\CommentTok{\# Проверка структуры результата}
\FunctionTok{str}\NormalTok{(RASP)}

\CommentTok{\# {-}{-}{-}{-}{-}{-}{-}{-}{-}{-}{-}{-}{-}{-}{-}{-}{-}{-}{-}{-}{-}{-}{-}{-}{-}{-}{-}{-}{-}{-}{-}{-}{-}{-}{-}{-}{-}{-}{-}{-}{-}{-}{-}{-}{-}}
\CommentTok{\# 11. ВИЗУАЛИЗАЦИЯ РЕЗУЛЬТАТОВ (КАРТА)}
\CommentTok{\# {-}{-}{-}{-}{-}{-}{-}{-}{-}{-}{-}{-}{-}{-}{-}{-}{-}{-}{-}{-}{-}{-}{-}{-}{-}{-}{-}{-}{-}{-}{-}{-}{-}{-}{-}{-}{-}{-}{-}{-}{-}{-}{-}{-}{-}}

\CommentTok{\# Загрузка картографических данных}
\NormalTok{world }\OtherTok{\textless{}{-}} \FunctionTok{ne\_countries}\NormalTok{(}\AttributeTok{scale =} \StringTok{"medium"}\NormalTok{, }\AttributeTok{returnclass =} \StringTok{"sf"}\NormalTok{)}

\CommentTok{\# Определение региона интереса (Арктика России)}
\NormalTok{arctic\_bbox }\OtherTok{\textless{}{-}} \FunctionTok{st\_bbox}\NormalTok{(}\FunctionTok{c}\NormalTok{(}\AttributeTok{xmin =} \DecValTok{25}\NormalTok{, }\AttributeTok{xmax =} \DecValTok{70}\NormalTok{, }\AttributeTok{ymin =} \DecValTok{65}\NormalTok{, }\AttributeTok{ymax =} \DecValTok{80}\NormalTok{), }\AttributeTok{crs =} \DecValTok{4326}\NormalTok{)}
\NormalTok{arctic }\OtherTok{\textless{}{-}} \FunctionTok{st\_crop}\NormalTok{(world, arctic\_bbox)}

\CommentTok{\# Кастомные разрывы для цветовой шкалы}
\NormalTok{my\_breaks }\OtherTok{\textless{}{-}} \FunctionTok{c}\NormalTok{(}\FloatTok{0.001}\NormalTok{, }\FloatTok{0.1}\NormalTok{, }\DecValTok{1}\NormalTok{, }\DecValTok{200}\NormalTok{, }\DecValTok{10000}\NormalTok{)}

\CommentTok{\# Создание основной визуализации}
\FunctionTok{ggplot}\NormalTok{() }\SpecialCharTok{+}
  \CommentTok{\# Теплокарта плотности}
  \FunctionTok{geom\_point}\NormalTok{(}
    \AttributeTok{data =}\NormalTok{ RASP, }
    \FunctionTok{aes}\NormalTok{(}\AttributeTok{x =}\NormalTok{ X, }\AttributeTok{y =}\NormalTok{ Y, }\AttributeTok{color =} \FunctionTok{exp}\NormalTok{(est)), }
    \AttributeTok{size =} \FloatTok{0.8}\NormalTok{, }
    \AttributeTok{alpha =} \FloatTok{0.7}
\NormalTok{  ) }\SpecialCharTok{+} 
  \CommentTok{\# Наблюдаемые точки данных}
  \FunctionTok{geom\_point}\NormalTok{(}
    \AttributeTok{data =}\NormalTok{ data, }
    \FunctionTok{aes}\NormalTok{(}\AttributeTok{x =}\NormalTok{ X, }\AttributeTok{y =}\NormalTok{ Y, }\AttributeTok{size =}\NormalTok{ PROM), }\CommentTok{\# Размер по плотности}
    \AttributeTok{color =} \StringTok{"black"}\NormalTok{, }
    \AttributeTok{fill =} \ConstantTok{NA}\NormalTok{, }
    \AttributeTok{alpha =} \FloatTok{0.6}\NormalTok{,}
    \AttributeTok{shape =} \DecValTok{21} \CommentTok{\# Кружки с обводкой}
\NormalTok{  ) }\SpecialCharTok{+}
  \CommentTok{\# Картографическая подложка}
  \FunctionTok{geom\_sf}\NormalTok{(}\AttributeTok{data =}\NormalTok{ arctic, }\AttributeTok{fill =} \StringTok{"lightgrey"}\NormalTok{, }\AttributeTok{color =} \StringTok{"darkgrey"}\NormalTok{) }\SpecialCharTok{+}
  \CommentTok{\# Цветовая шкала (логарифмическая)}
  \FunctionTok{scale\_color\_viridis\_c}\NormalTok{(}
    \AttributeTok{name =} \StringTok{""}\NormalTok{,}
    \AttributeTok{option =} \StringTok{"H"}\NormalTok{, }
    \AttributeTok{trans =} \StringTok{"log"}\NormalTok{, }
    \AttributeTok{breaks =}\NormalTok{ my\_breaks, }
    \AttributeTok{labels =}\NormalTok{ my\_breaks}
\NormalTok{  ) }\SpecialCharTok{+}
  \CommentTok{\# Разделение по годам}
  \FunctionTok{facet\_wrap}\NormalTok{(}\SpecialCharTok{\textasciitilde{}}\NormalTok{ YEAR, }\AttributeTok{ncol =} \DecValTok{2}\NormalTok{) }\SpecialCharTok{+}
  \CommentTok{\# Настройка области просмотра}
  \FunctionTok{coord\_sf}\NormalTok{(}
    \AttributeTok{xlim =} \FunctionTok{c}\NormalTok{(}\FunctionTok{min}\NormalTok{(RASP}\SpecialCharTok{$}\NormalTok{X)}\SpecialCharTok{{-}}\DecValTok{1}\NormalTok{, }\FunctionTok{max}\NormalTok{(RASP}\SpecialCharTok{$}\NormalTok{X)}\SpecialCharTok{+}\DecValTok{1}\NormalTok{),}
    \AttributeTok{ylim =} \FunctionTok{c}\NormalTok{(}\FunctionTok{min}\NormalTok{(RASP}\SpecialCharTok{$}\NormalTok{Y)}\SpecialCharTok{{-}}\FloatTok{0.5}\NormalTok{, }\FunctionTok{max}\NormalTok{(RASP}\SpecialCharTok{$}\NormalTok{Y)}\SpecialCharTok{+}\FloatTok{0.5}\NormalTok{),}
    \AttributeTok{crs =} \DecValTok{4326}
\NormalTok{  ) }\SpecialCharTok{+}
  \CommentTok{\# Тема оформления}
  \FunctionTok{theme\_bw}\NormalTok{(}\AttributeTok{base\_size =} \DecValTok{12}\NormalTok{) }\SpecialCharTok{+}
  \FunctionTok{labs}\NormalTok{(}\AttributeTok{x =} \StringTok{"Долгота"}\NormalTok{, }\AttributeTok{y =} \StringTok{"Широта"}\NormalTok{, }\AttributeTok{title =} \StringTok{"Пространственное распределение плотности"}\NormalTok{) }\SpecialCharTok{+}
  \FunctionTok{theme}\NormalTok{(}
    \AttributeTok{panel.grid =} \FunctionTok{element\_line}\NormalTok{(}\AttributeTok{color =} \StringTok{"grey90"}\NormalTok{),}
    \AttributeTok{legend.position =} \StringTok{"bottom"}\NormalTok{,}
    \AttributeTok{legend.key.width =} \FunctionTok{unit}\NormalTok{(}\FloatTok{1.2}\NormalTok{, }\StringTok{"cm"}\NormalTok{),}
    \AttributeTok{strip.background =} \FunctionTok{element\_rect}\NormalTok{(}\AttributeTok{fill =} \StringTok{"white"}\NormalTok{)}
\NormalTok{  )}

\CommentTok{\# Сохранение графика (раскомментируйте)}
\CommentTok{\# ggsave("sdmTMBmap10.jpg", width = 8, height = 8, dpi = 300)}
\end{Highlighting}
\end{Shaded}

\begin{figure}[H]

{\centering \includegraphics[width=0.9\linewidth,height=\textheight,keepaspectratio]{images/sdmTMBmap10.jpg}

}

\caption{Рис. 7.: Визуализация результатов (КАРТА)}

\end{figure}%

\begin{Shaded}
\begin{Highlighting}[]
\CommentTok{\# {-}{-}{-}{-}{-}{-}{-}{-}{-}{-}{-}{-}{-}{-}{-}{-}{-}{-}{-}{-}{-}{-}{-}{-}{-}{-}{-}{-}{-}{-}{-}{-}{-}{-}{-}{-}{-}{-}{-}{-}{-}{-}{-}{-}{-}{-}{-}{-}{-}{-}{-}}
\CommentTok{\# 12. РАСЧЕТ ИНДЕКСОВ ОБИЛИЯ}
\CommentTok{\# {-}{-}{-}{-}{-}{-}{-}{-}{-}{-}{-}{-}{-}{-}{-}{-}{-}{-}{-}{-}{-}{-}{-}{-}{-}{-}{-}{-}{-}{-}{-}{-}{-}{-}{-}{-}{-}{-}{-}{-}{-}{-}{-}{-}{-}{-}{-}{-}{-}{-}{-}}

\CommentTok{\# Расчет индексов с разными доверительными интервалами}
\NormalTok{index }\OtherTok{\textless{}{-}} \FunctionTok{get\_index}\NormalTok{(predictions, }\AttributeTok{area =} \DecValTok{4}\NormalTok{, }\AttributeTok{level =} \FloatTok{0.95}\NormalTok{, }\AttributeTok{bias\_correct =} \ConstantTok{TRUE}\NormalTok{)}
\NormalTok{index2 }\OtherTok{\textless{}{-}} \FunctionTok{get\_index}\NormalTok{(predictions, }\AttributeTok{area =} \DecValTok{4}\NormalTok{, }\AttributeTok{level =} \FloatTok{0.5}\NormalTok{, }\AttributeTok{bias\_correct =} \ConstantTok{TRUE}\NormalTok{)}

\CommentTok{\# Формирование сводной таблицы результатов}
\NormalTok{total }\OtherTok{\textless{}{-}} \FunctionTok{data.frame}\NormalTok{(}
  \AttributeTok{YEAR =}\NormalTok{ index}\SpecialCharTok{$}\NormalTok{YEAR,}
  \AttributeTok{lwr\_95 =}\NormalTok{ index}\SpecialCharTok{$}\NormalTok{lwr,}
  \AttributeTok{lwr\_50 =}\NormalTok{ index2}\SpecialCharTok{$}\NormalTok{lwr,}
  \AttributeTok{estimate =}\NormalTok{ index}\SpecialCharTok{$}\NormalTok{est,}
  \AttributeTok{upr\_50 =}\NormalTok{ index2}\SpecialCharTok{$}\NormalTok{upr,}
  \AttributeTok{upr\_95 =}\NormalTok{ index}\SpecialCharTok{$}\NormalTok{upr,}
  \AttributeTok{se =}\NormalTok{ index}\SpecialCharTok{$}\NormalTok{se,}
  \AttributeTok{cv =} \FunctionTok{sqrt}\NormalTok{(}\FunctionTok{exp}\NormalTok{(index}\SpecialCharTok{$}\NormalTok{se}\SpecialCharTok{\^{}}\DecValTok{2}\NormalTok{) }\SpecialCharTok{{-}} \DecValTok{1}\NormalTok{) }\CommentTok{\# Коэффициент вариации}
\NormalTok{)}

\CommentTok{\# Визуализация индексов обилия}
\FunctionTok{ggplot}\NormalTok{(total, }\FunctionTok{aes}\NormalTok{(}\AttributeTok{x =}\NormalTok{ YEAR, }\AttributeTok{y =}\NormalTok{ estimate}\SpecialCharTok{/}\DecValTok{1000000}\NormalTok{)) }\SpecialCharTok{+} 
  \CommentTok{\# Основная линия оценки}
  \FunctionTok{geom\_line}\NormalTok{(}\AttributeTok{linewidth =} \DecValTok{1}\NormalTok{, }\AttributeTok{color =} \StringTok{"steelblue"}\NormalTok{) }\SpecialCharTok{+}
  
  \CommentTok{\# 95\% доверительный интервал (более широкий и прозрачный)}
  \FunctionTok{geom\_ribbon}\NormalTok{(}
    \FunctionTok{aes}\NormalTok{(}\AttributeTok{ymin =}\NormalTok{ lwr\_95}\SpecialCharTok{/}\DecValTok{1000000}\NormalTok{, }\AttributeTok{ymax =}\NormalTok{ upr\_95}\SpecialCharTok{/}\DecValTok{1000000}\NormalTok{),}
    \AttributeTok{alpha =} \FloatTok{0.2}\NormalTok{,  }\CommentTok{\# Полупрозрачность}
    \AttributeTok{fill =} \StringTok{"steelblue"}\NormalTok{,}
    \AttributeTok{color =} \ConstantTok{NA}     \CommentTok{\# Без контура}
\NormalTok{  ) }\SpecialCharTok{+}
  
  \CommentTok{\# 50\% доверительный интервал (менее прозрачный)}
  \FunctionTok{geom\_ribbon}\NormalTok{(}
    \FunctionTok{aes}\NormalTok{(}\AttributeTok{ymin =}\NormalTok{ lwr\_50}\SpecialCharTok{/}\DecValTok{1000000}\NormalTok{, }\AttributeTok{ymax =}\NormalTok{ upr\_50}\SpecialCharTok{/}\DecValTok{1000000}\NormalTok{),}
    \AttributeTok{alpha =} \FloatTok{0.4}\NormalTok{,  }\CommentTok{\# Меньшая прозрачность}
    \AttributeTok{fill =} \StringTok{"steelblue"}\NormalTok{,}
    \AttributeTok{color =} \ConstantTok{NA}
\NormalTok{  ) }\SpecialCharTok{+}
  
  \CommentTok{\# Настройки осей и заголовков}
  \FunctionTok{ylab}\NormalTok{(}\StringTok{\textquotesingle{}Промысловый запас, млн. экз\textquotesingle{}}\NormalTok{) }\SpecialCharTok{+}
  \FunctionTok{xlab}\NormalTok{(}\StringTok{\textquotesingle{}Год\textquotesingle{}}\NormalTok{) }\SpecialCharTok{+}
  
  \CommentTok{\# Вертикальные линии для годов}
  \FunctionTok{geom\_vline}\NormalTok{(}
    \AttributeTok{xintercept =}\NormalTok{ total}\SpecialCharTok{$}\NormalTok{YEAR, }
    \AttributeTok{linetype =} \StringTok{"dotted"}\NormalTok{, }
    \AttributeTok{color =} \StringTok{"grey60"}\NormalTok{, }
    \AttributeTok{alpha =} \FloatTok{0.6}
\NormalTok{  ) }\SpecialCharTok{+}
  
  \CommentTok{\# Точки с значениями оценок}
  \FunctionTok{geom\_point}\NormalTok{(}
    \AttributeTok{size =} \DecValTok{3}\NormalTok{,}
    \AttributeTok{color =} \StringTok{"navyblue"}\NormalTok{,}
    \AttributeTok{fill =} \StringTok{"white"}\NormalTok{,}
    \AttributeTok{shape =} \DecValTok{21}
\NormalTok{  ) }\SpecialCharTok{+}
  
  \CommentTok{\# Настройка темы}
  \FunctionTok{theme\_minimal}\NormalTok{(}\AttributeTok{base\_size =} \DecValTok{14}\NormalTok{) }\SpecialCharTok{+}
  \FunctionTok{theme}\NormalTok{(}
    \AttributeTok{plot.title =} \FunctionTok{element\_text}\NormalTok{(}\AttributeTok{hjust =} \FloatTok{0.5}\NormalTok{, }\AttributeTok{face =} \StringTok{"bold"}\NormalTok{),}
    \AttributeTok{panel.grid.minor =} \FunctionTok{element\_blank}\NormalTok{(),}
    \AttributeTok{panel.grid.major =} \FunctionTok{element\_line}\NormalTok{(}\AttributeTok{color =} \StringTok{"grey90"}\NormalTok{),}
    \AttributeTok{axis.line =} \FunctionTok{element\_line}\NormalTok{(}\AttributeTok{color =} \StringTok{"grey30"}\NormalTok{),}
    \AttributeTok{legend.position =} \StringTok{"none"}
\NormalTok{  )}
\end{Highlighting}
\end{Shaded}

\begin{figure}[H]

{\centering \includegraphics[width=0.6\linewidth,height=\textheight,keepaspectratio]{images/sdmTMB7.PNG}

}

\caption{Рис. 8.: Визуализация индексов обилия}

\end{figure}%

\begin{Shaded}
\begin{Highlighting}[]
\CommentTok{\# Форматированный вывод результатов}
\NormalTok{total }\SpecialCharTok{\%\textgreater{}\%} 
  \FunctionTok{mutate}\NormalTok{(}\AttributeTok{cv\_percent =} \DecValTok{100} \SpecialCharTok{*}\NormalTok{ cv) }\SpecialCharTok{\%\textgreater{}\%} 
  \FunctionTok{select}\NormalTok{(}
\NormalTok{    YEAR, }
\NormalTok{    estimate, }
\NormalTok{    lwr\_50,  }\CommentTok{\# Нижняя граница 50\% ДИ}
\NormalTok{    upr\_50,  }\CommentTok{\# Верхняя граница 50\% ДИ}
\NormalTok{    lwr\_95,  }\CommentTok{\# Нижняя граница 95\% ДИ}
\NormalTok{    upr\_95,  }\CommentTok{\# Верхняя граница 95\% ДИ}
\NormalTok{    cv\_percent}
\NormalTok{  ) }\SpecialCharTok{\%\textgreater{}\%}
\NormalTok{  knitr}\SpecialCharTok{::}\FunctionTok{kable}\NormalTok{(}
    \AttributeTok{format =} \StringTok{"pandoc"}\NormalTok{, }
    \AttributeTok{digits =} \FunctionTok{c}\NormalTok{(}\DecValTok{0}\NormalTok{, }\DecValTok{0}\NormalTok{, }\DecValTok{0}\NormalTok{, }\DecValTok{0}\NormalTok{, }\DecValTok{0}\NormalTok{, }\DecValTok{0}\NormalTok{, }\DecValTok{1}\NormalTok{),}
    \AttributeTok{col.names =} \FunctionTok{c}\NormalTok{(}
      \StringTok{"Год"}\NormalTok{, }
      \StringTok{"Оценка"}\NormalTok{, }
      \StringTok{"Нижняя 50\%"}\NormalTok{, }
      \StringTok{"Верхняя 50\%"}\NormalTok{, }
      \StringTok{"Нижняя 95\%"}\NormalTok{, }
      \StringTok{"Верхняя 95\%"}\NormalTok{, }
      \StringTok{"CV\%"}
\NormalTok{    )}
\NormalTok{  )}
\end{Highlighting}
\end{Shaded}

\begin{Shaded}
\begin{Highlighting}[]
\NormalTok{  Год    Оценка   Нижняя }\DecValTok{50}\SpecialCharTok{\%   Верхняя 50\%}\NormalTok{   Нижняя }\DecValTok{95}\SpecialCharTok{\%   Верхняя 95\%}\NormalTok{    CV\%}
\SpecialCharTok{{-}{-}{-}{-}{-}}  \SpecialCharTok{{-}{-}{-}{-}{-}{-}{-}{-}}  \SpecialCharTok{{-}{-}{-}{-}{-}{-}{-}{-}{-}{-}{-}}  \SpecialCharTok{{-}{-}{-}{-}{-}{-}{-}{-}{-}{-}{-}{-}}  \SpecialCharTok{{-}{-}{-}{-}{-}{-}{-}{-}{-}{-}{-}}  \SpecialCharTok{{-}{-}{-}{-}{-}{-}{-}{-}{-}{-}{-}{-}}  \SpecialCharTok{{-}{-}{-}{-}{-}}
 \DecValTok{2019}   \DecValTok{2381774}      \DecValTok{2177448}       \DecValTok{2605274}      \DecValTok{1835312}       \DecValTok{3090946}   \FloatTok{13.4}
 \DecValTok{2020}   \DecValTok{1634549}      \DecValTok{1539111}       \DecValTok{1735906}      \DecValTok{1372377}       \DecValTok{1946805}    \FloatTok{8.9}
 \DecValTok{2021}   \DecValTok{1920507}      \DecValTok{1794122}       \DecValTok{2055795}      \DecValTok{1575823}       \DecValTok{2340584}   \FloatTok{10.1}
 \DecValTok{2022}   \DecValTok{1036673}       \DecValTok{959251}       \DecValTok{1120344}       \DecValTok{827345}       \DecValTok{1298963}   \FloatTok{11.5}
 \DecValTok{2023}   \DecValTok{1147685}      \DecValTok{1068401}       \DecValTok{1232853}       \DecValTok{932147}       \DecValTok{1413062}   \FloatTok{10.6}
 \DecValTok{2024}   \DecValTok{1055733}       \DecValTok{985624}       \DecValTok{1130829}       \DecValTok{864640}       \DecValTok{1289060}   \FloatTok{10.2}
\SpecialCharTok{\textgreater{}} 
\end{Highlighting}
\end{Shaded}

\section{Базовая оценка +
предикторы}\label{ux431ux430ux437ux43eux432ux430ux44f-ux43eux446ux435ux43dux43aux430-ux43fux440ux435ux434ux438ux43aux442ux43eux440ux44b}

Сравнение пространственно-временных моделей sdmTMB с учетом типа съемки
(SURV) и года (YEAR)

Рассмотрим 4 пространственно-временные модели, оценивая их по:

Качеству подгонки (AIC) Стабильности оценок (sanity check) Значимости
ковариат Биологическому смыслу

4 модели: базовая модель, модель с глубиной (DEPTH),модель с
температурой (TEMP), модель с расстоянием до берега (DIST)

\begin{Shaded}
\begin{Highlighting}[]
\SpecialCharTok{\textgreater{}} \CommentTok{\# 8. ПОСТРОЕНИЕ ПРОСТРАНСТВЕННО{-}ВРЕМЕННОЙ МОДЕЛИ}
\ErrorTok{\textgreater{}} \CommentTok{\# {-}{-}{-}{-}{-}{-}{-}{-}{-}{-}{-}{-}{-}{-}{-}{-}{-}{-}{-}{-}{-}{-}{-}{-}{-}{-}{-}{-}{-}{-}{-}{-}{-}{-}{-}{-}{-}{-}{-}{-}{-}{-}{-}{-}{-}{-}{-}{-}{-}{-}{-}}
\ErrorTok{\textgreater{}} 
\ErrorTok{\textgreater{}}\NormalTok{ m }\OtherTok{\textless{}{-}} \FunctionTok{sdmTMB}\NormalTok{(}
\SpecialCharTok{+}   \AttributeTok{data =}\NormalTok{ data, }
\SpecialCharTok{+}   \AttributeTok{formula =}\NormalTok{ Density }\SpecialCharTok{\textasciitilde{}} \DecValTok{0}\SpecialCharTok{+} \FunctionTok{as.factor}\NormalTok{(SURV) }\SpecialCharTok{+} \FunctionTok{as.factor}\NormalTok{(YEAR),  }\CommentTok{\# Формула: плотность зависит от года}
\SpecialCharTok{+}   \AttributeTok{time =} \StringTok{"YEAR"}\NormalTok{,         }\CommentTok{\# Временная переменная}
\SpecialCharTok{+}   \AttributeTok{mesh =}\NormalTok{ mesh\_sdm,       }\CommentTok{\# Пространственная сетка}
\SpecialCharTok{+}   \AttributeTok{family =} \FunctionTok{tweedie}\NormalTok{(}\AttributeTok{link =} \StringTok{"log"}\NormalTok{),  }\CommentTok{\# Статистическое распределение}
\SpecialCharTok{+}   \AttributeTok{spatial =} \StringTok{"on"}\NormalTok{,        }\CommentTok{\# Включение пространственных эффектов}
\SpecialCharTok{+}   \AttributeTok{spatiotemporal =} \StringTok{"iid"} \CommentTok{\# Пространственно{-}временные эффекты}
\SpecialCharTok{+}\NormalTok{ )}
\SpecialCharTok{\textgreater{}} 
\ErrorTok{\textgreater{}} 
\ErrorTok{\textgreater{}} \CommentTok{\# Вывод результатов модели}
\ErrorTok{\textgreater{}} \FunctionTok{summary}\NormalTok{(m)}
\NormalTok{Spatiotemporal model fit by ML [}\StringTok{\textquotesingle{}sdmTMB\textquotesingle{}}\NormalTok{]}
\NormalTok{Formula}\SpecialCharTok{:}\NormalTok{ Density }\SpecialCharTok{\textasciitilde{}} \DecValTok{0} \SpecialCharTok{+} \FunctionTok{as.factor}\NormalTok{(SURV) }\SpecialCharTok{+} \FunctionTok{as.factor}\NormalTok{(YEAR)}
\NormalTok{Mesh}\SpecialCharTok{:} \FunctionTok{mesh\_sdm}\NormalTok{ (isotropic covariance)}
\NormalTok{Time column}\SpecialCharTok{:}\NormalTok{ YEAR}
\NormalTok{Data}\SpecialCharTok{:}\NormalTok{ data}
\NormalTok{Family}\SpecialCharTok{:} \FunctionTok{tweedie}\NormalTok{(}\AttributeTok{link =} \StringTok{\textquotesingle{}log\textquotesingle{}}\NormalTok{)}
 
\NormalTok{Conditional model}\SpecialCharTok{:}
\NormalTok{                    coef.est coef.se}
\FunctionTok{as.factor}\NormalTok{(SURV)CRAB     }\FloatTok{4.75}    \FloatTok{0.44}
\FunctionTok{as.factor}\NormalTok{(SURV)SUM      }\FloatTok{2.54}    \FloatTok{0.37}
\FunctionTok{as.factor}\NormalTok{(YEAR)}\DecValTok{2020}    \SpecialCharTok{{-}}\FloatTok{0.57}    \FloatTok{0.36}
\FunctionTok{as.factor}\NormalTok{(YEAR)}\DecValTok{2021}    \SpecialCharTok{{-}}\FloatTok{0.20}    \FloatTok{0.36}
\FunctionTok{as.factor}\NormalTok{(YEAR)}\DecValTok{2022}    \SpecialCharTok{{-}}\FloatTok{0.61}    \FloatTok{0.36}
\FunctionTok{as.factor}\NormalTok{(YEAR)}\DecValTok{2023}    \SpecialCharTok{{-}}\FloatTok{0.59}    \FloatTok{0.36}
\FunctionTok{as.factor}\NormalTok{(YEAR)}\DecValTok{2024}    \SpecialCharTok{{-}}\FloatTok{0.85}    \FloatTok{0.36}

\NormalTok{Dispersion parameter}\SpecialCharTok{:} \FloatTok{23.16}
\NormalTok{Tweedie p}\SpecialCharTok{:} \FloatTok{1.41}
\NormalTok{Matern range}\SpecialCharTok{:} \FloatTok{63.53}
\NormalTok{Spatial SD}\SpecialCharTok{:} \FloatTok{1.22}
\NormalTok{Spatiotemporal IID SD}\SpecialCharTok{:} \FloatTok{0.97}
\NormalTok{ML criterion at convergence}\SpecialCharTok{:} \FloatTok{5914.655}

\NormalTok{See ?tidy.sdmTMB to extract these values as a data frame.}
\SpecialCharTok{\textgreater{}} \FunctionTok{AIC}\NormalTok{(m)  }\CommentTok{\# Критерий Акаике}
\NormalTok{[}\DecValTok{1}\NormalTok{] }\FloatTok{11853.31}
\SpecialCharTok{\textgreater{}} \FunctionTok{sanity}\NormalTok{(m)  }\CommentTok{\# Проверка корректности модели}
\NormalTok{v Non}\SpecialCharTok{{-}}\NormalTok{linear minimizer suggests successful convergence}
\NormalTok{v Hessian matrix is positive definite}
\NormalTok{v No extreme or very small eigenvalues detected}
\NormalTok{v No gradients with respect to fixed effects are }\SpecialCharTok{\textgreater{}=} \FloatTok{0.001}
\NormalTok{v No fixed}\SpecialCharTok{{-}}\NormalTok{effect standard errors are }\ConstantTok{NA}
\NormalTok{v No standard errors look unreasonably large}
\NormalTok{v No sigma parameters are }\SpecialCharTok{\textless{}} \FloatTok{0.01}
\NormalTok{v No sigma parameters are }\SpecialCharTok{\textgreater{}} \DecValTok{100}
\NormalTok{v Range parameter doesn}\StringTok{\textquotesingle{}t look unreasonably large}
\StringTok{\textgreater{} }
\StringTok{\textgreater{} md \textless{}{-} sdmTMB(}
\StringTok{+   data = data, }
\StringTok{+   formula = Density \textasciitilde{} 0+ as.factor(SURV) + as.factor(YEAR)+s(DEPTH),  \# Формула: плотность зависит от года}
\StringTok{+   time = "YEAR",         \# Временная переменная}
\StringTok{+   mesh = mesh\_sdm,       \# Пространственная сетка}
\StringTok{+   family = tweedie(link = "log"),  \# Статистическое распределение}
\StringTok{+   spatial = "on",        \# Включение пространственных эффектов}
\StringTok{+   spatiotemporal = "iid" \# Пространственно{-}временные эффекты}
\StringTok{+ )}
\StringTok{\textgreater{} }
\StringTok{\textgreater{} }
\StringTok{\textgreater{} \# Вывод результатов модели}
\StringTok{\textgreater{} summary(md)}
\StringTok{Spatiotemporal model fit by ML [\textquotesingle{}}\NormalTok{sdmTMB}\StringTok{\textquotesingle{}]}
\StringTok{Formula: Density \textasciitilde{} 0 + as.factor(SURV) + as.factor(YEAR) + s(DEPTH)}
\StringTok{Mesh: mesh\_sdm (isotropic covariance)}
\StringTok{Time column: YEAR}
\StringTok{Data: data}
\StringTok{Family: tweedie(link = \textquotesingle{}}\NormalTok{log}\StringTok{\textquotesingle{})}
\StringTok{ }
\StringTok{Conditional model:}
\StringTok{                    coef.est coef.se}
\StringTok{as.factor(SURV)CRAB     5.42    0.32}
\StringTok{as.factor(SURV)SUM      3.09    0.28}
\StringTok{as.factor(YEAR)2020    {-}0.52    0.27}
\StringTok{as.factor(YEAR)2021    {-}0.15    0.27}
\StringTok{as.factor(YEAR)2022    {-}0.67    0.27}
\StringTok{as.factor(YEAR)2023    {-}0.63    0.27}
\StringTok{as.factor(YEAR)2024    {-}0.93    0.27}
\StringTok{sDEPTH                 {-}0.60    0.42}

\StringTok{Smooth terms:}
\StringTok{           Std. Dev.}
\StringTok{sds(DEPTH)      1.71}

\StringTok{Dispersion parameter: 24.43}
\StringTok{Tweedie p: 1.39}
\StringTok{Matern range: 40.20}
\StringTok{Spatial SD: 0.97}
\StringTok{Spatiotemporal IID SD: 0.94}
\StringTok{ML criterion at convergence: 5907.365}

\StringTok{See ?tidy.sdmTMB to extract these values as a data frame.}
\StringTok{\textgreater{} AIC(md)  \# Критерий Акаике}
\StringTok{[1] 11842.73}
\StringTok{\textgreater{} sanity(md)  \# Проверка корректности модели}
\StringTok{v Non{-}linear minimizer suggests successful convergence}
\StringTok{v Hessian matrix is positive definite}
\StringTok{v No extreme or very small eigenvalues detected}
\StringTok{v No gradients with respect to fixed effects are \textgreater{}= 0.001}
\StringTok{v No fixed{-}effect standard errors are NA}
\StringTok{v No standard errors look unreasonably large}
\StringTok{v No sigma parameters are \textless{} 0.01}
\StringTok{v No sigma parameters are \textgreater{} 100}
\StringTok{v Range parameter doesn\textquotesingle{}}\NormalTok{t look unreasonably large}
\SpecialCharTok{\textgreater{}} 
\ErrorTok{\textgreater{}} 
\ErrorTok{\textgreater{}}\NormalTok{ mt }\OtherTok{\textless{}{-}} \FunctionTok{sdmTMB}\NormalTok{(}
\SpecialCharTok{+}   \AttributeTok{data =}\NormalTok{ data, }
\SpecialCharTok{+}   \AttributeTok{formula =}\NormalTok{ Density }\SpecialCharTok{\textasciitilde{}} \DecValTok{0}\SpecialCharTok{+} \FunctionTok{as.factor}\NormalTok{(SURV) }\SpecialCharTok{+} \FunctionTok{as.factor}\NormalTok{(YEAR)}\SpecialCharTok{+}\FunctionTok{s}\NormalTok{(TEMP),  }\CommentTok{\# Формула: плотность зависит от года}
\SpecialCharTok{+}   \AttributeTok{time =} \StringTok{"YEAR"}\NormalTok{,         }\CommentTok{\# Временная переменная}
\SpecialCharTok{+}   \AttributeTok{mesh =}\NormalTok{ mesh\_sdm,       }\CommentTok{\# Пространственная сетка}
\SpecialCharTok{+}   \AttributeTok{family =} \FunctionTok{tweedie}\NormalTok{(}\AttributeTok{link =} \StringTok{"log"}\NormalTok{),  }\CommentTok{\# Статистическое распределение}
\SpecialCharTok{+}   \AttributeTok{spatial =} \StringTok{"on"}\NormalTok{,        }\CommentTok{\# Включение пространственных эффектов}
\SpecialCharTok{+}   \AttributeTok{spatiotemporal =} \StringTok{"iid"} \CommentTok{\# Пространственно{-}временные эффекты}
\SpecialCharTok{+}\NormalTok{ )}
\SpecialCharTok{\textgreater{}} 
\ErrorTok{\textgreater{}} 
\ErrorTok{\textgreater{}} \CommentTok{\# Вывод результатов модели}
\ErrorTok{\textgreater{}} \FunctionTok{summary}\NormalTok{(mt)}
\NormalTok{Spatiotemporal model fit by ML [}\StringTok{\textquotesingle{}sdmTMB\textquotesingle{}}\NormalTok{]}
\NormalTok{Formula}\SpecialCharTok{:}\NormalTok{ Density }\SpecialCharTok{\textasciitilde{}} \DecValTok{0} \SpecialCharTok{+} \FunctionTok{as.factor}\NormalTok{(SURV) }\SpecialCharTok{+} \FunctionTok{as.factor}\NormalTok{(YEAR) }\SpecialCharTok{+} \FunctionTok{s}\NormalTok{(TEMP)}
\NormalTok{Mesh}\SpecialCharTok{:} \FunctionTok{mesh\_sdm}\NormalTok{ (isotropic covariance)}
\NormalTok{Time column}\SpecialCharTok{:}\NormalTok{ YEAR}
\NormalTok{Data}\SpecialCharTok{:}\NormalTok{ data}
\NormalTok{Family}\SpecialCharTok{:} \FunctionTok{tweedie}\NormalTok{(}\AttributeTok{link =} \StringTok{\textquotesingle{}log\textquotesingle{}}\NormalTok{)}
 
\NormalTok{Conditional model}\SpecialCharTok{:}
\NormalTok{                    coef.est coef.se}
\FunctionTok{as.factor}\NormalTok{(SURV)CRAB     }\FloatTok{4.95}    \FloatTok{0.40}
\FunctionTok{as.factor}\NormalTok{(SURV)SUM      }\FloatTok{2.69}    \FloatTok{0.34}
\FunctionTok{as.factor}\NormalTok{(YEAR)}\DecValTok{2020}    \SpecialCharTok{{-}}\FloatTok{0.14}    \FloatTok{0.43}
\FunctionTok{as.factor}\NormalTok{(YEAR)}\DecValTok{2021}    \SpecialCharTok{{-}}\FloatTok{0.19}    \FloatTok{0.33}
\FunctionTok{as.factor}\NormalTok{(YEAR)}\DecValTok{2022}    \SpecialCharTok{{-}}\FloatTok{0.77}    \FloatTok{0.42}
\FunctionTok{as.factor}\NormalTok{(YEAR)}\DecValTok{2023}    \SpecialCharTok{{-}}\FloatTok{0.62}    \FloatTok{0.33}
\FunctionTok{as.factor}\NormalTok{(YEAR)}\DecValTok{2024}    \SpecialCharTok{{-}}\FloatTok{0.91}    \FloatTok{0.34}
\NormalTok{sTEMP                   }\FloatTok{0.80}    \FloatTok{0.83}

\NormalTok{Smooth terms}\SpecialCharTok{:}
\NormalTok{          Std. Dev.}
\FunctionTok{sds}\NormalTok{(TEMP)      }\FloatTok{3.15}

\NormalTok{Dispersion parameter}\SpecialCharTok{:} \FloatTok{23.42}
\NormalTok{Tweedie p}\SpecialCharTok{:} \FloatTok{1.40}
\NormalTok{Matern range}\SpecialCharTok{:} \FloatTok{55.04}
\NormalTok{Spatial SD}\SpecialCharTok{:} \FloatTok{1.12}
\NormalTok{Spatiotemporal IID SD}\SpecialCharTok{:} \FloatTok{0.96}
\NormalTok{ML criterion at convergence}\SpecialCharTok{:} \FloatTok{5912.795}

\NormalTok{See ?tidy.sdmTMB to extract these values as a data frame.}
\SpecialCharTok{\textgreater{}} \FunctionTok{AIC}\NormalTok{(mt)  }\CommentTok{\# Критерий Акаике}
\NormalTok{[}\DecValTok{1}\NormalTok{] }\FloatTok{11853.59}
\SpecialCharTok{\textgreater{}} \FunctionTok{sanity}\NormalTok{(mt)  }\CommentTok{\# Проверка корректности модели}
\NormalTok{v Non}\SpecialCharTok{{-}}\NormalTok{linear minimizer suggests successful convergence}
\NormalTok{v Hessian matrix is positive definite}
\NormalTok{v No extreme or very small eigenvalues detected}
\NormalTok{v No gradients with respect to fixed effects are }\SpecialCharTok{\textgreater{}=} \FloatTok{0.001}
\NormalTok{v No fixed}\SpecialCharTok{{-}}\NormalTok{effect standard errors are }\ConstantTok{NA}
\NormalTok{v No standard errors look unreasonably large}
\NormalTok{v No sigma parameters are }\SpecialCharTok{\textless{}} \FloatTok{0.01}
\NormalTok{v No sigma parameters are }\SpecialCharTok{\textgreater{}} \DecValTok{100}
\NormalTok{v Range parameter doesn}\StringTok{\textquotesingle{}t look unreasonably large}
\StringTok{\textgreater{} }
\StringTok{\textgreater{} mdist \textless{}{-} sdmTMB(}
\StringTok{+   data = data, }
\StringTok{+   formula = Density \textasciitilde{} 0+ as.factor(SURV) + as.factor(YEAR)+s(DIST),  \# Формула: плотность зависит от года}
\StringTok{+   time = "YEAR",         \# Временная переменная}
\StringTok{+   mesh = mesh\_sdm,       \# Пространственная сетка}
\StringTok{+   family = tweedie(link = "log"),  \# Статистическое распределение}
\StringTok{+   spatial = "on",        \# Включение пространственных эффектов}
\StringTok{+   spatiotemporal = "iid" \# Пространственно{-}временные эффекты}
\StringTok{+ )}
\StringTok{\textgreater{} }
\StringTok{\textgreater{} }
\StringTok{\textgreater{} \# Вывод результатов модели}
\StringTok{\textgreater{} summary(mdist)}
\StringTok{Spatiotemporal model fit by ML [\textquotesingle{}}\NormalTok{sdmTMB}\StringTok{\textquotesingle{}]}
\StringTok{Formula: Density \textasciitilde{} 0 + as.factor(SURV) + as.factor(YEAR) + s(DIST)}
\StringTok{Mesh: mesh\_sdm (isotropic covariance)}
\StringTok{Time column: YEAR}
\StringTok{Data: data}
\StringTok{Family: tweedie(link = \textquotesingle{}}\NormalTok{log}\StringTok{\textquotesingle{})}
\StringTok{ }
\StringTok{Conditional model:}
\StringTok{                    coef.est coef.se}
\StringTok{as.factor(SURV)CRAB     4.74    0.44}
\StringTok{as.factor(SURV)SUM      2.55    0.37}
\StringTok{as.factor(YEAR)2020    {-}0.57    0.36}
\StringTok{as.factor(YEAR)2021    {-}0.20    0.36}
\StringTok{as.factor(YEAR)2022    {-}0.61    0.36}
\StringTok{as.factor(YEAR)2023    {-}0.60    0.36}
\StringTok{as.factor(YEAR)2024    {-}0.85    0.36}
\StringTok{sDIST                  {-}0.06    0.16}

\StringTok{Smooth terms:}
\StringTok{          Std. Dev.}
\StringTok{sds(DIST)         0}

\StringTok{Dispersion parameter: 23.11}
\StringTok{Tweedie p: 1.41}
\StringTok{Matern range: 63.83}
\StringTok{Spatial SD: 1.22}
\StringTok{Spatiotemporal IID SD: 0.97}
\StringTok{ML criterion at convergence: 5914.594}

\StringTok{See ?tidy.sdmTMB to extract these values as a data frame.}

\StringTok{**Possible issues detected! Check output of sanity().**}
\StringTok{\textgreater{} AIC(mdist)  \# Критерий Акаике}
\StringTok{[1] 11857.19}
\StringTok{\textgreater{} sanity(mdist)  \# Проверка корректности модели}
\StringTok{v Non{-}linear minimizer suggests successful convergence}
\StringTok{v Hessian matrix is positive definite}
\StringTok{v No extreme or very small eigenvalues detected}
\StringTok{v No gradients with respect to fixed effects are \textgreater{}= 0.001}
\StringTok{v No fixed{-}effect standard errors are NA}
\StringTok{x \textasciigrave{}ln\_smooth\_sigma\textasciigrave{} standard error may be large}
\StringTok{i Try simplifying the model, adjusting the mesh, or adding priors}

\StringTok{v No sigma parameters are \textless{} 0.01}
\StringTok{v No sigma parameters are \textgreater{} 100}
\StringTok{v Range parameter doesn\textquotesingle{}}\NormalTok{t look unreasonably large}
\SpecialCharTok{\textgreater{}} 
\end{Highlighting}
\end{Shaded}

\subsubsection{\texorpdfstring{\textbf{1. Базовая модель (SURV +
YEAR)}}{1. Базовая модель (SURV + YEAR)}}\label{ux431ux430ux437ux43eux432ux430ux44f-ux43cux43eux434ux435ux43bux44c-surv-year}

\begin{verbatim}
Density ~ 0 + as.factor(SURV) + as.factor(YEAR)
\end{verbatim}

\begin{itemize}
\item
  \textbf{AIC}: 11853.31
\item
  \textbf{Проверка стабильности}: Все параметры стабильны
\item
  \textbf{Ключевые эффекты}:

  \begin{itemize}
  \item
    Высокая плотность в съемках CRAB (коэф. 4.75)
  \item
    Снижение плотности во всех годах относительно базового уровня
    (2020-2024: -0.57 до -0.85)
  \end{itemize}
\item
  \textbf{Пространственные параметры}:

  \begin{itemize}
  \item
    Диапазон Матерна: 63.53 км
  \item
    Пространственная SD: 1.22
  \end{itemize}
\end{itemize}

\subsubsection{\texorpdfstring{\textbf{2. Модель с глубиной
(DEPTH)}}{2. Модель с глубиной (DEPTH)}}\label{ux43cux43eux434ux435ux43bux44c-ux441-ux433ux43bux443ux431ux438ux43dux43eux439-depth}

\begin{verbatim}
Density ~ 0 + as.factor(SURV) + as.factor(YEAR) + s(DEPTH)
\end{verbatim}

\begin{itemize}
\item
  \textbf{AIC}: 11842.73 (наилучший)
\item
  \textbf{Проверка стабильности}: Все параметры стабильны
\item
  \textbf{Ключевые эффекты}:

  \begin{itemize}
  \item
    Сильное отрицательное влияние глубины (коэф. -0.60, SE=0.42)
  \item
    Усиление контраста между съемками CRAB/SUM (CRAB: 5.42 vs SUM: 3.09)
  \end{itemize}
\item
  \textbf{Улучшения}:

  \begin{itemize}
  \item
    Снижение AIC на 10.58 пунктов
  \item
    Уменьшение пространственного диапазона (40.20 км)
  \end{itemize}
\item
  \textbf{Интерпретация}: Глубина --- значимый экологический фактор
  распределения
\end{itemize}

\subsubsection{\texorpdfstring{\textbf{3. Модель с температурой
(TEMP)}}{3. Модель с температурой (TEMP)}}\label{ux43cux43eux434ux435ux43bux44c-ux441-ux442ux435ux43cux43fux435ux440ux430ux442ux443ux440ux43eux439-temp}

\begin{verbatim}
Density ~ 0 + as.factor(SURV) + as.factor(YEAR) + s(TEMP)
\end{verbatim}

\begin{itemize}
\item
  \textbf{AIC}: 11853.59 (хуже базовой)
\item
  \textbf{Проверка стабильности}: Стабильна, но высокий SE сглаживания
\item
  \textbf{Ключевые эффекты}:

  \begin{itemize}
  \item
    Слабый положительный эффект температуры (коэф. 0.80, SE=0.83)
  \item
    Незначительное изменение годовых эффектов
  \end{itemize}
\item
  \textbf{Проблемы}: Минимальное улучшение модели, высокая
  неопределенность эффекта температуры
\end{itemize}

\subsubsection{\texorpdfstring{\textbf{4. Модель с расстоянием
(DIST)}}{4. Модель с расстоянием (DIST)}}\label{ux43cux43eux434ux435ux43bux44c-ux441-ux440ux430ux441ux441ux442ux43eux44fux43dux438ux435ux43c-dist}

\begin{verbatim}
Density ~ 0 + as.factor(SURV) + as.factor(YEAR) + s(DIST)
\end{verbatim}

\begin{itemize}
\item
  \textbf{AIC}: 11857.19 (наихудший)
\item
  \textbf{Проверка стабильности}: Проблемы со сглаживанием
\item
  \textbf{Ключевые эффекты}:

  \begin{itemize}
  \item
    Незначительный эффект расстояния (коэф. -0.06, SE=0.16)
  \item
    Практически идентична базовой модели
  \end{itemize}
\item
  \textbf{Проблемы}: Наихудший AIC, предупреждения о нестабильности
\end{itemize}

\subsection{\texorpdfstring{\textbf{Сводка сравнения
моделей}}{Сводка сравнения моделей}}\label{ux441ux432ux43eux434ux43aux430-ux441ux440ux430ux432ux43dux435ux43dux438ux44f-ux43cux43eux434ux435ux43bux435ux439}

\begin{longtable}[]{@{}
  >{\raggedright\arraybackslash}p{(\linewidth - 10\tabcolsep) * \real{0.1667}}
  >{\raggedright\arraybackslash}p{(\linewidth - 10\tabcolsep) * \real{0.1667}}
  >{\raggedright\arraybackslash}p{(\linewidth - 10\tabcolsep) * \real{0.1667}}
  >{\raggedright\arraybackslash}p{(\linewidth - 10\tabcolsep) * \real{0.1667}}
  >{\raggedright\arraybackslash}p{(\linewidth - 10\tabcolsep) * \real{0.1667}}
  >{\raggedright\arraybackslash}p{(\linewidth - 10\tabcolsep) * \real{0.1667}}@{}}
\toprule\noalign{}
\begin{minipage}[b]{\linewidth}\raggedright
\textbf{Модель}
\end{minipage} & \begin{minipage}[b]{\linewidth}\raggedright
\textbf{AIC}
\end{minipage} & \begin{minipage}[b]{\linewidth}\raggedright
\textbf{ΔAIC}
\end{minipage} & \begin{minipage}[b]{\linewidth}\raggedright
\textbf{Стабильность}
\end{minipage} & \begin{minipage}[b]{\linewidth}\raggedright
\textbf{Ключевой предиктор}
\end{minipage} & \begin{minipage}[b]{\linewidth}\raggedright
\textbf{Эффект ковариаты}
\end{minipage} \\
\midrule\noalign{}
\endhead
\bottomrule\noalign{}
\endlastfoot
\textbf{DEPTH} & 11842.73 & - & ✓✓✓ & Глубина & Сильный (-0.60) \\
Базовая & 11853.31 & +10.6 & ✓✓✓ & - & - \\
TEMP & 11853.59 & +10.9 & ✓✓ & Температура & Слабый (+0.80) \\
DIST & 11857.19 & +14.5 & ✗ & Расстояние & Незначительный (-0.06) \\
\end{longtable}

\subsection{\texorpdfstring{\textbf{Рекомендации}}{Рекомендации}}\label{ux440ux435ux43aux43eux43cux435ux43dux434ux430ux446ux438ux438-1}

\begin{enumerate}
\def\labelenumi{\arabic{enumi}.}
\item
  \textbf{Лучшая модель}: С глубиной (DEPTH)

  \begin{itemize}
  \item
    Значительное улучшение AIC (-10.58)
  \item
    Биологически интерпретируемый эффект (глубина --- ключевой фактор
    распределения краба)
  \item
    Стабильные оценки параметров
  \end{itemize}
\item
  \textbf{Практическое значение}:

  \begin{itemize}
  \item
    Глубина объясняет \textasciitilde12\% пространственной
    вариабельности (судя по изменению пространственной SD)
  \item
    Модель адекватно отражает экологические предпочтения вида
  \end{itemize}
\end{enumerate}

\begin{quote}
\textbf{Вывод}: Включение глубины как ковариаты существенно улучшает
модель, тогда как температура и расстояние не дают значимых улучшений.
\end{quote}

\section{Визуализация
эффектов}\label{ux432ux438ux437ux443ux430ux43bux438ux437ux430ux446ux438ux44f-ux44dux444ux444ux435ux43aux442ux43eux432}

Модель с глубиной - md (см. передыдущий скрипт)

\begin{Shaded}
\begin{Highlighting}[]
\CommentTok{\# {-}{-}{-}{-}{-}{-}{-}{-}{-}{-}{-}{-}{-}{-}{-}{-}{-}{-}{-}{-}{-}{-}{-}{-}{-}{-}{-}{-}{-}{-}{-}{-}{-}{-}{-}{-}{-}{-}{-}{-}{-}{-}{-}{-}{-}{-}{-}{-}{-}{-}{-}}
\CommentTok{\# 8.1. ВИЗУАЛИЗАЦИЯ ЭФФЕКТА ГЛУБИНЫ}
\CommentTok{\# {-}{-}{-}{-}{-}{-}{-}{-}{-}{-}{-}{-}{-}{-}{-}{-}{-}{-}{-}{-}{-}{-}{-}{-}{-}{-}{-}{-}{-}{-}{-}{-}{-}{-}{-}{-}{-}{-}{-}{-}{-}{-}{-}{-}{-}{-}{-}{-}{-}{-}{-}}

\CommentTok{\# Создаем новый датафрейм для предсказаний}
\NormalTok{newdata }\OtherTok{\textless{}{-}} \FunctionTok{expand.grid}\NormalTok{(}
  \AttributeTok{DEPTH =} \FunctionTok{seq}\NormalTok{(}\DecValTok{50}\NormalTok{, }\DecValTok{400}\NormalTok{, }\AttributeTok{by =} \DecValTok{2}\NormalTok{),}
  \AttributeTok{YEAR =} \DecValTok{2020}\NormalTok{,}
  \AttributeTok{SURV =} \StringTok{"CRAB"}\NormalTok{,}
  \AttributeTok{xkm =} \FunctionTok{mean}\NormalTok{(data}\SpecialCharTok{$}\NormalTok{xkm),}
  \AttributeTok{ykm =} \FunctionTok{mean}\NormalTok{(data}\SpecialCharTok{$}\NormalTok{ykm)}
\NormalTok{)}

\CommentTok{\# Делаем предсказания с расчетом стандартных ошибок}
\NormalTok{pred }\OtherTok{\textless{}{-}} \FunctionTok{predict}\NormalTok{(md, }\AttributeTok{newdata =}\NormalTok{ newdata, }\AttributeTok{re\_formula =} \ConstantTok{NA}\NormalTok{, }\AttributeTok{se\_fit =} \ConstantTok{TRUE}\NormalTok{)}


\CommentTok{\# Визуализируем эффект глубины}
\FunctionTok{ggplot}\NormalTok{(pred, }\FunctionTok{aes}\NormalTok{(}\AttributeTok{x =}\NormalTok{ DEPTH, }\AttributeTok{y =} \FunctionTok{exp}\NormalTok{(est))) }\SpecialCharTok{+}
  \FunctionTok{geom\_line}\NormalTok{(}\AttributeTok{linewidth =} \FloatTok{1.2}\NormalTok{, }\AttributeTok{color =} \StringTok{"blue4"}\NormalTok{) }\SpecialCharTok{+}
  \FunctionTok{geom\_ribbon}\NormalTok{(}
    \FunctionTok{aes}\NormalTok{(}
      \AttributeTok{ymin =} \FunctionTok{exp}\NormalTok{(est }\SpecialCharTok{{-}} \FloatTok{1.96} \SpecialCharTok{*}\NormalTok{ est\_se), }
      \AttributeTok{ymax =} \FunctionTok{exp}\NormalTok{(est }\SpecialCharTok{+} \FloatTok{1.96} \SpecialCharTok{*}\NormalTok{ est\_se)  }\CommentTok{\# Исправлено на se.fit}
\NormalTok{    ),}
    \AttributeTok{alpha =} \FloatTok{0.3}\NormalTok{, }
    \AttributeTok{fill =} \StringTok{"steelblue"}
\NormalTok{  ) }\SpecialCharTok{+}
  \FunctionTok{labs}\NormalTok{(}
    \AttributeTok{title =} \StringTok{"Эффект глубины на плотность краба"}\NormalTok{,}
    \AttributeTok{subtitle =} \StringTok{"Год: 2020, Тип съемки: CRAB"}\NormalTok{,}
    \AttributeTok{x =} \StringTok{"Глубина (м)"}\NormalTok{,}
    \AttributeTok{y =} \StringTok{"Предсказанная плотность (особей/км²)"}
\NormalTok{  ) }\SpecialCharTok{+}
  \FunctionTok{theme\_bw}\NormalTok{(}\AttributeTok{base\_size =} \DecValTok{14}\NormalTok{) }\SpecialCharTok{+}
  \FunctionTok{theme}\NormalTok{(}\AttributeTok{panel.grid.minor =} \FunctionTok{element\_blank}\NormalTok{())}
\end{Highlighting}
\end{Shaded}

\begin{figure}[H]

{\centering \includegraphics[width=0.6\linewidth,height=\textheight,keepaspectratio]{images/sdmTMB8.PNG}

}

\caption{Рис. 9.: Визуализация эффекта глубины на плотность краба}

\end{figure}%

\section{Карта с акцентом на нулевые
уловы}\label{ux43aux430ux440ux442ux430-ux441-ux430ux43aux446ux435ux43dux442ux43eux43c-ux43dux430-ux43dux443ux43bux435ux432ux44bux435-ux443ux43bux43eux432ux44b}

Повторяем базовую оценку, но меняем в карте нулевые уловы на крестики

\begin{figure}[H]

{\centering \includegraphics[width=0.9\linewidth,height=\textheight,keepaspectratio]{images/sdmTMBmapZero.jpg}

}

\caption{Рис. 10.: Визуализация результатов (КАРТА)}

\end{figure}%

Скрипт для карты с базовой оценкой

\begin{Shaded}
\begin{Highlighting}[]
\CommentTok{\# {-}{-}{-}{-}{-}{-}{-}{-}{-}{-}{-}{-}{-}{-}{-}{-}{-}{-}{-}{-}{-}{-}{-}{-}{-}{-}{-}}
\CommentTok{\# 1. ПОДГОТОВКА СРЕДЫ И ДАННЫХ}
\CommentTok{\# {-}{-}{-}{-}{-}{-}{-}{-}{-}{-}{-}{-}{-}{-}{-}{-}{-}{-}{-}{-}{-}{-}{-}{-}{-}{-}{-}}

\CommentTok{\# Очистка рабочей среды}
\FunctionTok{rm}\NormalTok{(}\AttributeTok{list =} \FunctionTok{ls}\NormalTok{())}

\CommentTok{\# Установка рабочей директории (замените на свою)}
\FunctionTok{setwd}\NormalTok{(}\StringTok{"C:/COMBINE/"}\NormalTok{)}

\CommentTok{\# Загрузка необходимых пакетов}
\FunctionTok{library}\NormalTok{(readxl)       }\CommentTok{\# Для чтения Excel{-}файлов}
\FunctionTok{library}\NormalTok{(ggplot2)      }\CommentTok{\# Визуализация данных}
\FunctionTok{library}\NormalTok{(dplyr)        }\CommentTok{\# Обработка данных}
\FunctionTok{library}\NormalTok{(PBSmapping)   }\CommentTok{\# Для работы с пространственными данными}
\FunctionTok{library}\NormalTok{(sdmTMB)       }\CommentTok{\# Пространственно{-}временное моделирование}
\FunctionTok{library}\NormalTok{(INLA)         }\CommentTok{\# Продвинутые пространственные модели}
\FunctionTok{library}\NormalTok{(sp)           }\CommentTok{\# Классы для пространственных данных}
\FunctionTok{library}\NormalTok{(sf)           }\CommentTok{\# Пространственные данные (современный формат)}
\FunctionTok{library}\NormalTok{(rnaturalearth) }\CommentTok{\# Загрузка картографических данных}

\CommentTok{\# Загрузка данных из Excel{-}файла}
\NormalTok{data }\OtherTok{\textless{}{-}}\NormalTok{ readxl}\SpecialCharTok{::}\FunctionTok{read\_excel}\NormalTok{(}\StringTok{"KARTOGRAPHIC.xlsx"}\NormalTok{, }\AttributeTok{sheet =} \StringTok{"SURVEY"}\NormalTok{)}

\CommentTok{\# Просмотр структуры данных}
\FunctionTok{str}\NormalTok{(data)}


\CommentTok{\# {-}{-}{-}{-}{-}{-}{-}{-}{-}{-}{-}{-}{-}{-}{-}{-}{-}{-}{-}{-}{-}{-}{-}{-}{-}{-}{-}{-}{-}{-}{-}{-}{-}{-}{-}{-}{-}{-}{-}{-}{-}{-}{-}{-}{-}{-}{-}{-}{-}{-}}
\CommentTok{\# 2. ПРЕОБРАЗОВАНИЕ КООРДИНАТ В ПРОЕКЦИЮ UTM (в км)}
\CommentTok{\# {-}{-}{-}{-}{-}{-}{-}{-}{-}{-}{-}{-}{-}{-}{-}{-}{-}{-}{-}{-}{-}{-}{-}{-}{-}{-}{-}{-}{-}{-}{-}{-}{-}{-}{-}{-}{-}{-}{-}{-}{-}{-}{-}{-}{-}{-}{-}{-}{-}{-}}

\CommentTok{\# Создание пространственного объекта из данных}
\NormalTok{data\_sf }\OtherTok{\textless{}{-}} \FunctionTok{st\_as\_sf}\NormalTok{(}
\NormalTok{  data, }
  \AttributeTok{coords =} \FunctionTok{c}\NormalTok{(}\StringTok{"X"}\NormalTok{, }\StringTok{"Y"}\NormalTok{), }\CommentTok{\# Указание столбцов с координатами}
  \AttributeTok{crs =} \DecValTok{4326}            \CommentTok{\# Система координат WGS84 (широта/долгота)}
\NormalTok{) }

\CommentTok{\# Преобразование в UTM зону 37N (метры)}
\NormalTok{data\_utm }\OtherTok{\textless{}{-}} \FunctionTok{st\_transform}\NormalTok{(data\_sf, }\AttributeTok{crs =} \DecValTok{32637}\NormalTok{) }

\CommentTok{\# Извлечение координат и перевод в километры}
\NormalTok{utm\_coords }\OtherTok{\textless{}{-}} \FunctionTok{st\_coordinates}\NormalTok{(data\_utm)}
\NormalTok{data}\SpecialCharTok{$}\NormalTok{xkm }\OtherTok{\textless{}{-}}\NormalTok{ utm\_coords[, }\DecValTok{1}\NormalTok{] }\SpecialCharTok{/} \DecValTok{1000}  \CommentTok{\# X в км}
\NormalTok{data}\SpecialCharTok{$}\NormalTok{ykm }\OtherTok{\textless{}{-}}\NormalTok{ utm\_coords[, }\DecValTok{2}\NormalTok{] }\SpecialCharTok{/} \DecValTok{1000}  \CommentTok{\# Y в км}

\CommentTok{\# Очистка временных объектов}
\FunctionTok{rm}\NormalTok{(data\_sf, data\_utm, utm\_coords)}

\CommentTok{\# {-}{-}{-}{-}{-}{-}{-}{-}{-}{-}{-}{-}{-}{-}{-}{-}{-}{-}{-}{-}{-}{-}{-}{-}{-}{-}{-}{-}{-}{-}{-}{-}{-}{-}{-}{-}{-}{-}{-}{-}{-}}
\CommentTok{\# 3. ОПРЕДЕЛЕНИЕ ГРАНИЦ ИССЛЕДОВАНИЯ}
\CommentTok{\# {-}{-}{-}{-}{-}{-}{-}{-}{-}{-}{-}{-}{-}{-}{-}{-}{-}{-}{-}{-}{-}{-}{-}{-}{-}{-}{-}{-}{-}{-}{-}{-}{-}{-}{-}{-}{-}{-}{-}{-}{-}}

\CommentTok{\# Вычисление границ исследовательского полигона}
\NormalTok{xl }\OtherTok{\textless{}{-}} \FunctionTok{c}\NormalTok{(}\FunctionTok{min}\NormalTok{(data}\SpecialCharTok{$}\NormalTok{xkm), }\FunctionTok{max}\NormalTok{(data}\SpecialCharTok{$}\NormalTok{xkm))  }\CommentTok{\# Границы по X}
\NormalTok{yl }\OtherTok{\textless{}{-}} \FunctionTok{c}\NormalTok{(}\FunctionTok{min}\NormalTok{(data}\SpecialCharTok{$}\NormalTok{ykm), }\FunctionTok{max}\NormalTok{(data}\SpecialCharTok{$}\NormalTok{ykm))  }\CommentTok{\# Границы по Y}

\CommentTok{\# {-}{-}{-}{-}{-}{-}{-}{-}{-}{-}{-}{-}{-}{-}{-}{-}{-}{-}{-}{-}{-}{-}{-}{-}{-}{-}{-}{-}{-}{-}{-}{-}{-}{-}{-}{-}{-}{-}{-}{-}}
\CommentTok{\# 4. СОЗДАНИЕ РАСТРОВОЙ СЕТКИ ДЛЯ МОДЕЛИ}
\CommentTok{\# {-}{-}{-}{-}{-}{-}{-}{-}{-}{-}{-}{-}{-}{-}{-}{-}{-}{-}{-}{-}{-}{-}{-}{-}{-}{-}{-}{-}{-}{-}{-}{-}{-}{-}{-}{-}{-}{-}{-}{-}}

\CommentTok{\# Создание равномерной сетки с шагом 10 км (для визуализации карты использовался шаг 2 км}
\NormalTok{GRID }\OtherTok{\textless{}{-}} \FunctionTok{makeGrid}\NormalTok{(}
  \AttributeTok{x =} \FunctionTok{seq}\NormalTok{(xl[}\DecValTok{1}\NormalTok{], xl[}\DecValTok{2}\NormalTok{], }\DecValTok{10}\NormalTok{), }
  \AttributeTok{y =} \FunctionTok{seq}\NormalTok{(yl[}\DecValTok{1}\NormalTok{], yl[}\DecValTok{2}\NormalTok{], }\DecValTok{10}\NormalTok{),}
  \AttributeTok{byrow =} \ConstantTok{FALSE}\NormalTok{,}
  \AttributeTok{projection =} \StringTok{"UTM"}\NormalTok{, }
  \AttributeTok{zone =} \DecValTok{37}
\NormalTok{)}

\CommentTok{\# Расчет центроидов ячеек сетки}
\NormalTok{GRID }\OtherTok{\textless{}{-}} \FunctionTok{calcCentroid}\NormalTok{(GRID, }\AttributeTok{rollup =} \DecValTok{3}\NormalTok{)}

\CommentTok{\# {-}{-}{-}{-}{-}{-}{-}{-}{-}{-}{-}{-}{-}{-}{-}{-}{-}{-}{-}{-}{-}{-}{-}{-}{-}{-}{-}{-}{-}{-}{-}{-}{-}{-}{-}{-}{-}{-}{-}{-}{-}{-}{-}{-}{-}{-}{-}{-}{-}{-}{-}{-}{-}{-}{-}{-}{-}{-}{-}}
\CommentTok{\# 5. ПОСТРОЕНИЕ ВЫПУКЛОЙ ОБОЛОЧКИ (CONVEX HULL) ДЛЯ ДАННЫХ}
\CommentTok{\# {-}{-}{-}{-}{-}{-}{-}{-}{-}{-}{-}{-}{-}{-}{-}{-}{-}{-}{-}{-}{-}{-}{-}{-}{-}{-}{-}{-}{-}{-}{-}{-}{-}{-}{-}{-}{-}{-}{-}{-}{-}{-}{-}{-}{-}{-}{-}{-}{-}{-}{-}{-}{-}{-}{-}{-}{-}{-}{-}}

\CommentTok{\# Создание выпуклой оболочки вокруг точек данных}
\NormalTok{Hull }\OtherTok{\textless{}{-}} \FunctionTok{inla.nonconvex.hull}\NormalTok{(}\FunctionTok{cbind}\NormalTok{(data}\SpecialCharTok{$}\NormalTok{xkm, data}\SpecialCharTok{$}\NormalTok{ykm), }\AttributeTok{convex =} \SpecialCharTok{{-}}\FloatTok{0.03}\NormalTok{)}

\CommentTok{\# Визуализация оболочки }
 \FunctionTok{plot}\NormalTok{(Hull)}

\CommentTok{\# Визуализация оболочки и точек съемок 2019{-}2024}
\FunctionTok{points}\NormalTok{(data}\SpecialCharTok{$}\NormalTok{xkm, data}\SpecialCharTok{$}\NormalTok{ykm, }\AttributeTok{pch=}\DecValTok{1}\NormalTok{, }\AttributeTok{cex=}\FloatTok{0.55}\NormalTok{,}\AttributeTok{col=}\StringTok{"black"}\NormalTok{)}


\CommentTok{\# Фильтрация сетки: оставляем только точки внутри оболочки}
\NormalTok{line }\OtherTok{\textless{}{-}}\NormalTok{ Hull}\SpecialCharTok{$}\NormalTok{loc[, }\DecValTok{1}\SpecialCharTok{:}\DecValTok{2}\NormalTok{] }\SpecialCharTok{\%\textgreater{}\%} \FunctionTok{as.data.frame}\NormalTok{()}
\FunctionTok{colnames}\NormalTok{(line) }\OtherTok{\textless{}{-}} \FunctionTok{c}\NormalTok{(}\StringTok{"X"}\NormalTok{, }\StringTok{"Y"}\NormalTok{)}
\NormalTok{GRID}\SpecialCharTok{$}\NormalTok{AREA }\OtherTok{\textless{}{-}} \FunctionTok{point.in.polygon}\NormalTok{(GRID}\SpecialCharTok{$}\NormalTok{X, GRID}\SpecialCharTok{$}\NormalTok{Y, line}\SpecialCharTok{$}\NormalTok{X, line}\SpecialCharTok{$}\NormalTok{Y)}
\NormalTok{GRID }\OtherTok{\textless{}{-}}\NormalTok{ GRID[GRID}\SpecialCharTok{$}\NormalTok{AREA }\SpecialCharTok{\textgreater{}} \FloatTok{0.1}\NormalTok{, }\FunctionTok{c}\NormalTok{(}\StringTok{"X"}\NormalTok{, }\StringTok{"Y"}\NormalTok{)]  }\CommentTok{\# Только внутренние точки}

\CommentTok{\# {-}{-}{-}{-}{-}{-}{-}{-}{-}{-}{-}{-}{-}{-}{-}{-}{-}{-}{-}{-}{-}{-}{-}{-}{-}{-}{-}{-}{-}{-}{-}{-}{-}{-}{-}{-}{-}{-}{-}{-}{-}{-}{-}{-}{-}{-}{-}{-}{-}}
\CommentTok{\# 6. ПОДГОТОВКА СЕТКИ ДЛЯ ПРОГНОЗИРОВАНИЯ}
\CommentTok{\# {-}{-}{-}{-}{-}{-}{-}{-}{-}{-}{-}{-}{-}{-}{-}{-}{-}{-}{-}{-}{-}{-}{-}{-}{-}{-}{-}{-}{-}{-}{-}{-}{-}{-}{-}{-}{-}{-}{-}{-}{-}{-}{-}{-}{-}{-}{-}{-}{-}}

\CommentTok{\# Создание временной сетки (для каждого года)}
\NormalTok{grid }\OtherTok{\textless{}{-}} \FunctionTok{replicate\_df}\NormalTok{(GRID, }\StringTok{"YEAR"}\NormalTok{, }\FunctionTok{unique}\NormalTok{(data}\SpecialCharTok{$}\NormalTok{YEAR))}
\FunctionTok{colnames}\NormalTok{(grid) }\OtherTok{\textless{}{-}} \FunctionTok{c}\NormalTok{(}\StringTok{"xkm"}\NormalTok{, }\StringTok{"ykm"}\NormalTok{, }\StringTok{"YEAR"}\NormalTok{)}
\NormalTok{grid}\SpecialCharTok{$}\NormalTok{SURV }\OtherTok{\textless{}{-}} \StringTok{"CRAB"}  \CommentTok{\# Добавляем информацию о типе съемки}

\CommentTok{\# Визуализация оболочки и сетки для прогнозирования (grid\}}
 \FunctionTok{plot}\NormalTok{(Hull)}
 \FunctionTok{points}\NormalTok{(grid}\SpecialCharTok{$}\NormalTok{xkm, grid}\SpecialCharTok{$}\NormalTok{ykm, }\AttributeTok{pch=}\DecValTok{1}\NormalTok{, }\AttributeTok{cex=}\FloatTok{0.55}\NormalTok{,}\AttributeTok{col=}\StringTok{"black"}\NormalTok{)}

\CommentTok{\# {-}{-}{-}{-}{-}{-}{-}{-}{-}{-}{-}{-}{-}{-}{-}{-}{-}{-}{-}{-}{-}{-}{-}{-}{-}{-}{-}{-}{-}{-}{-}{-}{-}{-}{-}{-}{-}{-}{-}{-}{-}{-}{-}{-}{-}{-}{-}{-}{-}{-}{-}}
\CommentTok{\# 7. ПОСТРОЕНИЕ ПРОСТРАНСТВЕННОЙ СЕТКИ (MESH)}
\CommentTok{\# {-}{-}{-}{-}{-}{-}{-}{-}{-}{-}{-}{-}{-}{-}{-}{-}{-}{-}{-}{-}{-}{-}{-}{-}{-}{-}{-}{-}{-}{-}{-}{-}{-}{-}{-}{-}{-}{-}{-}{-}{-}{-}{-}{-}{-}{-}{-}{-}{-}{-}{-}}

\CommentTok{\# Создание треугольной сетки для пространственного моделирования}
\NormalTok{mesh\_sdm }\OtherTok{\textless{}{-}} \FunctionTok{make\_mesh}\NormalTok{(}
\NormalTok{  data, }
  \FunctionTok{c}\NormalTok{(}\StringTok{"xkm"}\NormalTok{, }\StringTok{"ykm"}\NormalTok{),  }\CommentTok{\# Координаты}
  \AttributeTok{cutoff =} \DecValTok{10}        \CommentTok{\# Минимальное расстояние между узлами (км)}
\NormalTok{)}

\CommentTok{\# Визуализация сетки (раскомментируйте)}
 \FunctionTok{plot}\NormalTok{(mesh\_sdm)}

\CommentTok{\# {-}{-}{-}{-}{-}{-}{-}{-}{-}{-}{-}{-}{-}{-}{-}{-}{-}{-}{-}{-}{-}{-}{-}{-}{-}{-}{-}{-}{-}{-}{-}{-}{-}{-}{-}{-}{-}{-}{-}{-}{-}{-}{-}{-}{-}{-}{-}{-}{-}{-}{-}}
\CommentTok{\# 8. ПОСТРОЕНИЕ ПРОСТРАНСТВЕННО{-}ВРЕМЕННОЙ МОДЕЛИ}
\CommentTok{\# {-}{-}{-}{-}{-}{-}{-}{-}{-}{-}{-}{-}{-}{-}{-}{-}{-}{-}{-}{-}{-}{-}{-}{-}{-}{-}{-}{-}{-}{-}{-}{-}{-}{-}{-}{-}{-}{-}{-}{-}{-}{-}{-}{-}{-}{-}{-}{-}{-}{-}{-}}

\NormalTok{m }\OtherTok{\textless{}{-}} \FunctionTok{sdmTMB}\NormalTok{(}
  \AttributeTok{data =}\NormalTok{ data, }
  \AttributeTok{formula =}\NormalTok{ Density }\SpecialCharTok{\textasciitilde{}} \DecValTok{0} \SpecialCharTok{+} \FunctionTok{as.factor}\NormalTok{(YEAR),  }\CommentTok{\# Формула: плотность зависит от года}
  \AttributeTok{time =} \StringTok{"YEAR"}\NormalTok{,         }\CommentTok{\# Временная переменная}
  \AttributeTok{mesh =}\NormalTok{ mesh\_sdm,       }\CommentTok{\# Пространственная сетка}
  \AttributeTok{family =} \FunctionTok{tweedie}\NormalTok{(}\AttributeTok{link =} \StringTok{"log"}\NormalTok{),  }\CommentTok{\# Статистическое распределение}
  \AttributeTok{spatial =} \StringTok{"on"}\NormalTok{,        }\CommentTok{\# Включение пространственных эффектов}
  \AttributeTok{spatiotemporal =} \StringTok{"iid"} \CommentTok{\# Пространственно{-}временные эффекты}
\NormalTok{)}


\CommentTok{\# Вывод результатов модели}
\FunctionTok{summary}\NormalTok{(m)}
\FunctionTok{AIC}\NormalTok{(m)  }\CommentTok{\# Критерий Акаике}
\FunctionTok{sanity}\NormalTok{(m)  }\CommentTok{\# Проверка корректности модели}

\CommentTok{\# {-}{-}{-}{-}{-}{-}{-}{-}{-}{-}{-}{-}{-}{-}{-}{-}{-}{-}{-}{-}{-}{-}{-}{-}{-}{-}{-}{-}{-}{-}{-}{-}{-}{-}{-}{-}{-}{-}{-}{-}{-}{-}{-}{-}{-}{-}{-}{-}{-}{-}{-}}
\CommentTok{\# 9. ДИАГНОСТИКА МОДЕЛИ}
\CommentTok{\# {-}{-}{-}{-}{-}{-}{-}{-}{-}{-}{-}{-}{-}{-}{-}{-}{-}{-}{-}{-}{-}{-}{-}{-}{-}{-}{-}{-}{-}{-}{-}{-}{-}{-}{-}{-}{-}{-}{-}{-}{-}{-}{-}{-}{-}{-}{-}{-}{-}{-}{-}}

\CommentTok{\# Расчет остатков модели}
\NormalTok{data}\SpecialCharTok{$}\NormalTok{resids }\OtherTok{\textless{}{-}} \FunctionTok{residuals}\NormalTok{(m) }

\CommentTok{\# Гистограмма остатков}
\FunctionTok{hist}\NormalTok{(data}\SpecialCharTok{$}\NormalTok{resids)}

\CommentTok{\# График квантиль{-}квантиль}
\FunctionTok{qqnorm}\NormalTok{(data}\SpecialCharTok{$}\NormalTok{resids)}
\FunctionTok{abline}\NormalTok{(}\AttributeTok{a =} \DecValTok{0}\NormalTok{, }\AttributeTok{b =} \DecValTok{1}\NormalTok{)}

\CommentTok{\# {-}{-}{-}{-}{-}{-}{-}{-}{-}{-}{-}{-}{-}{-}{-}{-}{-}{-}{-}{-}{-}{-}{-}{-}{-}{-}{-}{-}{-}{-}{-}{-}{-}{-}{-}{-}{-}{-}{-}{-}{-}{-}{-}{-}{-}{-}{-}{-}{-}{-}{-}}
\CommentTok{\# 10. ПРОГНОЗИРОВАНИЕ НА СЕТКЕ}
\CommentTok{\# {-}{-}{-}{-}{-}{-}{-}{-}{-}{-}{-}{-}{-}{-}{-}{-}{-}{-}{-}{-}{-}{-}{-}{-}{-}{-}{-}{-}{-}{-}{-}{-}{-}{-}{-}{-}{-}{-}{-}{-}{-}{-}{-}{-}{-}{-}{-}{-}{-}{-}{-}}

\CommentTok{\# Прогноз значений плотности на сетке}
\NormalTok{predictions }\OtherTok{\textless{}{-}} \FunctionTok{predict}\NormalTok{(m, }\AttributeTok{newdata =}\NormalTok{ grid, }\AttributeTok{return\_tmb\_object =} \ConstantTok{TRUE}\NormalTok{)}
\NormalTok{RASP }\OtherTok{\textless{}{-}}\NormalTok{ predictions}\SpecialCharTok{$}\NormalTok{data}

\CommentTok{\# Преобразование координат обратно в широту/долготу}
\NormalTok{RASP}\SpecialCharTok{$}\NormalTok{xkm\_m }\OtherTok{\textless{}{-}}\NormalTok{ RASP}\SpecialCharTok{$}\NormalTok{xkm }\SpecialCharTok{*} \DecValTok{1000}  \CommentTok{\# Обратно в метры}
\NormalTok{RASP}\SpecialCharTok{$}\NormalTok{ykm\_m }\OtherTok{\textless{}{-}}\NormalTok{ RASP}\SpecialCharTok{$}\NormalTok{ykm }\SpecialCharTok{*} \DecValTok{1000}

\CommentTok{\# Создание пространственного объекта в UTM}
\NormalTok{utm\_proj }\OtherTok{\textless{}{-}} \FunctionTok{CRS}\NormalTok{(}\StringTok{"+proj=utm +zone=37 +datum=WGS84 +units=m +no\_defs"}\NormalTok{)}
\NormalTok{coords }\OtherTok{\textless{}{-}} \FunctionTok{cbind}\NormalTok{(RASP}\SpecialCharTok{$}\NormalTok{xkm\_m, RASP}\SpecialCharTok{$}\NormalTok{ykm\_m)}
\NormalTok{sp\_points }\OtherTok{\textless{}{-}} \FunctionTok{SpatialPoints}\NormalTok{(coords, }\AttributeTok{proj4string =}\NormalTok{ utm\_proj)}

\CommentTok{\# Преобразование в WGS84 (широта/долгота)}
\NormalTok{wgs84\_proj }\OtherTok{\textless{}{-}} \FunctionTok{CRS}\NormalTok{(}\StringTok{"+proj=longlat +datum=WGS84"}\NormalTok{)}
\NormalTok{sp\_points\_latlon }\OtherTok{\textless{}{-}} \FunctionTok{spTransform}\NormalTok{(sp\_points, wgs84\_proj)}

\CommentTok{\# Добавление координат в основной датафрейм}
\NormalTok{RASP}\SpecialCharTok{$}\NormalTok{X }\OtherTok{\textless{}{-}} \FunctionTok{coordinates}\NormalTok{(sp\_points\_latlon)[, }\DecValTok{1}\NormalTok{]  }\CommentTok{\# Долгота}
\NormalTok{RASP}\SpecialCharTok{$}\NormalTok{Y }\OtherTok{\textless{}{-}} \FunctionTok{coordinates}\NormalTok{(sp\_points\_latlon)[, }\DecValTok{2}\NormalTok{]  }\CommentTok{\# Широта}

\CommentTok{\# Удаление временных столбцов}
\NormalTok{RASP}\SpecialCharTok{$}\NormalTok{xkm\_m }\OtherTok{\textless{}{-}} \ConstantTok{NULL}
\NormalTok{RASP}\SpecialCharTok{$}\NormalTok{ykm\_m }\OtherTok{\textless{}{-}} \ConstantTok{NULL}

\CommentTok{\# Проверка структуры результата}
\FunctionTok{str}\NormalTok{(RASP)}

\CommentTok{\# {-}{-}{-}{-}{-}{-}{-}{-}{-}{-}{-}{-}{-}{-}{-}{-}{-}{-}{-}{-}{-}{-}{-}{-}{-}{-}{-}{-}{-}{-}{-}{-}{-}{-}{-}{-}{-}{-}{-}{-}{-}{-}{-}{-}{-}}
\CommentTok{\# 11. ВИЗУАЛИЗАЦИЯ РЕЗУЛЬТАТОВ (КАРТА)}
\CommentTok{\# {-}{-}{-}{-}{-}{-}{-}{-}{-}{-}{-}{-}{-}{-}{-}{-}{-}{-}{-}{-}{-}{-}{-}{-}{-}{-}{-}{-}{-}{-}{-}{-}{-}{-}{-}{-}{-}{-}{-}{-}{-}{-}{-}{-}{-}}

\CommentTok{\# Загрузка картографических данных}
\NormalTok{world }\OtherTok{\textless{}{-}} \FunctionTok{ne\_countries}\NormalTok{(}\AttributeTok{scale =} \StringTok{"medium"}\NormalTok{, }\AttributeTok{returnclass =} \StringTok{"sf"}\NormalTok{)}

\CommentTok{\# Определение региона интереса (Арктика России)}
\NormalTok{arctic\_bbox }\OtherTok{\textless{}{-}} \FunctionTok{st\_bbox}\NormalTok{(}\FunctionTok{c}\NormalTok{(}\AttributeTok{xmin =} \DecValTok{25}\NormalTok{, }\AttributeTok{xmax =} \DecValTok{70}\NormalTok{, }\AttributeTok{ymin =} \DecValTok{65}\NormalTok{, }\AttributeTok{ymax =} \DecValTok{80}\NormalTok{), }\AttributeTok{crs =} \DecValTok{4326}\NormalTok{)}
\NormalTok{arctic }\OtherTok{\textless{}{-}} \FunctionTok{st\_crop}\NormalTok{(world, arctic\_bbox)}

\CommentTok{\# Определяем кастомные breaks для шкалы}
\NormalTok{my\_breaks }\OtherTok{\textless{}{-}} \FunctionTok{c}\NormalTok{(}\FloatTok{0.001}\NormalTok{,}\FloatTok{0.1}\NormalTok{,}\DecValTok{1}\NormalTok{,  }\DecValTok{200}\NormalTok{, }\DecValTok{10000}\NormalTok{)}

\CommentTok{\# Создаем категории для PROM}
\NormalTok{data }\OtherTok{\textless{}{-}}\NormalTok{ data }\SpecialCharTok{\%\textgreater{}\%}
  \FunctionTok{mutate}\NormalTok{(}
    \AttributeTok{PROM\_cat =} \FunctionTok{case\_when}\NormalTok{(}
\NormalTok{      PROM }\SpecialCharTok{==} \DecValTok{0} \SpecialCharTok{\textasciitilde{}} \StringTok{"0"}\NormalTok{,}
\NormalTok{      PROM }\SpecialCharTok{\textgreater{}=} \DecValTok{1} \SpecialCharTok{\&}\NormalTok{ PROM }\SpecialCharTok{\textless{}} \DecValTok{10} \SpecialCharTok{\textasciitilde{}} \StringTok{"1{-}9"}\NormalTok{,}
\NormalTok{      PROM }\SpecialCharTok{\textgreater{}=} \DecValTok{10} \SpecialCharTok{\&}\NormalTok{ PROM }\SpecialCharTok{\textless{}} \DecValTok{100} \SpecialCharTok{\textasciitilde{}} \StringTok{"10{-}99"}\NormalTok{,}
\NormalTok{      PROM }\SpecialCharTok{\textgreater{}=} \DecValTok{100} \SpecialCharTok{\textasciitilde{}} \StringTok{"100+"}
\NormalTok{    ),}
    \AttributeTok{PROM\_cat =} \FunctionTok{factor}\NormalTok{(PROM\_cat, }\AttributeTok{levels =} \FunctionTok{c}\NormalTok{(}\StringTok{"0"}\NormalTok{, }\StringTok{"1{-}9"}\NormalTok{, }\StringTok{"10{-}99"}\NormalTok{, }\StringTok{"100+"}\NormalTok{)),}
    \AttributeTok{shape\_cat =} \FunctionTok{ifelse}\NormalTok{(PROM\_cat }\SpecialCharTok{==} \StringTok{"0"}\NormalTok{, }\StringTok{"zero"}\NormalTok{, }\StringTok{"non\_zero"}\NormalTok{)}
\NormalTok{  )}

\CommentTok{\# Обновляем график}
\FunctionTok{ggplot}\NormalTok{() }\SpecialCharTok{+}
  \FunctionTok{geom\_point}\NormalTok{(}
    \AttributeTok{data =}\NormalTok{ RASP, }
    \FunctionTok{aes}\NormalTok{(}\AttributeTok{x =}\NormalTok{ X, }\AttributeTok{y =}\NormalTok{ Y, }\AttributeTok{color =} \FunctionTok{exp}\NormalTok{(est)), }
    \AttributeTok{size =} \FloatTok{0.8}\NormalTok{, }
    \AttributeTok{alpha =} \FloatTok{0.7}
\NormalTok{  ) }\SpecialCharTok{+} 
  \FunctionTok{geom\_point}\NormalTok{(}
    \AttributeTok{data =}\NormalTok{ data, }
    \FunctionTok{aes}\NormalTok{(}\AttributeTok{x =}\NormalTok{ X, }\AttributeTok{y =}\NormalTok{ Y, }\AttributeTok{size =}\NormalTok{ PROM\_cat, }\AttributeTok{shape =}\NormalTok{ shape\_cat),}
    \AttributeTok{color =} \StringTok{"black"}\NormalTok{, }
    \AttributeTok{fill =} \ConstantTok{NA}\NormalTok{, }
    \AttributeTok{alpha =} \FloatTok{0.6}
\NormalTok{  ) }\SpecialCharTok{+}
  \FunctionTok{scale\_size\_manual}\NormalTok{(}
    \AttributeTok{name =} \ConstantTok{NULL}\NormalTok{,}
    \AttributeTok{values =} \FunctionTok{c}\NormalTok{(}\StringTok{"0"} \OtherTok{=} \DecValTok{1}\NormalTok{, }\StringTok{"1{-}9"} \OtherTok{=} \DecValTok{2}\NormalTok{, }\StringTok{"10{-}99"} \OtherTok{=} \DecValTok{3}\NormalTok{),}
    \AttributeTok{labels =} \FunctionTok{c}\NormalTok{(}\StringTok{"0"}\NormalTok{, }\StringTok{"10"}\NormalTok{, }\StringTok{"100"}\NormalTok{)}
\NormalTok{  ) }\SpecialCharTok{+}
  \FunctionTok{scale\_shape\_manual}\NormalTok{(}
    \AttributeTok{values =} \FunctionTok{c}\NormalTok{(}\StringTok{"zero"} \OtherTok{=} \DecValTok{4}\NormalTok{, }\StringTok{"non\_zero"} \OtherTok{=} \DecValTok{21}\NormalTok{),}
    \AttributeTok{guide =} \StringTok{"none"} \CommentTok{\# Скрываем легенду для формы}
\NormalTok{  ) }\SpecialCharTok{+}
  \FunctionTok{guides}\NormalTok{(}
    \AttributeTok{size =} \FunctionTok{guide\_legend}\NormalTok{(}
      \AttributeTok{override.aes =} \FunctionTok{list}\NormalTok{(}\AttributeTok{shape =} \FunctionTok{c}\NormalTok{(}\DecValTok{4}\NormalTok{, }\DecValTok{21}\NormalTok{, }\DecValTok{21}\NormalTok{)) }\CommentTok{\# Крестик только для первого элемента}
\NormalTok{    )}
\NormalTok{  ) }\SpecialCharTok{+}
  \FunctionTok{geom\_sf}\NormalTok{(}\AttributeTok{data =}\NormalTok{ arctic, }\AttributeTok{fill =} \StringTok{"lightgrey"}\NormalTok{, }\AttributeTok{color =} \StringTok{"darkgrey"}\NormalTok{) }\SpecialCharTok{+}
  \FunctionTok{scale\_color\_viridis\_c}\NormalTok{(}
    \AttributeTok{name =} \ConstantTok{NULL}\NormalTok{,}
    \AttributeTok{option =} \StringTok{"H"}\NormalTok{, }
    \AttributeTok{trans =} \StringTok{"log"}\NormalTok{, }
    \AttributeTok{breaks =}\NormalTok{ my\_breaks, }
    \AttributeTok{labels =}\NormalTok{ my\_breaks, }
    \AttributeTok{limits =} \FunctionTok{range}\NormalTok{(my\_breaks),}
    \AttributeTok{guide =} \FunctionTok{guide\_colorbar}\NormalTok{(}
      \AttributeTok{barwidth =} \FunctionTok{unit}\NormalTok{(}\DecValTok{5}\NormalTok{, }\StringTok{"cm"}\NormalTok{),}
      \AttributeTok{title.position =} \StringTok{"top"}\NormalTok{,}
      \AttributeTok{direction =} \StringTok{"horizontal"}
\NormalTok{    )}
\NormalTok{  ) }\SpecialCharTok{+}
  \FunctionTok{facet\_wrap}\NormalTok{(}\SpecialCharTok{\textasciitilde{}}\NormalTok{ YEAR, }\AttributeTok{ncol =} \DecValTok{2}\NormalTok{) }\SpecialCharTok{+}
  \FunctionTok{coord\_sf}\NormalTok{(}
    \AttributeTok{xlim =} \FunctionTok{c}\NormalTok{(}\FunctionTok{min}\NormalTok{(RASP}\SpecialCharTok{$}\NormalTok{X)}\SpecialCharTok{{-}}\DecValTok{1}\NormalTok{, }\FunctionTok{max}\NormalTok{(RASP}\SpecialCharTok{$}\NormalTok{X)}\SpecialCharTok{+}\DecValTok{1}\NormalTok{),}
    \AttributeTok{ylim =} \FunctionTok{c}\NormalTok{(}\FunctionTok{min}\NormalTok{(RASP}\SpecialCharTok{$}\NormalTok{Y)}\SpecialCharTok{{-}}\FloatTok{0.5}\NormalTok{, }\FunctionTok{max}\NormalTok{(RASP}\SpecialCharTok{$}\NormalTok{Y)}\SpecialCharTok{+}\FloatTok{0.5}\NormalTok{),}
    \AttributeTok{crs =} \DecValTok{4326}
\NormalTok{  ) }\SpecialCharTok{+}
  \FunctionTok{theme\_bw}\NormalTok{(}\AttributeTok{base\_size =} \DecValTok{12}\NormalTok{) }\SpecialCharTok{+}
  \FunctionTok{labs}\NormalTok{(}\AttributeTok{x =} \ConstantTok{NULL}\NormalTok{, }\AttributeTok{y =} \ConstantTok{NULL}\NormalTok{) }\SpecialCharTok{+}
  \FunctionTok{theme}\NormalTok{(}
    \AttributeTok{panel.grid =} \FunctionTok{element\_line}\NormalTok{(}\AttributeTok{color =} \StringTok{"grey90"}\NormalTok{),}
    \AttributeTok{legend.position =} \StringTok{"bottom"}\NormalTok{,}
    \AttributeTok{legend.key.width =} \FunctionTok{unit}\NormalTok{(}\FloatTok{1.2}\NormalTok{, }\StringTok{"cm"}\NormalTok{),}
    \AttributeTok{strip.background =} \FunctionTok{element\_rect}\NormalTok{(}\AttributeTok{fill =} \StringTok{"white"}\NormalTok{)}
\NormalTok{  )}


\CommentTok{\# Сохранение графика (раскомментируйте)}
\CommentTok{\# ggsave("sdmTMBmapZero.jpg", width = 8, height = 8, dpi = 300)}
\end{Highlighting}
\end{Shaded}

\section{Определение площади
съемки}\label{ux43eux43fux440ux435ux434ux435ux43bux435ux43dux438ux435-ux43fux43bux43eux449ux430ux434ux438-ux441ux44aux435ux43cux43aux438}

\begin{Shaded}
\begin{Highlighting}[]
\CommentTok{\# {-}{-}{-}{-}{-}{-}{-}{-}{-}{-}{-}{-}{-}{-}{-}{-}{-}{-}{-}{-}{-}{-}{-}{-}{-}{-}{-}}
\CommentTok{\# 1. ПОДГОТОВКА СРЕДЫ И ДАННЫХ}
\CommentTok{\# {-}{-}{-}{-}{-}{-}{-}{-}{-}{-}{-}{-}{-}{-}{-}{-}{-}{-}{-}{-}{-}{-}{-}{-}{-}{-}{-}}

\CommentTok{\# Очистка рабочей среды}
\FunctionTok{rm}\NormalTok{(}\AttributeTok{list =} \FunctionTok{ls}\NormalTok{())}

\CommentTok{\# Установка рабочей директории (замените на свою)}
\FunctionTok{setwd}\NormalTok{(}\StringTok{"C:/COMBINE/"}\NormalTok{)}

\CommentTok{\# Загрузка необходимых пакетов}
\FunctionTok{library}\NormalTok{(readxl)       }\CommentTok{\# Для чтения Excel{-}файлов}
\FunctionTok{library}\NormalTok{(PBSmapping)   }\CommentTok{\# Для работы с пространственными данными}
\FunctionTok{library}\NormalTok{(sdmTMB)       }\CommentTok{\# Пространственно{-}временное моделирование}
\FunctionTok{library}\NormalTok{(INLA)         }\CommentTok{\# Продвинутые пространственные модели}

\CommentTok{\# Загрузка данных из Excel{-}файла}
\NormalTok{data }\OtherTok{\textless{}{-}}\NormalTok{ readxl}\SpecialCharTok{::}\FunctionTok{read\_excel}\NormalTok{(}\StringTok{"KARTOGRAPHIC.xlsx"}\NormalTok{, }\AttributeTok{sheet =} \StringTok{"SURVEY"}\NormalTok{)}

\CommentTok{\# Просмотр структуры данных}
\FunctionTok{str}\NormalTok{(data)}


\CommentTok{\# {-}{-}{-}{-}{-}{-}{-}{-}{-}{-}{-}{-}{-}{-}{-}{-}{-}{-}{-}{-}{-}{-}{-}{-}{-}{-}{-}{-}{-}{-}{-}{-}{-}{-}{-}{-}{-}{-}{-}{-}{-}{-}{-}{-}{-}{-}{-}{-}{-}{-}}
\CommentTok{\# 2. ПРЕОБРАЗОВАНИЕ КООРДИНАТ В ПРОЕКЦИЮ UTM (в км)}
\CommentTok{\# {-}{-}{-}{-}{-}{-}{-}{-}{-}{-}{-}{-}{-}{-}{-}{-}{-}{-}{-}{-}{-}{-}{-}{-}{-}{-}{-}{-}{-}{-}{-}{-}{-}{-}{-}{-}{-}{-}{-}{-}{-}{-}{-}{-}{-}{-}{-}{-}{-}{-}}

\CommentTok{\# Создание пространственного объекта из данных}
\NormalTok{data\_sf }\OtherTok{\textless{}{-}} \FunctionTok{st\_as\_sf}\NormalTok{(}
\NormalTok{  data, }
  \AttributeTok{coords =} \FunctionTok{c}\NormalTok{(}\StringTok{"X"}\NormalTok{, }\StringTok{"Y"}\NormalTok{), }\CommentTok{\# Указание столбцов с координатами}
  \AttributeTok{crs =} \DecValTok{4326}            \CommentTok{\# Система координат WGS84 (широта/долгота)}
\NormalTok{) }

\CommentTok{\# Преобразование в UTM зону 37N (метры)}
\NormalTok{data\_utm }\OtherTok{\textless{}{-}} \FunctionTok{st\_transform}\NormalTok{(data\_sf, }\AttributeTok{crs =} \DecValTok{32637}\NormalTok{) }

\CommentTok{\# Извлечение координат и перевод в километры}
\NormalTok{utm\_coords }\OtherTok{\textless{}{-}} \FunctionTok{st\_coordinates}\NormalTok{(data\_utm)}
\NormalTok{data}\SpecialCharTok{$}\NormalTok{xkm }\OtherTok{\textless{}{-}}\NormalTok{ utm\_coords[, }\DecValTok{1}\NormalTok{] }\SpecialCharTok{/} \DecValTok{1000}  \CommentTok{\# X в км}
\NormalTok{data}\SpecialCharTok{$}\NormalTok{ykm }\OtherTok{\textless{}{-}}\NormalTok{ utm\_coords[, }\DecValTok{2}\NormalTok{] }\SpecialCharTok{/} \DecValTok{1000}  \CommentTok{\# Y в км}

\CommentTok{\# Очистка временных объектов}
\FunctionTok{rm}\NormalTok{(data\_sf, data\_utm, utm\_coords)}

\CommentTok{\# {-}{-}{-}{-}{-}{-}{-}{-}{-}{-}{-}{-}{-}{-}{-}{-}{-}{-}{-}{-}{-}{-}{-}{-}{-}{-}{-}{-}{-}{-}{-}{-}{-}{-}{-}{-}{-}{-}{-}{-}{-}}
\CommentTok{\# 3. ОПРЕДЕЛЕНИЕ ГРАНИЦ ИССЛЕДОВАНИЯ}
\CommentTok{\# {-}{-}{-}{-}{-}{-}{-}{-}{-}{-}{-}{-}{-}{-}{-}{-}{-}{-}{-}{-}{-}{-}{-}{-}{-}{-}{-}{-}{-}{-}{-}{-}{-}{-}{-}{-}{-}{-}{-}{-}{-}}

\CommentTok{\# Вычисление границ исследовательского полигона}
\NormalTok{xl }\OtherTok{\textless{}{-}} \FunctionTok{c}\NormalTok{(}\FunctionTok{min}\NormalTok{(data}\SpecialCharTok{$}\NormalTok{xkm), }\FunctionTok{max}\NormalTok{(data}\SpecialCharTok{$}\NormalTok{xkm))  }\CommentTok{\# Границы по X}
\NormalTok{yl }\OtherTok{\textless{}{-}} \FunctionTok{c}\NormalTok{(}\FunctionTok{min}\NormalTok{(data}\SpecialCharTok{$}\NormalTok{ykm), }\FunctionTok{max}\NormalTok{(data}\SpecialCharTok{$}\NormalTok{ykm))  }\CommentTok{\# Границы по Y}

\CommentTok{\# {-}{-}{-}{-}{-}{-}{-}{-}{-}{-}{-}{-}{-}{-}{-}{-}{-}{-}{-}{-}{-}{-}{-}{-}{-}{-}{-}{-}{-}{-}{-}{-}{-}{-}{-}{-}{-}{-}{-}{-}}
\CommentTok{\# 4. СОЗДАНИЕ РАСТРОВОЙ СЕТКИ ДЛЯ МОДЕЛИ}
\CommentTok{\# {-}{-}{-}{-}{-}{-}{-}{-}{-}{-}{-}{-}{-}{-}{-}{-}{-}{-}{-}{-}{-}{-}{-}{-}{-}{-}{-}{-}{-}{-}{-}{-}{-}{-}{-}{-}{-}{-}{-}{-}}

\CommentTok{\# Создание равномерной сетки с шагом 10 км (для визуализации карты использовался шаг 2 км}
\NormalTok{GRID }\OtherTok{\textless{}{-}} \FunctionTok{makeGrid}\NormalTok{(}
  \AttributeTok{x =} \FunctionTok{seq}\NormalTok{(xl[}\DecValTok{1}\NormalTok{], xl[}\DecValTok{2}\NormalTok{], }\DecValTok{10}\NormalTok{), }
  \AttributeTok{y =} \FunctionTok{seq}\NormalTok{(yl[}\DecValTok{1}\NormalTok{], yl[}\DecValTok{2}\NormalTok{], }\DecValTok{10}\NormalTok{),}
  \AttributeTok{byrow =} \ConstantTok{FALSE}\NormalTok{,}
  \AttributeTok{projection =} \StringTok{"UTM"}\NormalTok{, }
  \AttributeTok{zone =} \DecValTok{37}
\NormalTok{)}

\CommentTok{\# Расчет центроидов ячеек сетки}
\NormalTok{GRID }\OtherTok{\textless{}{-}} \FunctionTok{calcCentroid}\NormalTok{(GRID, }\AttributeTok{rollup =} \DecValTok{3}\NormalTok{)}

\CommentTok{\# {-}{-}{-}{-}{-}{-}{-}{-}{-}{-}{-}{-}{-}{-}{-}{-}{-}{-}{-}{-}{-}{-}{-}{-}{-}{-}{-}{-}{-}{-}{-}{-}{-}{-}{-}{-}{-}{-}{-}{-}{-}{-}{-}{-}{-}{-}{-}{-}{-}{-}{-}{-}{-}{-}{-}{-}{-}{-}{-}}
\CommentTok{\# 5. ПОСТРОЕНИЕ ВЫПУКЛОЙ ОБОЛОЧКИ (CONVEX HULL) ДЛЯ ДАННЫХ}
\CommentTok{\# {-}{-}{-}{-}{-}{-}{-}{-}{-}{-}{-}{-}{-}{-}{-}{-}{-}{-}{-}{-}{-}{-}{-}{-}{-}{-}{-}{-}{-}{-}{-}{-}{-}{-}{-}{-}{-}{-}{-}{-}{-}{-}{-}{-}{-}{-}{-}{-}{-}{-}{-}{-}{-}{-}{-}{-}{-}{-}{-}}

\CommentTok{\# Создание выпуклой оболочки вокруг точек данных}
\NormalTok{Hull }\OtherTok{\textless{}{-}} \FunctionTok{inla.nonconvex.hull}\NormalTok{(}\FunctionTok{cbind}\NormalTok{(data}\SpecialCharTok{$}\NormalTok{xkm, data}\SpecialCharTok{$}\NormalTok{ykm), }\AttributeTok{convex =} \SpecialCharTok{{-}}\FloatTok{0.03}\NormalTok{)}

\CommentTok{\# Визуализация оболочки }
 \FunctionTok{plot}\NormalTok{(Hull)}

\CommentTok{\# Преобразование Hull в объект PolySet:}
\NormalTok{polys }\OtherTok{\textless{}{-}} \FunctionTok{data.frame}\NormalTok{(}
  \AttributeTok{PID =} \FunctionTok{rep}\NormalTok{(}\DecValTok{1}\NormalTok{, }\FunctionTok{nrow}\NormalTok{(Hull}\SpecialCharTok{$}\NormalTok{loc)), }\CommentTok{\# ID полигона}
  \AttributeTok{POS =} \DecValTok{1}\SpecialCharTok{:}\FunctionTok{nrow}\NormalTok{(Hull}\SpecialCharTok{$}\NormalTok{loc),       }\CommentTok{\# Порядок точек}
  \AttributeTok{X =}\NormalTok{ Hull}\SpecialCharTok{$}\NormalTok{loc[, }\DecValTok{1}\NormalTok{],            }\CommentTok{\# Координата X (в км)}
  \AttributeTok{Y =}\NormalTok{ Hull}\SpecialCharTok{$}\NormalTok{loc[, }\DecValTok{2}\NormalTok{]             }\CommentTok{\# Координата Y (в км)}
\NormalTok{)}
\NormalTok{polys }\OtherTok{\textless{}{-}}\NormalTok{ PBSmapping}\SpecialCharTok{::}\FunctionTok{as.PolySet}\NormalTok{(polys, }\AttributeTok{projection =} \StringTok{"UTM"}\NormalTok{, }\AttributeTok{zone =} \DecValTok{37}\NormalTok{)}

\CommentTok{\# Расчет площади:}
\NormalTok{area }\OtherTok{\textless{}{-}}\NormalTok{ PBSmapping}\SpecialCharTok{::}\FunctionTok{calcArea}\NormalTok{(polys)}
\FunctionTok{print}\NormalTok{(}\FunctionTok{paste}\NormalTok{(}\StringTok{"Площадь Hull:"}\NormalTok{, }\FunctionTok{round}\NormalTok{(area}\SpecialCharTok{$}\NormalTok{area, }\DecValTok{2}\NormalTok{), }\StringTok{"км кв."}\NormalTok{))}
\end{Highlighting}
\end{Shaded}

{[}1{]} ``Площадь Hull: 283947.5 км кв.''

\bookmarksetup{startatroot}

\chapter{Основы
геостатистики}\label{ux43eux441ux43dux43eux432ux44b-ux433ux435ux43eux441ux442ux430ux442ux438ux441ux442ux438ux43aux438}

\section{Введение}\label{ux432ux432ux435ux434ux435ux43dux438ux435-4}

\section{Загрузка данных и первичный
осмотр}\label{ux437ux430ux433ux440ux443ux437ux43aux430-ux434ux430ux43dux43dux44bux445-ux438-ux43fux435ux440ux432ux438ux447ux43dux44bux439-ux43eux441ux43cux43eux442ux440-1}

Текст. Текст. Текст.

\section{Описательная статистика и
визуализация}\label{ux43eux43fux438ux441ux430ux442ux435ux43bux44cux43dux430ux44f-ux441ux442ux430ux442ux438ux441ux442ux438ux43aux430-ux438-ux432ux438ux437ux443ux430ux43bux438ux437ux430ux446ux438ux44f-1}

Текст. Текст. Текст.

\bookmarksetup{startatroot}

\chapter{Байесовский
подход}\label{ux431ux430ux439ux435ux441ux43eux432ux441ux43aux438ux439-ux43fux43eux434ux445ux43eux434}

\section{Лекция-Введение}\label{ux43bux435ux43aux446ux438ux44f-ux432ux432ux435ux434ux435ux43dux438ux435}

\[  
P(\theta|x)=\frac{L(x|\theta)P(\theta)}{\sum\limits_{i=1}^{n}{\left[ L(x_i|\theta)P(\theta)\right]}}  
\]

\section{Описательная статистика и
визуализация}\label{ux43eux43fux438ux441ux430ux442ux435ux43bux44cux43dux430ux44f-ux441ux442ux430ux442ux438ux441ux442ux438ux43aux430-ux438-ux432ux438ux437ux443ux430ux43bux438ux437ux430ux446ux438ux44f-2}

Текст. Текст. Текст.

\bookmarksetup{startatroot}

\chapter{Продукционные
модели}\label{ux43fux440ux43eux434ux443ux43aux446ux438ux43eux43dux43dux44bux435-ux43cux43eux434ux435ux43bux438}

\section{Введение}\label{ux432ux432ux435ux434ux435ux43dux438ux435-5}

\section{Загрузка данных и первичный
осмотр}\label{ux437ux430ux433ux440ux443ux437ux43aux430-ux434ux430ux43dux43dux44bux445-ux438-ux43fux435ux440ux432ux438ux447ux43dux44bux439-ux43eux441ux43cux43eux442ux440-2}

Текст. Текст. Текст.

\section{Описательная статистика и
визуализация}\label{ux43eux43fux438ux441ux430ux442ux435ux43bux44cux43dux430ux44f-ux441ux442ux430ux442ux438ux441ux442ux438ux43aux430-ux438-ux432ux438ux437ux443ux430ux43bux438ux437ux430ux446ux438ux44f-3}

Текст. Текст. Текст.

\bookmarksetup{startatroot}

\chapter{Введение в
прогнозирование}\label{ux432ux432ux435ux434ux435ux43dux438ux435-ux432-ux43fux440ux43eux433ux43dux43eux437ux438ux440ux43eux432ux430ux43dux438ux435}

\section{Введение}\label{ux432ux432ux435ux434ux435ux43dux438ux435-6}

\section{Загрузка данных и первичный
осмотр}\label{ux437ux430ux433ux440ux443ux437ux43aux430-ux434ux430ux43dux43dux44bux445-ux438-ux43fux435ux440ux432ux438ux447ux43dux44bux439-ux43eux441ux43cux43eux442ux440-3}

Текст. Текст. Текст.

\section{Описательная статистика и
визуализация}\label{ux43eux43fux438ux441ux430ux442ux435ux43bux44cux43dux430ux44f-ux441ux442ux430ux442ux438ux441ux442ux438ux43aux430-ux438-ux432ux438ux437ux443ux430ux43bux438ux437ux430ux446ux438ux44f-4}

Текст. Текст. Текст.

\bookmarksetup{startatroot}

\chapter{ПРП и стратегии
управления}\label{ux43fux440ux43f-ux438-ux441ux442ux440ux430ux442ux435ux433ux438ux438-ux443ux43fux440ux430ux432ux43bux435ux43dux438ux44f}

\section{Введение}\label{ux432ux432ux435ux434ux435ux43dux438ux435-7}

\section{Загрузка данных и первичный
осмотр}\label{ux437ux430ux433ux440ux443ux437ux43aux430-ux434ux430ux43dux43dux44bux445-ux438-ux43fux435ux440ux432ux438ux447ux43dux44bux439-ux43eux441ux43cux43eux442ux440-4}

Текст. Текст. Текст.

\section{Описательная статистика и
визуализация}\label{ux43eux43fux438ux441ux430ux442ux435ux43bux44cux43dux430ux44f-ux441ux442ux430ux442ux438ux441ux442ux438ux43aux430-ux438-ux432ux438ux437ux443ux430ux43bux438ux437ux430ux446ux438ux44f-5}

Текст. Текст. Текст.

\bookmarksetup{startatroot}

\chapter*{References}\label{references}
\addcontentsline{toc}{chapter}{References}

\markboth{References}{References}

\phantomsection\label{refs}




\end{document}
