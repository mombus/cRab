% Options for packages loaded elsewhere
% Options for packages loaded elsewhere
\PassOptionsToPackage{unicode}{hyperref}
\PassOptionsToPackage{hyphens}{url}
\PassOptionsToPackage{dvipsnames,svgnames,x11names}{xcolor}
%
\documentclass[
  letterpaper,
  DIV=11,
  numbers=noendperiod]{scrreprt}
\usepackage{xcolor}
\usepackage{amsmath,amssymb}
\setcounter{secnumdepth}{5}
\usepackage{iftex}
\ifPDFTeX
  \usepackage[T1]{fontenc}
  \usepackage[utf8]{inputenc}
  \usepackage{textcomp} % provide euro and other symbols
\else % if luatex or xetex
  \usepackage{unicode-math} % this also loads fontspec
  \defaultfontfeatures{Scale=MatchLowercase}
  \defaultfontfeatures[\rmfamily]{Ligatures=TeX,Scale=1}
\fi
\usepackage{lmodern}
\ifPDFTeX\else
  % xetex/luatex font selection
\fi
% Use upquote if available, for straight quotes in verbatim environments
\IfFileExists{upquote.sty}{\usepackage{upquote}}{}
\IfFileExists{microtype.sty}{% use microtype if available
  \usepackage[]{microtype}
  \UseMicrotypeSet[protrusion]{basicmath} % disable protrusion for tt fonts
}{}
\makeatletter
\@ifundefined{KOMAClassName}{% if non-KOMA class
  \IfFileExists{parskip.sty}{%
    \usepackage{parskip}
  }{% else
    \setlength{\parindent}{0pt}
    \setlength{\parskip}{6pt plus 2pt minus 1pt}}
}{% if KOMA class
  \KOMAoptions{parskip=half}}
\makeatother
% Make \paragraph and \subparagraph free-standing
\makeatletter
\ifx\paragraph\undefined\else
  \let\oldparagraph\paragraph
  \renewcommand{\paragraph}{
    \@ifstar
      \xxxParagraphStar
      \xxxParagraphNoStar
  }
  \newcommand{\xxxParagraphStar}[1]{\oldparagraph*{#1}\mbox{}}
  \newcommand{\xxxParagraphNoStar}[1]{\oldparagraph{#1}\mbox{}}
\fi
\ifx\subparagraph\undefined\else
  \let\oldsubparagraph\subparagraph
  \renewcommand{\subparagraph}{
    \@ifstar
      \xxxSubParagraphStar
      \xxxSubParagraphNoStar
  }
  \newcommand{\xxxSubParagraphStar}[1]{\oldsubparagraph*{#1}\mbox{}}
  \newcommand{\xxxSubParagraphNoStar}[1]{\oldsubparagraph{#1}\mbox{}}
\fi
\makeatother

\usepackage{color}
\usepackage{fancyvrb}
\newcommand{\VerbBar}{|}
\newcommand{\VERB}{\Verb[commandchars=\\\{\}]}
\DefineVerbatimEnvironment{Highlighting}{Verbatim}{commandchars=\\\{\}}
% Add ',fontsize=\small' for more characters per line
\usepackage{framed}
\definecolor{shadecolor}{RGB}{241,243,245}
\newenvironment{Shaded}{\begin{snugshade}}{\end{snugshade}}
\newcommand{\AlertTok}[1]{\textcolor[rgb]{0.68,0.00,0.00}{#1}}
\newcommand{\AnnotationTok}[1]{\textcolor[rgb]{0.37,0.37,0.37}{#1}}
\newcommand{\AttributeTok}[1]{\textcolor[rgb]{0.40,0.45,0.13}{#1}}
\newcommand{\BaseNTok}[1]{\textcolor[rgb]{0.68,0.00,0.00}{#1}}
\newcommand{\BuiltInTok}[1]{\textcolor[rgb]{0.00,0.23,0.31}{#1}}
\newcommand{\CharTok}[1]{\textcolor[rgb]{0.13,0.47,0.30}{#1}}
\newcommand{\CommentTok}[1]{\textcolor[rgb]{0.37,0.37,0.37}{#1}}
\newcommand{\CommentVarTok}[1]{\textcolor[rgb]{0.37,0.37,0.37}{\textit{#1}}}
\newcommand{\ConstantTok}[1]{\textcolor[rgb]{0.56,0.35,0.01}{#1}}
\newcommand{\ControlFlowTok}[1]{\textcolor[rgb]{0.00,0.23,0.31}{\textbf{#1}}}
\newcommand{\DataTypeTok}[1]{\textcolor[rgb]{0.68,0.00,0.00}{#1}}
\newcommand{\DecValTok}[1]{\textcolor[rgb]{0.68,0.00,0.00}{#1}}
\newcommand{\DocumentationTok}[1]{\textcolor[rgb]{0.37,0.37,0.37}{\textit{#1}}}
\newcommand{\ErrorTok}[1]{\textcolor[rgb]{0.68,0.00,0.00}{#1}}
\newcommand{\ExtensionTok}[1]{\textcolor[rgb]{0.00,0.23,0.31}{#1}}
\newcommand{\FloatTok}[1]{\textcolor[rgb]{0.68,0.00,0.00}{#1}}
\newcommand{\FunctionTok}[1]{\textcolor[rgb]{0.28,0.35,0.67}{#1}}
\newcommand{\ImportTok}[1]{\textcolor[rgb]{0.00,0.46,0.62}{#1}}
\newcommand{\InformationTok}[1]{\textcolor[rgb]{0.37,0.37,0.37}{#1}}
\newcommand{\KeywordTok}[1]{\textcolor[rgb]{0.00,0.23,0.31}{\textbf{#1}}}
\newcommand{\NormalTok}[1]{\textcolor[rgb]{0.00,0.23,0.31}{#1}}
\newcommand{\OperatorTok}[1]{\textcolor[rgb]{0.37,0.37,0.37}{#1}}
\newcommand{\OtherTok}[1]{\textcolor[rgb]{0.00,0.23,0.31}{#1}}
\newcommand{\PreprocessorTok}[1]{\textcolor[rgb]{0.68,0.00,0.00}{#1}}
\newcommand{\RegionMarkerTok}[1]{\textcolor[rgb]{0.00,0.23,0.31}{#1}}
\newcommand{\SpecialCharTok}[1]{\textcolor[rgb]{0.37,0.37,0.37}{#1}}
\newcommand{\SpecialStringTok}[1]{\textcolor[rgb]{0.13,0.47,0.30}{#1}}
\newcommand{\StringTok}[1]{\textcolor[rgb]{0.13,0.47,0.30}{#1}}
\newcommand{\VariableTok}[1]{\textcolor[rgb]{0.07,0.07,0.07}{#1}}
\newcommand{\VerbatimStringTok}[1]{\textcolor[rgb]{0.13,0.47,0.30}{#1}}
\newcommand{\WarningTok}[1]{\textcolor[rgb]{0.37,0.37,0.37}{\textit{#1}}}

\usepackage{longtable,booktabs,array}
\usepackage{calc} % for calculating minipage widths
% Correct order of tables after \paragraph or \subparagraph
\usepackage{etoolbox}
\makeatletter
\patchcmd\longtable{\par}{\if@noskipsec\mbox{}\fi\par}{}{}
\makeatother
% Allow footnotes in longtable head/foot
\IfFileExists{footnotehyper.sty}{\usepackage{footnotehyper}}{\usepackage{footnote}}
\makesavenoteenv{longtable}
\usepackage{graphicx}
\makeatletter
\newsavebox\pandoc@box
\newcommand*\pandocbounded[1]{% scales image to fit in text height/width
  \sbox\pandoc@box{#1}%
  \Gscale@div\@tempa{\textheight}{\dimexpr\ht\pandoc@box+\dp\pandoc@box\relax}%
  \Gscale@div\@tempb{\linewidth}{\wd\pandoc@box}%
  \ifdim\@tempb\p@<\@tempa\p@\let\@tempa\@tempb\fi% select the smaller of both
  \ifdim\@tempa\p@<\p@\scalebox{\@tempa}{\usebox\pandoc@box}%
  \else\usebox{\pandoc@box}%
  \fi%
}
% Set default figure placement to htbp
\def\fps@figure{htbp}
\makeatother





\setlength{\emergencystretch}{3em} % prevent overfull lines

\providecommand{\tightlist}{%
  \setlength{\itemsep}{0pt}\setlength{\parskip}{0pt}}



 


\KOMAoption{captions}{tableheading}
\makeatletter
\@ifpackageloaded{bookmark}{}{\usepackage{bookmark}}
\makeatother
\makeatletter
\@ifpackageloaded{caption}{}{\usepackage{caption}}
\AtBeginDocument{%
\ifdefined\contentsname
  \renewcommand*\contentsname{Table of contents}
\else
  \newcommand\contentsname{Table of contents}
\fi
\ifdefined\listfigurename
  \renewcommand*\listfigurename{List of Figures}
\else
  \newcommand\listfigurename{List of Figures}
\fi
\ifdefined\listtablename
  \renewcommand*\listtablename{List of Tables}
\else
  \newcommand\listtablename{List of Tables}
\fi
\ifdefined\figurename
  \renewcommand*\figurename{Figure}
\else
  \newcommand\figurename{Figure}
\fi
\ifdefined\tablename
  \renewcommand*\tablename{Table}
\else
  \newcommand\tablename{Table}
\fi
}
\@ifpackageloaded{float}{}{\usepackage{float}}
\floatstyle{ruled}
\@ifundefined{c@chapter}{\newfloat{codelisting}{h}{lop}}{\newfloat{codelisting}{h}{lop}[chapter]}
\floatname{codelisting}{Listing}
\newcommand*\listoflistings{\listof{codelisting}{List of Listings}}
\makeatother
\makeatletter
\makeatother
\makeatletter
\@ifpackageloaded{caption}{}{\usepackage{caption}}
\@ifpackageloaded{subcaption}{}{\usepackage{subcaption}}
\makeatother
\usepackage{bookmark}
\IfFileExists{xurl.sty}{\usepackage{xurl}}{} % add URL line breaks if available
\urlstyle{same}
\hypersetup{
  pdftitle={Оценка водных биоресурсов при недостатке данных в среде R (для начинающих)},
  pdfauthor={Сергей Баканёв},
  colorlinks=true,
  linkcolor={blue},
  filecolor={Maroon},
  citecolor={Blue},
  urlcolor={Blue},
  pdfcreator={LaTeX via pandoc}}


\title{Оценка водных биоресурсов при недостатке данных в среде R (для
начинающих)}
\author{Сергей Баканёв}
\date{2025-04-30}
\begin{document}
\maketitle

\renewcommand*\contentsname{Table of contents}
{
\hypersetup{linkcolor=}
\setcounter{tocdepth}{2}
\tableofcontents
}

\bookmarksetup{startatroot}

\chapter*{Аннотация}\label{ux430ux43dux43dux43eux442ux430ux446ux438ux44f}
\addcontentsline{toc}{chapter}{Аннотация}

\markboth{Аннотация}{Аннотация}

\textbf{Баканев С. В. (2025) Оценка водных биоресурсов при недостатке
данных в среде R (для начинающих). --- Курс лекций и практических
занятий, адрес доступа: https://mombus.github.io/cRab}

Данный ресурс представляет собой сборник практических решений по
применению современных статистических методов в гидробиологии и
рыбохозяйственных исследованиях при анализе ``неполных'' данных. Курс
включает пошаговые алгоритмы реализации методов анализа и оценки водных
биоресурсов в среде R и ориентирован как на начинающих, так и просто -
интересующихся специалистов.

Практическая программа охватывает ключевые методы оценки водных
биоресурсов, выстроенные в логической последовательности: от основ
анализа данных улова и картографии до продвинутого
пространственно-временного моделирования (sdmTMB) и методов машинного
обучения.

Ресурс постоянно развивается и дополняется новыми практическими
примерами.

\begin{center}
\includegraphics[width=0.5\linewidth,height=\textheight,keepaspectratio]{images/cRab_logo.png}
\end{center}

\bookmarksetup{startatroot}

\chapter{Введение}\label{ux432ux432ux435ux434ux435ux43dux438ux435}

``How data treat you?'' --- Аристотель

Настоящий ресурс посвящен применению современных методов анализа данных
для оценки водных биоресурсов, поскольку, как заметил бы Роберт
Сапольски, игнорировать статистику в рыбохозяйственной науке так же
безрассудно, как павиану-самцу пренебрегать иерархией стаи перед брачным
сезоном. Изначально эти материалы создавались для очного курса, который
пока не состоялся и, в силу разных обстоятельств, возможно, не
состоится. Что ж, эволюция любит слепые переулки, и вот теперь
практические занятия выставлены на всеобщее обозрение --- в свободное
пользование для всех заинтересованных специалистов. Лекции, быть может,
последуют позже, а может, и нет --- как повезет.

Мы живем в эпоху больших языковых моделей, когда любая стандартная
методика может быть разжевана нейросетью, а обучающий скрипт
сгенерирован за время, необходимое чтобы моргнуть. В таких условиях
настоящую ценность представляют уже не шаблонные решения, а
профессиональный опыт, причудливые идеи и то самое глубокое понимание,
которое позволяет видеть данные изнутри. Молодые специалисты, как
правило, достаточно быстро учатся собирать материал «в поле» --- этому
способствует и врожденная склонность человека к исследованию, и вполне
себе социализированное желание проводить время на свежем воздухе. Но вот
ключевой вопрос, перед которым многие замирают: а что делать с этими
данными дальше? Ограничиться стандартной картой и графиком с
коэффициентом корреляции --- сегодня не вариант. Мир анализа данных
полон мощных и поразительных инструментов, и наша задача --- не только
показать их, но и научить заставлять данные рассказывать о себе так, как
им, возможно, не всегда хочется: по-разному, подробно, иногда даже
против их воли.

\emph{How data treat you?} --- как-то раз ернически спросил меня\ldots{}
нет, не Аристотель, конечно, а Aristoteles, молодой специалист из
Венесуэлы. Произошло это во время одной научно-исследовательской съемки
у берегов Фолклендских островов. Его вопрос я понял не сразу ---
возможно, сказалась усталость, а может, когнитивный диссонанс от того,
что столь глубинно-философское прозрение пришло к человеку в ярком плаще
и с гамаком за спиной. Но смысл его оказался точен: данные относятся к
тебе так, как ты относишься к ним.

Сегодня от исследователя уже не требуется безупречного владения навыками
программирования --- достаточно иметь терпение, любопытство и немного
смирения перед лицом технологии. LLM стали нашими компаньонами,
цифровыми шимпанзе-ассистентами, способными написать, исправить,
прокомментировать или продолжить почти любой скрипт. Это значительно
снижает порог входа в мир R и анализа. Такой подход можно было бы
назвать «vibe coding» --- итеративный, почти медитативный процесс
творческого диалога с машиной, где ты формулируешь задачи на
естественном языке, прототипируешь идеи через ИИ, а затем оцениваешь
результат, фокусируясь не на синтаксисе, а на смысле. Данный практикум
стремится культивировать именно такой --- более человечный --- стиль
работы. Многие из этих занятий родились в нескончаемых диалогах и
импровизированных дискуссиях с Cursor, DeepSeek, Qwen и KIMI --- моими
цифровыми коллегами по цеху.

Практикум представляет собой своего рода путеводитель --- или, если
угодно, field guide --- по применению современных методов анализа
данных, ориентированный на начинающих специалистов. Материалы
структурированы так, чтобы охватить ключевые этапы работы: от первичной
загрузки и обработки данных до продвинутого моделирования и
визуализации. Здесь вы найдёте подробные примеры кода, пояснения к
методам и --- что особенно важно --- интерпретацию результатов, которая
позволяет не только освоить R, но и понять, каким образом эти выводы
встраиваются в более широкий биологический и управленческий контекст.

Особое внимание уделено работе с ограниченными и неполными данными ---
ситуацией, типичной для многих гидробиологических и рыбохозяйственных
исследований. Потому что, let's face it, идеальные датасеты существуют
разве что в учебниках. В реальности же нам приходится иметь дело с тем,
что есть --- и находить красоту в несовершенстве. Практикум включает как
классические статистические методы (линейные и логистические регрессии,
кластеризация, сравнение групп), так и современные подходы:
пространственно-временное моделирование (sdmTMB), нейронные сети и
байесовские методы оценки запасов (SPiCT, JABBA). Отдельный раздел
посвящен картографированию и визуализации --- потому что карта всё ещё
иногда говорит громче, чем сто пятьсот p-value.

Материалы продолжают пополняться --- медленно, неравномерно, с
переменным успехом --- и доступны в открытом доступе. Возможно, они
станут для кого-то тем самым ресурсом, которого не хватало.
Приветствуются предложения по сотрудничеству и материалы от коллег --- с
обязательным указанием авторства. Также принимаются вопросы, идеи и даже
деликатно оформленные предложения по улучшению --- потому что ни один
мозг, даже при поддержке LLM, не может объять необъятное.

Контакты для связи: Сергей Баканёв mombus@gmail.com

\begin{center}
\includegraphics[width=0.5\linewidth,height=\textheight,keepaspectratio]{images/Bob.png}
\end{center}

\bookmarksetup{startatroot}

\chapter{Анализ и визуализация данных
улова}\label{ux430ux43dux430ux43bux438ux437-ux438-ux432ux438ux437ux443ux430ux43bux438ux437ux430ux446ux438ux44f-ux434ux430ux43dux43dux44bux445-ux443ux43bux43eux432ux430}

\section{Введение}\label{ux432ux432ux435ux434ux435ux43dux438ux435-1}

Это занятие --- про первый шаг в анализе уловов: аккуратно загрузить
данные, посмотреть на них без иллюзий и задать простые, проверяемые
вопросы. Мы будем работать в R, потому что он честно показывает
структуру данных и не скрывает неудобные детали. Наша цель сейчас не
«сделать красиво», а убедиться, что мы видим именно то, что реально
записано в файле: сколько строк, какие переменные, каких они типов и не
прячутся ли среди них ошибки, способные испортить весь последующий
анализ.

Начинаем с того, чтобы R «видел» правильную папку. Рабочая директория
должна указывать туда, где лежит файл shrimp\_catch.csv. Простая
установка пути --- это не бюрократия, а воспроизводимость: на другом
компьютере тот же код должен читать те же данные, а не «что‑то похожее».
После этого подключаем tidyverse: это набор инструментов, который
унифицирует чтение, преобразование и визуализацию. read\_csv считывает
таблицу и сразу создаёт tibble --- «вежливую» версию data.frame с чётким
хранением типов. Уже на этом шаге стоит помнить о банальных, но частых
ловушках: десятичный разделитель должен совпадать с вашими региональными
настройками, пустые строки и «NA» в файле должны превращаться в
пропуски, а не в нули или текст.

Первичный осмотр --- это короткий разговор с данными без интерпретации.
Команда glimpse выдаёт компактный снимок: сколько строк, каковы названия
столбцов и их классы, примеры значений. В нашем наборе ожидаем пять
столбцов: id как целое число, age как целое число 1--4, length как
числовая величина длины, weight как числовая величина массы и sex как
текстовый признак пола. Если вы видите, что length внезапно «chr» или
sex закодирован числами --- это сигнал остановиться и привести типы в
порядок сейчас, а не объяснять странные результаты потом. Аналогичная
команда str показывает внутреннюю структуру и подтверждает, что R
понимает объект так же, как и вы. Эти две команды --- «микроскоп 4×»:
быстро и без украшательств.

Дальше имеет смысл задать несколько контрольных вопросов, которые
одновременно проверяют здравый смысл и раскрывают базовую статистику.
summary покажет минимумы, медианы и квартильные точки для количественных
переменных и распределение для категориальных. Если где‑то возникает
отрицательный вес, нулевая длина или возраст за пределами 1--4 --- это
не «особенности популяции», это данные, требующие чистки. table и
prop.table дадут частоты по полу; если соотношение полов выглядит
нереалистично для вашей промысловой выборки --- проверьте этап
предобработки. Наконец, простой cor.test между длиной и весом покажет,
есть ли ожидаемая сильная положительная связь; но здесь важно помнить о
дисциплине: корреляция --- это не причинность, и даже высокая r требует
подтверждения графиком рассеяния и проверкой на аутлаеры.

Зачем столько внимания «мелочам» до любых моделей? Потому что в
прикладной биостатистике именно этот участок пути отделяет полезные
выводы от красивых, но пустых графиков. Проверка типов и диапазонов,
явное обращение с пропусками, подтверждение структурой --- это те
скромные процедуры, которые экономят часы на поздних этапах. И если
позволить себе лёгкую ремарку, то лучший способ повысить
интеллектуальную честность анализа --- не верить по умолчанию ни себе,
ни данным, пока вы не посмотрели на них под простейшим светом glimpse и
str.

Когда эти шаги пройдены, можно переходить к описательной статистике и
первичной визуализации. Гистограмма длины даст быстрый набросок формы
распределения, а простые группировки по возрасту покажут, как меняются
средние и разброс. Но это уже следующий раздел. Сейчас важнее, чтобы R и
вы одинаково понимали, что такое «наши данные», и чтобы каждый
последующий результат опирался на корректно загруженную и проверенную
таблицу
\href{https://mombus.github.io/cRab/data/shrimp_catch.csv}{shrimp\_catch.csv}.

\section{Загрузка данных и первичный
осмотр}\label{ux437ux430ux433ux440ux443ux437ux43aux430-ux434ux430ux43dux43dux44bux445-ux438-ux43fux435ux440ux432ux438ux447ux43dux44bux439-ux43eux441ux43cux43eux442ux440}

ссылка на файл:
\href{https://mombus.github.io/cRab/data/shrimp_catch.csv}{shrimp\_catch.csv}

\begin{Shaded}
\begin{Highlighting}[]
\CommentTok{\# Установка рабочей директории}
\FunctionTok{setwd}\NormalTok{(}\StringTok{"C:/TEXTBOOK/"}\NormalTok{)}
\CommentTok{\# Загрузка библиотек}
\FunctionTok{library}\NormalTok{(tidyverse)}
\CommentTok{\# Загрузка данных}
\NormalTok{data }\OtherTok{\textless{}{-}} \FunctionTok{read\_csv}\NormalTok{(}\StringTok{"shrimp\_catch.csv"}\NormalTok{)}
\end{Highlighting}
\end{Shaded}

Команда \texttt{glimpse} знакомит со структурой данных:

\begin{Shaded}
\begin{Highlighting}[]
\CommentTok{\# Просмотр структуры и первых строк загруженных данных}
\FunctionTok{glimpse}\NormalTok{(data)}
\end{Highlighting}
\end{Shaded}

\begin{Shaded}
\begin{Highlighting}[]
\NormalTok{Rows}\SpecialCharTok{:} \DecValTok{230}
\NormalTok{Columns}\SpecialCharTok{:} \DecValTok{5}
\SpecialCharTok{$}\NormalTok{ id     }\SpecialCharTok{\textless{}}\NormalTok{int}\SpecialCharTok{\textgreater{}} \DecValTok{1}\NormalTok{, }\DecValTok{2}\NormalTok{, }\DecValTok{3}\NormalTok{, }\DecValTok{4}\NormalTok{, }\DecValTok{5}\NormalTok{, }\DecValTok{6}\NormalTok{, }\DecValTok{7}\NormalTok{, }\DecValTok{8}\NormalTok{, }\DecValTok{9}\NormalTok{, }\DecValTok{10}\NormalTok{, }\DecValTok{11}\NormalTok{, }\DecValTok{12}\NormalTok{, }\DecValTok{13}\NormalTok{, }\DecValTok{14}\NormalTok{, }\DecValTok{15}\NormalTok{, }\DecValTok{16}\NormalTok{, }\DecValTok{17}\NormalTok{, }\DecValTok{18}\NormalTok{, }\SpecialCharTok{\textasciitilde{}}
\ErrorTok{$}\NormalTok{ age    }\SpecialCharTok{\textless{}}\NormalTok{int}\SpecialCharTok{\textgreater{}} \DecValTok{2}\NormalTok{, }\DecValTok{4}\NormalTok{, }\DecValTok{4}\NormalTok{, }\DecValTok{4}\NormalTok{, }\DecValTok{1}\NormalTok{, }\DecValTok{4}\NormalTok{, }\DecValTok{2}\NormalTok{, }\DecValTok{2}\NormalTok{, }\DecValTok{4}\NormalTok{, }\DecValTok{3}\NormalTok{, }\DecValTok{4}\NormalTok{, }\DecValTok{3}\NormalTok{, }\DecValTok{2}\NormalTok{, }\DecValTok{1}\NormalTok{, }\DecValTok{2}\NormalTok{, }\DecValTok{1}\NormalTok{, }\DecValTok{2}\NormalTok{, }\DecValTok{2}\NormalTok{, }\DecValTok{2}\NormalTok{, }\DecValTok{2}\NormalTok{, }\DecValTok{3}\NormalTok{, }\SpecialCharTok{\textasciitilde{}}
\ErrorTok{$}\NormalTok{ length }\SpecialCharTok{\textless{}}\NormalTok{dbl}\SpecialCharTok{\textgreater{}} \FloatTok{20.45450}\NormalTok{, }\FloatTok{25.88928}\NormalTok{, }\FloatTok{29.42257}\NormalTok{, }\FloatTok{30.68292}\NormalTok{, }\FloatTok{12.46059}\NormalTok{, }\FloatTok{28.52152}\NormalTok{, }\FloatTok{17.}\SpecialCharTok{\textasciitilde{}}
\ErrorTok{$}\NormalTok{ weight }\SpecialCharTok{\textless{}}\NormalTok{dbl}\SpecialCharTok{\textgreater{}} \FloatTok{1.28221748}\NormalTok{, }\FloatTok{1.97476899}\NormalTok{, }\FloatTok{2.65412595}\NormalTok{, }\FloatTok{3.44746476}\NormalTok{, }\FloatTok{0.13404801}\NormalTok{, }\FloatTok{2.3}\SpecialCharTok{\textasciitilde{}}
\ErrorTok{$}\NormalTok{ sex    }\SpecialCharTok{\textless{}}\NormalTok{chr}\SpecialCharTok{\textgreater{}} \StringTok{"M"}\NormalTok{, }\StringTok{"F"}\NormalTok{, }\StringTok{"F"}\NormalTok{, }\StringTok{"F"}\NormalTok{, }\StringTok{"M"}\NormalTok{, }\StringTok{"F"}\NormalTok{, }\StringTok{"M"}\NormalTok{, }\StringTok{"M"}\NormalTok{, }\StringTok{"F"}\NormalTok{, }\StringTok{"F"}\NormalTok{, }\StringTok{"F"}\NormalTok{, }\StringTok{"F"}\NormalTok{, }\StringTok{"M"}\SpecialCharTok{\textasciitilde{}}
\ErrorTok{\textgreater{}} 
\end{Highlighting}
\end{Shaded}

Можно использовать команду \texttt{str} --- показывает внутреннюю
\textbf{структуру} объекта , включая количество строк, столбцов,
названия переменных, их типы (\texttt{chr}, \texttt{num}, \texttt{int} и
др.), а также несколько первых значений.

\begin{Shaded}
\begin{Highlighting}[]
\FunctionTok{str}\NormalTok{(data)}
\end{Highlighting}
\end{Shaded}

\begin{Shaded}
\begin{Highlighting}[]
\StringTok{\textquotesingle{}data.frame\textquotesingle{}}\SpecialCharTok{:}   \DecValTok{230}\NormalTok{ obs. of  }\DecValTok{5}\NormalTok{ variables}\SpecialCharTok{:}
 \ErrorTok{$}\NormalTok{ id    }\SpecialCharTok{:}\NormalTok{ int  }\DecValTok{1} \DecValTok{2} \DecValTok{3} \DecValTok{4} \DecValTok{5} \DecValTok{6} \DecValTok{7} \DecValTok{8} \DecValTok{9} \DecValTok{10}\NormalTok{ ...}
 \SpecialCharTok{$}\NormalTok{ age   }\SpecialCharTok{:}\NormalTok{ int  }\DecValTok{2} \DecValTok{4} \DecValTok{4} \DecValTok{4} \DecValTok{1} \DecValTok{4} \DecValTok{2} \DecValTok{2} \DecValTok{4} \DecValTok{3}\NormalTok{ ...}
 \SpecialCharTok{$}\NormalTok{ length}\SpecialCharTok{:}\NormalTok{ num  }\FloatTok{20.5} \FloatTok{25.9} \FloatTok{29.4} \FloatTok{30.7} \FloatTok{12.5}\NormalTok{ ...}
 \SpecialCharTok{$}\NormalTok{ weight}\SpecialCharTok{:}\NormalTok{ num  }\FloatTok{1.282} \FloatTok{1.975} \FloatTok{2.654} \FloatTok{3.447} \FloatTok{0.134}\NormalTok{ ...}
 \SpecialCharTok{$}\NormalTok{ sex   }\SpecialCharTok{:}\NormalTok{ chr  }\StringTok{"M"} \StringTok{"F"} \StringTok{"F"} \StringTok{"F"}\NormalTok{ ...}
\SpecialCharTok{\textgreater{}}
\end{Highlighting}
\end{Shaded}

\section{Описательная статистика и
визуализация}\label{ux43eux43fux438ux441ux430ux442ux435ux43bux44cux43dux430ux44f-ux441ux442ux430ux442ux438ux441ux442ux438ux43aux430-ux438-ux432ux438ux437ux443ux430ux43bux438ux437ux430ux446ux438ux44f}

Команда \texttt{summary} выводит \textbf{описательную статистику} для
каждой числовой переменной: минимум, 1-й квартиль, медиана, среднее, 3-й
квартиль, максимум; для категориальных переменных --- частоты.

\begin{Shaded}
\begin{Highlighting}[]
\CommentTok{\# Общая статистика}
\FunctionTok{summary}\NormalTok{(data)}
\end{Highlighting}
\end{Shaded}

\begin{Shaded}
\begin{Highlighting}[]
\NormalTok{       id              age            length          weight       }
\NormalTok{ Min.   }\SpecialCharTok{:}  \FloatTok{1.00}\NormalTok{   Min.   }\SpecialCharTok{:}\FloatTok{1.000}\NormalTok{   Min.   }\SpecialCharTok{:} \FloatTok{7.65}\NormalTok{   Min.   }\SpecialCharTok{:{-}}\FloatTok{0.3334}  
 \DecValTok{1}\NormalTok{st Qu.}\SpecialCharTok{:} \FloatTok{58.25}   \DecValTok{1}\NormalTok{st Qu.}\SpecialCharTok{:}\FloatTok{2.000}   \DecValTok{1}\NormalTok{st Qu.}\SpecialCharTok{:}\FloatTok{17.62}   \DecValTok{1}\NormalTok{st Qu.}\SpecialCharTok{:} \FloatTok{0.6320}  
\NormalTok{ Median }\SpecialCharTok{:}\FloatTok{115.50}\NormalTok{   Median }\SpecialCharTok{:}\FloatTok{3.000}\NormalTok{   Median }\SpecialCharTok{:}\FloatTok{22.49}\NormalTok{   Median }\SpecialCharTok{:} \FloatTok{1.3660}  
\NormalTok{ Mean   }\SpecialCharTok{:}\FloatTok{115.50}\NormalTok{   Mean   }\SpecialCharTok{:}\FloatTok{2.509}\NormalTok{   Mean   }\SpecialCharTok{:}\FloatTok{21.68}\NormalTok{   Mean   }\SpecialCharTok{:} \FloatTok{1.4933}  
 \DecValTok{3}\NormalTok{rd Qu.}\SpecialCharTok{:}\FloatTok{172.75}   \DecValTok{3}\NormalTok{rd Qu.}\SpecialCharTok{:}\FloatTok{3.000}   \DecValTok{3}\NormalTok{rd Qu.}\SpecialCharTok{:}\FloatTok{26.03}   \DecValTok{3}\NormalTok{rd Qu.}\SpecialCharTok{:} \FloatTok{2.1148}  
\NormalTok{ Max.   }\SpecialCharTok{:}\FloatTok{230.00}\NormalTok{   Max.   }\SpecialCharTok{:}\FloatTok{4.000}\NormalTok{   Max.   }\SpecialCharTok{:}\FloatTok{36.02}\NormalTok{   Max.   }\SpecialCharTok{:} \FloatTok{5.1316}  
\NormalTok{     sex           }
\NormalTok{ Length}\SpecialCharTok{:}\DecValTok{230}        
\NormalTok{ Class }\SpecialCharTok{:}\NormalTok{character  }
\NormalTok{ Mode  }\SpecialCharTok{:}\NormalTok{character  }
\end{Highlighting}
\end{Shaded}

Простейшими командами можно вычислить, например, соотоношение полов или
корреляцию длина-вес.

\begin{Shaded}
\begin{Highlighting}[]
\CommentTok{\# Соотношение полов}
\FunctionTok{prop.table}\NormalTok{(}\FunctionTok{table}\NormalTok{(data}\SpecialCharTok{$}\NormalTok{sex)) }\SpecialCharTok{\%\textgreater{}\%} \FunctionTok{round}\NormalTok{(}\DecValTok{2}\NormalTok{)}
\end{Highlighting}
\end{Shaded}

\begin{Shaded}
\begin{Highlighting}[]
\NormalTok{   F    M }
\FloatTok{0.35} \FloatTok{0.65} 
\end{Highlighting}
\end{Shaded}

\begin{Shaded}
\begin{Highlighting}[]
\CommentTok{\# Корреляция длина{-}вес с p{-}value}
\NormalTok{cor\_test }\OtherTok{\textless{}{-}} \FunctionTok{cor.test}\NormalTok{(data}\SpecialCharTok{$}\NormalTok{length, data}\SpecialCharTok{$}\NormalTok{weight, }
                     \AttributeTok{method =} \StringTok{"pearson"}\NormalTok{, }
                     \AttributeTok{exact =} \ConstantTok{FALSE}\NormalTok{,}
                     \AttributeTok{na.action =}\NormalTok{ na.omit)}
 
\NormalTok{cor\_coef }\OtherTok{\textless{}{-}} \FunctionTok{round}\NormalTok{(cor\_test}\SpecialCharTok{$}\NormalTok{estimate, }\DecValTok{2}\NormalTok{)}
\NormalTok{p\_value }\OtherTok{\textless{}{-}}\NormalTok{ scales}\SpecialCharTok{::}\FunctionTok{pvalue}\NormalTok{(cor\_test}\SpecialCharTok{$}\NormalTok{p.value, }\AttributeTok{accuracy =}\NormalTok{ .}\DecValTok{001}\NormalTok{)}
 
\FunctionTok{cat}\NormalTok{(}\StringTok{"Корреляция Пирсона: r ="}\NormalTok{, cor\_coef, }\StringTok{", p ="}\NormalTok{, p\_value, }\StringTok{"}\SpecialCharTok{\textbackslash{}n}\StringTok{"}\NormalTok{)}
\end{Highlighting}
\end{Shaded}

\begin{Shaded}
\begin{Highlighting}[]
\NormalTok{Корреляция Пирсона}\SpecialCharTok{:}\NormalTok{ r }\OtherTok{=} \FloatTok{0.95}\NormalTok{ , p }\OtherTok{=} \ErrorTok{\textless{}}\FloatTok{0.001} 
\end{Highlighting}
\end{Shaded}

\begin{Shaded}
\begin{Highlighting}[]
\CommentTok{\# Распределение возраста}
\FunctionTok{table}\NormalTok{(data}\SpecialCharTok{$}\NormalTok{age)}
\end{Highlighting}
\end{Shaded}

\begin{Shaded}
\begin{Highlighting}[]
\DecValTok{1}  \DecValTok{2}  \DecValTok{3}  \DecValTok{4} 
\DecValTok{43} \DecValTok{68} \DecValTok{77} \DecValTok{40} 
\end{Highlighting}
\end{Shaded}

\begin{Shaded}
\begin{Highlighting}[]
\CommentTok{\# Средние значения длины и веса по группам}
\NormalTok{data }\SpecialCharTok{\%\textgreater{}\%}
   \FunctionTok{group\_by}\NormalTok{(age) }\SpecialCharTok{\%\textgreater{}\%}
   \FunctionTok{summarise}\NormalTok{(}
     \AttributeTok{mean\_length =} \FunctionTok{mean}\NormalTok{(length),}
     \AttributeTok{sd\_length =} \FunctionTok{sd}\NormalTok{(length),}
     \AttributeTok{mean\_weight =} \FunctionTok{mean}\NormalTok{(weight),}
     \AttributeTok{sd\_weight =} \FunctionTok{sd}\NormalTok{(weight))}
\end{Highlighting}
\end{Shaded}

\begin{Shaded}
\begin{Highlighting}[]
\CommentTok{\# A tibble: 4 x 5}
\NormalTok{    age mean\_length sd\_length mean\_weight sd\_weight}
  \SpecialCharTok{\textless{}}\NormalTok{dbl}\SpecialCharTok{\textgreater{}}       \ErrorTok{\textless{}}\NormalTok{dbl}\SpecialCharTok{\textgreater{}}     \ErrorTok{\textless{}}\NormalTok{dbl}\SpecialCharTok{\textgreater{}}       \ErrorTok{\textless{}}\NormalTok{dbl}\SpecialCharTok{\textgreater{}}     \ErrorTok{\textless{}}\NormalTok{dbl}\SpecialCharTok{\textgreater{}}
\DecValTok{1}     \DecValTok{1}        \FloatTok{12.7}      \FloatTok{1.37}       \FloatTok{0.249}     \FloatTok{0.234}
\DecValTok{2}     \DecValTok{2}        \FloatTok{19.2}      \FloatTok{1.88}       \FloatTok{0.919}     \FloatTok{0.341}
\DecValTok{3}     \DecValTok{3}        \FloatTok{24.8}      \FloatTok{1.72}       \FloatTok{1.88}      \FloatTok{0.424}
\DecValTok{4}     \DecValTok{4}        \FloatTok{29.1}      \FloatTok{2.28}       \FloatTok{2.96}      \FloatTok{0.804}
\SpecialCharTok{\textgreater{}} 
\end{Highlighting}
\end{Shaded}

\subsection{Построение гистограммы для переменной `length' (длина
креветок)}\label{ux43fux43eux441ux442ux440ux43eux435ux43dux438ux435-ux433ux438ux441ux442ux43eux433ux440ux430ux43cux43cux44b-ux434ux43bux44f-ux43fux435ux440ux435ux43cux435ux43dux43dux43eux439-length-ux434ux43bux438ux43dux430-ux43aux440ux435ux432ux435ux442ux43eux43a}

Для первого визуального знакомства команда \texttt{hist} строит
гистограмму --- простой график, который показывает, как распределены
значения числовой переменной. В данном случае отображается распределение
длин креветок из набора данных.

\begin{Shaded}
\begin{Highlighting}[]
\FunctionTok{hist}\NormalTok{(data}\SpecialCharTok{$}\NormalTok{length, }
     \AttributeTok{main =} \StringTok{"Гистограмма длины креветок"}\NormalTok{,          }\CommentTok{\# Заголовок графика}
     \AttributeTok{xlab =} \StringTok{"Длина (см)"}\NormalTok{,                          }\CommentTok{\# Подпись оси X}
     \AttributeTok{ylab =} \StringTok{"Частота"}\NormalTok{,                             }\CommentTok{\# Подпись оси Y}
     \AttributeTok{col =} \StringTok{"lightblue"}\NormalTok{,                            }\CommentTok{\# Цвет столбцов}
     \AttributeTok{border =} \StringTok{"black"}\NormalTok{,                             }\CommentTok{\# Цвет границ столбцов}
     \AttributeTok{breaks =} \DecValTok{15}\NormalTok{)                                   }\CommentTok{\# Количество интервалов}
\end{Highlighting}
\end{Shaded}

\begin{figure}[H]

{\centering \includegraphics[width=0.6\linewidth,height=\textheight,keepaspectratio]{images/hist_shrimp.PNG}

}

\caption{Рис. 1.1: Гистограмма длины креветок}

\end{figure}%

\subsection{\texorpdfstring{Визуализация в
\texttt{ggridges}}{Визуализация в ggridges}}\label{ux432ux438ux437ux443ux430ux43bux438ux437ux430ux446ux438ux44f-ux432-ggridges}

Для элегантных и компактных графиков подходит библиотека
\texttt{ggridges}. Построим распределение длины креветки в зависимости
от пола и возраста.

\begin{Shaded}
\begin{Highlighting}[]
\FunctionTok{library}\NormalTok{(ggplot2)}
\FunctionTok{library}\NormalTok{(ggridges)}

\FunctionTok{ggplot}\NormalTok{(data, }\FunctionTok{aes}\NormalTok{(}\AttributeTok{x =}\NormalTok{ length, }
                 \AttributeTok{y =}\NormalTok{ sex, }
                 \AttributeTok{group =}\NormalTok{ sex, }
                 \AttributeTok{fill =}\NormalTok{ sex)) }\SpecialCharTok{+}
  \FunctionTok{geom\_density\_ridges}\NormalTok{(}\AttributeTok{scale =} \DecValTok{2}\NormalTok{, }\AttributeTok{alpha =} \FloatTok{0.7}\NormalTok{) }\SpecialCharTok{+}
  \FunctionTok{scale\_y\_discrete}\NormalTok{(}\AttributeTok{expand =} \FunctionTok{c}\NormalTok{(}\DecValTok{0}\NormalTok{, }\DecValTok{0}\NormalTok{)) }\SpecialCharTok{+}
  \FunctionTok{scale\_x\_continuous}\NormalTok{(}\AttributeTok{expand =} \FunctionTok{c}\NormalTok{(}\DecValTok{0}\NormalTok{, }\DecValTok{0}\NormalTok{)) }\SpecialCharTok{+}
  \FunctionTok{labs}\NormalTok{(}
    \AttributeTok{title =} \StringTok{"Распределение длины карапакса по полу"}\NormalTok{,}
    \AttributeTok{x =} \StringTok{"Длина карапакса (мм)"}\NormalTok{,}
    \AttributeTok{y =} \StringTok{"Пол"}
\NormalTok{  ) }\SpecialCharTok{+}
  \FunctionTok{theme}\NormalTok{(}
    \AttributeTok{panel.border =} \FunctionTok{element\_blank}\NormalTok{(),  }\CommentTok{\# Убирает рамку вокруг графика}
    \AttributeTok{axis.line =} \FunctionTok{element\_line}\NormalTok{(}\AttributeTok{color =} \StringTok{"black"}\NormalTok{)  }\CommentTok{\# Сохраняет осевые линии (опционально)}
\NormalTok{  )}
\end{Highlighting}
\end{Shaded}

\begin{figure}[H]

{\centering \includegraphics[width=0.6\linewidth,height=\textheight,keepaspectratio]{images/ggridges_shrimp.PNG}

}

\caption{Рис. 1.2: Пол-длина креветок с использованием
\texttt{ggridges}}

\end{figure}%

\section{Выявление аутлайеров
(выбросов)}\label{ux432ux44bux44fux432ux43bux435ux43dux438ux435-ux430ux443ux442ux43bux430ux439ux435ux440ux43eux432-ux432ux44bux431ux440ux43eux441ux43eux432}

Аутлаеры (выбросы) --- наблюдения, значительно отклоняющиеся от общего
распределения данных. Их идентификация критически важна, так как они
могут искажать результаты анализа. Один из надёжных методов обнаружения
выбросов --- \textbf{метод межквартильного размаха (IQR)}.

\subsection{\texorpdfstring{\textbf{Теория
метода}}{Теория метода}}\label{ux442ux435ux43eux440ux438ux44f-ux43cux435ux442ux43eux434ux430}

\begin{enumerate}
\def\labelenumi{\arabic{enumi}.}
\item
  \textbf{Расчёт квартилей}:

  \begin{itemize}
  \item
    \textbf{Q1} (25-й перцентиль): значение, ниже которого находится
    25\% данных.
  \item
    \textbf{Q3} (75-й перцентиль): значение, ниже которого находится
    75\% данных.
  \item
    \textbf{IQR = Q3 - Q1}: мера разброса средней половины данных.
  \end{itemize}
\item
  \textbf{Границы аутлаеров}:

  \begin{itemize}
  \item
    \textbf{Нижняя граница}: Q1−1.5×IQRQ1−1.5×IQR
  \item
    \textbf{Верхняя граница}: Q3+1.5×IQRQ3+1.5×IQR\\
    Наблюдения за этими пределами считаются выбросами.
  \end{itemize}
\end{enumerate}

\subsection{\texorpdfstring{\textbf{Преимущества
метода}}{Преимущества метода}}\label{ux43fux440ux435ux438ux43cux443ux449ux435ux441ux442ux432ux430-ux43cux435ux442ux43eux434ux430}

\begin{itemize}
\item
  Устойчивость к асимметрии распределения.
\item
  Не требует предположения о нормальности данных.
\end{itemize}

\begin{Shaded}
\begin{Highlighting}[]
\CommentTok{\# Метод межквартильного размаха}
\NormalTok{outliers }\OtherTok{\textless{}{-}}\NormalTok{ data }\SpecialCharTok{\%\textgreater{}\%}
  \FunctionTok{mutate}\NormalTok{(}
    \AttributeTok{length\_z =} \FunctionTok{scale}\NormalTok{(length),}
    \AttributeTok{weight\_z =} \FunctionTok{scale}\NormalTok{(weight)}
\NormalTok{  ) }\SpecialCharTok{\%\textgreater{}\%} 
  \FunctionTok{filter}\NormalTok{(}\FunctionTok{abs}\NormalTok{(length\_z) }\SpecialCharTok{\textgreater{}} \DecValTok{3} \SpecialCharTok{|} \FunctionTok{abs}\NormalTok{(weight\_z) }\SpecialCharTok{\textgreater{}} \DecValTok{3}\NormalTok{)}

\CommentTok{\# Визуализация}
\FunctionTok{ggplot}\NormalTok{(data, }\FunctionTok{aes}\NormalTok{(}\AttributeTok{x =}\NormalTok{ length, }\AttributeTok{y =}\NormalTok{ weight)) }\SpecialCharTok{+}
  \FunctionTok{geom\_point}\NormalTok{(}\FunctionTok{aes}\NormalTok{(}\AttributeTok{color =} \StringTok{"Обычные"}\NormalTok{), }\AttributeTok{alpha =} \FloatTok{0.5}\NormalTok{) }\SpecialCharTok{+}
  \FunctionTok{geom\_point}\NormalTok{(}\AttributeTok{data =}\NormalTok{ outliers, }\FunctionTok{aes}\NormalTok{(}\AttributeTok{color =} \StringTok{"Аутлаеры"}\NormalTok{), }\AttributeTok{size =} \DecValTok{3}\NormalTok{) }\SpecialCharTok{+}
  \FunctionTok{scale\_color\_manual}\NormalTok{(}\AttributeTok{values =} \FunctionTok{c}\NormalTok{(}\StringTok{"Обычные"} \OtherTok{=} \StringTok{"grey50"}\NormalTok{, }\StringTok{"Аутлаеры"} \OtherTok{=} \StringTok{"red"}\NormalTok{)) }\SpecialCharTok{+}
  \FunctionTok{labs}\NormalTok{(}\AttributeTok{title =} \StringTok{"Выявление аномальных наблюдений"}\NormalTok{, }\AttributeTok{color =} \StringTok{"Тип"}\NormalTok{)}
\end{Highlighting}
\end{Shaded}

\begin{figure}[H]

{\centering \includegraphics[width=0.6\linewidth,height=\textheight,keepaspectratio]{images/outliers_shrimp.PNG}

}

\caption{Рис. 1.3: Распределение длины карапакса}

\end{figure}%

\section{Определение возрастной структуры: статистические методы анализа
размерных
данных}\label{ux43eux43fux440ux435ux434ux435ux43bux435ux43dux438ux435-ux432ux43eux437ux440ux430ux441ux442ux43dux43eux439-ux441ux442ux440ux443ux43aux442ux443ux440ux44b-ux441ux442ux430ux442ux438ux441ux442ux438ux447ux435ux441ux43aux438ux435-ux43cux435ux442ux43eux434ux44b-ux430ux43dux430ux43bux438ux437ux430-ux440ux430ux437ux43cux435ux440ux43dux44bux445-ux434ux430ux43dux43dux44bux445}

\textbf{Лирическое отступление}

Определять возрастную структуру по размерным данным --- это не попытка
«угадать» возраст, а дисциплинированный способ выделить в общей смеси
несколько закономерных мод, за которыми почти всегда стоят биологические
процессы: вариации в пополнении, различия в темпах роста, селективность
промысла. Мы не видим возраст напрямую, зато видим след его накопления в
длине, и задача статистики здесь --- разложить смешанное распределение
на осмысленные компоненты, не выдавая желаемое за действительное.
Начинать разумно с простого: гистограмма и сглаженная плотность дают
первичную картину, где «пики» --- это кандидаты на возрастные группы.
Выбор ширины бинов (диапазонов) --- не косметика: слишком широкие бины
сливают моды, слишком узкие создают шумовые «зубцы». Ядерная плотность
полезна как независимая проверка: если пик виден и на гистограмме, и на
сглаженной плотности, это хороший знак. Уже на этом шаге важно исключить
явные аутлаеры и убедиться, что анализируется однородная по сезону и
району выборка: смешение сезонов способно превратить один чёткий пик в
два слабых и наоборот.

K‑means привлекателен скоростью и простотой, но его допущения жестковаты
для биологии: он делит по ближайшему центру и фактически предполагает
равные дисперсии у групп. Для черновой разметки это приемлемо: задали
\emph{K}, получили кластеры, посмотрели, не распилили ли явный пик
пополам и не смешали ли крайние хвосты. Но трактовать эти кластеры как
«возрастные классы» без дополнительных проверок нельзя. Минимальный
набор проверок --- «локальный смысл»: центры кластеров должны быть
упорядочены по длине, доли групп не должны выглядеть абсурдно для вашей
системы, а границы между кластерами --- приходиться на спады между
модами гистограммы. Полезно пробежать несколько значений \emph{K},
посмотреть «локоть» по внутрикластерной дисперсии или силуэт; если
модель жадно «доедает» шум, она не помогает задаче.

Декомпозиция смесью нормалей с EM‑алгоритмом ближе к тому, что нам
нужно: каждая компонента имеет свой средний размер, свою дисперсию и
свою долю, а принадлежность особи --- вероятностная, а не «жёсткая». Это
лучше отражает реальность: возрастные группы перекрываются, и жёсткое
отнесение на границе избыточно уверенно. Здесь ключевая инженерная мысль
--- инициализация и выбор числа компонент. Стартовать можно от пиковой
структуры гистограммы или от грубых центров k‑means; число компонент
выбирать по BIC/AIC и здравому смыслу, помня, что каждая лишняя
компонента почти всегда «объясняет» шум. Параметры смеси имеют
прозрачную интерпретацию: \emph{μ} --- модальный размер группы, \emph{σ}
--- разброс (ростовая гетерогенность плюс измерительная ошибка),
\emph{λ} --- доля группы в выборке. Для отчётности полезно ранжировать
компоненты по μ, чтобы избежать «перескока меток» между запусками, и
дать доверительные интервалы (обычный приём --- бутстрэп).

Метод Бхаттачарии, классика промысловой статистики, по сути делает то же
в терминах гистограммы: линейнизует лог‑разности соседних бинов и
позволяет визуально «вынуть» наклон, соответствующий компоненте
нормального распределения. Он чувствителен к выбору ширины бина и к
ровности хвостов, зато нагляден и хорошо работает там, где пики
действительно нормальны и отделены. В паре с EM это сильная связка:
Бхаттачария помогает выбрать разумное \emph{K} и старт, EM --- уточняет
параметры и даёт вероятностные принадлежности. Сопоставление результатов
этих двух подходов повышает доверие: если оба «видят» четыре группы с
близкими μ, это гораздо лучше, чем красивый рисунок одного метода.

На практике полезно придерживаться алгоритма «снизу вверх». Сначала ---
чистая визуализация: гистограмма, ядерная плотность, по возможности
разрезы по полу и возрасту полевой маркировки; это помогает понять, не
смешиваем ли мы биологически разные контексты. Затем --- черновая
кластеризация k‑means для ориентировочных центров и грубого \emph{K}.
Далее --- смеси нормалей с EM, выбор \emph{K} по BIC и проверка
стабильности решения от разных стартов. После подгонки --- диагностика:
наложить компоненты и суммарную смесь на гистограмму, проверить, не
«улетели» ли \emph{σ}, нет ли «дублирующих» компонент с почти
одинаковыми \emph{μ}, сопоставить \emph{λ} с ожидаемыми долями когорт. И
главное --- помнить, что «модальные группы по длине» и «возрастные
классы» совпадают не автоматом: для перевода мод в возраст нужен
ростовой ключ (например, параметры Берталанфи) или независимая
информация о когортности. Без этого честнее говорить «модальные
размерные группы».

Наконец, стоит держать в голове пару дисциплинарных напоминаний. Любая
смесь будет пытаться объяснить артефакты данных, поэтому контроль
качества измерений и фильтрация аутлаеров --- не опция, а необходимость.
Сезонная выборка и выборочная выловленность сдвигают доли и средние:
если орудия ловят неравномерно, \emph{λ} смеси --- это не «структура
популяции», а «структура улова». И, как бы прозаично это ни звучало,
фиксируйте зерно генератора случайных чисел и документируйте выбор
\emph{K} и стартовые значения: это делает ваш результат воспроизводимым,
а спор --- предметным. Такой дисциплинированный ход --- от картинки к
модели, от модели к диагностике, от диагностики к осторожной
интерпретации --- позволяет извлечь из длины то, что она действительно
хранит про возраст, и не больше.

И так, возрастная структура популяции --- часто важна для расчёта
промысловой смертности, оценки репродуктивного потенциала и
прогнозирования динамики запасов. Поскольку прямое измерение возраста
часто невозможно (например, у беспозвоночных или рыб без четких
возрастных меток), используются статистические методы, выделяющие группы
в смешанных распределениях размеров.

\textbf{Основные подходы:}

\begin{enumerate}
\def\labelenumi{\arabic{enumi}.}
\item
  \textbf{Метод k-средних (k-means)} --- алгоритм кластеризации,
  группирующий особи в заданное число кластеров (возрастных групп) на
  основе их размеров.
\item
  \textbf{Метод Бхаттачарии} --- статистический подход для разделения
  смешанных нормальных распределений, часто применяемый для
  идентификации мод в гистограммах.
\item
  \textbf{EM-алгоритм} --- оценка параметров смеси распределений,
  подходящая для данных с перекрывающимися возрастными группами.
\item
  \textbf{Гауссовы смеси (GMM)} --- расширение метода Бхаттачарии для
  многомерного анализа.
\item
  \textbf{Ядерное сглаживание} --- непараметрический метод визуализации
  плотности, помогающий выявить скрытые моды.
\end{enumerate}

Рассмотрим метод k-средних (k-means) и метод Бхаттачарии, предварительно
построив гистограмму.

\begin{Shaded}
\begin{Highlighting}[]
\CommentTok{\# Загрузка библиотек}
\FunctionTok{library}\NormalTok{(tidyverse)}
\FunctionTok{library}\NormalTok{(mixtools)}
\CommentTok{\# Гистограмма длины с наложением плотности}
\FunctionTok{ggplot}\NormalTok{(data, }\FunctionTok{aes}\NormalTok{(}\AttributeTok{x =}\NormalTok{ length)) }\SpecialCharTok{+}
  \FunctionTok{geom\_histogram}\NormalTok{(}\FunctionTok{aes}\NormalTok{(}\AttributeTok{y =} \FunctionTok{after\_stat}\NormalTok{(density)), }\AttributeTok{fill =} \StringTok{"steelblue"}\NormalTok{, }\AttributeTok{bins =} \DecValTok{20}\NormalTok{, }\AttributeTok{alpha =} \FloatTok{0.7}\NormalTok{) }\SpecialCharTok{+}
  \FunctionTok{geom\_density}\NormalTok{(}\AttributeTok{color =} \StringTok{"\#FC4E07"}\NormalTok{, }\AttributeTok{linewidth =} \DecValTok{1}\NormalTok{) }\SpecialCharTok{+}
  \FunctionTok{labs}\NormalTok{(}\AttributeTok{title =} \StringTok{"Распределение длины карапакса"}\NormalTok{, }
       \AttributeTok{subtitle =} \StringTok{"Пики могут соответствовать возрастным группам"}\NormalTok{,}
       \AttributeTok{x =} \StringTok{"Длина (мм)"}\NormalTok{)}
\end{Highlighting}
\end{Shaded}

\begin{figure}[H]

{\centering \includegraphics[width=0.6\linewidth,height=\textheight,keepaspectratio]{images/hist_dens_shrimp.PNG}

}

\caption{Рис. 1.3: Распределение длины карапакса}

\end{figure}%

\begin{Shaded}
\begin{Highlighting}[]
\CommentTok{\# Кластеризация по длине (K{-}means как пример)}
\FunctionTok{set.seed}\NormalTok{(}\DecValTok{123}\NormalTok{)}
\NormalTok{clusters }\OtherTok{\textless{}{-}} \FunctionTok{kmeans}\NormalTok{(data}\SpecialCharTok{$}\NormalTok{length, }\AttributeTok{centers =} \DecValTok{4}\NormalTok{)  }\CommentTok{\# Предполагаем 4 возрастные группы}
\NormalTok{data}\SpecialCharTok{$}\NormalTok{cluster }\OtherTok{\textless{}{-}} \FunctionTok{factor}\NormalTok{(clusters}\SpecialCharTok{$}\NormalTok{cluster)}

\CommentTok{\# Визуализация кластеров}
\FunctionTok{ggplot}\NormalTok{(data, }\FunctionTok{aes}\NormalTok{(}\AttributeTok{x =}\NormalTok{ length, }\AttributeTok{fill =}\NormalTok{ cluster)) }\SpecialCharTok{+}
  \FunctionTok{geom\_histogram}\NormalTok{(}\AttributeTok{bins =} \DecValTok{25}\NormalTok{, }\AttributeTok{alpha =} \FloatTok{0.7}\NormalTok{) }\SpecialCharTok{+}
  \FunctionTok{labs}\NormalTok{(}\AttributeTok{title =} \StringTok{"Кластеризация по длине)"}\NormalTok{, }
       \AttributeTok{x =} \StringTok{"Длина (мм)"}\NormalTok{)}
\end{Highlighting}
\end{Shaded}

\begin{figure}[H]

{\centering \includegraphics[width=0.6\linewidth,height=\textheight,keepaspectratio]{images/cluster_shrimp.PNG}

}

\caption{Рис. 1.4: Кластеризация по длине}

\end{figure}%

\begin{Shaded}
\begin{Highlighting}[]
\CommentTok{\# Установка рабочей директории}
\FunctionTok{setwd}\NormalTok{(}\StringTok{"C:/TEXTBOOK/"}\NormalTok{)}

\CommentTok{\# Загрузка библиотек}
\FunctionTok{library}\NormalTok{(tidyverse)}
\FunctionTok{library}\NormalTok{(mixtools)}

\CommentTok{\# Загрузка данных}
\NormalTok{data }\OtherTok{\textless{}{-}} \FunctionTok{read.csv}\NormalTok{(}\StringTok{"shrimp\_catch.csv"}\NormalTok{)}

\CommentTok{\# 1. Построение и отображение гистограммы}
\FunctionTok{hist}\NormalTok{(data}\SpecialCharTok{$}\NormalTok{length, }\AttributeTok{breaks =} \DecValTok{20}\NormalTok{, }\AttributeTok{main =} \StringTok{"Гистограмма распределения длин карапаксов"}\NormalTok{,}
     \AttributeTok{xlab =} \StringTok{"Длина карапакса (мм)"}\NormalTok{, }\AttributeTok{ylab =} \StringTok{"Частота"}\NormalTok{)}

\CommentTok{\# 2. Инициализация параметров (предположим 4 возрастные группы)}
\NormalTok{init\_params }\OtherTok{\textless{}{-}} \FunctionTok{list}\NormalTok{(}
  \AttributeTok{lambda =} \FunctionTok{rep}\NormalTok{(}\DecValTok{1}\SpecialCharTok{/}\DecValTok{4}\NormalTok{, }\DecValTok{4}\NormalTok{),}
  \AttributeTok{mu =} \FunctionTok{c}\NormalTok{(}\DecValTok{13}\NormalTok{, }\DecValTok{19}\NormalTok{, }\DecValTok{25}\NormalTok{, }\DecValTok{32}\NormalTok{),}
  \AttributeTok{sigma =} \FunctionTok{c}\NormalTok{(}\FloatTok{1.5}\NormalTok{, }\FloatTok{1.75}\NormalTok{, }\FloatTok{1.75}\NormalTok{, }\FloatTok{2.5}\NormalTok{)}
\NormalTok{)}

\CommentTok{\# 3. Разделение смеси распределений методом EM}
\NormalTok{fit }\OtherTok{\textless{}{-}} \FunctionTok{normalmixEM}\NormalTok{(data}\SpecialCharTok{$}\NormalTok{length, }\AttributeTok{k =} \DecValTok{4}\NormalTok{, }\AttributeTok{maxit =} \DecValTok{1000}\NormalTok{, }\AttributeTok{epsilon =} \FloatTok{1e{-}3}\NormalTok{,}
                   \AttributeTok{lambda =}\NormalTok{ init\_params}\SpecialCharTok{$}\NormalTok{lambda,}
                   \AttributeTok{mu =}\NormalTok{ init\_params}\SpecialCharTok{$}\NormalTok{mu,}
                   \AttributeTok{sigma =}\NormalTok{ init\_params}\SpecialCharTok{$}\NormalTok{sigma)}

\CommentTok{\# 4. Визуализация результатов с ggplot2}
\CommentTok{\# Генерация сетки для построения кривых}
\NormalTok{x\_grid }\OtherTok{\textless{}{-}} \FunctionTok{seq}\NormalTok{(}\FunctionTok{min}\NormalTok{(data}\SpecialCharTok{$}\NormalTok{length), }\FunctionTok{max}\NormalTok{(data}\SpecialCharTok{$}\NormalTok{length), }\AttributeTok{length.out =} \DecValTok{500}\NormalTok{)}

\CommentTok{\# Функция смеси}
\NormalTok{mixture\_density }\OtherTok{\textless{}{-}} \ControlFlowTok{function}\NormalTok{(x) \{}
\NormalTok{  fit}\SpecialCharTok{$}\NormalTok{lambda[}\DecValTok{1}\NormalTok{] }\SpecialCharTok{*} \FunctionTok{dnorm}\NormalTok{(x, fit}\SpecialCharTok{$}\NormalTok{mu[}\DecValTok{1}\NormalTok{], fit}\SpecialCharTok{$}\NormalTok{sigma[}\DecValTok{1}\NormalTok{]) }\SpecialCharTok{+}
\NormalTok{  fit}\SpecialCharTok{$}\NormalTok{lambda[}\DecValTok{2}\NormalTok{] }\SpecialCharTok{*} \FunctionTok{dnorm}\NormalTok{(x, fit}\SpecialCharTok{$}\NormalTok{mu[}\DecValTok{2}\NormalTok{], fit}\SpecialCharTok{$}\NormalTok{sigma[}\DecValTok{2}\NormalTok{]) }\SpecialCharTok{+}
\NormalTok{  fit}\SpecialCharTok{$}\NormalTok{lambda[}\DecValTok{3}\NormalTok{] }\SpecialCharTok{*} \FunctionTok{dnorm}\NormalTok{(x, fit}\SpecialCharTok{$}\NormalTok{mu[}\DecValTok{3}\NormalTok{], fit}\SpecialCharTok{$}\NormalTok{sigma[}\DecValTok{3}\NormalTok{]) }\SpecialCharTok{+}
\NormalTok{  fit}\SpecialCharTok{$}\NormalTok{lambda[}\DecValTok{4}\NormalTok{] }\SpecialCharTok{*} \FunctionTok{dnorm}\NormalTok{(x, fit}\SpecialCharTok{$}\NormalTok{mu[}\DecValTok{4}\NormalTok{], fit}\SpecialCharTok{$}\NormalTok{sigma[}\DecValTok{4}\NormalTok{])}
\NormalTok{\}}

\CommentTok{\# График}
\FunctionTok{ggplot}\NormalTok{(data, }\FunctionTok{aes}\NormalTok{(}\AttributeTok{x =}\NormalTok{ length)) }\SpecialCharTok{+}
  \CommentTok{\# Гистограмма}
  \FunctionTok{geom\_histogram}\NormalTok{(}\FunctionTok{aes}\NormalTok{(}\AttributeTok{y =} \FunctionTok{after\_stat}\NormalTok{(density)), }\AttributeTok{bins =} \DecValTok{20}\NormalTok{, }\AttributeTok{fill =} \StringTok{"white"}\NormalTok{, }\AttributeTok{color =} \StringTok{"black"}\NormalTok{, }\AttributeTok{alpha =} \FloatTok{0.7}\NormalTok{) }\SpecialCharTok{+}
  \CommentTok{\# Исходное распределение (гладкая линия)}
  \FunctionTok{geom\_density}\NormalTok{(}\AttributeTok{color =} \StringTok{"red"}\NormalTok{, }\AttributeTok{lwd =} \FloatTok{1.2}\NormalTok{) }\SpecialCharTok{+}
  \CommentTok{\# Смесь распределений}
  \FunctionTok{stat\_function}\NormalTok{(}\AttributeTok{fun =}\NormalTok{ mixture\_density, }\AttributeTok{color =} \StringTok{"black"}\NormalTok{, }\AttributeTok{lwd =} \FloatTok{1.5}\NormalTok{) }\SpecialCharTok{+}
  \CommentTok{\# Компоненты смеси}
  \FunctionTok{stat\_function}\NormalTok{(}\AttributeTok{fun =} \ControlFlowTok{function}\NormalTok{(x) fit}\SpecialCharTok{$}\NormalTok{lambda[}\DecValTok{1}\NormalTok{] }\SpecialCharTok{*} \FunctionTok{dnorm}\NormalTok{(x, fit}\SpecialCharTok{$}\NormalTok{mu[}\DecValTok{1}\NormalTok{], fit}\SpecialCharTok{$}\NormalTok{sigma[}\DecValTok{1}\NormalTok{]), }\AttributeTok{color =} \StringTok{"blue"}\NormalTok{, }\AttributeTok{lwd =} \DecValTok{1}\NormalTok{) }\SpecialCharTok{+}
  \FunctionTok{stat\_function}\NormalTok{(}\AttributeTok{fun =} \ControlFlowTok{function}\NormalTok{(x) fit}\SpecialCharTok{$}\NormalTok{lambda[}\DecValTok{2}\NormalTok{] }\SpecialCharTok{*} \FunctionTok{dnorm}\NormalTok{(x, fit}\SpecialCharTok{$}\NormalTok{mu[}\DecValTok{2}\NormalTok{], fit}\SpecialCharTok{$}\NormalTok{sigma[}\DecValTok{2}\NormalTok{]), }\AttributeTok{color =} \StringTok{"green"}\NormalTok{, }\AttributeTok{lwd =} \DecValTok{1}\NormalTok{) }\SpecialCharTok{+}
  \FunctionTok{stat\_function}\NormalTok{(}\AttributeTok{fun =} \ControlFlowTok{function}\NormalTok{(x) fit}\SpecialCharTok{$}\NormalTok{lambda[}\DecValTok{3}\NormalTok{] }\SpecialCharTok{*} \FunctionTok{dnorm}\NormalTok{(x, fit}\SpecialCharTok{$}\NormalTok{mu[}\DecValTok{3}\NormalTok{], fit}\SpecialCharTok{$}\NormalTok{sigma[}\DecValTok{3}\NormalTok{]), }\AttributeTok{color =} \StringTok{"orange"}\NormalTok{, }\AttributeTok{lwd =} \DecValTok{1}\NormalTok{) }\SpecialCharTok{+}
  \FunctionTok{stat\_function}\NormalTok{(}\AttributeTok{fun =} \ControlFlowTok{function}\NormalTok{(x) fit}\SpecialCharTok{$}\NormalTok{lambda[}\DecValTok{4}\NormalTok{] }\SpecialCharTok{*} \FunctionTok{dnorm}\NormalTok{(x, fit}\SpecialCharTok{$}\NormalTok{mu[}\DecValTok{4}\NormalTok{], fit}\SpecialCharTok{$}\NormalTok{sigma[}\DecValTok{4}\NormalTok{]), }\AttributeTok{color =} \StringTok{"purple"}\NormalTok{, }\AttributeTok{lwd =} \DecValTok{1}\NormalTok{) }\SpecialCharTok{+}
  
  \CommentTok{\# Настройка темы и легенды}
  \FunctionTok{theme\_minimal}\NormalTok{() }\SpecialCharTok{+}
  \FunctionTok{labs}\NormalTok{(}
    \AttributeTok{x =} \StringTok{"Длина карапакса (мм)"}\NormalTok{,}
    \AttributeTok{y =} \StringTok{"Плотность"}\NormalTok{,}
    \AttributeTok{title =} \StringTok{"Разделение возрастных групп методом EM"}
\NormalTok{  )}
\end{Highlighting}
\end{Shaded}

\begin{figure}[H]

{\centering \includegraphics[width=0.6\linewidth,height=\textheight,keepaspectratio]{images/bhattacharya_shrimp.PNG}

}

\caption{Рис. 1.5: Метод Бхаттачарии}

\end{figure}%

\section{Уравнение
Берталанфи}\label{ux443ux440ux430ux432ux43dux435ux43dux438ux435-ux431ux435ux440ux442ux430ux43bux430ux43dux444ux438}

Уравнение Берталанфи --- фундаментальная модель в рыбохозяйственной
науке, описывающая асимптотический рост организмов. Оно имеет вид: \[
L(t) = L_{\infty} \cdot \left(1 - e^{-k \cdot (t - t_0)}\right)
\] где \emph{L\textsubscript{∞}}--- теоретическая максимальная длина
особи, \emph{k}--- коэффициент скорости роста,
\emph{t\textsubscript{0}}--- гипотетический возраст при нулевой длине.

В приведённом коде модель применяется для анализа роста северной
креветки :

\begin{enumerate}
\def\labelenumi{\arabic{enumi}.}
\item
  \textbf{Подготовка данных}: Удаление аутлаеров (например, строк 10 и
  50) повышает точность оценки параметров.
\item
  \textbf{Инициализация параметров}:

  \begin{itemize}
  \item
    \emph{L\textsubscript{∞}} задаётся как максимальная наблюдаемая
    длина в данных.
  \item
    \emph{k} и \emph{t\textsubscript{0}} подбираются итеративно методом
    нелинейных наименьших квадратов (\textbf{\texttt{nls}}).
  \end{itemize}
\item
  \textbf{Визуализация}: График сопоставляет эмпирические данные (точки)
  с предсказаниями модели (красная линия), демонстрируя, как рост
  замедляется с приближением к \emph{L∞}.
\end{enumerate}

\textbf{Интерпретация параметров}:

\begin{itemize}
\item
  Высокое значение \emph{k} (\textgreater0.3) указывает на быстрый рост
  молоди.
\item
  \emph{t\textsubscript{0}}\textless0 может отражать ранний метаморфоз
  личинок.
\end{itemize}

\begin{Shaded}
\begin{Highlighting}[]
\CommentTok{\# Загрузка библиотек}
\FunctionTok{library}\NormalTok{(ggplot2)}
\FunctionTok{library}\NormalTok{(dplyr)}
\FunctionTok{library}\NormalTok{(nlme)}

\CommentTok{\# Загрузка данных}
\NormalTok{data }\OtherTok{\textless{}{-}} \FunctionTok{read.csv}\NormalTok{(}\StringTok{"shrimp\_catch.csv"}\NormalTok{)}

\CommentTok{\# Преобразование возраста в числовой формат}
\NormalTok{data}\SpecialCharTok{$}\NormalTok{age\_num }\OtherTok{\textless{}{-}} \FunctionTok{as.numeric}\NormalTok{(data}\SpecialCharTok{$}\NormalTok{age)}

\CommentTok{\# Удаление аутлайеров (если необходимо)}
\NormalTok{data\_clean }\OtherTok{\textless{}{-}}\NormalTok{ data }\SpecialCharTok{\%\textgreater{}\%}
  \FunctionTok{filter}\NormalTok{(}\SpecialCharTok{!}\NormalTok{id }\SpecialCharTok{\%in\%} \FunctionTok{c}\NormalTok{(}\DecValTok{10}\NormalTok{, }\DecValTok{50}\NormalTok{))  }\CommentTok{\# Пример удаления строк с аномалиями}

\CommentTok{\# Начальные параметры на основе данных}
\NormalTok{L\_inf\_start }\OtherTok{\textless{}{-}} \FunctionTok{max}\NormalTok{(data\_clean}\SpecialCharTok{$}\NormalTok{length, }\AttributeTok{na.rm =} \ConstantTok{TRUE}\NormalTok{)  }\CommentTok{\# Максимальная длина}
\NormalTok{k\_start }\OtherTok{\textless{}{-}} \FloatTok{0.3}                                        \CommentTok{\# Средняя скорость роста}
\NormalTok{t0\_start }\OtherTok{\textless{}{-}} \SpecialCharTok{{-}}\FloatTok{0.5}                                      \CommentTok{\# Гипотетический возраст}

\CommentTok{\# Подгонка модели с увеличенным числом итераций}
\NormalTok{model }\OtherTok{\textless{}{-}} \FunctionTok{nls}\NormalTok{(}
\NormalTok{  length }\SpecialCharTok{\textasciitilde{}}\NormalTok{ L\_inf }\SpecialCharTok{*}\NormalTok{ (}\DecValTok{1} \SpecialCharTok{{-}} \FunctionTok{exp}\NormalTok{(}\SpecialCharTok{{-}}\NormalTok{k }\SpecialCharTok{*}\NormalTok{ (age\_num }\SpecialCharTok{{-}}\NormalTok{ t0))),}
  \AttributeTok{data =}\NormalTok{ data\_clean,}
  \AttributeTok{start =} \FunctionTok{list}\NormalTok{(}\AttributeTok{L\_inf =}\NormalTok{ L\_inf\_start, }\AttributeTok{k =}\NormalTok{ k\_start, }\AttributeTok{t0 =}\NormalTok{ t0\_start),}
  \AttributeTok{control =} \FunctionTok{nls.control}\NormalTok{(}\AttributeTok{maxiter =} \DecValTok{200}\NormalTok{, }\AttributeTok{warnOnly =} \ConstantTok{TRUE}\NormalTok{)  }\CommentTok{\# Увеличиваем лимит итераций}
\NormalTok{)}

\CommentTok{\# Вывод результатов}
\FunctionTok{summary}\NormalTok{(model)}

\CommentTok{\# Создание последовательности возрастов для предсказания}
\NormalTok{age\_seq }\OtherTok{\textless{}{-}} \FunctionTok{seq}\NormalTok{(}\FunctionTok{min}\NormalTok{(data\_clean}\SpecialCharTok{$}\NormalTok{age\_num), }\FunctionTok{max}\NormalTok{(data\_clean}\SpecialCharTok{$}\NormalTok{age\_num), }\AttributeTok{by =} \FloatTok{0.1}\NormalTok{)}

\CommentTok{\# Предсказание значений длины}
\NormalTok{length\_pred }\OtherTok{\textless{}{-}} \FunctionTok{predict}\NormalTok{(model, }\AttributeTok{newdata =} \FunctionTok{data.frame}\NormalTok{(}\AttributeTok{age\_num =}\NormalTok{ age\_seq))}

\CommentTok{\# Построение графика}
\FunctionTok{ggplot}\NormalTok{(data\_clean, }\FunctionTok{aes}\NormalTok{(}\AttributeTok{x =}\NormalTok{ age\_num, }\AttributeTok{y =}\NormalTok{ length)) }\SpecialCharTok{+}
  \FunctionTok{geom\_point}\NormalTok{(}\FunctionTok{aes}\NormalTok{(}\AttributeTok{color =}\NormalTok{ age), }\AttributeTok{alpha =} \FloatTok{0.7}\NormalTok{) }\SpecialCharTok{+}
  \FunctionTok{geom\_line}\NormalTok{(}\AttributeTok{data =} \FunctionTok{data.frame}\NormalTok{(}\AttributeTok{age\_num =}\NormalTok{ age\_seq, }\AttributeTok{length =}\NormalTok{ length\_pred), }
            \FunctionTok{aes}\NormalTok{(}\AttributeTok{x =}\NormalTok{ age\_num, }\AttributeTok{y =}\NormalTok{ length), }\AttributeTok{color =} \StringTok{"red"}\NormalTok{, }\AttributeTok{linewidth =} \FloatTok{1.2}\NormalTok{) }\SpecialCharTok{+}
  \FunctionTok{labs}\NormalTok{(}
    \AttributeTok{title =} \StringTok{"Рост креветок по уравнению Берталанфи"}\NormalTok{,}
    \AttributeTok{x =} \StringTok{"Возраст (годы)"}\NormalTok{,}
    \AttributeTok{y =} \StringTok{"Длина карапакса (мм)"}\NormalTok{,}
    \AttributeTok{color =} \StringTok{"Возрастная группа"}
\NormalTok{  ) }\SpecialCharTok{+}
  \FunctionTok{theme\_minimal}\NormalTok{()}

\CommentTok{\# Сохранение графика}
\FunctionTok{ggsave}\NormalTok{(}\StringTok{"bertalanffy\_model.png"}\NormalTok{, }\AttributeTok{width =} \DecValTok{8}\NormalTok{, }\AttributeTok{height =} \DecValTok{6}\NormalTok{)}
\end{Highlighting}
\end{Shaded}

\begin{figure}[H]

{\centering \includegraphics[width=0.6\linewidth,height=\textheight,keepaspectratio]{images/bertalanffy_model.PNG}

}

\caption{Рис. 1.6: Рост креветок по уравнению Берталанфи}

\end{figure}%

\section{Огива, логистическая кривая и 50\%-ное
созревание}\label{ux43eux433ux438ux432ux430-ux43bux43eux433ux438ux441ux442ux438ux447ux435ux441ux43aux430ux44f-ux43aux440ux438ux432ux430ux44f-ux438-50-ux43dux43eux435-ux441ux43eux437ux440ux435ux432ux430ux43dux438ux435}

Логистическая регрессия удобна там, где исход --- бинарный: созрел/не
созрел, самка/самец. Для протоандрической креветки вероятность быть
самкой естественно растёт с длиной, и логистическая кривая описывает это
гладким переходом от 0 к 1; её центральная точка даёт L50 = −β0/β1 ---
длину, при которой половина особей уже самки. Огива --- это та же
история, но накопительно: как доля самок нарастает по мере увеличения
длины; она наглядна для сравнения годов/районов и проверки сдвигов
зрелости. Качество модели удобно проверять ROC/AUC: AUC ≈ 0.9+ означает,
что длина хорошо ранжирует вероятность женского пола, но не отменяет
проверки калибровки. Знак и величина β1 интерпретируются просто:
положительный β1 --- с каждым миллиметром шансы быть самкой растут,
exp(β1) --- во сколько раз растут эти шансы на единицу длины.
Биологически L50 концентрирует ключевой сигнал: при стабильных условиях
он держится в узком интервале (для \emph{Pandalus borealis} около 25--28
мм), а его снижение обычно маркирует стресс среды или избирательный
вылов, «подталкивающий» к более раннему созреванию. В прикладном учёте
это даёт два практичных числа --- L50 и AUC --- и две опоры для
интерпретации: насколько резко идёт переход (крутизна кривой) и
насколько надёжен прогноз (дискриминация и калибровка).

Логистическая кривая --- ключевой инструмент для моделирования бинарных
процессов, таких как созревание или смена пола у организмов. В случае
протоандрических креветок (\emph{Pandalus borealis}), которые меняют пол
с возрастом, зависимость вероятности быть самкой от длины карапакса
можно описать логистической функцией:

\[
P(F) = \frac{1}{1 + e^{-(\beta_0 + \beta_1 \cdot длина)}}
\]

где \emph{P(F)} --- вероятность принадлежности к женскому полу,
\emph{β\textsubscript{0}} --- интерсепт, \emph{β\textsubscript{1}} ---
коэффициент влияния длины.

Точка перегиба логистической кривой соответствует длине, при которой
вероятность быть самкой равна 50\%: \[
L_{50} = -\frac{\beta_0}{\beta_1}
\]

\begin{figure}[H]

{\centering \includegraphics[width=0.6\linewidth,height=\textheight,keepaspectratio]{images/logistic_model_shrimp.PNG}

}

\caption{Рис. 1.7: Логистическая кривая}

\end{figure}%

Огива (кумулятивная кривая) показывает накопление вероятности с
увеличением длины. Для анализа созревания её можно построить через
интеграл логистической функции. Визуально она демонстрирует, как доля
самок возрастает с размером.

\begin{figure}[H]

{\centering \includegraphics[width=0.6\linewidth,height=\textheight,keepaspectratio]{images/ogive_shrimp.PNG}

}

\caption{Рис. 1.8: Огива}

\end{figure}%

\subsection{\texorpdfstring{\textbf{Оценка
модели}}{Оценка модели}}\label{ux43eux446ux435ux43dux43aux430-ux43cux43eux434ux435ux43bux438}

\begin{enumerate}
\def\labelenumi{\arabic{enumi}.}
\item
  \textbf{ROC-кривая и AUC}:

  \begin{itemize}
  \item
    Площадь под ROC-кривой (AUC) \textgreater0.7 указывает на хорошую
    предсказательную способность модели.
  \item
    Значение AUC = 0.94(пример из кода) подтверждает сильную связь длины
    и пола.
  \end{itemize}
\end{enumerate}

\begin{figure}[H]

{\centering \includegraphics[width=0.6\linewidth,height=\textheight,keepaspectratio]{images/ROC_shrimp.PNG}

}

\caption{Рис. 1.9: ROC-кривая и AUC}

\end{figure}%

\begin{enumerate}
\def\labelenumi{\arabic{enumi}.}
\setcounter{enumi}{1}
\item
  \textbf{Интерпретация коэффициентов}:

  \begin{itemize}
  \item
    Положительный \emph{β\textsubscript{1}} означает: с ростом длины
    вероятность быть самкой увеличивается.
  \item
    Например, \emph{β\textsubscript{1}}=0.25 → увеличение длины на 1 мм
    повышает шансы в e\textsuperscript{0.25}≈1.28 раза.
  \end{itemize}
\end{enumerate}

\subsection{\texorpdfstring{\textbf{Биологический
контекст}}{Биологический контекст}}\label{ux431ux438ux43eux43bux43eux433ux438ux447ux435ux441ux43aux438ux439-ux43aux43eux43dux442ux435ux43aux441ux442}

\begin{itemize}
\item
  \textbf{Протоандрический гермафродитизм}: У креветок смена пола с
  самцов на самок происходит при достижении критического размера
  (\textasciitilde25-28 мм).
\item
  \textbf{L50 как индикатор}: Снижение \emph{L\textsubscript{50}} в
  популяции может сигнализировать о стрессовых условиях (перелов,
  изменение среды), ускоряющих созревание.
\end{itemize}

\begin{Shaded}
\begin{Highlighting}[]
\CommentTok{\# Установка рабочей директории}
\FunctionTok{setwd}\NormalTok{(}\StringTok{"C:/TEXTBOOK/"}\NormalTok{)}

\CommentTok{\# Загрузка библиотек}
\FunctionTok{library}\NormalTok{(tidyverse)}
\FunctionTok{library}\NormalTok{(pROC)}
\FunctionTok{library}\NormalTok{(ggplot2)}

\CommentTok{\# Загрузка данных}
\NormalTok{data }\OtherTok{\textless{}{-}} \FunctionTok{read\_csv}\NormalTok{(}\StringTok{"shrimp\_catch.csv"}\NormalTok{)}

\CommentTok{\# 1. Предобработка данных {-}{-}{-}{-}{-}{-}{-}{-}{-}{-}{-}{-}{-}{-}{-}{-}{-}{-}{-}{-}{-}{-}{-}{-}{-}{-}{-}{-}{-}{-}{-}{-}{-}{-}{-}{-}{-}{-}{-}{-}{-}{-}{-}{-}{-}{-}{-}{-}{-}{-}{-}{-}{-}}
\CommentTok{\# Удаление аутлаеров методом IQR}
\NormalTok{Q1 }\OtherTok{\textless{}{-}} \FunctionTok{quantile}\NormalTok{(data}\SpecialCharTok{$}\NormalTok{length, }\FloatTok{0.25}\NormalTok{)}
\NormalTok{Q3 }\OtherTok{\textless{}{-}} \FunctionTok{quantile}\NormalTok{(data}\SpecialCharTok{$}\NormalTok{length, }\FloatTok{0.75}\NormalTok{)}
\NormalTok{IQR }\OtherTok{\textless{}{-}}\NormalTok{ Q3 }\SpecialCharTok{{-}}\NormalTok{ Q1}
\NormalTok{data\_clean }\OtherTok{\textless{}{-}}\NormalTok{ data }\SpecialCharTok{\%\textgreater{}\%}
  \FunctionTok{filter}\NormalTok{(length }\SpecialCharTok{\textgreater{}=}\NormalTok{ Q1 }\SpecialCharTok{{-}} \FloatTok{1.5}\SpecialCharTok{*}\NormalTok{IQR }\SpecialCharTok{\&}\NormalTok{ length }\SpecialCharTok{\textless{}=}\NormalTok{ Q3 }\SpecialCharTok{+} \FloatTok{1.5}\SpecialCharTok{*}\NormalTok{IQR)}

\CommentTok{\# 2. Логистическая регрессия {-}{-}{-}{-}{-}{-}{-}{-}{-}{-}{-}{-}{-}{-}{-}{-}{-}{-}{-}{-}{-}{-}{-}{-}{-}{-}{-}{-}{-}{-}{-}{-}{-}{-}{-}{-}{-}{-}{-}{-}{-}{-}{-}{-}{-}{-}{-}{-}{-}{-}}
\CommentTok{\# Преобразование пола в бинарную переменную}
\NormalTok{data\_clean}\SpecialCharTok{$}\NormalTok{sex\_binary }\OtherTok{\textless{}{-}} \FunctionTok{ifelse}\NormalTok{(data\_clean}\SpecialCharTok{$}\NormalTok{sex }\SpecialCharTok{==} \StringTok{"F"}\NormalTok{, }\DecValTok{1}\NormalTok{, }\DecValTok{0}\NormalTok{)}

\CommentTok{\# Подгонка модели}
\NormalTok{model\_logit }\OtherTok{\textless{}{-}} \FunctionTok{glm}\NormalTok{(sex\_binary }\SpecialCharTok{\textasciitilde{}}\NormalTok{ length, }
                   \AttributeTok{data =}\NormalTok{ data\_clean, }
                   \AttributeTok{family =} \FunctionTok{binomial}\NormalTok{(}\AttributeTok{link =} \StringTok{"logit"}\NormalTok{))}

\CommentTok{\# Расчет коэффициентов}
\NormalTok{beta0 }\OtherTok{\textless{}{-}} \FunctionTok{coef}\NormalTok{(model\_logit)[}\DecValTok{1}\NormalTok{]}
\NormalTok{beta1 }\OtherTok{\textless{}{-}} \FunctionTok{coef}\NormalTok{(model\_logit)[}\DecValTok{2}\NormalTok{]}

\CommentTok{\# Вычисление L50 (длина 50\% созревания)}
\NormalTok{L50 }\OtherTok{\textless{}{-}} \FunctionTok{round}\NormalTok{(}\SpecialCharTok{{-}}\NormalTok{beta0}\SpecialCharTok{/}\NormalTok{beta1, }\DecValTok{1}\NormalTok{)}

\CommentTok{\# 3. Визуализация {-}{-}{-}{-}{-}{-}{-}{-}{-}{-}{-}{-}{-}{-}{-}{-}{-}{-}{-}{-}{-}{-}{-}{-}{-}{-}{-}{-}{-}{-}{-}{-}{-}{-}{-}{-}{-}{-}{-}{-}{-}{-}{-}{-}{-}{-}{-}{-}{-}{-}{-}{-}{-}{-}{-}{-}{-}{-}{-}{-}}
\CommentTok{\# Логистическая кривая}
\FunctionTok{ggplot}\NormalTok{(data\_clean, }\FunctionTok{aes}\NormalTok{(}\AttributeTok{x =}\NormalTok{ length, }\AttributeTok{y =}\NormalTok{ sex\_binary)) }\SpecialCharTok{+}
  \FunctionTok{geom\_point}\NormalTok{(}\FunctionTok{aes}\NormalTok{(}\AttributeTok{color =}\NormalTok{ sex), }\AttributeTok{alpha =} \FloatTok{0.6}\NormalTok{, }\AttributeTok{size =} \DecValTok{2}\NormalTok{) }\SpecialCharTok{+}
  \FunctionTok{geom\_line}\NormalTok{(}\FunctionTok{aes}\NormalTok{(}\AttributeTok{y =} \FunctionTok{predict}\NormalTok{(model\_logit, }\AttributeTok{type =} \StringTok{"response"}\NormalTok{)), }
            \AttributeTok{color =} \StringTok{"\#D81B60"}\NormalTok{, }\AttributeTok{linewidth =} \FloatTok{1.5}\NormalTok{) }\SpecialCharTok{+}
  \FunctionTok{geom\_vline}\NormalTok{(}\AttributeTok{xintercept =}\NormalTok{ L50, }\AttributeTok{linetype =} \StringTok{"dashed"}\NormalTok{, }\AttributeTok{color =} \StringTok{"\#1E88E5"}\NormalTok{) }\SpecialCharTok{+}
  \FunctionTok{annotate}\NormalTok{(}\StringTok{"text"}\NormalTok{, }\AttributeTok{x =}\NormalTok{ L50 }\SpecialCharTok{+} \DecValTok{2}\NormalTok{, }\AttributeTok{y =} \FloatTok{0.2}\NormalTok{, }
           \AttributeTok{label =} \FunctionTok{paste}\NormalTok{(}\StringTok{"L50 ="}\NormalTok{, L50, }\StringTok{"мм"}\NormalTok{), }\AttributeTok{color =} \StringTok{"\#1E88E5"}\NormalTok{) }\SpecialCharTok{+}
  \FunctionTok{scale\_color\_manual}\NormalTok{(}\AttributeTok{values =} \FunctionTok{c}\NormalTok{(}\StringTok{"\#FFC107"}\NormalTok{, }\StringTok{"\#1976D2"}\NormalTok{)) }\SpecialCharTok{+}
  \FunctionTok{labs}\NormalTok{(}
    \AttributeTok{title =} \StringTok{"Зависимость пола от длины карапакса"}\NormalTok{,}
    \AttributeTok{subtitle =} \StringTok{"Логистическая регрессия с 50\%{-}ной точкой созревания"}\NormalTok{,}
    \AttributeTok{x =} \StringTok{"Длина карапакса (мм)"}\NormalTok{,}
    \AttributeTok{y =} \StringTok{"Вероятность быть самкой (P(F))"}
\NormalTok{  ) }\SpecialCharTok{+}
  \FunctionTok{theme\_minimal}\NormalTok{(}\AttributeTok{base\_size =} \DecValTok{12}\NormalTok{)}

\CommentTok{\# Огива (кумулятивное распределение)}
\NormalTok{data\_ogive }\OtherTok{\textless{}{-}}\NormalTok{ data\_clean }\SpecialCharTok{\%\textgreater{}\%}
  \FunctionTok{arrange}\NormalTok{(length) }\SpecialCharTok{\%\textgreater{}\%}
  \FunctionTok{mutate}\NormalTok{(}
    \AttributeTok{cum\_females =} \FunctionTok{cumsum}\NormalTok{(sex\_binary),}
    \AttributeTok{cum\_prob =}\NormalTok{ cum\_females }\SpecialCharTok{/} \FunctionTok{max}\NormalTok{(cum\_females)}
\NormalTok{  )}

\FunctionTok{ggplot}\NormalTok{(data\_ogive, }\FunctionTok{aes}\NormalTok{(}\AttributeTok{x =}\NormalTok{ length, }\AttributeTok{y =}\NormalTok{ cum\_prob)) }\SpecialCharTok{+}
  \FunctionTok{geom\_line}\NormalTok{(}\AttributeTok{color =} \StringTok{"\#4CAF50"}\NormalTok{, }\AttributeTok{linewidth =} \FloatTok{1.5}\NormalTok{) }\SpecialCharTok{+}
  \FunctionTok{geom\_vline}\NormalTok{(}\AttributeTok{xintercept =}\NormalTok{ L50, }\AttributeTok{linetype =} \StringTok{"dashed"}\NormalTok{, }\AttributeTok{color =} \StringTok{"\#1E88E5"}\NormalTok{) }\SpecialCharTok{+}
  \FunctionTok{geom\_hline}\NormalTok{(}\AttributeTok{yintercept =} \FloatTok{0.5}\NormalTok{, }\AttributeTok{linetype =} \StringTok{"dotted"}\NormalTok{, }\AttributeTok{color =} \StringTok{"\#757575"}\NormalTok{) }\SpecialCharTok{+}
  \FunctionTok{annotate}\NormalTok{(}\StringTok{"text"}\NormalTok{, }\AttributeTok{x =}\NormalTok{ L50 }\SpecialCharTok{+} \DecValTok{2}\NormalTok{, }\AttributeTok{y =} \FloatTok{0.55}\NormalTok{, }
           \AttributeTok{label =} \FunctionTok{paste}\NormalTok{(}\StringTok{"50\% созревание при"}\NormalTok{, L50, }\StringTok{"мм"}\NormalTok{), }\AttributeTok{color =} \StringTok{"\#1E88E5"}\NormalTok{) }\SpecialCharTok{+}
  \FunctionTok{scale\_y\_continuous}\NormalTok{(}\AttributeTok{labels =}\NormalTok{ scales}\SpecialCharTok{::}\NormalTok{percent) }\SpecialCharTok{+}
  \FunctionTok{labs}\NormalTok{(}
    \AttributeTok{title =} \StringTok{"Огива: Кумулятивное распределение самок"}\NormalTok{,}
    \AttributeTok{x =} \StringTok{"Длина карапакса (мм)"}\NormalTok{,}
    \AttributeTok{y =} \StringTok{"Накопленная доля самок"}
\NormalTok{  ) }\SpecialCharTok{+}
  \FunctionTok{theme\_minimal}\NormalTok{(}\AttributeTok{base\_size =} \DecValTok{12}\NormalTok{)}

\CommentTok{\# 4. Оценка модели {-}{-}{-}{-}{-}{-}{-}{-}{-}{-}{-}{-}{-}{-}{-}{-}{-}{-}{-}{-}{-}{-}{-}{-}{-}{-}{-}{-}{-}{-}{-}{-}{-}{-}{-}{-}{-}{-}{-}{-}{-}{-}{-}{-}{-}{-}{-}{-}{-}{-}{-}{-}{-}{-}{-}{-}{-}{-}{-}}
\CommentTok{\# ROC{-}анализ}
\NormalTok{roc\_obj }\OtherTok{\textless{}{-}} \FunctionTok{roc}\NormalTok{(data\_clean}\SpecialCharTok{$}\NormalTok{sex\_binary, }\FunctionTok{predict}\NormalTok{(model\_logit, }\AttributeTok{type =} \StringTok{"response"}\NormalTok{))}
\NormalTok{auc\_value }\OtherTok{\textless{}{-}} \FunctionTok{round}\NormalTok{(}\FunctionTok{auc}\NormalTok{(roc\_obj), }\DecValTok{2}\NormalTok{)}

\CommentTok{\# График ROC{-}кривой}
\FunctionTok{plot}\NormalTok{(roc\_obj, }\AttributeTok{col =} \StringTok{"\#E53935"}\NormalTok{, }\AttributeTok{main =} \FunctionTok{paste}\NormalTok{(}\StringTok{"ROC{-}кривая (AUC ="}\NormalTok{, auc\_value, }\StringTok{")"}\NormalTok{))}

\CommentTok{\# 5. Сохранение результатов {-}{-}{-}{-}{-}{-}{-}{-}{-}{-}{-}{-}{-}{-}{-}{-}{-}{-}{-}{-}{-}{-}{-}{-}{-}{-}{-}{-}{-}{-}{-}{-}{-}{-}{-}{-}{-}{-}{-}{-}{-}{-}{-}{-}{-}{-}{-}{-}{-}{-}}
\FunctionTok{ggsave}\NormalTok{(}\StringTok{"logistic\_curve.png"}\NormalTok{, }\AttributeTok{width =} \DecValTok{8}\NormalTok{, }\AttributeTok{height =} \DecValTok{6}\NormalTok{, }\AttributeTok{dpi =} \DecValTok{300}\NormalTok{)}
\FunctionTok{ggsave}\NormalTok{(}\StringTok{"ogive\_curve.png"}\NormalTok{, }\AttributeTok{width =} \DecValTok{8}\NormalTok{, }\AttributeTok{height =} \DecValTok{6}\NormalTok{, }\AttributeTok{dpi =} \DecValTok{300}\NormalTok{)}

\CommentTok{\# Вывод ключевых метрик}
\FunctionTok{cat}\NormalTok{(}\StringTok{"Результаты анализа:}\SpecialCharTok{\textbackslash{}n}\StringTok{"}\NormalTok{)}
\FunctionTok{cat}\NormalTok{(}\StringTok{"{-} Длина 50\%{-}ного созревания (L50):"}\NormalTok{, L50, }\StringTok{"мм}\SpecialCharTok{\textbackslash{}n}\StringTok{"}\NormalTok{)}
\FunctionTok{cat}\NormalTok{(}\StringTok{"{-} AUC модели:"}\NormalTok{, auc\_value, }\StringTok{"}\SpecialCharTok{\textbackslash{}n}\StringTok{"}\NormalTok{)}
\FunctionTok{cat}\NormalTok{(}\StringTok{"{-} Коэффициенты модели:}\SpecialCharTok{\textbackslash{}n}\StringTok{"}\NormalTok{)}
\FunctionTok{cat}\NormalTok{(}\StringTok{"  Intercept (β0):"}\NormalTok{, }\FunctionTok{round}\NormalTok{(beta0, }\DecValTok{2}\NormalTok{), }\StringTok{"}\SpecialCharTok{\textbackslash{}n}\StringTok{"}\NormalTok{)}
\FunctionTok{cat}\NormalTok{(}\StringTok{"  Slope (β1):"}\NormalTok{, }\FunctionTok{round}\NormalTok{(beta1, }\DecValTok{2}\NormalTok{), }\StringTok{"}\SpecialCharTok{\textbackslash{}n}\StringTok{"}\NormalTok{)}
\end{Highlighting}
\end{Shaded}

\section{Сравнение групп, параметров,
моделей}\label{ux441ux440ux430ux432ux43dux435ux43dux438ux435-ux433ux440ux443ux43fux43f-ux43fux430ux440ux430ux43cux435ux442ux440ux43eux432-ux43cux43eux434ux435ux43bux435ux439}

Сравнивать группы --- это не про охоту за маленькими p-value, а про
проверяемые ответы на конкретные биологические вопросы. «Самки длиннее
самцов?» --- переводим в аккуратную статистическую формулировку,
начинаем с гигиены данных и только потом подбираем тест. Предобработка
банальна, но критична: убираем очевидные аутлаеры по понятному правилу
(IQR или заранее согласованный протокол), не «чистим» хвосты до
совершенства, сохраняем независимость наблюдений. Разбиваем выборку на
подмножества по полу, проверяем типы переменных, смотрим на формы
распределений и пропуски. И дальше --- не прыжок к t‑тесту, а короткая
остановка у предпосылок: нормальность и гомогенность дисперсий --- это
про остатки и разумность аппроксимации, а не про «магическое число
0.05». При несхожих дисперсиях уместнее Уэлч, при явной ненормальности и
неробастности --- Манн--Уитни, а при больших n классический t‑тест часто
держится благодаря центральной предельной теореме. В любом случае голые
p‑значения не заканчивают разговор: эффект размера (Cohen's d) и
доверительные интервалы говорят «на сколько», а не только «есть/нет».

Визуализация в этом месте --- не иллюстрация, а часть доказательства.
Boxplot/violin помогают увидеть медианы, разброс и асимметрию;
добавленная на график оценка p‑value дисциплинирует интерпретацию, но не
подменяет её. Полезно в той же системе координат показать точки, чтобы
помнить: каждая точка --- отдельная особь, а не абстрактная «генеральная
совокупность». И если позволить себе короткую «сапольскину» ремарку:
мозг с удовольствием «видит» разницу там, где её нет, поэтому лучше
сначала смотреть на график, потом на число, а не наоборот.

Когда вопрос --- уже не «кто крупнее», а «кто растёт быстрее», сравнение
средних сменяется сравнением параметров модели. Самый прозрачный путь
--- объединённая линейная модель с взаимодействием: length
\textasciitilde{} age * sex. Значимый коэффициент при взаимодействии ---
это формализованная фраза «наклоны различаются». Диагностика здесь
важнее, чем когда‑либо: линейность, разброс остатков, потенциальные
leverage‑точки. Альтернатива --- раздельные модели по полу и прямое
сравнение наклонов через тест Вальда; он удобен как независимая проверка
и часто даёт те же выводы, что и взаимодействие, если структура данных
не экзотична. Интерпретация должна оставаться биологической:
различающиеся наклоны --- это не «магия пола», а потенциальная разница в
темпе роста, доступе к корму или сезоне отбора проб.

Дальше мы неизбежно приходим к форме связи «вес--длина». Линейная модель
соблазнительно проста, но биологически мир чаще степенной: масса
масштабируется примерно как длина в степени 3, с поправками на форму и
состояние. Полиномиальная регрессия третьего порядка часто выигрывает в
AIC и R², потому что ловит сгибы и плечи; у неё есть и оборотная сторона
--- склонность к переобучению и слабая интерпретируемость коэффициентов.
Степенная модель почти всегда немного проигрывает по «сухим метрикам»,
зато даёт ясный смысл: параметр b близок к 3 --- всё ожидаемо; заметное
отклонение --- есть предмет для обсуждения физиологии, питания,
сезонности. Какой из подходов «лучший»? Тот, у которого остатки ведут
себя прилично, AIC не кричит о лишней сложности, а биолог рассказывает
связную историю, не пряча глаза. Хорошая практика --- сопоставить все
три, показать таблицу R²/AIC, приложить графики остатков и проговорить
компромисс между точностью и объяснимостью.

И в сравнении групп, и в сравнении параметров, и в выборе модели
действуют три простых правила. Первое --- формулируйте вопрос до теста:
это экономит десятки необязательных проверок. Второе --- показывайте
эффект с интервалами: «на сколько» важнее «насколько значимо». Третье
--- проверяйте устойчивость: замены теста (t ↔ Уэлч ↔ Манн--Уитни),
альтернативная спецификация модели, бутстрэп интервалов --- всё это
помогает отличить сигнал от удачного совпадения. И, наконец, не
забывайте про контекст отбора проб: если улов по орудиям и глубинам
неоднороден, то и выводы про «среднего самца» или «типичную самку» легко
превращаются в выводы про «типичный улов». Статистическая аккуратность
здесь --- это не педантизм, а способ говорить о биологии без самообмана.

\subsection{Сравнение групп (на примере самцов и
самок)}\label{ux441ux440ux430ux432ux43dux435ux43dux438ux435-ux433ux440ux443ux43fux43f-ux43dux430-ux43fux440ux438ux43cux435ux440ux435-ux441ux430ux43cux446ux43eux432-ux438-ux441ux430ux43cux43eux43a}

Рассмотрим методы сравнения количественных характеристик (длина, вес)
между самцами и самками северной креветки. Анализ включает проверку
нормальности распределения, выбор подходящего статистического теста и
визуализацию различий.

\subsubsection{Подготовка
данных}\label{ux43fux43eux434ux433ux43eux442ux43eux432ux43aux430-ux434ux430ux43dux43dux44bux445}

Загрузим данные и выделим подвыборки для самцов и самок:

\begin{Shaded}
\begin{Highlighting}[]
\CommentTok{\# Установка рабочей директории}
\FunctionTok{setwd}\NormalTok{(}\StringTok{"C:/TEXTBOOK/"}\NormalTok{)}

\CommentTok{\# Загрузка библиотек  }
\FunctionTok{library}\NormalTok{(tidyverse)  }
\FunctionTok{library}\NormalTok{(ggplot2)  }
\FunctionTok{library}\NormalTok{(rstatix)}
\FunctionTok{library}\NormalTok{(ggpubr)}

\CommentTok{\# Загрузка данных  }
\NormalTok{data }\OtherTok{\textless{}{-}} \FunctionTok{read\_csv}\NormalTok{(}\StringTok{"shrimp\_catch.csv"}\NormalTok{) }\SpecialCharTok{\%\textgreater{}\%}
  \FunctionTok{filter}\NormalTok{(}\SpecialCharTok{!}\NormalTok{id }\SpecialCharTok{\%in\%} \FunctionTok{c}\NormalTok{(}\DecValTok{10}\NormalTok{, }\DecValTok{50}\NormalTok{))  }\CommentTok{\# Удаление аномальных наблюдений }

\CommentTok{\# Фильтрация данных по полу  }
\NormalTok{males }\OtherTok{\textless{}{-}}\NormalTok{ data }\SpecialCharTok{\%\textgreater{}\%} \FunctionTok{filter}\NormalTok{(sex }\SpecialCharTok{==} \StringTok{"M"}\NormalTok{)  }
\NormalTok{females }\OtherTok{\textless{}{-}}\NormalTok{ data }\SpecialCharTok{\%\textgreater{}\%} \FunctionTok{filter}\NormalTok{(sex }\SpecialCharTok{==} \StringTok{"F"}\NormalTok{) }
\end{Highlighting}
\end{Shaded}

\subsubsection{Проверка нормальности
распределения}\label{ux43fux440ux43eux432ux435ux440ux43aux430-ux43dux43eux440ux43cux430ux43bux44cux43dux43eux441ux442ux438-ux440ux430ux441ux43fux440ux435ux434ux435ux43bux435ux43dux438ux44f}

Перед сравнением групп проверим, соответствуют ли данные нормальному
распределению (тест Шапиро-Уилка):

\begin{Shaded}
\begin{Highlighting}[]
\CommentTok{\# Проверка нормальности для длины самцов  }
\FunctionTok{shapiro\_test}\NormalTok{(males}\SpecialCharTok{$}\NormalTok{length)  }
\CommentTok{\# Проверка нормальности для длины самок  }
\FunctionTok{shapiro\_test}\NormalTok{(females}\SpecialCharTok{$}\NormalTok{length) }
\end{Highlighting}
\end{Shaded}

Если p-value \textgreater{} 0.05, распределение считается нормальным. В
противном случае используем непараметрические методы.

\subsubsection{Сравнение средних
значений}\label{ux441ux440ux430ux432ux43dux435ux43dux438ux435-ux441ux440ux435ux434ux43dux438ux445-ux437ux43dux430ux447ux435ux43dux438ux439}

Если данные нормальны: t-тест

\begin{Shaded}
\begin{Highlighting}[]
\CommentTok{\# T{-}тест для сравнения длин самцов и самок  }
\NormalTok{t\_test\_result }\OtherTok{\textless{}{-}} \FunctionTok{t\_test}\NormalTok{(length }\SpecialCharTok{\textasciitilde{}}\NormalTok{ sex, }\AttributeTok{data =}\NormalTok{ data)  }
\NormalTok{t\_test\_result }
\end{Highlighting}
\end{Shaded}

Если данные не нормальны: U-тест Манна-Уитни

\begin{Shaded}
\begin{Highlighting}[]
\CommentTok{\# U{-}тест для сравнения длин самцов и самок  }
\NormalTok{mannwhitney\_result }\OtherTok{\textless{}{-}} \FunctionTok{wilcox\_test}\NormalTok{(length }\SpecialCharTok{\textasciitilde{}}\NormalTok{ sex, }\AttributeTok{data =}\NormalTok{ data)  }
\NormalTok{mannwhitney\_result }
\end{Highlighting}
\end{Shaded}

\subsubsection{Эффект размера (коэффициент
Коэна)}\label{ux44dux444ux444ux435ux43aux442-ux440ux430ux437ux43cux435ux440ux430-ux43aux43eux44dux444ux444ux438ux446ux438ux435ux43dux442-ux43aux43eux44dux43dux430}

Для оценки практической значимости различий рассчитаем коэффициент
Коэна:

\begin{Shaded}
\begin{Highlighting}[]
\CommentTok{\# Расчет коэффициента Коэна  }
\NormalTok{cohens\_d\_result }\OtherTok{\textless{}{-}} \FunctionTok{cohens\_d}\NormalTok{(length }\SpecialCharTok{\textasciitilde{}}\NormalTok{ sex, }\AttributeTok{data =}\NormalTok{ data)  }
\NormalTok{cohens\_d\_result  }
\end{Highlighting}
\end{Shaded}

\begin{itemize}
\item
  \textbf{d \textless{} 0.2} : малый эффект,
\item
  \textbf{d ≈ 0.5} : средний эффект,
\item
  \textbf{d \textgreater{} 0.8} : большой эффект.
\end{itemize}

\subsubsection{\texorpdfstring{\textbf{Визуализация
различий}}{Визуализация различий}}\label{ux432ux438ux437ux443ux430ux43bux438ux437ux430ux446ux438ux44f-ux440ux430ux437ux43bux438ux447ux438ux439}

Построим boxplot для визуального сравнения длин самцов и самок:

\begin{Shaded}
\begin{Highlighting}[]
\FunctionTok{ggplot}\NormalTok{(data, }\FunctionTok{aes}\NormalTok{(}\AttributeTok{x =}\NormalTok{ sex, }\AttributeTok{y =}\NormalTok{ length, }\AttributeTok{fill =}\NormalTok{ sex)) }\SpecialCharTok{+}  
  \FunctionTok{geom\_boxplot}\NormalTok{(}\AttributeTok{color =} \StringTok{"black"}\NormalTok{, }\AttributeTok{alpha =} \FloatTok{0.7}\NormalTok{) }\SpecialCharTok{+}  
  \FunctionTok{stat\_compare\_means}\NormalTok{(}\AttributeTok{method =} \StringTok{"t.test"}\NormalTok{) }\SpecialCharTok{+}  \CommentTok{\# Добавление p{-}value  }
  \FunctionTok{labs}\NormalTok{(}\AttributeTok{title =} \StringTok{"Сравнение длин самцов и самок"}\NormalTok{,  }
       \AttributeTok{x =} \StringTok{"Пол"}\NormalTok{, }\AttributeTok{y =} \StringTok{"Длина карапакса (мм)"}\NormalTok{) }\SpecialCharTok{+}  
  \FunctionTok{theme\_minimal}\NormalTok{() }
\end{Highlighting}
\end{Shaded}

\begin{figure}[H]

{\centering \includegraphics[width=0.6\linewidth,height=\textheight,keepaspectratio]{images/ttest_shrimp.PNG}

}

\caption{Рис. 1.10: Boxplot сравнения длин самцов и самок}

\end{figure}%

\subsubsection{\texorpdfstring{\textbf{Интерпретация
результатов}}{Интерпретация результатов}}\label{ux438ux43dux442ux435ux440ux43fux440ux435ux442ux430ux446ux438ux44f-ux440ux435ux437ux443ux43bux44cux442ux430ux442ux43eux432}

\begin{enumerate}
\def\labelenumi{\arabic{enumi}.}
\item
  Если p-value \textless{} 0.05, различия между группами статистически
  значимы.
\item
  Эффект размера помогает оценить биологическую важность различий.
  Например, если самки значительно крупнее самцов (d = 1.2), это может
  указывать на половой диморфизм, связанный с репродуктивной стратегией.

  \subsubsection{\texorpdfstring{\textbf{Пример полного анализа для
  веса}}{Пример полного анализа для веса}}\label{ux43fux440ux438ux43cux435ux440-ux43fux43eux43bux43dux43eux433ux43e-ux430ux43dux430ux43bux438ux437ux430-ux434ux43bux44f-ux432ux435ux441ux430}
\end{enumerate}

\begin{Shaded}
\begin{Highlighting}[]
\CommentTok{\# Полный анализ для веса  }
\NormalTok{weight\_analysis }\OtherTok{\textless{}{-}}\NormalTok{ data }\SpecialCharTok{\%\textgreater{}\%}  
  \FunctionTok{group\_by}\NormalTok{(sex) }\SpecialCharTok{\%\textgreater{}\%}  
  \FunctionTok{summarise}\NormalTok{(  }
    \AttributeTok{mean\_weight =} \FunctionTok{mean}\NormalTok{(weight),  }
    \AttributeTok{sd\_weight =} \FunctionTok{sd}\NormalTok{(weight),  }
    \AttributeTok{n =} \FunctionTok{n}\NormalTok{()  }
\NormalTok{  ) }\SpecialCharTok{\%\textgreater{}\%}  
  \FunctionTok{mutate}\NormalTok{(  }
    \AttributeTok{t\_test =} \FunctionTok{list}\NormalTok{(}\FunctionTok{t\_test}\NormalTok{(weight }\SpecialCharTok{\textasciitilde{}}\NormalTok{ sex, }\AttributeTok{data =}\NormalTok{ data)),  }
    \AttributeTok{cohens\_d =} \FunctionTok{list}\NormalTok{(}\FunctionTok{cohens\_d}\NormalTok{(weight }\SpecialCharTok{\textasciitilde{}}\NormalTok{ sex, }\AttributeTok{data =}\NormalTok{ data))  }
\NormalTok{  )  }

\CommentTok{\# Вывод результатов  }
\FunctionTok{print}\NormalTok{(weight\_analysis) }

\CommentTok{\# Распределение веса по полу}
\FunctionTok{ggplot}\NormalTok{(data, }\FunctionTok{aes}\NormalTok{(}\AttributeTok{x =} \FunctionTok{factor}\NormalTok{(sex), }\AttributeTok{y =}\NormalTok{ weight, }\AttributeTok{fill =} \FunctionTok{factor}\NormalTok{(sex))) }\SpecialCharTok{+}
  \FunctionTok{geom\_violin}\NormalTok{(}\AttributeTok{trim =} \ConstantTok{FALSE}\NormalTok{, }\AttributeTok{alpha =} \FloatTok{0.7}\NormalTok{) }\SpecialCharTok{+}
  \FunctionTok{geom\_boxplot}\NormalTok{(}\AttributeTok{width =} \FloatTok{0.2}\NormalTok{, }\AttributeTok{outlier.shape =} \ConstantTok{NA}\NormalTok{, }\AttributeTok{fill =} \StringTok{"white"}\NormalTok{) }\SpecialCharTok{+}
  \FunctionTok{labs}\NormalTok{(}\AttributeTok{title =} \StringTok{"Распределение веса по полу"}\NormalTok{, }\AttributeTok{x =} \StringTok{"Пол"}\NormalTok{, }\AttributeTok{y =} \StringTok{"Вес (г)"}\NormalTok{) }\SpecialCharTok{+}
  \FunctionTok{theme\_minimal}\NormalTok{()}
\end{Highlighting}
\end{Shaded}

\begin{figure}[H]

{\centering \includegraphics[width=0.6\linewidth,height=\textheight,keepaspectratio]{images/violin_shrimp.PNG}

}

\caption{Рис. 1.12: Violin plot для визуализации распределения веса}

\end{figure}%

\subsubsection{\texorpdfstring{\textbf{Выводы}}{Выводы}}\label{ux432ux44bux432ux43eux434ux44b}

\begin{enumerate}
\def\labelenumi{\arabic{enumi}.}
\item
  Используйте t-тест для нормальных данных и U-тест для ненормальных.
\item
  Дополните анализ оценкой эффекта размера для биологической
  интерпретации.
\item
  Визуализируйте различия с помощью boxplot или violin plot.
\end{enumerate}

\textbf{Рекомендации} :

\begin{itemize}
\item
  Для многомерных данных (например, одновременное сравнение длины, веса
  и возраста) применяйте MANOVA.
\item
  Если группы неоднородны (например, разный возрастной состав),
  используйте ковариационный анализ (ANCOVA).

  \subsection{\texorpdfstring{\textbf{Что делать, если тест на
  нормальность не пройден для одной из
  групп?}}{Что делать, если тест на нормальность не пройден для одной из групп?}}\label{ux447ux442ux43e-ux434ux435ux43bux430ux442ux44c-ux435ux441ux43bux438-ux442ux435ux441ux442-ux43dux430-ux43dux43eux440ux43cux430ux43bux44cux43dux43eux441ux442ux44c-ux43dux435-ux43fux440ux43eux439ux434ux435ux43d-ux434ux43bux44f-ux43eux434ux43dux43eux439-ux438ux437-ux433ux440ux443ux43fux43f}

  При сравнении количественных характеристик (например, длины карапакса
  у самцов и самок) важно учитывать, соответствуют ли данные нормальному
  распределению. Если тест на нормальность (например, Шапиро-Уилка)
  показывает значимое отклонение от нормальности для одной из групп, это
  влияет на выбор статистического теста и интерпретацию результатов.

  \subsubsection{\texorpdfstring{\textbf{Пример из нашего
  анализа}}{Пример из нашего анализа}}\label{ux43fux440ux438ux43cux435ux440-ux438ux437-ux43dux430ux448ux435ux433ux43e-ux430ux43dux430ux43bux438ux437ux430}

  Мы провели сравнение длины карапакса между самцами и самками:

  \begin{itemize}
  \item
    Для самцов: \textbf{\texttt{shapiro\_test(males\$length)}} → p-value
    = \textbf{0.000574} (нормальность отвергнута).
  \item
    Для самок: \textbf{\texttt{shapiro\_test(females\$length)}} →
    p-value = \textbf{0.891} (нормальность подтверждена).
  \end{itemize}

  Несмотря на это, мы применили как \textbf{t-тест} , так и
  \textbf{U-тест Манна-Уитни} :

  \begin{itemize}
  \item
    \textbf{t-тест} : p-value = 1.46e-40 (значимо).
  \item
    \textbf{U-тест} : p-value = 1.97e-27 (значимо).
  \item
    Коэффициент Коэна: d = 2.14 (большой эффект).
  \end{itemize}

  \subsubsection{\texorpdfstring{\textbf{Почему это
  работает?}}{Почему это работает?}}\label{ux43fux43eux447ux435ux43cux443-ux44dux442ux43e-ux440ux430ux431ux43eux442ux430ux435ux442}

  \begin{enumerate}
  \def\labelenumi{\arabic{enumi}.}
  \item
    \textbf{t-тест устойчив к умеренным отклонениям от нормальности} :

    \begin{itemize}
    \item
      При больших выборках (n \textgreater{} 30) центральная предельная
      теорема позволяет использовать t-тест даже при слабо выраженной
      асимметрии.
    \item
      В вашем случае выборка самцов (n = 149) достаточно велика, чтобы
      компенсировать отклонение от нормальности.
    \end{itemize}
  \item
    \textbf{U-тест Манна-Уитни --- непараметрическая альтернатива} :

    \begin{itemize}
    \item
      Этот тест не требует нормальности и сравнивает ранги, а не средние
      значения.
    \item
      Он подтверждает значимость различий, что усиливает доверие к
      выводу.
    \end{itemize}
  \item
    \textbf{Эффект размера (коэффициент Кобена)} :

    \begin{itemize}
    \tightlist
    \item
      d = 2.14 указывает на \textbf{большой эффект} , что важно для
      биологической интерпретации, даже если p-values значимы.
    \end{itemize}
  \end{enumerate}
\end{itemize}

\subsection{Сравнение параметров (линейные модели для оценки
межгрупповых
различий)}\label{ux441ux440ux430ux432ux43dux435ux43dux438ux435-ux43fux430ux440ux430ux43cux435ux442ux440ux43eux432-ux43bux438ux43dux435ux439ux43dux44bux435-ux43cux43eux434ux435ux43bux438-ux434ux43bux44f-ux43eux446ux435ux43dux43aux438-ux43cux435ux436ux433ux440ux443ux43fux43fux43eux432ux44bux445-ux440ux430ux437ux43bux438ux447ux438ux439}

Для сравнения параметров двух линейных моделей (например, скорости роста
самцов и самок) используем следующий подход.

\begin{Shaded}
\begin{Highlighting}[]
\CommentTok{\# Установка рабочей директории}
\FunctionTok{setwd}\NormalTok{(}\StringTok{"C:/TEXTBOOK/"}\NormalTok{)}

\CommentTok{\# Загрузка библиотек}
\FunctionTok{library}\NormalTok{(tidyverse)}
\FunctionTok{library}\NormalTok{(ggplot2)}
\FunctionTok{library}\NormalTok{(broom)}
\FunctionTok{library}\NormalTok{(knitr)}

\CommentTok{\# Загрузка данных}
\NormalTok{data }\OtherTok{\textless{}{-}} \FunctionTok{read\_csv}\NormalTok{(}\StringTok{"shrimp\_catch.csv"}\NormalTok{) }\SpecialCharTok{\%\textgreater{}\%}
  \FunctionTok{filter}\NormalTok{(}\SpecialCharTok{!}\NormalTok{id }\SpecialCharTok{\%in\%} \FunctionTok{c}\NormalTok{(}\DecValTok{10}\NormalTok{, }\DecValTok{50}\NormalTok{))  }\CommentTok{\# Удаление аномальных наблюдений}

\CommentTok{\# Фильтрация данных по полу}
\NormalTok{data\_male }\OtherTok{\textless{}{-}}\NormalTok{ data }\SpecialCharTok{\%\textgreater{}\%} \FunctionTok{filter}\NormalTok{(sex }\SpecialCharTok{==} \StringTok{"M"}\NormalTok{)}
\NormalTok{data\_female }\OtherTok{\textless{}{-}}\NormalTok{ data }\SpecialCharTok{\%\textgreater{}\%} \FunctionTok{filter}\NormalTok{(sex }\SpecialCharTok{==} \StringTok{"F"}\NormalTok{)}

\CommentTok{\# Построение моделей}
\NormalTok{model\_male }\OtherTok{\textless{}{-}} \FunctionTok{lm}\NormalTok{(length }\SpecialCharTok{\textasciitilde{}}\NormalTok{ age, }\AttributeTok{data =}\NormalTok{ data\_male)}
\NormalTok{model\_female }\OtherTok{\textless{}{-}} \FunctionTok{lm}\NormalTok{(length }\SpecialCharTok{\textasciitilde{}}\NormalTok{ age, }\AttributeTok{data =}\NormalTok{ data\_female)}

\FunctionTok{ggplot}\NormalTok{(data, }\FunctionTok{aes}\NormalTok{(age, length, }\AttributeTok{color =}\NormalTok{ sex)) }\SpecialCharTok{+}
  \FunctionTok{geom\_point}\NormalTok{(}\AttributeTok{alpha =} \FloatTok{0.5}\NormalTok{) }\SpecialCharTok{+}
  \FunctionTok{geom\_smooth}\NormalTok{(}\AttributeTok{method =} \StringTok{"lm"}\NormalTok{, }\AttributeTok{formula =}\NormalTok{ y }\SpecialCharTok{\textasciitilde{}}\NormalTok{ x) }\SpecialCharTok{+}
  \FunctionTok{scale\_color\_manual}\NormalTok{(}\AttributeTok{values =} \FunctionTok{c}\NormalTok{(}\StringTok{"\#E7B800"}\NormalTok{, }\StringTok{"\#00AFBB"}\NormalTok{)) }\SpecialCharTok{+}
  \FunctionTok{labs}\NormalTok{(}\AttributeTok{x =} \StringTok{"Возраст"}\NormalTok{, }\AttributeTok{y =} \StringTok{"Длина (мм)"}\NormalTok{) }\SpecialCharTok{+}
  \FunctionTok{theme\_minimal}\NormalTok{()}
\end{Highlighting}
\end{Shaded}

\begin{figure}[H]

{\centering \includegraphics[width=0.6\linewidth,height=\textheight,keepaspectratio]{images/comparison_shrimp.PNG}

}

\caption{Рис. 1.15: Визуализация моделей}

\end{figure}%

\textbf{Метод 1: Объединенная модель с взаимодействиями}

\begin{Shaded}
\begin{Highlighting}[]
\CommentTok{\# Установка рабочей директории}
\NormalTok{joint\_model }\OtherTok{\textless{}{-}} \FunctionTok{lm}\NormalTok{(length }\SpecialCharTok{\textasciitilde{}}\NormalTok{ age }\SpecialCharTok{*}\NormalTok{ sex, }\AttributeTok{data =}\NormalTok{ data)}
\FunctionTok{summary}\NormalTok{(joint\_model) }\SpecialCharTok{\%\textgreater{}\%} 
\NormalTok{  broom}\SpecialCharTok{::}\FunctionTok{tidy}\NormalTok{() }\SpecialCharTok{\%\textgreater{}\%} 
  \FunctionTok{filter}\NormalTok{(term }\SpecialCharTok{==} \StringTok{"age:sexM"}\NormalTok{) }\SpecialCharTok{\%\textgreater{}\%} 
  \FunctionTok{kable}\NormalTok{(}\AttributeTok{caption =} \StringTok{"Проверка различия наклонов"}\NormalTok{, }\AttributeTok{digits =} \DecValTok{3}\NormalTok{)}
\end{Highlighting}
\end{Shaded}

\begin{Shaded}
\begin{Highlighting}[]
\NormalTok{Table}\SpecialCharTok{:}\NormalTok{ Проверка различия наклонов}

\SpecialCharTok{|}\NormalTok{term     }\SpecialCharTok{|}\NormalTok{ estimate}\SpecialCharTok{|}\NormalTok{ std.error}\SpecialCharTok{|}\NormalTok{ statistic}\SpecialCharTok{|}\NormalTok{ p.value}\SpecialCharTok{|}
\ErrorTok{|:}\SpecialCharTok{{-}{-}{-}{-}{-}{-}{-}{-}}\ErrorTok{|}\SpecialCharTok{{-}{-}{-}{-}{-}{-}{-}{-}}\ErrorTok{:|}\SpecialCharTok{{-}{-}{-}{-}{-}{-}{-}{-}{-}}\ErrorTok{:|}\SpecialCharTok{{-}{-}{-}{-}{-}{-}{-}{-}{-}}\ErrorTok{:|}\SpecialCharTok{{-}{-}{-}{-}{-}{-}{-}}\ErrorTok{:|}
\ErrorTok{|}\NormalTok{age}\SpecialCharTok{:}\NormalTok{sexM }\SpecialCharTok{|}     \FloatTok{1.86}\SpecialCharTok{|}     \FloatTok{0.459}\SpecialCharTok{|}     \FloatTok{4.053}\SpecialCharTok{|}       \DecValTok{0}\SpecialCharTok{|}
\ErrorTok{\textgreater{}} 
\end{Highlighting}
\end{Shaded}

\textbf{Интерпретация:}\\
Значимый коэффициент взаимодействия \textbf{\texttt{age:sexM}} (p
\textless{} 0.05) указывает на статистически значимые различия в
скорости роста между полами.

\textbf{Метод 2: Тест Вальда}

\begin{Shaded}
\begin{Highlighting}[]
\FunctionTok{library}\NormalTok{(car)}
\NormalTok{delta\_beta }\OtherTok{\textless{}{-}} \FunctionTok{coef}\NormalTok{(model\_male)[}\StringTok{"age"}\NormalTok{] }\SpecialCharTok{{-}} \FunctionTok{coef}\NormalTok{(model\_female)[}\StringTok{"age"}\NormalTok{]}
\NormalTok{se\_diff }\OtherTok{\textless{}{-}} \FunctionTok{sqrt}\NormalTok{(}\FunctionTok{vcov}\NormalTok{(model\_male)[}\StringTok{"age"}\NormalTok{,}\StringTok{"age"}\NormalTok{] }\SpecialCharTok{+} \FunctionTok{vcov}\NormalTok{(model\_female)[}\StringTok{"age"}\NormalTok{,}\StringTok{"age"}\NormalTok{])}
\NormalTok{z\_score }\OtherTok{\textless{}{-}}\NormalTok{ delta\_beta }\SpecialCharTok{/}\NormalTok{ se\_diff}
\NormalTok{p\_value }\OtherTok{\textless{}{-}} \DecValTok{2} \SpecialCharTok{*} \FunctionTok{pnorm}\NormalTok{(}\SpecialCharTok{{-}}\FunctionTok{abs}\NormalTok{(z\_score))}

\FunctionTok{cat}\NormalTok{(}\StringTok{"Разница коэффициентов:"}\NormalTok{, }\FunctionTok{round}\NormalTok{(delta\_beta, }\DecValTok{3}\NormalTok{), }
    \StringTok{"}\SpecialCharTok{\textbackslash{}n}\StringTok{Z{-}статистика:"}\NormalTok{, }\FunctionTok{round}\NormalTok{(z\_score, }\DecValTok{3}\NormalTok{),}
    \StringTok{"}\SpecialCharTok{\textbackslash{}n}\StringTok{p{-}value:"}\NormalTok{, }\FunctionTok{format.pval}\NormalTok{(p\_value, }\AttributeTok{digits =} \DecValTok{2}\NormalTok{))}


\NormalTok{comparison\_table }\OtherTok{\textless{}{-}} \FunctionTok{data.frame}\NormalTok{(}
\NormalTok{  Параметр }\OtherTok{=} \FunctionTok{c}\NormalTok{(}\StringTok{"Скорость роста самцов"}\NormalTok{, }\StringTok{"Скорость роста самок"}\NormalTok{, }\StringTok{"Разница"}\NormalTok{),}
\NormalTok{  Значение }\OtherTok{=} \FunctionTok{c}\NormalTok{(}
    \FunctionTok{round}\NormalTok{(}\FunctionTok{coef}\NormalTok{(model\_male)[}\StringTok{"age"}\NormalTok{], }\DecValTok{2}\NormalTok{),}
    \FunctionTok{round}\NormalTok{(}\FunctionTok{coef}\NormalTok{(model\_female)[}\StringTok{"age"}\NormalTok{], }\DecValTok{2}\NormalTok{),}
    \FunctionTok{round}\NormalTok{(delta\_beta, }\DecValTok{2}\NormalTok{)}
\NormalTok{  ),}
  \StringTok{\textasciigrave{}}\AttributeTok{p{-}value}\StringTok{\textasciigrave{}} \OtherTok{=} \FunctionTok{c}\NormalTok{(}
    \FunctionTok{format.pval}\NormalTok{(}\FunctionTok{summary}\NormalTok{(model\_male)}\SpecialCharTok{$}\NormalTok{coefficients[}\StringTok{"age"}\NormalTok{,}\DecValTok{4}\NormalTok{], }\AttributeTok{digits =} \DecValTok{2}\NormalTok{),}
    \FunctionTok{format.pval}\NormalTok{(}\FunctionTok{summary}\NormalTok{(model\_female)}\SpecialCharTok{$}\NormalTok{coefficients[}\StringTok{"age"}\NormalTok{,}\DecValTok{4}\NormalTok{], }\AttributeTok{digits =} \DecValTok{2}\NormalTok{),}
    \FunctionTok{format.pval}\NormalTok{(p\_value, }\AttributeTok{digits =} \DecValTok{2}\NormalTok{)}
\NormalTok{  )}
\NormalTok{)}
\FunctionTok{kable}\NormalTok{(comparison\_table, }\AttributeTok{caption =} \StringTok{"Сравнение коэффициентов роста"}\NormalTok{)}
\end{Highlighting}
\end{Shaded}

Вывод

\begin{Shaded}
\begin{Highlighting}[]
\SpecialCharTok{:}\NormalTok{ Сравнение коэффициентов роста}

\SpecialCharTok{|}\NormalTok{Параметр              }\SpecialCharTok{|}\NormalTok{ Значение}\SpecialCharTok{|}\NormalTok{p.value }\SpecialCharTok{|}
\ErrorTok{|:}\SpecialCharTok{{-}{-}{-}{-}{-}{-}{-}{-}{-}{-}{-}{-}{-}{-}{-}{-}{-}{-}{-}{-}{-}}\ErrorTok{|}\SpecialCharTok{{-}{-}{-}{-}{-}{-}{-}{-}}\ErrorTok{:|:}\SpecialCharTok{{-}{-}{-}{-}{-}{-}{-}}\ErrorTok{|}
\ErrorTok{|}\NormalTok{Скорость роста самцов }\SpecialCharTok{|}     \FloatTok{5.95}\SpecialCharTok{|}\ErrorTok{\textless{}}\FloatTok{2e{-}16}  \SpecialCharTok{|}
\ErrorTok{|}\NormalTok{Скорость роста самок  }\SpecialCharTok{|}     \FloatTok{4.09}\SpecialCharTok{|}\FloatTok{5.2e{-}13} \SpecialCharTok{|}
\ErrorTok{|}\NormalTok{Разница               }\SpecialCharTok{|}     \FloatTok{1.86}\SpecialCharTok{|}\FloatTok{0.00024} \SpecialCharTok{|}
\ErrorTok{\textgreater{}} 
\end{Highlighting}
\end{Shaded}

\textbf{Интерпретация:}\\
Значимая \emph{разница} (p \textless{} 0.05) указывает на статистически
значимые различия в скорости роста между полами.

\subsection{Сравнение
моделей}\label{ux441ux440ux430ux432ux43dux435ux43dux438ux435-ux43cux43eux434ux435ux43bux435ux439}

Одним из ключевых аспектов анализа биологических данных является
определение формы зависимости между переменными. В данном разделе мы
рассмотрим основы подбора модели зависимости между длиной и весом
креветок. Начиная с простой линейной модели, мы постепенно перейдем к
более сложным нелинейным моделям, чтобы продемонстрировать методику
выбора наилучшей модели. Cравним три модели --- линейную, полиномиальную
и степенную --- чтобы определить, какая из них наилучшим образом
описывает данные. Цель анализа --- найти математическую зависимость,
которая:

\begin{enumerate}
\def\labelenumi{\arabic{enumi}.}
\item
  Точно предсказывает вес креветки по её длине.
\item
  Имеет биологическую интерпретацию.
\item
  Минимизирует ошибку предсказания.
\end{enumerate}

\subsubsection{Модели и их
параметры}\label{ux43cux43eux434ux435ux43bux438-ux438-ux438ux445-ux43fux430ux440ux430ux43cux435ux442ux440ux44b}

\begin{enumerate}
\def\labelenumi{\arabic{enumi}.}
\tightlist
\item
  \textbf{Линейная}:
  \(\text{weight} = \beta_0 + \beta_1\cdot\text{length}\)
\item
  \textbf{Полиномиальная 3-й степени}:
  \(\text{weight} = \beta_0 + \beta_1\cdot\text{length} + \beta_2\cdot\text{length}^2 + \beta_3\cdot\text{length}^3\)
\item
  \textbf{Степенная}: \(\text{weight} = a\cdot\text{length}^b\)
\end{enumerate}

\subsubsection{Метрики}\label{ux43cux435ux442ux440ux438ux43aux438}

\begin{itemize}
\tightlist
\item
  \textbf{R²} - (коэффициент детерминации): чем ближе к 1, тем лучше
  модель объясняет данные.
\item
  \textbf{AIC} -(информационный критерий Акаике): чем меньше значение,
  тем лучше модель с учётом её сложности.
\end{itemize}

\subsubsection{\texorpdfstring{\textbf{Результаты}}{Результаты}}\label{ux440ux435ux437ux443ux43bux44cux442ux430ux442ux44b}

\paragraph{\texorpdfstring{\textbf{1. Линейная
модель}}{1. Линейная модель}}\label{ux43bux438ux43dux435ux439ux43dux430ux44f-ux43cux43eux434ux435ux43bux44c}

\begin{Shaded}
\begin{Highlighting}[]
\NormalTok{Coefficients}\SpecialCharTok{:}
\NormalTok{             Estimate Std. Error t value }\FunctionTok{Pr}\NormalTok{(}\SpecialCharTok{\textgreater{}}\ErrorTok{|}\NormalTok{t}\SpecialCharTok{|}\NormalTok{)    }
\NormalTok{(Intercept) }\SpecialCharTok{{-}}\FloatTok{2.115}      \FloatTok{0.085}     \SpecialCharTok{{-}}\FloatTok{24.86}   \SpecialCharTok{\textless{}}\FloatTok{2e{-}16} \SpecialCharTok{**}\ErrorTok{*}
\NormalTok{length       }\FloatTok{0.1665}     \FloatTok{0.0038}    \FloatTok{43.71}    \SpecialCharTok{\textless{}}\FloatTok{2e{-}16} \SpecialCharTok{**}\ErrorTok{*}
\end{Highlighting}
\end{Shaded}

\begin{itemize}
\item
  \textbf{R² = 0.894}
\item
  \textbf{AIC = 148.02}
\end{itemize}

\begin{figure}[H]

{\centering \includegraphics[width=0.6\linewidth,height=\textheight,keepaspectratio]{images/linear_shrimp.PNG}

}

\caption{Рис. 1.5: Линейная модель}

\end{figure}%

\paragraph{\texorpdfstring{\textbf{2. Полиномиальная
модель}}{2. Полиномиальная модель}}\label{ux43fux43eux43bux438ux43dux43eux43cux438ux430ux43bux44cux43dux430ux44f-ux43cux43eux434ux435ux43bux44c}

\begin{Shaded}
\begin{Highlighting}[]
\NormalTok{Coefficients}\SpecialCharTok{:}
\NormalTok{                 Estimate Std. Error t value }\FunctionTok{Pr}\NormalTok{(}\SpecialCharTok{\textgreater{}}\ErrorTok{|}\NormalTok{t}\SpecialCharTok{|}\NormalTok{)    }
\FunctionTok{poly}\NormalTok{(length,}\DecValTok{3}\NormalTok{)}\DecValTok{1}  \FloatTok{14.5038}    \FloatTok{0.2127}    \FloatTok{68.18}   \SpecialCharTok{\textless{}}\FloatTok{2e{-}16} \SpecialCharTok{**}\ErrorTok{*}
\FunctionTok{poly}\NormalTok{(length,}\DecValTok{3}\NormalTok{)}\DecValTok{2}   \FloatTok{3.7209}    \FloatTok{0.2127}    \FloatTok{17.49}   \SpecialCharTok{\textless{}}\FloatTok{2e{-}16} \SpecialCharTok{**}\ErrorTok{*}
\FunctionTok{poly}\NormalTok{(length,}\DecValTok{3}\NormalTok{)}\DecValTok{3}   \FloatTok{0.9526}    \FloatTok{0.2127}     \FloatTok{4.48}  \FloatTok{1.2e{-}05} \SpecialCharTok{**}\ErrorTok{*}
\end{Highlighting}
\end{Shaded}

\begin{itemize}
\item
  \textbf{R² = 0.957}
\item
  \textbf{AIC = -52.80}
\end{itemize}

\begin{figure}[H]

{\centering \includegraphics[width=0.6\linewidth,height=\textheight,keepaspectratio]{images/poly_shrimp.PNG}

}

\caption{Рис. 1.5: Полиномиальная модель}

\end{figure}%

\paragraph{\texorpdfstring{\textbf{3. Степенная
модель}}{3. Степенная модель}}\label{ux441ux442ux435ux43fux435ux43dux43dux430ux44f-ux43cux43eux434ux435ux43bux44c}

\begin{Shaded}
\begin{Highlighting}[]
\NormalTok{Parameters}\SpecialCharTok{:}
\NormalTok{   Estimate Std. Error t value }\FunctionTok{Pr}\NormalTok{(}\SpecialCharTok{\textgreater{}}\ErrorTok{|}\NormalTok{t}\SpecialCharTok{|}\NormalTok{)    }
\NormalTok{a }\FloatTok{0.000157}   \FloatTok{0.000028}    \FloatTok{5.60}  \FloatTok{6.3e{-}08} \SpecialCharTok{**}\ErrorTok{*}
\NormalTok{b }\FloatTok{2.920160}   \FloatTok{0.054102}   \FloatTok{53.98}   \SpecialCharTok{\textless{}}\FloatTok{2e{-}16} \SpecialCharTok{**}\ErrorTok{*}
\end{Highlighting}
\end{Shaded}

\begin{itemize}
\item
  \textbf{R² = 0.955}
\item
  \textbf{AIC = -48.43} \begin{center}
  \includegraphics[width=0.6\linewidth,height=\textheight,keepaspectratio]{images/power_shrimp.PNG}
  \end{center}
\end{itemize}

\subsubsection{\texorpdfstring{\textbf{3. Сравнение
моделей}}{3. Сравнение моделей}}\label{ux441ux440ux430ux432ux43dux435ux43dux438ux435-ux43cux43eux434ux435ux43bux435ux439-1}

\begin{longtable}[]{@{}lll@{}}
\toprule\noalign{}
\textbf{Модель} & \textbf{R²} & \textbf{AIC} \\
\midrule\noalign{}
\endhead
\bottomrule\noalign{}
\endlastfoot
Линейная & 0.894 & 148.02 \\
Полиномиальная & 0.957 & -52.80 \\
Степенная & 0.955 & -48.43 \\
\end{longtable}

\textbf{Выводы:}

\begin{enumerate}
\def\labelenumi{\arabic{enumi}.}
\item
  \textbf{Полиномиальная модель} демонстрирует наилучшие показатели
  (максимальный R² и минимальный AIC).
\item
  \textbf{Степенная модель} близка по качеству, но её параметр
  \emph{b}≈2.92 близок к биологически ожидаемому значению 3 (вес
  пропорционален объёму).
\item
  \textbf{Линейная модель} существенно уступает по точности.
\end{enumerate}

\subsubsection{\texorpdfstring{\textbf{4.
Рекомендации}}{4. Рекомендации}}\label{ux440ux435ux43aux43eux43cux435ux43dux434ux430ux446ux438ux438}

\begin{itemize}
\item
  \textbf{Для прогнозирования} используйте полиномиальную модель, так
  как она минимизирует ошибку.
\item
  \textbf{Для биологической интерпретации} предпочтительна степенная
  модель: weight∝length\textsuperscript{2.92}.
\item
  \textbf{Избегайте переобучения:} Полиномиальные модели высокой степени
  могут терять интерпретируемость.
\end{itemize}

\subsubsection{\texorpdfstring{\textbf{5. Визуализация
остатков}}{5. Визуализация остатков}}\label{ux432ux438ux437ux443ux430ux43bux438ux437ux430ux446ux438ux44f-ux43eux441ux442ux430ux442ux43aux43eux432}

Остатки степенной модели распределены равномерно, что подтверждает её
адекватность: \begin{center}
\includegraphics[width=0.6\linewidth,height=\textheight,keepaspectratio]{images/residuals_shrimp.PNG}
\end{center}

\subsubsection{\texorpdfstring{\textbf{Заключение}}{Заключение}}\label{ux437ux430ux43aux43bux44eux447ux435ux43dux438ux435}

Для анализа зависимости веса от длины северной креветки
\textbf{рекомендуется}:

\begin{enumerate}
\def\labelenumi{\arabic{enumi}.}
\item
  \textbf{Полиномиальная модель} --- для задач, требующих максимальной
  точности.
\item
  \textbf{Степенная модель} --- для интерпретации биологических
  закономерностей.
\end{enumerate}

Скрипт вышеописанных событий:

\begin{Shaded}
\begin{Highlighting}[]
\CommentTok{\# Установка рабочей директории}
\FunctionTok{setwd}\NormalTok{(}\StringTok{"C:/TEXTBOOK/"}\NormalTok{)}

\CommentTok{\# Загрузка библиотек}
\FunctionTok{library}\NormalTok{(tidyverse)}
\FunctionTok{library}\NormalTok{(ggplot2)}

\CommentTok{\# Загрузка данных}
\NormalTok{data }\OtherTok{\textless{}{-}} \FunctionTok{read\_csv}\NormalTok{(}\StringTok{"shrimp\_catch.csv"}\NormalTok{) }\SpecialCharTok{\%\textgreater{}\%}
  \FunctionTok{filter}\NormalTok{(}\SpecialCharTok{!}\NormalTok{id }\SpecialCharTok{\%in\%} \FunctionTok{c}\NormalTok{(}\DecValTok{10}\NormalTok{, }\DecValTok{50}\NormalTok{))  }\CommentTok{\# Удаление аномальных наблюдений}

\CommentTok{\# Проверка структуры}
\FunctionTok{glimpse}\NormalTok{(data)}

\CommentTok{\# Линейная модель: вес \textasciitilde{} длина}
\NormalTok{model\_linear }\OtherTok{\textless{}{-}} \FunctionTok{lm}\NormalTok{(weight }\SpecialCharTok{\textasciitilde{}}\NormalTok{ length, }\AttributeTok{data =}\NormalTok{ data)}
\FunctionTok{summary}\NormalTok{(model\_linear)}

\CommentTok{\# Визуализация}
\FunctionTok{ggplot}\NormalTok{(data, }\FunctionTok{aes}\NormalTok{(}\AttributeTok{x =}\NormalTok{ length, }\AttributeTok{y =}\NormalTok{ weight)) }\SpecialCharTok{+}
  \FunctionTok{geom\_point}\NormalTok{(}\AttributeTok{color =} \StringTok{"steelblue"}\NormalTok{, }\AttributeTok{alpha =} \FloatTok{0.7}\NormalTok{) }\SpecialCharTok{+}
  \FunctionTok{geom\_smooth}\NormalTok{(}\AttributeTok{method =} \StringTok{"lm"}\NormalTok{, }\AttributeTok{color =} \StringTok{"\#FC4E07"}\NormalTok{) }\SpecialCharTok{+}
  \FunctionTok{labs}\NormalTok{(}\AttributeTok{title =} \StringTok{"Линейная модель"}\NormalTok{, }\AttributeTok{x =} \StringTok{"Длина (мм)"}\NormalTok{, }\AttributeTok{y =} \StringTok{"Вес (г)"}\NormalTok{)}


\CommentTok{\# Полиномиальная модель: вес \textasciitilde{} длина + длина? + длина?}
\NormalTok{model\_poly }\OtherTok{\textless{}{-}} \FunctionTok{lm}\NormalTok{(weight }\SpecialCharTok{\textasciitilde{}} \FunctionTok{poly}\NormalTok{(length, }\DecValTok{3}\NormalTok{), }\AttributeTok{data =}\NormalTok{ data)}
\FunctionTok{summary}\NormalTok{(model\_poly)}

\CommentTok{\# Визуализация}
\FunctionTok{ggplot}\NormalTok{(data, }\FunctionTok{aes}\NormalTok{(}\AttributeTok{x =}\NormalTok{ length, }\AttributeTok{y =}\NormalTok{ weight)) }\SpecialCharTok{+}
  \FunctionTok{geom\_point}\NormalTok{(}\AttributeTok{color =} \StringTok{"steelblue"}\NormalTok{, }\AttributeTok{alpha =} \FloatTok{0.7}\NormalTok{) }\SpecialCharTok{+}
  \FunctionTok{geom\_smooth}\NormalTok{(}\AttributeTok{method =} \StringTok{"lm"}\NormalTok{, }\AttributeTok{formula =}\NormalTok{ y }\SpecialCharTok{\textasciitilde{}} \FunctionTok{poly}\NormalTok{(x, }\DecValTok{3}\NormalTok{), }\AttributeTok{color =} \StringTok{"\#E7B800"}\NormalTok{) }\SpecialCharTok{+}
  \FunctionTok{labs}\NormalTok{(}\AttributeTok{title =} \StringTok{"Полиномиальная модель"}\NormalTok{, }\AttributeTok{x =} \StringTok{"Длина (мм)"}\NormalTok{, }\AttributeTok{y =} \StringTok{"Вес (г)"}\NormalTok{)}


\CommentTok{\# Степенная модель: вес \textasciitilde{} длина\^{}k (k подбирается)}
\NormalTok{model\_power }\OtherTok{\textless{}{-}} \FunctionTok{nls}\NormalTok{(weight }\SpecialCharTok{\textasciitilde{}}\NormalTok{ a }\SpecialCharTok{*}\NormalTok{ length}\SpecialCharTok{\^{}}\NormalTok{b, }
                   \AttributeTok{data =}\NormalTok{ data, }
                   \AttributeTok{start =} \FunctionTok{list}\NormalTok{(}\AttributeTok{a =} \FloatTok{0.001}\NormalTok{, }\AttributeTok{b =} \DecValTok{3}\NormalTok{))  }\CommentTok{\# Начальные значения}
\FunctionTok{summary}\NormalTok{(model\_power)}

\CommentTok{\# Визуализация}
\NormalTok{data}\SpecialCharTok{$}\NormalTok{pred\_power }\OtherTok{\textless{}{-}} \FunctionTok{predict}\NormalTok{(model\_power)}
\FunctionTok{ggplot}\NormalTok{(data, }\FunctionTok{aes}\NormalTok{(}\AttributeTok{x =}\NormalTok{ length, }\AttributeTok{y =}\NormalTok{ weight)) }\SpecialCharTok{+}
  \FunctionTok{geom\_point}\NormalTok{(}\AttributeTok{color =} \StringTok{"steelblue"}\NormalTok{, }\AttributeTok{alpha =} \FloatTok{0.7}\NormalTok{) }\SpecialCharTok{+}
  \FunctionTok{geom\_line}\NormalTok{(}\FunctionTok{aes}\NormalTok{(}\AttributeTok{y =}\NormalTok{ pred\_power), }\AttributeTok{color =} \StringTok{"\#00BA38"}\NormalTok{, }\AttributeTok{linewidth =} \FloatTok{1.2}\NormalTok{) }\SpecialCharTok{+}
  \FunctionTok{labs}\NormalTok{(}\AttributeTok{title =} \StringTok{"Степенная модель"}\NormalTok{, }\AttributeTok{x =} \StringTok{"Длина (мм)"}\NormalTok{, }\AttributeTok{y =} \StringTok{"Вес (г)"}\NormalTok{)}

\CommentTok{\# Расчет AIC}
\FunctionTok{AIC}\NormalTok{(model\_linear, model\_poly, model\_power)}

\CommentTok{\# Расчет R?}
\NormalTok{r2\_linear }\OtherTok{\textless{}{-}} \FunctionTok{summary}\NormalTok{(model\_linear)}\SpecialCharTok{$}\NormalTok{r.squared}
\NormalTok{r2\_poly }\OtherTok{\textless{}{-}} \FunctionTok{summary}\NormalTok{(model\_poly)}\SpecialCharTok{$}\NormalTok{r.squared}
\NormalTok{r2\_power }\OtherTok{\textless{}{-}} \DecValTok{1} \SpecialCharTok{{-}} \FunctionTok{sum}\NormalTok{(}\FunctionTok{residuals}\NormalTok{(model\_power)}\SpecialCharTok{\^{}}\DecValTok{2}\NormalTok{) }\SpecialCharTok{/} \FunctionTok{sum}\NormalTok{((data}\SpecialCharTok{$}\NormalTok{weight }\SpecialCharTok{{-}} \FunctionTok{mean}\NormalTok{(data}\SpecialCharTok{$}\NormalTok{weight))}\SpecialCharTok{\^{}}\DecValTok{2}\NormalTok{)}

\CommentTok{\# Создание таблицы сравнения моделей}
\NormalTok{comparison\_table }\OtherTok{\textless{}{-}} \FunctionTok{data.frame}\NormalTok{(}
\NormalTok{  Модель }\OtherTok{=} \FunctionTok{c}\NormalTok{(}\StringTok{"Линейная"}\NormalTok{, }\StringTok{"Полиномиальная"}\NormalTok{, }\StringTok{"Степенная"}\NormalTok{),}
  \AttributeTok{R\_square =} \FunctionTok{c}\NormalTok{(r2\_linear, r2\_poly, r2\_power),}
  \AttributeTok{AIC =} \FunctionTok{c}\NormalTok{(}\FunctionTok{AIC}\NormalTok{(model\_linear), }\FunctionTok{AIC}\NormalTok{(model\_poly), }\FunctionTok{AIC}\NormalTok{(model\_power))}
\NormalTok{)}

\CommentTok{\# Вывод таблицы}
\FunctionTok{print}\NormalTok{(comparison\_table)}

\CommentTok{\# Остатки для степенной модели}
\NormalTok{data}\SpecialCharTok{$}\NormalTok{residuals }\OtherTok{\textless{}{-}} \FunctionTok{residuals}\NormalTok{(model\_power)}

\FunctionTok{ggplot}\NormalTok{(data, }\FunctionTok{aes}\NormalTok{(}\AttributeTok{x =}\NormalTok{ length, }\AttributeTok{y =}\NormalTok{ residuals)) }\SpecialCharTok{+}
  \FunctionTok{geom\_point}\NormalTok{(}\AttributeTok{color =} \StringTok{"\#FC4E07"}\NormalTok{, }\AttributeTok{alpha =} \FloatTok{0.7}\NormalTok{) }\SpecialCharTok{+}
  \FunctionTok{geom\_hline}\NormalTok{(}\AttributeTok{yintercept =} \DecValTok{0}\NormalTok{, }\AttributeTok{linetype =} \StringTok{"dashed"}\NormalTok{) }\SpecialCharTok{+}
  \FunctionTok{labs}\NormalTok{(}\AttributeTok{title =} \StringTok{"Остатки степенной модели"}\NormalTok{, }\AttributeTok{x =} \StringTok{"Длина (мм)"}\NormalTok{, }\AttributeTok{y =} \StringTok{"Ошибка"}\NormalTok{)}
\end{Highlighting}
\end{Shaded}

\bookmarksetup{startatroot}

\chapter{Нейронные сети в экологии: практическое
введение}\label{ux43dux435ux439ux440ux43eux43dux43dux44bux435-ux441ux435ux442ux438-ux432-ux44dux43aux43eux43bux43eux433ux438ux438-ux43fux440ux430ux43aux442ux438ux447ux435ux441ux43aux43eux435-ux432ux432ux435ux434ux435ux43dux438ux435}

\section{Введение}\label{ux432ux432ux435ux434ux435ux43dux438ux435-2}

Это практическое занятие --- про то, как из разрозненных чисел сделать
внятную экологическую историю и как перейти от простых регрессий к
нейронным сетям, оставаясь честными перед данными. Мы используем R не из
эстетики, а из прагматики: он позволяет прозрачно воспроизводить анализ,
контролировать каждую трансформацию и быстро проверять гипотезы. В
основе занятия --- логика и примеры из статьи Андрея Викторовича
Коросова
«\href{https://ecopri.ru/journal/article.php?id=14002}{Нейронные сети в
экологии: введение}» (Принципы экологии, 2023, №3, 76--96). Там хорошо
показан путь от классических линейных моделей к нелинейным конструкциям
и дальше --- к искусственным нейронным сетям, способным решать задачи
классификации и прогнозирования. Мы пойдём тем же маршрутом, но с
учебной расстановкой акцентов: сначала поймём, как работает «молоток»
(регрессия), прежде чем брать в руки «многофункциональный инструмент»
(сеть).

Задача занятия двоякая. Во‑первых, усвоить минимально достаточный набор
статистических практик, чтобы не путать «эффект» с «удачным
совпадением»: проверка предпосылок, визуальная диагностика, простые и
понятные метрики качества, раздельные обучающие и тестовые выборки.
Во‑вторых, увидеть, как усложнение модели должно быть мотивировано
данными и биологией, а не нашей любовью к сложным методам. Если более
простая модель объясняет всё, что вам нужно для решения прикладной
задачи, смело берите её --- мозг склонен влюбляться в красивое, но нам
нужна работающая гипотеза.

Структура занятия отражает эволюцию инструментов. Начнём с линейной
регрессии на предельно понятном примере: связь массы и длины. Здесь
важны не только коэффициенты и p‑значения, но и остатки, проверка
линейности, гомоскедастичность, доверительные интервалы. Затем
познакомимся с численной оптимизацией: когда аналитического решения нет,
мы используем итерационные алгоритмы (nls) и учимся задавать стартовые
значения, контролировать сходимость и чувствительность. Далее ---
множественная регрессия и вопрос интерпретации: что реально добавляет
предиктор, а что «ездит зайцем» на коллинеарности. Оттуда естественно
перейти к нелинейным зависимостям: аллометрия, линеаризация через
логарифмы, сопоставление качества моделей не только по R², но и по AIC,
и --- что особенно важно --- по поведению остатков. Логистическая
регрессия вводит нас в мир пороговых процессов и бинарных исходов:
S‑кривая, L50, ROC/AUC, калибровка вероятностей --- всё это работает
одинаково хорошо для токсичности дафний и для созревания по длине.

Когда базовые кирпичики стоят, делаем шаг к нейронным сетям. Сначала
показываем, что сеть без скрытых слоёв фактически воспроизводит линейную
модель. Затем добавляем один скрытый нейрон и видим, как появляется
возможность описывать нелинейности и пороговые эффекты. Дальше ---
классификация по нескольким признакам и небольшие архитектуры: оцениваем
точность, избегаем утечки информации, фиксируем случайные зерна,
обязательно сравниваем с простыми бэйзлайнами, чтобы не путать «мощнее»
с «лучше». В финале --- пример пространственного моделирования
численности по биотопам: разделение на train/test, прогноз на новых
условиях, разговор о переносимости моделей и ограничениях, без которых
любые «красивые карты» остаются просто эстетикой.

Организация работы предельно проста. Даны три версии скрипта:
\href{https://mombus.github.io/cRab/data/KOROSOV.R}{KOROSOV.R} ---
максимально близко к оригиналу;
\href{https://mombus.github.io/cRab/data/KOROSOV_updated.R}{KOROSOV\_updated.R}
--- тот же код с подробными комментариями и пояснениями (основной
учебный вариант);
\href{https://mombus.github.io/cRab/data/KOROSOV_visual.R}{KOROSOV\_visual.R}
--- дополненный продвинутой визуализацией и небольшой аналитикой
качества. Для запуска понадобятся данные
\href{https://mombus.github.io/cRab/data/vipkar.csv}{vipkar.csv} и
\href{https://mombus.github.io/cRab/data/kihzsdat.csv}{kihzsdat.csv},
корректная рабочая директория в setwd() и набор пакетов (как минимум
neuralnet и ggplot2). Мы сознательно держим зависимости минимальными,
чтобы главный фокус был на методе и интерпретации, а не на обвязке.

Чему вы научитесь и на что обращать внимание. Во‑первых, всегда
проверять, что модель решает именно ваш вопрос: чёткая формулировка
задачи до выбора алгоритма экономит половину времени. Во‑вторых, всегда
показывать эффект и неопределённость: коэффициенты с интервалами,
калибровка вероятностей, ошибки прогноза на независимых данных.
В‑третьих, всегда сравнивать с простым бэйзлайном: если «сеть» не лучше
честной регрессии на чистых признаках, значит, проблема не в
архитектуре, а в данных или постановке. И да, старайтесь говорить языком
биологии: «параметр \emph{b} близок к 3» --- это про объём, «L50
сдвинулся» --- про созревание, «AUC высок, но калибровка плывёт» --- про
надёжность решений на уровне индивидуальных вероятностей.

Наконец, про дисциплину и воспроизводимость. Фиксируйте seed,
документируйте версии пакетов и исходные предположения, храните все
промежуточные шаги в скриптах. Это скучно минуту, но экономит дни. И
даже когда вы дойдёте до «сетей», помните: сложная модель --- это не
билет в истину, а всего лишь более гибкий аппроксиматор. Хорошая
практика --- держать рядом простой, интерпретируемый аналог и объяснять
расхождения между ними. Тогда ваши результаты будут не просто
«работать», а выдерживать обсуждение с биологами, инженерами и
управленцами --- то есть приносить пользу за пределами экрана.

\textbf{Для работы скрипта:}

\begin{enumerate}
\def\labelenumi{\arabic{enumi}.}
\item
  Скачайте файлы данных
  (\href{https://mombus.github.io/cRab/data/vipkar.csv}{vipkar.csv} и
  \href{https://mombus.github.io/cRab/data/kihzsdat.csv}{kihzsdat.csv})
\item
  Установите рабочую директорию в setwd()
\item
  Установите необходимые пакеты :
  \textbf{\texttt{install.packages(c("neuralnet",\ "ggplot2"))}}
\end{enumerate}

\begin{Shaded}
\begin{Highlighting}[]
\CommentTok{\# ЗАГРУЗКА БИБЛИОТЕК И НАСТРОЙКА СРЕДЫ ================================}
\FunctionTok{library}\NormalTok{(neuralnet)   }\CommentTok{\# Для построения нейронных сетей}
\FunctionTok{library}\NormalTok{(ggplot2)     }\CommentTok{\# Для продвинутой визуализации (в данном скрипте не используется напрямую)}

\CommentTok{\# Установите свою рабочую директорию (где лежат файлы данных)}
\CommentTok{\# setwd("C:/ВАША\_ДИРЕКТОРИЯ/")}
\end{Highlighting}
\end{Shaded}

\section{ЛИНЕЙНАЯ
РЕГРЕССИЯ}\label{ux43bux438ux43dux435ux439ux43dux430ux44f-ux440ux435ux433ux440ux435ux441ux441ux438ux44f}

В этом разделе мы изучим основы экологического моделирования на примере
зависимости массы тела гадюки от ее длины. Вы построите простую линейную
регрессионную модель, визуализируете данные и линию регрессии, а также
интерпретируете результаты с помощью функции \texttt{summary()}.

Загружаем данные

\begin{Shaded}
\begin{Highlighting}[]
\CommentTok{\# Данные: масса (w) и длина тела (lt) гадюк (в см и граммах)}
\NormalTok{w }\OtherTok{\textless{}{-}} \FunctionTok{c}\NormalTok{(}\DecValTok{85}\NormalTok{, }\DecValTok{90}\NormalTok{, }\DecValTok{85}\NormalTok{, }\DecValTok{95}\NormalTok{, }\DecValTok{95}\NormalTok{, }\DecValTok{135}\NormalTok{, }\DecValTok{165}\NormalTok{, }\DecValTok{135}\NormalTok{, }\DecValTok{140}\NormalTok{)}
\NormalTok{lt }\OtherTok{\textless{}{-}} \FunctionTok{c}\NormalTok{(}\DecValTok{51}\NormalTok{, }\DecValTok{51}\NormalTok{, }\DecValTok{52}\NormalTok{, }\DecValTok{54}\NormalTok{, }\DecValTok{54}\NormalTok{, }\DecValTok{59}\NormalTok{, }\DecValTok{59}\NormalTok{, }\DecValTok{60}\NormalTok{, }\DecValTok{62}\NormalTok{)}
\end{Highlighting}
\end{Shaded}

Строим и запускаем модель \[
w_t = a_0 + a_1 \cdot l_t
\]

где: - \(w_t\) --- зависимая переменная, - \(a_0\) --- свободный член, -
\(a_1\) --- коэффициент регрессии, - \(l_t\) --- независимая переменная.

\begin{Shaded}
\begin{Highlighting}[]
\CommentTok{\# Построение линейной модели: w = a0 + a1*lt}
\NormalTok{lreg }\OtherTok{\textless{}{-}} \FunctionTok{lm}\NormalTok{(w }\SpecialCharTok{\textasciitilde{}}\NormalTok{ lt)}
\end{Highlighting}
\end{Shaded}

Выведем результаты модели

\begin{Shaded}
\begin{Highlighting}[]
\CommentTok{\# Просмотр результатов модели:}
\FunctionTok{summary}\NormalTok{(lreg)  }\CommentTok{\# Обратите внимание на коэффициенты и p{-}значения}
\end{Highlighting}
\end{Shaded}

На экране появится:

\begin{Shaded}
\begin{Highlighting}[]
\NormalTok{Call}\SpecialCharTok{:}
\FunctionTok{lm}\NormalTok{(}\AttributeTok{formula =}\NormalTok{ w }\SpecialCharTok{\textasciitilde{}}\NormalTok{ lt)}

\NormalTok{Residuals}\SpecialCharTok{:}
\NormalTok{    Min      }\DecValTok{1}\NormalTok{Q  Median      }\DecValTok{3}\NormalTok{Q     Max }
\SpecialCharTok{{-}}\FloatTok{13.452}  \SpecialCharTok{{-}}\FloatTok{7.585}  \SpecialCharTok{{-}}\FloatTok{4.868}   \FloatTok{1.490}  \FloatTok{30.623} 

\NormalTok{Coefficients}\SpecialCharTok{:}
\NormalTok{            Estimate Std. Error t value }\FunctionTok{Pr}\NormalTok{(}\SpecialCharTok{\textgreater{}}\ErrorTok{|}\NormalTok{t}\SpecialCharTok{|}\NormalTok{)    }
\NormalTok{(Intercept) }\SpecialCharTok{{-}}\FloatTok{240.766}     \FloatTok{64.457}  \SpecialCharTok{{-}}\FloatTok{3.735} \FloatTok{0.007308} \SpecialCharTok{**} 
\NormalTok{lt             }\FloatTok{6.358}      \FloatTok{1.153}   \FloatTok{5.516} \FloatTok{0.000891} \SpecialCharTok{**}\ErrorTok{*}
\SpecialCharTok{{-}{-}{-}}
\NormalTok{Signif. codes}\SpecialCharTok{:}  \DecValTok{0}\NormalTok{ ‘}\SpecialCharTok{**}\ErrorTok{*}\NormalTok{’ }\FloatTok{0.001}\NormalTok{ ‘}\SpecialCharTok{**}\NormalTok{’ }\FloatTok{0.01}\NormalTok{ ‘}\SpecialCharTok{*}\NormalTok{’ }\FloatTok{0.05}\NormalTok{ ‘.’ }\FloatTok{0.1}\NormalTok{ ‘ ’ }\DecValTok{1}

\NormalTok{Residual standard error}\SpecialCharTok{:} \FloatTok{13.81}\NormalTok{ on }\DecValTok{7}\NormalTok{ degrees of freedom}
\NormalTok{Multiple R}\SpecialCharTok{{-}}\NormalTok{squared}\SpecialCharTok{:}  \FloatTok{0.813}\NormalTok{,     Adjusted R}\SpecialCharTok{{-}}\NormalTok{squared}\SpecialCharTok{:}  \FloatTok{0.7863} 
\NormalTok{F}\SpecialCharTok{{-}}\NormalTok{statistic}\SpecialCharTok{:} \FloatTok{30.43}\NormalTok{ on }\DecValTok{1}\NormalTok{ and }\DecValTok{7}\NormalTok{ DF,  p}\SpecialCharTok{{-}}\NormalTok{value}\SpecialCharTok{:} \FloatTok{0.0008911}
\end{Highlighting}
\end{Shaded}

Мы получили результаты линейной регрессии, где зависимая переменная ---
масса тела гадюки (w), а независимая переменная --- длина тела (lt).
Разберем каждый параметр:

1. **Call (Вызов модели):**

`lm(formula = w \textasciitilde{} lt)`

Это просто напоминание, какая модель была построена. Здесь указано, что
мы моделировали зависимость массы (w) от длины тела (lt) с помощью
линейной регрессии.

2. **Residuals (Остатки):**

Остатки --- это разница между наблюдаемыми значениями массы и
предсказанными моделью значениями. Они показывают, насколько хорошо
модель описывает данные.

\begin{itemize}
\item
  `Min`: минимальный остаток = -13.452 (наибольшее недооцененное
  значение)
\item
  `1Q`: первый квартиль = -7.585 (25\% остатков меньше этого значения)
\item
  `Median`: медиана остатков = -4.868 (середина распределения остатков)
\item
  `3Q`: третий квартиль = 1.490 (75\% остатков меньше этого значения)
\item
  `Max`: максимальный остаток = 30.623 (наибольшее переоцененное
  значение)
\end{itemize}

Распределение остатков: медиана немного смещена влево (отрицательное
значение), а размах между 1Q и 3Q составляет примерно 9 единиц. Это
может указывать на легкую асимметрию, но выборка мала.

3. **Coefficients (Коэффициенты):**

\begin{itemize}
\item
  `(Intercept)`: свободный член (a0) = -240.766. Это предсказанное
  значение массы при длине тела, равной нулю. Биологически это не имеет
  смысла (длина не может быть нулевой), но это математическая
  особенность модели.
\item
  `lt`: коэффициент регрессии (a1) = 6.358. Это означает, что при
  увеличении длины тела на 1 см масса тела увеличивается в среднем на
  6.358 г.
\end{itemize}

Для каждого коэффициента приведены:

\begin{itemize}
\item
  `Estimate`: точечная оценка коэффициента.
\item
  `Std. Error`: стандартная ошибка оценки коэффициента. Для intercept =
  64.457, для lt = 1.153. Это мера изменчивости оценки.
\item
  `t value`: t-статистика. Рассчитывается как Estimate / Std.Error. Для
  intercept: -240.766 / 64.457 ≈ -3.735; для lt: 6.358 / 1.153 ≈ 5.516.
\item
  `Pr(\textgreater\textbar t\textbar)`: p-значение для проверки гипотезы
  о равенстве коэффициента нулю.
\item
  Для intercept: p=0.007308 (значим на уровне α=0.01, т.е. intercept
  статистически значимо отличается от нуля).
\item
  Для lt: p=0.000891 (значим на уровне α=0.001). Это означает, что длина
  тела значимо влияет на массу.
\end{itemize}

Значимость кодов: три звездочки (`***`) означают, что коэффициент значим
на уровне 0.001.

4. **Residual standard error (Стандартная ошибка остатков):** 13.81 на 7
степенях свободы. Это мера разброса остатков. В среднем, предсказания
модели отклоняются от реальных значений на ±13.81 г. Степени свободы
(df) = n - 2 = 9 - 2 = 7 (n --- количество наблюдений).

5. **Multiple R-squared (Коэффициент детерминации R²):** 0.813. Это
означает, что 81.3\% вариации массы тела объясняется длиной тела.
Остальные 18.7\% --- это неучтенные факторы и случайная изменчивость.

6. **Adjusted R-squared (Скорректированный R²):** 0.7863. Этот
показатель корректирует R² с учетом числа предикторов. Он полезен при
сравнении моделей с разным числом предикторов. Здесь он немного меньше
R², так как учитывает, что в модели один предиктор.

7. **F-statistic (F-статистика):** 30.43 на 1 и 7 степенях свободы.
Проверяет гипотезу о том, что все коэффициенты (кроме intercept) равны
нулю (т.е. модель не лучше, чем модель только с константой).

\begin{itemize}
\tightlist
\item
  p-value: 0.0008911 (крайне значимый), что означает, что модель в целом
  адекватна.
\end{itemize}

**Выводы:**

- Уравнение модели: `w = -240.77 + 6.36 * lt`

- Длина тела значимо влияет на массу (p\textless0.001).

- Модель объясняет 81.3\% вариации массы.

- На каждый сантиметр длины тела масса увеличивается примерно на 6.36 г.

- Остатки модели показывают, что есть несколько точек, которые модель
предсказывает с заметной ошибкой (особенно максимальный остаток в 30.6
г). Возможно, для более точного прогноза нужна нелинейная модель или
учет дополнительных факторов.

**Рекомендации:**

- Проверить допущения линейной регрессии (нормальность остатков,
гомоскедастичность) с помощью диагностических графиков.

- Рассмотреть возможность включения других переменных (например,
возраста, пола) в модель.

- Убедиться, что в данных нет выбросов, которые могут влиять на
коэффициенты.

\begin{Shaded}
\begin{Highlighting}[]
\CommentTok{\# Визуализация зависимости}
\FunctionTok{plot}\NormalTok{(lt, w, }
     \AttributeTok{main =} \StringTok{"Зависимость массы от длины тела гадюки"}\NormalTok{, }
     \AttributeTok{xlab =} \StringTok{"Длина тела (см)"}\NormalTok{, }
     \AttributeTok{ylab =} \StringTok{"Масса (г)"}\NormalTok{, }
     \AttributeTok{pch =} \DecValTok{19}\NormalTok{,        }\CommentTok{\# Кружки вместо стандартных точек}
     \AttributeTok{col =} \StringTok{"darkgreen"}\NormalTok{)}
\FunctionTok{abline}\NormalTok{(lreg, }\AttributeTok{col =} \StringTok{"red"}\NormalTok{, }\AttributeTok{lwd =} \DecValTok{2}\NormalTok{)  }\CommentTok{\# Добавляем линию регрессии}
\end{Highlighting}
\end{Shaded}

\begin{figure}[H]

{\centering \includegraphics[width=0.6\linewidth,height=\textheight,keepaspectratio]{images/KOROSOV1.PNG}

}

\caption{Рис. 1.: Пример линейной регрессии}

\end{figure}%

\section{ЧИСЛЕННАЯ
ОПТИМИЗАЦИЯ}\label{ux447ux438ux441ux43bux435ux43dux43dux430ux44f-ux43eux43fux442ux438ux43cux438ux437ux430ux446ux438ux44f}

Здесь вы познакомитесь с численными методами оптимизации параметров
моделей, которые применяются, когда аналитическое решение невозможно. На
примере той же зависимости массы от длины вы подгоните параметры модели
с помощью функции \texttt{nls()} и сравните результаты с аналитическим
решением.

Аналитические методы дают точное решение в виде математической формулы,
используя алгебраические преобразования и теоремы математического
анализа. Они идеальны для простых моделей, где существуют явные решения,
обеспечивая прозрачную интерпретацию параметров. В экологии такие методы
применимы для базовых зависимостей типа линейной регрессии. Численные
методы используются, когда аналитическое решение невозможно, и работают
через последовательные приближения, начиная со стартовых значений и
итеративно улучшая параметры модели. Они незаменимы для сложных
экологических моделей с нелинейными зависимостями, взаимодействиями
факторов и ``шумными'' полевыми данными, позволяя решать задачи,
недоступные для аналитических подходов.

\begin{Shaded}
\begin{Highlighting}[]
\CommentTok{\# Подгонка параметров через оптимизацию}
\NormalTok{nls\_model }\OtherTok{\textless{}{-}} \FunctionTok{nls}\NormalTok{(w }\SpecialCharTok{\textasciitilde{}}\NormalTok{ a0 }\SpecialCharTok{+}\NormalTok{ a1 }\SpecialCharTok{*}\NormalTok{ lt, }\AttributeTok{start =} \FunctionTok{list}\NormalTok{(}\AttributeTok{a0 =} \DecValTok{1}\NormalTok{, }\AttributeTok{a1 =} \DecValTok{1}\NormalTok{))}
\FunctionTok{summary}\NormalTok{(nls\_model)}
\end{Highlighting}
\end{Shaded}

На экране появится:

\begin{Shaded}
\begin{Highlighting}[]
\NormalTok{Formula}\SpecialCharTok{:}\NormalTok{ w }\SpecialCharTok{\textasciitilde{}}\NormalTok{ a0 }\SpecialCharTok{+}\NormalTok{ a1 }\SpecialCharTok{*}\NormalTok{ lt}

\NormalTok{Parameters}\SpecialCharTok{:}
\NormalTok{   Estimate Std. Error t value }\FunctionTok{Pr}\NormalTok{(}\SpecialCharTok{\textgreater{}}\ErrorTok{|}\NormalTok{t}\SpecialCharTok{|}\NormalTok{)    }
\NormalTok{a0 }\SpecialCharTok{{-}}\FloatTok{240.766}     \FloatTok{64.457}  \SpecialCharTok{{-}}\FloatTok{3.735} \FloatTok{0.007308} \SpecialCharTok{**} 
\NormalTok{a1    }\FloatTok{6.358}      \FloatTok{1.153}   \FloatTok{5.516} \FloatTok{0.000891} \SpecialCharTok{**}\ErrorTok{*}
\SpecialCharTok{{-}{-}{-}}
\NormalTok{Signif. codes}\SpecialCharTok{:}  \DecValTok{0}\NormalTok{ ‘}\SpecialCharTok{**}\ErrorTok{*}\NormalTok{’ }\FloatTok{0.001}\NormalTok{ ‘}\SpecialCharTok{**}\NormalTok{’ }\FloatTok{0.01}\NormalTok{ ‘}\SpecialCharTok{*}\NormalTok{’ }\FloatTok{0.05}\NormalTok{ ‘.’ }\FloatTok{0.1}\NormalTok{ ‘ ’ }\DecValTok{1}

\NormalTok{Residual standard error}\SpecialCharTok{:} \FloatTok{13.81}\NormalTok{ on }\DecValTok{7}\NormalTok{ degrees of freedom}

\NormalTok{Number of iterations to convergence}\SpecialCharTok{:} \DecValTok{1} 
\NormalTok{Achieved convergence tolerance}\SpecialCharTok{:} \FloatTok{3.247e{-}08}
\end{Highlighting}
\end{Shaded}

\subsection{\texorpdfstring{\textbf{Интерпретация результатов
модели}}{Интерпретация результатов модели}}\label{ux438ux43dux442ux435ux440ux43fux440ux435ux442ux430ux446ux438ux44f-ux440ux435ux437ux443ux43bux44cux442ux430ux442ux43eux432-ux43cux43eux434ux435ux43bux438}

Мы построили линейную модель зависимости массы гадюки (w) от длины её
тела (lt) по формуле:\\
\textbf{\texttt{w\ =\ a0\ +\ a1\ *\ lt}}

\textbf{Ключевые параметры модели:}

\begin{itemize}
\item
  \textbf{a0 (свободный член)}: -240.8 г\\
  Это теоретическая масса при нулевой длине тела. Отрицательное значение
  указывает, что модель не подходит для очень молодых особей.
\item
  \textbf{a1 (коэффициент при lt)}: 6.36 г/см\\
  Каждый дополнительный сантиметр длины тела увеличивает массу в среднем
  на 6.36 г.
\end{itemize}

\textbf{Точность и значимость:}

\begin{itemize}
\item
  Оба коэффициента \textbf{высоко значимы} (p \textless{} 0.01), что
  подтверждает реальность зависимости.
\item
  Стандартная ошибка для a1 составляет 1.15 г/см - это значит, что
  реальное значение, вероятно, находится между 5.2 и 7.5 г/см.
\item
  Модель хорошо сошлась за 1 шаг (итерацию), что говорит об удачном
  подборе параметров.
\end{itemize}

\textbf{Ошибка прогноза:}\\
Среднее отклонение предсказаний от реальных значений - 13.8 г
(стандартная ошибка остатков). Для особи массой 100 г это означает
возможную ошибку прогноза около 14\%.

\begin{quote}
\textbf{Биологический смысл:} Модель подтверждает сильную аллометрию -
крупные гадюки имеют относительно большую массу тела. Каждый сантиметр
длины добавляет около 6.4 г массы. Для особи длиной 55 см прогнозируемая
масса составит: -240.8 + 6.36*55 ≈ 109 г.
\end{quote}

\#\#МНОЖЕСТВЕННАЯ РЕГРЕССИЯ

В этом разделе мы расширим модель, включив несколько факторов. Вы
построите множественную регрессию, учитывающую одновременно длину тела и
длину хвоста гадюки, и научитесь интерпретировать влияние нескольких
предикторов на зависимую переменную.

\begin{Shaded}
\begin{Highlighting}[]
\CommentTok{\# Добавляем новый признак {-} длину хвоста (lc)}
\NormalTok{w }\OtherTok{\textless{}{-}} \FunctionTok{c}\NormalTok{(}\DecValTok{40}\NormalTok{, }\DecValTok{156}\NormalTok{, }\DecValTok{105}\NormalTok{, }\DecValTok{85}\NormalTok{, }\DecValTok{80}\NormalTok{, }\DecValTok{50}\NormalTok{, }\DecValTok{75}\NormalTok{, }\DecValTok{48}\NormalTok{, }\DecValTok{75}\NormalTok{, }\DecValTok{67}\NormalTok{)}
\NormalTok{lt }\OtherTok{\textless{}{-}} \FunctionTok{c}\NormalTok{(}\DecValTok{44}\NormalTok{, }\DecValTok{59}\NormalTok{, }\DecValTok{49}\NormalTok{, }\DecValTok{50}\NormalTok{, }\DecValTok{54}\NormalTok{, }\DecValTok{43}\NormalTok{, }\DecValTok{49}\NormalTok{, }\DecValTok{42}\NormalTok{, }\DecValTok{47}\NormalTok{, }\DecValTok{47}\NormalTok{)}
\NormalTok{lc }\OtherTok{\textless{}{-}} \FunctionTok{c}\NormalTok{(}\DecValTok{70}\NormalTok{, }\DecValTok{78}\NormalTok{, }\DecValTok{66}\NormalTok{, }\DecValTok{90}\NormalTok{, }\DecValTok{83}\NormalTok{, }\DecValTok{70}\NormalTok{, }\DecValTok{62}\NormalTok{, }\DecValTok{75}\NormalTok{, }\DecValTok{40}\NormalTok{, }\DecValTok{80}\NormalTok{)}
\end{Highlighting}
\end{Shaded}

Используя glm-функцию, построим модель с двумя предикторами: \[
w = a_0 + a_1 \cdot l_t + a_2 \cdot l_c
\]

где: - \(w\) --- масса гадюки, - \(l_t\) --- длина тела гадюки, -
\(l_c\) --- длина хвоста гадюки, - \(a_0\) --- свободный член
(константа), - \(a_1\) --- коэффициент регрессии при длине тела, -
\(a_2\) --- коэффициент регрессии при длине хвоста.

\begin{Shaded}
\begin{Highlighting}[]
\CommentTok{\# Множественная регрессия: w = a0 + a1*lt + a2*lc}
\NormalTok{multi\_reg }\OtherTok{\textless{}{-}} \FunctionTok{glm}\NormalTok{(w }\SpecialCharTok{\textasciitilde{}}\NormalTok{ lt }\SpecialCharTok{+}\NormalTok{ lc)}
\FunctionTok{summary}\NormalTok{(multi\_reg)}
\end{Highlighting}
\end{Shaded}

На экране появится:

\begin{Shaded}
\begin{Highlighting}[]
\NormalTok{Call}\SpecialCharTok{:}
\FunctionTok{glm}\NormalTok{(}\AttributeTok{formula =}\NormalTok{ w }\SpecialCharTok{\textasciitilde{}}\NormalTok{ lt }\SpecialCharTok{+}\NormalTok{ lc)}

\NormalTok{Coefficients}\SpecialCharTok{:}
\NormalTok{             Estimate Std. Error t value }\FunctionTok{Pr}\NormalTok{(}\SpecialCharTok{\textgreater{}}\ErrorTok{|}\NormalTok{t}\SpecialCharTok{|}\NormalTok{)    }
\NormalTok{(Intercept) }\SpecialCharTok{{-}}\FloatTok{191.2982}    \FloatTok{53.6908}  \SpecialCharTok{{-}}\FloatTok{3.563} \FloatTok{0.009183} \SpecialCharTok{**} 
\NormalTok{lt             }\FloatTok{6.0308}     \FloatTok{1.1051}   \FloatTok{5.457} \FloatTok{0.000949} \SpecialCharTok{**}\ErrorTok{*}
\NormalTok{lc            }\SpecialCharTok{{-}}\FloatTok{0.3150}     \FloatTok{0.4133}  \SpecialCharTok{{-}}\FloatTok{0.762} \FloatTok{0.470913}    
\SpecialCharTok{{-}{-}{-}}
\NormalTok{Signif. codes}\SpecialCharTok{:}  \DecValTok{0}\NormalTok{ ‘}\SpecialCharTok{**}\ErrorTok{*}\NormalTok{’ }\FloatTok{0.001}\NormalTok{ ‘}\SpecialCharTok{**}\NormalTok{’ }\FloatTok{0.01}\NormalTok{ ‘}\SpecialCharTok{*}\NormalTok{’ }\FloatTok{0.05}\NormalTok{ ‘.’ }\FloatTok{0.1}\NormalTok{ ‘ ’ }\DecValTok{1}

\NormalTok{(Dispersion parameter }\ControlFlowTok{for}\NormalTok{ gaussian family taken to be }\FloatTok{270.9752}\NormalTok{)}

\NormalTok{    Null deviance}\SpecialCharTok{:} \FloatTok{10132.9}\NormalTok{  on }\DecValTok{9}\NormalTok{  degrees of freedom}
\NormalTok{Residual deviance}\SpecialCharTok{:}  \FloatTok{1896.8}\NormalTok{  on }\DecValTok{7}\NormalTok{  degrees of freedom}
\NormalTok{AIC}\SpecialCharTok{:} \FloatTok{88.832}

\NormalTok{Number of Fisher Scoring iterations}\SpecialCharTok{:} \DecValTok{2}
\end{Highlighting}
\end{Shaded}

\subsection{\texorpdfstring{\textbf{Интерпретация результатов
множественной
регрессии}}{Интерпретация результатов множественной регрессии}}\label{ux438ux43dux442ux435ux440ux43fux440ux435ux442ux430ux446ux438ux44f-ux440ux435ux437ux443ux43bux44cux442ux430ux442ux43eux432-ux43cux43dux43eux436ux435ux441ux442ux432ux435ux43dux43dux43eux439-ux440ux435ux433ux440ux435ux441ux441ux438ux438}

Мы исследовали зависимость массы гадюки (w) от длины тела (lt) и длины
хвоста (lc) с помощью модели:\\
\textbf{\texttt{w\ =\ b0\ +\ b1*lt\ +\ b2*lc}}

\textbf{Ключевые выводы модели:}

\begin{enumerate}
\def\labelenumi{\arabic{enumi}.}
\item
  \textbf{Длина тела (lt) сильно влияет на массу}:

  \begin{itemize}
  \item
    Коэффициент: +6.03 г/см
  \item
    Каждый сантиметр длины тела увеличивает массу на \textasciitilde6 г
  \item
    Высокая значимость (p = 0.00095)
  \end{itemize}
\item
  \textbf{Длина хвоста (lc) не влияет значимо на массу}:

  \begin{itemize}
  \item
    Коэффициент: -0.315 г/см (незначимый)
  \item
    p-значение 0.47 \textgreater{} 0.05 - статистически недостоверно
  \item
    После учета длины тела, длина хвоста не добавляет информации
  \end{itemize}
\item
  \textbf{Свободный член (b0)}: -191.3 г\\
  Отрицательное значение подтверждает нелинейность роста у молодых
  особей
\end{enumerate}

\textbf{Качество модели:}

\begin{itemize}
\item
  Модель объясняет значительную часть вариации:\\
  Общая вариация (Null deviance) = 10132.9\\
  Остаточная вариация (Residual deviance) = 1896.8 → \textbf{Объяснено
  81\% вариации}
\item
  AIC = 88.8 (ниже, чем у модели без lc - 92.1, что указывает на лучшее
  качество)
\item
  Модель быстро сошлась за 2 итерации
\end{itemize}

\textbf{Биологическая интерпретация:}

\begin{enumerate}
\def\labelenumi{\arabic{enumi}.}
\item
  Масса тела определяется в основном длиной туловища, а не хвоста
\item
  Для прогноза массы достаточно учитывать только длину тела
\item
  Пример прогноза для особи (lt=50 см, lc=70 см):\\
  \textbf{\texttt{-191.3\ +\ 6.03*50\ -\ 0.315*70\ ≈\ 111\ г}}
\end{enumerate}

\begin{quote}
\textbf{Рекомендация}: При изучении массы гадюк можно исключить длину
хвоста из модели, так как она не вносит значимого вклада в предсказание.
Основным морфометрическим показателем остается длина тела.
\end{quote}

\section{НЕЛИНЕЙНЫЕ
ЗАВИСИМОСТИ}\label{ux43dux435ux43bux438ux43dux435ux439ux43dux44bux435-ux437ux430ux432ux438ux441ux438ux43cux43eux441ux442ux438}

Экологические данные часто имеют нелинейный характер. Здесь вы
смоделируете степенную зависимость (аллометрию) между массой и длиной
тела, используя линеаризацию через логарифмирование, а затем
визуализируете кривую модели.

\begin{Shaded}
\begin{Highlighting}[]
\CommentTok{\# Часто в экологии связи имеют степенной характер: w = a0 * lt\^{}a1}
\CommentTok{\# Линеаризация через логарифмирование}
\NormalTok{log\_model }\OtherTok{\textless{}{-}} \FunctionTok{lm}\NormalTok{(}\FunctionTok{log}\NormalTok{(w) }\SpecialCharTok{\textasciitilde{}} \FunctionTok{log}\NormalTok{(lt))}

\CommentTok{\# Преобразование коэффициентов обратно}
\NormalTok{a0 }\OtherTok{\textless{}{-}} \FunctionTok{exp}\NormalTok{(}\FunctionTok{coef}\NormalTok{(log\_model)[}\DecValTok{1}\NormalTok{])  }\CommentTok{\# Переход от логарифмов}
\NormalTok{a1 }\OtherTok{\textless{}{-}} \FunctionTok{coef}\NormalTok{(log\_model)[}\DecValTok{2}\NormalTok{]       }\CommentTok{\# Показатель степени}

\CommentTok{\# Визуализация степенной зависимости}
\FunctionTok{plot}\NormalTok{(lt, w, }
     \AttributeTok{main =} \StringTok{"Степенная зависимость массы от длины"}\NormalTok{, }
     \AttributeTok{xlab =} \StringTok{"Длина тела (см)"}\NormalTok{, }
     \AttributeTok{ylab =} \StringTok{"Масса (г)"}\NormalTok{,}
     \AttributeTok{pch =} \DecValTok{17}\NormalTok{,}
     \AttributeTok{col =} \StringTok{"blue"}\NormalTok{)}
\FunctionTok{curve}\NormalTok{(a0 }\SpecialCharTok{*}\NormalTok{ x}\SpecialCharTok{\^{}}\NormalTok{a1, }\AttributeTok{add =} \ConstantTok{TRUE}\NormalTok{, }\AttributeTok{col =} \StringTok{"red"}\NormalTok{, }\AttributeTok{lwd =} \DecValTok{2}\NormalTok{)  }\CommentTok{\# Кривая модели}
\end{Highlighting}
\end{Shaded}

\begin{figure}[H]

{\centering \includegraphics[width=0.6\linewidth,height=\textheight,keepaspectratio]{images/KOROSOV2.PNG}

}

\caption{Рис. 2.: Расчет степенной функции}

\end{figure}%

\section{ЛОГИСТИЧЕСКАЯ
РЕГРЕССИЯ}\label{ux43bux43eux433ux438ux441ux442ux438ux447ux435ux441ux43aux430ux44f-ux440ux435ux433ux440ux435ux441ux441ux438ux44f}

Вы изучите моделирование пороговых эффектов в экологии на примере
смертности дафний в зависимости от концентрации токсиканта. Построив
логистическую регрессию, вы получите S-образную кривую, характерную для
таких процессов.

\begin{Shaded}
\begin{Highlighting}[]
\CommentTok{\# Пример: смертность дафний при разных концентрациях токсиканта}
\CommentTok{\# Данные:}
\NormalTok{K }\OtherTok{\textless{}{-}} \FunctionTok{c}\NormalTok{(}\DecValTok{100}\NormalTok{, }\DecValTok{126}\NormalTok{, }\DecValTok{158}\NormalTok{, }\DecValTok{200}\NormalTok{, }\DecValTok{251}\NormalTok{, }\DecValTok{316}\NormalTok{, }\DecValTok{398}\NormalTok{, }\DecValTok{501}\NormalTok{, }\DecValTok{631}\NormalTok{, }\DecValTok{794}\NormalTok{, }\DecValTok{1000}\NormalTok{)}
\NormalTok{p }\OtherTok{\textless{}{-}} \FunctionTok{c}\NormalTok{(}\DecValTok{0}\NormalTok{, }\DecValTok{0}\NormalTok{, }\DecValTok{0}\NormalTok{, }\DecValTok{0}\NormalTok{, }\DecValTok{0}\NormalTok{, }\FloatTok{0.5}\NormalTok{, }\FloatTok{0.5}\NormalTok{, }\DecValTok{1}\NormalTok{, }\DecValTok{1}\NormalTok{, }\DecValTok{1}\NormalTok{, }\DecValTok{1}\NormalTok{)  }\CommentTok{\# Доля погибших}
\NormalTok{d }\OtherTok{\textless{}{-}} \FunctionTok{data.frame}\NormalTok{(K, p)}

\CommentTok{\# Построение логистической модели}
\NormalTok{logit\_model }\OtherTok{\textless{}{-}} \FunctionTok{glm}\NormalTok{(p }\SpecialCharTok{\textasciitilde{}}\NormalTok{ K, }\AttributeTok{family =} \FunctionTok{binomial}\NormalTok{(), }\AttributeTok{data =}\NormalTok{ d)}

\CommentTok{\# Визуализация S{-}образной кривой}
\FunctionTok{plot}\NormalTok{(d}\SpecialCharTok{$}\NormalTok{K, d}\SpecialCharTok{$}\NormalTok{p, }
     \AttributeTok{xlab =} \StringTok{"Концентрация токсиканта (мг/л)"}\NormalTok{, }
     \AttributeTok{ylab =} \StringTok{"Доля погибших"}\NormalTok{, }
     \AttributeTok{main =} \StringTok{"Токсическое воздействие на дафний"}\NormalTok{,}
     \AttributeTok{pch =} \DecValTok{19}\NormalTok{,}
     \AttributeTok{col =} \StringTok{"red"}\NormalTok{)}
\FunctionTok{lines}\NormalTok{(d}\SpecialCharTok{$}\NormalTok{K, }\FunctionTok{predict}\NormalTok{(logit\_model, }\AttributeTok{type =} \StringTok{"response"}\NormalTok{), }
      \AttributeTok{col =} \StringTok{"blue"}\NormalTok{, }\AttributeTok{lwd =} \DecValTok{2}\NormalTok{, }\AttributeTok{lty =} \DecValTok{1}\NormalTok{)}
\end{Highlighting}
\end{Shaded}

\begin{figure}[H]

{\centering \includegraphics[width=0.6\linewidth,height=\textheight,keepaspectratio]{images/KOROSOV3.PNG}

}

\caption{Рис. 3.: Расчет логистической регрессии гибели дафний в
токсиканте}

\end{figure}%

\section{ПЕРЕХОД К
СЕТЯМ}\label{ux43fux435ux440ux435ux445ux43eux434-ux43a-ux441ux435ux442ux44fux43c}

Сделаем первый шаг к нейронным сетям, построив простейшую сеть без
скрытых слоев (аналог линейной регрессии) для модели токсичности. Вы
визуализируете структуру сети и убедитесь, что она дает результат,
аналогичный линейной модели.

\begin{Shaded}
\begin{Highlighting}[]
\CommentTok{\# Простейшая нейросеть (аналог линейной регрессии)}
\NormalTok{nn\_simple }\OtherTok{\textless{}{-}} \FunctionTok{neuralnet}\NormalTok{(p }\SpecialCharTok{\textasciitilde{}}\NormalTok{ K, }\AttributeTok{data =}\NormalTok{ d, }\AttributeTok{hidden =} \DecValTok{0}\NormalTok{)}

\CommentTok{\# Визуализация структуры сети}
\FunctionTok{plot}\NormalTok{(nn\_simple, }\AttributeTok{rep =} \StringTok{"best"}\NormalTok{)}
\end{Highlighting}
\end{Shaded}

\begin{figure}[H]

{\centering \includegraphics[width=0.4\linewidth,height=\textheight,keepaspectratio]{images/KOROSOV4.PNG}

}

\caption{Рис. 4.: Схема нейрона}

\end{figure}%

\section{НЕЙРОНЫ КАК НЕЛИНЕЙНЫЕ
ПРЕОБРАЗОВАТЕЛИ}\label{ux43dux435ux439ux440ux43eux43dux44b-ux43aux430ux43a-ux43dux435ux43bux438ux43dux435ux439ux43dux44bux435-ux43fux440ux435ux43eux431ux440ux430ux437ux43eux432ux430ux442ux435ux43bux438}

Здесь вы добавите в нейронную сеть скрытый слой с одним нейроном, что
позволит моделировать нелинейные зависимости. Вы сравните результат
работы такой сети с логистической регрессией и увидите, как нейронная
сеть имитирует пороговый эффект.

\begin{Shaded}
\begin{Highlighting}[]
\CommentTok{\# Сеть с одним скрытым нейроном (имитирует логистическую регрессию)}
\NormalTok{nn\_1hidden }\OtherTok{\textless{}{-}} \FunctionTok{neuralnet}\NormalTok{(p }\SpecialCharTok{\textasciitilde{}}\NormalTok{ K, }\AttributeTok{data =}\NormalTok{ d, }\AttributeTok{hidden =} \DecValTok{1}\NormalTok{)}

\CommentTok{\# Сравнение с логистической регрессией}
\FunctionTok{plot}\NormalTok{(d}\SpecialCharTok{$}\NormalTok{K, }\FunctionTok{predict}\NormalTok{(logit\_model, }\AttributeTok{type =} \StringTok{"response"}\NormalTok{), }
     \AttributeTok{type =} \StringTok{"l"}\NormalTok{, }
     \AttributeTok{col =} \StringTok{"darkgreen"}\NormalTok{, }
     \AttributeTok{lwd =} \DecValTok{2}\NormalTok{,}
     \AttributeTok{xlab =} \StringTok{"Концентрация"}\NormalTok{, }
     \AttributeTok{ylab =} \StringTok{"Смертность"}\NormalTok{,}
     \AttributeTok{main =} \StringTok{"Сравнение моделей"}\NormalTok{)}
\FunctionTok{lines}\NormalTok{(d}\SpecialCharTok{$}\NormalTok{K, }\FunctionTok{predict}\NormalTok{(nn\_1hidden, d), }\AttributeTok{col =} \StringTok{"blue"}\NormalTok{, }\AttributeTok{lty =} \DecValTok{2}\NormalTok{, }\AttributeTok{lwd =} \DecValTok{2}\NormalTok{)}
\FunctionTok{legend}\NormalTok{(}\StringTok{"bottomright"}\NormalTok{, }
       \AttributeTok{legend =} \FunctionTok{c}\NormalTok{(}\StringTok{"Логистическая регрессия"}\NormalTok{, }\StringTok{"Нейронная сеть (1 нейрон)"}\NormalTok{),}
       \AttributeTok{col =} \FunctionTok{c}\NormalTok{(}\StringTok{"darkgreen"}\NormalTok{, }\StringTok{"blue"}\NormalTok{), }
       \AttributeTok{lty =} \DecValTok{1}\SpecialCharTok{:}\DecValTok{2}\NormalTok{,}
       \AttributeTok{lwd =} \DecValTok{2}\NormalTok{)}
\end{Highlighting}
\end{Shaded}

\begin{figure}[H]

{\centering \includegraphics[width=0.4\linewidth,height=\textheight,keepaspectratio]{images/KOROSOV5.PNG}

}

\caption{Рис. 5.: Сравнение работы}

\end{figure}%

\section{КЛАССИФИКАЦИЯ В
ЭКОЛОГИИ}\label{ux43aux43bux430ux441ux441ux438ux444ux438ux43aux430ux446ux438ux44f-ux432-ux44dux43aux43eux43bux43eux433ux438ux438}

Вы примените нейронные сети для решения задачи классификации -
определения пола гадюк по морфометрическим признакам. Построив и сравнив
несколько архитектур сетей (без скрытых нейронов, с одним и тремя
нейронами), вы оцените их точность.

\begin{Shaded}
\begin{Highlighting}[]
\CommentTok{\# Загрузка данных по гадюкам (пол, длина тела, длина хвоста, масса)}
\NormalTok{v }\OtherTok{\textless{}{-}} \FunctionTok{read.csv}\NormalTok{(}\StringTok{"vipkar.csv"}\NormalTok{)}
\FunctionTok{head}\NormalTok{(v, }\DecValTok{3}\NormalTok{)  }\CommentTok{\# Просмотр первых строк данных}
\end{Highlighting}
\end{Shaded}

Модель без скрытых нейронов (аналог линейной регрессии)

\begin{Shaded}
\begin{Highlighting}[]
\NormalTok{nv0 }\OtherTok{\textless{}{-}} \FunctionTok{neuralnet}\NormalTok{(ns }\SpecialCharTok{\textasciitilde{}}\NormalTok{ lc, }\AttributeTok{data =}\NormalTok{ v, }\AttributeTok{hidden =} \DecValTok{0}\NormalTok{)}
\FunctionTok{plot}\NormalTok{(nv0)  }\CommentTok{\# Визуализация простейшей сети}
\end{Highlighting}
\end{Shaded}

\begin{figure}[H]

{\centering \includegraphics[width=0.4\linewidth,height=\textheight,keepaspectratio]{images/KOROSOV6.PNG}

}

\caption{Рис. 6.: Визуализация простейшей сети}

\end{figure}%

Модель с одним скрытым нейроном

\begin{Shaded}
\begin{Highlighting}[]
\NormalTok{nv1 }\OtherTok{\textless{}{-}} \FunctionTok{neuralnet}\NormalTok{(ns }\SpecialCharTok{\textasciitilde{}}\NormalTok{ lc, }\AttributeTok{data =}\NormalTok{ v, }\AttributeTok{hidden =} \DecValTok{1}\NormalTok{)}
\FunctionTok{plot}\NormalTok{(nv1)  }\CommentTok{\# Схема сети с одним нейроном}
\end{Highlighting}
\end{Shaded}

\begin{figure}[H]

{\centering \includegraphics[width=0.4\linewidth,height=\textheight,keepaspectratio]{images/KOROSOV7.PNG}

}

\caption{Рис. 7.: Схема сети с одним нейроном}

\end{figure}%

Модель с тремя скрытыми нейронами (полноценная нейросеть)

\begin{Shaded}
\begin{Highlighting}[]
\NormalTok{nv3 }\OtherTok{\textless{}{-}} \FunctionTok{neuralnet}\NormalTok{(ns }\SpecialCharTok{\textasciitilde{}}\NormalTok{ lc }\SpecialCharTok{+}\NormalTok{ lt }\SpecialCharTok{+}\NormalTok{ w, }\AttributeTok{data =}\NormalTok{ v, }\AttributeTok{hidden =} \DecValTok{3}\NormalTok{)}
\FunctionTok{plot}\NormalTok{(nv3)  }\CommentTok{\# Визуализация сложной сети}
\end{Highlighting}
\end{Shaded}

\begin{figure}[H]

{\centering \includegraphics[width=0.4\linewidth,height=\textheight,keepaspectratio]{images/KOROSOV8.PNG}

}

\caption{Рис. 8.: Модель с тремя скрытыми нейронами}

\end{figure}%

Оценка точности классификации

\begin{Shaded}
\begin{Highlighting}[]
\NormalTok{predictions }\OtherTok{\textless{}{-}} \FunctionTok{predict}\NormalTok{(nv3, v)}
\NormalTok{predicted\_sex }\OtherTok{\textless{}{-}} \FunctionTok{round}\NormalTok{(predictions)}
\NormalTok{accuracy }\OtherTok{\textless{}{-}} \FunctionTok{mean}\NormalTok{(v}\SpecialCharTok{$}\NormalTok{ns }\SpecialCharTok{==}\NormalTok{ predicted\_sex)}
\FunctionTok{cat}\NormalTok{(}\StringTok{"Точность классификации:"}\NormalTok{, }\FunctionTok{round}\NormalTok{(accuracy}\SpecialCharTok{*}\DecValTok{100}\NormalTok{, }\DecValTok{1}\NormalTok{), }\StringTok{"\%}\SpecialCharTok{\textbackslash{}n}\StringTok{"}\NormalTok{)}
\end{Highlighting}
\end{Shaded}

Сравнение разных архитектур нейронных сетей (см. срипт
\href{https://mombus.github.io/cRab/data/KOROSOV_visual.R}{KOROSOV\_visual.R})

\begin{figure}[H]

{\centering \includegraphics[width=0.6\linewidth,height=\textheight,keepaspectratio]{images/KOROSOV9.PNG}

}

\caption{Рис. 9.: Точность определения пола гадюк}

\end{figure}%

\section{ПРОСТРАНСТВЕННОЕ
МОДЕЛИРОВАНИЕ}\label{ux43fux440ux43eux441ux442ux440ux430ux43dux441ux442ux432ux435ux43dux43dux43eux435-ux43cux43eux434ux435ux43bux438ux440ux43eux432ux430ux43dux438ux435}

В завершение вы построите нейронную сеть для прогнозирования численности
гадюк на островах по характеристикам биотопов. Вы разделите данные на
обучающую и тестовую выборки, оцените точность модели и используете ее
для прогноза в новых условиях.

\begin{Shaded}
\begin{Highlighting}[]
\CommentTok{\# Данные по островам Кижского архипелага}
\NormalTok{v }\OtherTok{\textless{}{-}} \FunctionTok{read.csv}\NormalTok{(}\StringTok{"kihzsdat.csv"}\NormalTok{)}
\FunctionTok{head}\NormalTok{(v, }\DecValTok{3}\NormalTok{)  }\CommentTok{\# Структура данных: площадь, биотопы, численность видов}

\CommentTok{\# Случайное разделение данных на обучающую и тестовую выборки}
\FunctionTok{set.seed}\NormalTok{(}\DecValTok{123}\NormalTok{)  }\CommentTok{\# Для воспроизводимости}
\NormalTok{train\_indices }\OtherTok{\textless{}{-}} \FunctionTok{sample}\NormalTok{(}\DecValTok{1}\SpecialCharTok{:}\FunctionTok{nrow}\NormalTok{(v), }\DecValTok{12}\NormalTok{)}
\NormalTok{train\_data }\OtherTok{\textless{}{-}}\NormalTok{ v[train\_indices, ]}
\NormalTok{test\_data }\OtherTok{\textless{}{-}}\NormalTok{ v[}\SpecialCharTok{{-}}\NormalTok{train\_indices, ]}

\CommentTok{\# Построение нейросети с 5 нейронами в скрытом слое}
\NormalTok{model }\OtherTok{\textless{}{-}} \FunctionTok{neuralnet}\NormalTok{(vb }\SpecialCharTok{\textasciitilde{}}\NormalTok{ fo }\SpecialCharTok{+}\NormalTok{ me }\SpecialCharTok{+}\NormalTok{ bo, }\AttributeTok{data =}\NormalTok{ train\_data, }\AttributeTok{hidden =} \DecValTok{5}\NormalTok{)}

\CommentTok{\# Прогнозирование на обучающей выборке}
\NormalTok{train\_pred }\OtherTok{\textless{}{-}} \FunctionTok{predict}\NormalTok{(model, train\_data)}
\NormalTok{train\_accuracy }\OtherTok{\textless{}{-}} \FunctionTok{mean}\NormalTok{(}\FunctionTok{round}\NormalTok{(train\_pred) }\SpecialCharTok{==}\NormalTok{ train\_data}\SpecialCharTok{$}\NormalTok{vb)}
\FunctionTok{cat}\NormalTok{(}\StringTok{"Точность на обучающей выборке:"}\NormalTok{, }\FunctionTok{round}\NormalTok{(train\_accuracy}\SpecialCharTok{*}\DecValTok{100}\NormalTok{, }\DecValTok{1}\NormalTok{), }\StringTok{"\%}\SpecialCharTok{\textbackslash{}n}\StringTok{"}\NormalTok{)}

\CommentTok{\# Прогнозирование на тестовой выборке}
\NormalTok{test\_pred }\OtherTok{\textless{}{-}} \FunctionTok{predict}\NormalTok{(model, test\_data)}
\NormalTok{test\_accuracy }\OtherTok{\textless{}{-}} \FunctionTok{mean}\NormalTok{(}\FunctionTok{round}\NormalTok{(test\_pred) }\SpecialCharTok{==}\NormalTok{ test\_data}\SpecialCharTok{$}\NormalTok{vb)}
\FunctionTok{cat}\NormalTok{(}\StringTok{"Точность на тестовой выборке:"}\NormalTok{, }\FunctionTok{round}\NormalTok{(test\_accuracy}\SpecialCharTok{*}\DecValTok{100}\NormalTok{, }\DecValTok{1}\NormalTok{), }\StringTok{"\%}\SpecialCharTok{\textbackslash{}n}\StringTok{"}\NormalTok{)}

\CommentTok{\# Прогноз для новых условий (пример)}
\NormalTok{new\_conditions }\OtherTok{\textless{}{-}} \FunctionTok{data.frame}\NormalTok{(}
  \AttributeTok{fo =} \FunctionTok{c}\NormalTok{(}\FloatTok{57.9}\NormalTok{, }\FloatTok{35.3}\NormalTok{, }\FloatTok{83.0}\NormalTok{),  }\CommentTok{\# Площадь лесов (\%)}
  \AttributeTok{me =} \FunctionTok{c}\NormalTok{(}\FloatTok{4.1}\NormalTok{, }\FloatTok{0.0}\NormalTok{, }\FloatTok{7.3}\NormalTok{),     }\CommentTok{\# Площадь лугов (\%)}
  \AttributeTok{bo =} \FunctionTok{c}\NormalTok{(}\FloatTok{3.4}\NormalTok{, }\FloatTok{7.9}\NormalTok{, }\FloatTok{11.5}\NormalTok{)     }\CommentTok{\# Площадь болот (\%)}
\NormalTok{)}

\NormalTok{future\_pred }\OtherTok{\textless{}{-}} \FunctionTok{predict}\NormalTok{(model, new\_conditions)}
\FunctionTok{cat}\NormalTok{(}\StringTok{"Прогнозируемая численность гадюк:"}\NormalTok{, }\FunctionTok{round}\NormalTok{(future\_pred), }\StringTok{"}\SpecialCharTok{\textbackslash{}n}\StringTok{"}\NormalTok{)}
\end{Highlighting}
\end{Shaded}

\bookmarksetup{startatroot}

\chapter{Основы
картографии}\label{ux43eux441ux43dux43eux432ux44b-ux43aux430ux440ux442ux43eux433ux440ux430ux444ux438ux438}

\section{Введение}\label{ux432ux432ux435ux434ux435ux43dux438ux435-3}

Это занятие --- про то, как превратить координаты и уловы в честные
карты, которые помогают думать, а не просто украшать отчёт. Мы будем
работать в R, потому что он дисциплинирует: каждая операция видна,
воспроизводима и проверяема, а любой красивый результат можно разобрать
до строчки кода. В логике курса мы пойдём от простого к сложному: от
базовой точечной карты распределения уловов --- к картам с береговой
линией, к учёту нулевых уловов и разбиению по квартилям, к фасетам для
сравнения лет, к локальной автокорреляции (LISA), к промысловым картам и
картограммам, и, наконец, к гибридным решениям, где данные съёмок и
промысла встречаются на одной карте. В конце добавим «служебные» карты
для раздела «Материал и методы» и карты с врезками --- тот самый
минимум, который ожидают рецензенты. В духе Ноама Хомского напомним:
карта --- это не территория, а модель наших предположений; чем
прозрачнее исходные решения (данные, проекция, шкалы), тем меньше
поводов для самообмана.

В основе любой качественной карты лежат три вещи: корректная подготовка
данных, грамотная картографическая основа и осмысленная визуальная
метафора. Сначала убеждаемся, что координаты в единой системе (WGS84),
отсутствуют перепутанные долготы и широты, а нулевые уловы размечены как
нули, а не пропуски. Затем выбираем основу: береговая линия и полигоны
стран из rnaturalearth, при необходимости батиметрия из marmap,
корректная проекция для расстояний и площадей (в задачах локального
масштаба --- UTM). И только после этого --- визуальные решения:
непрерывные шкалы с воспринимаемо ровной палитрой (viridis), понятная
легенда, единицы измерения, подписи, масштабная линейка и, где уместно,
стрелка «север». Важный этический момент: размер и цвет несут разные
смыслы; не заставляйте читателя угадывать, что из них интенсивность, а
что --- частота или категория. Нулевые уловы --- это не «мусор», а
сильный сигнал об отсутствии; показывайте их отдельным слоем и символом,
чтобы не переоценивать «горячие точки».

По мере усложнения задач мы добавляем структуру. Разбиение по квартилям
даёт сопоставимость между годами, фасеты позволяют увидеть межгодовую
динамику без наложения, LISA подсвечивает кластеры высокой и низкой
интенсивности и аномалии, где значение точки расходится с окружением.
Картограммы или сеточная агрегация помогают уйти от шумной точки к
устойчивой картинке на уровне промысловых квадратов; гибридные карты
честно показывают возможный разрыв между научной съёмкой и промыслом.
Здесь важно помнить про «анатомию ошибки»: выбор размера ячейки, числа
соседей в LISA, границ квартилей и способа агрегации --- это не
техническая деталь, а модельное решение; фиксируйте его явно, чтобы
завтра вы сами могли воспроизвести сегодняшнюю карту.

Практический результат должен быть пригоден для публикации. Все примеры
в R можно экспортировать в векторные форматы (PDF, SVG) и растровые
(PNG, TIFF) с высоким разрешением, где подписи, легенды и цвета
сохраняют читаемость при печати. В скрипте мы покажем, как автоматически
подбирать границы области с небольшим буфером --- и почему чаще лучше
задать их вручную, чтобы карта не «гуляла» между рис. 1 и рис. 2. Мы
разберём, как организовать легенды, чтобы они не спорили друг с другом
при фасетировании, как синхронизировать цветовые шкалы между годами,
чтобы зелёное «вчера» и зелёное «сегодня» значили одно и то же, и как
использовать врезку, чтобы читатель понял контекст региона, а не искал
его на глобусе.

Наконец, про дисциплину и воспроизводимость. Данные берём из одного
файла
(\href{https://mombus.github.io/cRab/data/KARTOGRAPHIC.xlsx}{KARTOGRAPHIC.xlsx}),
зависимости минимальны и явно перечислены, рабочая директория задаётся в
начале, все параметры карт --- на виду. Такой стиль не только ускоряет
работу, но и воспитывает привычку проверять себя: если карта получилась
слишком «красивая», вернитесь и взгляните на нули, на шкалы, на
проекцию, на подписи. Хорошая карта в рыбохозяйственной и
гидробиологической практике --- это не ``арт‑объект'', а прозрачный
инструмент коммуникации: с ней удобно спорить, её можно повторить и на
её основе можно принять решение. Если к концу занятия вы без подсказок
соберёте три‑четыре типовых карты для результатов и одну аккуратную для
«Материалов и методов», задача занятия выполнена.

\textbf{Для работы скрипта:}

\begin{enumerate}
\def\labelenumi{\arabic{enumi}.}
\item
  Скачайте файл данных
  (\href{https://mombus.github.io/cRab/data/KARTOGRAPHIC.xlsx}{KARTOGRAPHIC.xlsx})
\item
  Установите рабочую директорию в setwd()
\item
  Установите необходимые пакеты :
  \textbf{\texttt{install.packages(c("readxl",\ "tidyverse,\ "rnaturalearth",\ "sf",\ "viridis"\ ))}}
  \texttt{и\ др.}
\end{enumerate}

\section{Карта распределения уловов в
съемке}\label{ux43aux430ux440ux442ux430-ux440ux430ux441ux43fux440ux435ux434ux435ux43bux435ux43dux438ux44f-ux443ux43bux43eux432ux43eux432-ux432-ux441ux44aux435ux43cux43aux435}

Данная карта демонстрирует распределение уловов краба в ходе
исследовательской съемки. На ней отображены точки наблюдений, где размер
и цвет точек соответствуют величине улова.

\begin{figure}[H]

{\centering \includegraphics[width=0.7\linewidth,height=\textheight,keepaspectratio]{images/KARTOGRAPH1.jpg}

}

\caption{Рис. 1.: Пример карты распределения уловов в съемке}

\end{figure}%

В скрипте границы карты (лимиты) определяются автоматически с буфером,
но чаще их просто устанавливают вручную, например:

\begin{Shaded}
\begin{Highlighting}[]
\NormalTok{xmin }\OtherTok{\textless{}{-}} \DecValTok{37}
\NormalTok{xmax }\OtherTok{\textless{}{-}} \DecValTok{49}
\NormalTok{ymin }\OtherTok{\textless{}{-}} \FloatTok{68.5}
\NormalTok{ymax }\OtherTok{\textless{}{-}} \FloatTok{70.5}
\end{Highlighting}
\end{Shaded}

Скрипт карты целиком:

\begin{Shaded}
\begin{Highlighting}[]
\CommentTok{\# Очистка памяти и установка рабочей папки}
\FunctionTok{rm}\NormalTok{(}\AttributeTok{list =} \FunctionTok{ls}\NormalTok{())}
\FunctionTok{setwd}\NormalTok{(}\StringTok{"C:/COURSES/KARTOGRAPH/"}\NormalTok{)}

\CommentTok{\# Загрузка необходимых пакетов}
\FunctionTok{library}\NormalTok{(tidyverse)  }\CommentTok{\# Обработка данных и визуализация}
\FunctionTok{library}\NormalTok{(readxl)     }\CommentTok{\# Чтение Excel{-}файлов}

\CommentTok{\# 1. ЗАГРУЗКА ДАННЫХ}
\NormalTok{DATA }\OtherTok{\textless{}{-}} \FunctionTok{read\_excel}\NormalTok{(}\StringTok{"KARTOGRAPHIC.xlsx"}\NormalTok{, }\AttributeTok{sheet =} \StringTok{"SURVEY"}\NormalTok{) }\SpecialCharTok{\%\textgreater{}\%} 
  \FunctionTok{filter}\NormalTok{(YEAR }\SpecialCharTok{==} \DecValTok{2023}\NormalTok{, SURV }\SpecialCharTok{==} \StringTok{"CRAB"}\NormalTok{)  }\CommentTok{\# Фильтр для 2023 года и съемки CRAB}

\CommentTok{\# 2. АВТОМАТИЧЕСКИЙ РАСЧЕТ ГРАНИЦ С БУФЕРОМ 5\%}
\CommentTok{\# Расчет диапазонов координат}
\NormalTok{x\_range }\OtherTok{\textless{}{-}} \FunctionTok{range}\NormalTok{(DATA}\SpecialCharTok{$}\NormalTok{X, }\AttributeTok{na.rm =} \ConstantTok{TRUE}\NormalTok{)}
\NormalTok{y\_range }\OtherTok{\textless{}{-}} \FunctionTok{range}\NormalTok{(DATA}\SpecialCharTok{$}\NormalTok{Y, }\AttributeTok{na.rm =} \ConstantTok{TRUE}\NormalTok{)}

\CommentTok{\# Расчет 5\% буфера}
\NormalTok{x\_buffer }\OtherTok{\textless{}{-}} \FloatTok{0.05} \SpecialCharTok{*} \FunctionTok{diff}\NormalTok{(x\_range)}
\NormalTok{y\_buffer }\OtherTok{\textless{}{-}} \FloatTok{0.05} \SpecialCharTok{*} \FunctionTok{diff}\NormalTok{(y\_range)}

\CommentTok{\# Установка границ с буфером}
\NormalTok{xmin }\OtherTok{\textless{}{-}}\NormalTok{ x\_range[}\DecValTok{1}\NormalTok{] }\SpecialCharTok{{-}}\NormalTok{ x\_buffer}
\NormalTok{xmax }\OtherTok{\textless{}{-}}\NormalTok{ x\_range[}\DecValTok{2}\NormalTok{] }\SpecialCharTok{+}\NormalTok{ x\_buffer}
\NormalTok{ymin }\OtherTok{\textless{}{-}}\NormalTok{ y\_range[}\DecValTok{1}\NormalTok{] }\SpecialCharTok{{-}}\NormalTok{ y\_buffer}
\NormalTok{ymax }\OtherTok{\textless{}{-}}\NormalTok{ y\_range[}\DecValTok{2}\NormalTok{] }\SpecialCharTok{+}\NormalTok{ y\_buffer}

\CommentTok{\# 3. ВИЗУАЛИЗАЦИЯ ТОЧЕК}
\FunctionTok{ggplot}\NormalTok{(DATA) }\SpecialCharTok{+}
  \CommentTok{\# Точки наблюдений с размером и цветом по величине улова}
  \FunctionTok{geom\_point}\NormalTok{(}\FunctionTok{aes}\NormalTok{(}\AttributeTok{x =}\NormalTok{ X, }\AttributeTok{y =}\NormalTok{ Y, }\AttributeTok{size =}\NormalTok{ PROM, }\AttributeTok{color =}\NormalTok{ PROM), }\AttributeTok{alpha =} \FloatTok{0.7}\NormalTok{) }\SpecialCharTok{+}
  
  \CommentTok{\# Цветовая шкала (виридисная палитра)}
  \FunctionTok{scale\_color\_viridis\_c}\NormalTok{(}\AttributeTok{option =} \StringTok{"H"}\NormalTok{, }\AttributeTok{name =} \StringTok{"Улов"}\NormalTok{) }\SpecialCharTok{+}
  
  \CommentTok{\# Шкала размеров точек}
  \FunctionTok{scale\_size\_continuous}\NormalTok{(}\AttributeTok{name =} \StringTok{"Улов"}\NormalTok{) }\SpecialCharTok{+}
  
  \CommentTok{\# Настройка границ с автоматически рассчитанными значениями}
  \FunctionTok{coord\_cartesian}\NormalTok{(}\AttributeTok{xlim =} \FunctionTok{c}\NormalTok{(xmin, xmax), }\AttributeTok{ylim =} \FunctionTok{c}\NormalTok{(ymin, ymax)) }\SpecialCharTok{+}
  
  \CommentTok{\# Подписи осей}
  \FunctionTok{labs}\NormalTok{(}\AttributeTok{x =} \StringTok{"Долгота"}\NormalTok{, }\AttributeTok{y =} \StringTok{"Широта"}\NormalTok{, }
       \AttributeTok{title =} \StringTok{"Распределение уловов краба"}\NormalTok{, }
       \AttributeTok{subtitle =} \StringTok{"2023 год, тип съемки: CRAB"}\NormalTok{) }\SpecialCharTok{+}
  
  \CommentTok{\# Оформление графика}
  \FunctionTok{theme\_bw}\NormalTok{() }\SpecialCharTok{+}
  \FunctionTok{theme}\NormalTok{(}
    \AttributeTok{panel.grid =} \FunctionTok{element\_line}\NormalTok{(}\AttributeTok{color =} \StringTok{"grey90"}\NormalTok{),}
    \AttributeTok{legend.position =} \StringTok{"bottom"}
\NormalTok{  )}
\end{Highlighting}
\end{Shaded}

\section{Карта распределения уловов в съемке с береговой
линией}\label{ux43aux430ux440ux442ux430-ux440ux430ux441ux43fux440ux435ux434ux435ux43bux435ux43dux438ux44f-ux443ux43bux43eux432ux43eux432-ux432-ux441ux44aux435ux43cux43aux435-ux441-ux431ux435ux440ux435ux433ux43eux432ux43eux439-ux43bux438ux43dux438ux435ux439}

\begin{figure}[H]

{\centering \includegraphics[width=0.7\linewidth,height=\textheight,keepaspectratio]{images/KARTOGRAPH2.jpg}

}

\caption{Рис. 2.: Пример карты распределения уловов в съемке с береговой
линией}

\end{figure}%

\begin{Shaded}
\begin{Highlighting}[]
\CommentTok{\# Очистка окружения и установка рабочей директории}
\FunctionTok{rm}\NormalTok{(}\AttributeTok{list =} \FunctionTok{ls}\NormalTok{())  }\CommentTok{\# Удаление всех объектов из глобального окружения}
\FunctionTok{setwd}\NormalTok{(}\StringTok{"C:/COURSES/KARTOGRAPH/"}\NormalTok{)  }\CommentTok{\# Установка рабочей директории}

\CommentTok{\# Загрузка необходимых библиотек}
\FunctionTok{library}\NormalTok{(rnaturalearth)  }\CommentTok{\# Для получения векторных карт мира}
\FunctionTok{library}\NormalTok{(tidyverse)      }\CommentTok{\# Коллекция пакетов для работы с данными}
\FunctionTok{library}\NormalTok{(sf)             }\CommentTok{\# Пространственный анализ}

\DocumentationTok{\#\#\#\#\#\#\# ЗАГРУЗКА ДАННЫХ И ПОДГОТОВКА ПРОСТРАНСТВЕННЫХ ОБЪЕКТОВ \#\#\#\#\#\#\#\#\#\#\#\#\#\#\#\#}

\CommentTok{\# Чтение и фильтрация данных}
\NormalTok{DATA }\OtherTok{\textless{}{-}}\NormalTok{ readxl}\SpecialCharTok{::}\FunctionTok{read\_excel}\NormalTok{(}\StringTok{"KARTOGRAPHIC.xlsx"}\NormalTok{, }\AttributeTok{sheet =} \StringTok{"SURVEY"}\NormalTok{) }\SpecialCharTok{\%\textgreater{}\%} 
  \FunctionTok{filter}\NormalTok{(YEAR }\SpecialCharTok{==} \DecValTok{2023}\NormalTok{, SURV }\SpecialCharTok{==} \StringTok{"CRAB"}\NormalTok{)  }\CommentTok{\# Фильтр данных за 2023 год по типу съемки}

\CommentTok{\# Получение границ России}
\NormalTok{russia }\OtherTok{\textless{}{-}} \FunctionTok{ne\_countries}\NormalTok{(}\AttributeTok{scale =} \DecValTok{10}\NormalTok{, }\AttributeTok{country =} \StringTok{"Russia"}\NormalTok{)  }\CommentTok{\# Загрузка векторных границ (масштаб 1:10м)}

\CommentTok{\# Установка границ отображаемой области (долгота/широта)}
\NormalTok{xmin}\OtherTok{=}\DecValTok{37}  \CommentTok{\# Западная граница}
\NormalTok{xmax}\OtherTok{=}\DecValTok{49}  \CommentTok{\# Восточная граница}
\NormalTok{ymin}\OtherTok{=}\FloatTok{68.5} \CommentTok{\# Южная граница}
\NormalTok{ymax}\OtherTok{=}\FloatTok{70.5} \CommentTok{\# Северная граница}

\CommentTok{\# Построение карты}
\FunctionTok{ggplot}\NormalTok{() }\SpecialCharTok{+}
  \CommentTok{\# Базовая карта России}
  \FunctionTok{geom\_sf}\NormalTok{(}\AttributeTok{data =}\NormalTok{ russia, }\AttributeTok{fill =} \StringTok{"lightblue"}\NormalTok{) }\SpecialCharTok{+} 
  \CommentTok{\# Ограничение области отображения}
  \FunctionTok{coord\_sf}\NormalTok{(}\AttributeTok{xlim =} \FunctionTok{c}\NormalTok{(xmin, xmax), }\AttributeTok{ylim =} \FunctionTok{c}\NormalTok{(ymin, ymax)) }\SpecialCharTok{+}
  \CommentTok{\# Точки наблюдений с размером и цветом по переменной PROM}
  \FunctionTok{geom\_point}\NormalTok{(}\FunctionTok{aes}\NormalTok{(}\AttributeTok{x =}\NormalTok{ X, }\AttributeTok{y =}\NormalTok{ Y, }\AttributeTok{size =}\NormalTok{ PROM, }\AttributeTok{color =}\NormalTok{ PROM),}
             \AttributeTok{data =}\NormalTok{ DATA, }\AttributeTok{alpha =} \FloatTok{0.6}\NormalTok{) }\SpecialCharTok{+}
  \CommentTok{\# Цветовая шкала (viridis, вариант H)}
  \FunctionTok{scale\_color\_viridis\_c}\NormalTok{(}\AttributeTok{option =} \StringTok{"H"}\NormalTok{)}
\end{Highlighting}
\end{Shaded}

\section{Карта распределения уловов, включая
нулевые}\label{ux43aux430ux440ux442ux430-ux440ux430ux441ux43fux440ux435ux434ux435ux43bux435ux43dux438ux44f-ux443ux43bux43eux432ux43eux432-ux432ux43aux43bux44eux447ux430ux44f-ux43dux443ux43bux435ux432ux44bux435}

\begin{figure}[H]

{\centering \includegraphics[width=0.7\linewidth,height=\textheight,keepaspectratio]{images/KARTOGRAPH3.jpg}

}

\caption{Рис. 3.: Карта распределения уловов, включая нулевые}

\end{figure}%

\begin{Shaded}
\begin{Highlighting}[]
\CommentTok{\# Очистка окружения и установка рабочей директории}
\FunctionTok{rm}\NormalTok{(}\AttributeTok{list =} \FunctionTok{ls}\NormalTok{())  }\CommentTok{\# Удаление всех объектов из глобального окружения}
\FunctionTok{setwd}\NormalTok{(}\StringTok{"C:/COURSES/KARTOGRAPH/"}\NormalTok{)  }\CommentTok{\# Установка рабочей директории}

\CommentTok{\# Загрузка необходимых библиотек}
\FunctionTok{library}\NormalTok{(rnaturalearth)  }\CommentTok{\# Для получения векторных карт мира}
\FunctionTok{library}\NormalTok{(tidyverse)      }\CommentTok{\# Коллекция пакетов для работы с данными}
\FunctionTok{library}\NormalTok{(sf)             }\CommentTok{\# Пространственный анализ}

\DocumentationTok{\#\#\#\#\#\#\# ЗАГРУЗКА ДАННЫХ И ПОДГОТОВКА ПРОСТРАНСТВЕННЫХ ОБЪЕКТОВ \#\#\#\#\#\#\#\#\#\#\#\#\#\#\#\#}

\CommentTok{\# Чтение и фильтрация данных}
\NormalTok{DATA }\OtherTok{\textless{}{-}}\NormalTok{ readxl}\SpecialCharTok{::}\FunctionTok{read\_excel}\NormalTok{(}\StringTok{"KARTOGRAPHIC.xlsx"}\NormalTok{, }\AttributeTok{sheet =} \StringTok{"SURVEY"}\NormalTok{) }\SpecialCharTok{\%\textgreater{}\%} 
  \FunctionTok{filter}\NormalTok{(YEAR }\SpecialCharTok{==} \DecValTok{2023}\NormalTok{, SURV }\SpecialCharTok{==} \StringTok{"CRAB"}\NormalTok{)  }\CommentTok{\# Фильтр данных за 2023 год по типу съемки}

\CommentTok{\# Получение границ России}
\NormalTok{russia }\OtherTok{\textless{}{-}} \FunctionTok{ne\_countries}\NormalTok{(}\AttributeTok{scale =} \DecValTok{10}\NormalTok{, }\AttributeTok{country =} \StringTok{"Russia"}\NormalTok{)  }\CommentTok{\# Загрузка векторных границ (масштаб 1:10м)}

\CommentTok{\# Установка границ отображаемой области (долгота/широта)}
\NormalTok{xmin}\OtherTok{=}\DecValTok{37}  \CommentTok{\# Западная граница}
\NormalTok{xmax}\OtherTok{=}\DecValTok{49}  \CommentTok{\# Восточная граница}
\NormalTok{ymin}\OtherTok{=}\FloatTok{68.5} \CommentTok{\# Южная граница}
\NormalTok{ymax}\OtherTok{=}\FloatTok{70.5} \CommentTok{\# Северная граница}

\CommentTok{\# Построение карты}
\FunctionTok{ggplot}\NormalTok{() }\SpecialCharTok{+}
  \CommentTok{\# Базовая карта России}
  \FunctionTok{geom\_sf}\NormalTok{(}\AttributeTok{data =}\NormalTok{ russia, }\AttributeTok{fill =} \StringTok{"lightblue"}\NormalTok{) }\SpecialCharTok{+} 
  \CommentTok{\# Ограничение области отображения}
  \FunctionTok{coord\_sf}\NormalTok{(}\AttributeTok{xlim =} \FunctionTok{c}\NormalTok{(xmin, xmax), }\AttributeTok{ylim =} \FunctionTok{c}\NormalTok{(ymin, ymax)) }\SpecialCharTok{+}
  \CommentTok{\# Точки наблюдений с размером и цветом по переменной PROM (ненулевые уловы)}
  \FunctionTok{geom\_point}\NormalTok{(}\FunctionTok{aes}\NormalTok{(}\AttributeTok{x =}\NormalTok{ X, }\AttributeTok{y =}\NormalTok{ Y, }\AttributeTok{size =}\NormalTok{ PROM, }\AttributeTok{color =}\NormalTok{ PROM),}
             \AttributeTok{data =} \FunctionTok{filter}\NormalTok{(DATA, PROM }\SpecialCharTok{\textgreater{}} \DecValTok{0}\NormalTok{), }\AttributeTok{alpha =} \FloatTok{0.6}\NormalTok{) }\SpecialCharTok{+}
  \CommentTok{\# Точки для нулевых уловов (крестики)}
  \FunctionTok{geom\_point}\NormalTok{(}\FunctionTok{aes}\NormalTok{(}\AttributeTok{x =}\NormalTok{ X, }\AttributeTok{y =}\NormalTok{ Y),}
             \AttributeTok{data =} \FunctionTok{filter}\NormalTok{(DATA, PROM }\SpecialCharTok{==} \DecValTok{0}\NormalTok{),}
             \AttributeTok{shape =} \DecValTok{4}\NormalTok{, }\AttributeTok{size =} \DecValTok{1}\NormalTok{, }\AttributeTok{stroke =} \DecValTok{1}\NormalTok{, }\AttributeTok{color =} \StringTok{"black"}\NormalTok{) }\SpecialCharTok{+}
  \CommentTok{\# Цветовая шкала (viridis, вариант H)}
  \FunctionTok{scale\_color\_viridis\_c}\NormalTok{(}\AttributeTok{option =} \StringTok{"H"}\NormalTok{)}
\end{Highlighting}
\end{Shaded}

\section{Карта распределения уловов, распределенных по
квартилям}\label{ux43aux430ux440ux442ux430-ux440ux430ux441ux43fux440ux435ux434ux435ux43bux435ux43dux438ux44f-ux443ux43bux43eux432ux43eux432-ux440ux430ux441ux43fux440ux435ux434ux435ux43bux435ux43dux43dux44bux445-ux43fux43e-ux43aux432ux430ux440ux442ux438ux43bux44fux43c}

\begin{figure}[H]

{\centering \includegraphics[width=0.7\linewidth,height=\textheight,keepaspectratio]{images/KARTOGRAPH4.jpg}

}

\caption{Рис. 4.: Карта распределения уловов, распределенных по
квартилям}

\end{figure}%

\begin{Shaded}
\begin{Highlighting}[]
\CommentTok{\# Очистка окружения и установка рабочей директории}
\FunctionTok{rm}\NormalTok{(}\AttributeTok{list =} \FunctionTok{ls}\NormalTok{())}
\FunctionTok{setwd}\NormalTok{(}\StringTok{"C:/COURSES/KARTOGRAPH/"}\NormalTok{)}

\CommentTok{\# Загрузка необходимых библиотек}
\FunctionTok{library}\NormalTok{(rnaturalearth)}
\FunctionTok{library}\NormalTok{(tidyverse)}
\FunctionTok{library}\NormalTok{(sf)}

\DocumentationTok{\#\#\#\#\#\#\# ЗАГРУЗКА ДАННЫХ И ПОДГОТОВКА ПРОСТРАНСТВЕННЫХ ОБЪЕКТОВ \#\#\#\#\#\#\#\#\#\#\#\#\#\#\#\#}

\CommentTok{\# Чтение и фильтрация данных}
\NormalTok{DATA }\OtherTok{\textless{}{-}}\NormalTok{ readxl}\SpecialCharTok{::}\FunctionTok{read\_excel}\NormalTok{(}\StringTok{"KARTOGRAPHIC.xlsx"}\NormalTok{, }\AttributeTok{sheet =} \StringTok{"SURVEY"}\NormalTok{) }\SpecialCharTok{\%\textgreater{}\%} 
  \FunctionTok{filter}\NormalTok{(YEAR }\SpecialCharTok{==} \DecValTok{2023}\NormalTok{, SURV }\SpecialCharTok{==} \StringTok{"CRAB"}\NormalTok{)}

\CommentTok{\# Получение границ России}
\NormalTok{russia }\OtherTok{\textless{}{-}} \FunctionTok{ne\_countries}\NormalTok{(}\AttributeTok{scale =} \DecValTok{10}\NormalTok{, }\AttributeTok{country =} \StringTok{"Russia"}\NormalTok{) }\SpecialCharTok{\%\textgreater{}\%} 
  \FunctionTok{st\_as\_sf}\NormalTok{()}

\CommentTok{\# Установка границ отображаемой области}
\NormalTok{xmin}\OtherTok{=}\DecValTok{37}\NormalTok{; xmax}\OtherTok{=}\DecValTok{49}\NormalTok{; ymin}\OtherTok{=}\FloatTok{68.5}\NormalTok{; ymax}\OtherTok{=}\FloatTok{70.5}

\DocumentationTok{\#\#\#\#\#\#\# ПОДГОТОВКА ДАННЫХ ДЛЯ ВИЗУАЛИЗАЦИИ \#\#\#\#\#\#\#\#\#\#\#\#\#\#\#\#}
\CommentTok{\# Вычисляем квартили отдельно}
\NormalTok{quantiles }\OtherTok{\textless{}{-}} \FunctionTok{quantile}\NormalTok{(DATA}\SpecialCharTok{$}\NormalTok{PROM[DATA}\SpecialCharTok{$}\NormalTok{PROM }\SpecialCharTok{\textgreater{}} \DecValTok{0}\NormalTok{], }\AttributeTok{probs =} \FunctionTok{seq}\NormalTok{(}\DecValTok{0}\NormalTok{, }\DecValTok{1}\NormalTok{, }\FloatTok{0.25}\NormalTok{))}

\CommentTok{\# Создаем 4 категории с реальными диапазонами значений}
\NormalTok{nonzero\_data }\OtherTok{\textless{}{-}}\NormalTok{ DATA }\SpecialCharTok{\%\textgreater{}\%} 
  \FunctionTok{filter}\NormalTok{(PROM }\SpecialCharTok{\textgreater{}} \DecValTok{0}\NormalTok{) }\SpecialCharTok{\%\textgreater{}\%}
  \FunctionTok{mutate}\NormalTok{(}
    \AttributeTok{PROM\_cat =} \FunctionTok{cut}\NormalTok{(}
\NormalTok{      PROM,}
      \AttributeTok{breaks =}\NormalTok{ quantiles,}
      \AttributeTok{include.lowest =} \ConstantTok{TRUE}\NormalTok{,}
      \AttributeTok{labels =} \FunctionTok{c}\NormalTok{(}
        \FunctionTok{sprintf}\NormalTok{(}\StringTok{"\%.1f {-} \%.1f"}\NormalTok{, quantiles[}\DecValTok{1}\NormalTok{], quantiles[}\DecValTok{2}\NormalTok{]),}
        \FunctionTok{sprintf}\NormalTok{(}\StringTok{"\%.1f {-} \%.1f"}\NormalTok{, quantiles[}\DecValTok{2}\NormalTok{], quantiles[}\DecValTok{3}\NormalTok{]),}
        \FunctionTok{sprintf}\NormalTok{(}\StringTok{"\%.1f {-} \%.1f"}\NormalTok{, quantiles[}\DecValTok{3}\NormalTok{], quantiles[}\DecValTok{4}\NormalTok{]),}
        \FunctionTok{sprintf}\NormalTok{(}\StringTok{"\%.1f {-} \%.1f"}\NormalTok{, quantiles[}\DecValTok{4}\NormalTok{], quantiles[}\DecValTok{5}\NormalTok{])}
\NormalTok{      )}
\NormalTok{    )}
\NormalTok{  )}

\CommentTok{\# Построение карты}
\FunctionTok{ggplot}\NormalTok{() }\SpecialCharTok{+}
  \CommentTok{\# Базовая карта России}
  \FunctionTok{geom\_sf}\NormalTok{(}\AttributeTok{data =}\NormalTok{ russia, }\AttributeTok{fill =} \StringTok{"lightblue"}\NormalTok{, }\AttributeTok{color =} \StringTok{"gray40"}\NormalTok{) }\SpecialCharTok{+} 
  \CommentTok{\# Ограничение области отображения}
  \FunctionTok{coord\_sf}\NormalTok{(}\AttributeTok{xlim =} \FunctionTok{c}\NormalTok{(xmin, xmax), }\AttributeTok{ylim =} \FunctionTok{c}\NormalTok{(ymin, ymax)) }\SpecialCharTok{+}
  \CommentTok{\# Точки наблюдений с категориальным размером}
  \FunctionTok{geom\_point}\NormalTok{(}
    \AttributeTok{data =}\NormalTok{ nonzero\_data,}
    \FunctionTok{aes}\NormalTok{(}\AttributeTok{x =}\NormalTok{ X, }\AttributeTok{y =}\NormalTok{ Y, }\AttributeTok{size =}\NormalTok{ PROM\_cat, }\AttributeTok{color =}\NormalTok{ PROM),}
    \AttributeTok{alpha =} \FloatTok{0.7}
\NormalTok{  ) }\SpecialCharTok{+}
  \CommentTok{\# Точки для нулевых уловов (крестики)}
  \FunctionTok{geom\_point}\NormalTok{(}
    \AttributeTok{data =} \FunctionTok{filter}\NormalTok{(DATA, PROM }\SpecialCharTok{==} \DecValTok{0}\NormalTok{),}
    \FunctionTok{aes}\NormalTok{(}\AttributeTok{x =}\NormalTok{ X, }\AttributeTok{y =}\NormalTok{ Y),}
    \AttributeTok{shape =} \DecValTok{4}\NormalTok{, }\AttributeTok{size =} \FloatTok{1.2}\NormalTok{, }\AttributeTok{stroke =} \DecValTok{1}\NormalTok{, }\AttributeTok{color =} \StringTok{"black"}
\NormalTok{  ) }\SpecialCharTok{+}
  \CommentTok{\# Цветовая шкала (непрерывная)}
  \FunctionTok{scale\_color\_viridis\_c}\NormalTok{(}\AttributeTok{option =} \StringTok{"H"}\NormalTok{, }\AttributeTok{name =} \ConstantTok{NULL}\NormalTok{) }\SpecialCharTok{+}
  \CommentTok{\# Ручная настройка размеров для категорий}
  \FunctionTok{scale\_size\_manual}\NormalTok{(}
    \AttributeTok{name =} \StringTok{"Улов (экз./ч)"}\NormalTok{,}
    \AttributeTok{values =} \FunctionTok{c}\NormalTok{(}\DecValTok{2}\NormalTok{, }\DecValTok{4}\NormalTok{, }\DecValTok{6}\NormalTok{, }\DecValTok{8}\NormalTok{),  }\CommentTok{\# Размеры точек для категорий}
    \AttributeTok{drop =} \ConstantTok{FALSE}
\NormalTok{  ) }\SpecialCharTok{+}
  \CommentTok{\# Настройки темы}
  \FunctionTok{theme\_bw}\NormalTok{() }\SpecialCharTok{+}
  \FunctionTok{labs}\NormalTok{(}
    \AttributeTok{title =} \StringTok{"Распределение уловов краба (2023)"}\NormalTok{,}
    \AttributeTok{subtitle =} \StringTok{"Черные крестики {-} нулевые уловы"}\NormalTok{,}
    \AttributeTok{x =} \StringTok{"Долгота"}\NormalTok{, }
    \AttributeTok{y =} \StringTok{"Широта"}
\NormalTok{  ) }\SpecialCharTok{+}
  \FunctionTok{theme}\NormalTok{(}
    \AttributeTok{panel.grid =} \FunctionTok{element\_line}\NormalTok{(}\AttributeTok{color =} \StringTok{"gray90"}\NormalTok{),}
    \AttributeTok{legend.position =} \StringTok{"bottom"}
\NormalTok{  )}
\end{Highlighting}
\end{Shaded}

\section{Карта распределения уловов по
фасеткам}\label{ux43aux430ux440ux442ux430-ux440ux430ux441ux43fux440ux435ux434ux435ux43bux435ux43dux438ux44f-ux443ux43bux43eux432ux43eux432-ux43fux43e-ux444ux430ux441ux435ux442ux43aux430ux43c}

\begin{figure}[H]

{\centering \includegraphics[width=0.8\linewidth,height=\textheight,keepaspectratio]{images/KARTOGRAPH5.jpg}

}

\caption{Рис. 5.: Карта распределения уловов по фасеткам}

\end{figure}%

\begin{Shaded}
\begin{Highlighting}[]
\CommentTok{\# Очистка окружения и установка рабочей директории}
\FunctionTok{rm}\NormalTok{(}\AttributeTok{list =} \FunctionTok{ls}\NormalTok{())}
\FunctionTok{setwd}\NormalTok{(}\StringTok{"C:/COURSES/KARTOGRAPH/"}\NormalTok{)}

\CommentTok{\# Установка и подключение библиотек (если не установлено — раскомментируй)}
\CommentTok{\# install.packages(c("rnaturalearth", "tidyverse", "sf", "readxl", "viridis"))}
\FunctionTok{library}\NormalTok{(rnaturalearth)}
\FunctionTok{library}\NormalTok{(tidyverse)}
\FunctionTok{library}\NormalTok{(sf)}
\FunctionTok{library}\NormalTok{(readxl)}
\FunctionTok{library}\NormalTok{(viridis)}

\DocumentationTok{\#\#\#\#\#\#\# ЗАГРУЗКА ДАННЫХ И ПОДГОТОВКА ПРОСТРАНСТВЕННЫХ ОБЪЕКТОВ \#\#\#\#\#\#\#\#\#\#\#\#\#\#\#\#}

\CommentTok{\# Чтение и фильтрация данных (убираем фильтр по году, чтобы работать со всеми годами)}
\NormalTok{DATA }\OtherTok{\textless{}{-}}\NormalTok{ readxl}\SpecialCharTok{::}\FunctionTok{read\_excel}\NormalTok{(}\StringTok{"KARTOGRAPHIC.xlsx"}\NormalTok{, }\AttributeTok{sheet =} \StringTok{"SURVEY"}\NormalTok{) }\SpecialCharTok{\%\textgreater{}\%} 
  \FunctionTok{filter}\NormalTok{(SURV }\SpecialCharTok{==} \StringTok{"CRAB"}\NormalTok{)}

\CommentTok{\# Получение границ России}
\NormalTok{russia }\OtherTok{\textless{}{-}} \FunctionTok{ne\_countries}\NormalTok{(}\AttributeTok{scale =} \DecValTok{10}\NormalTok{, }\AttributeTok{country =} \StringTok{"Russia"}\NormalTok{) }\SpecialCharTok{\%\textgreater{}\%} 
  \FunctionTok{st\_as\_sf}\NormalTok{()}

\CommentTok{\# Установка границ отображаемой области}
\NormalTok{xmin }\OtherTok{\textless{}{-}} \DecValTok{37}\NormalTok{; xmax }\OtherTok{\textless{}{-}} \DecValTok{49}
\NormalTok{ymin }\OtherTok{\textless{}{-}} \FloatTok{68.5}\NormalTok{; ymax }\OtherTok{\textless{}{-}} \FloatTok{70.5}

\CommentTok{\# Вычисляем общие квартили для всех лет (чтобы категории были сопоставимыми)}
\NormalTok{quantiles }\OtherTok{\textless{}{-}} \FunctionTok{quantile}\NormalTok{(DATA}\SpecialCharTok{$}\NormalTok{PROM[DATA}\SpecialCharTok{$}\NormalTok{PROM }\SpecialCharTok{\textgreater{}} \DecValTok{0}\NormalTok{], }\AttributeTok{probs =} \FunctionTok{seq}\NormalTok{(}\DecValTok{0}\NormalTok{, }\DecValTok{1}\NormalTok{, }\FloatTok{0.25}\NormalTok{))}

\CommentTok{\# Создаем данные с ненулевыми уловами и категориями}
\NormalTok{nonzero\_data }\OtherTok{\textless{}{-}}\NormalTok{ DATA }\SpecialCharTok{\%\textgreater{}\%}
  \FunctionTok{filter}\NormalTok{(PROM }\SpecialCharTok{\textgreater{}} \DecValTok{0}\NormalTok{) }\SpecialCharTok{\%\textgreater{}\%}
  \FunctionTok{mutate}\NormalTok{(}
\AttributeTok{PROM\_cat =} \FunctionTok{cut}\NormalTok{(}
\NormalTok{  PROM,}
  \AttributeTok{breaks =} \FunctionTok{c}\NormalTok{(}\SpecialCharTok{{-}}\ConstantTok{Inf}\NormalTok{, quantiles[}\DecValTok{2}\SpecialCharTok{:}\DecValTok{4}\NormalTok{], }\ConstantTok{Inf}\NormalTok{),}
  \AttributeTok{include.lowest =} \ConstantTok{TRUE}\NormalTok{,}
  \AttributeTok{labels =} \FunctionTok{c}\NormalTok{(}
    \FunctionTok{sprintf}\NormalTok{(}\StringTok{"\%d {-} \%d"}\NormalTok{, }\FunctionTok{floor}\NormalTok{(quantiles[}\DecValTok{1}\NormalTok{]), }\FunctionTok{floor}\NormalTok{(quantiles[}\DecValTok{2}\NormalTok{])),}
    \FunctionTok{sprintf}\NormalTok{(}\StringTok{"\%d {-} \%d"}\NormalTok{, }\FunctionTok{floor}\NormalTok{(quantiles[}\DecValTok{2}\NormalTok{]), }\FunctionTok{floor}\NormalTok{(quantiles[}\DecValTok{3}\NormalTok{])),}
    \FunctionTok{sprintf}\NormalTok{(}\StringTok{"\%d {-} \%d"}\NormalTok{, }\FunctionTok{floor}\NormalTok{(quantiles[}\DecValTok{3}\NormalTok{]), }\FunctionTok{floor}\NormalTok{(quantiles[}\DecValTok{4}\NormalTok{])),}
    \FunctionTok{sprintf}\NormalTok{(}\StringTok{"\%d {-} \%d"}\NormalTok{, }\FunctionTok{floor}\NormalTok{(quantiles[}\DecValTok{4}\NormalTok{]), }\FunctionTok{floor}\NormalTok{(}\FunctionTok{max}\NormalTok{(DATA}\SpecialCharTok{$}\NormalTok{PROM)))}
\NormalTok{  )}
\NormalTok{)}
\NormalTok{  )}

\CommentTok{\# Отдельно выделяем точки с нулевым уловом}
\NormalTok{zero\_data }\OtherTok{\textless{}{-}}\NormalTok{ DATA }\SpecialCharTok{\%\textgreater{}\%} \FunctionTok{filter}\NormalTok{(PROM }\SpecialCharTok{==} \DecValTok{0}\NormalTok{)}

\DocumentationTok{\#\#\#\#\#\#\# ВИЗУАЛИЗАЦИЯ \#\#\#\#\#\#\#\#\#\#\#\#\#\#\#\#}

\CommentTok{\# Фасеточная карта по годам}
\FunctionTok{ggplot}\NormalTok{() }\SpecialCharTok{+}
  \CommentTok{\# Граница России}
  \FunctionTok{geom\_sf}\NormalTok{(}\AttributeTok{data =}\NormalTok{ russia, }\AttributeTok{fill =} \StringTok{"lightblue"}\NormalTok{, }\AttributeTok{color =} \StringTok{"gray40"}\NormalTok{) }\SpecialCharTok{+}
  
  \CommentTok{\# Ограничение области отображения}
  \FunctionTok{coord\_sf}\NormalTok{(}\AttributeTok{xlim =} \FunctionTok{c}\NormalTok{(xmin, xmax), }\AttributeTok{ylim =} \FunctionTok{c}\NormalTok{(ymin, ymax)) }\SpecialCharTok{+}
  
  \CommentTok{\# Точки с уловом}
  \FunctionTok{geom\_point}\NormalTok{(}
    \AttributeTok{data =}\NormalTok{ nonzero\_data,}
    \FunctionTok{aes}\NormalTok{(}\AttributeTok{x =}\NormalTok{ X, }\AttributeTok{y =}\NormalTok{ Y, }\AttributeTok{size =}\NormalTok{ PROM\_cat, }\AttributeTok{color =}\NormalTok{ PROM),}
    \AttributeTok{alpha =} \FloatTok{0.7}
\NormalTok{  ) }\SpecialCharTok{+}
  
  \CommentTok{\# Нулевые уловы — крестики}
  \FunctionTok{geom\_point}\NormalTok{(}
    \AttributeTok{data =}\NormalTok{ zero\_data,}
    \FunctionTok{aes}\NormalTok{(}\AttributeTok{x =}\NormalTok{ X, }\AttributeTok{y =}\NormalTok{ Y),}
    \AttributeTok{shape =} \DecValTok{4}\NormalTok{, }\AttributeTok{size =} \FloatTok{1.2}\NormalTok{, }\AttributeTok{stroke =} \DecValTok{1}\NormalTok{, }\AttributeTok{color =} \StringTok{"black"}
\NormalTok{  ) }\SpecialCharTok{+}
  
  \CommentTok{\# Цветовая шкала}
  \FunctionTok{scale\_color\_viridis\_c}\NormalTok{(}\AttributeTok{option =} \StringTok{"H"}\NormalTok{, }\AttributeTok{name =} \ConstantTok{NULL}\NormalTok{) }\SpecialCharTok{+}
  
  \CommentTok{\# Настройка размеров точек по категориям}
  \FunctionTok{scale\_size\_manual}\NormalTok{(}
    \AttributeTok{name =} \StringTok{"Улов (экз./ч)"}\NormalTok{,}
    \AttributeTok{values =} \FunctionTok{c}\NormalTok{(}\DecValTok{1}\NormalTok{, }\DecValTok{2}\NormalTok{,}\DecValTok{4}\NormalTok{, }\DecValTok{6}\NormalTok{),}
    \AttributeTok{drop =} \ConstantTok{FALSE}
\NormalTok{  ) }\SpecialCharTok{+}
  
  \CommentTok{\# Фасет по годам}
  \FunctionTok{facet\_wrap}\NormalTok{(}\SpecialCharTok{\textasciitilde{}}\NormalTok{ YEAR, }\AttributeTok{ncol =} \DecValTok{2}\NormalTok{, }\AttributeTok{labeller =}\NormalTok{ label\_value) }\SpecialCharTok{+}
  
  \CommentTok{\# Тема и заголовок}
  \FunctionTok{theme\_bw}\NormalTok{() }\SpecialCharTok{+}
  \FunctionTok{labs}\NormalTok{(}
    \AttributeTok{title =} \StringTok{"Распределение уловов краба по годам"}\NormalTok{,}
    \AttributeTok{subtitle =} \ConstantTok{NULL}\NormalTok{,}
    \AttributeTok{x =} \StringTok{"Долгота"}\NormalTok{, }
    \AttributeTok{y =} \StringTok{"Широта"}
\NormalTok{  ) }\SpecialCharTok{+}
  \FunctionTok{theme}\NormalTok{(}
    \AttributeTok{panel.grid =} \FunctionTok{element\_line}\NormalTok{(}\AttributeTok{color =} \StringTok{"gray90"}\NormalTok{),}
    \AttributeTok{legend.position =} \StringTok{"bottom"}
\NormalTok{  )}
\end{Highlighting}
\end{Shaded}

\section{Карта распределения уловов с автокорреляцией
LISA}\label{ux43aux430ux440ux442ux430-ux440ux430ux441ux43fux440ux435ux434ux435ux43bux435ux43dux438ux44f-ux443ux43bux43eux432ux43eux432-ux441-ux430ux432ux442ux43eux43aux43eux440ux440ux435ux43bux44fux446ux438ux435ux439-lisa}

Алгоритм LISA (Local Indicators of Spatial Association) представляет
собой инструмент выявления пространственных закономерностей на уровне
отдельных объектов. В отличие от глобальных показателей, которые дают
обобщенную оценку автокорреляции для всего региона, LISA позволяет
идентифицировать конкретные кластеры и аномалии, определяя, какие именно
участки вносят основной вклад в пространственную структуру данных. В
контексте анализа промысловых данных краба за 2023 год, этот метод
позволяет выявить зоны концентрации уловов и территории с аномальными
показателями.

Суть кластеризации по методу LISA заключается в сравнении значения
каждого конкретного объекта (точки съемки) со значениями его соседей.
Алгоритм последовательно выполняет несколько ключевых шагов: сначала
создается матрица пространственных весов, где для каждой точки
определяются k ближайших соседей (в данном случае k=4). Затем для каждой
точки рассчитывается локальный индекс Морана (Ii), который количественно
оценивает степень сходства между значением в точке и ее окружением.
Статистическая значимость кластеризации проверяется через p-значение,
полученное методом Монте-Карло.

Биологическая интерпретация выявленных кластеров основана на их
классификации:

\begin{itemize}
\item
  \textbf{High-High} (красные точки): зоны высокой концентрации уловов,
  окруженные такими же продуктивными участками --- потенциальные
  ``горячие точки'' скопления краба
\item
  \textbf{Low-Low} (синие точки): территории с устойчиво низкими
  уловами, окруженные аналогичными участками --- возможные акватории с
  неблагоприятными условиями
\item
  \textbf{High-Low} (розовые точки): аномалии высоких уловов на фоне
  низкопродуктивного окружения --- требуют проверки на ошибки данных или
  изучения уникальных локальных факторов
\item
  \textbf{Low-High} (голубые точки): участки неожиданно низких уловов в
  окружении продуктивных зон --- возможные признаки перелова или
  деградации среды
\end{itemize}

Визуализация результатов (рис. 6) сочетает картографическую основу с
семантикой цвета и размера: размер точки пропорционален величине улова
(PROM), а цвет отражает тип кластера. Серые точки обозначают территории
без статистически значимой автокорреляции. Ограничение области
исследования координатами 37-49° в.д. и 68.5-70.5° с.ш. фокусирует
анализ на ключевом промысловом районе, а преобразование в проекцию UTM
(32638) обеспечивает точность расчетов расстояний.

Практическая ценность анализа заключается в возможности целевого
управления промыслом: выявленные кластеры High-High могут стать
объектами особого мониторинга для предотвращения перелова, тогда как
зоны Low-Low требуют изучения причин низкой продуктивности (например,
исследования донных сообществ или океанографических условий). Аномальные
точки (High-Low/Low-High) служат индикаторами для выборочного контроля
достоверности данных. Такой подход переводит сырые данные съемки в
пространственно-стратифицированную основу для принятия управленческих
решений, позволяя оптимизировать промысловое усилие и минимизировать
воздействие на уязвимые участки донных экосистем.

\begin{figure}[H]

{\centering \includegraphics[width=0.8\linewidth,height=\textheight,keepaspectratio]{images/KARTOGRAPH6.jpg}

}

\caption{Рис. 6.: Карта распределения уловов с автокорреляцией LISA}

\end{figure}%

\begin{Shaded}
\begin{Highlighting}[]
\CommentTok{\# Очистка окружения и установка рабочей директории}
\FunctionTok{rm}\NormalTok{(}\AttributeTok{list =} \FunctionTok{ls}\NormalTok{())}
\FunctionTok{setwd}\NormalTok{(}\StringTok{"C:/COURSES/KARTOGRAPH/"}\NormalTok{)}

\CommentTok{\# Загрузка библиотек}
\FunctionTok{library}\NormalTok{(rnaturalearth)}
\FunctionTok{library}\NormalTok{(tidyverse)}
\FunctionTok{library}\NormalTok{(sf)}
\FunctionTok{library}\NormalTok{(spdep)}
\FunctionTok{library}\NormalTok{(ggspatial)}
\FunctionTok{library}\NormalTok{(readxl)}

\CommentTok{\# 1. ЗАГРУЗКА И ПОДГОТОВКА ДАННЫХ}
\NormalTok{DATA }\OtherTok{\textless{}{-}} \FunctionTok{read\_excel}\NormalTok{(}\StringTok{"KARTOGRAPHIC.xlsx"}\NormalTok{, }\AttributeTok{sheet =} \StringTok{"SURVEY"}\NormalTok{) }\SpecialCharTok{\%\textgreater{}\%} 
  \FunctionTok{filter}\NormalTok{(YEAR }\SpecialCharTok{==} \DecValTok{2023}\NormalTok{, SURV }\SpecialCharTok{==} \StringTok{"CRAB"}\NormalTok{)}

\CommentTok{\# Проверка названий колонок}
\FunctionTok{print}\NormalTok{(}\FunctionTok{names}\NormalTok{(DATA))  }\CommentTok{\# Убедитесь, что координаты названы правильно}

\CommentTok{\# Преобразование в пространственные данные (замените X/Y на ваши названия)}
\NormalTok{points\_sf }\OtherTok{\textless{}{-}} \FunctionTok{st\_as\_sf}\NormalTok{(DATA, }\AttributeTok{coords =} \FunctionTok{c}\NormalTok{(}\StringTok{"X"}\NormalTok{, }\StringTok{"Y"}\NormalTok{), }\AttributeTok{crs =} \DecValTok{4326}\NormalTok{)}

\CommentTok{\# 2. ПОЛУЧЕНИЕ КАРТЫ РОССИИ}
\CommentTok{\# Задаем границы области}
\NormalTok{xmin }\OtherTok{\textless{}{-}} \DecValTok{37}
\NormalTok{xmax }\OtherTok{\textless{}{-}} \DecValTok{49}
\NormalTok{ymin }\OtherTok{\textless{}{-}} \FloatTok{68.5}
\NormalTok{ymax }\OtherTok{\textless{}{-}} \FloatTok{70.5}

\CommentTok{\# Создаём ограничивающий прямоугольник}
\NormalTok{bbox }\OtherTok{\textless{}{-}} \FunctionTok{st\_bbox}\NormalTok{(}\FunctionTok{c}\NormalTok{(}\AttributeTok{xmin =}\NormalTok{ xmin, }\AttributeTok{xmax =}\NormalTok{ xmax, }\AttributeTok{ymin =}\NormalTok{ ymin, }\AttributeTok{ymax =}\NormalTok{ ymax), }\AttributeTok{crs =} \DecValTok{4326}\NormalTok{)}
\NormalTok{bbox\_poly }\OtherTok{\textless{}{-}} \FunctionTok{st\_as\_sfc}\NormalTok{(bbox)}

\CommentTok{\# Карта России}
\NormalTok{russia }\OtherTok{\textless{}{-}} \FunctionTok{ne\_countries}\NormalTok{(}\AttributeTok{country =} \StringTok{"Russia"}\NormalTok{, }\AttributeTok{scale =} \DecValTok{10}\NormalTok{) }\SpecialCharTok{\%\textgreater{}\%} 
  \FunctionTok{st\_as\_sf}\NormalTok{() }\SpecialCharTok{\%\textgreater{}\%} 
  \FunctionTok{st\_crop}\NormalTok{(bbox)  }\CommentTok{\# Обрезка без st\_intersection}

\CommentTok{\# 3. ПОДГОТОВКА ТОЧЕК}
\CommentTok{\# Удаление дубликатов по координатам}
\NormalTok{coords }\OtherTok{\textless{}{-}} \FunctionTok{st\_coordinates}\NormalTok{(points\_sf)}
\NormalTok{points\_sf }\OtherTok{\textless{}{-}}\NormalTok{ points\_sf[}\SpecialCharTok{!}\FunctionTok{duplicated}\NormalTok{(coords), , drop }\OtherTok{=} \ConstantTok{FALSE}\NormalTok{]}

\CommentTok{\# Перевод в UTM}
\NormalTok{points\_utm }\OtherTok{\textless{}{-}} \FunctionTok{st\_transform}\NormalTok{(points\_sf, }\AttributeTok{crs =} \DecValTok{32638}\NormalTok{)}

\CommentTok{\# 4. АНАЛИЗ LISA}
\CommentTok{\# Матрица весов}
\NormalTok{knn }\OtherTok{\textless{}{-}} \FunctionTok{knearneigh}\NormalTok{(points\_utm, }\AttributeTok{k =} \DecValTok{4}\NormalTok{)}
\NormalTok{nb }\OtherTok{\textless{}{-}} \FunctionTok{knn2nb}\NormalTok{(knn)}
\NormalTok{listw }\OtherTok{\textless{}{-}} \FunctionTok{nb2listw}\NormalTok{(nb, }\AttributeTok{style =} \StringTok{"W"}\NormalTok{)}

\CommentTok{\# Локальный Моран}
\NormalTok{local\_moran }\OtherTok{\textless{}{-}} \FunctionTok{localmoran}\NormalTok{(points\_utm}\SpecialCharTok{$}\NormalTok{PROM, listw)}

\CommentTok{\# Добавляем кластеры}
\NormalTok{points\_utm }\OtherTok{\textless{}{-}}\NormalTok{ points\_utm }\SpecialCharTok{\%\textgreater{}\%}
  \FunctionTok{mutate}\NormalTok{(}
    \AttributeTok{Local\_I =}\NormalTok{ local\_moran[, }\StringTok{"Ii"}\NormalTok{],}
    \AttributeTok{P\_value =}\NormalTok{ local\_moran[, }\StringTok{"Pr(z != E(Ii))"}\NormalTok{],}
    \AttributeTok{Mean\_PROM =} \FunctionTok{mean}\NormalTok{(PROM, }\AttributeTok{na.rm =} \ConstantTok{TRUE}\NormalTok{),  }\CommentTok{\# Добавляем среднее значение}
    \AttributeTok{Cluster =} \FunctionTok{case\_when}\NormalTok{(}
\NormalTok{      Local\_I }\SpecialCharTok{\textgreater{}} \DecValTok{0} \SpecialCharTok{\&}\NormalTok{ PROM }\SpecialCharTok{\textgreater{}}\NormalTok{ Mean\_PROM }\SpecialCharTok{\textasciitilde{}} \StringTok{"High{-}High"}\NormalTok{,}
\NormalTok{      Local\_I }\SpecialCharTok{\textgreater{}} \DecValTok{0} \SpecialCharTok{\&}\NormalTok{ PROM }\SpecialCharTok{\textless{}=}\NormalTok{ Mean\_PROM }\SpecialCharTok{\textasciitilde{}} \StringTok{"Low{-}Low"}\NormalTok{,  }\CommentTok{\# Включаем PROM == 0}
\NormalTok{      Local\_I }\SpecialCharTok{\textless{}} \DecValTok{0} \SpecialCharTok{\&}\NormalTok{ PROM }\SpecialCharTok{\textgreater{}}\NormalTok{ Mean\_PROM }\SpecialCharTok{\textasciitilde{}} \StringTok{"High{-}Low"}\NormalTok{,}
\NormalTok{      Local\_I }\SpecialCharTok{\textless{}} \DecValTok{0} \SpecialCharTok{\&}\NormalTok{ PROM }\SpecialCharTok{\textless{}=}\NormalTok{ Mean\_PROM }\SpecialCharTok{\textasciitilde{}} \StringTok{"Low{-}High"}\NormalTok{,  }\CommentTok{\# PROM == 0 попадает сюда}
      \ConstantTok{TRUE} \SpecialCharTok{\textasciitilde{}} \StringTok{"Not significant"}
\NormalTok{    )}
\NormalTok{  )}

\CommentTok{\# Обратно в WGS84}
\NormalTok{points\_result }\OtherTok{\textless{}{-}} \FunctionTok{st\_transform}\NormalTok{(points\_utm, }\AttributeTok{crs =} \DecValTok{4326}\NormalTok{)}

\CommentTok{\# 5. ВИЗУАЛИЗАЦИЯ}
\NormalTok{cluster\_colors }\OtherTok{\textless{}{-}} \FunctionTok{c}\NormalTok{(}
  \StringTok{"High{-}High"} \OtherTok{=} \StringTok{"red"}\NormalTok{,}
  \StringTok{"Low{-}Low"} \OtherTok{=} \StringTok{"blue"}\NormalTok{,}
  \StringTok{"High{-}Low"} \OtherTok{=} \StringTok{"pink"}\NormalTok{,}
  \StringTok{"Low{-}High"} \OtherTok{=} \StringTok{"lightblue"}\NormalTok{,}
  \StringTok{"Not significant"} \OtherTok{=} \StringTok{"gray"}
\NormalTok{)}

\FunctionTok{ggplot}\NormalTok{() }\SpecialCharTok{+}
  \CommentTok{\# Карта России}
  \FunctionTok{geom\_sf}\NormalTok{(}\AttributeTok{data =}\NormalTok{ russia, }\AttributeTok{fill =} \StringTok{"lightblue"}\NormalTok{, }\AttributeTok{color =} \StringTok{"black"}\NormalTok{) }\SpecialCharTok{+}
  
  \CommentTok{\# Все точки (включая PROM == 0) — в одном слое}
  \FunctionTok{geom\_sf}\NormalTok{(}
    \AttributeTok{data =}\NormalTok{ points\_result,}
    \FunctionTok{aes}\NormalTok{(}\AttributeTok{color =}\NormalTok{ Cluster, }\AttributeTok{size =}\NormalTok{ PROM),}
    \AttributeTok{alpha =} \FloatTok{0.8}
\NormalTok{  ) }\SpecialCharTok{+}
  
  \CommentTok{\# Настройки координат и масштаба}
  \FunctionTok{coord\_sf}\NormalTok{(}\AttributeTok{xlim =} \FunctionTok{c}\NormalTok{(xmin, xmax), }\AttributeTok{ylim =} \FunctionTok{c}\NormalTok{(ymin, ymax), }\AttributeTok{expand =} \ConstantTok{FALSE}\NormalTok{) }\SpecialCharTok{+}
  \FunctionTok{annotation\_scale}\NormalTok{(}\AttributeTok{location =} \StringTok{"tl"}\NormalTok{, }\AttributeTok{width\_hint =} \FloatTok{0.3}\NormalTok{) }\SpecialCharTok{+}
  
  \CommentTok{\# Цвет и размер}
  \FunctionTok{scale\_color\_manual}\NormalTok{(}\AttributeTok{values =}\NormalTok{ cluster\_colors) }\SpecialCharTok{+}
  \FunctionTok{scale\_size\_continuous}\NormalTok{(}\AttributeTok{range =} \FunctionTok{c}\NormalTok{(}\DecValTok{1}\NormalTok{, }\DecValTok{8}\NormalTok{), }\AttributeTok{name =} \StringTok{"Величина улова"}\NormalTok{) }\SpecialCharTok{+}
  
  \CommentTok{\# Заголовки и тема}
  \FunctionTok{labs}\NormalTok{(}
    \AttributeTok{title =} \StringTok{"Пространственная автокорреляция уловов краба (LISA)"}\NormalTok{,}
    \AttributeTok{subtitle =} \StringTok{"2023 год, тип съемки: CRAB"}\NormalTok{,}
    \AttributeTok{color =} \StringTok{"Тип кластера"}
\NormalTok{  ) }\SpecialCharTok{+}
  
\FunctionTok{theme\_minimal}\NormalTok{() }\SpecialCharTok{+}
  \FunctionTok{theme}\NormalTok{(}
    \AttributeTok{plot.title =} \FunctionTok{element\_text}\NormalTok{(}\AttributeTok{hjust =} \FloatTok{0.5}\NormalTok{, }\AttributeTok{face =} \StringTok{"bold"}\NormalTok{),}
    \AttributeTok{plot.subtitle =} \FunctionTok{element\_text}\NormalTok{(}\AttributeTok{hjust =} \FloatTok{0.5}\NormalTok{),}
    \AttributeTok{legend.position =} \StringTok{"right"}\NormalTok{,}
    \AttributeTok{panel.border =} \FunctionTok{element\_rect}\NormalTok{(}\AttributeTok{colour =} \StringTok{"black"}\NormalTok{, }\AttributeTok{size =} \DecValTok{1}\NormalTok{, }\AttributeTok{fill =} \ConstantTok{NA}\NormalTok{)  }\CommentTok{\# Рамка вокруг карты}
\NormalTok{  )}
\end{Highlighting}
\end{Shaded}

\section{Карта распределения уловов с автокорреляцией LISA по
фасеткам}\label{ux43aux430ux440ux442ux430-ux440ux430ux441ux43fux440ux435ux434ux435ux43bux435ux43dux438ux44f-ux443ux43bux43eux432ux43eux432-ux441-ux430ux432ux442ux43eux43aux43eux440ux440ux435ux43bux44fux446ux438ux435ux439-lisa-ux43fux43e-ux444ux430ux441ux435ux442ux43aux430ux43c}

\begin{figure}[H]

{\centering \includegraphics[width=0.8\linewidth,height=\textheight,keepaspectratio]{images/KARTOGRAPH7.jpg}

}

\caption{Рис. 7.: Карта распределения уловов с автокорреляцией LISA по
фасеткам}

\end{figure}%

\begin{Shaded}
\begin{Highlighting}[]
\CommentTok{\# Очистка памяти и установка рабочей папки}
\FunctionTok{rm}\NormalTok{(}\AttributeTok{list =} \FunctionTok{ls}\NormalTok{())}
\FunctionTok{setwd}\NormalTok{(}\StringTok{"C:/COURSES/KARTOGRAPH/"}\NormalTok{)}

\CommentTok{\# Загрузка необходимых пакетов}
\FunctionTok{library}\NormalTok{(rnaturalearth)  }\CommentTok{\# Географические карты}
\FunctionTok{library}\NormalTok{(tidyverse)      }\CommentTok{\# Обработка данных и визуализация}
\FunctionTok{library}\NormalTok{(sf)             }\CommentTok{\# Пространственные данные}
\FunctionTok{library}\NormalTok{(spdep)          }\CommentTok{\# Пространственная статистика}
\FunctionTok{library}\NormalTok{(ggspatial)      }\CommentTok{\# Дополнения для карт в ggplot}
\FunctionTok{library}\NormalTok{(readxl)         }\CommentTok{\# Чтение Excel{-}файлов}

\CommentTok{\# 1. ЗАГРУЗКА И ПРЕОБРАЗОВАНИЕ ДАННЫХ}
\CommentTok{\# {-} Чтение данных из Excel}
\CommentTok{\# {-} Фильтрация только данных по крабу}
\NormalTok{DATA }\OtherTok{\textless{}{-}} \FunctionTok{read\_excel}\NormalTok{(}\StringTok{"KARTOGRAPHIC.xlsx"}\NormalTok{, }\AttributeTok{sheet =} \StringTok{"SURVEY"}\NormalTok{) }\SpecialCharTok{\%\textgreater{}\%} 
  \FunctionTok{filter}\NormalTok{(SURV }\SpecialCharTok{==} \StringTok{"CRAB"}\NormalTok{)}

\CommentTok{\# Преобразование в пространственный объект с координатами}
\NormalTok{points\_sf }\OtherTok{\textless{}{-}} \FunctionTok{st\_as\_sf}\NormalTok{(DATA, }\AttributeTok{coords =} \FunctionTok{c}\NormalTok{(}\StringTok{"X"}\NormalTok{, }\StringTok{"Y"}\NormalTok{), }\AttributeTok{crs =} \DecValTok{4326}\NormalTok{)}

\CommentTok{\# 2. ПОДГОТОВКА КАРТОГРАФИЧЕСКОЙ ОСНОВЫ}
\CommentTok{\# {-} Определение границ области исследования}
\NormalTok{xmin }\OtherTok{\textless{}{-}} \DecValTok{37}\NormalTok{; xmax }\OtherTok{\textless{}{-}} \DecValTok{49}\NormalTok{; ymin }\OtherTok{\textless{}{-}} \FloatTok{68.5}\NormalTok{; ymax }\OtherTok{\textless{}{-}} \FloatTok{70.7}

\CommentTok{\# {-} Создание ограничивающего прямоугольника}
\NormalTok{bbox }\OtherTok{\textless{}{-}} \FunctionTok{st\_bbox}\NormalTok{(}\FunctionTok{c}\NormalTok{(}\AttributeTok{xmin =}\NormalTok{ xmin, }\AttributeTok{xmax =}\NormalTok{ xmax, }\AttributeTok{ymin =}\NormalTok{ ymin, }\AttributeTok{ymax =}\NormalTok{ ymax), }\AttributeTok{crs =} \DecValTok{4326}\NormalTok{)}

\CommentTok{\# {-} Загрузка и обрезка карты России по заданным границам}
\NormalTok{russia }\OtherTok{\textless{}{-}} \FunctionTok{ne\_countries}\NormalTok{(}\AttributeTok{country =} \StringTok{"Russia"}\NormalTok{, }\AttributeTok{scale =} \DecValTok{10}\NormalTok{) }\SpecialCharTok{\%\textgreater{}\%} 
  \FunctionTok{st\_as\_sf}\NormalTok{() }\SpecialCharTok{\%\textgreater{}\%} 
  \FunctionTok{st\_crop}\NormalTok{(bbox)}

\CommentTok{\# 3. ФУНКЦИЯ ДЛЯ ПРОСТРАНСТВЕННОГО АНАЛИЗА ПО ГОДАМ}
\NormalTok{analyze\_year }\OtherTok{\textless{}{-}} \ControlFlowTok{function}\NormalTok{(data\_year) \{}
  \CommentTok{\# Удаление дубликатов координат}
\NormalTok{  coords }\OtherTok{\textless{}{-}} \FunctionTok{st\_coordinates}\NormalTok{(data\_year)}
\NormalTok{  data\_year }\OtherTok{\textless{}{-}}\NormalTok{ data\_year[}\SpecialCharTok{!}\FunctionTok{duplicated}\NormalTok{(coords), , drop }\OtherTok{=} \ConstantTok{FALSE}\NormalTok{]}
  
  \CommentTok{\# Перепроецирование в UTM для точных расчетов}
\NormalTok{  points\_utm }\OtherTok{\textless{}{-}} \FunctionTok{st\_transform}\NormalTok{(data\_year, }\AttributeTok{crs =} \DecValTok{32638}\NormalTok{)}
  
  \CommentTok{\# Построение матрицы пространственных весов (4 ближайших соседа)}
\NormalTok{  knn }\OtherTok{\textless{}{-}} \FunctionTok{knearneigh}\NormalTok{(points\_utm, }\AttributeTok{k =} \DecValTok{4}\NormalTok{)}
\NormalTok{  nb }\OtherTok{\textless{}{-}} \FunctionTok{knn2nb}\NormalTok{(knn)}
\NormalTok{  listw }\OtherTok{\textless{}{-}} \FunctionTok{nb2listw}\NormalTok{(nb, }\AttributeTok{style =} \StringTok{"W"}\NormalTok{)  }\CommentTok{\# Стандартизованная матрица}
  
  \CommentTok{\# Расчет локальной пространственной автокорреляции (LISA)}
\NormalTok{  local\_moran }\OtherTok{\textless{}{-}} \FunctionTok{localmoran}\NormalTok{(points\_utm}\SpecialCharTok{$}\NormalTok{PROM, listw)}
  
  \CommentTok{\# Классификация кластеров на основе результатов}
\NormalTok{  points\_utm }\OtherTok{\textless{}{-}}\NormalTok{ points\_utm }\SpecialCharTok{\%\textgreater{}\%}
    \FunctionTok{mutate}\NormalTok{(}
      \AttributeTok{Local\_I =}\NormalTok{ local\_moran[, }\StringTok{"Ii"}\NormalTok{],}
      \AttributeTok{P\_value =}\NormalTok{ local\_moran[, }\StringTok{"Pr(z != E(Ii))"}\NormalTok{],}
      \AttributeTok{Mean\_PROM =} \FunctionTok{mean}\NormalTok{(PROM, }\AttributeTok{na.rm =} \ConstantTok{TRUE}\NormalTok{),}
      \AttributeTok{Cluster =} \FunctionTok{case\_when}\NormalTok{(}
\NormalTok{        Local\_I }\SpecialCharTok{\textgreater{}} \DecValTok{0} \SpecialCharTok{\&}\NormalTok{ PROM }\SpecialCharTok{\textgreater{}}\NormalTok{ Mean\_PROM }\SpecialCharTok{\textasciitilde{}} \StringTok{"High{-}High"}\NormalTok{,     }\CommentTok{\# Горячая точка}
\NormalTok{        Local\_I }\SpecialCharTok{\textgreater{}} \DecValTok{0} \SpecialCharTok{\&}\NormalTok{ PROM }\SpecialCharTok{\textless{}=}\NormalTok{ Mean\_PROM }\SpecialCharTok{\textasciitilde{}} \StringTok{"Low{-}Low"}\NormalTok{,      }\CommentTok{\# Холодная точка}
\NormalTok{        Local\_I }\SpecialCharTok{\textless{}} \DecValTok{0} \SpecialCharTok{\&}\NormalTok{ PROM }\SpecialCharTok{\textgreater{}}\NormalTok{ Mean\_PROM }\SpecialCharTok{\textasciitilde{}} \StringTok{"High{-}Low"}\NormalTok{,      }\CommentTok{\# Выброс (высокий среди низких)}
\NormalTok{        Local\_I }\SpecialCharTok{\textless{}} \DecValTok{0} \SpecialCharTok{\&}\NormalTok{ PROM }\SpecialCharTok{\textless{}=}\NormalTok{ Mean\_PROM }\SpecialCharTok{\textasciitilde{}} \StringTok{"Low{-}High"}\NormalTok{,     }\CommentTok{\# Выброс (низкий среди высоких)}
        \ConstantTok{TRUE} \SpecialCharTok{\textasciitilde{}} \StringTok{"Not significant"}                          \CommentTok{\# Незначимые}
\NormalTok{      )}
\NormalTok{    )}
  
  \CommentTok{\# Возврат в географические координаты}
  \FunctionTok{st\_transform}\NormalTok{(points\_utm, }\AttributeTok{crs =} \DecValTok{4326}\NormalTok{)}
\NormalTok{\}}

\CommentTok{\# 4. ОБРАБОТКА ДАННЫХ ПО ГОДАМ}
\CommentTok{\# {-} Разделение данных по годам}
\CommentTok{\# {-} Применение анализа для каждого года}
\CommentTok{\# {-} Объединение результатов}
\NormalTok{results\_list }\OtherTok{\textless{}{-}}\NormalTok{ DATA }\SpecialCharTok{\%\textgreater{}\%}
  \FunctionTok{group\_split}\NormalTok{(YEAR) }\SpecialCharTok{\%\textgreater{}\%} 
  \FunctionTok{lapply}\NormalTok{(}\ControlFlowTok{function}\NormalTok{(group) \{}
    \FunctionTok{analyze\_year}\NormalTok{(}\FunctionTok{st\_as\_sf}\NormalTok{(group, }\AttributeTok{coords =} \FunctionTok{c}\NormalTok{(}\StringTok{"X"}\NormalTok{, }\StringTok{"Y"}\NormalTok{), }\AttributeTok{crs =} \DecValTok{4326}\NormalTok{))}
\NormalTok{  \}) }\SpecialCharTok{\%\textgreater{}\%}
  \FunctionTok{bind\_rows}\NormalTok{()}

\CommentTok{\# 5. КАТЕГОРИЗАЦИЯ УЛОВОВ}
\CommentTok{\# {-} Расчет квантилей для всего набора данных}
\NormalTok{PROM\_breaks }\OtherTok{\textless{}{-}} \FunctionTok{quantile}\NormalTok{(results\_list}\SpecialCharTok{$}\NormalTok{PROM, }
                         \AttributeTok{probs =} \FunctionTok{c}\NormalTok{(}\DecValTok{0}\NormalTok{, }\FloatTok{0.25}\NormalTok{, }\FloatTok{0.5}\NormalTok{, }\FloatTok{0.75}\NormalTok{, }\DecValTok{1}\NormalTok{), }
                         \AttributeTok{na.rm =} \ConstantTok{TRUE}\NormalTok{) }\SpecialCharTok{\%\textgreater{}\%} 
  \FunctionTok{round}\NormalTok{(}\DecValTok{1}\NormalTok{)  }\CommentTok{\# Округление значений}

\CommentTok{\# {-} Создание меток с реальными диапазонами}
\NormalTok{PROM\_labels }\OtherTok{\textless{}{-}} \FunctionTok{sprintf}\NormalTok{(}\StringTok{"\%.1f {-} \%.1f"}\NormalTok{, PROM\_breaks[}\DecValTok{1}\SpecialCharTok{:}\DecValTok{4}\NormalTok{], PROM\_breaks[}\DecValTok{2}\SpecialCharTok{:}\DecValTok{5}\NormalTok{])}

\CommentTok{\# {-} Добавление категорий уловов в данные}
\NormalTok{results\_list }\OtherTok{\textless{}{-}}\NormalTok{ results\_list }\SpecialCharTok{\%\textgreater{}\%}
  \FunctionTok{mutate}\NormalTok{(}
    \AttributeTok{PROM\_category =} \FunctionTok{cut}\NormalTok{(}
\NormalTok{      PROM, }
      \AttributeTok{breaks =}\NormalTok{ PROM\_breaks, }
      \AttributeTok{labels =}\NormalTok{ PROM\_labels,}
      \AttributeTok{include.lowest =} \ConstantTok{TRUE}
\NormalTok{    )}
\NormalTok{  )}

\CommentTok{\# 6. ВИЗУАЛИЗАЦИЯ РЕЗУЛЬТАТОВ}
\CommentTok{\# Цветовая схема для типов кластеров}
\NormalTok{cluster\_colors }\OtherTok{\textless{}{-}} \FunctionTok{c}\NormalTok{(}
  \StringTok{"High{-}High"} \OtherTok{=} \StringTok{"red"}\NormalTok{,       }\CommentTok{\# Горячие точки}
  \StringTok{"Low{-}Low"} \OtherTok{=} \StringTok{"blue"}\NormalTok{,        }\CommentTok{\# Холодные точки}
  \StringTok{"High{-}Low"} \OtherTok{=} \StringTok{"pink"}\NormalTok{,       }\CommentTok{\# Выбросы высокие}
  \StringTok{"Low{-}High"} \OtherTok{=} \StringTok{"lightblue"}\NormalTok{,  }\CommentTok{\# Выбросы низкие}
  \StringTok{"Not significant"} \OtherTok{=} \StringTok{"gray"} \CommentTok{\# Незначимые}
\NormalTok{)}

\CommentTok{\# Построение карты}
\FunctionTok{ggplot}\NormalTok{(}\AttributeTok{data =}\NormalTok{ results\_list) }\SpecialCharTok{+}
  \CommentTok{\# Базовая карта России}
  \FunctionTok{geom\_sf}\NormalTok{(}\AttributeTok{data =}\NormalTok{ russia, }\AttributeTok{fill =} \StringTok{"\#E8E5D6"}\NormalTok{, }\AttributeTok{color =} \StringTok{"black"}\NormalTok{, }\AttributeTok{inherit.aes =} \ConstantTok{FALSE}\NormalTok{) }\SpecialCharTok{+}
  
  \CommentTok{\# Точки наблюдений с цветом по кластерам и размером по уловам}
  \FunctionTok{geom\_sf}\NormalTok{(}\FunctionTok{aes}\NormalTok{(}\AttributeTok{color =}\NormalTok{ Cluster, }\AttributeTok{size =}\NormalTok{ PROM\_category), }\AttributeTok{alpha =} \FloatTok{0.8}\NormalTok{) }\SpecialCharTok{+}
  
  \CommentTok{\# Разделение на панели по годам}
  \FunctionTok{facet\_wrap}\NormalTok{(}\SpecialCharTok{\textasciitilde{}}\NormalTok{ YEAR, }\AttributeTok{ncol =} \DecValTok{2}\NormalTok{) }\SpecialCharTok{+}
  
  \CommentTok{\# Установка границ карты}
  \FunctionTok{coord\_sf}\NormalTok{(}\AttributeTok{xlim =} \FunctionTok{c}\NormalTok{(xmin, xmax), }\AttributeTok{ylim =} \FunctionTok{c}\NormalTok{(ymin, ymax), }\AttributeTok{expand =} \ConstantTok{FALSE}\NormalTok{) }\SpecialCharTok{+}
  
  \CommentTok{\# Настройка легенды для кластеров}
  \FunctionTok{scale\_color\_manual}\NormalTok{(}
    \AttributeTok{values =}\NormalTok{ cluster\_colors,}
    \AttributeTok{name =} \StringTok{"Тип кластера"}\NormalTok{,}
    \AttributeTok{guide =} \FunctionTok{guide\_legend}\NormalTok{(}\AttributeTok{nrow =} \DecValTok{2}\NormalTok{)}
\NormalTok{  ) }\SpecialCharTok{+}
  
  \CommentTok{\# Настройка легенды для уловов (реальные диапазоны)}
  \FunctionTok{scale\_size\_manual}\NormalTok{(}
    \AttributeTok{name =} \StringTok{"Величина улова"}\NormalTok{,}
    \AttributeTok{values =} \FunctionTok{c}\NormalTok{(}\DecValTok{1}\NormalTok{, }\DecValTok{2}\NormalTok{, }\DecValTok{3}\NormalTok{, }\DecValTok{5}\NormalTok{),  }\CommentTok{\# Размеры точек для 4{-}х категорий}
    \AttributeTok{breaks =} \FunctionTok{levels}\NormalTok{(results\_list}\SpecialCharTok{$}\NormalTok{PROM\_category),}
    \AttributeTok{guide =} \FunctionTok{guide\_legend}\NormalTok{(}\AttributeTok{nrow =} \DecValTok{2}\NormalTok{)}
\NormalTok{  ) }\SpecialCharTok{+}
  
  \CommentTok{\# Заголовки и подписи}
  \FunctionTok{labs}\NormalTok{(}
    \AttributeTok{title =} \StringTok{"Пространственная автокорреляция уловов краба (LISA)"}\NormalTok{,}
    \AttributeTok{subtitle =} \StringTok{"Тип съемки: CRAB"}
\NormalTok{  ) }\SpecialCharTok{+}
  
  \CommentTok{\# Оформление графика}
  \FunctionTok{theme\_minimal}\NormalTok{() }\SpecialCharTok{+}
  \FunctionTok{theme}\NormalTok{(}
    \AttributeTok{axis.text.x =} \FunctionTok{element\_text}\NormalTok{(}\AttributeTok{size =} \DecValTok{9}\NormalTok{, }\AttributeTok{margin =} \FunctionTok{margin}\NormalTok{(}\AttributeTok{t =} \DecValTok{5}\NormalTok{)),}
    \AttributeTok{axis.text.y =} \FunctionTok{element\_text}\NormalTok{(}\AttributeTok{size =} \DecValTok{9}\NormalTok{, }\AttributeTok{angle =} \DecValTok{90}\NormalTok{, }\AttributeTok{hjust =} \FloatTok{0.5}\NormalTok{, }\AttributeTok{margin =} \FunctionTok{margin}\NormalTok{(}\AttributeTok{r =} \DecValTok{5}\NormalTok{)),}
    \AttributeTok{panel.background =} \FunctionTok{element\_rect}\NormalTok{(}\AttributeTok{fill =} \StringTok{"\#F0F8FF"}\NormalTok{, }\AttributeTok{color =} \ConstantTok{NA}\NormalTok{),  }\CommentTok{\# Фон океана}
    \AttributeTok{panel.grid.major =} \FunctionTok{element\_line}\NormalTok{(}\AttributeTok{color =} \StringTok{"grey90"}\NormalTok{, }\AttributeTok{linetype =} \StringTok{"dotted"}\NormalTok{),}
    \AttributeTok{legend.position =} \StringTok{"bottom"}\NormalTok{,           }\CommentTok{\# Легенда внизу}
    \AttributeTok{legend.box =} \StringTok{"horizontal"}\NormalTok{,            }\CommentTok{\# Горизонтальное расположение}
    \AttributeTok{panel.border =} \FunctionTok{element\_rect}\NormalTok{(}\AttributeTok{fill =} \ConstantTok{NA}\NormalTok{, }\AttributeTok{color =} \StringTok{"black"}\NormalTok{, }\AttributeTok{size =} \FloatTok{0.7}\NormalTok{),}
    \AttributeTok{strip.background =} \FunctionTok{element\_rect}\NormalTok{(}\AttributeTok{fill =} \StringTok{"white"}\NormalTok{, }\AttributeTok{color =} \StringTok{"black"}\NormalTok{, }\AttributeTok{size =} \FloatTok{0.7}\NormalTok{),  }\CommentTok{\# Заголовки панелей}
    \AttributeTok{strip.text =} \FunctionTok{element\_text}\NormalTok{(}\AttributeTok{size =} \DecValTok{11}\NormalTok{, }\AttributeTok{face =} \StringTok{"bold"}\NormalTok{)}
\NormalTok{  ) }\SpecialCharTok{+}
  
  \CommentTok{\# Разметка осей (долгота с шагом 2°, широта с шагом 1°)}
  \FunctionTok{scale\_x\_continuous}\NormalTok{(}
    \AttributeTok{breaks =} \FunctionTok{seq}\NormalTok{(}\FunctionTok{floor}\NormalTok{(xmin), }\FunctionTok{ceiling}\NormalTok{(xmax), }\AttributeTok{by =} \DecValTok{2}\NormalTok{),}
    \AttributeTok{labels =} \ControlFlowTok{function}\NormalTok{(x) }\FunctionTok{paste0}\NormalTok{(x, }\StringTok{"°E"}\NormalTok{)}
\NormalTok{  ) }\SpecialCharTok{+}
  \FunctionTok{scale\_y\_continuous}\NormalTok{(}
    \AttributeTok{breaks =} \FunctionTok{seq}\NormalTok{(}\FunctionTok{floor}\NormalTok{(ymin), }\FunctionTok{ceiling}\NormalTok{(ymax), }\AttributeTok{by =} \DecValTok{1}\NormalTok{),  }
    \AttributeTok{labels =} \ControlFlowTok{function}\NormalTok{(y) }\FunctionTok{paste0}\NormalTok{(y, }\StringTok{"°N"}\NormalTok{)}
\NormalTok{  )}
\end{Highlighting}
\end{Shaded}

\section{Промысловые карты с квартильным распределением
уловов}\label{ux43fux440ux43eux43cux44bux441ux43bux43eux432ux44bux435-ux43aux430ux440ux442ux44b-ux441-ux43aux432ux430ux440ux442ux438ux43bux44cux43dux44bux43c-ux440ux430ux441ux43fux440ux435ux434ux435ux43bux435ux43dux438ux435ux43c-ux443ux43bux43eux432ux43eux432}

\begin{figure}[H]

{\centering \includegraphics[width=0.8\linewidth,height=\textheight,keepaspectratio]{images/KARTOGRAPH8.jpg}

}

\caption{Рис. 8.: Промысловые карты с квартильным распределением уловов}

\end{figure}%

\begin{Shaded}
\begin{Highlighting}[]
\CommentTok{\# Очистка окружения и установка рабочей директории}
\FunctionTok{rm}\NormalTok{(}\AttributeTok{list =} \FunctionTok{ls}\NormalTok{())}
\FunctionTok{setwd}\NormalTok{(}\StringTok{"C:/COURSES/KARTOGRAPH/"}\NormalTok{)}

\CommentTok{\# Загрузка необходимых библиотек}
\FunctionTok{library}\NormalTok{(rnaturalearth)}
\FunctionTok{library}\NormalTok{(tidyverse)}
\FunctionTok{library}\NormalTok{(sf)}

\DocumentationTok{\#\#\#\#\#\#\# ЗАГРУЗКА ДАННЫХ И ПОДГОТОВКА ПРОСТРАНСТВЕННЫХ ОБЪЕКТОВ \#\#\#\#\#\#\#\#\#\#\#\#\#\#\#\#}

\CommentTok{\# Чтение и фильтрация данных}
\NormalTok{DATA }\OtherTok{\textless{}{-}}\NormalTok{ readxl}\SpecialCharTok{::}\FunctionTok{read\_excel}\NormalTok{(}\StringTok{"KARTOGRAPHIC.xlsx"}\NormalTok{, }\AttributeTok{sheet =} \StringTok{"FISHERY"}\NormalTok{) }\SpecialCharTok{\%\textgreater{}\%} 
  \FunctionTok{filter}\NormalTok{(YEAR }\SpecialCharTok{==} \DecValTok{2023}\NormalTok{)}

\CommentTok{\# Получение границ России}
\NormalTok{russia }\OtherTok{\textless{}{-}} \FunctionTok{ne\_countries}\NormalTok{(}\AttributeTok{scale =} \DecValTok{10}\NormalTok{, }\AttributeTok{country =} \StringTok{"Russia"}\NormalTok{) }\SpecialCharTok{\%\textgreater{}\%} 
  \FunctionTok{st\_as\_sf}\NormalTok{()}

\CommentTok{\# Установка границ отображаемой области}
\NormalTok{xmin}\OtherTok{=}\DecValTok{37}\NormalTok{; xmax}\OtherTok{=}\DecValTok{48}\NormalTok{; ymin}\OtherTok{=}\FloatTok{68.6}\NormalTok{; ymax}\OtherTok{=}\DecValTok{71}

\DocumentationTok{\#\#\#\#\#\#\# ПОДГОТОВКА ДАННЫХ ДЛЯ ВИЗУАЛИЗАЦИИ \#\#\#\#\#\#\#\#\#\#\#\#\#\#\#\#}
\CommentTok{\# Вычисляем квартили отдельно}
\NormalTok{quantiles }\OtherTok{\textless{}{-}} \FunctionTok{quantile}\NormalTok{(DATA}\SpecialCharTok{$}\NormalTok{CPUE[DATA}\SpecialCharTok{$}\NormalTok{CPUE }\SpecialCharTok{\textgreater{}} \DecValTok{0}\NormalTok{], }\AttributeTok{probs =} \FunctionTok{seq}\NormalTok{(}\DecValTok{0}\NormalTok{, }\DecValTok{1}\NormalTok{, }\FloatTok{0.25}\NormalTok{))}

\CommentTok{\# Создаем 4 категории с реальными диапазонами значений}
\NormalTok{nonzero\_data }\OtherTok{\textless{}{-}}\NormalTok{ DATA }\SpecialCharTok{\%\textgreater{}\%} 
  \FunctionTok{filter}\NormalTok{(CPUE }\SpecialCharTok{\textgreater{}} \DecValTok{0}\NormalTok{) }\SpecialCharTok{\%\textgreater{}\%}
  \FunctionTok{mutate}\NormalTok{(}
    \AttributeTok{CPUE\_cat =} \FunctionTok{cut}\NormalTok{(}
\NormalTok{      CPUE,}
      \AttributeTok{breaks =}\NormalTok{ quantiles,}
      \AttributeTok{include.lowest =} \ConstantTok{TRUE}\NormalTok{,}
      \AttributeTok{labels =} \FunctionTok{c}\NormalTok{(}
        \FunctionTok{sprintf}\NormalTok{(}\StringTok{"\%.1f {-} \%.1f"}\NormalTok{, quantiles[}\DecValTok{1}\NormalTok{], quantiles[}\DecValTok{2}\NormalTok{]),}
        \FunctionTok{sprintf}\NormalTok{(}\StringTok{"\%.1f {-} \%.1f"}\NormalTok{, quantiles[}\DecValTok{2}\NormalTok{], quantiles[}\DecValTok{3}\NormalTok{]),}
        \FunctionTok{sprintf}\NormalTok{(}\StringTok{"\%.1f {-} \%.1f"}\NormalTok{, quantiles[}\DecValTok{3}\NormalTok{], quantiles[}\DecValTok{4}\NormalTok{]),}
        \FunctionTok{sprintf}\NormalTok{(}\StringTok{"\%.1f {-} \%.1f"}\NormalTok{, quantiles[}\DecValTok{4}\NormalTok{], quantiles[}\DecValTok{5}\NormalTok{])}
\NormalTok{      )}
\NormalTok{    )}
\NormalTok{  )}

\CommentTok{\# Построение карты}
\FunctionTok{ggplot}\NormalTok{() }\SpecialCharTok{+}
  \CommentTok{\# Базовая карта России}
  \FunctionTok{geom\_sf}\NormalTok{(}\AttributeTok{data =}\NormalTok{ russia, }\AttributeTok{fill =} \StringTok{"lightblue"}\NormalTok{, }\AttributeTok{color =} \StringTok{"gray40"}\NormalTok{) }\SpecialCharTok{+} 
  \CommentTok{\# Ограничение области отображения}
  \FunctionTok{coord\_sf}\NormalTok{(}\AttributeTok{xlim =} \FunctionTok{c}\NormalTok{(xmin, xmax), }\AttributeTok{ylim =} \FunctionTok{c}\NormalTok{(ymin, ymax)) }\SpecialCharTok{+}
  \CommentTok{\# Точки наблюдений с категориальным размером}
  \FunctionTok{geom\_point}\NormalTok{(}
    \AttributeTok{data =}\NormalTok{ nonzero\_data,}
    \FunctionTok{aes}\NormalTok{(}\AttributeTok{x =}\NormalTok{ X, }\AttributeTok{y =}\NormalTok{ Y, }\AttributeTok{size =}\NormalTok{ CPUE\_cat, }\AttributeTok{color =}\NormalTok{ CPUE),}
    \AttributeTok{alpha =} \FloatTok{0.7}
\NormalTok{  ) }\SpecialCharTok{+}
  \CommentTok{\# Точки для нулевых уловов (крестики)}
  \FunctionTok{geom\_point}\NormalTok{(}
    \AttributeTok{data =} \FunctionTok{filter}\NormalTok{(DATA, CPUE }\SpecialCharTok{==} \DecValTok{0}\NormalTok{),}
    \FunctionTok{aes}\NormalTok{(}\AttributeTok{x =}\NormalTok{ X, }\AttributeTok{y =}\NormalTok{ Y),}
    \AttributeTok{shape =} \DecValTok{4}\NormalTok{, }\AttributeTok{size =} \FloatTok{1.2}\NormalTok{, }\AttributeTok{stroke =} \DecValTok{1}\NormalTok{, }\AttributeTok{color =} \StringTok{"black"}
\NormalTok{  ) }\SpecialCharTok{+}
  \CommentTok{\# Цветовая шкала (непрерывная)}
  \FunctionTok{scale\_color\_viridis\_c}\NormalTok{(}\AttributeTok{option =} \StringTok{"H"}\NormalTok{, }\AttributeTok{name =} \ConstantTok{NULL}\NormalTok{) }\SpecialCharTok{+}
  \CommentTok{\# Ручная настройка размеров для категорий}
  \FunctionTok{scale\_size\_manual}\NormalTok{(}
    \AttributeTok{name =} \StringTok{"CPUE"}\NormalTok{,}
    \AttributeTok{values =} \FunctionTok{c}\NormalTok{(}\DecValTok{2}\NormalTok{, }\DecValTok{4}\NormalTok{, }\DecValTok{6}\NormalTok{, }\DecValTok{8}\NormalTok{),  }\CommentTok{\# Размеры точек для категорий}
    \AttributeTok{drop =} \ConstantTok{FALSE}
\NormalTok{  ) }\SpecialCharTok{+}
  \CommentTok{\# Настройки темы}
  \FunctionTok{theme\_bw}\NormalTok{() }\SpecialCharTok{+}
  \FunctionTok{labs}\NormalTok{(}
    \AttributeTok{title =} \StringTok{"Распределение CPUE краба (2023)"}\NormalTok{,}
    \AttributeTok{subtitle =} \ConstantTok{NULL}\NormalTok{,}
    \AttributeTok{x =} \StringTok{"Долгота"}\NormalTok{, }
    \AttributeTok{y =} \StringTok{"Широта"}
\NormalTok{  ) }\SpecialCharTok{+}
  \FunctionTok{theme}\NormalTok{(}
    \AttributeTok{panel.grid =} \FunctionTok{element\_line}\NormalTok{(}\AttributeTok{color =} \StringTok{"gray90"}\NormalTok{),}
    \AttributeTok{legend.position =} \StringTok{"bottom"}
\NormalTok{  )}
\end{Highlighting}
\end{Shaded}

\section{Промысловые карты с агрегацией в центрах полигонов
(промквадратов)}\label{ux43fux440ux43eux43cux44bux441ux43bux43eux432ux44bux435-ux43aux430ux440ux442ux44b-ux441-ux430ux433ux440ux435ux433ux430ux446ux438ux435ux439-ux432-ux446ux435ux43dux442ux440ux430ux445-ux43fux43eux43bux438ux433ux43eux43dux43eux432-ux43fux440ux43eux43cux43aux432ux430ux434ux440ux430ux442ux43eux432}

\begin{figure}[H]

{\centering \includegraphics[width=0.8\linewidth,height=\textheight,keepaspectratio]{images/KARTOGRAPH9.jpg}

}

\caption{Рис. 9.: Промысловые карты с агрегацией в центрах полигонов
(промквадратов)}

\end{figure}%

\begin{Shaded}
\begin{Highlighting}[]
\CommentTok{\# Очистка окружения и установка рабочей директории}
\FunctionTok{rm}\NormalTok{(}\AttributeTok{list =} \FunctionTok{ls}\NormalTok{())}
\FunctionTok{setwd}\NormalTok{(}\StringTok{"C:/COURSES/KARTOGRAPH/"}\NormalTok{)}

\CommentTok{\# Загрузка необходимых библиотек}
\FunctionTok{library}\NormalTok{(rnaturalearth)}
\FunctionTok{library}\NormalTok{(tidyverse)}
\FunctionTok{library}\NormalTok{(sf)}

\DocumentationTok{\#\#\#\#\#\#\# ЗАГРУЗКА ДАННЫХ И ПОДГОТОВКА ПРОСТРАНСТВЕННЫХ ОБЪЕКТОВ \#\#\#\#\#\#\#\#\#\#\#\#\#\#\#\#}

\CommentTok{\# Чтение и фильтрация данных}
\NormalTok{DATA }\OtherTok{\textless{}{-}}\NormalTok{ readxl}\SpecialCharTok{::}\FunctionTok{read\_excel}\NormalTok{(}\StringTok{"KARTOGRAPHIC.xlsx"}\NormalTok{, }\AttributeTok{sheet =} \StringTok{"FISHERY"}\NormalTok{) }\SpecialCharTok{\%\textgreater{}\%} 
  \FunctionTok{filter}\NormalTok{(YEAR }\SpecialCharTok{==} \DecValTok{2023}\NormalTok{)}

\CommentTok{\# Преобразуем CPUE в пространственные точки}
\NormalTok{spec\_points }\OtherTok{\textless{}{-}} \FunctionTok{st\_as\_sf}\NormalTok{(DATA, }\AttributeTok{coords =} \FunctionTok{c}\NormalTok{(}\StringTok{"X"}\NormalTok{, }\StringTok{"Y"}\NormalTok{), }\AttributeTok{crs =} \DecValTok{4326}\NormalTok{)}

\CommentTok{\# Карта России}
\NormalTok{russia }\OtherTok{\textless{}{-}} \FunctionTok{ne\_countries}\NormalTok{(}\AttributeTok{scale =} \DecValTok{10}\NormalTok{, }\AttributeTok{country =} \StringTok{"Russia"}\NormalTok{) }

\CommentTok{\# Параметры карты и сетки}
\NormalTok{xmin }\OtherTok{\textless{}{-}} \DecValTok{32}\NormalTok{; xmax }\OtherTok{\textless{}{-}} \DecValTok{48}\NormalTok{; ymin }\OtherTok{\textless{}{-}} \DecValTok{68}\NormalTok{; ymax }\OtherTok{\textless{}{-}} \DecValTok{72}
\NormalTok{xcs }\OtherTok{\textless{}{-}} \DecValTok{1}\NormalTok{; ycs }\OtherTok{\textless{}{-}} \FloatTok{0.25}


\CommentTok{\# Создание основного датафрейма и пространственных объектов}
\NormalTok{points\_sf }\OtherTok{\textless{}{-}} \FunctionTok{st\_as\_sf}\NormalTok{(DATA, }\AttributeTok{coords =} \FunctionTok{c}\NormalTok{(}\StringTok{"X"}\NormalTok{, }\StringTok{"Y"}\NormalTok{), }\AttributeTok{crs =} \DecValTok{4326}\NormalTok{)}

\CommentTok{\# Создание сетки}
\NormalTok{grid\_sf }\OtherTok{\textless{}{-}} \FunctionTok{st\_make\_grid}\NormalTok{(points\_sf, }
                        \AttributeTok{cellsize =} \FunctionTok{c}\NormalTok{(xcs, ycs),}
                        \AttributeTok{n =} \FunctionTok{c}\NormalTok{(}\DecValTok{2} \SpecialCharTok{+}\NormalTok{ (xmax }\SpecialCharTok{{-}}\NormalTok{ xmin)}\SpecialCharTok{/}\NormalTok{xcs, }\DecValTok{2} \SpecialCharTok{+}\NormalTok{ (ymax }\SpecialCharTok{{-}}\NormalTok{ ymin)}\SpecialCharTok{/}\NormalTok{ycs),}
                        \AttributeTok{offset =} \FunctionTok{c}\NormalTok{(xmin }\SpecialCharTok{{-}}\NormalTok{ xcs, ymin }\SpecialCharTok{{-}}\NormalTok{ ycs)) }\SpecialCharTok{\%\textgreater{}\%} 
  \FunctionTok{st\_sf}\NormalTok{() }\SpecialCharTok{\%\textgreater{}\%} 
  \FunctionTok{mutate}\NormalTok{(}\AttributeTok{cell\_id =} \FunctionTok{row\_number}\NormalTok{())}

\CommentTok{\# Присоединяем точки Catch к сетке и агрегируем по ячейкам и годам}
\NormalTok{shares\_df\_catch }\OtherTok{\textless{}{-}} \FunctionTok{st\_join}\NormalTok{(points\_sf, grid\_sf) }\SpecialCharTok{\%\textgreater{}\%} 
  \FunctionTok{st\_drop\_geometry}\NormalTok{() }\SpecialCharTok{\%\textgreater{}\%} 
  \FunctionTok{group\_by}\NormalTok{(cell\_id, YEAR) }\SpecialCharTok{\%\textgreater{}\%} 
  \FunctionTok{summarise}\NormalTok{(}
    \AttributeTok{Count =} \FunctionTok{n}\NormalTok{(),}
    \AttributeTok{CATCH =} \FunctionTok{mean}\NormalTok{(CPUE, }\AttributeTok{na.rm =} \ConstantTok{TRUE}\NormalTok{)}
\NormalTok{  ) }\SpecialCharTok{\%\textgreater{}\%} 
  \FunctionTok{ungroup}\NormalTok{()}

\CommentTok{\# Присоединяем статистику Catch к сетке}
\NormalTok{gird\_shares\_catch }\OtherTok{\textless{}{-}} \FunctionTok{right\_join}\NormalTok{(grid\_sf, shares\_df\_catch, }\AttributeTok{by =} \StringTok{"cell\_id"}\NormalTok{)}



\CommentTok{\# Центроиды сетки по W}
\NormalTok{CENTROIDS\_W }\OtherTok{\textless{}{-}}\NormalTok{ gird\_shares\_catch }\SpecialCharTok{\%\textgreater{}\%} 
  \FunctionTok{st\_centroid}\NormalTok{()}

\DocumentationTok{\#\#\#\#\#\#\#\#\#\#\#\#\#\#\#\#\#\#\#\# ВИЗУАЛИЗАЦИЯ \#\#\#\#\#\#\#\#\#\#\#\#\#\#\#\#\#\#\#\#\#\#\#\#\#\#\#\#\#\#\#\#\#\#\#\#\#\#\#\#\#}

\FunctionTok{ggplot}\NormalTok{() }\SpecialCharTok{+}
  \CommentTok{\# 1. Сетка без заливки}
  \FunctionTok{geom\_sf}\NormalTok{(}\AttributeTok{data =}\NormalTok{ grid\_sf, }\AttributeTok{fill =} \ConstantTok{NA}\NormalTok{, }\AttributeTok{color =} \StringTok{"grey80"}\NormalTok{, }\AttributeTok{linewidth =} \FloatTok{0.3}\NormalTok{) }\SpecialCharTok{+}
  
  \CommentTok{\# 2. Границы России}
  \FunctionTok{geom\_sf}\NormalTok{(}\AttributeTok{data =}\NormalTok{ russia, }\AttributeTok{fill =} \StringTok{"grey95"}\NormalTok{) }\SpecialCharTok{+}
  
  \CommentTok{\# 3. Центроиды ячеек с CATCH (цвет и размер по значению)}
  \FunctionTok{geom\_sf}\NormalTok{(}
    \AttributeTok{data =}\NormalTok{ CENTROIDS\_W, }
    \FunctionTok{aes}\NormalTok{(}\AttributeTok{size =}\NormalTok{ CATCH, }\AttributeTok{color =}\NormalTok{ CATCH),}
    \AttributeTok{shape =} \DecValTok{16}\NormalTok{, }
    \AttributeTok{alpha =} \FloatTok{0.7}
\NormalTok{  ) }\SpecialCharTok{+}
  
  \CommentTok{\# 4. Цветовая шкала (viridis как в первом скрипте)}
  \FunctionTok{scale\_color\_viridis\_c}\NormalTok{(}
    \AttributeTok{option =} \StringTok{"H"}\NormalTok{, }
    \AttributeTok{name =} \ConstantTok{NULL}\NormalTok{,}
    \AttributeTok{limits =} \FunctionTok{c}\NormalTok{(}\DecValTok{0}\NormalTok{, }\FunctionTok{max}\NormalTok{(gird\_shares\_catch}\SpecialCharTok{$}\NormalTok{CATCH, }\AttributeTok{na.rm =} \ConstantTok{TRUE}\NormalTok{))}
\NormalTok{  ) }\SpecialCharTok{+}
  
  \CommentTok{\# 5. Шкала размера центроидов}
  \FunctionTok{scale\_size\_continuous}\NormalTok{(}
    \AttributeTok{range =} \FunctionTok{c}\NormalTok{(}\DecValTok{1}\NormalTok{, }\DecValTok{10}\NormalTok{), }
    \AttributeTok{name =} \StringTok{"CPUE"}
\NormalTok{  ) }\SpecialCharTok{+}
  
  \CommentTok{\# 6. Обрезаем область отображения}
  \FunctionTok{coord\_sf}\NormalTok{(}
    \AttributeTok{xlim =} \FunctionTok{c}\NormalTok{(xmin, xmax), }
    \AttributeTok{ylim =} \FunctionTok{c}\NormalTok{(ymin, ymax),}
    \AttributeTok{expand =} \ConstantTok{FALSE}  \CommentTok{\# Точное соответствие границ}
\NormalTok{  ) }\SpecialCharTok{+}
  
  \CommentTok{\# 7. Шкалы для осей координат}
  \FunctionTok{scale\_x\_continuous}\NormalTok{(}
    \AttributeTok{breaks =} \FunctionTok{seq}\NormalTok{(xmin, xmax, }\AttributeTok{by =} \DecValTok{2}\NormalTok{),  }\CommentTok{\# Метки каждые 2 градуса}
    \AttributeTok{name =} \StringTok{"Долгота"}
\NormalTok{  ) }\SpecialCharTok{+}
  \FunctionTok{scale\_y\_continuous}\NormalTok{(}
    \AttributeTok{breaks =} \FunctionTok{seq}\NormalTok{(ymin, ymax, }\AttributeTok{by =} \FloatTok{0.5}\NormalTok{),  }\CommentTok{\# Метки каждые 0.5 градуса}
    \AttributeTok{name =} \StringTok{"Широта"}
\NormalTok{  ) }\SpecialCharTok{+}
  
  \CommentTok{\# 8. Тема оформления}
  \FunctionTok{theme\_minimal}\NormalTok{() }\SpecialCharTok{+}
  \FunctionTok{theme}\NormalTok{(}
    \AttributeTok{panel.grid =} \FunctionTok{element\_blank}\NormalTok{(),}
    \AttributeTok{legend.position =} \StringTok{"bottom"}\NormalTok{,}
    \AttributeTok{panel.border =} \FunctionTok{element\_rect}\NormalTok{(}\AttributeTok{fill =} \ConstantTok{NA}\NormalTok{, }\AttributeTok{color =} \StringTok{"black"}\NormalTok{, }\AttributeTok{size =} \FloatTok{0.5}\NormalTok{),}
    \CommentTok{\# Добавляем сетку для осей координат}
    \AttributeTok{panel.grid.major =} \FunctionTok{element\_line}\NormalTok{(}\AttributeTok{color =} \StringTok{"gray90"}\NormalTok{, }\AttributeTok{linewidth =} \FloatTok{0.2}\NormalTok{)}
\NormalTok{  ) }\SpecialCharTok{+}
  
  \CommentTok{\# 9. Явное указание названий осей (дублируем для надежности)}
  \FunctionTok{labs}\NormalTok{(}\AttributeTok{x =} \StringTok{"Долгота"}\NormalTok{, }\AttributeTok{y =} \StringTok{"Широта"}\NormalTok{)}
\end{Highlighting}
\end{Shaded}

\section{Промысловые карты -
картограммы}\label{ux43fux440ux43eux43cux44bux441ux43bux43eux432ux44bux435-ux43aux430ux440ux442ux44b---ux43aux430ux440ux442ux43eux433ux440ux430ux43cux43cux44b}

\begin{figure}[H]

{\centering \includegraphics[width=0.8\linewidth,height=\textheight,keepaspectratio]{images/KARTOGRAPH10.jpg}

}

\caption{Рис. 10.: Промысловые карты - картограммы}

\end{figure}%

\begin{Shaded}
\begin{Highlighting}[]
\CommentTok{\# Очистка окружения и установка рабочей директории}
\FunctionTok{rm}\NormalTok{(}\AttributeTok{list =} \FunctionTok{ls}\NormalTok{())}
\FunctionTok{setwd}\NormalTok{(}\StringTok{"C:/COURSES/KARTOGRAPH/"}\NormalTok{)}

\CommentTok{\# Загрузка необходимых библиотек}
\FunctionTok{library}\NormalTok{(rnaturalearth)}
\FunctionTok{library}\NormalTok{(tidyverse)}
\FunctionTok{library}\NormalTok{(sf)}

\DocumentationTok{\#\#\#\#\#\#\# ЗАГРУЗКА ДАННЫХ И ПОДГОТОВКА ПРОСТРАНСТВЕННЫХ ОБЪЕКТОВ \#\#\#\#\#\#\#\#\#\#\#\#\#\#\#\#}

\CommentTok{\# Чтение и фильтрация данных}
\NormalTok{DATA }\OtherTok{\textless{}{-}}\NormalTok{ readxl}\SpecialCharTok{::}\FunctionTok{read\_excel}\NormalTok{(}\StringTok{"KARTOGRAPHIC.xlsx"}\NormalTok{, }\AttributeTok{sheet =} \StringTok{"FISHERY"}\NormalTok{) }\SpecialCharTok{\%\textgreater{}\%} 
  \FunctionTok{filter}\NormalTok{(YEAR }\SpecialCharTok{==} \DecValTok{2023}\NormalTok{)}

\CommentTok{\# Преобразуем CPUE в пространственные точки}
\NormalTok{spec\_points }\OtherTok{\textless{}{-}} \FunctionTok{st\_as\_sf}\NormalTok{(DATA, }\AttributeTok{coords =} \FunctionTok{c}\NormalTok{(}\StringTok{"X"}\NormalTok{, }\StringTok{"Y"}\NormalTok{), }\AttributeTok{crs =} \DecValTok{4326}\NormalTok{)}

\CommentTok{\# Карта России}
\NormalTok{russia }\OtherTok{\textless{}{-}} \FunctionTok{ne\_countries}\NormalTok{(}\AttributeTok{scale =} \DecValTok{10}\NormalTok{, }\AttributeTok{country =} \StringTok{"Russia"}\NormalTok{) }

\CommentTok{\# Параметры карты и сетки}
\NormalTok{xmin }\OtherTok{\textless{}{-}} \DecValTok{32}\NormalTok{; xmax }\OtherTok{\textless{}{-}} \DecValTok{48}\NormalTok{; ymin }\OtherTok{\textless{}{-}} \DecValTok{68}\NormalTok{; ymax }\OtherTok{\textless{}{-}} \DecValTok{72}
\NormalTok{xcs }\OtherTok{\textless{}{-}} \DecValTok{1}\NormalTok{; ycs }\OtherTok{\textless{}{-}} \FloatTok{0.25}


\CommentTok{\# Создание основного датафрейма и пространственных объектов}
\NormalTok{points\_sf }\OtherTok{\textless{}{-}} \FunctionTok{st\_as\_sf}\NormalTok{(DATA, }\AttributeTok{coords =} \FunctionTok{c}\NormalTok{(}\StringTok{"X"}\NormalTok{, }\StringTok{"Y"}\NormalTok{), }\AttributeTok{crs =} \DecValTok{4326}\NormalTok{)}

\CommentTok{\# Создание сетки}
\NormalTok{grid\_sf }\OtherTok{\textless{}{-}} \FunctionTok{st\_make\_grid}\NormalTok{(points\_sf, }
                        \AttributeTok{cellsize =} \FunctionTok{c}\NormalTok{(xcs, ycs),}
                        \AttributeTok{n =} \FunctionTok{c}\NormalTok{(}\DecValTok{2} \SpecialCharTok{+}\NormalTok{ (xmax }\SpecialCharTok{{-}}\NormalTok{ xmin)}\SpecialCharTok{/}\NormalTok{xcs, }\DecValTok{2} \SpecialCharTok{+}\NormalTok{ (ymax }\SpecialCharTok{{-}}\NormalTok{ ymin)}\SpecialCharTok{/}\NormalTok{ycs),}
                        \AttributeTok{offset =} \FunctionTok{c}\NormalTok{(xmin }\SpecialCharTok{{-}}\NormalTok{ xcs, ymin }\SpecialCharTok{{-}}\NormalTok{ ycs)) }\SpecialCharTok{\%\textgreater{}\%} 
  \FunctionTok{st\_sf}\NormalTok{() }\SpecialCharTok{\%\textgreater{}\%} 
  \FunctionTok{mutate}\NormalTok{(}\AttributeTok{cell\_id =} \FunctionTok{row\_number}\NormalTok{())}

\CommentTok{\# Присоединяем точки Catch к сетке и агрегируем по ячейкам и годам}
\NormalTok{shares\_df\_catch }\OtherTok{\textless{}{-}} \FunctionTok{st\_join}\NormalTok{(points\_sf, grid\_sf) }\SpecialCharTok{\%\textgreater{}\%} 
  \FunctionTok{st\_drop\_geometry}\NormalTok{() }\SpecialCharTok{\%\textgreater{}\%} 
  \FunctionTok{group\_by}\NormalTok{(cell\_id, YEAR) }\SpecialCharTok{\%\textgreater{}\%} 
  \FunctionTok{summarise}\NormalTok{(}
    \AttributeTok{Count =} \FunctionTok{n}\NormalTok{(),}
    \AttributeTok{CATCH =} \FunctionTok{mean}\NormalTok{(CPUE, }\AttributeTok{na.rm =} \ConstantTok{TRUE}\NormalTok{)}
\NormalTok{  ) }\SpecialCharTok{\%\textgreater{}\%} 
  \FunctionTok{ungroup}\NormalTok{()}

\CommentTok{\# Присоединяем статистику Catch к сетке}
\NormalTok{gird\_shares\_catch }\OtherTok{\textless{}{-}} \FunctionTok{right\_join}\NormalTok{(grid\_sf, shares\_df\_catch, }\AttributeTok{by =} \StringTok{"cell\_id"}\NormalTok{)}



\CommentTok{\# Центроиды сетки по W}
\NormalTok{CENTROIDS\_W }\OtherTok{\textless{}{-}}\NormalTok{ gird\_shares\_catch }\SpecialCharTok{\%\textgreater{}\%} 
  \FunctionTok{st\_centroid}\NormalTok{()}

\DocumentationTok{\#\#\#\#\#\#\#\#\#\#\#\#\#\#\#\#\#\#\#\# ВИЗУАЛИЗАЦИЯ \#\#\#\#\#\#\#\#\#\#\#\#\#\#\#\#\#\#\#\#\#\#\#\#\#\#\#\#\#\#\#\#\#\#\#\#\#\#\#\#\#}

\FunctionTok{ggplot}\NormalTok{() }\SpecialCharTok{+}
  \CommentTok{\# 1. Сетка без заливки}
  \FunctionTok{geom\_sf}\NormalTok{(}\AttributeTok{data =}\NormalTok{ grid\_sf, }\AttributeTok{fill =} \ConstantTok{NA}\NormalTok{, }\AttributeTok{color =} \StringTok{"grey80"}\NormalTok{, }\AttributeTok{linewidth =} \FloatTok{0.3}\NormalTok{) }\SpecialCharTok{+}
  
  \CommentTok{\# 2. Границы России}
  \FunctionTok{geom\_sf}\NormalTok{(}\AttributeTok{data =}\NormalTok{ russia, }\AttributeTok{fill =} \StringTok{"grey95"}\NormalTok{) }\SpecialCharTok{+}
  
  \CommentTok{\# 3. Заливка по улову с палитрой viridis option "H"}
  \FunctionTok{geom\_sf}\NormalTok{(}\AttributeTok{data =}\NormalTok{ gird\_shares\_catch, }\FunctionTok{aes}\NormalTok{(}\AttributeTok{fill =}\NormalTok{ CATCH), }\AttributeTok{color =} \ConstantTok{NA}\NormalTok{) }\SpecialCharTok{+}
  
  \CommentTok{\# 4. Цветовая шкала viridis option "H" для заливки}
  \FunctionTok{scale\_fill\_viridis\_c}\NormalTok{(}
    \AttributeTok{option =} \StringTok{"H"}\NormalTok{, }
    \AttributeTok{name =} \StringTok{"CPUE"}\NormalTok{,}
    \AttributeTok{limits =} \FunctionTok{c}\NormalTok{(}\DecValTok{0}\NormalTok{, }\FunctionTok{max}\NormalTok{(gird\_shares\_catch}\SpecialCharTok{$}\NormalTok{CATCH, }\AttributeTok{na.rm =} \ConstantTok{TRUE}\NormalTok{)),}
    \AttributeTok{na.value =} \StringTok{"transparent"}
\NormalTok{  ) }\SpecialCharTok{+}
  
  \CommentTok{\# 5. Обрезаем область отображения}
  \FunctionTok{coord\_sf}\NormalTok{(}
    \AttributeTok{xlim =} \FunctionTok{c}\NormalTok{(xmin, xmax), }
    \AttributeTok{ylim =} \FunctionTok{c}\NormalTok{(ymin, ymax),}
    \AttributeTok{expand =} \ConstantTok{FALSE}
\NormalTok{  ) }\SpecialCharTok{+}
  
  \CommentTok{\# 6. Шкалы для осей координат}
  \FunctionTok{scale\_x\_continuous}\NormalTok{(}
    \AttributeTok{breaks =} \FunctionTok{seq}\NormalTok{(xmin, xmax, }\AttributeTok{by =} \DecValTok{2}\NormalTok{),}
    \AttributeTok{name =} \StringTok{"Долгота"}
\NormalTok{  ) }\SpecialCharTok{+}
  \FunctionTok{scale\_y\_continuous}\NormalTok{(}
    \AttributeTok{breaks =} \FunctionTok{seq}\NormalTok{(ymin, ymax, }\AttributeTok{by =} \FloatTok{0.5}\NormalTok{),}
    \AttributeTok{name =} \StringTok{"Широта"}
\NormalTok{  ) }\SpecialCharTok{+}
  
  \CommentTok{\# 7. Тема оформления}
  \FunctionTok{theme\_minimal}\NormalTok{() }\SpecialCharTok{+}
  \FunctionTok{theme}\NormalTok{(}
    \AttributeTok{panel.grid =} \FunctionTok{element\_blank}\NormalTok{(),}
    \AttributeTok{legend.position =} \StringTok{"bottom"}\NormalTok{,}
    \AttributeTok{panel.border =} \FunctionTok{element\_rect}\NormalTok{(}\AttributeTok{fill =} \ConstantTok{NA}\NormalTok{, }\AttributeTok{color =} \StringTok{"black"}\NormalTok{, }\AttributeTok{size =} \FloatTok{0.5}\NormalTok{),}
    \AttributeTok{panel.grid.major =} \FunctionTok{element\_line}\NormalTok{(}\AttributeTok{color =} \StringTok{"gray90"}\NormalTok{, }\AttributeTok{size =} \FloatTok{0.2}\NormalTok{)}
\NormalTok{  ) }\SpecialCharTok{+}
  \FunctionTok{labs}\NormalTok{(}\AttributeTok{x =} \StringTok{"Долгота"}\NormalTok{, }\AttributeTok{y =} \StringTok{"Широта"}\NormalTok{)}
\end{Highlighting}
\end{Shaded}

\section{Промысловые карты - картограммы по
фасеткам}\label{ux43fux440ux43eux43cux44bux441ux43bux43eux432ux44bux435-ux43aux430ux440ux442ux44b---ux43aux430ux440ux442ux43eux433ux440ux430ux43cux43cux44b-ux43fux43e-ux444ux430ux441ux435ux442ux43aux430ux43c}

\begin{figure}[H]

{\centering \includegraphics[width=0.8\linewidth,height=\textheight,keepaspectratio]{images/KARTOGRAPH11.jpg}

}

\caption{Рис. 11.: Промысловые карты - картограммы по фасеткам}

\end{figure}%

\begin{Shaded}
\begin{Highlighting}[]
\CommentTok{\# Очистка окружения и установка рабочей директории}
\FunctionTok{rm}\NormalTok{(}\AttributeTok{list =} \FunctionTok{ls}\NormalTok{())}
\FunctionTok{setwd}\NormalTok{(}\StringTok{"C:/COURSES/KARTOGRAPH/"}\NormalTok{)}

\FunctionTok{library}\NormalTok{(rnaturalearth)}
\FunctionTok{library}\NormalTok{(tidyverse)}
\FunctionTok{library}\NormalTok{(ggspatial)}
\FunctionTok{library}\NormalTok{(sf)}

\DocumentationTok{\#\#\#\#\#\#\# READ DATA AND PREPARE SPATIAL OBJECTS \#\#\#\#\#\#\#\#\#\#\#\#\#\#\#\#\#\#\#\#\#\#\#\#\#\#\#\#}

\CommentTok{\# Чтение и фильтрация данных}
\NormalTok{DATA }\OtherTok{\textless{}{-}}\NormalTok{ readxl}\SpecialCharTok{::}\FunctionTok{read\_excel}\NormalTok{(}\StringTok{"KARTOGRAPHIC.xlsx"}\NormalTok{, }\AttributeTok{sheet =} \StringTok{"FISHERY"}\NormalTok{) }\SpecialCharTok{\%\textgreater{}\%} 
  \FunctionTok{filter}\NormalTok{(YEAR }\SpecialCharTok{\textgreater{}} \DecValTok{2020} \SpecialCharTok{\&}\NormalTok{ YEAR }\SpecialCharTok{\textless{}} \DecValTok{2025}\NormalTok{)}

\CommentTok{\# Карта России}
\NormalTok{russia }\OtherTok{\textless{}{-}} \FunctionTok{ne\_countries}\NormalTok{(}\AttributeTok{scale =} \DecValTok{10}\NormalTok{, }\AttributeTok{country =} \StringTok{"Russia"}\NormalTok{) }

\CommentTok{\# Параметры карты и сетки}
\NormalTok{xmin }\OtherTok{\textless{}{-}} \DecValTok{32}\NormalTok{; xmax }\OtherTok{\textless{}{-}} \DecValTok{48}\NormalTok{; ymin }\OtherTok{\textless{}{-}} \DecValTok{68}\NormalTok{; ymax }\OtherTok{\textless{}{-}} \DecValTok{72}
\NormalTok{xcs }\OtherTok{\textless{}{-}} \DecValTok{2}\NormalTok{; ycs }\OtherTok{\textless{}{-}} \FloatTok{0.5}

\CommentTok{\# Преобразование в пространственные объекты}
\NormalTok{points\_sf }\OtherTok{\textless{}{-}} \FunctionTok{st\_as\_sf}\NormalTok{(DATA, }\AttributeTok{coords =} \FunctionTok{c}\NormalTok{(}\StringTok{"X"}\NormalTok{, }\StringTok{"Y"}\NormalTok{), }\AttributeTok{crs =} \DecValTok{4326}\NormalTok{)}

\CommentTok{\# Создание сетки}
\NormalTok{grid\_sf }\OtherTok{\textless{}{-}} \FunctionTok{st\_make\_grid}\NormalTok{(}
\NormalTok{  points\_sf,}
  \AttributeTok{cellsize =} \FunctionTok{c}\NormalTok{(xcs, ycs),}
  \AttributeTok{n =} \FunctionTok{c}\NormalTok{(}\DecValTok{2} \SpecialCharTok{+}\NormalTok{ (xmax }\SpecialCharTok{{-}}\NormalTok{ xmin)}\SpecialCharTok{/}\NormalTok{xcs, }\DecValTok{2} \SpecialCharTok{+}\NormalTok{ (ymax }\SpecialCharTok{{-}}\NormalTok{ ymin)}\SpecialCharTok{/}\NormalTok{ycs),}
  \AttributeTok{offset =} \FunctionTok{c}\NormalTok{(xmin }\SpecialCharTok{{-}}\NormalTok{ xcs, ymin }\SpecialCharTok{{-}}\NormalTok{ ycs)}
\NormalTok{) }\SpecialCharTok{\%\textgreater{}\%} 
  \FunctionTok{st\_sf}\NormalTok{() }\SpecialCharTok{\%\textgreater{}\%} 
  \FunctionTok{mutate}\NormalTok{(}\AttributeTok{cell\_id =} \FunctionTok{row\_number}\NormalTok{())}

\CommentTok{\# Агрегация данных по сетке и годам}
\NormalTok{shares\_df\_catch }\OtherTok{\textless{}{-}}\NormalTok{ points\_sf }\SpecialCharTok{\%\textgreater{}\%} 
  \FunctionTok{st\_join}\NormalTok{(grid\_sf) }\SpecialCharTok{\%\textgreater{}\%} 
  \FunctionTok{st\_drop\_geometry}\NormalTok{() }\SpecialCharTok{\%\textgreater{}\%} 
  \FunctionTok{group\_by}\NormalTok{(cell\_id, YEAR) }\SpecialCharTok{\%\textgreater{}\%} 
  \FunctionTok{summarise}\NormalTok{(}\AttributeTok{CATCH =} \FunctionTok{mean}\NormalTok{(CPUE, }\AttributeTok{na.rm =} \ConstantTok{TRUE}\NormalTok{), }\AttributeTok{.groups =} \StringTok{\textquotesingle{}drop\textquotesingle{}}\NormalTok{)}

\CommentTok{\# Присоединение статистики к сетке}
\NormalTok{gird\_shares\_catch }\OtherTok{\textless{}{-}}\NormalTok{ grid\_sf }\SpecialCharTok{\%\textgreater{}\%} 
  \FunctionTok{right\_join}\NormalTok{(shares\_df\_catch, }\AttributeTok{by =} \StringTok{"cell\_id"}\NormalTok{)}

\DocumentationTok{\#\#\#\#\#\#\#\#\#\#\#\#\#\#\#\#\#\#\#\# ВИЗУАЛИЗАЦИЯ \#\#\#\#\#\#\#\#\#\#\#\#\#\#\#\#\#\#\#\#\#\#\#\#\#\#\#\#\#\#\#\#\#\#\#\#\#\#\#\#\#}

\CommentTok{\# Определяем общий максимум CPUE для единой шкалы цветов}
\NormalTok{catch\_max }\OtherTok{\textless{}{-}} \FunctionTok{max}\NormalTok{(gird\_shares\_catch}\SpecialCharTok{$}\NormalTok{CATCH, }\AttributeTok{na.rm =} \ConstantTok{TRUE}\NormalTok{)}

\CommentTok{\# Рассчитываем шаг для подписей (в 2 раза реже исходной сетки)}
\NormalTok{x\_breaks }\OtherTok{\textless{}{-}} \FunctionTok{seq}\NormalTok{(xmin, xmax, }\AttributeTok{by =}\NormalTok{ xcs }\SpecialCharTok{*} \DecValTok{2}\NormalTok{)  }\CommentTok{\# 4 градуса}
\NormalTok{y\_breaks }\OtherTok{\textless{}{-}} \FunctionTok{seq}\NormalTok{(ymin, ymax, }\AttributeTok{by =}\NormalTok{ ycs }\SpecialCharTok{*} \DecValTok{2}\NormalTok{)  }\CommentTok{\# 1 градус}

\CommentTok{\# Функция для форматирования подписей: пропускаем первую подпись}
\NormalTok{format\_labels }\OtherTok{\textless{}{-}} \ControlFlowTok{function}\NormalTok{(breaks) \{}
\NormalTok{  labels }\OtherTok{\textless{}{-}} \FunctionTok{paste0}\NormalTok{(breaks, }\StringTok{"°"}\NormalTok{)}
\NormalTok{  labels[}\DecValTok{1}\NormalTok{] }\OtherTok{\textless{}{-}} \StringTok{""}  \CommentTok{\# Пропускаем первую подпись}
  \FunctionTok{return}\NormalTok{(labels)}
\NormalTok{\}}

\FunctionTok{ggplot}\NormalTok{() }\SpecialCharTok{+}
  \CommentTok{\# Контуры сетки}
  \FunctionTok{geom\_sf}\NormalTok{(}\AttributeTok{data =}\NormalTok{ grid\_sf, }\AttributeTok{fill =} \ConstantTok{NA}\NormalTok{, }\AttributeTok{color =} \StringTok{"grey80"}\NormalTok{, }\AttributeTok{linewidth =} \FloatTok{0.3}\NormalTok{) }\SpecialCharTok{+}
  
  \CommentTok{\# Заливка по улову с цветовой схемой viridis}
  \FunctionTok{geom\_sf}\NormalTok{(}\AttributeTok{data =}\NormalTok{ gird\_shares\_catch, }\FunctionTok{aes}\NormalTok{(}\AttributeTok{fill =}\NormalTok{ CATCH), }\AttributeTok{color =} \ConstantTok{NA}\NormalTok{) }\SpecialCharTok{+}
  
  \CommentTok{\# Границы России}
  \FunctionTok{geom\_sf}\NormalTok{(}\AttributeTok{data =}\NormalTok{ russia, }\AttributeTok{fill =} \StringTok{"\#E8E5D6"}\NormalTok{) }\SpecialCharTok{+}
  
  \CommentTok{\# Фасетирование по годам}
  \FunctionTok{facet\_wrap}\NormalTok{(}\SpecialCharTok{\textasciitilde{}}\NormalTok{ YEAR, }\AttributeTok{nrow =} \DecValTok{2}\NormalTok{) }\SpecialCharTok{+}
  
  \CommentTok{\# Цветовая шкала}
  \FunctionTok{scale\_fill\_viridis\_c}\NormalTok{(}
    \AttributeTok{option =} \StringTok{"H"}\NormalTok{, }
    \AttributeTok{name =} \StringTok{"CPUE"}\NormalTok{,}
    \AttributeTok{limits =} \FunctionTok{c}\NormalTok{(}\DecValTok{0}\NormalTok{, catch\_max),}
    \AttributeTok{na.value =} \StringTok{"transparent"}
\NormalTok{  ) }\SpecialCharTok{+}
  
  \CommentTok{\# Область отображения}
  \FunctionTok{coord\_sf}\NormalTok{(}
    \AttributeTok{xlim =} \FunctionTok{c}\NormalTok{(xmin, xmax), }
    \AttributeTok{ylim =} \FunctionTok{c}\NormalTok{(ymin, ymax),}
    \AttributeTok{expand =} \ConstantTok{FALSE}
\NormalTok{  ) }\SpecialCharTok{+}
  
  \CommentTok{\# Управление подписями осей с символом градуса (пропускаем первую подпись)}
  \FunctionTok{scale\_x\_continuous}\NormalTok{(}
    \AttributeTok{breaks =}\NormalTok{ x\_breaks,}
    \AttributeTok{labels =}\NormalTok{ format\_labels}
\NormalTok{  ) }\SpecialCharTok{+}
  \FunctionTok{scale\_y\_continuous}\NormalTok{(}
    \AttributeTok{breaks =}\NormalTok{ y\_breaks,}
    \AttributeTok{labels =}\NormalTok{ format\_labels}
\NormalTok{  ) }\SpecialCharTok{+}
  
  \CommentTok{\# Оформление с тиками на осях}
  \FunctionTok{theme\_minimal}\NormalTok{() }\SpecialCharTok{+}
  \FunctionTok{theme}\NormalTok{(}
    \AttributeTok{panel.grid =} \FunctionTok{element\_blank}\NormalTok{(),}
    \AttributeTok{legend.position =} \StringTok{"bottom"}\NormalTok{,}
    \AttributeTok{legend.key.width =} \FunctionTok{unit}\NormalTok{(}\FloatTok{2.5}\NormalTok{, }\StringTok{"cm"}\NormalTok{),}
    \AttributeTok{legend.title =} \FunctionTok{element\_text}\NormalTok{(}\AttributeTok{vjust =} \FloatTok{0.8}\NormalTok{, }\AttributeTok{size =} \DecValTok{12}\NormalTok{),}
    \AttributeTok{panel.border =} \FunctionTok{element\_rect}\NormalTok{(}\AttributeTok{fill =} \ConstantTok{NA}\NormalTok{, }\AttributeTok{color =} \StringTok{"black"}\NormalTok{, }\AttributeTok{size =} \FloatTok{0.7}\NormalTok{),}
    \AttributeTok{panel.grid.major =} \FunctionTok{element\_line}\NormalTok{(}\AttributeTok{color =} \StringTok{"grey90"}\NormalTok{, }\AttributeTok{size =} \FloatTok{0.2}\NormalTok{),}
    \AttributeTok{strip.background =} \FunctionTok{element\_rect}\NormalTok{(}\AttributeTok{fill =} \StringTok{"\#E8E5D6"}\NormalTok{, }\AttributeTok{color =} \StringTok{"black"}\NormalTok{),}
    \AttributeTok{strip.text =} \FunctionTok{element\_text}\NormalTok{(}\AttributeTok{face =} \StringTok{"bold"}\NormalTok{, }\AttributeTok{size =} \DecValTok{12}\NormalTok{),}
    \AttributeTok{axis.text.x =} \FunctionTok{element\_text}\NormalTok{(}\AttributeTok{size =} \DecValTok{9}\NormalTok{, }\AttributeTok{angle =} \DecValTok{0}\NormalTok{, }\AttributeTok{margin =} \FunctionTok{margin}\NormalTok{(}\AttributeTok{t =} \DecValTok{5}\NormalTok{)),}
    \AttributeTok{axis.text.y =} \FunctionTok{element\_text}\NormalTok{(}\AttributeTok{size =} \DecValTok{9}\NormalTok{, }\AttributeTok{angle =} \DecValTok{90}\NormalTok{, }\AttributeTok{hjust =} \FloatTok{0.5}\NormalTok{, }\AttributeTok{margin =} \FunctionTok{margin}\NormalTok{(}\AttributeTok{r =} \DecValTok{5}\NormalTok{)),}
    \AttributeTok{axis.title.x =} \FunctionTok{element\_blank}\NormalTok{(),}
    \AttributeTok{axis.title.y =} \FunctionTok{element\_blank}\NormalTok{(),}
    
    \CommentTok{\# Тики (засечки) на оси}
    \AttributeTok{axis.ticks =} \FunctionTok{element\_line}\NormalTok{(}\AttributeTok{color =} \StringTok{"black"}\NormalTok{, }\AttributeTok{size =} \FloatTok{0.5}\NormalTok{),}
    \AttributeTok{axis.ticks.length =} \FunctionTok{unit}\NormalTok{(}\FloatTok{0.2}\NormalTok{, }\StringTok{"cm"}\NormalTok{),}
    \AttributeTok{axis.ticks.x =} \FunctionTok{element\_line}\NormalTok{(}\AttributeTok{color =} \StringTok{"black"}\NormalTok{, }\AttributeTok{size =} \FloatTok{0.5}\NormalTok{),}
    \AttributeTok{axis.ticks.y =} \FunctionTok{element\_line}\NormalTok{(}\AttributeTok{color =} \StringTok{"black"}\NormalTok{, }\AttributeTok{size =} \FloatTok{0.5}\NormalTok{)}
\NormalTok{  ) }\SpecialCharTok{+}
  
  \CommentTok{\# Настройка легенды}
  \FunctionTok{guides}\NormalTok{(}\AttributeTok{fill =} \FunctionTok{guide\_colorbar}\NormalTok{(}
    \AttributeTok{title.position =} \StringTok{"top"}\NormalTok{,}
    \AttributeTok{title.hjust =} \FloatTok{0.5}\NormalTok{,}
    \AttributeTok{barwidth =} \DecValTok{15}\NormalTok{,}
    \AttributeTok{frame.colour =} \StringTok{"black"}\NormalTok{,}
    \AttributeTok{ticks.colour =} \StringTok{"black"}
\NormalTok{  ))}

\CommentTok{\# Сохранение результата}
\FunctionTok{ggsave}\NormalTok{(}\StringTok{"KARTOGRAPH11.jpg"}\NormalTok{, }
       \AttributeTok{device =} \StringTok{"jpeg"}\NormalTok{, }
       \AttributeTok{dpi =} \DecValTok{300}\NormalTok{,}
       \AttributeTok{width =} \DecValTok{7}\NormalTok{,}
       \AttributeTok{height =} \DecValTok{5}\NormalTok{,}
       \AttributeTok{units =} \StringTok{"in"}\NormalTok{)}
\end{Highlighting}
\end{Shaded}

\section{Гибридные карты - картограммы и точки (съемка и промысловые
данные)}\label{ux433ux438ux431ux440ux438ux434ux43dux44bux435-ux43aux430ux440ux442ux44b---ux43aux430ux440ux442ux43eux433ux440ux430ux43cux43cux44b-ux438-ux442ux43eux447ux43aux438-ux441ux44aux435ux43cux43aux430-ux438-ux43fux440ux43eux43cux44bux441ux43bux43eux432ux44bux435-ux434ux430ux43dux43dux44bux435}

\begin{figure}[H]

{\centering \includegraphics[width=0.8\linewidth,height=\textheight,keepaspectratio]{images/KARTOGRAPH12.jpg}

}

\caption{Рис. 12.: Гибридные карты - картограммы и точки (съемка и
промысловые данные)}

\end{figure}%

\begin{Shaded}
\begin{Highlighting}[]
\CommentTok{\# Очистка окружения и установка рабочей директории}
\FunctionTok{rm}\NormalTok{(}\AttributeTok{list =} \FunctionTok{ls}\NormalTok{())}
\FunctionTok{setwd}\NormalTok{(}\StringTok{"C:/COURSES/KARTOGRAPH/"}\NormalTok{)}

\FunctionTok{library}\NormalTok{(rnaturalearth)}
\FunctionTok{library}\NormalTok{(tidyverse)}
\FunctionTok{library}\NormalTok{(ggspatial)}
\FunctionTok{library}\NormalTok{(sf)}

\DocumentationTok{\#\#\#\#\#\#\# READ DATA AND PREPARE SPATIAL OBJECTS \#\#\#\#\#\#\#\#\#\#\#\#\#\#\#\#\#\#\#\#\#\#\#\#\#\#\#\#}

\CommentTok{\# Чтение и фильтрация данных}
\NormalTok{DATA }\OtherTok{\textless{}{-}}\NormalTok{ readxl}\SpecialCharTok{::}\FunctionTok{read\_excel}\NormalTok{(}\StringTok{"KARTOGRAPHIC.xlsx"}\NormalTok{, }\AttributeTok{sheet =} \StringTok{"FISHERY"}\NormalTok{) }\SpecialCharTok{\%\textgreater{}\%} 
  \FunctionTok{filter}\NormalTok{(YEAR }\SpecialCharTok{\textgreater{}} \DecValTok{2020} \SpecialCharTok{\&}\NormalTok{ YEAR }\SpecialCharTok{\textless{}} \DecValTok{2025}\NormalTok{)}

\NormalTok{SURVEY }\OtherTok{\textless{}{-}}\NormalTok{ readxl}\SpecialCharTok{::}\FunctionTok{read\_excel}\NormalTok{(}\StringTok{"KARTOGRAPHIC.xlsx"}\NormalTok{, }\AttributeTok{sheet =} \StringTok{"SURVEY"}\NormalTok{) }\SpecialCharTok{\%\textgreater{}\%} 
  \FunctionTok{filter}\NormalTok{(YEAR }\SpecialCharTok{\textgreater{}} \DecValTok{2020} \SpecialCharTok{\&}\NormalTok{ YEAR }\SpecialCharTok{\textless{}} \DecValTok{2025}\NormalTok{, SURV }\SpecialCharTok{==} \StringTok{"CRAB"}\NormalTok{)}

\CommentTok{\# Создание 4 категорий для переменной PROM}
\NormalTok{breaks }\OtherTok{\textless{}{-}} \FunctionTok{quantile}\NormalTok{(SURVEY}\SpecialCharTok{$}\NormalTok{PROM, }
                   \AttributeTok{probs =} \FunctionTok{c}\NormalTok{(}\DecValTok{0}\NormalTok{, }\FloatTok{0.25}\NormalTok{, }\FloatTok{0.5}\NormalTok{, }\FloatTok{0.75}\NormalTok{, }\DecValTok{1}\NormalTok{), }
                   \AttributeTok{na.rm =} \ConstantTok{TRUE}\NormalTok{)}
\NormalTok{SURVEY}\SpecialCharTok{$}\NormalTok{PROM\_cat }\OtherTok{\textless{}{-}} \FunctionTok{cut}\NormalTok{(SURVEY}\SpecialCharTok{$}\NormalTok{PROM,}
                       \AttributeTok{breaks =}\NormalTok{ breaks,}
                       \AttributeTok{include.lowest =} \ConstantTok{TRUE}\NormalTok{,}
                       \AttributeTok{labels =} \FunctionTok{c}\NormalTok{(}\StringTok{"Q1 (Low)"}\NormalTok{, }\StringTok{"Q2"}\NormalTok{, }\StringTok{"Q3"}\NormalTok{, }\StringTok{"Q4 (High)"}\NormalTok{))}

\CommentTok{\# Карта России}
\NormalTok{russia }\OtherTok{\textless{}{-}} \FunctionTok{ne\_countries}\NormalTok{(}\AttributeTok{scale =} \DecValTok{10}\NormalTok{, }\AttributeTok{country =} \StringTok{"Russia"}\NormalTok{) }

\CommentTok{\# Параметры карты и сетки}
\NormalTok{xmin }\OtherTok{\textless{}{-}} \DecValTok{32}\NormalTok{; xmax }\OtherTok{\textless{}{-}} \DecValTok{48}\NormalTok{; ymin }\OtherTok{\textless{}{-}} \DecValTok{68}\NormalTok{; ymax }\OtherTok{\textless{}{-}} \DecValTok{72}
\NormalTok{xcs }\OtherTok{\textless{}{-}} \DecValTok{2}\NormalTok{; ycs }\OtherTok{\textless{}{-}} \FloatTok{0.5}

\CommentTok{\# Преобразование в пространственные объекты}
\NormalTok{points\_sf }\OtherTok{\textless{}{-}} \FunctionTok{st\_as\_sf}\NormalTok{(DATA, }\AttributeTok{coords =} \FunctionTok{c}\NormalTok{(}\StringTok{"X"}\NormalTok{, }\StringTok{"Y"}\NormalTok{), }\AttributeTok{crs =} \DecValTok{4326}\NormalTok{)}
\NormalTok{survey\_sf }\OtherTok{\textless{}{-}} \FunctionTok{st\_as\_sf}\NormalTok{(SURVEY, }\AttributeTok{coords =} \FunctionTok{c}\NormalTok{(}\StringTok{"X"}\NormalTok{, }\StringTok{"Y"}\NormalTok{), }\AttributeTok{crs =} \DecValTok{4326}\NormalTok{)}

\CommentTok{\# Создание сетки}
\NormalTok{grid\_sf }\OtherTok{\textless{}{-}} \FunctionTok{st\_make\_grid}\NormalTok{(}
\NormalTok{  points\_sf,}
  \AttributeTok{cellsize =} \FunctionTok{c}\NormalTok{(xcs, ycs),}
  \AttributeTok{n =} \FunctionTok{c}\NormalTok{(}\DecValTok{2} \SpecialCharTok{+}\NormalTok{ (xmax }\SpecialCharTok{{-}}\NormalTok{ xmin)}\SpecialCharTok{/}\NormalTok{xcs, }\DecValTok{2} \SpecialCharTok{+}\NormalTok{ (ymax }\SpecialCharTok{{-}}\NormalTok{ ymin)}\SpecialCharTok{/}\NormalTok{ycs),}
  \AttributeTok{offset =} \FunctionTok{c}\NormalTok{(xmin }\SpecialCharTok{{-}}\NormalTok{ xcs, ymin }\SpecialCharTok{{-}}\NormalTok{ ycs)}
\NormalTok{) }\SpecialCharTok{\%\textgreater{}\%} 
  \FunctionTok{st\_sf}\NormalTok{() }\SpecialCharTok{\%\textgreater{}\%} 
  \FunctionTok{mutate}\NormalTok{(}\AttributeTok{cell\_id =} \FunctionTok{row\_number}\NormalTok{())}

\CommentTok{\# Агрегация данных по сетке и годам}
\NormalTok{shares\_df\_catch }\OtherTok{\textless{}{-}}\NormalTok{ points\_sf }\SpecialCharTok{\%\textgreater{}\%} 
  \FunctionTok{st\_join}\NormalTok{(grid\_sf) }\SpecialCharTok{\%\textgreater{}\%} 
  \FunctionTok{st\_drop\_geometry}\NormalTok{() }\SpecialCharTok{\%\textgreater{}\%} 
  \FunctionTok{group\_by}\NormalTok{(cell\_id, YEAR) }\SpecialCharTok{\%\textgreater{}\%} 
  \FunctionTok{summarise}\NormalTok{(}\AttributeTok{CATCH =} \FunctionTok{mean}\NormalTok{(CPUE, }\AttributeTok{na.rm =} \ConstantTok{TRUE}\NormalTok{), }\AttributeTok{.groups =} \StringTok{\textquotesingle{}drop\textquotesingle{}}\NormalTok{)}

\CommentTok{\# Присоединение статистики к сетке}
\NormalTok{gird\_shares\_catch }\OtherTok{\textless{}{-}}\NormalTok{ grid\_sf }\SpecialCharTok{\%\textgreater{}\%} 
  \FunctionTok{right\_join}\NormalTok{(shares\_df\_catch, }\AttributeTok{by =} \StringTok{"cell\_id"}\NormalTok{)}

\DocumentationTok{\#\#\#\#\#\#\#\#\#\#\#\#\#\#\#\#\#\#\# ВИЗУАЛИЗАЦИЯ \#\#\#\#\#\#\#\#\#\#\#\#\#\#\#\#\#\#\#\#\#\#\#\#\#\#\#\#\#\#\#\#\#\#\#\#\#\#\#\#\#}
\FunctionTok{ggplot}\NormalTok{() }\SpecialCharTok{+}
  \CommentTok{\# Контуры сетки}
  \FunctionTok{geom\_sf}\NormalTok{(}\AttributeTok{data =}\NormalTok{ grid\_sf, }\AttributeTok{fill =} \ConstantTok{NA}\NormalTok{, }\AttributeTok{color =} \StringTok{"grey80"}\NormalTok{, }\AttributeTok{linewidth =} \FloatTok{0.3}\NormalTok{) }\SpecialCharTok{+}
  
  \CommentTok{\# Заливка по улову (средний CPUE)}
  \FunctionTok{geom\_sf}\NormalTok{(}\AttributeTok{data =}\NormalTok{ gird\_shares\_catch, }\FunctionTok{aes}\NormalTok{(}\AttributeTok{fill =}\NormalTok{ CATCH)) }\SpecialCharTok{+}
  
  \CommentTok{\# Границы России}
  \FunctionTok{geom\_sf}\NormalTok{(}\AttributeTok{data =}\NormalTok{ russia, }\AttributeTok{fill =} \StringTok{"\#E8E5D6"}\NormalTok{) }\SpecialCharTok{+}
  
  \CommentTok{\# Точки SURVEY: голубые с черной окантовкой}
  \FunctionTok{geom\_sf}\NormalTok{(}\AttributeTok{data =}\NormalTok{ survey\_sf, }
          \FunctionTok{aes}\NormalTok{(}\AttributeTok{size =}\NormalTok{ PROM\_cat),}
          \AttributeTok{fill =} \StringTok{"lightblue"}\NormalTok{,    }\CommentTok{\# Голубая заливка}
          \AttributeTok{color =} \StringTok{"black"}\NormalTok{,        }\CommentTok{\# Черная окантовка}
          \AttributeTok{alpha =} \FloatTok{0.7}\NormalTok{,}
          \AttributeTok{shape =} \DecValTok{21}\NormalTok{,             }\CommentTok{\# Круг с обводкой}
          \AttributeTok{stroke =} \FloatTok{0.5}\NormalTok{,           }\CommentTok{\# Толщина окантовки}
          \AttributeTok{show.legend =} \StringTok{"point"}\NormalTok{) }\SpecialCharTok{+}
  
  \CommentTok{\# Фасетирование по годам}
  \FunctionTok{facet\_wrap}\NormalTok{(}\SpecialCharTok{\textasciitilde{}}\NormalTok{ YEAR, }\AttributeTok{nrow =} \DecValTok{2}\NormalTok{) }\SpecialCharTok{+}
  
  \CommentTok{\# Цветовая шкала для заливки}
  \FunctionTok{scale\_fill\_gradient}\NormalTok{(}
    \AttributeTok{low =} \StringTok{"white"}\NormalTok{, }
    \AttributeTok{high =} \StringTok{"red"}\NormalTok{,}
    \AttributeTok{na.value =} \ConstantTok{NA}\NormalTok{,}
    \AttributeTok{limits =} \FunctionTok{c}\NormalTok{(}\DecValTok{0}\NormalTok{, }\FunctionTok{max}\NormalTok{(gird\_shares\_catch}\SpecialCharTok{$}\NormalTok{CATCH, }\AttributeTok{na.rm =} \ConstantTok{TRUE}\NormalTok{)),}
    \AttributeTok{name =} \StringTok{"Catch (CPUE)"}
\NormalTok{  ) }\SpecialCharTok{+}
  
  \CommentTok{\# Шкала размеров для точек}
  \FunctionTok{scale\_size\_manual}\NormalTok{(}
    \AttributeTok{name =} \StringTok{"PROM Category"}\NormalTok{,}
    \AttributeTok{values =} \FunctionTok{c}\NormalTok{(}\FloatTok{1.5}\NormalTok{, }\FloatTok{2.5}\NormalTok{, }\FloatTok{3.5}\NormalTok{, }\FloatTok{4.5}\NormalTok{)  }\CommentTok{\# Размеры точек для 4 категорий}
\NormalTok{  ) }\SpecialCharTok{+}
  
  \DocumentationTok{\#\#\# ОСИ С ГЕОГРАФИЧЕСКИМИ КООРДИНАТАМИ }\AlertTok{\#\#\#}
  \FunctionTok{scale\_x\_continuous}\NormalTok{(}
    \AttributeTok{breaks =} \FunctionTok{c}\NormalTok{(}\DecValTok{32}\NormalTok{, }\DecValTok{38}\NormalTok{, }\DecValTok{44}\NormalTok{, }\DecValTok{48}\NormalTok{),                    }
    \AttributeTok{labels =} \FunctionTok{c}\NormalTok{(}\StringTok{"32°E"}\NormalTok{, }\StringTok{"38°E"}\NormalTok{, }\StringTok{"44°E"}\NormalTok{, }\StringTok{"48°E"}\NormalTok{),    }
    \AttributeTok{name =} \ConstantTok{NULL}
\NormalTok{  ) }\SpecialCharTok{+}
  \FunctionTok{scale\_y\_continuous}\NormalTok{(}
    \AttributeTok{breaks =} \FunctionTok{c}\NormalTok{(}\FloatTok{68.5}\NormalTok{, }\FloatTok{69.5}\NormalTok{, }\FloatTok{70.5}\NormalTok{, }\FloatTok{71.5}\NormalTok{),          }
    \AttributeTok{labels =} \FunctionTok{c}\NormalTok{(}\StringTok{"68.5°N"}\NormalTok{, }\StringTok{"69.5°N"}\NormalTok{, }\StringTok{"70.5°N"}\NormalTok{, }\StringTok{"71.5°N"}\NormalTok{),}
    \AttributeTok{name =} \ConstantTok{NULL}
\NormalTok{  ) }\SpecialCharTok{+}
  
  \CommentTok{\# Область отображения}
  \FunctionTok{coord\_sf}\NormalTok{(}\AttributeTok{xlim =} \FunctionTok{c}\NormalTok{(xmin, xmax), }\AttributeTok{ylim =} \FunctionTok{c}\NormalTok{(ymin, ymax)) }\SpecialCharTok{+}
  
  \CommentTok{\# Оформление}
  \FunctionTok{theme\_minimal}\NormalTok{() }\SpecialCharTok{+}
  \FunctionTok{theme}\NormalTok{(}
    \AttributeTok{axis.text.x =} \FunctionTok{element\_text}\NormalTok{(}\AttributeTok{size =} \DecValTok{8}\NormalTok{),}
    \AttributeTok{axis.text.y =} \FunctionTok{element\_text}\NormalTok{(}\AttributeTok{size =} \DecValTok{8}\NormalTok{, }\AttributeTok{angle =} \DecValTok{90}\NormalTok{, }\AttributeTok{hjust =} \FloatTok{0.5}\NormalTok{),}
    \AttributeTok{panel.grid =} \FunctionTok{element\_line}\NormalTok{(}\AttributeTok{color =} \StringTok{"grey90"}\NormalTok{),}
    \AttributeTok{legend.position =} \StringTok{"bottom"}\NormalTok{,}
    \AttributeTok{legend.box =} \StringTok{"horizontal"}\NormalTok{,  }\CommentTok{\# Размещение легенд в одну строку}
    \AttributeTok{panel.border =} \FunctionTok{element\_rect}\NormalTok{(}\AttributeTok{fill =} \ConstantTok{NA}\NormalTok{, }\AttributeTok{color =} \StringTok{"black"}\NormalTok{, }\AttributeTok{size =} \FloatTok{0.5}\NormalTok{),}
    \AttributeTok{strip.background =} \FunctionTok{element\_rect}\NormalTok{(}\AttributeTok{fill =} \StringTok{"white"}\NormalTok{, }\AttributeTok{color =} \StringTok{"black"}\NormalTok{)}
\NormalTok{  ) }\SpecialCharTok{+}
  \CommentTok{\# Управление легендами}
  \FunctionTok{guides}\NormalTok{(}
    \AttributeTok{fill =} \FunctionTok{guide\_colorbar}\NormalTok{(}\AttributeTok{title.position =} \StringTok{"top"}\NormalTok{, }\AttributeTok{title.hjust =} \FloatTok{0.5}\NormalTok{),}
    \AttributeTok{size =} \FunctionTok{guide\_legend}\NormalTok{(}\AttributeTok{title.position =} \StringTok{"top"}\NormalTok{, }\AttributeTok{title.hjust =} \FloatTok{0.5}\NormalTok{)}
\NormalTok{  )}
\end{Highlighting}
\end{Shaded}

\section{Карты для ``главы Материал и
методы''}\label{ux43aux430ux440ux442ux44b-ux434ux43bux44f-ux433ux43bux430ux432ux44b-ux43cux430ux442ux435ux440ux438ux430ux43b-ux438-ux43cux435ux442ux43eux434ux44b}

\begin{figure}[H]

{\centering \includegraphics[width=0.8\linewidth,height=\textheight,keepaspectratio]{images/KARTOGRAPH13.jpg}

}

\caption{Рис. 13.: Карты для ``главы Материал и методы''}

\end{figure}%

\begin{Shaded}
\begin{Highlighting}[]
\CommentTok{\# Очистка окружения и установка рабочей директории}
\FunctionTok{rm}\NormalTok{(}\AttributeTok{list =} \FunctionTok{ls}\NormalTok{()) }\CommentTok{\# Удаление всех объектов из глобального окружения}
\FunctionTok{setwd}\NormalTok{(}\StringTok{"C:/COURSES/KARTOGRAPH/"}\NormalTok{) }\CommentTok{\# Установка рабочей директории}

\CommentTok{\# {-}{-}{-}{-}{-}{-}{-}{-}{-}{-}{-}{-}{-}{-}{-}{-}{-}}
\CommentTok{\# ЗАГРУЗКА ПАКЕТОВ}
\CommentTok{\# {-}{-}{-}{-}{-}{-}{-}{-}{-}{-}{-}{-}{-}{-}{-}{-}{-}}
\FunctionTok{library}\NormalTok{(sf)          }\CommentTok{\# Пространственные операции с векторными данными}
\FunctionTok{library}\NormalTok{(marmap)      }\CommentTok{\# Работа с батиметрическими данными (карты глубин)}
\FunctionTok{library}\NormalTok{(tidyverse)   }\CommentTok{\# Коллекция пакетов для обработки данных (dplyr, ggplot2 и др.)}
\FunctionTok{library}\NormalTok{(rnaturalearth) }\CommentTok{\# Векторные картографические данные (границы, береговые линии)}
\FunctionTok{library}\NormalTok{(ggspatial)   }\CommentTok{\# Инструменты для пространственной визуализации в ggplot}
\FunctionTok{library}\NormalTok{(readxl)      }\CommentTok{\# Импорт данных из Excel{-}файлов}

\CommentTok{\# {-}{-}{-}{-}{-}{-}{-}{-}{-}{-}{-}{-}{-}{-}{-}{-}{-}}
\CommentTok{\# ЗАГРУЗКА ДАННЫХ}
\CommentTok{\# {-}{-}{-}{-}{-}{-}{-}{-}{-}{-}{-}{-}{-}{-}{-}{-}{-}}
\CommentTok{\# Чтение данных из Excel{-}листа}
\NormalTok{DATA }\OtherTok{\textless{}{-}}\NormalTok{ readxl}\SpecialCharTok{::}\FunctionTok{read\_excel}\NormalTok{(}\StringTok{"KARTOGRAPHIC.xlsx"}\NormalTok{, }\AttributeTok{sheet =} \StringTok{"SURVEY"}\NormalTok{)}

\CommentTok{\# Фильтрация данных:}
\NormalTok{SUMMER }\OtherTok{\textless{}{-}}\NormalTok{ DATA[DATA}\SpecialCharTok{$}\NormalTok{SURV }\SpecialCharTok{==} \StringTok{"SUM"} \SpecialCharTok{\&}\NormalTok{ DATA}\SpecialCharTok{$}\NormalTok{YEAR }\SpecialCharTok{==} \DecValTok{2024}\NormalTok{, ] }\CommentTok{\# Летние исследования 2024}
\NormalTok{CRAB }\OtherTok{\textless{}{-}}\NormalTok{ DATA[DATA}\SpecialCharTok{$}\NormalTok{SURV }\SpecialCharTok{==} \StringTok{"CRAB"} \SpecialCharTok{\&}\NormalTok{ DATA}\SpecialCharTok{$}\NormalTok{YEAR }\SpecialCharTok{==} \DecValTok{2024}\NormalTok{, ]   }\CommentTok{\# Крабовые исследования 2024}

\CommentTok{\# {-}{-}{-}{-}{-}{-}{-}{-}{-}{-}{-}{-}{-}{-}{-}{-}{-}}
\CommentTok{\# ПОДГОТОВКА КАРТОГРАФИЧЕСКОЙ ОСНОВЫ}
\CommentTok{\# {-}{-}{-}{-}{-}{-}{-}{-}{-}{-}{-}{-}{-}{-}{-}{-}{-}}
\CommentTok{\# Загрузка векторных границ России}
\NormalTok{russia\_map }\OtherTok{\textless{}{-}} \FunctionTok{ne\_states}\NormalTok{(}\AttributeTok{country =} \StringTok{"russia"}\NormalTok{, }\AttributeTok{returnclass =} \StringTok{"sf"}\NormalTok{)}

\CommentTok{\# Загрузка береговой линии мирового океана}
\NormalTok{coast }\OtherTok{\textless{}{-}} \FunctionTok{ne\_coastline}\NormalTok{(}\AttributeTok{scale =} \DecValTok{10}\NormalTok{, }\AttributeTok{returnclass =} \StringTok{"sf"}\NormalTok{)}

\CommentTok{\# Создание сетки для навигации (5° по долготе, 1° по широте)}
\NormalTok{ga\_grid }\OtherTok{\textless{}{-}}\NormalTok{ russia\_map }\SpecialCharTok{\%\textgreater{}\%} 
  \FunctionTok{st\_make\_grid}\NormalTok{(}\AttributeTok{cellsize =} \FunctionTok{c}\NormalTok{(}\DecValTok{5}\NormalTok{, }\DecValTok{1}\NormalTok{), }\AttributeTok{offset =} \FunctionTok{c}\NormalTok{(}\DecValTok{30}\NormalTok{, }\DecValTok{67}\NormalTok{))}

\CommentTok{\# Установка границ региона интереса}
\NormalTok{xmin }\OtherTok{\textless{}{-}} \DecValTok{30}\NormalTok{; xmax }\OtherTok{\textless{}{-}} \DecValTok{58}
\NormalTok{ymin }\OtherTok{\textless{}{-}} \DecValTok{67}\NormalTok{; ymax }\OtherTok{\textless{}{-}} \FloatTok{72.5}

\CommentTok{\# {-}{-}{-}{-}{-}{-}{-}{-}{-}{-}{-}{-}{-}{-}{-}{-}{-}}
\CommentTok{\# БАТИМЕТРИЧЕСКИЕ ДАННЫЕ}
\CommentTok{\# {-}{-}{-}{-}{-}{-}{-}{-}{-}{-}{-}{-}{-}{-}{-}{-}{-}}
\CommentTok{\# Загрузка данных о глубинах из базы NOAA}
\NormalTok{bat }\OtherTok{\textless{}{-}} \FunctionTok{getNOAA.bathy}\NormalTok{(}
  \AttributeTok{lon1 =}\NormalTok{ xmin, }\AttributeTok{lon2 =}\NormalTok{ xmax,}
  \AttributeTok{lat1 =}\NormalTok{ ymin, }\AttributeTok{lat2 =}\NormalTok{ ymax,}
  \AttributeTok{resolution =} \DecValTok{1}\NormalTok{,   }\CommentTok{\# Разрешение данных (1 минута дуги)}
  \AttributeTok{keep =} \ConstantTok{TRUE}       \CommentTok{\# Сохранить кэш на диске}
\NormalTok{)}

\CommentTok{\# Преобразование в таблицу XYZ (долгота, широта, глубина)}
\NormalTok{bat\_xyz }\OtherTok{\textless{}{-}} \FunctionTok{as.xyz}\NormalTok{(bat)}

\CommentTok{\# Создание цветовой схемы для глубин:}
\NormalTok{breaks }\OtherTok{\textless{}{-}} \FunctionTok{c}\NormalTok{(}\SpecialCharTok{{-}}\DecValTok{10000}\NormalTok{, }\SpecialCharTok{{-}}\DecValTok{7000}\NormalTok{, }\SpecialCharTok{{-}}\DecValTok{6000}\NormalTok{, }\SpecialCharTok{{-}}\DecValTok{5000}\NormalTok{, }\SpecialCharTok{{-}}\DecValTok{4000}\NormalTok{, }\SpecialCharTok{{-}}\DecValTok{3000}\NormalTok{, }\SpecialCharTok{{-}}\DecValTok{2000}\NormalTok{, }\SpecialCharTok{{-}}\DecValTok{1000}\NormalTok{, }
            \SpecialCharTok{{-}}\DecValTok{500}\NormalTok{, }\SpecialCharTok{{-}}\DecValTok{200}\NormalTok{, }\SpecialCharTok{{-}}\DecValTok{50}\NormalTok{, }\SpecialCharTok{{-}}\DecValTok{1}\NormalTok{, }\DecValTok{5}\NormalTok{, }\DecValTok{50}\NormalTok{, }\DecValTok{100}\NormalTok{, }\DecValTok{150}\NormalTok{, }\DecValTok{200}\NormalTok{, }\DecValTok{300}\NormalTok{, }\DecValTok{400}\NormalTok{, }\DecValTok{500}\NormalTok{, }\DecValTok{1000}\NormalTok{, }\DecValTok{3000}\NormalTok{)}
\NormalTok{cols }\OtherTok{\textless{}{-}} \FunctionTok{c}\NormalTok{(}
  \StringTok{"\#5e99d6"}\NormalTok{, }\StringTok{"\#669cd4"}\NormalTok{, }\StringTok{"\#6c9fd4"}\NormalTok{, }\StringTok{"\#96bce3"}\NormalTok{, }\StringTok{"\#AEC8E3"}\NormalTok{, }\StringTok{"\#a6c4e3"}\NormalTok{,}
  \StringTok{"\#AEC8E3"}\NormalTok{, }\StringTok{"\#BBD0EB"}\NormalTok{, }\StringTok{"\#C7DCF1"}\NormalTok{, }\StringTok{"\#DAECFA"}\NormalTok{, }\StringTok{"\#D2E5F6"}\NormalTok{, }\StringTok{"\#e1f2d8"}\NormalTok{,}
  \StringTok{"\#B8D3AA"}\NormalTok{, }\StringTok{"\#b3b387"}\NormalTok{, }\StringTok{"\#9EC187"}\NormalTok{, }\StringTok{"\#C7D097"}\NormalTok{, }\StringTok{"\#DADBAF"}\NormalTok{, }\StringTok{"\#F3F0C7"}\NormalTok{,}
  \StringTok{"\#E6DBA8"}\NormalTok{, }\StringTok{"\#DACFA1"}\NormalTok{, }\StringTok{"\#D1BF81"}\NormalTok{, }\StringTok{"\#C69D45"}
\NormalTok{)}

\CommentTok{\# Категоризация глубин для визуализации}
\NormalTok{bat\_xyz}\SpecialCharTok{$}\NormalTok{V4 }\OtherTok{\textless{}{-}} \FunctionTok{cut}\NormalTok{(bat\_xyz}\SpecialCharTok{$}\NormalTok{V3, }\AttributeTok{breaks =}\NormalTok{ breaks)}
\NormalTok{niveles }\OtherTok{\textless{}{-}} \FunctionTok{levels}\NormalTok{(bat\_xyz}\SpecialCharTok{$}\NormalTok{V4)  }\CommentTok{\# Сохранение уровней для легенды}

\CommentTok{\# {-}{-}{-}{-}{-}{-}{-}{-}{-}{-}{-}{-}{-}{-}{-}{-}{-}}
\CommentTok{\# ПОСТРОЕНИЕ БАЗОВОЙ КАРТЫ}
\CommentTok{\# {-}{-}{-}{-}{-}{-}{-}{-}{-}{-}{-}{-}{-}{-}{-}{-}{-}}
\NormalTok{map }\OtherTok{\textless{}{-}} \FunctionTok{ggplot}\NormalTok{() }\SpecialCharTok{+}
  \CommentTok{\# Векторные границы России}
  \FunctionTok{geom\_sf}\NormalTok{(}\AttributeTok{data =}\NormalTok{ russia\_map) }\SpecialCharTok{+}
  \CommentTok{\# Батиметрическая подложка (цвет = глубина)}
  \FunctionTok{geom\_tile}\NormalTok{(}\AttributeTok{data =}\NormalTok{ bat\_xyz, }\FunctionTok{aes}\NormalTok{(}\AttributeTok{x =}\NormalTok{ V1, }\AttributeTok{y =}\NormalTok{ V2, }\AttributeTok{fill =}\NormalTok{ V4), }\AttributeTok{show.legend =} \ConstantTok{FALSE}\NormalTok{) }\SpecialCharTok{+}
  \CommentTok{\# Цветовая схема для глубин}
  \FunctionTok{scale\_fill\_manual}\NormalTok{(}\AttributeTok{name =} \StringTok{"Глубина"}\NormalTok{, }\AttributeTok{values =}\NormalTok{ cols, }\AttributeTok{breaks =}\NormalTok{ niveles) }\SpecialCharTok{+}
  \CommentTok{\# Наложение сетки}
  \FunctionTok{geom\_sf}\NormalTok{(}\AttributeTok{data =}\NormalTok{ ga\_grid, }\AttributeTok{alpha =} \FloatTok{0.01}\NormalTok{, }\AttributeTok{linetype =} \DecValTok{3}\NormalTok{) }\SpecialCharTok{+}
  \CommentTok{\# Береговая линия}
  \FunctionTok{geom\_sf}\NormalTok{(}\AttributeTok{data =}\NormalTok{ coast, }\AttributeTok{linewidth =} \FloatTok{0.2}\NormalTok{, }\AttributeTok{fill =} \ConstantTok{NA}\NormalTok{) }\SpecialCharTok{+}
  \CommentTok{\# Ограничение области карты}
  \FunctionTok{coord\_sf}\NormalTok{(}\AttributeTok{xlim =} \FunctionTok{c}\NormalTok{(}\DecValTok{32}\NormalTok{, }\DecValTok{56}\NormalTok{), }\AttributeTok{ylim =} \FunctionTok{c}\NormalTok{(}\FloatTok{68.5}\NormalTok{, }\FloatTok{72.3}\NormalTok{)) }\SpecialCharTok{+}
  \CommentTok{\# Масштабная линейка (top{-}left)}
  \FunctionTok{annotation\_scale}\NormalTok{(}\AttributeTok{location =} \StringTok{"tl"}\NormalTok{, }\AttributeTok{width\_hint =} \FloatTok{0.2}\NormalTok{) }\SpecialCharTok{+}
  \CommentTok{\# Оформление}
  \FunctionTok{labs}\NormalTok{(}\AttributeTok{x =} \ConstantTok{NULL}\NormalTok{, }\AttributeTok{y =} \ConstantTok{NULL}\NormalTok{, }\AttributeTok{fill =} \StringTok{"Глубина (м)"}\NormalTok{) }\SpecialCharTok{+}
  \FunctionTok{theme}\NormalTok{(}\AttributeTok{panel.border =} \FunctionTok{element\_rect}\NormalTok{(}\AttributeTok{colour =} \StringTok{"black"}\NormalTok{, }\AttributeTok{fill =} \ConstantTok{NA}\NormalTok{, }\AttributeTok{linewidth =} \DecValTok{1}\NormalTok{))}

\CommentTok{\# {-}{-}{-}{-}{-}{-}{-}{-}{-}{-}{-}{-}{-}{-}{-}{-}{-}}
\CommentTok{\# ДОБАВЛЕНИЕ АННОТАЦИЙ}
\CommentTok{\# {-}{-}{-}{-}{-}{-}{-}{-}{-}{-}{-}{-}{-}{-}{-}{-}{-}}
\NormalTok{map }\OtherTok{\textless{}{-}}\NormalTok{ map }\SpecialCharTok{+}
  \FunctionTok{annotate}\NormalTok{(}\StringTok{"text"}\NormalTok{, }\AttributeTok{x =} \DecValTok{40}\NormalTok{, }\AttributeTok{y =} \FloatTok{72.1}\NormalTok{, }\AttributeTok{size =} \DecValTok{5}\NormalTok{, }
           \AttributeTok{label =} \StringTok{"Баренцево море"}\NormalTok{, }\AttributeTok{fontface =} \StringTok{"bold"}\NormalTok{) }\SpecialCharTok{+}
  \FunctionTok{annotate}\NormalTok{(}\StringTok{"text"}\NormalTok{, }\AttributeTok{x =} \FloatTok{52.2}\NormalTok{, }\AttributeTok{y =} \FloatTok{69.1}\NormalTok{, }\AttributeTok{size =} \DecValTok{4}\NormalTok{, }
           \AttributeTok{label =} \StringTok{"о. Колгуев"}\NormalTok{, }\AttributeTok{fontface =} \StringTok{"bold"}\NormalTok{) }\SpecialCharTok{+}
  \FunctionTok{annotate}\NormalTok{(}\StringTok{"text"}\NormalTok{, }\AttributeTok{x =} \DecValTok{33}\NormalTok{, }\AttributeTok{y =} \FloatTok{68.9}\NormalTok{, }\AttributeTok{size =} \DecValTok{4}\NormalTok{, }
           \AttributeTok{label =} \StringTok{"Кольский"}\NormalTok{, }\AttributeTok{fontface =} \StringTok{"bold"}\NormalTok{) }\SpecialCharTok{+}
  \FunctionTok{annotate}\NormalTok{(}\StringTok{"text"}\NormalTok{, }\AttributeTok{x =} \DecValTok{33}\NormalTok{, }\AttributeTok{y =} \FloatTok{68.6}\NormalTok{, }\AttributeTok{size =} \DecValTok{4}\NormalTok{, }
           \AttributeTok{label =} \StringTok{"п{-}ов"}\NormalTok{, }\AttributeTok{fontface =} \StringTok{"bold"}\NormalTok{)}

\CommentTok{\# {-}{-}{-}{-}{-}{-}{-}{-}{-}{-}{-}{-}{-}{-}{-}{-}{-}}
\CommentTok{\# ДОБАВЛЕНИЕ ТОЧЕК НАБЛЮДЕНИЙ}
\CommentTok{\# {-}{-}{-}{-}{-}{-}{-}{-}{-}{-}{-}{-}{-}{-}{-}{-}{-}}
\NormalTok{map }\OtherTok{\textless{}{-}}\NormalTok{ map }\SpecialCharTok{+}
  \CommentTok{\# Точки исследований краба (синие)}
  \FunctionTok{geom\_point}\NormalTok{(}
    \AttributeTok{data =}\NormalTok{ CRAB, }
    \FunctionTok{aes}\NormalTok{(}\AttributeTok{x =}\NormalTok{ X }\SpecialCharTok{+} \FloatTok{0.2}\NormalTok{, }\AttributeTok{y =}\NormalTok{ Y), }\CommentTok{\# Смещение для визуального разделения}
    \AttributeTok{size =} \DecValTok{3}\NormalTok{, }\AttributeTok{color =} \StringTok{"black"}\NormalTok{, }\AttributeTok{fill =} \StringTok{"\#1E90FF"}\NormalTok{, }
    \AttributeTok{shape =} \DecValTok{21}\NormalTok{, }\AttributeTok{alpha =} \DecValTok{1}
\NormalTok{  ) }\SpecialCharTok{+}
  \CommentTok{\# Точки летних исследований (оранжевые)}
  \FunctionTok{geom\_point}\NormalTok{(}
    \AttributeTok{data =}\NormalTok{ SUMMER, }
    \FunctionTok{aes}\NormalTok{(}\AttributeTok{x =}\NormalTok{ X, }\AttributeTok{y =}\NormalTok{ Y), }
    \AttributeTok{size =} \DecValTok{3}\NormalTok{, }\AttributeTok{color =} \StringTok{"black"}\NormalTok{, }\AttributeTok{fill =} \StringTok{"\#FFA500"}\NormalTok{, }
    \AttributeTok{shape =} \DecValTok{21}\NormalTok{, }\AttributeTok{alpha =} \DecValTok{1}
\NormalTok{  )}

\CommentTok{\# Вывод финальной карты}
\FunctionTok{print}\NormalTok{(map)}

\CommentTok{\# {-}{-}{-}{-}{-}{-}{-}{-}{-}{-}{-}{-}{-}{-}{-}{-}{-}}
\CommentTok{\# СОХРАНЕНИЕ РЕЗУЛЬТАТА}
\CommentTok{\# {-}{-}{-}{-}{-}{-}{-}{-}{-}{-}{-}{-}{-}{-}{-}{-}{-}}
\FunctionTok{ggsave}\NormalTok{(}\StringTok{"DATA\_MAP.jpg"}\NormalTok{, }
       \AttributeTok{plot =}\NormalTok{ map,          }\CommentTok{\# Используем явное указание объекта}
       \AttributeTok{device =} \StringTok{"jpeg"}\NormalTok{, }
       \AttributeTok{dpi =} \DecValTok{600}\NormalTok{,           }\CommentTok{\# Высокое разрешение}
       \AttributeTok{width =} \DecValTok{7}\NormalTok{,           }\CommentTok{\# Ширина в дюймах}
       \AttributeTok{height =} \DecValTok{5}\NormalTok{,          }\CommentTok{\# Высота в дюймах}
       \AttributeTok{units =} \StringTok{"in"}\NormalTok{)}
\end{Highlighting}
\end{Shaded}

\section{Карты с картой-врезкой и
маршрутом}\label{ux43aux430ux440ux442ux44b-ux441-ux43aux430ux440ux442ux43eux439-ux432ux440ux435ux437ux43aux43eux439-ux438-ux43cux430ux440ux448ux440ux443ux442ux43eux43c}

\begin{figure}[H]

{\centering \includegraphics[width=0.8\linewidth,height=\textheight,keepaspectratio]{images/KARTOGRAPH14.jpg}

}

\caption{Рис. 14.: Карты с картой-врезкой и маршрутом}

\end{figure}%

\begin{Shaded}
\begin{Highlighting}[]
\CommentTok{\# Очистка окружения и установка рабочей директории}
\FunctionTok{rm}\NormalTok{(}\AttributeTok{list =} \FunctionTok{ls}\NormalTok{()) }\CommentTok{\# Удаление всех объектов из глобального окружения}
\FunctionTok{setwd}\NormalTok{(}\StringTok{"C:/COURSES/KARTOGRAPH/"}\NormalTok{) }\CommentTok{\# Установка рабочей директории}

\CommentTok{\# {-}{-}{-}{-}{-}{-}{-}{-}{-}{-}{-}{-}{-}{-}{-}{-}{-}}
\CommentTok{\# ЗАГРУЗКА ПАКЕТОВ}
\CommentTok{\# {-}{-}{-}{-}{-}{-}{-}{-}{-}{-}{-}{-}{-}{-}{-}{-}{-}}
\FunctionTok{library}\NormalTok{(sf)          }\CommentTok{\# Пространственные операции с векторными данными}
\FunctionTok{library}\NormalTok{(marmap)      }\CommentTok{\# Работа с батиметрическими данными (карты глубин)}
\FunctionTok{library}\NormalTok{(tidyverse)   }\CommentTok{\# Коллекция пакетов для обработки данных}
\FunctionTok{library}\NormalTok{(rnaturalearth) }\CommentTok{\# Векторные картографические данные}
\FunctionTok{library}\NormalTok{(ggspatial)   }\CommentTok{\# Инструменты для пространственной визуализации}
\FunctionTok{library}\NormalTok{(readxl)      }\CommentTok{\# Импорт данных из Excel}
\FunctionTok{library}\NormalTok{(ggOceanMaps) }\CommentTok{\# Специализированные карты океанов}
\FunctionTok{library}\NormalTok{(cowplot)     }\CommentTok{\# Компоновка графиков и добавление элементов}

\CommentTok{\# {-}{-}{-}{-}{-}{-}{-}{-}{-}{-}{-}{-}{-}{-}{-}{-}{-}}
\CommentTok{\# ЗАГРУЗКА ДАННЫХ}
\CommentTok{\# {-}{-}{-}{-}{-}{-}{-}{-}{-}{-}{-}{-}{-}{-}{-}{-}{-}}
\CommentTok{\# Чтение данных из Excel}
\NormalTok{DATA }\OtherTok{\textless{}{-}}\NormalTok{ readxl}\SpecialCharTok{::}\FunctionTok{read\_excel}\NormalTok{(}\StringTok{"KARTOGRAPHIC.xlsx"}\NormalTok{, }\AttributeTok{sheet =} \StringTok{"SURVEY"}\NormalTok{)}

\CommentTok{\# Фильтрация данных (крабовые исследования 2022)}
\NormalTok{DATA }\OtherTok{\textless{}{-}}\NormalTok{ DATA[DATA}\SpecialCharTok{$}\NormalTok{SURV }\SpecialCharTok{==} \StringTok{"CRAB"} \SpecialCharTok{\&}\NormalTok{ DATA}\SpecialCharTok{$}\NormalTok{YEAR }\SpecialCharTok{==} \DecValTok{2022}\NormalTok{, ]}

\CommentTok{\# Загрузка векторных границ России}
\NormalTok{russia\_map }\OtherTok{\textless{}{-}} \FunctionTok{ne\_states}\NormalTok{(}\AttributeTok{country =} \StringTok{"russia"}\NormalTok{, }\AttributeTok{returnclass =} \StringTok{"sf"}\NormalTok{)}

\CommentTok{\# Установка границ региона интереса}
\NormalTok{xmin }\OtherTok{\textless{}{-}} \DecValTok{35}\NormalTok{; xmax }\OtherTok{\textless{}{-}} \DecValTok{50}
\NormalTok{ymin }\OtherTok{\textless{}{-}} \FloatTok{67.2}\NormalTok{; ymax }\OtherTok{\textless{}{-}} \DecValTok{71}

\CommentTok{\# {-}{-}{-}{-}{-}{-}{-}{-}{-}{-}{-}{-}{-}{-}{-}{-}{-}}
\CommentTok{\# БАТИМЕТРИЧЕСКИЕ ДАННЫЕ}
\CommentTok{\# {-}{-}{-}{-}{-}{-}{-}{-}{-}{-}{-}{-}{-}{-}{-}{-}{-}}
\CommentTok{\# Загрузка данных о глубинах}
\NormalTok{bat }\OtherTok{\textless{}{-}} \FunctionTok{getNOAA.bathy}\NormalTok{(xmin, xmax, ymin, ymax, }\AttributeTok{resolution =} \DecValTok{1}\NormalTok{, }\AttributeTok{keep =} \ConstantTok{TRUE}\NormalTok{)}
\NormalTok{bat\_xyz }\OtherTok{\textless{}{-}} \FunctionTok{as.xyz}\NormalTok{(bat)}

\CommentTok{\# Определение цветовых уровней для глубин}
\NormalTok{breaks }\OtherTok{\textless{}{-}} \FunctionTok{c}\NormalTok{(}\SpecialCharTok{{-}}\DecValTok{10000}\NormalTok{, }\SpecialCharTok{{-}}\DecValTok{7000}\NormalTok{, }\SpecialCharTok{{-}}\DecValTok{6000}\NormalTok{, }\SpecialCharTok{{-}}\DecValTok{5000}\NormalTok{, }\SpecialCharTok{{-}}\DecValTok{4000}\NormalTok{, }\SpecialCharTok{{-}}\DecValTok{3000}\NormalTok{, }\SpecialCharTok{{-}}\DecValTok{2000}\NormalTok{, }\SpecialCharTok{{-}}\DecValTok{1000}\NormalTok{, }
            \SpecialCharTok{{-}}\DecValTok{500}\NormalTok{, }\SpecialCharTok{{-}}\DecValTok{200}\NormalTok{, }\SpecialCharTok{{-}}\DecValTok{50}\NormalTok{, }\SpecialCharTok{{-}}\DecValTok{1}\NormalTok{, }\DecValTok{5}\NormalTok{, }\DecValTok{50}\NormalTok{, }\DecValTok{100}\NormalTok{, }\DecValTok{150}\NormalTok{, }\DecValTok{200}\NormalTok{, }\DecValTok{300}\NormalTok{, }\DecValTok{400}\NormalTok{, }\DecValTok{500}\NormalTok{, }\DecValTok{1000}\NormalTok{, }\DecValTok{3000}\NormalTok{)}
\NormalTok{cols }\OtherTok{\textless{}{-}} \FunctionTok{c}\NormalTok{(}
  \StringTok{"\#5e99d6"}\NormalTok{, }\StringTok{"\#669cd4"}\NormalTok{, }\StringTok{"\#6c9fd4"}\NormalTok{, }\StringTok{"\#96bce3"}\NormalTok{, }\StringTok{"\#AEC8E3"}\NormalTok{, }\StringTok{"\#a6c4e3"}\NormalTok{,}
  \StringTok{"\#AEC8E3"}\NormalTok{, }\StringTok{"\#BBD0EB"}\NormalTok{, }\StringTok{"\#C7DCF1"}\NormalTok{, }\StringTok{"\#DAECFA"}\NormalTok{, }\StringTok{"\#D2E5F6"}\NormalTok{, }\StringTok{"\#e1f2d8"}\NormalTok{,}
  \StringTok{"\#B8D3AA"}\NormalTok{, }\StringTok{"\#b3b387"}\NormalTok{, }\StringTok{"\#9EC187"}\NormalTok{, }\StringTok{"\#C7D097"}\NormalTok{, }\StringTok{"\#DADBAF"}\NormalTok{, }\StringTok{"\#F3F0C7"}\NormalTok{,}
  \StringTok{"\#E6DBA8"}\NormalTok{, }\StringTok{"\#DACFA1"}\NormalTok{, }\StringTok{"\#D1BF81"}\NormalTok{, }\StringTok{"\#C69D45"}
\NormalTok{)}

\CommentTok{\# Категоризация глубин}
\NormalTok{bat\_xyz}\SpecialCharTok{$}\NormalTok{V4 }\OtherTok{\textless{}{-}} \FunctionTok{cut}\NormalTok{(bat\_xyz}\SpecialCharTok{$}\NormalTok{V3, }\AttributeTok{breaks =}\NormalTok{ breaks)}
\NormalTok{niveles }\OtherTok{\textless{}{-}} \FunctionTok{levels}\NormalTok{(bat\_xyz}\SpecialCharTok{$}\NormalTok{V4)}

\CommentTok{\# Создание координатной сетки}
\NormalTok{ga\_grid }\OtherTok{\textless{}{-}}\NormalTok{ russia\_map }\SpecialCharTok{\%\textgreater{}\%} 
  \FunctionTok{st\_make\_grid}\NormalTok{(}\AttributeTok{cellsize =} \FunctionTok{c}\NormalTok{(}\DecValTok{2}\NormalTok{, }\FloatTok{0.5}\NormalTok{), }\AttributeTok{offset =} \FunctionTok{c}\NormalTok{(}\DecValTok{34}\NormalTok{, }\DecValTok{67}\NormalTok{))}

\CommentTok{\# {-}{-}{-}{-}{-}{-}{-}{-}{-}{-}{-}{-}{-}{-}{-}{-}{-}}
\CommentTok{\# ПОСТРОЕНИЕ ОСНОВНОЙ КАРТЫ}
\CommentTok{\# {-}{-}{-}{-}{-}{-}{-}{-}{-}{-}{-}{-}{-}{-}{-}{-}{-}}
\NormalTok{map }\OtherTok{\textless{}{-}} \FunctionTok{ggplot}\NormalTok{() }\SpecialCharTok{+}
  \CommentTok{\# Векторные границы России}
  \FunctionTok{geom\_sf}\NormalTok{(}\AttributeTok{data =}\NormalTok{ russia\_map) }\SpecialCharTok{+}
  \CommentTok{\# Батиметрическая подложка}
  \FunctionTok{geom\_tile}\NormalTok{(}\AttributeTok{data =}\NormalTok{ bat\_xyz, }\FunctionTok{aes}\NormalTok{(}\AttributeTok{x =}\NormalTok{ V1, }\AttributeTok{y =}\NormalTok{ V2, }\AttributeTok{fill =}\NormalTok{ V4), }\AttributeTok{show.legend =} \ConstantTok{FALSE}\NormalTok{) }\SpecialCharTok{+}
  \FunctionTok{scale\_fill\_manual}\NormalTok{(}\AttributeTok{values =}\NormalTok{ cols, }\AttributeTok{breaks =}\NormalTok{ niveles) }\SpecialCharTok{+}
  \CommentTok{\# Контур нулевой глубины (береговая линия)}
  \FunctionTok{geom\_contour}\NormalTok{(}\AttributeTok{data =}\NormalTok{ bat\_xyz, }\FunctionTok{aes}\NormalTok{(}\AttributeTok{x =}\NormalTok{ V1, }\AttributeTok{y =}\NormalTok{ V2, }\AttributeTok{z =}\NormalTok{ V3), }
               \AttributeTok{breaks =} \DecValTok{0}\NormalTok{, }\AttributeTok{color =} \StringTok{"black"}\NormalTok{, }\AttributeTok{linewidth =} \FloatTok{0.5}\NormalTok{) }\SpecialCharTok{+} 
  \CommentTok{\# Координатная сетка}
  \FunctionTok{geom\_sf}\NormalTok{(}\AttributeTok{data =}\NormalTok{ ga\_grid, }\AttributeTok{alpha =} \FloatTok{0.01}\NormalTok{, }\AttributeTok{linetype =} \DecValTok{3}\NormalTok{) }\SpecialCharTok{+}
  \CommentTok{\# Ограничение области карты}
  \FunctionTok{coord\_sf}\NormalTok{(}\AttributeTok{xlim =} \FunctionTok{c}\NormalTok{(}\DecValTok{36}\NormalTok{, }\DecValTok{49}\NormalTok{), }\AttributeTok{ylim =} \FunctionTok{c}\NormalTok{(}\FloatTok{67.4}\NormalTok{, }\FloatTok{70.8}\NormalTok{)) }\SpecialCharTok{+} 
  \CommentTok{\# Масштабная линейка}
  \FunctionTok{annotation\_scale}\NormalTok{(}\AttributeTok{location =} \StringTok{"tr"}\NormalTok{, }\AttributeTok{width\_hint =} \FloatTok{0.5}\NormalTok{) }\SpecialCharTok{+}
  \FunctionTok{labs}\NormalTok{(}\AttributeTok{x =} \ConstantTok{NULL}\NormalTok{, }\AttributeTok{y =} \ConstantTok{NULL}\NormalTok{) }\SpecialCharTok{+}
  \CommentTok{\# Географические подписи}
  \FunctionTok{annotate}\NormalTok{(}\StringTok{"text"}\NormalTok{, }\AttributeTok{x =} \DecValTok{47}\NormalTok{, }\AttributeTok{y =} \FloatTok{70.7}\NormalTok{, }\AttributeTok{size =} \DecValTok{5}\NormalTok{, }
           \AttributeTok{label =} \StringTok{"Баренцево море"}\NormalTok{, }\AttributeTok{fontface =} \StringTok{"bold"}\NormalTok{) }\SpecialCharTok{+}
  \FunctionTok{annotate}\NormalTok{(}\StringTok{"text"}\NormalTok{, }\AttributeTok{x =} \FloatTok{48.4}\NormalTok{, }\AttributeTok{y =} \FloatTok{68.62}\NormalTok{, }\AttributeTok{size =} \DecValTok{4}\NormalTok{,}
           \AttributeTok{label =} \StringTok{"о. Колгуев"}\NormalTok{, }\AttributeTok{fontface =} \StringTok{"bold"}\NormalTok{) }\SpecialCharTok{+}
  \FunctionTok{annotate}\NormalTok{(}\StringTok{"text"}\NormalTok{, }\AttributeTok{x =} \FloatTok{37.5}\NormalTok{, }\AttributeTok{y =} \FloatTok{67.7}\NormalTok{, }\AttributeTok{size =} \DecValTok{5}\NormalTok{,}
           \AttributeTok{label =} \StringTok{"Кольский п{-}ов"}\NormalTok{, }\AttributeTok{fontface =} \StringTok{"bold"}\NormalTok{) }\SpecialCharTok{+}
  \CommentTok{\# Маршрут и точки исследований}
  \FunctionTok{geom\_path}\NormalTok{(}\AttributeTok{data =}\NormalTok{ DATA, }\FunctionTok{aes}\NormalTok{(}\AttributeTok{x =}\NormalTok{ X, }\AttributeTok{y =}\NormalTok{ Y), }\AttributeTok{color =} \StringTok{"black"}\NormalTok{) }\SpecialCharTok{+}
  \FunctionTok{geom\_point}\NormalTok{(}\AttributeTok{data =}\NormalTok{ DATA, }\FunctionTok{aes}\NormalTok{(}\AttributeTok{x =}\NormalTok{ X, }\AttributeTok{y =}\NormalTok{ Y), }
             \AttributeTok{size =} \DecValTok{3}\NormalTok{, }\AttributeTok{color =} \StringTok{"black"}\NormalTok{, }\AttributeTok{fill =} \StringTok{"white"}\NormalTok{, }
             \AttributeTok{shape =} \DecValTok{21}\NormalTok{, }\AttributeTok{alpha =} \FloatTok{0.8}\NormalTok{) }\SpecialCharTok{+}
  \CommentTok{\# ДОБАВЛЕНИЕ РАМКИ {-} ключевое изменение}
  \FunctionTok{theme}\NormalTok{(}\AttributeTok{panel.border =} \FunctionTok{element\_rect}\NormalTok{(}\AttributeTok{colour =} \StringTok{"black"}\NormalTok{, }\AttributeTok{fill =} \ConstantTok{NA}\NormalTok{, }\AttributeTok{linewidth =} \FloatTok{1.5}\NormalTok{))}

\CommentTok{\# {-}{-}{-}{-}{-}{-}{-}{-}{-}{-}{-}{-}{-}{-}{-}{-}{-}}
\CommentTok{\# СОЗДАНИЕ ВСТАВКИ{-}ЛОКАЦИИ}
\CommentTok{\# {-}{-}{-}{-}{-}{-}{-}{-}{-}{-}{-}{-}{-}{-}{-}{-}{-}}
\CommentTok{\# Область для вставки}
\NormalTok{ins }\OtherTok{\textless{}{-}} \FunctionTok{data.frame}\NormalTok{(}\AttributeTok{lon =} \FunctionTok{c}\NormalTok{(}\DecValTok{10}\NormalTok{, }\DecValTok{10}\NormalTok{, }\DecValTok{70}\NormalTok{, }\DecValTok{70}\NormalTok{), }\AttributeTok{lat =} \FunctionTok{c}\NormalTok{(}\DecValTok{67}\NormalTok{, }\DecValTok{80}\NormalTok{, }\DecValTok{80}\NormalTok{, }\DecValTok{67}\NormalTok{))}

\CommentTok{\# Получение данных для вставки}
\NormalTok{mar\_bathy }\OtherTok{\textless{}{-}} \FunctionTok{getNOAA.bathy}\NormalTok{(}\DecValTok{9}\NormalTok{, }\DecValTok{71}\NormalTok{, }\FloatTok{66.5}\NormalTok{, }\DecValTok{83}\NormalTok{, }\AttributeTok{res =} \DecValTok{4}\NormalTok{, }\AttributeTok{keep =} \ConstantTok{TRUE}\NormalTok{)}
\NormalTok{bathy }\OtherTok{\textless{}{-}} \FunctionTok{raster\_bathymetry}\NormalTok{(stars}\SpecialCharTok{::}\FunctionTok{st\_as\_stars}\NormalTok{(marmap}\SpecialCharTok{::}\FunctionTok{as.raster}\NormalTok{(mar\_bathy)), }
                           \AttributeTok{depths =} \ConstantTok{NULL}\NormalTok{, }\AttributeTok{verbose =} \ConstantTok{FALSE}\NormalTok{)}

\CommentTok{\# Построение вставки}
\NormalTok{insetmap }\OtherTok{\textless{}{-}} \FunctionTok{basemap}\NormalTok{(ins, }\AttributeTok{shapefiles =} \FunctionTok{list}\NormalTok{(}\AttributeTok{land =}\NormalTok{ dd\_land, }\AttributeTok{bathy =}\NormalTok{ bathy), }
                   \AttributeTok{bathy.style =} \StringTok{"rub"}\NormalTok{, }\AttributeTok{legends =} \ConstantTok{FALSE}\NormalTok{) }\SpecialCharTok{+}
  \CommentTok{\# Прямоугольник, обозначающий область основной карты}
  \FunctionTok{geom\_rect}\NormalTok{(}\FunctionTok{aes}\NormalTok{(}\AttributeTok{xmin =} \DecValTok{35}\NormalTok{, }\AttributeTok{xmax =} \DecValTok{51}\NormalTok{, }\AttributeTok{ymin =} \FloatTok{67.5}\NormalTok{, }\AttributeTok{ymax =} \DecValTok{71}\NormalTok{), }
            \AttributeTok{fill =} \StringTok{"black"}\NormalTok{, }\AttributeTok{color =} \StringTok{"black"}\NormalTok{, }\AttributeTok{alpha =} \FloatTok{0.2}\NormalTok{) }\SpecialCharTok{+}
  \FunctionTok{labs}\NormalTok{(}\AttributeTok{y =} \ConstantTok{NULL}\NormalTok{, }\AttributeTok{x =} \ConstantTok{NULL}\NormalTok{) }\SpecialCharTok{+}
  \CommentTok{\# Упрощение оформления}
  \FunctionTok{theme}\NormalTok{(}\AttributeTok{axis.text.x =} \FunctionTok{element\_blank}\NormalTok{(), }
        \AttributeTok{axis.text.y =} \FunctionTok{element\_blank}\NormalTok{(),}
        \CommentTok{\# Рамка для вставки}
        \AttributeTok{panel.border =} \FunctionTok{element\_rect}\NormalTok{(}\AttributeTok{colour =} \StringTok{"black"}\NormalTok{, }\AttributeTok{fill =} \ConstantTok{NA}\NormalTok{, }\AttributeTok{linewidth =} \DecValTok{1}\NormalTok{))}

\CommentTok{\# {-}{-}{-}{-}{-}{-}{-}{-}{-}{-}{-}{-}{-}{-}{-}{-}{-}}
\CommentTok{\# ФИНАЛЬНАЯ КОМПОНОВКА С РАМКОЙ}
\CommentTok{\# {-}{-}{-}{-}{-}{-}{-}{-}{-}{-}{-}{-}{-}{-}{-}{-}{-}}
\NormalTok{MAP }\OtherTok{\textless{}{-}} \FunctionTok{ggdraw}\NormalTok{() }\SpecialCharTok{+}
  \CommentTok{\# Основная карта}
  \FunctionTok{draw\_plot}\NormalTok{(map) }\SpecialCharTok{+}
  \CommentTok{\# Вставка с позиционированием}
  \FunctionTok{draw\_plot}\NormalTok{(insetmap,}
            \AttributeTok{height =} \FloatTok{0.3}\NormalTok{,}
            \AttributeTok{x =} \SpecialCharTok{{-}}\FloatTok{0.26}\NormalTok{,}
            \AttributeTok{y =} \FloatTok{0.55}\NormalTok{) }

\CommentTok{\# Вывод финальной карты}
\FunctionTok{print}\NormalTok{(MAP)}

\CommentTok{\# {-}{-}{-}{-}{-}{-}{-}{-}{-}{-}{-}{-}{-}{-}{-}{-}{-}}
\CommentTok{\# СОХРАНЕНИЕ РЕЗУЛЬТАТА}
\CommentTok{\# {-}{-}{-}{-}{-}{-}{-}{-}{-}{-}{-}{-}{-}{-}{-}{-}{-}}
\FunctionTok{ggsave}\NormalTok{(}\StringTok{"DATA\_MAP\_FRAMED.jpg"}\NormalTok{, }
       \AttributeTok{plot =}\NormalTok{ MAP,}
       \AttributeTok{device =} \StringTok{"jpeg"}\NormalTok{, }
       \AttributeTok{dpi =} \DecValTok{600}\NormalTok{,}
       \AttributeTok{width =} \DecValTok{7}\NormalTok{,}
       \AttributeTok{height =} \DecValTok{6}\NormalTok{,}
       \AttributeTok{units =} \StringTok{"in"}\NormalTok{)}
\end{Highlighting}
\end{Shaded}

\bookmarksetup{startatroot}

\chapter{sdmTMB - оценка и визуализация индекса обилия по
съемке}\label{sdmtmb---ux43eux446ux435ux43dux43aux430-ux438-ux432ux438ux437ux443ux430ux43bux438ux437ux430ux446ux438ux44f-ux438ux43dux434ux435ux43aux441ux430-ux43eux431ux438ux43bux438ux44f-ux43fux43e-ux441ux44aux435ux43cux43aux435}

\section{Введение}\label{ux432ux432ux435ux434ux435ux43dux438ux435-4}

Это практическое занятие --- про то, как превратить данные съёмок в
строгую, воспроизводимую оценку индекса запаса, например, камчатского
краба во времени и пространстве. Мы используем sdmTMB как рабочую
лошадку: это современная реализация пространственно‑временных смешанных
моделей (GAMM/GLMM) с гауссовским марковским полем (SPDE) и
распределением Твиди для «рыбных» данных с нулями и передисперсией. В
духе Нассима Талеба начнём с честного признания: карта --- не
территория, а модель --- не «истина», а аккуратно сформулированная
гипотеза, которую мы обязаны проверять. В духе Даниэля Канемана будем
сознательно притормаживать «Систему 1» --- желание скорее получить
красивый график --- и переводить себя в «Систему 2»: чёткие допущения,
диагностика, чувствительность к альтернативам.

Мы ставим две цели. Первая --- стандартизировать индексы обилия краба по
годам, разложив наблюдаемую вариацию на «сигнал» (год, экология) и «шум»
(пространственная автокорреляция, структура съёмок, нули). Вторая ---
показать полный воспроизводимый конвейер: от проекции координат до карт
плотности и временного индекса с 50\% и 95\% доверительными интервалами.
По дороге мы дисциплинируем себя против типичных когнитивных ловушек:
WYSIATI («то, что видим --- и есть всё») --- когда пара удачных карт
заставляет игнорировать нули; подтверждающая предвзятость --- когда
заранее «знаем», что глубже «хуже»; эффект красивой истории --- когда
гладкая лента доверия соблазняет не проверять остатки. Антидот --- ясные
процедуры и проверки, описанные заранее и выполненные последовательно.

Мы начнём с данных. Координаты переводим в метры/километры (UTM), потому
что расстояния и сетка «mesh» живут в евклидовом пространстве, а не в
градусах. Уборка артефактов и выпуклая оболочка (convex hull) --- не
косметика: мы явно ограничиваем область предсказаний тем местом, где у
нас есть информация. Сетку прогнозирования строим равномерно (шаг 10 км
для индексов, более плотный --- для карт) и размножаем по годам; если в
данных есть переменные усилия (длительность, ширина трала, скорость),
используем плотность или offset(log(покрытой площади)) --- это
превращает «уловы» в сопоставимые «на единицу усилия». Здесь полезно
напоминание: если модель «слышит» усилие как сигнал обилия, мы сами
подменили биологию методикой отбора.

Дальше --- пространство и время. Мы строим треугольную сетку (mesh) c
разумным cutoff (например, 10 км) и запасом до границы полигона (чтобы
край не «ломал» ковариацию). Пространственный член --- стационарный
Матерн; пространственно-временной --- сначала iid по годам (как
минимум), а затем, при необходимости, AR1, если данные поддерживают
«инерцию» в годах. Распределение Твиди с лог‑ссылкой --- стандарт для
траловых съёмок: допускает нули как «структурные» и «стохастические»,
контролирует передисперсию. Это не догма: дельта‑подход (биномиальная
вероятность + гамма интенсивность) иногда лучше, но здесь мы остаёмся в
Твиди как в хорошо зарекомендовавшем себя компромиссе.

Формула модели консервативна и объяснима: фиксированные эффекты включают
год (как индикатор индекса) и, при необходимости, тип съёмки (SURV),
чтобы стандартизовать методические различия. Экологические ковариаты ---
глубина, температура, дистанция до берега --- добавляются в виде
сглаживаний s(DEPTH), s(TEMP), s(DIST) или как монотонные трансформации;
каждая ковариата --- это гипотеза, а не украшение. Для каждого
добавления задаём вопрос: «какую конкретную ошибку я уменьшаю этой
переменной?» Если ответ «никакую», это кандидат в удаление. Слишком
«гладкая» глубина, которая «улучшает» AIC на единицы и не меняет
биологического смысла, вероятно, эксплуатирует шум.

Оценивание --- через ML/REML; критерии --- AIC/ΔAIC, но не только. Мы
смотрим на sanity‑проверки (сходимость, положительно определённая
Гессиан, разумный диапазон Матерна, адекватные сигмы), на остатки
(гистограмма, QQ‑плот), на семивариограмму резервов, на карты
предсказаний и стандартных ошибок. Индекс рассчитываем функцией
get\_index с bias\_correct: на лог‑шкале с случайными эффектами
геометрическое среднее систематически смещается --- коррекция устраняет
этот эффект. Лента неопределённости --- 50\% и 95\% --- полезна для
управленческих разговоров: медианная динамика и «широкая страховочная»
область.

Визуализация --- не финальный штрих, а проверка гипотез. Карты плотности
по годам (facet) показывают миграции «горячих пятен» и «дырок» покрытия;
точки наблюдений поверх --- предохранитель от «галлюцинаций» модели там,
где данных не было. Нули --- крестиками, чтобы не исчезали в непрерывной
шкале. Эффекты ковариат мы рисуем на шкале плотности (после экспоненты),
с лентами доверия, фиксируя год и тип съёмки --- и спрашиваем себя:
«могу ли я объяснить это процессом, а не только формой сглаживания?».
Индекс обилия --- линия с лентами, с подписью единиц и масштаба (млн
экз.), без «спрятанных» преобразований оси.

В плане дисциплины мы будем систематичны. Сначала --- базовая модель
времени (год), чтобы увидеть «скелет» динамики. Затем --- добавление
SURV (стандартизация методики). После --- по одной экологической
ковариате, с сопоставлением моделей (AIC, стабильность, биологический
смысл, поведение остаточного пространства). Любое улучшение проверяется
на чувствительность к настройкам mesh (cutoff, расширение границы), к
выбору семейства (дельта vs Твиди), к структуре времени (iid vs AR1).
Если модель с глубиной уменьшает AIC и делает диапазон Матерна менее
расплывчатым --- это аргумент в её пользу; если добавление температуры
почти не меняет карт и даёт большой SE сглаживания --- это сигнал о
слабом или неоднородном эффекте.

Чтобы минимизировать «быстрые ошибки» --- утечки информации и
переобучение --- мы делаем простые защитные шаги. Мы не подстраиваем
сетку по виденному индексу; мы не выбираем шаг прогнозной сетки
постфактум «чтобы было красиво»; мы отделяем данные для индекса от
«демо‑карт» других лет, если демонстрируем экстраполяции. И мы держим
под рукой «план Б»: если в данных явная и сильная избытка нулей,
тестируем дельта‑подход; если глубина и температура «ходят» вместе,
проверяем их по отдельности или используем частичную зависимость.

Что вы получите в итоге. Полный воспроизводимый конвейер: проекция
координат в UTM (км), выпуклая оболочка и прогнозная сетка, треугольная
mesh, базовая и расширенные модели sdmTMB, карты плотности по годам с
точками съёмок, графики эффекта глубины (или других ковариат) с
доверительными лентами, и --- ключевое --- индекс обилия с 50\% и 95\%
ДИ, рассчитанный с коррекцией смещения. Вместе с этим --- набор
диагностик, который формирует «иммунитет» к излишней уверенности:
sanity‑чек модели, остатки, вариограммы, сравнение AIC, проверка
масштабов (диапазон Матерна, SD компонентов), и минимальный блок
чувствительности к архитектуре mesh и структуре времени.

И несколько практичных замечаний напоследок. 1) Проекция и единицы ---
это не бухгалтерия: неправильные километры превращают «корреляцию на 140
км» в географическую фантазию. 2) Выпуклая оболочка экономит силы модели
и удерживает нас от прогнозов «за краем карты»; слишком агрессивная
оболочка может «отрезать» настоящее пространство --- сравнивайте два
уровня жёсткости. 3) Твиди --- не панацея: если нули доминируют,
дельта‑модель и/или hurdle может дать более прозрачную интерпретацию. 4)
Индекс --- управленческий инструмент: показывайте медиану и ленты,
обозначайте годы со сменой методик, не скрывайте ширину
неопределённости. 5) Воспроизводимость --- ваш лучший адвокат:
фиксируйте seed, версии пакетов, все ключевые параметры сетки и модели
прямо в скрипте.

Этот курс --- про ремесло. Мы не «ловим» красивый график, а выстраиваем
цепочку решений, на каждом шаге задавая себе вопрос номер один: «почему
это должно быть правдой?» и номер два: вопрос «где здесь могу ошибиться
из‑за собственной уверенности?». Если к концу занятия у вас получится
модель, чьи выводы понятны, а неопределённость --- показана, вы сделали
ровно то, что нужно прикладной экологии: превратили разрозненные
наблюдения в аккуратное знание, которое можно проверить, воспроизвести и
использовать.

\textbf{Цель:}\\
Продемонстрировать применение современных методов SDM (Species
Distribution Modeling) и GAMM (Generalized Additive Mixed Models) для
стандартизации оценки запасов промысловых видов на примере камчатского
краба.

\textbf{Ключевые аспекты:}

\begin{enumerate}
\def\labelenumi{\arabic{enumi}.}
\item
  \textbf{Подготовка данных}:

  \begin{itemize}
  \item
    Преобразование координат в проекцию UTM (км)
  \item
    Фильтрация данных через выпуклую оболочку (convex hull)
  \item
    Создание прогнозной сетки с шагом 10 км (2 км)
  \end{itemize}
\item
  \textbf{Моделирование}:

  \begin{itemize}
  \item
    Построение треугольной сетки (mesh) для учета пространственной
    автокорреляции
  \item
    Подбор модели sdmTMB с пространственно-временными случайными
    эффектами
  \item
    Учет ключевых факторов: температура, глубина, тип съемки
  \end{itemize}
\item
  \textbf{Визуализация}:

  \begin{itemize}
  \item
    Карты распределения плотности с наложением данных съемок
  \item
    Динамика индекса обилия с 50\% и 95\% доверительными интервалами
  \end{itemize}
\end{enumerate}

\textbf{Для работы скрипта:}

\begin{enumerate}
\def\labelenumi{\arabic{enumi}.}
\item
  Скачайте файл данных
  (\href{https://mombus.github.io/cRab/data/KARTOGRAPHIC.xlsx}{KARTOGRAPHIC.xlsx})
\item
  Установите рабочую директорию в setwd()
\item
  Установите необходимые пакеты (см. начало скрипта).
\end{enumerate}

\section{Базовая
оценка}\label{ux431ux430ux437ux43eux432ux430ux44f-ux43eux446ux435ux43dux43aux430}

\begin{Shaded}
\begin{Highlighting}[]
\CommentTok{\# {-}{-}{-}{-}{-}{-}{-}{-}{-}{-}{-}{-}{-}{-}{-}{-}{-}{-}{-}{-}{-}{-}{-}{-}{-}{-}{-}}
\CommentTok{\# 1. ПОДГОТОВКА СРЕДЫ И ДАННЫХ}
\CommentTok{\# {-}{-}{-}{-}{-}{-}{-}{-}{-}{-}{-}{-}{-}{-}{-}{-}{-}{-}{-}{-}{-}{-}{-}{-}{-}{-}{-}}

\CommentTok{\# Очистка рабочей среды}
\FunctionTok{rm}\NormalTok{(}\AttributeTok{list =} \FunctionTok{ls}\NormalTok{())}

\CommentTok{\# Установка рабочей директории (замените на свою)}
\FunctionTok{setwd}\NormalTok{(}\StringTok{"C:/COMBINE/"}\NormalTok{)}

\CommentTok{\# Загрузка необходимых пакетов}
\FunctionTok{library}\NormalTok{(readxl)       }\CommentTok{\# Для чтения Excel{-}файлов}
\FunctionTok{library}\NormalTok{(ggplot2)      }\CommentTok{\# Визуализация данных}
\FunctionTok{library}\NormalTok{(dplyr)        }\CommentTok{\# Обработка данных}
\FunctionTok{library}\NormalTok{(PBSmapping)   }\CommentTok{\# Для работы с пространственными данными}
\FunctionTok{library}\NormalTok{(sdmTMB)       }\CommentTok{\# Пространственно{-}временное моделирование}
\FunctionTok{library}\NormalTok{(INLA)         }\CommentTok{\# Продвинутые пространственные модели}
\FunctionTok{library}\NormalTok{(sp)           }\CommentTok{\# Классы для пространственных данных}
\FunctionTok{library}\NormalTok{(sf)           }\CommentTok{\# Пространственные данные (современный формат)}
\FunctionTok{library}\NormalTok{(rnaturalearth) }\CommentTok{\# Загрузка картографических данных}

\CommentTok{\# Загрузка данных из Excel{-}файла}
\NormalTok{data }\OtherTok{\textless{}{-}}\NormalTok{ readxl}\SpecialCharTok{::}\FunctionTok{read\_excel}\NormalTok{(}\StringTok{"KARTOGRAPHIC.xlsx"}\NormalTok{, }\AttributeTok{sheet =} \StringTok{"SURVEY"}\NormalTok{)}

\CommentTok{\# Просмотр структуры данных}
\FunctionTok{str}\NormalTok{(data)}
\end{Highlighting}
\end{Shaded}

Должно выглядеть так:

\begin{Shaded}
\begin{Highlighting}[]
\SpecialCharTok{\textgreater{}} \FunctionTok{str}\NormalTok{(data)}
\NormalTok{tibble [}\DecValTok{1}\NormalTok{,}\DecValTok{126}\NormalTok{ x }\DecValTok{20}\NormalTok{] (S3}\SpecialCharTok{:}\NormalTok{ tbl\_df}\SpecialCharTok{/}\NormalTok{tbl}\SpecialCharTok{/}\NormalTok{data.frame)}
 \SpecialCharTok{$}\NormalTok{ NUM    }\SpecialCharTok{:}\NormalTok{ num [}\DecValTok{1}\SpecialCharTok{:}\DecValTok{1126}\NormalTok{] }\DecValTok{1} \DecValTok{2} \DecValTok{3} \DecValTok{4} \DecValTok{5} \DecValTok{6} \DecValTok{7} \DecValTok{8} \DecValTok{9} \DecValTok{10}\NormalTok{ ...}
 \SpecialCharTok{$}\NormalTok{ CALL   }\SpecialCharTok{:}\NormalTok{ chr [}\DecValTok{1}\SpecialCharTok{:}\DecValTok{1126}\NormalTok{] }\StringTok{"UFJN"} \StringTok{"UFJN"} \StringTok{"UFJN"} \StringTok{"UFJN"}\NormalTok{ ...}
 \SpecialCharTok{$}\NormalTok{ CRUSE  }\SpecialCharTok{:}\NormalTok{ num [}\DecValTok{1}\SpecialCharTok{:}\DecValTok{1126}\NormalTok{] }\DecValTok{112} \DecValTok{112} \DecValTok{112} \DecValTok{112} \DecValTok{112} \DecValTok{112} \DecValTok{112} \DecValTok{112} \DecValTok{112} \DecValTok{112}\NormalTok{ ...}
 \SpecialCharTok{$}\NormalTok{ SURV   }\SpecialCharTok{:}\NormalTok{ chr [}\DecValTok{1}\SpecialCharTok{:}\DecValTok{1126}\NormalTok{] }\StringTok{"SUM"} \StringTok{"SUM"} \StringTok{"SUM"} \StringTok{"SUM"}\NormalTok{ ...}
 \SpecialCharTok{$}\NormalTok{ TRAL   }\SpecialCharTok{:}\NormalTok{ num [}\DecValTok{1}\SpecialCharTok{:}\DecValTok{1126}\NormalTok{] }\DecValTok{2} \DecValTok{3} \DecValTok{5} \DecValTok{7} \DecValTok{9} \DecValTok{11} \DecValTok{13} \DecValTok{15} \DecValTok{17} \DecValTok{19}\NormalTok{ ...}
 \SpecialCharTok{$}\NormalTok{ DATE   }\SpecialCharTok{:}\NormalTok{ POSIXct[}\DecValTok{1}\SpecialCharTok{:}\DecValTok{1126}\NormalTok{], format}\SpecialCharTok{:} \StringTok{"2019{-}08{-}16"} \StringTok{"2019{-}08{-}16"}\NormalTok{ ...}
 \SpecialCharTok{$}\NormalTok{ MONTH  }\SpecialCharTok{:}\NormalTok{ num [}\DecValTok{1}\SpecialCharTok{:}\DecValTok{1126}\NormalTok{] }\DecValTok{8} \DecValTok{8} \DecValTok{8} \DecValTok{8} \DecValTok{8} \DecValTok{8} \DecValTok{8} \DecValTok{8} \DecValTok{8} \DecValTok{8}\NormalTok{ ...}
 \SpecialCharTok{$}\NormalTok{ YEAR   }\SpecialCharTok{:}\NormalTok{ num [}\DecValTok{1}\SpecialCharTok{:}\DecValTok{1126}\NormalTok{] }\DecValTok{2019} \DecValTok{2019} \DecValTok{2019} \DecValTok{2019} \DecValTok{2019}\NormalTok{ ...}
 \SpecialCharTok{$}\NormalTok{ TIME   }\SpecialCharTok{:}\NormalTok{ chr [}\DecValTok{1}\SpecialCharTok{:}\DecValTok{1126}\NormalTok{] }\StringTok{"9:43"} \StringTok{"14:19"} \StringTok{"19:33"} \StringTok{"2:47"}\NormalTok{ ...}
 \SpecialCharTok{$}\NormalTok{ DECMIN }\SpecialCharTok{:}\NormalTok{ num [}\DecValTok{1}\SpecialCharTok{:}\DecValTok{1126}\NormalTok{] }\FloatTok{1.04} \FloatTok{0.15} \FloatTok{0.15} \FloatTok{0.15} \FloatTok{0.15} \FloatTok{0.15} \FloatTok{0.15} \FloatTok{0.15} \FloatTok{0.15} \FloatTok{0.15}\NormalTok{ ...}
 \SpecialCharTok{$}\NormalTok{ DUR    }\SpecialCharTok{:}\NormalTok{ num [}\DecValTok{1}\SpecialCharTok{:}\DecValTok{1126}\NormalTok{] }\FloatTok{1.73} \FloatTok{0.25} \FloatTok{0.25} \FloatTok{0.25} \FloatTok{0.25}\NormalTok{ ...}
 \SpecialCharTok{$}\NormalTok{ DEPTH  }\SpecialCharTok{:}\NormalTok{ num [}\DecValTok{1}\SpecialCharTok{:}\DecValTok{1126}\NormalTok{] }\DecValTok{200} \DecValTok{198} \DecValTok{196} \DecValTok{132} \DecValTok{128} \DecValTok{131} \DecValTok{64} \DecValTok{73} \DecValTok{91} \DecValTok{62}\NormalTok{ ...}
 \SpecialCharTok{$}\NormalTok{ SPEED  }\SpecialCharTok{:}\NormalTok{ num [}\DecValTok{1}\SpecialCharTok{:}\DecValTok{1126}\NormalTok{] }\DecValTok{3} \DecValTok{3} \DecValTok{3} \DecValTok{3} \DecValTok{3} \DecValTok{3} \DecValTok{3} \DecValTok{3} \DecValTok{3} \DecValTok{3}\NormalTok{ ...}
 \SpecialCharTok{$}\NormalTok{ CATCH  }\SpecialCharTok{:}\NormalTok{ num [}\DecValTok{1}\SpecialCharTok{:}\DecValTok{1126}\NormalTok{] }\FloatTok{12.9} \FloatTok{365.3} \DecValTok{253} \FloatTok{163.9} \FloatTok{55.7}\NormalTok{ ...}
 \SpecialCharTok{$}\NormalTok{ Y      }\SpecialCharTok{:}\NormalTok{ num [}\DecValTok{1}\SpecialCharTok{:}\DecValTok{1126}\NormalTok{] }\FloatTok{69.5} \FloatTok{69.5} \FloatTok{69.4} \FloatTok{68.8} \FloatTok{69.4}\NormalTok{ ...}
 \SpecialCharTok{$}\NormalTok{ X      }\SpecialCharTok{:}\NormalTok{ num [}\DecValTok{1}\SpecialCharTok{:}\DecValTok{1126}\NormalTok{] }\FloatTok{35.8} \FloatTok{35.9} \FloatTok{37.4} \FloatTok{38.6} \DecValTok{39}\NormalTok{ ...}
 \SpecialCharTok{$}\NormalTok{ PROM   }\SpecialCharTok{:}\NormalTok{ num [}\DecValTok{1}\SpecialCharTok{:}\DecValTok{1126}\NormalTok{] }\DecValTok{2} \DecValTok{0} \DecValTok{0} \DecValTok{3} \DecValTok{0} \DecValTok{6} \DecValTok{6} \DecValTok{34} \DecValTok{22} \DecValTok{9}\NormalTok{ ...}
 \SpecialCharTok{$}\NormalTok{ Density}\SpecialCharTok{:}\NormalTok{ num [}\DecValTok{1}\SpecialCharTok{:}\DecValTok{1126}\NormalTok{] }\DecValTok{30} \DecValTok{0} \DecValTok{0} \DecValTok{45} \DecValTok{0}\NormalTok{ ...}
 \SpecialCharTok{$}\NormalTok{ DIST   }\SpecialCharTok{:}\NormalTok{ num [}\DecValTok{1}\SpecialCharTok{:}\DecValTok{1126}\NormalTok{] }\FloatTok{28.7} \FloatTok{28.7} \FloatTok{49.9} \FloatTok{37.3} \FloatTok{90.8}\NormalTok{ ...}
 \SpecialCharTok{$}\NormalTok{ TEMP   }\SpecialCharTok{:}\NormalTok{ num [}\DecValTok{1}\SpecialCharTok{:}\DecValTok{1126}\NormalTok{] }\FloatTok{5.57} \FloatTok{5.49} \FloatTok{4.99} \FloatTok{4.8} \FloatTok{4.4}\NormalTok{ ...}
\SpecialCharTok{\textgreater{}} 
\end{Highlighting}
\end{Shaded}

Далее:

\begin{Shaded}
\begin{Highlighting}[]
\CommentTok{\# {-}{-}{-}{-}{-}{-}{-}{-}{-}{-}{-}{-}{-}{-}{-}{-}{-}{-}{-}{-}{-}{-}{-}{-}{-}{-}{-}{-}{-}{-}{-}{-}{-}{-}{-}{-}{-}{-}{-}{-}{-}{-}{-}{-}{-}{-}{-}{-}{-}{-}}
\CommentTok{\# 2. ПРЕОБРАЗОВАНИЕ КООРДИНАТ В ПРОЕКЦИЮ UTM (в км)}
\CommentTok{\# {-}{-}{-}{-}{-}{-}{-}{-}{-}{-}{-}{-}{-}{-}{-}{-}{-}{-}{-}{-}{-}{-}{-}{-}{-}{-}{-}{-}{-}{-}{-}{-}{-}{-}{-}{-}{-}{-}{-}{-}{-}{-}{-}{-}{-}{-}{-}{-}{-}{-}}

\CommentTok{\# Создание пространственного объекта из данных}
\NormalTok{data\_sf }\OtherTok{\textless{}{-}} \FunctionTok{st\_as\_sf}\NormalTok{(}
\NormalTok{  data, }
  \AttributeTok{coords =} \FunctionTok{c}\NormalTok{(}\StringTok{"X"}\NormalTok{, }\StringTok{"Y"}\NormalTok{), }\CommentTok{\# Указание столбцов с координатами}
  \AttributeTok{crs =} \DecValTok{4326}            \CommentTok{\# Система координат WGS84 (широта/долгота)}
\NormalTok{) }

\CommentTok{\# Преобразование в UTM зону 37N (метры)}
\NormalTok{data\_utm }\OtherTok{\textless{}{-}} \FunctionTok{st\_transform}\NormalTok{(data\_sf, }\AttributeTok{crs =} \DecValTok{32637}\NormalTok{) }

\CommentTok{\# Извлечение координат и перевод в километры}
\NormalTok{utm\_coords }\OtherTok{\textless{}{-}} \FunctionTok{st\_coordinates}\NormalTok{(data\_utm)}
\NormalTok{data}\SpecialCharTok{$}\NormalTok{xkm }\OtherTok{\textless{}{-}}\NormalTok{ utm\_coords[, }\DecValTok{1}\NormalTok{] }\SpecialCharTok{/} \DecValTok{1000}  \CommentTok{\# X в км}
\NormalTok{data}\SpecialCharTok{$}\NormalTok{ykm }\OtherTok{\textless{}{-}}\NormalTok{ utm\_coords[, }\DecValTok{2}\NormalTok{] }\SpecialCharTok{/} \DecValTok{1000}  \CommentTok{\# Y в км}

\CommentTok{\# Очистка временных объектов}
\FunctionTok{rm}\NormalTok{(data\_sf, data\_utm, utm\_coords)}

\CommentTok{\# {-}{-}{-}{-}{-}{-}{-}{-}{-}{-}{-}{-}{-}{-}{-}{-}{-}{-}{-}{-}{-}{-}{-}{-}{-}{-}{-}{-}{-}{-}{-}{-}{-}{-}{-}{-}{-}{-}{-}{-}{-}}
\CommentTok{\# 3. ОПРЕДЕЛЕНИЕ ГРАНИЦ ИССЛЕДОВАНИЯ}
\CommentTok{\# {-}{-}{-}{-}{-}{-}{-}{-}{-}{-}{-}{-}{-}{-}{-}{-}{-}{-}{-}{-}{-}{-}{-}{-}{-}{-}{-}{-}{-}{-}{-}{-}{-}{-}{-}{-}{-}{-}{-}{-}{-}}

\CommentTok{\# Вычисление границ исследовательского полигона}
\NormalTok{xl }\OtherTok{\textless{}{-}} \FunctionTok{c}\NormalTok{(}\FunctionTok{min}\NormalTok{(data}\SpecialCharTok{$}\NormalTok{xkm), }\FunctionTok{max}\NormalTok{(data}\SpecialCharTok{$}\NormalTok{xkm))  }\CommentTok{\# Границы по X}
\NormalTok{yl }\OtherTok{\textless{}{-}} \FunctionTok{c}\NormalTok{(}\FunctionTok{min}\NormalTok{(data}\SpecialCharTok{$}\NormalTok{ykm), }\FunctionTok{max}\NormalTok{(data}\SpecialCharTok{$}\NormalTok{ykm))  }\CommentTok{\# Границы по Y}

\CommentTok{\# {-}{-}{-}{-}{-}{-}{-}{-}{-}{-}{-}{-}{-}{-}{-}{-}{-}{-}{-}{-}{-}{-}{-}{-}{-}{-}{-}{-}{-}{-}{-}{-}{-}{-}{-}{-}{-}{-}{-}{-}}
\CommentTok{\# 4. СОЗДАНИЕ РАСТРОВОЙ СЕТКИ ДЛЯ МОДЕЛИ}
\CommentTok{\# {-}{-}{-}{-}{-}{-}{-}{-}{-}{-}{-}{-}{-}{-}{-}{-}{-}{-}{-}{-}{-}{-}{-}{-}{-}{-}{-}{-}{-}{-}{-}{-}{-}{-}{-}{-}{-}{-}{-}{-}}

\CommentTok{\# Создание равномерной сетки с шагом 10 км }
\NormalTok{GRID }\OtherTok{\textless{}{-}} \FunctionTok{makeGrid}\NormalTok{(}
  \AttributeTok{x =} \FunctionTok{seq}\NormalTok{(xl[}\DecValTok{1}\NormalTok{], xl[}\DecValTok{2}\NormalTok{], }\DecValTok{10}\NormalTok{), }
  \AttributeTok{y =} \FunctionTok{seq}\NormalTok{(yl[}\DecValTok{1}\NormalTok{], yl[}\DecValTok{2}\NormalTok{], }\DecValTok{10}\NormalTok{),}
  \AttributeTok{byrow =} \ConstantTok{FALSE}\NormalTok{,}
  \AttributeTok{projection =} \StringTok{"UTM"}\NormalTok{, }
  \AttributeTok{zone =} \DecValTok{37}
\NormalTok{)}

\CommentTok{\# Расчет центроидов ячеек сетки}
\NormalTok{GRID }\OtherTok{\textless{}{-}} \FunctionTok{calcCentroid}\NormalTok{(GRID, }\AttributeTok{rollup =} \DecValTok{3}\NormalTok{)}

\CommentTok{\# {-}{-}{-}{-}{-}{-}{-}{-}{-}{-}{-}{-}{-}{-}{-}{-}{-}{-}{-}{-}{-}{-}{-}{-}{-}{-}{-}{-}{-}{-}{-}{-}{-}{-}{-}{-}{-}{-}{-}{-}{-}{-}{-}{-}{-}{-}{-}{-}{-}{-}{-}{-}{-}{-}{-}{-}{-}{-}{-}}
\CommentTok{\# 5. ПОСТРОЕНИЕ ВЫПУКЛОЙ ОБОЛОЧКИ (CONVEX HULL) ДЛЯ ДАННЫХ}
\CommentTok{\# {-}{-}{-}{-}{-}{-}{-}{-}{-}{-}{-}{-}{-}{-}{-}{-}{-}{-}{-}{-}{-}{-}{-}{-}{-}{-}{-}{-}{-}{-}{-}{-}{-}{-}{-}{-}{-}{-}{-}{-}{-}{-}{-}{-}{-}{-}{-}{-}{-}{-}{-}{-}{-}{-}{-}{-}{-}{-}{-}}

\CommentTok{\# Создание выпуклой оболочки вокруг точек данных}
\NormalTok{Hull }\OtherTok{\textless{}{-}} \FunctionTok{inla.nonconvex.hull}\NormalTok{(}\FunctionTok{cbind}\NormalTok{(data}\SpecialCharTok{$}\NormalTok{xkm, data}\SpecialCharTok{$}\NormalTok{ykm), }\AttributeTok{convex =} \SpecialCharTok{{-}}\FloatTok{0.03}\NormalTok{)}

\CommentTok{\# Визуализация оболочки }
 \FunctionTok{plot}\NormalTok{(Hull)}
\end{Highlighting}
\end{Shaded}

\begin{figure}[H]

{\centering \includegraphics[width=0.6\linewidth,height=\textheight,keepaspectratio]{images/sdmTMB1.PNG}

}

\caption{Рис. 1.: Визуализация оболочки съемок}

\end{figure}%

\begin{Shaded}
\begin{Highlighting}[]
\CommentTok{\# Визуализация оболочки и точек съемок 2019{-}2024}
\FunctionTok{points}\NormalTok{(data}\SpecialCharTok{$}\NormalTok{xkm, data}\SpecialCharTok{$}\NormalTok{ykm, }\AttributeTok{pch=}\DecValTok{1}\NormalTok{, }\AttributeTok{cex=}\FloatTok{0.55}\NormalTok{,}\AttributeTok{col=}\StringTok{"black"}\NormalTok{)}
\end{Highlighting}
\end{Shaded}

\begin{figure}[H]

{\centering \includegraphics[width=0.6\linewidth,height=\textheight,keepaspectratio]{images/sdmTMB2.PNG}

}

\caption{Рис. 2.: Визуализация оболочки и точек съемок}

\end{figure}%

\begin{Shaded}
\begin{Highlighting}[]
\CommentTok{\# Фильтрация сетки: оставляем только точки внутри оболочки}
\NormalTok{line }\OtherTok{\textless{}{-}}\NormalTok{ Hull}\SpecialCharTok{$}\NormalTok{loc[, }\DecValTok{1}\SpecialCharTok{:}\DecValTok{2}\NormalTok{] }\SpecialCharTok{\%\textgreater{}\%} \FunctionTok{as.data.frame}\NormalTok{()}
\FunctionTok{colnames}\NormalTok{(line) }\OtherTok{\textless{}{-}} \FunctionTok{c}\NormalTok{(}\StringTok{"X"}\NormalTok{, }\StringTok{"Y"}\NormalTok{)}
\NormalTok{GRID}\SpecialCharTok{$}\NormalTok{AREA }\OtherTok{\textless{}{-}} \FunctionTok{point.in.polygon}\NormalTok{(GRID}\SpecialCharTok{$}\NormalTok{X, GRID}\SpecialCharTok{$}\NormalTok{Y, line}\SpecialCharTok{$}\NormalTok{X, line}\SpecialCharTok{$}\NormalTok{Y)}
\NormalTok{GRID }\OtherTok{\textless{}{-}}\NormalTok{ GRID[GRID}\SpecialCharTok{$}\NormalTok{AREA }\SpecialCharTok{\textgreater{}} \FloatTok{0.1}\NormalTok{, }\FunctionTok{c}\NormalTok{(}\StringTok{"X"}\NormalTok{, }\StringTok{"Y"}\NormalTok{)]  }\CommentTok{\# Только внутренние точки}

\CommentTok{\# {-}{-}{-}{-}{-}{-}{-}{-}{-}{-}{-}{-}{-}{-}{-}{-}{-}{-}{-}{-}{-}{-}{-}{-}{-}{-}{-}{-}{-}{-}{-}{-}{-}{-}{-}{-}{-}{-}{-}{-}{-}{-}{-}{-}{-}{-}{-}{-}{-}}
\CommentTok{\# 6. ПОДГОТОВКА СЕТКИ ДЛЯ ПРОГНОЗИРОВАНИЯ}
\CommentTok{\# {-}{-}{-}{-}{-}{-}{-}{-}{-}{-}{-}{-}{-}{-}{-}{-}{-}{-}{-}{-}{-}{-}{-}{-}{-}{-}{-}{-}{-}{-}{-}{-}{-}{-}{-}{-}{-}{-}{-}{-}{-}{-}{-}{-}{-}{-}{-}{-}{-}}

\CommentTok{\# Создание временной сетки (для каждого года)}
\NormalTok{grid }\OtherTok{\textless{}{-}} \FunctionTok{replicate\_df}\NormalTok{(GRID, }\StringTok{"YEAR"}\NormalTok{, }\FunctionTok{unique}\NormalTok{(data}\SpecialCharTok{$}\NormalTok{YEAR))}
\FunctionTok{colnames}\NormalTok{(grid) }\OtherTok{\textless{}{-}} \FunctionTok{c}\NormalTok{(}\StringTok{"xkm"}\NormalTok{, }\StringTok{"ykm"}\NormalTok{, }\StringTok{"YEAR"}\NormalTok{)}
\NormalTok{grid}\SpecialCharTok{$}\NormalTok{SURV }\OtherTok{\textless{}{-}} \StringTok{"CRAB"}  \CommentTok{\# Добавляем информацию о типе съемки}

\CommentTok{\# Визуализация оболочки и сетки для прогнозирования (grid)}
 \FunctionTok{plot}\NormalTok{(Hull)}
 \FunctionTok{points}\NormalTok{(grid}\SpecialCharTok{$}\NormalTok{xkm, grid}\SpecialCharTok{$}\NormalTok{ykm, }\AttributeTok{pch=}\DecValTok{1}\NormalTok{, }\AttributeTok{cex=}\FloatTok{0.55}\NormalTok{,}\AttributeTok{col=}\StringTok{"black"}\NormalTok{)}
\end{Highlighting}
\end{Shaded}

\begin{figure}[H]

{\centering \includegraphics[width=0.6\linewidth,height=\textheight,keepaspectratio]{images/sdmTMB3.PNG}

}

\caption{Рис. 3.: Визуализация оболочки и сетки для прогнозирования
(grid)}

\end{figure}%

\begin{Shaded}
\begin{Highlighting}[]
\CommentTok{\# {-}{-}{-}{-}{-}{-}{-}{-}{-}{-}{-}{-}{-}{-}{-}{-}{-}{-}{-}{-}{-}{-}{-}{-}{-}{-}{-}{-}{-}{-}{-}{-}{-}{-}{-}{-}{-}{-}{-}{-}{-}{-}{-}{-}{-}{-}{-}{-}{-}{-}{-}}
\CommentTok{\# 7. ПОСТРОЕНИЕ ПРОСТРАНСТВЕННОЙ СЕТКИ (MESH)}
\CommentTok{\# {-}{-}{-}{-}{-}{-}{-}{-}{-}{-}{-}{-}{-}{-}{-}{-}{-}{-}{-}{-}{-}{-}{-}{-}{-}{-}{-}{-}{-}{-}{-}{-}{-}{-}{-}{-}{-}{-}{-}{-}{-}{-}{-}{-}{-}{-}{-}{-}{-}{-}{-}}

\CommentTok{\# Создание треугольной сетки для пространственного моделирования}
\NormalTok{mesh\_sdm }\OtherTok{\textless{}{-}} \FunctionTok{make\_mesh}\NormalTok{(}
\NormalTok{  data, }
  \FunctionTok{c}\NormalTok{(}\StringTok{"xkm"}\NormalTok{, }\StringTok{"ykm"}\NormalTok{),  }\CommentTok{\# Координаты}
  \AttributeTok{cutoff =} \DecValTok{10}        \CommentTok{\# Минимальное расстояние между узлами (км)}
\NormalTok{)}

\CommentTok{\# Визуализация сетки }
 \FunctionTok{plot}\NormalTok{(mesh\_sdm)}
\end{Highlighting}
\end{Shaded}

\begin{figure}[H]

{\centering \includegraphics[width=0.6\linewidth,height=\textheight,keepaspectratio]{images/sdmTMB4.PNG}

}

\caption{Рис. 4.: Визуализация сетки (mesh)}

\end{figure}%

\begin{Shaded}
\begin{Highlighting}[]
\CommentTok{\# {-}{-}{-}{-}{-}{-}{-}{-}{-}{-}{-}{-}{-}{-}{-}{-}{-}{-}{-}{-}{-}{-}{-}{-}{-}{-}{-}{-}{-}{-}{-}{-}{-}{-}{-}{-}{-}{-}{-}{-}{-}{-}{-}{-}{-}{-}{-}{-}{-}{-}{-}}
\CommentTok{\# 8. ПОСТРОЕНИЕ ПРОСТРАНСТВЕННО{-}ВРЕМЕННОЙ МОДЕЛИ}
\CommentTok{\# {-}{-}{-}{-}{-}{-}{-}{-}{-}{-}{-}{-}{-}{-}{-}{-}{-}{-}{-}{-}{-}{-}{-}{-}{-}{-}{-}{-}{-}{-}{-}{-}{-}{-}{-}{-}{-}{-}{-}{-}{-}{-}{-}{-}{-}{-}{-}{-}{-}{-}{-}}

\NormalTok{m }\OtherTok{\textless{}{-}} \FunctionTok{sdmTMB}\NormalTok{(}
  \AttributeTok{data =}\NormalTok{ data, }
  \AttributeTok{formula =}\NormalTok{ Density }\SpecialCharTok{\textasciitilde{}} \DecValTok{0} \SpecialCharTok{+} \FunctionTok{as.factor}\NormalTok{(YEAR),  }\CommentTok{\# Формула: плотность зависит от года}
  \AttributeTok{time =} \StringTok{"YEAR"}\NormalTok{,         }\CommentTok{\# Временная переменная}
  \AttributeTok{mesh =}\NormalTok{ mesh\_sdm,       }\CommentTok{\# Пространственная сетка}
  \AttributeTok{family =} \FunctionTok{tweedie}\NormalTok{(}\AttributeTok{link =} \StringTok{"log"}\NormalTok{),  }\CommentTok{\# Статистическое распределение}
  \AttributeTok{spatial =} \StringTok{"on"}\NormalTok{,        }\CommentTok{\# Включение пространственных эффектов}
  \AttributeTok{spatiotemporal =} \StringTok{"iid"} \CommentTok{\# Пространственно{-}временные эффекты}
\NormalTok{)}

\CommentTok{\# Вывод результатов модели}
\FunctionTok{summary}\NormalTok{(m)}
\FunctionTok{AIC}\NormalTok{(m)  }\CommentTok{\# Критерий Акаике}
\FunctionTok{sanity}\NormalTok{(m)  }\CommentTok{\# Проверка корректности модели}
\end{Highlighting}
\end{Shaded}

Получили результаты:

\begin{Shaded}
\begin{Highlighting}[]
\SpecialCharTok{\textgreater{}} \CommentTok{\# Вывод результатов модели}
\ErrorTok{\textgreater{}} \FunctionTok{summary}\NormalTok{(m)}
\NormalTok{Spatiotemporal model fit by ML [}\StringTok{\textquotesingle{}sdmTMB\textquotesingle{}}\NormalTok{]}
\NormalTok{Formula}\SpecialCharTok{:}\NormalTok{ Density }\SpecialCharTok{\textasciitilde{}} \DecValTok{0} \SpecialCharTok{+} \FunctionTok{as.factor}\NormalTok{(YEAR)}
\NormalTok{Mesh}\SpecialCharTok{:} \FunctionTok{mesh\_sdm}\NormalTok{ (isotropic covariance)}
\NormalTok{Time column}\SpecialCharTok{:}\NormalTok{ YEAR}
\NormalTok{Data}\SpecialCharTok{:}\NormalTok{ data}
\NormalTok{Family}\SpecialCharTok{:} \FunctionTok{tweedie}\NormalTok{(}\AttributeTok{link =} \StringTok{\textquotesingle{}log\textquotesingle{}}\NormalTok{)}
 
\NormalTok{Conditional model}\SpecialCharTok{:}
\NormalTok{                    coef.est coef.se}
\FunctionTok{as.factor}\NormalTok{(YEAR)}\DecValTok{2019}     \FloatTok{2.15}    \FloatTok{0.72}
\FunctionTok{as.factor}\NormalTok{(YEAR)}\DecValTok{2020}     \FloatTok{1.49}    \FloatTok{0.76}
\FunctionTok{as.factor}\NormalTok{(YEAR)}\DecValTok{2021}     \FloatTok{1.74}    \FloatTok{0.76}
\FunctionTok{as.factor}\NormalTok{(YEAR)}\DecValTok{2022}     \FloatTok{1.62}    \FloatTok{0.73}
\FunctionTok{as.factor}\NormalTok{(YEAR)}\DecValTok{2023}     \FloatTok{1.50}    \FloatTok{0.74}
\FunctionTok{as.factor}\NormalTok{(YEAR)}\DecValTok{2024}     \FloatTok{1.56}    \FloatTok{0.72}

\NormalTok{Dispersion parameter}\SpecialCharTok{:} \FloatTok{19.71}
\NormalTok{Tweedie p}\SpecialCharTok{:} \FloatTok{1.50}
\NormalTok{Matern range}\SpecialCharTok{:} \FloatTok{142.65}
\NormalTok{Spatial SD}\SpecialCharTok{:} \FloatTok{2.01}
\NormalTok{Spatiotemporal IID SD}\SpecialCharTok{:} \FloatTok{0.95}
\NormalTok{ML criterion at convergence}\SpecialCharTok{:} \FloatTok{5984.224}

\NormalTok{See ?tidy.sdmTMB to extract these values as a data frame.}
\SpecialCharTok{\textgreater{}} \FunctionTok{AIC}\NormalTok{(m)  }\CommentTok{\# Критерий Акаике}
\NormalTok{[}\DecValTok{1}\NormalTok{] }\FloatTok{11990.45}
\SpecialCharTok{\textgreater{}} \FunctionTok{sanity}\NormalTok{(m)  }\CommentTok{\# Проверка корректности модели}
\NormalTok{v Non}\SpecialCharTok{{-}}\NormalTok{linear minimizer suggests successful convergence}
\NormalTok{v Hessian matrix is positive definite}
\NormalTok{v No extreme or very small eigenvalues detected}
\NormalTok{v No gradients with respect to fixed effects are }\SpecialCharTok{\textgreater{}=} \FloatTok{0.001}
\NormalTok{v No fixed}\SpecialCharTok{{-}}\NormalTok{effect standard errors are }\ConstantTok{NA}
\NormalTok{v No standard errors look unreasonably large}
\NormalTok{v No sigma parameters are }\SpecialCharTok{\textless{}} \FloatTok{0.01}
\NormalTok{v No sigma parameters are }\SpecialCharTok{\textgreater{}} \DecValTok{100}
\NormalTok{v Range parameter doesn}\StringTok{\textquotesingle{}t look unreasonably large}
\end{Highlighting}
\end{Shaded}

\subsubsection{\texorpdfstring{\textbf{Годовые
эффекты:}}{Годовые эффекты:}}\label{ux433ux43eux434ux43eux432ux44bux435-ux44dux444ux444ux435ux43aux442ux44b}

\begin{verbatim}
2019: 2.15 ± 0.72 → exp(2.15) ≈ 8.58 экз./км²
2020: 1.49 ± 0.76 → exp(1.49) ≈ 4.44 экз./км²
2024: 1.56 ± 0.72 → exp(1.56) ≈ 4.76 экз./км²
\end{verbatim}

\begin{itemize}
\item
  \textbf{2019 год} - пик запаса (8.58 экз./км²)
\item
  \textbf{2020 год} - резкое снижение (-52\% к 2019)
\item
  \textbf{2021-2024} - стабилизация на уровне \textasciitilde4.5-5.0
  экз./км²
\item
  \textbf{Стандартные ошибки} \textasciitilde0.75:

  \begin{itemize}
  \item
    Приемлемая точность для данных такого объема
  \item
    Все годовые оценки статистически значимы
  \end{itemize}

  Модель пространственно-временного распределения плотности камчатского
  краба успешно прошла все диагностические проверки, демонстрируя
  отличную сходимость и статистическую надежность. Параметр
  распределения Твиди (p=1.50) оптимально соответствует данным траловых
  съемок, учитывая характерную для уловов передисперсию и избыток
  нулевых значений.

  Годовые оценки показывают выраженную динамику запаса: в 2019 году
  зафиксирован пик плотности (8.58 экз./км²), после чего в 2020 году
  произошло резкое снижение до 4.44 экз./км². В последующие годы
  (2021-2024) плотность стабилизировалась на уровне 4.5-5.0 экз./км²,
  что составляет примерно 55\% от максимальных значений 2019 года.
  Стандартные ошибки годовых коэффициентов (0.72-0.76) свидетельствуют о
  хорошей точности оценок при текущем объеме данных.

  Пространственная структура распределения характеризуется
  крупномасштабными скоплениями с диапазоном корреляции 143 км (Matern
  range: 142.65), что согласуется с известными особенностями миграций
  камчатского краба. Высокое значение пространственной изменчивости
  (SD=2.01) отражает типичную для вида мозаичность распределения, где
  участки высокой плотности соседствуют с зонами отсутствия особей.
  Умеренная пространственно-временная изменчивость (IID SD=0.95)
  указывает на относительную стабильность пространственной структуры
  запаса между годами.

  Параметр дисперсии (19.71) подтверждает ожидаемо высокую
  вариабельность данных, характерную для траловых съемок морских
  гидробионтов. Полученные результаты надежно фиксируют значительное
  сокращение запаса после 2019 года с последующей стабилизацией на
  пониженном уровне.

  \subsection{\texorpdfstring{\textbf{Пояснение результатов
  \texttt{sanity(m)} для начинающих гидробиологов (от
  DeepSeek):}}{Пояснение результатов sanity(m) для начинающих гидробиологов (от DeepSeek):}}\label{ux43fux43eux44fux441ux43dux435ux43dux438ux435-ux440ux435ux437ux443ux43bux44cux442ux430ux442ux43eux432-sanitym-ux434ux43bux44f-ux43dux430ux447ux438ux43dux430ux44eux449ux438ux445-ux433ux438ux434ux440ux43eux431ux438ux43eux43bux43eux433ux43eux432-ux43eux442-deepseek}

  \textbf{1.
  \texttt{v\ Non-linear\ minimizer\ suggests\ successful\ convergence}}\\
  (Нелинейный оптимизатор успешно сошелся)\\
  \emph{Пояснение:} Алгоритм поиска параметров модели корректно завершил
  работу. Это значит, что модель ``научилась'' описывать ваши данные и
  не застряла в промежуточных вычислениях. Как если бы вы успешно
  завершили лабораторный анализ без технических сбоев.

  \textbf{2. \texttt{v\ Hessian\ matrix\ is\ positive\ definite}}\\
  (Матрица Гессе положительно определена)\\
  \emph{Пояснение:} Математическое подтверждение, что найденные
  параметры модели действительно оптимальны. Аналогично тому, как в
  микроскопии вы видите четкий фокус - здесь модель ``четко видит''
  закономерности в данных.

  \textbf{3.
  \texttt{v\ No\ extreme\ or\ very\ small\ eigenvalues\ detected}}\\
  (Не обнаружено экстремальных или очень маленьких собственных
  значений)\\
  \emph{Пояснение:} Модель статистически стабильна. Представьте, что вы
  измеряете длину рыб - если бы ваш штангенциркуль иногда показывал 0
  или 1000 мм, это было бы проблемой. Здесь аналогично - вычисления
  надежны.

  \textbf{4.
  \texttt{v\ No\ gradients\ with\ respect\ to\ fixed\ effects\ are\ \textgreater{}=\ 0.001}}\\
  (Градиенты для фиксированных эффектов \textless{} 0.001)\\
  \emph{Пояснение:} Все ключевые параметры модели (например, влияние
  года на плотность) рассчитаны точно. Это как убедиться, что все
  измерения в вашем эксперименте выполнены с требуемой точностью (±0.1
  мг, ±1 см и т.д.).

  \textbf{5. \texttt{v\ No\ fixed-effect\ standard\ errors\ are\ NA}}\\
  (Стандартные ошибки для фиксированных эффектов не отсутствуют)\\
  \emph{Пояснение:} Для каждого рассчитанного параметра (например,
  годовых оценок) указана погрешность. Важно как в химическом анализе -
  если для концентрации вещества нет погрешности, результат ненадежен.

  \textbf{6.
  \texttt{v\ No\ standard\ errors\ look\ unreasonably\ large}}\\
  (Стандартные ошибки выглядят разумными)\\
  \emph{Пояснение:} Погрешности оценок адекватны. Например, если
  плотность краба 5±1 экз./км² - это нормально, но 5±100 экз./км² было
  бы бессмысленным.

  \textbf{7.
  \texttt{v\ No\ sigma\ parameters\ are\ \textless{}\ 0.01}}\\
  (Параметры сигма не меньше 0.01)\\
  \emph{Пояснение:} Модель не игнорирует важные источники изменчивости.
  Аналогично тому, что в пробе воды вы не упустили бы важный показатель,
  сказав ``он слишком мал''.

  \textbf{8.
  \texttt{v\ No\ sigma\ parameters\ are\ \textgreater{}\ 100}}\\
  (Параметры сигма не превышают 100)\\
  \emph{Пояснение:} Модель не преувеличивает случайные вариации. Как
  если бы вы не приписали естественные колебания температуры воды
  катастрофическому изменению климата.

  \textbf{9.
  \texttt{v\ Range\ parameter\ doesn\textquotesingle{}t\ look\ unreasonably\ large}}\\
  (Параметр диапазона не выглядит чрезмерно большим)\\
  \emph{Пояснение:} Пространственная автокорреляция имеет биологически
  осмысленный масштаб. Например, если модель показала бы, что скопления
  краба одинаковы на расстоянии 1000 км - это было бы нереалистично.
\end{itemize}

\begin{Shaded}
\begin{Highlighting}[]
\CommentTok{\# {-}{-}{-}{-}{-}{-}{-}{-}{-}{-}{-}{-}{-}{-}{-}{-}{-}{-}{-}{-}{-}{-}{-}{-}{-}{-}{-}{-}{-}{-}{-}{-}{-}{-}{-}{-}{-}{-}{-}{-}{-}{-}{-}{-}{-}{-}{-}{-}{-}{-}{-}}
\CommentTok{\# 9. ДИАГНОСТИКА МОДЕЛИ}
\CommentTok{\# {-}{-}{-}{-}{-}{-}{-}{-}{-}{-}{-}{-}{-}{-}{-}{-}{-}{-}{-}{-}{-}{-}{-}{-}{-}{-}{-}{-}{-}{-}{-}{-}{-}{-}{-}{-}{-}{-}{-}{-}{-}{-}{-}{-}{-}{-}{-}{-}{-}{-}{-}}

\CommentTok{\# Расчет остатков модели}
\NormalTok{data}\SpecialCharTok{$}\NormalTok{resids }\OtherTok{\textless{}{-}} \FunctionTok{residuals}\NormalTok{(m) }

\CommentTok{\# Гистограмма остатков}
\FunctionTok{hist}\NormalTok{(data}\SpecialCharTok{$}\NormalTok{resids)}

\CommentTok{\# График квантиль{-}квантиль}
\FunctionTok{qqnorm}\NormalTok{(data}\SpecialCharTok{$}\NormalTok{resids)}
\FunctionTok{abline}\NormalTok{(}\AttributeTok{a =} \DecValTok{0}\NormalTok{, }\AttributeTok{b =} \DecValTok{1}\NormalTok{)}
\end{Highlighting}
\end{Shaded}

\begin{figure}[H]

{\centering \includegraphics[width=0.6\linewidth,height=\textheight,keepaspectratio]{images/sdmTMB5.PNG}

}

\caption{Рис. 5.: Гистограмма остатков}

\end{figure}%

\begin{figure}[H]

{\centering \includegraphics[width=0.6\linewidth,height=\textheight,keepaspectratio]{images/sdmTMB6.PNG}

}

\caption{Рис. 6.: График квантиль-квантиль}

\end{figure}%

\begin{Shaded}
\begin{Highlighting}[]
\CommentTok{\# {-}{-}{-}{-}{-}{-}{-}{-}{-}{-}{-}{-}{-}{-}{-}{-}{-}{-}{-}{-}{-}{-}{-}{-}{-}{-}{-}{-}{-}{-}{-}{-}{-}{-}{-}{-}{-}{-}{-}{-}{-}{-}{-}{-}{-}{-}{-}{-}{-}{-}{-}}
\CommentTok{\# 10. ПРОГНОЗИРОВАНИЕ НА СЕТКЕ}
\CommentTok{\# {-}{-}{-}{-}{-}{-}{-}{-}{-}{-}{-}{-}{-}{-}{-}{-}{-}{-}{-}{-}{-}{-}{-}{-}{-}{-}{-}{-}{-}{-}{-}{-}{-}{-}{-}{-}{-}{-}{-}{-}{-}{-}{-}{-}{-}{-}{-}{-}{-}{-}{-}}

\CommentTok{\# Прогноз значений плотности на сетке}
\NormalTok{predictions }\OtherTok{\textless{}{-}} \FunctionTok{predict}\NormalTok{(m, }\AttributeTok{newdata =}\NormalTok{ grid, }\AttributeTok{return\_tmb\_object =} \ConstantTok{TRUE}\NormalTok{)}
\NormalTok{RASP }\OtherTok{\textless{}{-}}\NormalTok{ predictions}\SpecialCharTok{$}\NormalTok{data}

\CommentTok{\# Преобразование координат обратно в широту/долготу}
\NormalTok{RASP}\SpecialCharTok{$}\NormalTok{xkm\_m }\OtherTok{\textless{}{-}}\NormalTok{ RASP}\SpecialCharTok{$}\NormalTok{xkm }\SpecialCharTok{*} \DecValTok{1000}  \CommentTok{\# Обратно в метры}
\NormalTok{RASP}\SpecialCharTok{$}\NormalTok{ykm\_m }\OtherTok{\textless{}{-}}\NormalTok{ RASP}\SpecialCharTok{$}\NormalTok{ykm }\SpecialCharTok{*} \DecValTok{1000}

\CommentTok{\# Создание пространственного объекта в UTM}
\NormalTok{utm\_proj }\OtherTok{\textless{}{-}} \FunctionTok{CRS}\NormalTok{(}\StringTok{"+proj=utm +zone=37 +datum=WGS84 +units=m +no\_defs"}\NormalTok{)}
\NormalTok{coords }\OtherTok{\textless{}{-}} \FunctionTok{cbind}\NormalTok{(RASP}\SpecialCharTok{$}\NormalTok{xkm\_m, RASP}\SpecialCharTok{$}\NormalTok{ykm\_m)}
\NormalTok{sp\_points }\OtherTok{\textless{}{-}} \FunctionTok{SpatialPoints}\NormalTok{(coords, }\AttributeTok{proj4string =}\NormalTok{ utm\_proj)}

\CommentTok{\# Преобразование в WGS84 (широта/долгота)}
\NormalTok{wgs84\_proj }\OtherTok{\textless{}{-}} \FunctionTok{CRS}\NormalTok{(}\StringTok{"+proj=longlat +datum=WGS84"}\NormalTok{)}
\NormalTok{sp\_points\_latlon }\OtherTok{\textless{}{-}} \FunctionTok{spTransform}\NormalTok{(sp\_points, wgs84\_proj)}

\CommentTok{\# Добавление координат в основной датафрейм}
\NormalTok{RASP}\SpecialCharTok{$}\NormalTok{X }\OtherTok{\textless{}{-}} \FunctionTok{coordinates}\NormalTok{(sp\_points\_latlon)[, }\DecValTok{1}\NormalTok{]  }\CommentTok{\# Долгота}
\NormalTok{RASP}\SpecialCharTok{$}\NormalTok{Y }\OtherTok{\textless{}{-}} \FunctionTok{coordinates}\NormalTok{(sp\_points\_latlon)[, }\DecValTok{2}\NormalTok{]  }\CommentTok{\# Широта}

\CommentTok{\# Удаление временных столбцов}
\NormalTok{RASP}\SpecialCharTok{$}\NormalTok{xkm\_m }\OtherTok{\textless{}{-}} \ConstantTok{NULL}
\NormalTok{RASP}\SpecialCharTok{$}\NormalTok{ykm\_m }\OtherTok{\textless{}{-}} \ConstantTok{NULL}

\CommentTok{\# Проверка структуры результата}
\FunctionTok{str}\NormalTok{(RASP)}

\CommentTok{\# {-}{-}{-}{-}{-}{-}{-}{-}{-}{-}{-}{-}{-}{-}{-}{-}{-}{-}{-}{-}{-}{-}{-}{-}{-}{-}{-}{-}{-}{-}{-}{-}{-}{-}{-}{-}{-}{-}{-}{-}{-}{-}{-}{-}{-}}
\CommentTok{\# 11. ВИЗУАЛИЗАЦИЯ РЕЗУЛЬТАТОВ (КАРТА)}
\CommentTok{\# {-}{-}{-}{-}{-}{-}{-}{-}{-}{-}{-}{-}{-}{-}{-}{-}{-}{-}{-}{-}{-}{-}{-}{-}{-}{-}{-}{-}{-}{-}{-}{-}{-}{-}{-}{-}{-}{-}{-}{-}{-}{-}{-}{-}{-}}

\CommentTok{\# Загрузка картографических данных}
\NormalTok{world }\OtherTok{\textless{}{-}} \FunctionTok{ne\_countries}\NormalTok{(}\AttributeTok{scale =} \StringTok{"medium"}\NormalTok{, }\AttributeTok{returnclass =} \StringTok{"sf"}\NormalTok{)}

\CommentTok{\# Определение региона интереса (Арктика России)}
\NormalTok{arctic\_bbox }\OtherTok{\textless{}{-}} \FunctionTok{st\_bbox}\NormalTok{(}\FunctionTok{c}\NormalTok{(}\AttributeTok{xmin =} \DecValTok{25}\NormalTok{, }\AttributeTok{xmax =} \DecValTok{70}\NormalTok{, }\AttributeTok{ymin =} \DecValTok{65}\NormalTok{, }\AttributeTok{ymax =} \DecValTok{80}\NormalTok{), }\AttributeTok{crs =} \DecValTok{4326}\NormalTok{)}
\NormalTok{arctic }\OtherTok{\textless{}{-}} \FunctionTok{st\_crop}\NormalTok{(world, arctic\_bbox)}

\CommentTok{\# Кастомные разрывы для цветовой шкалы}
\NormalTok{my\_breaks }\OtherTok{\textless{}{-}} \FunctionTok{c}\NormalTok{(}\FloatTok{0.001}\NormalTok{, }\FloatTok{0.1}\NormalTok{, }\DecValTok{1}\NormalTok{, }\DecValTok{200}\NormalTok{, }\DecValTok{10000}\NormalTok{)}

\CommentTok{\# Создание основной визуализации}
\FunctionTok{ggplot}\NormalTok{() }\SpecialCharTok{+}
  \CommentTok{\# Теплокарта плотности}
  \FunctionTok{geom\_point}\NormalTok{(}
    \AttributeTok{data =}\NormalTok{ RASP, }
    \FunctionTok{aes}\NormalTok{(}\AttributeTok{x =}\NormalTok{ X, }\AttributeTok{y =}\NormalTok{ Y, }\AttributeTok{color =} \FunctionTok{exp}\NormalTok{(est)), }
    \AttributeTok{size =} \FloatTok{0.8}\NormalTok{, }
    \AttributeTok{alpha =} \FloatTok{0.7}
\NormalTok{  ) }\SpecialCharTok{+} 
  \CommentTok{\# Наблюдаемые точки данных}
  \FunctionTok{geom\_point}\NormalTok{(}
    \AttributeTok{data =}\NormalTok{ data, }
    \FunctionTok{aes}\NormalTok{(}\AttributeTok{x =}\NormalTok{ X, }\AttributeTok{y =}\NormalTok{ Y, }\AttributeTok{size =}\NormalTok{ PROM), }\CommentTok{\# Размер по плотности}
    \AttributeTok{color =} \StringTok{"black"}\NormalTok{, }
    \AttributeTok{fill =} \ConstantTok{NA}\NormalTok{, }
    \AttributeTok{alpha =} \FloatTok{0.6}\NormalTok{,}
    \AttributeTok{shape =} \DecValTok{21} \CommentTok{\# Кружки с обводкой}
\NormalTok{  ) }\SpecialCharTok{+}
  \CommentTok{\# Картографическая подложка}
  \FunctionTok{geom\_sf}\NormalTok{(}\AttributeTok{data =}\NormalTok{ arctic, }\AttributeTok{fill =} \StringTok{"lightgrey"}\NormalTok{, }\AttributeTok{color =} \StringTok{"darkgrey"}\NormalTok{) }\SpecialCharTok{+}
  \CommentTok{\# Цветовая шкала (логарифмическая)}
  \FunctionTok{scale\_color\_viridis\_c}\NormalTok{(}
    \AttributeTok{name =} \StringTok{""}\NormalTok{,}
    \AttributeTok{option =} \StringTok{"H"}\NormalTok{, }
    \AttributeTok{trans =} \StringTok{"log"}\NormalTok{, }
    \AttributeTok{breaks =}\NormalTok{ my\_breaks, }
    \AttributeTok{labels =}\NormalTok{ my\_breaks}
\NormalTok{  ) }\SpecialCharTok{+}
  \CommentTok{\# Разделение по годам}
  \FunctionTok{facet\_wrap}\NormalTok{(}\SpecialCharTok{\textasciitilde{}}\NormalTok{ YEAR, }\AttributeTok{ncol =} \DecValTok{2}\NormalTok{) }\SpecialCharTok{+}
  \CommentTok{\# Настройка области просмотра}
  \FunctionTok{coord\_sf}\NormalTok{(}
    \AttributeTok{xlim =} \FunctionTok{c}\NormalTok{(}\FunctionTok{min}\NormalTok{(RASP}\SpecialCharTok{$}\NormalTok{X)}\SpecialCharTok{{-}}\DecValTok{1}\NormalTok{, }\FunctionTok{max}\NormalTok{(RASP}\SpecialCharTok{$}\NormalTok{X)}\SpecialCharTok{+}\DecValTok{1}\NormalTok{),}
    \AttributeTok{ylim =} \FunctionTok{c}\NormalTok{(}\FunctionTok{min}\NormalTok{(RASP}\SpecialCharTok{$}\NormalTok{Y)}\SpecialCharTok{{-}}\FloatTok{0.5}\NormalTok{, }\FunctionTok{max}\NormalTok{(RASP}\SpecialCharTok{$}\NormalTok{Y)}\SpecialCharTok{+}\FloatTok{0.5}\NormalTok{),}
    \AttributeTok{crs =} \DecValTok{4326}
\NormalTok{  ) }\SpecialCharTok{+}
  \CommentTok{\# Тема оформления}
  \FunctionTok{theme\_bw}\NormalTok{(}\AttributeTok{base\_size =} \DecValTok{12}\NormalTok{) }\SpecialCharTok{+}
  \FunctionTok{labs}\NormalTok{(}\AttributeTok{x =} \StringTok{"Долгота"}\NormalTok{, }\AttributeTok{y =} \StringTok{"Широта"}\NormalTok{, }\AttributeTok{title =} \StringTok{"Пространственное распределение плотности"}\NormalTok{) }\SpecialCharTok{+}
  \FunctionTok{theme}\NormalTok{(}
    \AttributeTok{panel.grid =} \FunctionTok{element\_line}\NormalTok{(}\AttributeTok{color =} \StringTok{"grey90"}\NormalTok{),}
    \AttributeTok{legend.position =} \StringTok{"bottom"}\NormalTok{,}
    \AttributeTok{legend.key.width =} \FunctionTok{unit}\NormalTok{(}\FloatTok{1.2}\NormalTok{, }\StringTok{"cm"}\NormalTok{),}
    \AttributeTok{strip.background =} \FunctionTok{element\_rect}\NormalTok{(}\AttributeTok{fill =} \StringTok{"white"}\NormalTok{)}
\NormalTok{  )}

\CommentTok{\# Сохранение графика (раскомментируйте)}
\CommentTok{\# ggsave("sdmTMBmap10.jpg", width = 8, height = 8, dpi = 300)}
\end{Highlighting}
\end{Shaded}

\begin{figure}[H]

{\centering \includegraphics[width=0.9\linewidth,height=\textheight,keepaspectratio]{images/sdmTMBmap10.jpg}

}

\caption{Рис. 7.: Визуализация результатов (КАРТА)}

\end{figure}%

\begin{Shaded}
\begin{Highlighting}[]
\CommentTok{\# {-}{-}{-}{-}{-}{-}{-}{-}{-}{-}{-}{-}{-}{-}{-}{-}{-}{-}{-}{-}{-}{-}{-}{-}{-}{-}{-}{-}{-}{-}{-}{-}{-}{-}{-}{-}{-}{-}{-}{-}{-}{-}{-}{-}{-}{-}{-}{-}{-}{-}{-}}
\CommentTok{\# 12. РАСЧЕТ ИНДЕКСОВ ОБИЛИЯ}
\CommentTok{\# {-}{-}{-}{-}{-}{-}{-}{-}{-}{-}{-}{-}{-}{-}{-}{-}{-}{-}{-}{-}{-}{-}{-}{-}{-}{-}{-}{-}{-}{-}{-}{-}{-}{-}{-}{-}{-}{-}{-}{-}{-}{-}{-}{-}{-}{-}{-}{-}{-}{-}{-}}

\CommentTok{\# Расчет индексов с разными доверительными интервалами}
\NormalTok{index }\OtherTok{\textless{}{-}} \FunctionTok{get\_index}\NormalTok{(predictions, }\AttributeTok{area =} \DecValTok{4}\NormalTok{, }\AttributeTok{level =} \FloatTok{0.95}\NormalTok{, }\AttributeTok{bias\_correct =} \ConstantTok{TRUE}\NormalTok{)}
\NormalTok{index2 }\OtherTok{\textless{}{-}} \FunctionTok{get\_index}\NormalTok{(predictions, }\AttributeTok{area =} \DecValTok{4}\NormalTok{, }\AttributeTok{level =} \FloatTok{0.5}\NormalTok{, }\AttributeTok{bias\_correct =} \ConstantTok{TRUE}\NormalTok{)}

\CommentTok{\# Формирование сводной таблицы результатов}
\NormalTok{total }\OtherTok{\textless{}{-}} \FunctionTok{data.frame}\NormalTok{(}
  \AttributeTok{YEAR =}\NormalTok{ index}\SpecialCharTok{$}\NormalTok{YEAR,}
  \AttributeTok{lwr\_95 =}\NormalTok{ index}\SpecialCharTok{$}\NormalTok{lwr,}
  \AttributeTok{lwr\_50 =}\NormalTok{ index2}\SpecialCharTok{$}\NormalTok{lwr,}
  \AttributeTok{estimate =}\NormalTok{ index}\SpecialCharTok{$}\NormalTok{est,}
  \AttributeTok{upr\_50 =}\NormalTok{ index2}\SpecialCharTok{$}\NormalTok{upr,}
  \AttributeTok{upr\_95 =}\NormalTok{ index}\SpecialCharTok{$}\NormalTok{upr,}
  \AttributeTok{se =}\NormalTok{ index}\SpecialCharTok{$}\NormalTok{se,}
  \AttributeTok{cv =} \FunctionTok{sqrt}\NormalTok{(}\FunctionTok{exp}\NormalTok{(index}\SpecialCharTok{$}\NormalTok{se}\SpecialCharTok{\^{}}\DecValTok{2}\NormalTok{) }\SpecialCharTok{{-}} \DecValTok{1}\NormalTok{) }\CommentTok{\# Коэффициент вариации}
\NormalTok{)}

\CommentTok{\# Визуализация индексов обилия}
\FunctionTok{ggplot}\NormalTok{(total, }\FunctionTok{aes}\NormalTok{(}\AttributeTok{x =}\NormalTok{ YEAR, }\AttributeTok{y =}\NormalTok{ estimate}\SpecialCharTok{/}\DecValTok{1000000}\NormalTok{)) }\SpecialCharTok{+} 
  \CommentTok{\# Основная линия оценки}
  \FunctionTok{geom\_line}\NormalTok{(}\AttributeTok{linewidth =} \DecValTok{1}\NormalTok{, }\AttributeTok{color =} \StringTok{"steelblue"}\NormalTok{) }\SpecialCharTok{+}
  
  \CommentTok{\# 95\% доверительный интервал (более широкий и прозрачный)}
  \FunctionTok{geom\_ribbon}\NormalTok{(}
    \FunctionTok{aes}\NormalTok{(}\AttributeTok{ymin =}\NormalTok{ lwr\_95}\SpecialCharTok{/}\DecValTok{1000000}\NormalTok{, }\AttributeTok{ymax =}\NormalTok{ upr\_95}\SpecialCharTok{/}\DecValTok{1000000}\NormalTok{),}
    \AttributeTok{alpha =} \FloatTok{0.2}\NormalTok{,  }\CommentTok{\# Полупрозрачность}
    \AttributeTok{fill =} \StringTok{"steelblue"}\NormalTok{,}
    \AttributeTok{color =} \ConstantTok{NA}     \CommentTok{\# Без контура}
\NormalTok{  ) }\SpecialCharTok{+}
  
  \CommentTok{\# 50\% доверительный интервал (менее прозрачный)}
  \FunctionTok{geom\_ribbon}\NormalTok{(}
    \FunctionTok{aes}\NormalTok{(}\AttributeTok{ymin =}\NormalTok{ lwr\_50}\SpecialCharTok{/}\DecValTok{1000000}\NormalTok{, }\AttributeTok{ymax =}\NormalTok{ upr\_50}\SpecialCharTok{/}\DecValTok{1000000}\NormalTok{),}
    \AttributeTok{alpha =} \FloatTok{0.4}\NormalTok{,  }\CommentTok{\# Меньшая прозрачность}
    \AttributeTok{fill =} \StringTok{"steelblue"}\NormalTok{,}
    \AttributeTok{color =} \ConstantTok{NA}
\NormalTok{  ) }\SpecialCharTok{+}
  
  \CommentTok{\# Настройки осей и заголовков}
  \FunctionTok{ylab}\NormalTok{(}\StringTok{\textquotesingle{}Промысловый запас, млн. экз\textquotesingle{}}\NormalTok{) }\SpecialCharTok{+}
  \FunctionTok{xlab}\NormalTok{(}\StringTok{\textquotesingle{}Год\textquotesingle{}}\NormalTok{) }\SpecialCharTok{+}
  
  \CommentTok{\# Вертикальные линии для годов}
  \FunctionTok{geom\_vline}\NormalTok{(}
    \AttributeTok{xintercept =}\NormalTok{ total}\SpecialCharTok{$}\NormalTok{YEAR, }
    \AttributeTok{linetype =} \StringTok{"dotted"}\NormalTok{, }
    \AttributeTok{color =} \StringTok{"grey60"}\NormalTok{, }
    \AttributeTok{alpha =} \FloatTok{0.6}
\NormalTok{  ) }\SpecialCharTok{+}
  
  \CommentTok{\# Точки с значениями оценок}
  \FunctionTok{geom\_point}\NormalTok{(}
    \AttributeTok{size =} \DecValTok{3}\NormalTok{,}
    \AttributeTok{color =} \StringTok{"navyblue"}\NormalTok{,}
    \AttributeTok{fill =} \StringTok{"white"}\NormalTok{,}
    \AttributeTok{shape =} \DecValTok{21}
\NormalTok{  ) }\SpecialCharTok{+}
  
  \CommentTok{\# Настройка темы}
  \FunctionTok{theme\_minimal}\NormalTok{(}\AttributeTok{base\_size =} \DecValTok{14}\NormalTok{) }\SpecialCharTok{+}
  \FunctionTok{theme}\NormalTok{(}
    \AttributeTok{plot.title =} \FunctionTok{element\_text}\NormalTok{(}\AttributeTok{hjust =} \FloatTok{0.5}\NormalTok{, }\AttributeTok{face =} \StringTok{"bold"}\NormalTok{),}
    \AttributeTok{panel.grid.minor =} \FunctionTok{element\_blank}\NormalTok{(),}
    \AttributeTok{panel.grid.major =} \FunctionTok{element\_line}\NormalTok{(}\AttributeTok{color =} \StringTok{"grey90"}\NormalTok{),}
    \AttributeTok{axis.line =} \FunctionTok{element\_line}\NormalTok{(}\AttributeTok{color =} \StringTok{"grey30"}\NormalTok{),}
    \AttributeTok{legend.position =} \StringTok{"none"}
\NormalTok{  )}
\end{Highlighting}
\end{Shaded}

\begin{figure}[H]

{\centering \includegraphics[width=0.6\linewidth,height=\textheight,keepaspectratio]{images/sdmTMB7.PNG}

}

\caption{Рис. 8.: Визуализация индексов обилия}

\end{figure}%

\begin{Shaded}
\begin{Highlighting}[]
\CommentTok{\# Форматированный вывод результатов}
\NormalTok{total }\SpecialCharTok{\%\textgreater{}\%} 
  \FunctionTok{mutate}\NormalTok{(}\AttributeTok{cv\_percent =} \DecValTok{100} \SpecialCharTok{*}\NormalTok{ cv) }\SpecialCharTok{\%\textgreater{}\%} 
  \FunctionTok{select}\NormalTok{(}
\NormalTok{    YEAR, }
\NormalTok{    estimate, }
\NormalTok{    lwr\_50,  }\CommentTok{\# Нижняя граница 50\% ДИ}
\NormalTok{    upr\_50,  }\CommentTok{\# Верхняя граница 50\% ДИ}
\NormalTok{    lwr\_95,  }\CommentTok{\# Нижняя граница 95\% ДИ}
\NormalTok{    upr\_95,  }\CommentTok{\# Верхняя граница 95\% ДИ}
\NormalTok{    cv\_percent}
\NormalTok{  ) }\SpecialCharTok{\%\textgreater{}\%}
\NormalTok{  knitr}\SpecialCharTok{::}\FunctionTok{kable}\NormalTok{(}
    \AttributeTok{format =} \StringTok{"pandoc"}\NormalTok{, }
    \AttributeTok{digits =} \FunctionTok{c}\NormalTok{(}\DecValTok{0}\NormalTok{, }\DecValTok{0}\NormalTok{, }\DecValTok{0}\NormalTok{, }\DecValTok{0}\NormalTok{, }\DecValTok{0}\NormalTok{, }\DecValTok{0}\NormalTok{, }\DecValTok{1}\NormalTok{),}
    \AttributeTok{col.names =} \FunctionTok{c}\NormalTok{(}
      \StringTok{"Год"}\NormalTok{, }
      \StringTok{"Оценка"}\NormalTok{, }
      \StringTok{"Нижняя 50\%"}\NormalTok{, }
      \StringTok{"Верхняя 50\%"}\NormalTok{, }
      \StringTok{"Нижняя 95\%"}\NormalTok{, }
      \StringTok{"Верхняя 95\%"}\NormalTok{, }
      \StringTok{"CV\%"}
\NormalTok{    )}
\NormalTok{  )}
\end{Highlighting}
\end{Shaded}

\begin{Shaded}
\begin{Highlighting}[]
\NormalTok{  Год    Оценка   Нижняя }\DecValTok{50}\SpecialCharTok{\%   Верхняя 50\%}\NormalTok{   Нижняя }\DecValTok{95}\SpecialCharTok{\%   Верхняя 95\%}\NormalTok{    CV\%}
\SpecialCharTok{{-}{-}{-}{-}{-}}  \SpecialCharTok{{-}{-}{-}{-}{-}{-}{-}{-}}  \SpecialCharTok{{-}{-}{-}{-}{-}{-}{-}{-}{-}{-}{-}}  \SpecialCharTok{{-}{-}{-}{-}{-}{-}{-}{-}{-}{-}{-}{-}}  \SpecialCharTok{{-}{-}{-}{-}{-}{-}{-}{-}{-}{-}{-}}  \SpecialCharTok{{-}{-}{-}{-}{-}{-}{-}{-}{-}{-}{-}{-}}  \SpecialCharTok{{-}{-}{-}{-}{-}}
 \DecValTok{2019}   \DecValTok{2381774}      \DecValTok{2177448}       \DecValTok{2605274}      \DecValTok{1835312}       \DecValTok{3090946}   \FloatTok{13.4}
 \DecValTok{2020}   \DecValTok{1634549}      \DecValTok{1539111}       \DecValTok{1735906}      \DecValTok{1372377}       \DecValTok{1946805}    \FloatTok{8.9}
 \DecValTok{2021}   \DecValTok{1920507}      \DecValTok{1794122}       \DecValTok{2055795}      \DecValTok{1575823}       \DecValTok{2340584}   \FloatTok{10.1}
 \DecValTok{2022}   \DecValTok{1036673}       \DecValTok{959251}       \DecValTok{1120344}       \DecValTok{827345}       \DecValTok{1298963}   \FloatTok{11.5}
 \DecValTok{2023}   \DecValTok{1147685}      \DecValTok{1068401}       \DecValTok{1232853}       \DecValTok{932147}       \DecValTok{1413062}   \FloatTok{10.6}
 \DecValTok{2024}   \DecValTok{1055733}       \DecValTok{985624}       \DecValTok{1130829}       \DecValTok{864640}       \DecValTok{1289060}   \FloatTok{10.2}
\SpecialCharTok{\textgreater{}} 
\end{Highlighting}
\end{Shaded}

\section{Базовая оценка +
предикторы}\label{ux431ux430ux437ux43eux432ux430ux44f-ux43eux446ux435ux43dux43aux430-ux43fux440ux435ux434ux438ux43aux442ux43eux440ux44b}

Сравнение пространственно-временных моделей sdmTMB с учетом типа съемки
(SURV) и года (YEAR)

Рассмотрим 4 пространственно-временные модели, оценивая их по:

Качеству подгонки (AIC) Стабильности оценок (sanity check) Значимости
ковариат Биологическому смыслу

4 модели: базовая модель, модель с глубиной (DEPTH),модель с
температурой (TEMP), модель с расстоянием до берега (DIST)

\begin{Shaded}
\begin{Highlighting}[]
\SpecialCharTok{\textgreater{}} \CommentTok{\# 8. ПОСТРОЕНИЕ ПРОСТРАНСТВЕННО{-}ВРЕМЕННОЙ МОДЕЛИ}
\ErrorTok{\textgreater{}} \CommentTok{\# {-}{-}{-}{-}{-}{-}{-}{-}{-}{-}{-}{-}{-}{-}{-}{-}{-}{-}{-}{-}{-}{-}{-}{-}{-}{-}{-}{-}{-}{-}{-}{-}{-}{-}{-}{-}{-}{-}{-}{-}{-}{-}{-}{-}{-}{-}{-}{-}{-}{-}{-}}
\ErrorTok{\textgreater{}} 
\ErrorTok{\textgreater{}}\NormalTok{ m }\OtherTok{\textless{}{-}} \FunctionTok{sdmTMB}\NormalTok{(}
\SpecialCharTok{+}   \AttributeTok{data =}\NormalTok{ data, }
\SpecialCharTok{+}   \AttributeTok{formula =}\NormalTok{ Density }\SpecialCharTok{\textasciitilde{}} \DecValTok{0}\SpecialCharTok{+} \FunctionTok{as.factor}\NormalTok{(SURV) }\SpecialCharTok{+} \FunctionTok{as.factor}\NormalTok{(YEAR),  }\CommentTok{\# Формула: плотность зависит от года}
\SpecialCharTok{+}   \AttributeTok{time =} \StringTok{"YEAR"}\NormalTok{,         }\CommentTok{\# Временная переменная}
\SpecialCharTok{+}   \AttributeTok{mesh =}\NormalTok{ mesh\_sdm,       }\CommentTok{\# Пространственная сетка}
\SpecialCharTok{+}   \AttributeTok{family =} \FunctionTok{tweedie}\NormalTok{(}\AttributeTok{link =} \StringTok{"log"}\NormalTok{),  }\CommentTok{\# Статистическое распределение}
\SpecialCharTok{+}   \AttributeTok{spatial =} \StringTok{"on"}\NormalTok{,        }\CommentTok{\# Включение пространственных эффектов}
\SpecialCharTok{+}   \AttributeTok{spatiotemporal =} \StringTok{"iid"} \CommentTok{\# Пространственно{-}временные эффекты}
\SpecialCharTok{+}\NormalTok{ )}
\SpecialCharTok{\textgreater{}} 
\ErrorTok{\textgreater{}} 
\ErrorTok{\textgreater{}} \CommentTok{\# Вывод результатов модели}
\ErrorTok{\textgreater{}} \FunctionTok{summary}\NormalTok{(m)}
\NormalTok{Spatiotemporal model fit by ML [}\StringTok{\textquotesingle{}sdmTMB\textquotesingle{}}\NormalTok{]}
\NormalTok{Formula}\SpecialCharTok{:}\NormalTok{ Density }\SpecialCharTok{\textasciitilde{}} \DecValTok{0} \SpecialCharTok{+} \FunctionTok{as.factor}\NormalTok{(SURV) }\SpecialCharTok{+} \FunctionTok{as.factor}\NormalTok{(YEAR)}
\NormalTok{Mesh}\SpecialCharTok{:} \FunctionTok{mesh\_sdm}\NormalTok{ (isotropic covariance)}
\NormalTok{Time column}\SpecialCharTok{:}\NormalTok{ YEAR}
\NormalTok{Data}\SpecialCharTok{:}\NormalTok{ data}
\NormalTok{Family}\SpecialCharTok{:} \FunctionTok{tweedie}\NormalTok{(}\AttributeTok{link =} \StringTok{\textquotesingle{}log\textquotesingle{}}\NormalTok{)}
 
\NormalTok{Conditional model}\SpecialCharTok{:}
\NormalTok{                    coef.est coef.se}
\FunctionTok{as.factor}\NormalTok{(SURV)CRAB     }\FloatTok{4.75}    \FloatTok{0.44}
\FunctionTok{as.factor}\NormalTok{(SURV)SUM      }\FloatTok{2.54}    \FloatTok{0.37}
\FunctionTok{as.factor}\NormalTok{(YEAR)}\DecValTok{2020}    \SpecialCharTok{{-}}\FloatTok{0.57}    \FloatTok{0.36}
\FunctionTok{as.factor}\NormalTok{(YEAR)}\DecValTok{2021}    \SpecialCharTok{{-}}\FloatTok{0.20}    \FloatTok{0.36}
\FunctionTok{as.factor}\NormalTok{(YEAR)}\DecValTok{2022}    \SpecialCharTok{{-}}\FloatTok{0.61}    \FloatTok{0.36}
\FunctionTok{as.factor}\NormalTok{(YEAR)}\DecValTok{2023}    \SpecialCharTok{{-}}\FloatTok{0.59}    \FloatTok{0.36}
\FunctionTok{as.factor}\NormalTok{(YEAR)}\DecValTok{2024}    \SpecialCharTok{{-}}\FloatTok{0.85}    \FloatTok{0.36}

\NormalTok{Dispersion parameter}\SpecialCharTok{:} \FloatTok{23.16}
\NormalTok{Tweedie p}\SpecialCharTok{:} \FloatTok{1.41}
\NormalTok{Matern range}\SpecialCharTok{:} \FloatTok{63.53}
\NormalTok{Spatial SD}\SpecialCharTok{:} \FloatTok{1.22}
\NormalTok{Spatiotemporal IID SD}\SpecialCharTok{:} \FloatTok{0.97}
\NormalTok{ML criterion at convergence}\SpecialCharTok{:} \FloatTok{5914.655}

\NormalTok{See ?tidy.sdmTMB to extract these values as a data frame.}
\SpecialCharTok{\textgreater{}} \FunctionTok{AIC}\NormalTok{(m)  }\CommentTok{\# Критерий Акаике}
\NormalTok{[}\DecValTok{1}\NormalTok{] }\FloatTok{11853.31}
\SpecialCharTok{\textgreater{}} \FunctionTok{sanity}\NormalTok{(m)  }\CommentTok{\# Проверка корректности модели}
\NormalTok{v Non}\SpecialCharTok{{-}}\NormalTok{linear minimizer suggests successful convergence}
\NormalTok{v Hessian matrix is positive definite}
\NormalTok{v No extreme or very small eigenvalues detected}
\NormalTok{v No gradients with respect to fixed effects are }\SpecialCharTok{\textgreater{}=} \FloatTok{0.001}
\NormalTok{v No fixed}\SpecialCharTok{{-}}\NormalTok{effect standard errors are }\ConstantTok{NA}
\NormalTok{v No standard errors look unreasonably large}
\NormalTok{v No sigma parameters are }\SpecialCharTok{\textless{}} \FloatTok{0.01}
\NormalTok{v No sigma parameters are }\SpecialCharTok{\textgreater{}} \DecValTok{100}
\NormalTok{v Range parameter doesn}\StringTok{\textquotesingle{}t look unreasonably large}
\StringTok{\textgreater{} }
\StringTok{\textgreater{} md \textless{}{-} sdmTMB(}
\StringTok{+   data = data, }
\StringTok{+   formula = Density \textasciitilde{} 0+ as.factor(SURV) + as.factor(YEAR)+s(DEPTH),  \# Формула: плотность зависит от года}
\StringTok{+   time = "YEAR",         \# Временная переменная}
\StringTok{+   mesh = mesh\_sdm,       \# Пространственная сетка}
\StringTok{+   family = tweedie(link = "log"),  \# Статистическое распределение}
\StringTok{+   spatial = "on",        \# Включение пространственных эффектов}
\StringTok{+   spatiotemporal = "iid" \# Пространственно{-}временные эффекты}
\StringTok{+ )}
\StringTok{\textgreater{} }
\StringTok{\textgreater{} }
\StringTok{\textgreater{} \# Вывод результатов модели}
\StringTok{\textgreater{} summary(md)}
\StringTok{Spatiotemporal model fit by ML [\textquotesingle{}}\NormalTok{sdmTMB}\StringTok{\textquotesingle{}]}
\StringTok{Formula: Density \textasciitilde{} 0 + as.factor(SURV) + as.factor(YEAR) + s(DEPTH)}
\StringTok{Mesh: mesh\_sdm (isotropic covariance)}
\StringTok{Time column: YEAR}
\StringTok{Data: data}
\StringTok{Family: tweedie(link = \textquotesingle{}}\NormalTok{log}\StringTok{\textquotesingle{})}
\StringTok{ }
\StringTok{Conditional model:}
\StringTok{                    coef.est coef.se}
\StringTok{as.factor(SURV)CRAB     5.42    0.32}
\StringTok{as.factor(SURV)SUM      3.09    0.28}
\StringTok{as.factor(YEAR)2020    {-}0.52    0.27}
\StringTok{as.factor(YEAR)2021    {-}0.15    0.27}
\StringTok{as.factor(YEAR)2022    {-}0.67    0.27}
\StringTok{as.factor(YEAR)2023    {-}0.63    0.27}
\StringTok{as.factor(YEAR)2024    {-}0.93    0.27}
\StringTok{sDEPTH                 {-}0.60    0.42}

\StringTok{Smooth terms:}
\StringTok{           Std. Dev.}
\StringTok{sds(DEPTH)      1.71}

\StringTok{Dispersion parameter: 24.43}
\StringTok{Tweedie p: 1.39}
\StringTok{Matern range: 40.20}
\StringTok{Spatial SD: 0.97}
\StringTok{Spatiotemporal IID SD: 0.94}
\StringTok{ML criterion at convergence: 5907.365}

\StringTok{See ?tidy.sdmTMB to extract these values as a data frame.}
\StringTok{\textgreater{} AIC(md)  \# Критерий Акаике}
\StringTok{[1] 11842.73}
\StringTok{\textgreater{} sanity(md)  \# Проверка корректности модели}
\StringTok{v Non{-}linear minimizer suggests successful convergence}
\StringTok{v Hessian matrix is positive definite}
\StringTok{v No extreme or very small eigenvalues detected}
\StringTok{v No gradients with respect to fixed effects are \textgreater{}= 0.001}
\StringTok{v No fixed{-}effect standard errors are NA}
\StringTok{v No standard errors look unreasonably large}
\StringTok{v No sigma parameters are \textless{} 0.01}
\StringTok{v No sigma parameters are \textgreater{} 100}
\StringTok{v Range parameter doesn\textquotesingle{}}\NormalTok{t look unreasonably large}
\SpecialCharTok{\textgreater{}} 
\ErrorTok{\textgreater{}} 
\ErrorTok{\textgreater{}}\NormalTok{ mt }\OtherTok{\textless{}{-}} \FunctionTok{sdmTMB}\NormalTok{(}
\SpecialCharTok{+}   \AttributeTok{data =}\NormalTok{ data, }
\SpecialCharTok{+}   \AttributeTok{formula =}\NormalTok{ Density }\SpecialCharTok{\textasciitilde{}} \DecValTok{0}\SpecialCharTok{+} \FunctionTok{as.factor}\NormalTok{(SURV) }\SpecialCharTok{+} \FunctionTok{as.factor}\NormalTok{(YEAR)}\SpecialCharTok{+}\FunctionTok{s}\NormalTok{(TEMP),  }\CommentTok{\# Формула: плотность зависит от года}
\SpecialCharTok{+}   \AttributeTok{time =} \StringTok{"YEAR"}\NormalTok{,         }\CommentTok{\# Временная переменная}
\SpecialCharTok{+}   \AttributeTok{mesh =}\NormalTok{ mesh\_sdm,       }\CommentTok{\# Пространственная сетка}
\SpecialCharTok{+}   \AttributeTok{family =} \FunctionTok{tweedie}\NormalTok{(}\AttributeTok{link =} \StringTok{"log"}\NormalTok{),  }\CommentTok{\# Статистическое распределение}
\SpecialCharTok{+}   \AttributeTok{spatial =} \StringTok{"on"}\NormalTok{,        }\CommentTok{\# Включение пространственных эффектов}
\SpecialCharTok{+}   \AttributeTok{spatiotemporal =} \StringTok{"iid"} \CommentTok{\# Пространственно{-}временные эффекты}
\SpecialCharTok{+}\NormalTok{ )}
\SpecialCharTok{\textgreater{}} 
\ErrorTok{\textgreater{}} 
\ErrorTok{\textgreater{}} \CommentTok{\# Вывод результатов модели}
\ErrorTok{\textgreater{}} \FunctionTok{summary}\NormalTok{(mt)}
\NormalTok{Spatiotemporal model fit by ML [}\StringTok{\textquotesingle{}sdmTMB\textquotesingle{}}\NormalTok{]}
\NormalTok{Formula}\SpecialCharTok{:}\NormalTok{ Density }\SpecialCharTok{\textasciitilde{}} \DecValTok{0} \SpecialCharTok{+} \FunctionTok{as.factor}\NormalTok{(SURV) }\SpecialCharTok{+} \FunctionTok{as.factor}\NormalTok{(YEAR) }\SpecialCharTok{+} \FunctionTok{s}\NormalTok{(TEMP)}
\NormalTok{Mesh}\SpecialCharTok{:} \FunctionTok{mesh\_sdm}\NormalTok{ (isotropic covariance)}
\NormalTok{Time column}\SpecialCharTok{:}\NormalTok{ YEAR}
\NormalTok{Data}\SpecialCharTok{:}\NormalTok{ data}
\NormalTok{Family}\SpecialCharTok{:} \FunctionTok{tweedie}\NormalTok{(}\AttributeTok{link =} \StringTok{\textquotesingle{}log\textquotesingle{}}\NormalTok{)}
 
\NormalTok{Conditional model}\SpecialCharTok{:}
\NormalTok{                    coef.est coef.se}
\FunctionTok{as.factor}\NormalTok{(SURV)CRAB     }\FloatTok{4.95}    \FloatTok{0.40}
\FunctionTok{as.factor}\NormalTok{(SURV)SUM      }\FloatTok{2.69}    \FloatTok{0.34}
\FunctionTok{as.factor}\NormalTok{(YEAR)}\DecValTok{2020}    \SpecialCharTok{{-}}\FloatTok{0.14}    \FloatTok{0.43}
\FunctionTok{as.factor}\NormalTok{(YEAR)}\DecValTok{2021}    \SpecialCharTok{{-}}\FloatTok{0.19}    \FloatTok{0.33}
\FunctionTok{as.factor}\NormalTok{(YEAR)}\DecValTok{2022}    \SpecialCharTok{{-}}\FloatTok{0.77}    \FloatTok{0.42}
\FunctionTok{as.factor}\NormalTok{(YEAR)}\DecValTok{2023}    \SpecialCharTok{{-}}\FloatTok{0.62}    \FloatTok{0.33}
\FunctionTok{as.factor}\NormalTok{(YEAR)}\DecValTok{2024}    \SpecialCharTok{{-}}\FloatTok{0.91}    \FloatTok{0.34}
\NormalTok{sTEMP                   }\FloatTok{0.80}    \FloatTok{0.83}

\NormalTok{Smooth terms}\SpecialCharTok{:}
\NormalTok{          Std. Dev.}
\FunctionTok{sds}\NormalTok{(TEMP)      }\FloatTok{3.15}

\NormalTok{Dispersion parameter}\SpecialCharTok{:} \FloatTok{23.42}
\NormalTok{Tweedie p}\SpecialCharTok{:} \FloatTok{1.40}
\NormalTok{Matern range}\SpecialCharTok{:} \FloatTok{55.04}
\NormalTok{Spatial SD}\SpecialCharTok{:} \FloatTok{1.12}
\NormalTok{Spatiotemporal IID SD}\SpecialCharTok{:} \FloatTok{0.96}
\NormalTok{ML criterion at convergence}\SpecialCharTok{:} \FloatTok{5912.795}

\NormalTok{See ?tidy.sdmTMB to extract these values as a data frame.}
\SpecialCharTok{\textgreater{}} \FunctionTok{AIC}\NormalTok{(mt)  }\CommentTok{\# Критерий Акаике}
\NormalTok{[}\DecValTok{1}\NormalTok{] }\FloatTok{11853.59}
\SpecialCharTok{\textgreater{}} \FunctionTok{sanity}\NormalTok{(mt)  }\CommentTok{\# Проверка корректности модели}
\NormalTok{v Non}\SpecialCharTok{{-}}\NormalTok{linear minimizer suggests successful convergence}
\NormalTok{v Hessian matrix is positive definite}
\NormalTok{v No extreme or very small eigenvalues detected}
\NormalTok{v No gradients with respect to fixed effects are }\SpecialCharTok{\textgreater{}=} \FloatTok{0.001}
\NormalTok{v No fixed}\SpecialCharTok{{-}}\NormalTok{effect standard errors are }\ConstantTok{NA}
\NormalTok{v No standard errors look unreasonably large}
\NormalTok{v No sigma parameters are }\SpecialCharTok{\textless{}} \FloatTok{0.01}
\NormalTok{v No sigma parameters are }\SpecialCharTok{\textgreater{}} \DecValTok{100}
\NormalTok{v Range parameter doesn}\StringTok{\textquotesingle{}t look unreasonably large}
\StringTok{\textgreater{} }
\StringTok{\textgreater{} mdist \textless{}{-} sdmTMB(}
\StringTok{+   data = data, }
\StringTok{+   formula = Density \textasciitilde{} 0+ as.factor(SURV) + as.factor(YEAR)+s(DIST),  \# Формула: плотность зависит от года}
\StringTok{+   time = "YEAR",         \# Временная переменная}
\StringTok{+   mesh = mesh\_sdm,       \# Пространственная сетка}
\StringTok{+   family = tweedie(link = "log"),  \# Статистическое распределение}
\StringTok{+   spatial = "on",        \# Включение пространственных эффектов}
\StringTok{+   spatiotemporal = "iid" \# Пространственно{-}временные эффекты}
\StringTok{+ )}
\StringTok{\textgreater{} }
\StringTok{\textgreater{} }
\StringTok{\textgreater{} \# Вывод результатов модели}
\StringTok{\textgreater{} summary(mdist)}
\StringTok{Spatiotemporal model fit by ML [\textquotesingle{}}\NormalTok{sdmTMB}\StringTok{\textquotesingle{}]}
\StringTok{Formula: Density \textasciitilde{} 0 + as.factor(SURV) + as.factor(YEAR) + s(DIST)}
\StringTok{Mesh: mesh\_sdm (isotropic covariance)}
\StringTok{Time column: YEAR}
\StringTok{Data: data}
\StringTok{Family: tweedie(link = \textquotesingle{}}\NormalTok{log}\StringTok{\textquotesingle{})}
\StringTok{ }
\StringTok{Conditional model:}
\StringTok{                    coef.est coef.se}
\StringTok{as.factor(SURV)CRAB     4.74    0.44}
\StringTok{as.factor(SURV)SUM      2.55    0.37}
\StringTok{as.factor(YEAR)2020    {-}0.57    0.36}
\StringTok{as.factor(YEAR)2021    {-}0.20    0.36}
\StringTok{as.factor(YEAR)2022    {-}0.61    0.36}
\StringTok{as.factor(YEAR)2023    {-}0.60    0.36}
\StringTok{as.factor(YEAR)2024    {-}0.85    0.36}
\StringTok{sDIST                  {-}0.06    0.16}

\StringTok{Smooth terms:}
\StringTok{          Std. Dev.}
\StringTok{sds(DIST)         0}

\StringTok{Dispersion parameter: 23.11}
\StringTok{Tweedie p: 1.41}
\StringTok{Matern range: 63.83}
\StringTok{Spatial SD: 1.22}
\StringTok{Spatiotemporal IID SD: 0.97}
\StringTok{ML criterion at convergence: 5914.594}

\StringTok{See ?tidy.sdmTMB to extract these values as a data frame.}

\StringTok{**Possible issues detected! Check output of sanity().**}
\StringTok{\textgreater{} AIC(mdist)  \# Критерий Акаике}
\StringTok{[1] 11857.19}
\StringTok{\textgreater{} sanity(mdist)  \# Проверка корректности модели}
\StringTok{v Non{-}linear minimizer suggests successful convergence}
\StringTok{v Hessian matrix is positive definite}
\StringTok{v No extreme or very small eigenvalues detected}
\StringTok{v No gradients with respect to fixed effects are \textgreater{}= 0.001}
\StringTok{v No fixed{-}effect standard errors are NA}
\StringTok{x \textasciigrave{}ln\_smooth\_sigma\textasciigrave{} standard error may be large}
\StringTok{i Try simplifying the model, adjusting the mesh, or adding priors}

\StringTok{v No sigma parameters are \textless{} 0.01}
\StringTok{v No sigma parameters are \textgreater{} 100}
\StringTok{v Range parameter doesn\textquotesingle{}}\NormalTok{t look unreasonably large}
\SpecialCharTok{\textgreater{}} 
\end{Highlighting}
\end{Shaded}

\subsubsection{\texorpdfstring{\textbf{1. Базовая модель (SURV +
YEAR)}}{1. Базовая модель (SURV + YEAR)}}\label{ux431ux430ux437ux43eux432ux430ux44f-ux43cux43eux434ux435ux43bux44c-surv-year}

\begin{verbatim}
Density ~ 0 + as.factor(SURV) + as.factor(YEAR)
\end{verbatim}

\begin{itemize}
\item
  \textbf{AIC}: 11853.31
\item
  \textbf{Проверка стабильности}: Все параметры стабильны
\item
  \textbf{Ключевые эффекты}:

  \begin{itemize}
  \item
    Высокая плотность в съемках CRAB (коэф. 4.75)
  \item
    Снижение плотности во всех годах относительно базового уровня
    (2020-2024: -0.57 до -0.85)
  \end{itemize}
\item
  \textbf{Пространственные параметры}:

  \begin{itemize}
  \item
    Диапазон Матерна: 63.53 км
  \item
    Пространственная SD: 1.22
  \end{itemize}
\end{itemize}

\subsubsection{\texorpdfstring{\textbf{2. Модель с глубиной
(DEPTH)}}{2. Модель с глубиной (DEPTH)}}\label{ux43cux43eux434ux435ux43bux44c-ux441-ux433ux43bux443ux431ux438ux43dux43eux439-depth}

\begin{verbatim}
Density ~ 0 + as.factor(SURV) + as.factor(YEAR) + s(DEPTH)
\end{verbatim}

\begin{itemize}
\item
  \textbf{AIC}: 11842.73 (наилучший)
\item
  \textbf{Проверка стабильности}: Все параметры стабильны
\item
  \textbf{Ключевые эффекты}:

  \begin{itemize}
  \item
    Сильное отрицательное влияние глубины (коэф. -0.60, SE=0.42)
  \item
    Усиление контраста между съемками CRAB/SUM (CRAB: 5.42 vs SUM: 3.09)
  \end{itemize}
\item
  \textbf{Улучшения}:

  \begin{itemize}
  \item
    Снижение AIC на 10.58 пунктов
  \item
    Уменьшение пространственного диапазона (40.20 км)
  \end{itemize}
\item
  \textbf{Интерпретация}: Глубина --- значимый экологический фактор
  распределения
\end{itemize}

\subsubsection{\texorpdfstring{\textbf{3. Модель с температурой
(TEMP)}}{3. Модель с температурой (TEMP)}}\label{ux43cux43eux434ux435ux43bux44c-ux441-ux442ux435ux43cux43fux435ux440ux430ux442ux443ux440ux43eux439-temp}

\begin{verbatim}
Density ~ 0 + as.factor(SURV) + as.factor(YEAR) + s(TEMP)
\end{verbatim}

\begin{itemize}
\item
  \textbf{AIC}: 11853.59 (хуже базовой)
\item
  \textbf{Проверка стабильности}: Стабильна, но высокий SE сглаживания
\item
  \textbf{Ключевые эффекты}:

  \begin{itemize}
  \item
    Слабый положительный эффект температуры (коэф. 0.80, SE=0.83)
  \item
    Незначительное изменение годовых эффектов
  \end{itemize}
\item
  \textbf{Проблемы}: Минимальное улучшение модели, высокая
  неопределенность эффекта температуры
\end{itemize}

\subsubsection{\texorpdfstring{\textbf{4. Модель с расстоянием
(DIST)}}{4. Модель с расстоянием (DIST)}}\label{ux43cux43eux434ux435ux43bux44c-ux441-ux440ux430ux441ux441ux442ux43eux44fux43dux438ux435ux43c-dist}

\begin{verbatim}
Density ~ 0 + as.factor(SURV) + as.factor(YEAR) + s(DIST)
\end{verbatim}

\begin{itemize}
\item
  \textbf{AIC}: 11857.19 (наихудший)
\item
  \textbf{Проверка стабильности}: Проблемы со сглаживанием
\item
  \textbf{Ключевые эффекты}:

  \begin{itemize}
  \item
    Незначительный эффект расстояния (коэф. -0.06, SE=0.16)
  \item
    Практически идентична базовой модели
  \end{itemize}
\item
  \textbf{Проблемы}: Наихудший AIC, предупреждения о нестабильности
\end{itemize}

\subsection{\texorpdfstring{\textbf{Сводка сравнения
моделей}}{Сводка сравнения моделей}}\label{ux441ux432ux43eux434ux43aux430-ux441ux440ux430ux432ux43dux435ux43dux438ux44f-ux43cux43eux434ux435ux43bux435ux439}

\begin{longtable}[]{@{}
  >{\raggedright\arraybackslash}p{(\linewidth - 10\tabcolsep) * \real{0.1667}}
  >{\raggedright\arraybackslash}p{(\linewidth - 10\tabcolsep) * \real{0.1667}}
  >{\raggedright\arraybackslash}p{(\linewidth - 10\tabcolsep) * \real{0.1667}}
  >{\raggedright\arraybackslash}p{(\linewidth - 10\tabcolsep) * \real{0.1667}}
  >{\raggedright\arraybackslash}p{(\linewidth - 10\tabcolsep) * \real{0.1667}}
  >{\raggedright\arraybackslash}p{(\linewidth - 10\tabcolsep) * \real{0.1667}}@{}}
\toprule\noalign{}
\begin{minipage}[b]{\linewidth}\raggedright
\textbf{Модель}
\end{minipage} & \begin{minipage}[b]{\linewidth}\raggedright
\textbf{AIC}
\end{minipage} & \begin{minipage}[b]{\linewidth}\raggedright
\textbf{ΔAIC}
\end{minipage} & \begin{minipage}[b]{\linewidth}\raggedright
\textbf{Стабильность}
\end{minipage} & \begin{minipage}[b]{\linewidth}\raggedright
\textbf{Ключевой предиктор}
\end{minipage} & \begin{minipage}[b]{\linewidth}\raggedright
\textbf{Эффект ковариаты}
\end{minipage} \\
\midrule\noalign{}
\endhead
\bottomrule\noalign{}
\endlastfoot
\textbf{DEPTH} & 11842.73 & - & ✓✓✓ & Глубина & Сильный (-0.60) \\
Базовая & 11853.31 & +10.6 & ✓✓✓ & - & - \\
TEMP & 11853.59 & +10.9 & ✓✓ & Температура & Слабый (+0.80) \\
DIST & 11857.19 & +14.5 & ✗ & Расстояние & Незначительный (-0.06) \\
\end{longtable}

\subsection{\texorpdfstring{\textbf{Рекомендации}}{Рекомендации}}\label{ux440ux435ux43aux43eux43cux435ux43dux434ux430ux446ux438ux438-1}

\begin{enumerate}
\def\labelenumi{\arabic{enumi}.}
\item
  \textbf{Лучшая модель}: С глубиной (DEPTH)

  \begin{itemize}
  \item
    Значительное улучшение AIC (-10.58)
  \item
    Биологически интерпретируемый эффект (глубина --- ключевой фактор
    распределения краба)
  \item
    Стабильные оценки параметров
  \end{itemize}
\item
  \textbf{Практическое значение}:

  \begin{itemize}
  \item
    Глубина объясняет \textasciitilde12\% пространственной
    вариабельности (судя по изменению пространственной SD)
  \item
    Модель адекватно отражает экологические предпочтения вида
  \end{itemize}
\end{enumerate}

\begin{quote}
\textbf{Вывод}: Включение глубины как ковариаты существенно улучшает
модель, тогда как температура и расстояние не дают значимых улучшений.
\end{quote}

\section{Визуализация
эффектов}\label{ux432ux438ux437ux443ux430ux43bux438ux437ux430ux446ux438ux44f-ux44dux444ux444ux435ux43aux442ux43eux432}

Модель с глубиной - md (см. передыдущий скрипт)

\begin{Shaded}
\begin{Highlighting}[]
\CommentTok{\# {-}{-}{-}{-}{-}{-}{-}{-}{-}{-}{-}{-}{-}{-}{-}{-}{-}{-}{-}{-}{-}{-}{-}{-}{-}{-}{-}{-}{-}{-}{-}{-}{-}{-}{-}{-}{-}{-}{-}{-}{-}{-}{-}{-}{-}{-}{-}{-}{-}{-}{-}}
\CommentTok{\# 8.1. ВИЗУАЛИЗАЦИЯ ЭФФЕКТА ГЛУБИНЫ}
\CommentTok{\# {-}{-}{-}{-}{-}{-}{-}{-}{-}{-}{-}{-}{-}{-}{-}{-}{-}{-}{-}{-}{-}{-}{-}{-}{-}{-}{-}{-}{-}{-}{-}{-}{-}{-}{-}{-}{-}{-}{-}{-}{-}{-}{-}{-}{-}{-}{-}{-}{-}{-}{-}}

\CommentTok{\# Создаем новый датафрейм для предсказаний}
\NormalTok{newdata }\OtherTok{\textless{}{-}} \FunctionTok{expand.grid}\NormalTok{(}
  \AttributeTok{DEPTH =} \FunctionTok{seq}\NormalTok{(}\DecValTok{50}\NormalTok{, }\DecValTok{400}\NormalTok{, }\AttributeTok{by =} \DecValTok{2}\NormalTok{),}
  \AttributeTok{YEAR =} \DecValTok{2020}\NormalTok{,}
  \AttributeTok{SURV =} \StringTok{"CRAB"}\NormalTok{,}
  \AttributeTok{xkm =} \FunctionTok{mean}\NormalTok{(data}\SpecialCharTok{$}\NormalTok{xkm),}
  \AttributeTok{ykm =} \FunctionTok{mean}\NormalTok{(data}\SpecialCharTok{$}\NormalTok{ykm)}
\NormalTok{)}

\CommentTok{\# Делаем предсказания с расчетом стандартных ошибок}
\NormalTok{pred }\OtherTok{\textless{}{-}} \FunctionTok{predict}\NormalTok{(md, }\AttributeTok{newdata =}\NormalTok{ newdata, }\AttributeTok{re\_formula =} \ConstantTok{NA}\NormalTok{, }\AttributeTok{se\_fit =} \ConstantTok{TRUE}\NormalTok{)}


\CommentTok{\# Визуализируем эффект глубины}
\FunctionTok{ggplot}\NormalTok{(pred, }\FunctionTok{aes}\NormalTok{(}\AttributeTok{x =}\NormalTok{ DEPTH, }\AttributeTok{y =} \FunctionTok{exp}\NormalTok{(est))) }\SpecialCharTok{+}
  \FunctionTok{geom\_line}\NormalTok{(}\AttributeTok{linewidth =} \FloatTok{1.2}\NormalTok{, }\AttributeTok{color =} \StringTok{"blue4"}\NormalTok{) }\SpecialCharTok{+}
  \FunctionTok{geom\_ribbon}\NormalTok{(}
    \FunctionTok{aes}\NormalTok{(}
      \AttributeTok{ymin =} \FunctionTok{exp}\NormalTok{(est }\SpecialCharTok{{-}} \FloatTok{1.96} \SpecialCharTok{*}\NormalTok{ est\_se), }
      \AttributeTok{ymax =} \FunctionTok{exp}\NormalTok{(est }\SpecialCharTok{+} \FloatTok{1.96} \SpecialCharTok{*}\NormalTok{ est\_se)  }\CommentTok{\# Исправлено на se.fit}
\NormalTok{    ),}
    \AttributeTok{alpha =} \FloatTok{0.3}\NormalTok{, }
    \AttributeTok{fill =} \StringTok{"steelblue"}
\NormalTok{  ) }\SpecialCharTok{+}
  \FunctionTok{labs}\NormalTok{(}
    \AttributeTok{title =} \StringTok{"Эффект глубины на плотность краба"}\NormalTok{,}
    \AttributeTok{subtitle =} \StringTok{"Год: 2020, Тип съемки: CRAB"}\NormalTok{,}
    \AttributeTok{x =} \StringTok{"Глубина (м)"}\NormalTok{,}
    \AttributeTok{y =} \StringTok{"Предсказанная плотность (особей/км²)"}
\NormalTok{  ) }\SpecialCharTok{+}
  \FunctionTok{theme\_bw}\NormalTok{(}\AttributeTok{base\_size =} \DecValTok{14}\NormalTok{) }\SpecialCharTok{+}
  \FunctionTok{theme}\NormalTok{(}\AttributeTok{panel.grid.minor =} \FunctionTok{element\_blank}\NormalTok{())}
\end{Highlighting}
\end{Shaded}

\begin{figure}[H]

{\centering \includegraphics[width=0.6\linewidth,height=\textheight,keepaspectratio]{images/sdmTMB8.PNG}

}

\caption{Рис. 9.: Визуализация эффекта глубины на плотность краба}

\end{figure}%

\section{Карта с акцентом на нулевые
уловы}\label{ux43aux430ux440ux442ux430-ux441-ux430ux43aux446ux435ux43dux442ux43eux43c-ux43dux430-ux43dux443ux43bux435ux432ux44bux435-ux443ux43bux43eux432ux44b}

Повторяем базовую оценку, но меняем в карте нулевые уловы на крестики

\begin{figure}[H]

{\centering \includegraphics[width=0.9\linewidth,height=\textheight,keepaspectratio]{images/sdmTMBmapZero.jpg}

}

\caption{Рис. 10.: Визуализация результатов (КАРТА)}

\end{figure}%

Скрипт для карты с базовой оценкой

\begin{Shaded}
\begin{Highlighting}[]
\CommentTok{\# {-}{-}{-}{-}{-}{-}{-}{-}{-}{-}{-}{-}{-}{-}{-}{-}{-}{-}{-}{-}{-}{-}{-}{-}{-}{-}{-}}
\CommentTok{\# 1. ПОДГОТОВКА СРЕДЫ И ДАННЫХ}
\CommentTok{\# {-}{-}{-}{-}{-}{-}{-}{-}{-}{-}{-}{-}{-}{-}{-}{-}{-}{-}{-}{-}{-}{-}{-}{-}{-}{-}{-}}

\CommentTok{\# Очистка рабочей среды}
\FunctionTok{rm}\NormalTok{(}\AttributeTok{list =} \FunctionTok{ls}\NormalTok{())}

\CommentTok{\# Установка рабочей директории (замените на свою)}
\FunctionTok{setwd}\NormalTok{(}\StringTok{"C:/COMBINE/"}\NormalTok{)}

\CommentTok{\# Загрузка необходимых пакетов}
\FunctionTok{library}\NormalTok{(readxl)       }\CommentTok{\# Для чтения Excel{-}файлов}
\FunctionTok{library}\NormalTok{(ggplot2)      }\CommentTok{\# Визуализация данных}
\FunctionTok{library}\NormalTok{(dplyr)        }\CommentTok{\# Обработка данных}
\FunctionTok{library}\NormalTok{(PBSmapping)   }\CommentTok{\# Для работы с пространственными данными}
\FunctionTok{library}\NormalTok{(sdmTMB)       }\CommentTok{\# Пространственно{-}временное моделирование}
\FunctionTok{library}\NormalTok{(INLA)         }\CommentTok{\# Продвинутые пространственные модели}
\FunctionTok{library}\NormalTok{(sp)           }\CommentTok{\# Классы для пространственных данных}
\FunctionTok{library}\NormalTok{(sf)           }\CommentTok{\# Пространственные данные (современный формат)}
\FunctionTok{library}\NormalTok{(rnaturalearth) }\CommentTok{\# Загрузка картографических данных}

\CommentTok{\# Загрузка данных из Excel{-}файла}
\NormalTok{data }\OtherTok{\textless{}{-}}\NormalTok{ readxl}\SpecialCharTok{::}\FunctionTok{read\_excel}\NormalTok{(}\StringTok{"KARTOGRAPHIC.xlsx"}\NormalTok{, }\AttributeTok{sheet =} \StringTok{"SURVEY"}\NormalTok{)}

\CommentTok{\# Просмотр структуры данных}
\FunctionTok{str}\NormalTok{(data)}


\CommentTok{\# {-}{-}{-}{-}{-}{-}{-}{-}{-}{-}{-}{-}{-}{-}{-}{-}{-}{-}{-}{-}{-}{-}{-}{-}{-}{-}{-}{-}{-}{-}{-}{-}{-}{-}{-}{-}{-}{-}{-}{-}{-}{-}{-}{-}{-}{-}{-}{-}{-}{-}}
\CommentTok{\# 2. ПРЕОБРАЗОВАНИЕ КООРДИНАТ В ПРОЕКЦИЮ UTM (в км)}
\CommentTok{\# {-}{-}{-}{-}{-}{-}{-}{-}{-}{-}{-}{-}{-}{-}{-}{-}{-}{-}{-}{-}{-}{-}{-}{-}{-}{-}{-}{-}{-}{-}{-}{-}{-}{-}{-}{-}{-}{-}{-}{-}{-}{-}{-}{-}{-}{-}{-}{-}{-}{-}}

\CommentTok{\# Создание пространственного объекта из данных}
\NormalTok{data\_sf }\OtherTok{\textless{}{-}} \FunctionTok{st\_as\_sf}\NormalTok{(}
\NormalTok{  data, }
  \AttributeTok{coords =} \FunctionTok{c}\NormalTok{(}\StringTok{"X"}\NormalTok{, }\StringTok{"Y"}\NormalTok{), }\CommentTok{\# Указание столбцов с координатами}
  \AttributeTok{crs =} \DecValTok{4326}            \CommentTok{\# Система координат WGS84 (широта/долгота)}
\NormalTok{) }

\CommentTok{\# Преобразование в UTM зону 37N (метры)}
\NormalTok{data\_utm }\OtherTok{\textless{}{-}} \FunctionTok{st\_transform}\NormalTok{(data\_sf, }\AttributeTok{crs =} \DecValTok{32637}\NormalTok{) }

\CommentTok{\# Извлечение координат и перевод в километры}
\NormalTok{utm\_coords }\OtherTok{\textless{}{-}} \FunctionTok{st\_coordinates}\NormalTok{(data\_utm)}
\NormalTok{data}\SpecialCharTok{$}\NormalTok{xkm }\OtherTok{\textless{}{-}}\NormalTok{ utm\_coords[, }\DecValTok{1}\NormalTok{] }\SpecialCharTok{/} \DecValTok{1000}  \CommentTok{\# X в км}
\NormalTok{data}\SpecialCharTok{$}\NormalTok{ykm }\OtherTok{\textless{}{-}}\NormalTok{ utm\_coords[, }\DecValTok{2}\NormalTok{] }\SpecialCharTok{/} \DecValTok{1000}  \CommentTok{\# Y в км}

\CommentTok{\# Очистка временных объектов}
\FunctionTok{rm}\NormalTok{(data\_sf, data\_utm, utm\_coords)}

\CommentTok{\# {-}{-}{-}{-}{-}{-}{-}{-}{-}{-}{-}{-}{-}{-}{-}{-}{-}{-}{-}{-}{-}{-}{-}{-}{-}{-}{-}{-}{-}{-}{-}{-}{-}{-}{-}{-}{-}{-}{-}{-}{-}}
\CommentTok{\# 3. ОПРЕДЕЛЕНИЕ ГРАНИЦ ИССЛЕДОВАНИЯ}
\CommentTok{\# {-}{-}{-}{-}{-}{-}{-}{-}{-}{-}{-}{-}{-}{-}{-}{-}{-}{-}{-}{-}{-}{-}{-}{-}{-}{-}{-}{-}{-}{-}{-}{-}{-}{-}{-}{-}{-}{-}{-}{-}{-}}

\CommentTok{\# Вычисление границ исследовательского полигона}
\NormalTok{xl }\OtherTok{\textless{}{-}} \FunctionTok{c}\NormalTok{(}\FunctionTok{min}\NormalTok{(data}\SpecialCharTok{$}\NormalTok{xkm), }\FunctionTok{max}\NormalTok{(data}\SpecialCharTok{$}\NormalTok{xkm))  }\CommentTok{\# Границы по X}
\NormalTok{yl }\OtherTok{\textless{}{-}} \FunctionTok{c}\NormalTok{(}\FunctionTok{min}\NormalTok{(data}\SpecialCharTok{$}\NormalTok{ykm), }\FunctionTok{max}\NormalTok{(data}\SpecialCharTok{$}\NormalTok{ykm))  }\CommentTok{\# Границы по Y}

\CommentTok{\# {-}{-}{-}{-}{-}{-}{-}{-}{-}{-}{-}{-}{-}{-}{-}{-}{-}{-}{-}{-}{-}{-}{-}{-}{-}{-}{-}{-}{-}{-}{-}{-}{-}{-}{-}{-}{-}{-}{-}{-}}
\CommentTok{\# 4. СОЗДАНИЕ РАСТРОВОЙ СЕТКИ ДЛЯ МОДЕЛИ}
\CommentTok{\# {-}{-}{-}{-}{-}{-}{-}{-}{-}{-}{-}{-}{-}{-}{-}{-}{-}{-}{-}{-}{-}{-}{-}{-}{-}{-}{-}{-}{-}{-}{-}{-}{-}{-}{-}{-}{-}{-}{-}{-}}

\CommentTok{\# Создание равномерной сетки с шагом 10 км (для визуализации карты использовался шаг 2 км}
\NormalTok{GRID }\OtherTok{\textless{}{-}} \FunctionTok{makeGrid}\NormalTok{(}
  \AttributeTok{x =} \FunctionTok{seq}\NormalTok{(xl[}\DecValTok{1}\NormalTok{], xl[}\DecValTok{2}\NormalTok{], }\DecValTok{10}\NormalTok{), }
  \AttributeTok{y =} \FunctionTok{seq}\NormalTok{(yl[}\DecValTok{1}\NormalTok{], yl[}\DecValTok{2}\NormalTok{], }\DecValTok{10}\NormalTok{),}
  \AttributeTok{byrow =} \ConstantTok{FALSE}\NormalTok{,}
  \AttributeTok{projection =} \StringTok{"UTM"}\NormalTok{, }
  \AttributeTok{zone =} \DecValTok{37}
\NormalTok{)}

\CommentTok{\# Расчет центроидов ячеек сетки}
\NormalTok{GRID }\OtherTok{\textless{}{-}} \FunctionTok{calcCentroid}\NormalTok{(GRID, }\AttributeTok{rollup =} \DecValTok{3}\NormalTok{)}

\CommentTok{\# {-}{-}{-}{-}{-}{-}{-}{-}{-}{-}{-}{-}{-}{-}{-}{-}{-}{-}{-}{-}{-}{-}{-}{-}{-}{-}{-}{-}{-}{-}{-}{-}{-}{-}{-}{-}{-}{-}{-}{-}{-}{-}{-}{-}{-}{-}{-}{-}{-}{-}{-}{-}{-}{-}{-}{-}{-}{-}{-}}
\CommentTok{\# 5. ПОСТРОЕНИЕ ВЫПУКЛОЙ ОБОЛОЧКИ (CONVEX HULL) ДЛЯ ДАННЫХ}
\CommentTok{\# {-}{-}{-}{-}{-}{-}{-}{-}{-}{-}{-}{-}{-}{-}{-}{-}{-}{-}{-}{-}{-}{-}{-}{-}{-}{-}{-}{-}{-}{-}{-}{-}{-}{-}{-}{-}{-}{-}{-}{-}{-}{-}{-}{-}{-}{-}{-}{-}{-}{-}{-}{-}{-}{-}{-}{-}{-}{-}{-}}

\CommentTok{\# Создание выпуклой оболочки вокруг точек данных}
\NormalTok{Hull }\OtherTok{\textless{}{-}} \FunctionTok{inla.nonconvex.hull}\NormalTok{(}\FunctionTok{cbind}\NormalTok{(data}\SpecialCharTok{$}\NormalTok{xkm, data}\SpecialCharTok{$}\NormalTok{ykm), }\AttributeTok{convex =} \SpecialCharTok{{-}}\FloatTok{0.03}\NormalTok{)}

\CommentTok{\# Визуализация оболочки }
 \FunctionTok{plot}\NormalTok{(Hull)}

\CommentTok{\# Визуализация оболочки и точек съемок 2019{-}2024}
\FunctionTok{points}\NormalTok{(data}\SpecialCharTok{$}\NormalTok{xkm, data}\SpecialCharTok{$}\NormalTok{ykm, }\AttributeTok{pch=}\DecValTok{1}\NormalTok{, }\AttributeTok{cex=}\FloatTok{0.55}\NormalTok{,}\AttributeTok{col=}\StringTok{"black"}\NormalTok{)}


\CommentTok{\# Фильтрация сетки: оставляем только точки внутри оболочки}
\NormalTok{line }\OtherTok{\textless{}{-}}\NormalTok{ Hull}\SpecialCharTok{$}\NormalTok{loc[, }\DecValTok{1}\SpecialCharTok{:}\DecValTok{2}\NormalTok{] }\SpecialCharTok{\%\textgreater{}\%} \FunctionTok{as.data.frame}\NormalTok{()}
\FunctionTok{colnames}\NormalTok{(line) }\OtherTok{\textless{}{-}} \FunctionTok{c}\NormalTok{(}\StringTok{"X"}\NormalTok{, }\StringTok{"Y"}\NormalTok{)}
\NormalTok{GRID}\SpecialCharTok{$}\NormalTok{AREA }\OtherTok{\textless{}{-}} \FunctionTok{point.in.polygon}\NormalTok{(GRID}\SpecialCharTok{$}\NormalTok{X, GRID}\SpecialCharTok{$}\NormalTok{Y, line}\SpecialCharTok{$}\NormalTok{X, line}\SpecialCharTok{$}\NormalTok{Y)}
\NormalTok{GRID }\OtherTok{\textless{}{-}}\NormalTok{ GRID[GRID}\SpecialCharTok{$}\NormalTok{AREA }\SpecialCharTok{\textgreater{}} \FloatTok{0.1}\NormalTok{, }\FunctionTok{c}\NormalTok{(}\StringTok{"X"}\NormalTok{, }\StringTok{"Y"}\NormalTok{)]  }\CommentTok{\# Только внутренние точки}

\CommentTok{\# {-}{-}{-}{-}{-}{-}{-}{-}{-}{-}{-}{-}{-}{-}{-}{-}{-}{-}{-}{-}{-}{-}{-}{-}{-}{-}{-}{-}{-}{-}{-}{-}{-}{-}{-}{-}{-}{-}{-}{-}{-}{-}{-}{-}{-}{-}{-}{-}{-}}
\CommentTok{\# 6. ПОДГОТОВКА СЕТКИ ДЛЯ ПРОГНОЗИРОВАНИЯ}
\CommentTok{\# {-}{-}{-}{-}{-}{-}{-}{-}{-}{-}{-}{-}{-}{-}{-}{-}{-}{-}{-}{-}{-}{-}{-}{-}{-}{-}{-}{-}{-}{-}{-}{-}{-}{-}{-}{-}{-}{-}{-}{-}{-}{-}{-}{-}{-}{-}{-}{-}{-}}

\CommentTok{\# Создание временной сетки (для каждого года)}
\NormalTok{grid }\OtherTok{\textless{}{-}} \FunctionTok{replicate\_df}\NormalTok{(GRID, }\StringTok{"YEAR"}\NormalTok{, }\FunctionTok{unique}\NormalTok{(data}\SpecialCharTok{$}\NormalTok{YEAR))}
\FunctionTok{colnames}\NormalTok{(grid) }\OtherTok{\textless{}{-}} \FunctionTok{c}\NormalTok{(}\StringTok{"xkm"}\NormalTok{, }\StringTok{"ykm"}\NormalTok{, }\StringTok{"YEAR"}\NormalTok{)}
\NormalTok{grid}\SpecialCharTok{$}\NormalTok{SURV }\OtherTok{\textless{}{-}} \StringTok{"CRAB"}  \CommentTok{\# Добавляем информацию о типе съемки}

\CommentTok{\# Визуализация оболочки и сетки для прогнозирования (grid\}}
 \FunctionTok{plot}\NormalTok{(Hull)}
 \FunctionTok{points}\NormalTok{(grid}\SpecialCharTok{$}\NormalTok{xkm, grid}\SpecialCharTok{$}\NormalTok{ykm, }\AttributeTok{pch=}\DecValTok{1}\NormalTok{, }\AttributeTok{cex=}\FloatTok{0.55}\NormalTok{,}\AttributeTok{col=}\StringTok{"black"}\NormalTok{)}

\CommentTok{\# {-}{-}{-}{-}{-}{-}{-}{-}{-}{-}{-}{-}{-}{-}{-}{-}{-}{-}{-}{-}{-}{-}{-}{-}{-}{-}{-}{-}{-}{-}{-}{-}{-}{-}{-}{-}{-}{-}{-}{-}{-}{-}{-}{-}{-}{-}{-}{-}{-}{-}{-}}
\CommentTok{\# 7. ПОСТРОЕНИЕ ПРОСТРАНСТВЕННОЙ СЕТКИ (MESH)}
\CommentTok{\# {-}{-}{-}{-}{-}{-}{-}{-}{-}{-}{-}{-}{-}{-}{-}{-}{-}{-}{-}{-}{-}{-}{-}{-}{-}{-}{-}{-}{-}{-}{-}{-}{-}{-}{-}{-}{-}{-}{-}{-}{-}{-}{-}{-}{-}{-}{-}{-}{-}{-}{-}}

\CommentTok{\# Создание треугольной сетки для пространственного моделирования}
\NormalTok{mesh\_sdm }\OtherTok{\textless{}{-}} \FunctionTok{make\_mesh}\NormalTok{(}
\NormalTok{  data, }
  \FunctionTok{c}\NormalTok{(}\StringTok{"xkm"}\NormalTok{, }\StringTok{"ykm"}\NormalTok{),  }\CommentTok{\# Координаты}
  \AttributeTok{cutoff =} \DecValTok{10}        \CommentTok{\# Минимальное расстояние между узлами (км)}
\NormalTok{)}

\CommentTok{\# Визуализация сетки (раскомментируйте)}
 \FunctionTok{plot}\NormalTok{(mesh\_sdm)}

\CommentTok{\# {-}{-}{-}{-}{-}{-}{-}{-}{-}{-}{-}{-}{-}{-}{-}{-}{-}{-}{-}{-}{-}{-}{-}{-}{-}{-}{-}{-}{-}{-}{-}{-}{-}{-}{-}{-}{-}{-}{-}{-}{-}{-}{-}{-}{-}{-}{-}{-}{-}{-}{-}}
\CommentTok{\# 8. ПОСТРОЕНИЕ ПРОСТРАНСТВЕННО{-}ВРЕМЕННОЙ МОДЕЛИ}
\CommentTok{\# {-}{-}{-}{-}{-}{-}{-}{-}{-}{-}{-}{-}{-}{-}{-}{-}{-}{-}{-}{-}{-}{-}{-}{-}{-}{-}{-}{-}{-}{-}{-}{-}{-}{-}{-}{-}{-}{-}{-}{-}{-}{-}{-}{-}{-}{-}{-}{-}{-}{-}{-}}

\NormalTok{m }\OtherTok{\textless{}{-}} \FunctionTok{sdmTMB}\NormalTok{(}
  \AttributeTok{data =}\NormalTok{ data, }
  \AttributeTok{formula =}\NormalTok{ Density }\SpecialCharTok{\textasciitilde{}} \DecValTok{0} \SpecialCharTok{+} \FunctionTok{as.factor}\NormalTok{(YEAR),  }\CommentTok{\# Формула: плотность зависит от года}
  \AttributeTok{time =} \StringTok{"YEAR"}\NormalTok{,         }\CommentTok{\# Временная переменная}
  \AttributeTok{mesh =}\NormalTok{ mesh\_sdm,       }\CommentTok{\# Пространственная сетка}
  \AttributeTok{family =} \FunctionTok{tweedie}\NormalTok{(}\AttributeTok{link =} \StringTok{"log"}\NormalTok{),  }\CommentTok{\# Статистическое распределение}
  \AttributeTok{spatial =} \StringTok{"on"}\NormalTok{,        }\CommentTok{\# Включение пространственных эффектов}
  \AttributeTok{spatiotemporal =} \StringTok{"iid"} \CommentTok{\# Пространственно{-}временные эффекты}
\NormalTok{)}


\CommentTok{\# Вывод результатов модели}
\FunctionTok{summary}\NormalTok{(m)}
\FunctionTok{AIC}\NormalTok{(m)  }\CommentTok{\# Критерий Акаике}
\FunctionTok{sanity}\NormalTok{(m)  }\CommentTok{\# Проверка корректности модели}

\CommentTok{\# {-}{-}{-}{-}{-}{-}{-}{-}{-}{-}{-}{-}{-}{-}{-}{-}{-}{-}{-}{-}{-}{-}{-}{-}{-}{-}{-}{-}{-}{-}{-}{-}{-}{-}{-}{-}{-}{-}{-}{-}{-}{-}{-}{-}{-}{-}{-}{-}{-}{-}{-}}
\CommentTok{\# 9. ДИАГНОСТИКА МОДЕЛИ}
\CommentTok{\# {-}{-}{-}{-}{-}{-}{-}{-}{-}{-}{-}{-}{-}{-}{-}{-}{-}{-}{-}{-}{-}{-}{-}{-}{-}{-}{-}{-}{-}{-}{-}{-}{-}{-}{-}{-}{-}{-}{-}{-}{-}{-}{-}{-}{-}{-}{-}{-}{-}{-}{-}}

\CommentTok{\# Расчет остатков модели}
\NormalTok{data}\SpecialCharTok{$}\NormalTok{resids }\OtherTok{\textless{}{-}} \FunctionTok{residuals}\NormalTok{(m) }

\CommentTok{\# Гистограмма остатков}
\FunctionTok{hist}\NormalTok{(data}\SpecialCharTok{$}\NormalTok{resids)}

\CommentTok{\# График квантиль{-}квантиль}
\FunctionTok{qqnorm}\NormalTok{(data}\SpecialCharTok{$}\NormalTok{resids)}
\FunctionTok{abline}\NormalTok{(}\AttributeTok{a =} \DecValTok{0}\NormalTok{, }\AttributeTok{b =} \DecValTok{1}\NormalTok{)}

\CommentTok{\# {-}{-}{-}{-}{-}{-}{-}{-}{-}{-}{-}{-}{-}{-}{-}{-}{-}{-}{-}{-}{-}{-}{-}{-}{-}{-}{-}{-}{-}{-}{-}{-}{-}{-}{-}{-}{-}{-}{-}{-}{-}{-}{-}{-}{-}{-}{-}{-}{-}{-}{-}}
\CommentTok{\# 10. ПРОГНОЗИРОВАНИЕ НА СЕТКЕ}
\CommentTok{\# {-}{-}{-}{-}{-}{-}{-}{-}{-}{-}{-}{-}{-}{-}{-}{-}{-}{-}{-}{-}{-}{-}{-}{-}{-}{-}{-}{-}{-}{-}{-}{-}{-}{-}{-}{-}{-}{-}{-}{-}{-}{-}{-}{-}{-}{-}{-}{-}{-}{-}{-}}

\CommentTok{\# Прогноз значений плотности на сетке}
\NormalTok{predictions }\OtherTok{\textless{}{-}} \FunctionTok{predict}\NormalTok{(m, }\AttributeTok{newdata =}\NormalTok{ grid, }\AttributeTok{return\_tmb\_object =} \ConstantTok{TRUE}\NormalTok{)}
\NormalTok{RASP }\OtherTok{\textless{}{-}}\NormalTok{ predictions}\SpecialCharTok{$}\NormalTok{data}

\CommentTok{\# Преобразование координат обратно в широту/долготу}
\NormalTok{RASP}\SpecialCharTok{$}\NormalTok{xkm\_m }\OtherTok{\textless{}{-}}\NormalTok{ RASP}\SpecialCharTok{$}\NormalTok{xkm }\SpecialCharTok{*} \DecValTok{1000}  \CommentTok{\# Обратно в метры}
\NormalTok{RASP}\SpecialCharTok{$}\NormalTok{ykm\_m }\OtherTok{\textless{}{-}}\NormalTok{ RASP}\SpecialCharTok{$}\NormalTok{ykm }\SpecialCharTok{*} \DecValTok{1000}

\CommentTok{\# Создание пространственного объекта в UTM}
\NormalTok{utm\_proj }\OtherTok{\textless{}{-}} \FunctionTok{CRS}\NormalTok{(}\StringTok{"+proj=utm +zone=37 +datum=WGS84 +units=m +no\_defs"}\NormalTok{)}
\NormalTok{coords }\OtherTok{\textless{}{-}} \FunctionTok{cbind}\NormalTok{(RASP}\SpecialCharTok{$}\NormalTok{xkm\_m, RASP}\SpecialCharTok{$}\NormalTok{ykm\_m)}
\NormalTok{sp\_points }\OtherTok{\textless{}{-}} \FunctionTok{SpatialPoints}\NormalTok{(coords, }\AttributeTok{proj4string =}\NormalTok{ utm\_proj)}

\CommentTok{\# Преобразование в WGS84 (широта/долгота)}
\NormalTok{wgs84\_proj }\OtherTok{\textless{}{-}} \FunctionTok{CRS}\NormalTok{(}\StringTok{"+proj=longlat +datum=WGS84"}\NormalTok{)}
\NormalTok{sp\_points\_latlon }\OtherTok{\textless{}{-}} \FunctionTok{spTransform}\NormalTok{(sp\_points, wgs84\_proj)}

\CommentTok{\# Добавление координат в основной датафрейм}
\NormalTok{RASP}\SpecialCharTok{$}\NormalTok{X }\OtherTok{\textless{}{-}} \FunctionTok{coordinates}\NormalTok{(sp\_points\_latlon)[, }\DecValTok{1}\NormalTok{]  }\CommentTok{\# Долгота}
\NormalTok{RASP}\SpecialCharTok{$}\NormalTok{Y }\OtherTok{\textless{}{-}} \FunctionTok{coordinates}\NormalTok{(sp\_points\_latlon)[, }\DecValTok{2}\NormalTok{]  }\CommentTok{\# Широта}

\CommentTok{\# Удаление временных столбцов}
\NormalTok{RASP}\SpecialCharTok{$}\NormalTok{xkm\_m }\OtherTok{\textless{}{-}} \ConstantTok{NULL}
\NormalTok{RASP}\SpecialCharTok{$}\NormalTok{ykm\_m }\OtherTok{\textless{}{-}} \ConstantTok{NULL}

\CommentTok{\# Проверка структуры результата}
\FunctionTok{str}\NormalTok{(RASP)}

\CommentTok{\# {-}{-}{-}{-}{-}{-}{-}{-}{-}{-}{-}{-}{-}{-}{-}{-}{-}{-}{-}{-}{-}{-}{-}{-}{-}{-}{-}{-}{-}{-}{-}{-}{-}{-}{-}{-}{-}{-}{-}{-}{-}{-}{-}{-}{-}}
\CommentTok{\# 11. ВИЗУАЛИЗАЦИЯ РЕЗУЛЬТАТОВ (КАРТА)}
\CommentTok{\# {-}{-}{-}{-}{-}{-}{-}{-}{-}{-}{-}{-}{-}{-}{-}{-}{-}{-}{-}{-}{-}{-}{-}{-}{-}{-}{-}{-}{-}{-}{-}{-}{-}{-}{-}{-}{-}{-}{-}{-}{-}{-}{-}{-}{-}}

\CommentTok{\# Загрузка картографических данных}
\NormalTok{world }\OtherTok{\textless{}{-}} \FunctionTok{ne\_countries}\NormalTok{(}\AttributeTok{scale =} \StringTok{"medium"}\NormalTok{, }\AttributeTok{returnclass =} \StringTok{"sf"}\NormalTok{)}

\CommentTok{\# Определение региона интереса (Арктика России)}
\NormalTok{arctic\_bbox }\OtherTok{\textless{}{-}} \FunctionTok{st\_bbox}\NormalTok{(}\FunctionTok{c}\NormalTok{(}\AttributeTok{xmin =} \DecValTok{25}\NormalTok{, }\AttributeTok{xmax =} \DecValTok{70}\NormalTok{, }\AttributeTok{ymin =} \DecValTok{65}\NormalTok{, }\AttributeTok{ymax =} \DecValTok{80}\NormalTok{), }\AttributeTok{crs =} \DecValTok{4326}\NormalTok{)}
\NormalTok{arctic }\OtherTok{\textless{}{-}} \FunctionTok{st\_crop}\NormalTok{(world, arctic\_bbox)}

\CommentTok{\# Определяем кастомные breaks для шкалы}
\NormalTok{my\_breaks }\OtherTok{\textless{}{-}} \FunctionTok{c}\NormalTok{(}\FloatTok{0.001}\NormalTok{,}\FloatTok{0.1}\NormalTok{,}\DecValTok{1}\NormalTok{,  }\DecValTok{200}\NormalTok{, }\DecValTok{10000}\NormalTok{)}

\CommentTok{\# Создаем категории для PROM}
\NormalTok{data }\OtherTok{\textless{}{-}}\NormalTok{ data }\SpecialCharTok{\%\textgreater{}\%}
  \FunctionTok{mutate}\NormalTok{(}
    \AttributeTok{PROM\_cat =} \FunctionTok{case\_when}\NormalTok{(}
\NormalTok{      PROM }\SpecialCharTok{==} \DecValTok{0} \SpecialCharTok{\textasciitilde{}} \StringTok{"0"}\NormalTok{,}
\NormalTok{      PROM }\SpecialCharTok{\textgreater{}=} \DecValTok{1} \SpecialCharTok{\&}\NormalTok{ PROM }\SpecialCharTok{\textless{}} \DecValTok{10} \SpecialCharTok{\textasciitilde{}} \StringTok{"1{-}9"}\NormalTok{,}
\NormalTok{      PROM }\SpecialCharTok{\textgreater{}=} \DecValTok{10} \SpecialCharTok{\&}\NormalTok{ PROM }\SpecialCharTok{\textless{}} \DecValTok{100} \SpecialCharTok{\textasciitilde{}} \StringTok{"10{-}99"}\NormalTok{,}
\NormalTok{      PROM }\SpecialCharTok{\textgreater{}=} \DecValTok{100} \SpecialCharTok{\textasciitilde{}} \StringTok{"100+"}
\NormalTok{    ),}
    \AttributeTok{PROM\_cat =} \FunctionTok{factor}\NormalTok{(PROM\_cat, }\AttributeTok{levels =} \FunctionTok{c}\NormalTok{(}\StringTok{"0"}\NormalTok{, }\StringTok{"1{-}9"}\NormalTok{, }\StringTok{"10{-}99"}\NormalTok{, }\StringTok{"100+"}\NormalTok{)),}
    \AttributeTok{shape\_cat =} \FunctionTok{ifelse}\NormalTok{(PROM\_cat }\SpecialCharTok{==} \StringTok{"0"}\NormalTok{, }\StringTok{"zero"}\NormalTok{, }\StringTok{"non\_zero"}\NormalTok{)}
\NormalTok{  )}

\CommentTok{\# Обновляем график}
\FunctionTok{ggplot}\NormalTok{() }\SpecialCharTok{+}
  \FunctionTok{geom\_point}\NormalTok{(}
    \AttributeTok{data =}\NormalTok{ RASP, }
    \FunctionTok{aes}\NormalTok{(}\AttributeTok{x =}\NormalTok{ X, }\AttributeTok{y =}\NormalTok{ Y, }\AttributeTok{color =} \FunctionTok{exp}\NormalTok{(est)), }
    \AttributeTok{size =} \FloatTok{0.8}\NormalTok{, }
    \AttributeTok{alpha =} \FloatTok{0.7}
\NormalTok{  ) }\SpecialCharTok{+} 
  \FunctionTok{geom\_point}\NormalTok{(}
    \AttributeTok{data =}\NormalTok{ data, }
    \FunctionTok{aes}\NormalTok{(}\AttributeTok{x =}\NormalTok{ X, }\AttributeTok{y =}\NormalTok{ Y, }\AttributeTok{size =}\NormalTok{ PROM\_cat, }\AttributeTok{shape =}\NormalTok{ shape\_cat),}
    \AttributeTok{color =} \StringTok{"black"}\NormalTok{, }
    \AttributeTok{fill =} \ConstantTok{NA}\NormalTok{, }
    \AttributeTok{alpha =} \FloatTok{0.6}
\NormalTok{  ) }\SpecialCharTok{+}
  \FunctionTok{scale\_size\_manual}\NormalTok{(}
    \AttributeTok{name =} \ConstantTok{NULL}\NormalTok{,}
    \AttributeTok{values =} \FunctionTok{c}\NormalTok{(}\StringTok{"0"} \OtherTok{=} \DecValTok{1}\NormalTok{, }\StringTok{"1{-}9"} \OtherTok{=} \DecValTok{2}\NormalTok{, }\StringTok{"10{-}99"} \OtherTok{=} \DecValTok{3}\NormalTok{),}
    \AttributeTok{labels =} \FunctionTok{c}\NormalTok{(}\StringTok{"0"}\NormalTok{, }\StringTok{"10"}\NormalTok{, }\StringTok{"100"}\NormalTok{)}
\NormalTok{  ) }\SpecialCharTok{+}
  \FunctionTok{scale\_shape\_manual}\NormalTok{(}
    \AttributeTok{values =} \FunctionTok{c}\NormalTok{(}\StringTok{"zero"} \OtherTok{=} \DecValTok{4}\NormalTok{, }\StringTok{"non\_zero"} \OtherTok{=} \DecValTok{21}\NormalTok{),}
    \AttributeTok{guide =} \StringTok{"none"} \CommentTok{\# Скрываем легенду для формы}
\NormalTok{  ) }\SpecialCharTok{+}
  \FunctionTok{guides}\NormalTok{(}
    \AttributeTok{size =} \FunctionTok{guide\_legend}\NormalTok{(}
      \AttributeTok{override.aes =} \FunctionTok{list}\NormalTok{(}\AttributeTok{shape =} \FunctionTok{c}\NormalTok{(}\DecValTok{4}\NormalTok{, }\DecValTok{21}\NormalTok{, }\DecValTok{21}\NormalTok{)) }\CommentTok{\# Крестик только для первого элемента}
\NormalTok{    )}
\NormalTok{  ) }\SpecialCharTok{+}
  \FunctionTok{geom\_sf}\NormalTok{(}\AttributeTok{data =}\NormalTok{ arctic, }\AttributeTok{fill =} \StringTok{"lightgrey"}\NormalTok{, }\AttributeTok{color =} \StringTok{"darkgrey"}\NormalTok{) }\SpecialCharTok{+}
  \FunctionTok{scale\_color\_viridis\_c}\NormalTok{(}
    \AttributeTok{name =} \ConstantTok{NULL}\NormalTok{,}
    \AttributeTok{option =} \StringTok{"H"}\NormalTok{, }
    \AttributeTok{trans =} \StringTok{"log"}\NormalTok{, }
    \AttributeTok{breaks =}\NormalTok{ my\_breaks, }
    \AttributeTok{labels =}\NormalTok{ my\_breaks, }
    \AttributeTok{limits =} \FunctionTok{range}\NormalTok{(my\_breaks),}
    \AttributeTok{guide =} \FunctionTok{guide\_colorbar}\NormalTok{(}
      \AttributeTok{barwidth =} \FunctionTok{unit}\NormalTok{(}\DecValTok{5}\NormalTok{, }\StringTok{"cm"}\NormalTok{),}
      \AttributeTok{title.position =} \StringTok{"top"}\NormalTok{,}
      \AttributeTok{direction =} \StringTok{"horizontal"}
\NormalTok{    )}
\NormalTok{  ) }\SpecialCharTok{+}
  \FunctionTok{facet\_wrap}\NormalTok{(}\SpecialCharTok{\textasciitilde{}}\NormalTok{ YEAR, }\AttributeTok{ncol =} \DecValTok{2}\NormalTok{) }\SpecialCharTok{+}
  \FunctionTok{coord\_sf}\NormalTok{(}
    \AttributeTok{xlim =} \FunctionTok{c}\NormalTok{(}\FunctionTok{min}\NormalTok{(RASP}\SpecialCharTok{$}\NormalTok{X)}\SpecialCharTok{{-}}\DecValTok{1}\NormalTok{, }\FunctionTok{max}\NormalTok{(RASP}\SpecialCharTok{$}\NormalTok{X)}\SpecialCharTok{+}\DecValTok{1}\NormalTok{),}
    \AttributeTok{ylim =} \FunctionTok{c}\NormalTok{(}\FunctionTok{min}\NormalTok{(RASP}\SpecialCharTok{$}\NormalTok{Y)}\SpecialCharTok{{-}}\FloatTok{0.5}\NormalTok{, }\FunctionTok{max}\NormalTok{(RASP}\SpecialCharTok{$}\NormalTok{Y)}\SpecialCharTok{+}\FloatTok{0.5}\NormalTok{),}
    \AttributeTok{crs =} \DecValTok{4326}
\NormalTok{  ) }\SpecialCharTok{+}
  \FunctionTok{theme\_bw}\NormalTok{(}\AttributeTok{base\_size =} \DecValTok{12}\NormalTok{) }\SpecialCharTok{+}
  \FunctionTok{labs}\NormalTok{(}\AttributeTok{x =} \ConstantTok{NULL}\NormalTok{, }\AttributeTok{y =} \ConstantTok{NULL}\NormalTok{) }\SpecialCharTok{+}
  \FunctionTok{theme}\NormalTok{(}
    \AttributeTok{panel.grid =} \FunctionTok{element\_line}\NormalTok{(}\AttributeTok{color =} \StringTok{"grey90"}\NormalTok{),}
    \AttributeTok{legend.position =} \StringTok{"bottom"}\NormalTok{,}
    \AttributeTok{legend.key.width =} \FunctionTok{unit}\NormalTok{(}\FloatTok{1.2}\NormalTok{, }\StringTok{"cm"}\NormalTok{),}
    \AttributeTok{strip.background =} \FunctionTok{element\_rect}\NormalTok{(}\AttributeTok{fill =} \StringTok{"white"}\NormalTok{)}
\NormalTok{  )}


\CommentTok{\# Сохранение графика (раскомментируйте)}
\CommentTok{\# ggsave("sdmTMBmapZero.jpg", width = 8, height = 8, dpi = 300)}
\end{Highlighting}
\end{Shaded}

\section{Определение площади
съемки}\label{ux43eux43fux440ux435ux434ux435ux43bux435ux43dux438ux435-ux43fux43bux43eux449ux430ux434ux438-ux441ux44aux435ux43cux43aux438}

\begin{Shaded}
\begin{Highlighting}[]
\CommentTok{\# {-}{-}{-}{-}{-}{-}{-}{-}{-}{-}{-}{-}{-}{-}{-}{-}{-}{-}{-}{-}{-}{-}{-}{-}{-}{-}{-}}
\CommentTok{\# 1. ПОДГОТОВКА СРЕДЫ И ДАННЫХ}
\CommentTok{\# {-}{-}{-}{-}{-}{-}{-}{-}{-}{-}{-}{-}{-}{-}{-}{-}{-}{-}{-}{-}{-}{-}{-}{-}{-}{-}{-}}

\CommentTok{\# Очистка рабочей среды}
\FunctionTok{rm}\NormalTok{(}\AttributeTok{list =} \FunctionTok{ls}\NormalTok{())}

\CommentTok{\# Установка рабочей директории (замените на свою)}
\FunctionTok{setwd}\NormalTok{(}\StringTok{"C:/COMBINE/"}\NormalTok{)}

\CommentTok{\# Загрузка необходимых пакетов}
\FunctionTok{library}\NormalTok{(readxl)       }\CommentTok{\# Для чтения Excel{-}файлов}
\FunctionTok{library}\NormalTok{(PBSmapping)   }\CommentTok{\# Для работы с пространственными данными}
\FunctionTok{library}\NormalTok{(sdmTMB)       }\CommentTok{\# Пространственно{-}временное моделирование}
\FunctionTok{library}\NormalTok{(INLA)         }\CommentTok{\# Продвинутые пространственные модели}

\CommentTok{\# Загрузка данных из Excel{-}файла}
\NormalTok{data }\OtherTok{\textless{}{-}}\NormalTok{ readxl}\SpecialCharTok{::}\FunctionTok{read\_excel}\NormalTok{(}\StringTok{"KARTOGRAPHIC.xlsx"}\NormalTok{, }\AttributeTok{sheet =} \StringTok{"SURVEY"}\NormalTok{)}

\CommentTok{\# Просмотр структуры данных}
\FunctionTok{str}\NormalTok{(data)}


\CommentTok{\# {-}{-}{-}{-}{-}{-}{-}{-}{-}{-}{-}{-}{-}{-}{-}{-}{-}{-}{-}{-}{-}{-}{-}{-}{-}{-}{-}{-}{-}{-}{-}{-}{-}{-}{-}{-}{-}{-}{-}{-}{-}{-}{-}{-}{-}{-}{-}{-}{-}{-}}
\CommentTok{\# 2. ПРЕОБРАЗОВАНИЕ КООРДИНАТ В ПРОЕКЦИЮ UTM (в км)}
\CommentTok{\# {-}{-}{-}{-}{-}{-}{-}{-}{-}{-}{-}{-}{-}{-}{-}{-}{-}{-}{-}{-}{-}{-}{-}{-}{-}{-}{-}{-}{-}{-}{-}{-}{-}{-}{-}{-}{-}{-}{-}{-}{-}{-}{-}{-}{-}{-}{-}{-}{-}{-}}

\CommentTok{\# Создание пространственного объекта из данных}
\NormalTok{data\_sf }\OtherTok{\textless{}{-}} \FunctionTok{st\_as\_sf}\NormalTok{(}
\NormalTok{  data, }
  \AttributeTok{coords =} \FunctionTok{c}\NormalTok{(}\StringTok{"X"}\NormalTok{, }\StringTok{"Y"}\NormalTok{), }\CommentTok{\# Указание столбцов с координатами}
  \AttributeTok{crs =} \DecValTok{4326}            \CommentTok{\# Система координат WGS84 (широта/долгота)}
\NormalTok{) }

\CommentTok{\# Преобразование в UTM зону 37N (метры)}
\NormalTok{data\_utm }\OtherTok{\textless{}{-}} \FunctionTok{st\_transform}\NormalTok{(data\_sf, }\AttributeTok{crs =} \DecValTok{32637}\NormalTok{) }

\CommentTok{\# Извлечение координат и перевод в километры}
\NormalTok{utm\_coords }\OtherTok{\textless{}{-}} \FunctionTok{st\_coordinates}\NormalTok{(data\_utm)}
\NormalTok{data}\SpecialCharTok{$}\NormalTok{xkm }\OtherTok{\textless{}{-}}\NormalTok{ utm\_coords[, }\DecValTok{1}\NormalTok{] }\SpecialCharTok{/} \DecValTok{1000}  \CommentTok{\# X в км}
\NormalTok{data}\SpecialCharTok{$}\NormalTok{ykm }\OtherTok{\textless{}{-}}\NormalTok{ utm\_coords[, }\DecValTok{2}\NormalTok{] }\SpecialCharTok{/} \DecValTok{1000}  \CommentTok{\# Y в км}

\CommentTok{\# Очистка временных объектов}
\FunctionTok{rm}\NormalTok{(data\_sf, data\_utm, utm\_coords)}

\CommentTok{\# {-}{-}{-}{-}{-}{-}{-}{-}{-}{-}{-}{-}{-}{-}{-}{-}{-}{-}{-}{-}{-}{-}{-}{-}{-}{-}{-}{-}{-}{-}{-}{-}{-}{-}{-}{-}{-}{-}{-}{-}{-}}
\CommentTok{\# 3. ОПРЕДЕЛЕНИЕ ГРАНИЦ ИССЛЕДОВАНИЯ}
\CommentTok{\# {-}{-}{-}{-}{-}{-}{-}{-}{-}{-}{-}{-}{-}{-}{-}{-}{-}{-}{-}{-}{-}{-}{-}{-}{-}{-}{-}{-}{-}{-}{-}{-}{-}{-}{-}{-}{-}{-}{-}{-}{-}}

\CommentTok{\# Вычисление границ исследовательского полигона}
\NormalTok{xl }\OtherTok{\textless{}{-}} \FunctionTok{c}\NormalTok{(}\FunctionTok{min}\NormalTok{(data}\SpecialCharTok{$}\NormalTok{xkm), }\FunctionTok{max}\NormalTok{(data}\SpecialCharTok{$}\NormalTok{xkm))  }\CommentTok{\# Границы по X}
\NormalTok{yl }\OtherTok{\textless{}{-}} \FunctionTok{c}\NormalTok{(}\FunctionTok{min}\NormalTok{(data}\SpecialCharTok{$}\NormalTok{ykm), }\FunctionTok{max}\NormalTok{(data}\SpecialCharTok{$}\NormalTok{ykm))  }\CommentTok{\# Границы по Y}

\CommentTok{\# {-}{-}{-}{-}{-}{-}{-}{-}{-}{-}{-}{-}{-}{-}{-}{-}{-}{-}{-}{-}{-}{-}{-}{-}{-}{-}{-}{-}{-}{-}{-}{-}{-}{-}{-}{-}{-}{-}{-}{-}}
\CommentTok{\# 4. СОЗДАНИЕ РАСТРОВОЙ СЕТКИ ДЛЯ МОДЕЛИ}
\CommentTok{\# {-}{-}{-}{-}{-}{-}{-}{-}{-}{-}{-}{-}{-}{-}{-}{-}{-}{-}{-}{-}{-}{-}{-}{-}{-}{-}{-}{-}{-}{-}{-}{-}{-}{-}{-}{-}{-}{-}{-}{-}}

\CommentTok{\# Создание равномерной сетки с шагом 10 км (для визуализации карты использовался шаг 2 км}
\NormalTok{GRID }\OtherTok{\textless{}{-}} \FunctionTok{makeGrid}\NormalTok{(}
  \AttributeTok{x =} \FunctionTok{seq}\NormalTok{(xl[}\DecValTok{1}\NormalTok{], xl[}\DecValTok{2}\NormalTok{], }\DecValTok{10}\NormalTok{), }
  \AttributeTok{y =} \FunctionTok{seq}\NormalTok{(yl[}\DecValTok{1}\NormalTok{], yl[}\DecValTok{2}\NormalTok{], }\DecValTok{10}\NormalTok{),}
  \AttributeTok{byrow =} \ConstantTok{FALSE}\NormalTok{,}
  \AttributeTok{projection =} \StringTok{"UTM"}\NormalTok{, }
  \AttributeTok{zone =} \DecValTok{37}
\NormalTok{)}

\CommentTok{\# Расчет центроидов ячеек сетки}
\NormalTok{GRID }\OtherTok{\textless{}{-}} \FunctionTok{calcCentroid}\NormalTok{(GRID, }\AttributeTok{rollup =} \DecValTok{3}\NormalTok{)}

\CommentTok{\# {-}{-}{-}{-}{-}{-}{-}{-}{-}{-}{-}{-}{-}{-}{-}{-}{-}{-}{-}{-}{-}{-}{-}{-}{-}{-}{-}{-}{-}{-}{-}{-}{-}{-}{-}{-}{-}{-}{-}{-}{-}{-}{-}{-}{-}{-}{-}{-}{-}{-}{-}{-}{-}{-}{-}{-}{-}{-}{-}}
\CommentTok{\# 5. ПОСТРОЕНИЕ ВЫПУКЛОЙ ОБОЛОЧКИ (CONVEX HULL) ДЛЯ ДАННЫХ}
\CommentTok{\# {-}{-}{-}{-}{-}{-}{-}{-}{-}{-}{-}{-}{-}{-}{-}{-}{-}{-}{-}{-}{-}{-}{-}{-}{-}{-}{-}{-}{-}{-}{-}{-}{-}{-}{-}{-}{-}{-}{-}{-}{-}{-}{-}{-}{-}{-}{-}{-}{-}{-}{-}{-}{-}{-}{-}{-}{-}{-}{-}}

\CommentTok{\# Создание выпуклой оболочки вокруг точек данных}
\NormalTok{Hull }\OtherTok{\textless{}{-}} \FunctionTok{inla.nonconvex.hull}\NormalTok{(}\FunctionTok{cbind}\NormalTok{(data}\SpecialCharTok{$}\NormalTok{xkm, data}\SpecialCharTok{$}\NormalTok{ykm), }\AttributeTok{convex =} \SpecialCharTok{{-}}\FloatTok{0.03}\NormalTok{)}

\CommentTok{\# Визуализация оболочки }
 \FunctionTok{plot}\NormalTok{(Hull)}

\CommentTok{\# Преобразование Hull в объект PolySet:}
\NormalTok{polys }\OtherTok{\textless{}{-}} \FunctionTok{data.frame}\NormalTok{(}
  \AttributeTok{PID =} \FunctionTok{rep}\NormalTok{(}\DecValTok{1}\NormalTok{, }\FunctionTok{nrow}\NormalTok{(Hull}\SpecialCharTok{$}\NormalTok{loc)), }\CommentTok{\# ID полигона}
  \AttributeTok{POS =} \DecValTok{1}\SpecialCharTok{:}\FunctionTok{nrow}\NormalTok{(Hull}\SpecialCharTok{$}\NormalTok{loc),       }\CommentTok{\# Порядок точек}
  \AttributeTok{X =}\NormalTok{ Hull}\SpecialCharTok{$}\NormalTok{loc[, }\DecValTok{1}\NormalTok{],            }\CommentTok{\# Координата X (в км)}
  \AttributeTok{Y =}\NormalTok{ Hull}\SpecialCharTok{$}\NormalTok{loc[, }\DecValTok{2}\NormalTok{]             }\CommentTok{\# Координата Y (в км)}
\NormalTok{)}
\NormalTok{polys }\OtherTok{\textless{}{-}}\NormalTok{ PBSmapping}\SpecialCharTok{::}\FunctionTok{as.PolySet}\NormalTok{(polys, }\AttributeTok{projection =} \StringTok{"UTM"}\NormalTok{, }\AttributeTok{zone =} \DecValTok{37}\NormalTok{)}

\CommentTok{\# Расчет площади:}
\NormalTok{area }\OtherTok{\textless{}{-}}\NormalTok{ PBSmapping}\SpecialCharTok{::}\FunctionTok{calcArea}\NormalTok{(polys)}
\FunctionTok{print}\NormalTok{(}\FunctionTok{paste}\NormalTok{(}\StringTok{"Площадь Hull:"}\NormalTok{, }\FunctionTok{round}\NormalTok{(area}\SpecialCharTok{$}\NormalTok{area, }\DecValTok{2}\NormalTok{), }\StringTok{"км кв."}\NormalTok{))}
\end{Highlighting}
\end{Shaded}

{[}1{]} ``Площадь Hull: 283947.5 км кв.''

\bookmarksetup{startatroot}

\chapter{Продукционная модель
SPiCT}\label{ux43fux440ux43eux434ux443ux43aux446ux438ux43eux43dux43dux430ux44f-ux43cux43eux434ux435ux43bux44c-spict}

\section{Введение}\label{ux432ux432ux435ux434ux435ux43dux438ux435-5}

Вводное занятие по SPiCT'у, которое основаго на рекомендованом к
изучению обзоре, составленным разработчиками библиотеке. Однако в этом
репозитории (см.боковую вкладку оглавления) есть еще триптих скриптов,
посвященных SPiCT'у, включающих отдельную оценку ОДУ, ПРП и MSE (оценки
стратегии управления).И так, далее лирическое введение\ldots{}

Это практическое занятие --- приглашение работать с неопределённостью
честно и профессионально. Мы будем оценивать запас промыслового вида с
помощью \href{https://github.com/DTUAqua/spict}{SPiCT} ---
стохастической продукционной модели в байесовской постановке. В терминах
Даниэля Канемана: мы осознанно переводим себя из «быстрой» Системы 1
(интуиции и красивых историй) в «медленную» Систему 2 (проверяемые
допущения, приоры, диагностика, сценарии).

Зачем \href{https://github.com/DTUAqua/spict}{SPiCT} и почему именно
такой подход. \href{https://github.com/DTUAqua/spict}{SPiCT} реализует
стохастическую продукционную динамику (обобщение
Шефера/Пеллы--Томлинсона) и оценивает параметры через TMB* в байесовской
парадигме. Это важно по трём причинам. Во‑первых, реальный улов и
индексы шумные, с нулями, с передисперсией; стохастическая модель
учитывает процессные и наблюдательные ошибки, а не прячет их в остатки.
Во‑вторых, байесовские прайеры (или приоры) --- не «псевдонаучные
пожелания», а протокол того, что мы готовы считать правдоподобным до
данных: n≈2 (Шефер), разумный диапазон K, начальная доля B/K. Прайеры
делают оценку стабильной при слабых данных и позволяют явно «рассказать»
модели, что мы знаем из биологии. В‑третьих, пакет построен вокруг
воспроизводимой диагностики: OSA‑остатки, автокорреляция, сравнение
априор/апостериор, ретроспектива (коэффициент Мона), сценарии управления
--- это инструменты, уменьшающие риск когнитивных ловушек.

Про честные допущения и минимализм. «Простые» объяснения
предпочтительнее при прочих равных, пока они работают. Мы не перегружаем
модель лишними степенями свободы: фиксируем n=2, задаём информативные
прайеры на K и начальную долю B/K, аккуратно масштабируем
неопределённость последних лет (не наделяя «свежие» данные ролью
окончательных вердиктов), используем малый шаг интегрирования dteuler
для корректной динамики. Ричард Докинз сказал бы: эволюция --- это про
ограничения и компромиссы; продукционная модель --- тоже. Она не
объясняет всё, но хорошо решает задачу «сколько можем брать и оставаться
в зелёной зоне» при ограниченной информации.

Как держать под контролем наши ошибки мышления. Ведь мозг любит «истории
с концом», даже если данных мало. Мы противопоставляем этому протокол.
Сначала --- валидируем входные ряды (catch и индексы, лаги и
кросс‑корреляции), затем --- формируем единый вход SPiCT с явным
календарём измерений (timeI и obsI), задаём прайеры и повышаем
неопределённость там, где это честно (последние годы). Дальше ---
диагностика: OSA‑остатки без смещения и избыточной автокорреляции,
нормальность на QQ‑плотах, сравнение априора и апостериора (данные
«говорят», или всё держится на прайере?), корреляции параметров
(типичная антагония \emph{K} и \emph{q} --- не баг, а свойство задачи),
ретроспектива Мона (устойчивость оценок к добавлению новых лет). Это
«канемановская» дисциплина: мы строим защитные барьеры от своей
уверенности.

Про риск и «толстые хвосты». Напомним, что средние --- коварны. Даже
если B/B\textsubscript{MSY}\textgreater1 и
F/F\textsubscript{MSY}\textless1, управленческие правила должны
учитывать ширину доверительных интервалов, а не только точку. Поэтому мы
интерпретируем не одну кривую, а пучок сценариев: «держать текущий
вылов», «держать текущий F», «ловить на F\textsubscript{MSY}»,
«снизить/увеличить F», «хоккейная клюшка», «ICES‑правило», фиксированные
квоты. И для каждого --- не только прогноз B и F, но и риск превышения
F\textsubscript{MSY} и ухода ниже B\textsubscript{MSY}. Хорошая
рекомендация --- это баланс «надёжности» и «пользы»: консервативный
вылов, сохраняющий B/B\textsubscript{MSY}\textgreater1.2 при низкой
вероятности перелова, часто выигрывает у агрессивных схем, которые «в
среднем» немного выгоднее, но делают систему хрупкой.

Как читать результаты и не обмануться. Сходимость (convergence=0) и
финитные стандартные ошибки --- допуск к интерпретации. OSA‑диагностика
без смещения и лишней автокорреляции --- индикатор адекватности
структуры ошибок. Разумный K (с учётом прайера), r в биологическом
диапазоне, q1--q2 сопоставимые для CPUE/BESS --- признак
«реалистичности». Если апостериор на K почти совпадает с прайером --- не
беда, это честный сигнал: данных мало, рекомендацию стоит «страховать»
широкой лентой и консервативным сценарием. Если ковариации параметров
велики --- не прячем, а подчёркиваем в выводах. Если ретроспектива
показывает \textbar ρ\textbar{} близко к нулю --- модель устойчива; если
нет --- упрощаем, усиливаем приоры, перепроверяем данные.

Про «матчасть» и воспроизводимость. SPiCT --- это не только fit.spict(),
а экосистема справок (?check.inp, ?fit.spict), кратких обзоров и живого
«технического» руководства. Мы сохраняем код, версии пакетов, начальные
значения и приоры внутри скрипта --- так, чтобы завтра любой
исследователь смог повторить наши оценки. Это и есть ``антихрупкость'' :
система, которая выигрывает от проверок и критики. И это способ строить
доверие в сообществе --- не за счёт красноречия, а за счёт прозрачности.

Что вы освоите по итогам занятия. 1) Подготовку входов: единая шкала
времени, лаги индексов, кросс‑корреляции. 2) Настройку прайеров и их
роль при ограниченных данных. 3) Запуск и диагностику модели: от
конвергенции до OSA и ретроспективы. 4) Чтение ключевых ориентиров (MSY,
B\textsubscript{MSY}, F\textsubscript{MSY}, B/B\textsubscript{MSY},
F/F\textsubscript{MSY}) и их неопределённости. 5) Формирование и
интерпретацию сценариев управления с учётом риска, а не только «лучшей
точки». 6) Коммуникацию результатов для управленцев: одна картинка «Kobe
plot», одна таблица сценариев, одна короткая формулировка рекомендации с
оговорками и предпосылками.

И главное --- стиль мышления. Мы будем вспоминать Сапольского, когда
захочется превратить «красивую» картинку в факт; Канемана --- когда рука
потянется «довернуть» модель до нужного ответа; Талеба --- когда нужно
выбирать между «чуть больше сейчас» и «устойчиво много лет»; Хокинга ---
когда стоит убрать лишнюю сложность; Докинза --- когда интерпретируем
параметры через процессы; Харари --- когда формируем честный и открытый
нарратив о том, что модель знает и чего не знает. Такой образ действий
превращает SPiCT из «чёрного ящика» в дисциплину: воспроизводимый,
устойчивый и полезный для принятия решений анализ.

И так, библиотека SPiCT \url{https://github.com/DTUAqua/spict} - оценка
запаса с помощью стохастической версии продукционной модели и
байесовского подхода. Доступен краткий обзор пакета
\href{https://github.com/DTUAqua/spict/raw/master/spict/inst/doc/spict_handbook.pdf}{\texttt{здесь}},
который служит для ознакомления с пакетом и его функционалом. В обзоре
также содержится описание более продвинутых функций пакета.

Документ с техническими рекомендациями по использованию SPiCT доступен
\href{https://github.com/DTUAqua/spict/raw/master/spict/inst/doc/spict_guidelines.pdf}{здесь}
. Это постоянно обновляемый документ.

Пакет также содержит достаточно подробную документацию в виде справочных
текстов, связанных с каждой функцией (некоторые из них могут быть не
полностью актуальны). Доступ к ним можно получить обычным для R
способом, набрав, например \texttt{?check.inp}, . Для начала (помимо
изучения краткого обзора) рекомендуется прочитать \texttt{?check.inp}и
\texttt{?fit.spict}.

* - \textbf{TMB (Template Model Builder)}--- это специализированный
пакет для языка R, предназначенный для эффективной оценки параметров
сложных нелинейных статистических моделей, часто используемых в экологии
и fisheries science. Он позволяет реализовать байесовский вывод,
автоматически вычисляя производные и правдоподобие для сложных моделей,
что значительно ускоряет расчёты и делает возможной работу с
многомерными задачами, такими как оценка запасов с помощью SPiCT.

\section{Установка пакетов и Загрузка
данных}\label{ux443ux441ux442ux430ux43dux43eux432ux43aux430-ux43fux430ux43aux435ux442ux43eux432-ux438-ux437ux430ux433ux440ux443ux437ux43aux430-ux434ux430ux43dux43dux44bux445}

Полный скрипт находится по
\href{https://mombus.github.io/cRab/data/SPICT.R}{ссылке}

\begin{Shaded}
\begin{Highlighting}[]
\CommentTok{\# {-}{-}{-}{-}{-}{-}{-}{-}{-}{-}{-}{-}{-}{-}{-}{-}{-}{-}{-}{-}{-}{-}{-}{-}{-} 1. ПОДГОТОВКА СРЕДЫ {-}{-}{-}{-}{-}{-}{-}{-}{-}{-}{-}{-}{-}{-}{-}{-}{-}{-}{-}{-}{-}{-}{-}{-}{-}{-}{-}}

\DocumentationTok{\#\# 1.1 Установка пакетов (выполнить один раз)}
\CommentTok{\# install.packages("tidyverse")}
\CommentTok{\#remotes::install\_github("DTUAqua/spict/spict") \# Установка SPiCT}
\CommentTok{\#install.packages("TMB", type="source")}

\DocumentationTok{\#\# 1.2 Загрузка библиотек}
\FunctionTok{library}\NormalTok{(spict)   }\CommentTok{\# Основной пакет для моделирования}
\FunctionTok{library}\NormalTok{(tidyverse) }\CommentTok{\# Для обработки данных и визуализации}


\DocumentationTok{\#\# 1.3 Установка рабочей директории}
\FunctionTok{setwd}\NormalTok{(}\StringTok{"C:/SPICT"}\NormalTok{) }\CommentTok{\# Укажите вашу рабочую папку}


\CommentTok{\# {-}{-}{-}{-}{-}{-}{-}{-}{-}{-}{-}{-}{-}{-}{-}{-}{-}{-}{-}{-}{-}{-}{-}{-}{-} 2. ЗАГРУЗКА ДАННЫХ {-}{-}{-}{-}{-}{-}{-}{-}{-}{-}{-}{-}{-}{-}{-}{-}{-}{-}{-}{-}{-}{-}{-}{-}{-}{-}{-}}

\DocumentationTok{\#\# 2.1 Вектор лет наблюдений}
\NormalTok{Year }\OtherTok{\textless{}{-}} \DecValTok{2005}\SpecialCharTok{:}\DecValTok{2024}

\DocumentationTok{\#\# 2.2 Данные по вылову (тыс. тонн)}
\NormalTok{Catch }\OtherTok{\textless{}{-}} \FunctionTok{c}\NormalTok{(}\DecValTok{5}\NormalTok{,  }\DecValTok{7}\NormalTok{,  }\DecValTok{6}\NormalTok{, }\DecValTok{10}\NormalTok{, }\DecValTok{14}\NormalTok{, }\DecValTok{25}\NormalTok{, }\DecValTok{28}\NormalTok{, }\DecValTok{30}\NormalTok{, }\DecValTok{32}\NormalTok{, }\DecValTok{35}\NormalTok{, }\DecValTok{25}\NormalTok{, }\DecValTok{20}\NormalTok{, }\DecValTok{15}\NormalTok{, }\DecValTok{12}\NormalTok{, }\DecValTok{10}\NormalTok{, }\DecValTok{12}\NormalTok{, }\DecValTok{10}\NormalTok{, }\DecValTok{13}\NormalTok{, }\DecValTok{11}\NormalTok{, }\DecValTok{12}\NormalTok{)}

\DocumentationTok{\#\# 2.3 Индекс CPUE (промысловый индекс)}
\NormalTok{CPUEIndex }\OtherTok{\textless{}{-}} \FunctionTok{c}\NormalTok{(}\FloatTok{27.427120}\NormalTok{, }\FloatTok{26.775958}\NormalTok{, }\FloatTok{16.811997}\NormalTok{, }\FloatTok{22.979653}\NormalTok{, }\FloatTok{29.048568}\NormalTok{, }\FloatTok{29.996072}\NormalTok{, }\FloatTok{16.476301}\NormalTok{,}
\FloatTok{17.174455}\NormalTok{, }\FloatTok{10.537272}\NormalTok{, }\FloatTok{14.590435}\NormalTok{,  }\FloatTok{8.286352}\NormalTok{, }\FloatTok{11.394168}\NormalTok{, }\FloatTok{15.537878}\NormalTok{, }\FloatTok{13.791166}\NormalTok{,}
\FloatTok{11.527548}\NormalTok{, }\FloatTok{15.336093}\NormalTok{, }\FloatTok{12.154069}\NormalTok{, }\FloatTok{15.568450}\NormalTok{, }\FloatTok{16.221933}\NormalTok{, }\FloatTok{13.421132}\NormalTok{)}

\DocumentationTok{\#\# 2.4 Индекс BESS (научная съемка)}
\NormalTok{BESSIndex }\OtherTok{\textless{}{-}} \FunctionTok{c}\NormalTok{( }\ConstantTok{NA}\NormalTok{, }\FloatTok{16.270375}\NormalTok{, }\FloatTok{20.691355}\NormalTok{, }\FloatTok{15.141784}\NormalTok{, }\FloatTok{18.594620}\NormalTok{, }\FloatTok{15.975548}\NormalTok{, }\FloatTok{13.792012}\NormalTok{,}
\FloatTok{13.328805}\NormalTok{, }\FloatTok{11.659744}\NormalTok{, }\FloatTok{11.753855}\NormalTok{,  }\FloatTok{9.309859}\NormalTok{,  }\FloatTok{7.104886}\NormalTok{,  }\FloatTok{7.963839}\NormalTok{,  }\FloatTok{9.161322}\NormalTok{,}
\FloatTok{10.271221}\NormalTok{,  }\FloatTok{9.822960}\NormalTok{, }\FloatTok{10.347376}\NormalTok{, }\FloatTok{11.703610}\NormalTok{, }\FloatTok{13.679876}\NormalTok{, }\FloatTok{13.413696}\NormalTok{)}
\end{Highlighting}
\end{Shaded}

\section{Кросс-корреляции с временным лагом между индексами и
уловами}\label{ux43aux440ux43eux441ux441-ux43aux43eux440ux440ux435ux43bux44fux446ux438ux438-ux441-ux432ux440ux435ux43cux435ux43dux43dux44bux43c-ux43bux430ux433ux43eux43c-ux43cux435ux436ux434ux443-ux438ux43dux434ux435ux43aux441ux430ux43cux438-ux438-ux443ux43bux43eux432ux430ux43cux438}

\begin{Shaded}
\begin{Highlighting}[]
\CommentTok{\# График кросс{-}корреляции: Catch и BESSindex (только данные без пропусков)}
\FunctionTok{ccf}\NormalTok{(}\FunctionTok{na.omit}\NormalTok{(Catch), }\FunctionTok{na.omit}\NormalTok{(BESSIndex),}
    \AttributeTok{main =} \StringTok{"Кросс{-}корреляция: Уловы и BESSindex"}\NormalTok{,}
    \AttributeTok{xlab =} \StringTok{"Лаг (годы)"}\NormalTok{, }\AttributeTok{ylab =} \StringTok{"Корреляция"}\NormalTok{)}


\CommentTok{\# График кросс{-}корреляции: Catch и CPUEIndex (только данные без пропусков)}
\FunctionTok{ccf}\NormalTok{(}\FunctionTok{na.omit}\NormalTok{(Catch), }\FunctionTok{na.omit}\NormalTok{(CPUEIndex),}
    \AttributeTok{main =} \StringTok{"Кросс{-}корреляция: Уловы и CPUEIndex"}\NormalTok{,}
    \AttributeTok{xlab =} \StringTok{"Лаг (годы)"}\NormalTok{, }\AttributeTok{ylab =} \StringTok{"Корреляция"}\NormalTok{)}
\end{Highlighting}
\end{Shaded}

\begin{figure}[H]

{\centering \includegraphics[width=0.6\linewidth,height=\textheight,keepaspectratio]{images/SPICT1.PNG}

}

\caption{Рис. 1.: График кросс-корреляции: Catch и BESSindex}

\end{figure}%

\begin{figure}[H]

{\centering \includegraphics[width=0.6\linewidth,height=\textheight,keepaspectratio]{images/SPICT2.PNG}

}

\caption{Рис. 2.: График кросс-корреляции: Catch и CPUEIndex}

\end{figure}%

\section{Подготовка данных для
SPiCT}\label{ux43fux43eux434ux433ux43eux442ux43eux432ux43aux430-ux434ux430ux43dux43dux44bux445-ux434ux43bux44f-spict}

\begin{Shaded}
\begin{Highlighting}[]
\DocumentationTok{\#\# 3.1 Форматирование данных в список SPiCT}
\NormalTok{input\_new }\OtherTok{\textless{}{-}} \FunctionTok{list}\NormalTok{(}
  \AttributeTok{timeC =}\NormalTok{ Year,     }\CommentTok{\# Годы вылова}
  \AttributeTok{obsC =}\NormalTok{ Catch,     }\CommentTok{\# Значения вылова}
  \AttributeTok{timeI =} \FunctionTok{list}\NormalTok{(     }\CommentTok{\# Временные точки для индексов:}
\NormalTok{    Year }\SpecialCharTok{+} \FloatTok{0.5}\NormalTok{,     }\CommentTok{\#  CPUE {-} середина года (июль)}
\NormalTok{    Year }\SpecialCharTok{+} \FloatTok{0.75}     \CommentTok{\#  BESS {-} 3/4 года (октябрь)}
\NormalTok{  ),}
  \AttributeTok{obsI =} \FunctionTok{list}\NormalTok{(}
\NormalTok{    CPUEIndex,      }\CommentTok{\# Значения индекса CPUE}
\NormalTok{    BESSIndex       }\CommentTok{\# Значения индекса BESS}
\NormalTok{  )}
\NormalTok{)}

\DocumentationTok{\#\# 3.2 Проверка и подготовка входных данных}
\NormalTok{inp }\OtherTok{\textless{}{-}} \FunctionTok{check.inp}\NormalTok{(input\_new, }\AttributeTok{verbose =} \ConstantTok{TRUE}\NormalTok{)}
\end{Highlighting}
\end{Shaded}

Функция check.inp() выполняет подготовку и валидацию входных данных
перед моделированием. Основные задачи функции:

Проверяет наличие обязательных элементов: timeC (временные точки
вылова), obsC (значения вылова). Проверяет соответствие индексов (timeI
и obsI): одинаковое количество элементов в списках, соответствие длин
временных рядов. Обработка пропущенных значений. Автоматически
обрабатывает NA в начале вектора CPUEIndex (2005-2010). Удаляет или
маркирует пропущенные значения в соответствии с настройками пакета и пр.

\section{Настройка
модели}\label{ux43dux430ux441ux442ux440ux43eux439ux43aux430-ux43cux43eux434ux435ux43bux438}

\begin{Shaded}
\begin{Highlighting}[]
\CommentTok{\# {-}{-}{-}{-}{-}{-}{-}{-}{-}{-}{-}{-}{-}{-}{-}{-}{-}{-}{-} 4. НАСТРОЙКА МОДЕЛИ {-}{-}{-}{-}{-}{-}{-}{-}{-}{-}{-}{-}{-}{-}{-}{-}{-}{-}{-}{-}}

\DocumentationTok{\#\# 4.1 Установка априорных распределений}
\NormalTok{inp}\SpecialCharTok{$}\NormalTok{priors}\SpecialCharTok{$}\NormalTok{logn }\OtherTok{\textless{}{-}} \FunctionTok{c}\NormalTok{(}\FunctionTok{log}\NormalTok{(}\DecValTok{2}\NormalTok{), }\FloatTok{0.1}\NormalTok{, }\DecValTok{1}\NormalTok{)   }\CommentTok{\# Прайер для n (модель Шефера)}
\NormalTok{inp}\SpecialCharTok{$}\NormalTok{ini}\SpecialCharTok{$}\NormalTok{logn }\OtherTok{\textless{}{-}} \FunctionTok{log}\NormalTok{(}\DecValTok{2}\NormalTok{)                  }\CommentTok{\# Начальное значение}
\NormalTok{inp}\SpecialCharTok{$}\NormalTok{phases}\SpecialCharTok{$}\NormalTok{logn }\OtherTok{\textless{}{-}} \SpecialCharTok{{-}}\DecValTok{1}                   \CommentTok{\# Фиксируем параметр (не оцениваем)}

\NormalTok{inp}\SpecialCharTok{$}\NormalTok{priors}\SpecialCharTok{$}\NormalTok{logK }\OtherTok{\textless{}{-}} \FunctionTok{c}\NormalTok{(}\DecValTok{5}\NormalTok{, }\FloatTok{0.7}\NormalTok{, }\DecValTok{1}\NormalTok{)       }\CommentTok{\# Прайер для емкости среды (K)}
\NormalTok{inp}\SpecialCharTok{$}\NormalTok{priors}\SpecialCharTok{$}\NormalTok{logbkfrac }\OtherTok{\textless{}{-}} \FunctionTok{c}\NormalTok{(}\FunctionTok{log}\NormalTok{(}\FloatTok{0.75}\NormalTok{),}\FloatTok{0.25}\NormalTok{,}\DecValTok{1}\NormalTok{) }\CommentTok{\# Начальный уровень эксплуатации}



\DocumentationTok{\#\# 4.2 Настройка неопределенности данных}
\CommentTok{\# Повышаем неопределенность для последнего года вылова}
\NormalTok{inp}\SpecialCharTok{$}\NormalTok{stdevfacC[}\FunctionTok{length}\NormalTok{(inp}\SpecialCharTok{$}\NormalTok{stdevfacC)] }\OtherTok{\textless{}{-}} \DecValTok{2} 

\CommentTok{\# Повышаем неопределенность для последнего значения BESS}
\NormalTok{inp}\SpecialCharTok{$}\NormalTok{stdevfacI[[}\DecValTok{2}\NormalTok{]][}\FunctionTok{length}\NormalTok{(inp}\SpecialCharTok{$}\NormalTok{stdevfacI[[}\DecValTok{2}\NormalTok{]])] }\OtherTok{\textless{}{-}} \DecValTok{2} 

\DocumentationTok{\#\# 4.3 Настройка временного шага}
\NormalTok{inp}\SpecialCharTok{$}\NormalTok{dteuler }\OtherTok{\textless{}{-}} \DecValTok{1}\SpecialCharTok{/}\DecValTok{16}  \CommentTok{\# Более точная дискретизация (по умолчанию 1)}

\DocumentationTok{\#\# 4.4 Включение оценки ковариации}
\NormalTok{inp}\SpecialCharTok{$}\NormalTok{getJointPrecision }\OtherTok{\textless{}{-}} \ConstantTok{TRUE} \CommentTok{\# Для оценки случайных эффектов}
\end{Highlighting}
\end{Shaded}

\textbf{Установка априорных распределений}

\begin{verbatim}
inp$priors$logn <- c(log(2), 0.1, 1) inp$ini$logn <- log(2) inp$phases$logn <- -1
\end{verbatim}

\begin{enumerate}
\def\labelenumi{\arabic{enumi}.}
\item
  \textbf{Прайер для параметра n}: Устанавливаем лог-нормальное
  распределение для экспоненты в продукционном уравнении, фиксируя
  модель Шефера (n=2).
\item
  \textbf{Начальное значение}: Задаем стартовую точку для оптимизации
  как log(2), что соответствует n=2.
\item
  \textbf{Фиксация параметра}: Флаг \textbf{\texttt{-1}} исключает n из
  оценки, делая его константой (упрощает модель).
\item
  \textbf{Биологический смысл}: Обеспечивает реалистичную форму кривой
  производства (параболическую).
\end{enumerate}

\begin{verbatim}
inp$priors$logK <- c(5, 0.7, 1)
\end{verbatim}

\begin{enumerate}
\def\labelenumi{\arabic{enumi}.}
\item
  \textbf{Прайер для ёмкости среды}: Задаем лог-нормальное распределение
  для K.
\item
  \textbf{Параметры}: Медиана exp(5)≈148 тыс.т, SD=0.7 в лог-шкале.
\item
  \textbf{Значение 1}: Активирует использование прайера в расчетах.
\item
  \textbf{Назначение}: Отражает экспертные знания о возможном диапазоне
  K.
\end{enumerate}

\begin{verbatim}
inp$priors$logbkfrac <- c(log(0.75),0.25,1)
\end{verbatim}

\begin{enumerate}
\def\labelenumi{\arabic{enumi}.}
\item
  \textbf{Приор для начальной биомассы}: Определяет распределение для
  B₀/K.
\item
  \textbf{Параметры}: Медиана 0.75 (начальная биомасса 75\% от K),
  SD=0.25.
\item
  \textbf{Биологический смысл}: Отражает гипотезу, что запас изначально
  был близок к неиспользуемому.
\item
  \textbf{Важность}: Помогает оценить начальные условия при ограниченных
  данных.
\end{enumerate}

\textbf{Настройка неопределенности данных}

\begin{verbatim}
inp$stdevfacC[length(inp$stdevfacC)] <- 2
\end{verbatim}

\begin{enumerate}
\def\labelenumi{\arabic{enumi}.}
\item
  \textbf{Цель}: Увеличить неопределенность последнего года вылова.
\item
  \textbf{Значение 2}: Стандартная ошибка увеличивается вдвое.
\item
  \textbf{Причина}: Последние данные часто предварительные или неполные.
\item
  \textbf{Эффект}: Снижает влияние потенциально ненадежной точки на
  оценку запаса.
\end{enumerate}

\begin{verbatim}
inp$stdevfacI[[2]][length(inp$stdevfacI[[2]])] <- 2
\end{verbatim}

\begin{enumerate}
\def\labelenumi{\arabic{enumi}.}
\item
  \textbf{Цель}: Повысить неопределенность последнего значения научного
  индекса (BESS).
\item
  \textbf{Синтаксис}: \textbf{\texttt{{[}{[}2{]}{]}}} указывает на
  второй индекс в списке.
\item
  \textbf{Обоснование}: Данные съемок могут требовать последующих
  корректировок.
\item
  \textbf{Результат}: Модель становится менее чувствительной к
  потенциальным аномалиям.
\end{enumerate}

\textbf{Настройка временного шага}

\begin{verbatim}
inp$dteuler <- 1/16
\end{verbatim}

\begin{enumerate}
\def\labelenumi{\arabic{enumi}.}
\item
  \textbf{Цель}: Улучшить точность численного интегрирования.
\item
  \textbf{Значение}: Шаг расчета ≈23 дня (вместо годового).
\item
  \textbf{Необходимость}: Для короткоживущих видов (н-р, креветка) с
  быстрой динамикой.
\item
  \textbf{Эффект}: Точнее учитывает внутригодовые изменения и
  сезонность.
\item
  \textbf{Цена}: Увеличивает время расчета в 3-5 раз.
\end{enumerate}

\textbf{Включение оценки ковариации}

\begin{verbatim}
inp$getJointPrecision <- TRUE
\end{verbatim}

\begin{enumerate}
\def\labelenumi{\arabic{enumi}.}
\item
  \textbf{Цель}: Рассчитать полную ковариационную матрицу параметров.
\item
  \textbf{Необходимость}: Для корректной оценки неопределенности
  производных показателей (B/BMSY).
\item
  \textbf{Что делает}: Учитывает взаимосвязи между параметрами и
  скрытыми состояниями.
\item
  \textbf{Преимущество}: Более реалистичные доверительные интервалы.
\item
  \textbf{Ограничение}: Увеличивает время расчета на 20-30\%.
\end{enumerate}

\section{Визуализация входных
данных}\label{ux432ux438ux437ux443ux430ux43bux438ux437ux430ux446ux438ux44f-ux432ux445ux43eux434ux43dux44bux445-ux434ux430ux43dux43dux44bux445}

\begin{Shaded}
\begin{Highlighting}[]
\CommentTok{\# {-}{-}{-}{-}{-}{-}{-}{-}{-}{-}{-}{-}{-}{-}{-}{-}{-} 5. ВИЗУАЛИЗАЦИЯ ВХОДНЫХ ДАННЫХ {-}{-}{-}{-}{-}{-}{-}{-}{-}{-}{-}{-}{-}{-}{-}{-}{-}}

\DocumentationTok{\#\# 5.1 Общий график данных}
\FunctionTok{plotspict.data}\NormalTok{(inp)}

\FunctionTok{plotspict.ci}\NormalTok{(inp)}
\end{Highlighting}
\end{Shaded}

\begin{figure}[H]

{\centering \includegraphics[width=0.6\linewidth,height=\textheight,keepaspectratio]{images/SPICT3.PNG}

}

\caption{Рис. 3.: Визуализация входных данных}

\end{figure}%

\section{Запуск
модели}\label{ux437ux430ux43fux443ux441ux43a-ux43cux43eux434ux435ux43bux438}

\begin{Shaded}
\begin{Highlighting}[]
\CommentTok{\# {-}{-}{-}{-}{-}{-}{-}{-}{-}{-}{-}{-}{-}{-}{-}{-}{-}{-}{-} 6. ЗАПУСК МОДЕЛИ {-}{-}{-}{-}{-}{-}{-}{-}{-}{-}{-}{-}{-}{-}{-}{-}{-}{-}{-}{-}}

\DocumentationTok{\#\# 6.1 Настройка оптимизатора}
\NormalTok{inp}\SpecialCharTok{$}\NormalTok{optimiser.control }\OtherTok{=} \FunctionTok{list}\NormalTok{(}\AttributeTok{iter.max =} \FloatTok{1e5}\NormalTok{, }\AttributeTok{eval.max =} \FloatTok{1e5}\NormalTok{)}

\DocumentationTok{\#\# 6.2 Подгонка модели}
\NormalTok{fit }\OtherTok{\textless{}{-}} \FunctionTok{fit.spict}\NormalTok{(inp)}

\DocumentationTok{\#\# 6.3 Добавление OSA{-}остатков}
\NormalTok{fit }\OtherTok{\textless{}{-}} \FunctionTok{calc.osa.resid}\NormalTok{(fit) }
\end{Highlighting}
\end{Shaded}

\textbf{Настройка оптимизатора}

\begin{verbatim}
inp$optimiser.control = list(iter.max = 1e5, eval.max = 1e5)
\end{verbatim}

\textbf{Что это делает:}

\begin{enumerate}
\def\labelenumi{\arabic{enumi}.}
\item
  Увеличивает максимальное количество итераций (iter.max) и вычислений
  (eval.max) для алгоритма оптимизации до 100,000
\item
  Обеспечивает, что процесс оптимизации не остановится преждевременно
  из-за ограничений по умолчанию
\item
  Особенно важно для сложных моделей с несколькими индексами или при
  плохой сходимости
\item
  Помогает алгоритму найти глобальный минимум функции правдоподобия
\item
  Предотвращает ошибки типа ``maximum iterations reached''
\end{enumerate}

\textbf{Подгонка модели}

\begin{verbatim}
fit <- fit.spict(inp)
\end{verbatim}

\textbf{Что происходит:}

\begin{enumerate}
\def\labelenumi{\arabic{enumi}.}
\item
  \textbf{Оценка параметров}: Ищет значения параметров (K, r, q и др.),
  максимизирующие правдоподобие
\item
  \textbf{Интегрирование уравнений}: Решает дифференциальные уравнения
  модели с шагом dteuler
\item
  \textbf{Расчет неопределенности}: Оценивает стандартные ошибки через
  обратную матрицу Гессе
\item
  \textbf{Диагностика сходимости}: Проверяет успешность оптимизации
  (fit\$opt\$convergence)
\item
  \textbf{Сохраняет результаты}: Формирует объект fit со всеми выходами
  модели
\end{enumerate}

\textbf{Ключевые процессы:}

\begin{itemize}
\item
  Численная оптимизация (обычно алгоритм nlminb)
\item
  Интегрирование методом Эйлера
\item
  Расчет логарифмического правдоподобия
\item
  Оценка матрицы Гессе
\end{itemize}

\textbf{OSA-остатки}

\begin{verbatim}
fit <- calc.osa.resid(fit)
\end{verbatim}

\textbf{Что такое OSA-остатки:}

\begin{enumerate}
\def\labelenumi{\arabic{enumi}.}
\item
  \textbf{One-Step-Ahead residuals} - остатки ``на один шаг вперед''
\item
  \textbf{Диагностический инструмент}: Показывают, насколько хорошо
  модель предсказывает следующее наблюдение
\item
  \textbf{Расчет}: Для каждого года t модель подгоняется по данным до
  t-1, затем сравнивается предсказание с фактическим значением в t
\end{enumerate}

\textbf{Что делают:}

\begin{enumerate}
\def\labelenumi{\arabic{enumi}.}
\item
  \textbf{Обнаружение систематических ошибок}: Выявляют смещения в
  предсказаниях
\item
  \textbf{Проверка независимости}: Автокорреляция в остатках указывает
  на неучтенные зависимости
\item
  \textbf{Оценка распределения ошибок}: Проверяют соответствие
  нормальному распределению
\item
  \textbf{Идентификация выбросов}: Помогают найти аномальные точки
  данных
\end{enumerate}

\section{Диагностика
сходимости}\label{ux434ux438ux430ux433ux43dux43eux441ux442ux438ux43aux430-ux441ux445ux43eux434ux438ux43cux43eux441ux442ux438}

\begin{Shaded}
\begin{Highlighting}[]
\CommentTok{\# {-}{-}{-}{-}{-}{-}{-}{-}{-}{-}{-}{-}{-}{-}{-}{-}{-} 7. ДИАГНОСТИКА СХОДИМОСТИ {-}{-}{-}{-}{-}{-}{-}{-}{-}{-}{-}{-}{-}{-}{-}{-}{-}}

\DocumentationTok{\#\# 7.1 Проверка сходимости}
\NormalTok{fit}\SpecialCharTok{$}\NormalTok{opt}\SpecialCharTok{$}\NormalTok{convergence  }\CommentTok{\# 0 = успешная сходимость}

\DocumentationTok{\#\# 7.2 Проверка конечных значений}
\FunctionTok{all}\NormalTok{(}\FunctionTok{is.finite}\NormalTok{(fit}\SpecialCharTok{$}\NormalTok{sd)) }\CommentTok{\# TRUE = все параметры конечны}
\end{Highlighting}
\end{Shaded}

\textbf{Диагностика сходимости модели SPiCT}

\textbf{Проверка сходимости (7.1)}\\
\textbf{\texttt{fit\$opt\$convergence}} --- индикатор успешности
оптимизации. Возвращаемое значение \textbf{0} означает, что алгоритм
оптимизации успешно сошелся к точке максимума правдоподобия. Это важно,
так как гарантирует:

\begin{enumerate}
\def\labelenumi{\arabic{enumi}.}
\item
  Параметры модели достигли стабильных значений
\item
  Градиент функции правдоподобия близок к нулю
\item
  Результаты статистически надежны
\item
  Модель готова для дальнейшего анализа и прогнозирования
\end{enumerate}

\textbf{Интерпретация кодов:}

\begin{itemize}
\item
  \textbf{0}: Успешная сходимость (идеальный результат)
\item
  \textbf{1}: Достигнут лимит итераций (требует увеличения iter.max)
\item
  \textbf{10}: Дегенерация симплекса (проблемы с данными)
\item
  Другие коды указывают на специфические ошибки оптимизации
\end{itemize}

\textbf{Проверка конечных значений (7.2)}\\
\textbf{\texttt{all(is.finite(fit\$sd))}} --- комплексная проверка
корректности оценок неопределенности. Результат \textbf{TRUE} означает:

\begin{enumerate}
\def\labelenumi{\arabic{enumi}.}
\item
  Все стандартные ошибки параметров являются вещественными числами
\item
  Отсутствуют патологические значения (NaN, Inf, NA) в матрице Гессе
\item
  Ковариационная матрица положительно определена
\item
  Оценки неопределенности надежны для построения доверительных
  интервалов
\end{enumerate}

\textbf{Что проверяет:}

\begin{itemize}
\item
  Корректность расчета стандартных ошибок
\item
  Отсутствие вырожденных параметров
\item
  Численную стабильность решения
\item
  Возможность интерпретировать результаты
\end{itemize}

\textbf{Последствия FALSE:}

\begin{enumerate}
\def\labelenumi{\arabic{enumi}.}
\item
  Невозможно построить достоверные доверительные интервалы
\item
  Риск ошибочных управленческих рекомендаций
\item
  Требуется пересмотр модели (упрощение, изменение прайеров)
\end{enumerate}

\section{Диагностика
модели}\label{ux434ux438ux430ux433ux43dux43eux441ux442ux438ux43aux430-ux43cux43eux434ux435ux43bux438}

\begin{Shaded}
\begin{Highlighting}[]
\CommentTok{\# {-}{-}{-}{-}{-}{-}{-}{-}{-}{-}{-}{-}{-}{-}{-}{-}{-} 8. ДИАГНОСТИКА МОДЕЛИ {-}{-}{-}{-}{-}{-}{-}{-}{-}{-}{-}{-}{-}{-}{-}{-}{-}}

\DocumentationTok{\#\# 8.1 График остатков}
\FunctionTok{plotspict.osar}\NormalTok{(fit) }

\DocumentationTok{\#\# 8.2 Общая диагностика}
\FunctionTok{plotspict.diagnostic}\NormalTok{(fit)}

\DocumentationTok{\#\# 8.3 Сравнение приоров и апостериорных распределений}
\FunctionTok{plotspict.priors}\NormalTok{(fit)}

\DocumentationTok{\#\# 8.4 Проверка корреляции параметров}
\FunctionTok{cov2cor}\NormalTok{(fit}\SpecialCharTok{$}\NormalTok{cov.fixed)        }\CommentTok{\# Матрица корреляций}
\FunctionTok{cov2cor}\NormalTok{(fit}\SpecialCharTok{$}\NormalTok{cov.fixed) }\SpecialCharTok{\textgreater{}} \FloatTok{0.8}  \CommentTok{\# Выявление сильных корреляций (\textgreater{}0.8)}
\end{Highlighting}
\end{Shaded}

\begin{figure}[H]

{\centering \includegraphics[width=0.6\linewidth,height=\textheight,keepaspectratio]{images/SPICT4.PNG}

}

\caption{Рис. 4.: График остатков}

\end{figure}%

\textbf{Остатки по времени} (Residuals vs.~time):

\begin{verbatim}
-   Показывает разницу между наблюдаемыми значениями и предсказаниями модели

-   Идеал: случайное распределение вокруг нуля

    ### **Что означает "Index 1 Bias p-val 0.7813"?**

    Это результат **статистического теста на систематическую ошибку** (bias) для первого индекса (в вашем случае - CPUE):

    1.  **Index 1**: Это ваш индекс CPUE (первый в списке индексов)

    2.  **Bias p-val**: p-value теста на наличие систематического смещения

    3.  **0.7813**: Конкретное значение p-value

    ### **Интерпретация значения 0.7813:**

    -   **p-value \> 0.05**: Нет статистически значимых доказательств систематической ошибки (смещения)

    -   **Высокое значение (0.7813)**: Сильно выше 0.05 → модель **не имеет значимого смещения** для этого индекса

    -   **Практический смысл**: Модель адекватно описывает динамику CPUE без постоянного завышения или занижения
\end{verbatim}

\begin{figure}[H]

{\centering \includegraphics[width=0.8\linewidth,height=\textheight,keepaspectratio]{images/SPICT5.PNG}

}

\caption{Рис. 5.: Общая диагностика}

\end{figure}%

\textbf{Структура диагностического графика}

\begin{enumerate}
\def\labelenumi{\arabic{enumi}.}
\item
  \textbf{Первый столбец:} Информация, связанная с данными по
  \textbf{вылову (catch)}.
\item
  \textbf{Второй и третий столбцы:} Информация, связанная с данными по
  \textbf{индексам биомассы}.
\end{enumerate}

\textbf{Строки графика содержат (сверху вниз):}

\begin{enumerate}
\def\labelenumi{\arabic{enumi}.}
\item
  \textbf{Логарифмы исходных рядов данных:}

  \begin{itemize}
  \item
    Верхний левый график в строке: \textbf{\texttt{log\ catch\ data}}
    (Логарифм данных по вылову).
  \item
    Срелний и правый графики в строке:
    \textbf{\texttt{log\ index\ data}} (Логарифм данных по индексу).
  \item
    \emph{Цель:} Визуально оценить исходные данные.
  \end{itemize}
\item
  \textbf{OSA-остатки:}

  \begin{itemize}
  \item
    График показывает разницу между наблюдаемыми и предсказанными на
    один шаг вперед значениями (в логарифмической шкале).
  \item
    \textbf{Заголовок графика} содержит \textbf{p-значение теста на
    смещение (bias test)}. Этот тест проверяет, отличается ли среднее
    остатков от нуля (систематическая ошибка).

    \begin{itemize}
    \item
      \textbf{\texttt{Bias\ p-val:\ X.XXXX}} (p-значение теста на
      смещение).
    \item
      \textbf{Зеленый заголовок:} Тест НЕ значим (нет свидетельств
      систематической ошибки, p \textgreater{} 0.05).
    \item
      \textbf{Красный заголовок:} Тест значим (есть свидетельства
      систематической ошибки, p \textless= 0.05).
    \end{itemize}
  \item
    Три теста незначимы (p=0.4386 для вылова, p=0.7813 и p=0.9472 для
    индексов), заголовки зеленые.
  \end{itemize}
\item
  \textbf{Эмпирическая автокорреляция остатков (ACF - Autocorrelation
  Function):}

  \begin{itemize}
  \item
    График показывает корреляцию остатков с их собственными
    лагированными значениями.
  \item
    Выполняется \textbf{два теста на значимую автокорреляцию:}

    \begin{enumerate}
    \def\labelenumii{\arabic{enumii}.}
    \item
      \textbf{Тест Льюнга-Бокса (Ljung-Box test):} Одновременный тест
      для нескольких лагов (здесь 4). Результат:

      \begin{itemize}
      \item
        \textbf{\texttt{LBox\ p-val:\ X.XXXX}} в заголовке графика.
      \item
        На примере: Три теста незначимы (p=0.1348 для вылова, p=0.68 и
        p=0.3602 для индексов).
      \end{itemize}
    \item
      \textbf{Тесты для отдельных лагов:} Пунктирные горизонтальные
      линии на графике показывают критические значения для значимой
      автокорреляции на каждом конкретном лаге. Если столбики
      автокорреляции (вертикальные линии) выходят за эти пунктирные
      линии, это свидетельствует о значимой автокорреляции на данном
      лаге.
    \end{enumerate}
  \item
    На примере : Никаких нарушений (значимой автокорреляции) не
    выявлено.
  \end{itemize}
\item
  \textbf{Тесты на нормальность остатков:}

  \begin{itemize}
  \item
    \textbf{QQ-график (Quantile-Quantile plot):} Сравнивает квантили
    остатков с квантилями теоретического нормального распределения.
    Прямая линия указывает на нормальность.
  \item
    \textbf{Тест Шапиро-Уилка (Shapiro-Wilk test):} Формальный тест на
    нормальность. Результат:

    \begin{itemize}
    \item
      \textbf{\texttt{Shapiro\ p-val:\ X.XXXX}} в заголовке графика.
    \item
      На примере: Три теста незначимы (p=0.1268 для вылова, p=0.9554 и
      p=0.9825 для индекса), нет свидетельств против нормальности.
    \end{itemize}
  \end{itemize}
\end{enumerate}

\textbf{Вывод для примера :}\\
Данные в этом примере \textbf{не показали значимых нарушений}
предположений модели (нет систематической ошибки, автокорреляции или
отклонения от нормальности остатков). Это \textbf{повышает уверенность в
полученных результатах} моделирования.

\textbf{Для обсуждения возможных нарушений и способов их устранения}
читатель отсылается к Pedersen and Berg (2017) см.
\url{https://github.com/DTUAqua/spict/raw/master/spict/inst/doc/spict_handbook.pdf}.

\begin{figure}[H]

{\centering \includegraphics[width=0.8\linewidth,height=\textheight,keepaspectratio]{images/SPICT6.PNG}

}

\caption{Рис. 6.: Сравнение априорных и апостериорных распределений}

\end{figure}%

\textbf{\texttt{n} (параметр формы продукционной функции)}

\begin{itemize}
\item
  \textbf{Описание:} Определяет форму продукционной кривой (зависимость
  роста биомассы от самой биомассы).
\item
  \textbf{Интерпретация:}

  \begin{itemize}
  \item
    \textbf{\texttt{n\ =\ 2}}: Классическая модель Шефера (симметричная
    кривая, максимум производства при \textbf{\texttt{B/K\ =\ 0.5}}).
  \item
    \textbf{\texttt{n\ ≠\ 2}}: Обобщенная модель Пеллы-Томлинсона
    (асимметричная кривая).
  \end{itemize}
\item
  \textbf{Важность:} влияет на оценку \textbf{\texttt{Bmsy}} (биомасса
  при MSY) и статус запаса (\textbf{\texttt{B/Bmsy}}).
\end{itemize}

\textbf{\texttt{alpha1}, \texttt{alpha2} (Параметры соотношения шумов
для индексов)}

\begin{itemize}
\item
  \textbf{Описание:} Логарифмы отношений стандартных отклонений ошибок
  наблюдения индексов (\textbf{\texttt{sdi1}}, \textbf{\texttt{sdi2}}) к
  стандартному отклонению процесса биомассы (\textbf{\texttt{sdb}}):\\
  \textbf{\texttt{alpha1\ =\ log(sdi1)\ -\ log(sdb)}}\strut \\
  \textbf{\texttt{alpha2\ =\ log(sdi2)\ -\ log(sdb)}}
\item
  \textbf{Интерпретация:}

  \begin{itemize}
  \item
    Отражают \emph{относительную точность} каждого индекса биомассы по
    сравнению с изменчивостью самой биомассы.
  \item
    \textbf{\texttt{alpha\ =\ 0}} (\textbf{\texttt{sdi\ =\ sdb}}): Шум
    индекса равен шуму биомассы.
  \item
    \textbf{\texttt{alpha\ \textless{}\ 0}}
    (\textbf{\texttt{sdi\ \textless{}\ sdb}}): Индекс точнее, чем
    изменчивость биомассы (хорошо).
  \item
    \textbf{\texttt{alpha\ \textgreater{}\ 0}}
    (\textbf{\texttt{sdi\ \textgreater{}\ sdb}}): Индекс шумнее, чем
    изменчивость биомассы (плохо).
  \end{itemize}
\item
  \textbf{Контекст:} Появляются, только если в модели используется
  \textbf{два или более индексов} биомассы.
\end{itemize}

\textbf{\texttt{beta} (Параметр соотношения шумов для уловов)}

\begin{itemize}
\item
  \textbf{Описание:} Логарифм отношения стандартного отклонения ошибок
  наблюдения уловов (\textbf{\texttt{sdc}}) к стандартному отклонению
  процесса промысловой смертности (\textbf{\texttt{sdf}}):\\
  \textbf{\texttt{beta\ =\ log(sdc)\ -\ log(sdf)}}
\item
  \textbf{Интерпретация:}

  \begin{itemize}
  \item
    Отражает \emph{относительную точность} данных по уловам по сравнению
    с изменчивостью промысловой смертности.
  \item
    \textbf{\texttt{beta\ =\ 0}} (\textbf{\texttt{sdc\ =\ sdf}}): Шум
    уловов равен шуму F.
  \item
    \textbf{\texttt{beta\ \textless{}\ 0}}
    (\textbf{\texttt{sdc\ \textless{}\ sdf}}): Данные по уловам точнее,
    чем изменчивость F (хорошо).
  \item
    \textbf{\texttt{beta\ \textgreater{}\ 0}}
    (\textbf{\texttt{sdc\ \textgreater{}\ sdf}}): Данные по уловам
    шумнее, чем изменчивость F (плохо).
  \end{itemize}
\end{itemize}

\textbf{\texttt{K} (емкость среды)}

\begin{itemize}
\item
  \textbf{Описание:} Максимальная равновесная биомасса
  неэксплуатируемого запаса (carrying capacity).
\item
  \textbf{Интерпретация:} Верхняя асимптота кривой роста. Один из самых
  важных и часто трудных для оценки параметров, особенно при
  ограниченных данных.
\item
  \textbf{Единицы измерения:} Те же, что и у биомассы (например, тонны,
  тыс. особей).
\end{itemize}

\subsection{\texorpdfstring{\textbf{\texttt{bkfrac} (Начальная
биомасса)}}{bkfrac (Начальная биомасса)}}\label{bkfrac-ux43dux430ux447ux430ux43bux44cux43dux430ux44f-ux431ux438ux43eux43cux430ux441ux441ux430}

\begin{itemize}
\item
  \textbf{Описание:} Доля от \textbf{\texttt{K}}, которую составляла
  биомасса запаса в \textbf{начальный год} временного ряда:\\
  \textbf{\texttt{bkfrac\ =\ B₀\ /\ K}}
\item
  \textbf{Интерпретация:}

  \begin{itemize}
  \item
    \textbf{\texttt{bkfrac\ =\ 1}}: Запас был в нетронутом состоянии в
    начальный год (\textbf{\texttt{B₀\ =\ K}}).
  \item
    \textbf{\texttt{bkfrac\ \textless{}\ 1}}: Запас уже был
    эксплуатируемым к началу ряда данных.
  \end{itemize}
\item
  \textbf{Важность:} Сильно влияет на реконструкцию исторической
  динамики биомассы, особенно если данные начинаются с периода
  интенсивного промысла.
\end{itemize}

\textbf{Почему именно эти параметры?}

Функция \textbf{\texttt{plotspict.priors(fit)}} по умолчанию
фокусируется на параметрах, для которых:

\begin{enumerate}
\def\labelenumi{\arabic{enumi}.}
\item
  \textbf{Заданы явные априорные распределения} пользователем (как
  \textbf{\texttt{logK}} или \textbf{\texttt{bkfrac}} в примерах).
\item
  \textbf{Применены стандартные полу-информативные априоры SPiCT} для
  стабилизации оценки в условиях ограниченных данных. К ним относятся
  \textbf{\texttt{n}}, \textbf{\texttt{alpha1}},
  \textbf{\texttt{alpha2}}, \textbf{\texttt{beta}}. SPiCT использует их,
  так как эти параметры (особенно \textbf{\texttt{n}} и соотношения
  шумов) часто плохо определяются только данными улова и индекса.
\end{enumerate}

\textbf{Сравнение априора и апостериора показывает:}

\begin{itemize}
\item
  Насколько \textbf{данные обновили наши первоначальные представления}
  (априорные) о параметре.
\item
  Насколько \textbf{информативны были априорные распределения}.
\item
  \textbf{Надежность оценки:} Сильное сужение апостериорного
  распределения относительно априорного говорит о том, что данные
  содержат информацию о параметре. Если апостериорное распределение
  почти совпадает с априорным, данные не добавили новой информации
  (оценка держится на априорном распределение (прайере)).
\end{itemize}

\section{Проверка корреляции
параметров}\label{ux43fux440ux43eux432ux435ux440ux43aux430-ux43aux43eux440ux440ux435ux43bux44fux446ux438ux438-ux43fux430ux440ux430ux43cux435ux442ux440ux43eux432}

\begin{Shaded}
\begin{Highlighting}[]
\DocumentationTok{\#\# 8.4 Проверка корреляции параметров}
\SpecialCharTok{\textgreater{}} \FunctionTok{cov2cor}\NormalTok{(fit}\SpecialCharTok{$}\NormalTok{cov.fixed)        }\CommentTok{\# Матрица корреляций}
\NormalTok{              logm        logK        logq        logq      logsdb      logsdf}
\NormalTok{logm    }\FloatTok{1.00000000} \SpecialCharTok{{-}}\FloatTok{0.44139706}  \FloatTok{0.24731406}  \FloatTok{0.26680092}  \FloatTok{0.10226103}  \FloatTok{0.05866772}
\NormalTok{logK   }\SpecialCharTok{{-}}\FloatTok{0.44139706}  \FloatTok{1.00000000} \SpecialCharTok{{-}}\FloatTok{0.78966591} \SpecialCharTok{{-}}\FloatTok{0.84564098} \SpecialCharTok{{-}}\FloatTok{0.14502822}  \FloatTok{0.03159055}
\NormalTok{logq    }\FloatTok{0.24731406} \SpecialCharTok{{-}}\FloatTok{0.78966591}  \FloatTok{1.00000000}  \FloatTok{0.92073272}  \FloatTok{0.09936208} \SpecialCharTok{{-}}\FloatTok{0.06019367}
\NormalTok{logq    }\FloatTok{0.26680092} \SpecialCharTok{{-}}\FloatTok{0.84564098}  \FloatTok{0.92073272}  \FloatTok{1.00000000}  \FloatTok{0.10661957} \SpecialCharTok{{-}}\FloatTok{0.06022707}
\NormalTok{logsdb  }\FloatTok{0.10226103} \SpecialCharTok{{-}}\FloatTok{0.14502822}  \FloatTok{0.09936208}  \FloatTok{0.10661957}  \FloatTok{1.00000000} \SpecialCharTok{{-}}\FloatTok{0.07548411}
\NormalTok{logsdf  }\FloatTok{0.05866772}  \FloatTok{0.03159055} \SpecialCharTok{{-}}\FloatTok{0.06019367} \SpecialCharTok{{-}}\FloatTok{0.06022707} \SpecialCharTok{{-}}\FloatTok{0.07548411}  \FloatTok{1.00000000}
\NormalTok{logsdi  }\FloatTok{0.04194204} \SpecialCharTok{{-}}\FloatTok{0.06650572}  \FloatTok{0.12991301}  \FloatTok{0.13206072}  \FloatTok{0.05802810} \SpecialCharTok{{-}}\FloatTok{0.02826486}
\NormalTok{logsdi }\SpecialCharTok{{-}}\FloatTok{0.02144997}  \FloatTok{0.05063718} \SpecialCharTok{{-}}\FloatTok{0.08833322} \SpecialCharTok{{-}}\FloatTok{0.09060172}  \FloatTok{0.04142955}  \FloatTok{0.00688377}
\NormalTok{logsdc }\SpecialCharTok{{-}}\FloatTok{0.03142644} \SpecialCharTok{{-}}\FloatTok{0.10430756}  \FloatTok{0.07611708}  \FloatTok{0.07988704}  \FloatTok{0.19380037} \SpecialCharTok{{-}}\FloatTok{0.38345908}
\NormalTok{            logsdi       logsdi       logsdc}
\NormalTok{logm    }\FloatTok{0.04194204} \SpecialCharTok{{-}}\FloatTok{0.021449974} \SpecialCharTok{{-}}\FloatTok{0.031426440}
\NormalTok{logK   }\SpecialCharTok{{-}}\FloatTok{0.06650572}  \FloatTok{0.050637182} \SpecialCharTok{{-}}\FloatTok{0.104307558}
\NormalTok{logq    }\FloatTok{0.12991301} \SpecialCharTok{{-}}\FloatTok{0.088333221}  \FloatTok{0.076117083}
\NormalTok{logq    }\FloatTok{0.13206072} \SpecialCharTok{{-}}\FloatTok{0.090601717}  \FloatTok{0.079887038}
\NormalTok{logsdb  }\FloatTok{0.05802810}  \FloatTok{0.041429548}  \FloatTok{0.193800371}
\NormalTok{logsdf }\SpecialCharTok{{-}}\FloatTok{0.02826486}  \FloatTok{0.006883770} \SpecialCharTok{{-}}\FloatTok{0.383459078}
\NormalTok{logsdi  }\FloatTok{1.00000000} \SpecialCharTok{{-}}\FloatTok{0.024282304}  \FloatTok{0.029024203}
\NormalTok{logsdi }\SpecialCharTok{{-}}\FloatTok{0.02428230}  \FloatTok{1.000000000} \SpecialCharTok{{-}}\FloatTok{0.005068705}
\NormalTok{logsdc  }\FloatTok{0.02902420} \SpecialCharTok{{-}}\FloatTok{0.005068705}  \FloatTok{1.000000000}
\SpecialCharTok{\textgreater{}} \FunctionTok{cov2cor}\NormalTok{(fit}\SpecialCharTok{$}\NormalTok{cov.fixed) }\SpecialCharTok{\textgreater{}} \FloatTok{0.8}  \CommentTok{\# Выявление сильных корреляций (\textgreater{}0.8)}
\NormalTok{        logm  logK  logq  logq logsdb logsdf logsdi logsdi logsdc}
\NormalTok{logm    }\ConstantTok{TRUE} \ConstantTok{FALSE} \ConstantTok{FALSE} \ConstantTok{FALSE}  \ConstantTok{FALSE}  \ConstantTok{FALSE}  \ConstantTok{FALSE}  \ConstantTok{FALSE}  \ConstantTok{FALSE}
\NormalTok{logK   }\ConstantTok{FALSE}  \ConstantTok{TRUE} \ConstantTok{FALSE} \ConstantTok{FALSE}  \ConstantTok{FALSE}  \ConstantTok{FALSE}  \ConstantTok{FALSE}  \ConstantTok{FALSE}  \ConstantTok{FALSE}
\NormalTok{logq   }\ConstantTok{FALSE} \ConstantTok{FALSE}  \ConstantTok{TRUE}  \ConstantTok{TRUE}  \ConstantTok{FALSE}  \ConstantTok{FALSE}  \ConstantTok{FALSE}  \ConstantTok{FALSE}  \ConstantTok{FALSE}
\NormalTok{logq   }\ConstantTok{FALSE} \ConstantTok{FALSE}  \ConstantTok{TRUE}  \ConstantTok{TRUE}  \ConstantTok{FALSE}  \ConstantTok{FALSE}  \ConstantTok{FALSE}  \ConstantTok{FALSE}  \ConstantTok{FALSE}
\NormalTok{logsdb }\ConstantTok{FALSE} \ConstantTok{FALSE} \ConstantTok{FALSE} \ConstantTok{FALSE}   \ConstantTok{TRUE}  \ConstantTok{FALSE}  \ConstantTok{FALSE}  \ConstantTok{FALSE}  \ConstantTok{FALSE}
\NormalTok{logsdf }\ConstantTok{FALSE} \ConstantTok{FALSE} \ConstantTok{FALSE} \ConstantTok{FALSE}  \ConstantTok{FALSE}   \ConstantTok{TRUE}  \ConstantTok{FALSE}  \ConstantTok{FALSE}  \ConstantTok{FALSE}
\NormalTok{logsdi }\ConstantTok{FALSE} \ConstantTok{FALSE} \ConstantTok{FALSE} \ConstantTok{FALSE}  \ConstantTok{FALSE}  \ConstantTok{FALSE}   \ConstantTok{TRUE}  \ConstantTok{FALSE}  \ConstantTok{FALSE}
\NormalTok{logsdi }\ConstantTok{FALSE} \ConstantTok{FALSE} \ConstantTok{FALSE} \ConstantTok{FALSE}  \ConstantTok{FALSE}  \ConstantTok{FALSE}  \ConstantTok{FALSE}   \ConstantTok{TRUE}  \ConstantTok{FALSE}
\NormalTok{logsdc }\ConstantTok{FALSE} \ConstantTok{FALSE} \ConstantTok{FALSE} \ConstantTok{FALSE}  \ConstantTok{FALSE}  \ConstantTok{FALSE}  \ConstantTok{FALSE}  \ConstantTok{FALSE}   \ConstantTok{TRUE}
\SpecialCharTok{\textgreater{}} 
\end{Highlighting}
\end{Shaded}

\textbf{Анализ матрицы корреляций параметров модели SPiCT}

Команды выполняют два действия:

\begin{enumerate}
\def\labelenumi{\arabic{enumi}.}
\item
  \textbf{\texttt{cov2cor(fit\$cov.fixed)}} - преобразует матрицу
  ковариаций в матрицу корреляций
\item
  \textbf{\texttt{cov2cor(fit\$cov.fixed)\ \textgreater{}\ 0.8}} -
  выявляет сильные корреляции (\textgreater0.8)
\end{enumerate}

\textbf{Умеренные корреляции (\textbar r\textbar{} \textgreater{} 0.7):}

\begin{itemize}
\item
  \textbf{logK и logq (-0.79)}:\\
  Классическая \textbf{отрицательная корреляция между емкостью среды и
  уловистостью}. Означает, что:

  \begin{itemize}
  \item
    Данные можно объяснить либо:

    \begin{itemize}
    \item
      Большим запасом (высокий K) с низкой уловистостью (низкий q)
    \item
      Или малым запасом (низкий K) с высокой уловистостью (высокий q)
    \end{itemize}
  \item
    Типично для моделей с ограниченными данными
  \end{itemize}
\end{itemize}

\section{Визуализация
резульататов}\label{ux432ux438ux437ux443ux430ux43bux438ux437ux430ux446ux438ux44f-ux440ux435ux437ux443ux43bux44cux430ux442ux430ux442ux43eux432}

\begin{Shaded}
\begin{Highlighting}[]
\SpecialCharTok{{-}{-}{-}{-}{-}{-}{-}{-}{-}{-}{-}{-}{-}{-}{-}{-}{-}} \FloatTok{9.}\NormalTok{ ВИЗУАЛИЗАЦИЯ РЕЗУЛЬТАТОВ }\SpecialCharTok{{-}{-}{-}{-}{-}{-}{-}{-}{-}{-}{-}{-}{-}{-}{-}{-}{-}}

\DocumentationTok{\#\# 9.1 Основные графики}

\FunctionTok{plot}\NormalTok{(fit) \textbackslash{}}\CommentTok{\# Комплексный отчет}

\DocumentationTok{\#\# 9.2 Биомасса в абсолютных величинах}

\FunctionTok{plotspict.biomass}\NormalTok{( fit, }\AttributeTok{logax =} \ConstantTok{FALSE}\NormalTok{, \textbackslash{}}\CommentTok{\# Линейная шкала main = "Абсолютная биомасса", ylim = c(0, 250), \textbackslash{}\# Ограничение по оси Y plot.obs = TRUE, \textbackslash{}\# Отображать наблюдения xlab = "Год", CI = 0.95, \textbackslash{}\# 95\% доверительный интервал qlegend = FALSE, rel.axes = TRUE, rel.ci = TRUE )}

\DocumentationTok{\#\# 9.3 Относительная биомасса (B/Bmsy)}

\FunctionTok{plotspict.bbmsy}\NormalTok{(fit,}\AttributeTok{qlegend =} \ConstantTok{FALSE}\NormalTok{)}

\DocumentationTok{\#\# 9.4 Вылов}

\FunctionTok{plotspict.catch}\NormalTok{(fit,}\AttributeTok{qlegend =} \ConstantTok{FALSE}\NormalTok{)}

\DocumentationTok{\#\# 9.5 Относительная смертность (F/Fmsy)}

\FunctionTok{plotspict.ffmsy}\NormalTok{(fit,}\AttributeTok{qlegend =} \ConstantTok{FALSE}\NormalTok{)}

\DocumentationTok{\#\# 9.6 Продукционная кривая}

\FunctionTok{plotspict.production}\NormalTok{(fit)}

\DocumentationTok{\#\# 9.7 Kobe plot}

\FunctionTok{plotspict.fb}\NormalTok{(fit, }\AttributeTok{ylim=}\FunctionTok{c}\NormalTok{(}\DecValTok{0}\NormalTok{, }\FloatTok{0.5}\NormalTok{), }\AttributeTok{xlim=}\FunctionTok{c}\NormalTok{(}\DecValTok{0}\NormalTok{, }\DecValTok{200}\NormalTok{))}
\end{Highlighting}
\end{Shaded}

\begin{figure}[H]

{\centering \includegraphics[width=0.8\linewidth,height=\textheight,keepaspectratio]{images/SPICT7.PNG}

}

\caption{Рис. 7.: Комплексный отчет}

\end{figure}%

\begin{figure}[H]

{\centering \includegraphics[width=0.6\linewidth,height=\textheight,keepaspectratio]{images/SPICT8.PNG}

}

\caption{Рис. 8.: Динамика абсолютной биомассы}

\end{figure}%

\begin{figure}[H]

{\centering \includegraphics[width=0.6\linewidth,height=\textheight,keepaspectratio]{images/SPICT9.PNG}

}

\caption{Рис. 9.: Динамика относительной биомассы}

\end{figure}%

\begin{figure}[H]

{\centering \includegraphics[width=0.6\linewidth,height=\textheight,keepaspectratio]{images/SPICT10.PNG}

}

\caption{Рис. 10.: Динамика относительной смертности}

\end{figure}%

\begin{figure}[H]

{\centering \includegraphics[width=0.6\linewidth,height=\textheight,keepaspectratio]{images/SPICT11.PNG}

}

\caption{Рис. 11.: Продукционная кривая}

\end{figure}%

\begin{figure}[H]

{\centering \includegraphics[width=0.6\linewidth,height=\textheight,keepaspectratio]{images/SPICT12.PNG}

}

\caption{Рис. 12.: Kobe plot}

\end{figure}%

График показывает динамику биомассы и смертности от промысла с
начального года (здесь 2005), обозначенного кругом, до конечного года
(здесь 2025), обозначенного квадратом. Жёлтый ромб обозначает среднюю
биомассу за длительный период при сохранении текущей (2025) промысловой
нагрузки. Эта точка может быть интерпретирована как равновесное значение
вылова и обозначена в легенде как E(B∞) как статистический способ
выражения ожидания биомассы при t → ∞. Поскольку текущая промысловая
смертность близка к FMSY, ожидаемая долгосрочная биомасса близка к BMSY.
Серая затенённая область в форме банана обозначает 95\% доверительную
область пары FMSY, BMSY. Эту область важно визуализировать совместно,
поскольку две контрольные точки имеют сильную (отрицательную)
корреляцию.

\section{Анализ
результатов}\label{ux430ux43dux430ux43bux438ux437-ux440ux435ux437ux443ux43bux44cux442ux430ux442ux43eux432}

\begin{Shaded}
\begin{Highlighting}[]
\CommentTok{\# {-}{-}{-}{-}{-}{-}{-}{-}{-}{-}{-}{-}{-}{-}{-}{-}{-} 10. АНАЛИЗ РЕЗУЛЬТАТОВ {-}{-}{-}{-}{-}{-}{-}{-}{-}{-}{-}{-}{-}{-}{-}{-}{-}}

\DocumentationTok{\#\# 10.1 Краткий отчет}
\FunctionTok{summary}\NormalTok{(fit)}

\DocumentationTok{\#\# 10.2 Точечные оценки параметров}
\NormalTok{pars }\OtherTok{\textless{}{-}} \FunctionTok{sumspict.parest}\NormalTok{(fit)}

\DocumentationTok{\#\# 10.3 Ориентиры управления (стохастические)}
\FunctionTok{sumspict.srefpoints}\NormalTok{(fit)}

\DocumentationTok{\#\# 10.4 Ориентиры управления (детерминированные)}
\FunctionTok{sumspict.drefpoints}\NormalTok{(fit)}
\end{Highlighting}
\end{Shaded}

Краткий отчет

\begin{Shaded}
\begin{Highlighting}[]
\FunctionTok{mary}\NormalTok{(fit)}
\NormalTok{Convergence}\SpecialCharTok{:} \DecValTok{0}\NormalTok{  MSG}\SpecialCharTok{:}\NormalTok{ relative }\FunctionTok{convergence}\NormalTok{ (}\DecValTok{4}\NormalTok{)}
\NormalTok{Objective }\ControlFlowTok{function}\NormalTok{ at optimum}\SpecialCharTok{:} \SpecialCharTok{{-}}\FloatTok{4.8303552}
\NormalTok{Euler time }\FunctionTok{step}\NormalTok{ (years)}\SpecialCharTok{:}  \DecValTok{1}\SpecialCharTok{/}\DecValTok{16}\NormalTok{ or }\FloatTok{0.0625}
\NormalTok{Nobs C}\SpecialCharTok{:} \DecValTok{20}\NormalTok{,  Nobs I1}\SpecialCharTok{:} \DecValTok{20}\NormalTok{,  Nobs I2}\SpecialCharTok{:} \DecValTok{19}

\NormalTok{Residual }\FunctionTok{diagnostics}\NormalTok{ (p}\SpecialCharTok{{-}}\NormalTok{values)}
\NormalTok{    shapiro   bias    acf   LBox shapiro bias acf LBox  }
\NormalTok{ C   }\FloatTok{0.1268} \FloatTok{0.4386} \FloatTok{0.0562} \FloatTok{0.1348}       \SpecialCharTok{{-}}    \SpecialCharTok{{-}}\NormalTok{   .    }\SpecialCharTok{{-}}  
\NormalTok{ I1  }\FloatTok{0.9554} \FloatTok{0.7813} \FloatTok{0.3360} \FloatTok{0.6800}       \SpecialCharTok{{-}}    \SpecialCharTok{{-}}   \SpecialCharTok{{-}}    \SpecialCharTok{{-}}  
\NormalTok{ I2  }\FloatTok{0.9825} \FloatTok{0.9472} \FloatTok{0.1390} \FloatTok{0.3602}       \SpecialCharTok{{-}}    \SpecialCharTok{{-}}   \SpecialCharTok{{-}}    \SpecialCharTok{{-}}  

\NormalTok{Priors}
\NormalTok{      logn  }\SpecialCharTok{\textasciitilde{}}\NormalTok{  dnorm[}\FunctionTok{log}\NormalTok{(}\DecValTok{2}\NormalTok{), }\FloatTok{0.1}\SpecialCharTok{\^{}}\DecValTok{2}\NormalTok{]}
\NormalTok{  logalpha  }\SpecialCharTok{\textasciitilde{}}\NormalTok{  dnorm[}\FunctionTok{log}\NormalTok{(}\DecValTok{1}\NormalTok{), }\DecValTok{2}\SpecialCharTok{\^{}}\DecValTok{2}\NormalTok{]}
\NormalTok{   logbeta  }\SpecialCharTok{\textasciitilde{}}\NormalTok{  dnorm[}\FunctionTok{log}\NormalTok{(}\DecValTok{1}\NormalTok{), }\DecValTok{2}\SpecialCharTok{\^{}}\DecValTok{2}\NormalTok{]}
\NormalTok{      logK  }\SpecialCharTok{\textasciitilde{}}\NormalTok{  dnorm[}\FunctionTok{log}\NormalTok{(}\FloatTok{148.413}\NormalTok{), }\FloatTok{0.7}\SpecialCharTok{\^{}}\DecValTok{2}\NormalTok{]}
\NormalTok{ logbkfrac  }\SpecialCharTok{\textasciitilde{}}\NormalTok{  dnorm[}\FunctionTok{log}\NormalTok{(}\FloatTok{0.75}\NormalTok{), }\FloatTok{0.25}\SpecialCharTok{\^{}}\DecValTok{2}\NormalTok{]}

\NormalTok{Fixed parameters}
\NormalTok{   fixed.value  }
\NormalTok{ n           }\DecValTok{2}  

\NormalTok{Model parameter estimates w }\DecValTok{95}\NormalTok{\% CI }
\NormalTok{           estimate       cilow       ciupp    log.est  }
\NormalTok{ alpha1  }\FloatTok{11.9450241}   \FloatTok{2.6794895}  \FloatTok{53.2502930}  \FloatTok{2.4803148}  
\NormalTok{ alpha2   }\FloatTok{5.0683878}   \FloatTok{1.1265510}  \FloatTok{22.8028330}  \FloatTok{1.6230228}  
\NormalTok{ beta     }\FloatTok{0.1879597}   \FloatTok{0.0371609}   \FloatTok{0.9507000} \SpecialCharTok{{-}}\FloatTok{1.6715278}  
\NormalTok{ r        }\FloatTok{0.3769549}   \FloatTok{0.2895415}   \FloatTok{0.4907586} \SpecialCharTok{{-}}\FloatTok{0.9756298}  
\NormalTok{ rc       }\FloatTok{0.3769549}   \FloatTok{0.2895415}   \FloatTok{0.4907586} \SpecialCharTok{{-}}\FloatTok{0.9756298}  
\NormalTok{ rold     }\FloatTok{0.3769549}   \FloatTok{0.2895415}   \FloatTok{0.4907586} \SpecialCharTok{{-}}\FloatTok{0.9756298}  
\NormalTok{ m       }\FloatTok{17.8600178}  \FloatTok{16.2681570}  \FloatTok{19.6076444}  \FloatTok{2.8825646}  
\NormalTok{ K      }\FloatTok{189.5188972} \FloatTok{153.7796069} \FloatTok{233.5642100}  \FloatTok{5.2444887}  
\NormalTok{ q1       }\FloatTok{0.1446167}   \FloatTok{0.1110579}   \FloatTok{0.1883161} \SpecialCharTok{{-}}\FloatTok{1.9336685}  
\NormalTok{ q2       }\FloatTok{0.1105024}   \FloatTok{0.0861636}   \FloatTok{0.1417164} \SpecialCharTok{{-}}\FloatTok{2.2027177}  
\NormalTok{ sdb      }\FloatTok{0.0179239}   \FloatTok{0.0040818}   \FloatTok{0.0787065} \SpecialCharTok{{-}}\FloatTok{4.0216194}  
\NormalTok{ sdf      }\FloatTok{0.3205493}   \FloatTok{0.2181678}   \FloatTok{0.4709762} \SpecialCharTok{{-}}\FloatTok{1.1377191}  
\NormalTok{ sdi1     }\FloatTok{0.2141016}   \FloatTok{0.1563268}   \FloatTok{0.2932285} \SpecialCharTok{{-}}\FloatTok{1.5413046}  
\NormalTok{ sdi2     }\FloatTok{0.0908454}   \FloatTok{0.0648365}   \FloatTok{0.1272875} \SpecialCharTok{{-}}\FloatTok{2.3985967}  
\NormalTok{ sdc      }\FloatTok{0.0602503}   \FloatTok{0.0143610}   \FloatTok{0.2527760} \SpecialCharTok{{-}}\FloatTok{2.8092469}  
 
\NormalTok{Deterministic reference }\FunctionTok{points}\NormalTok{ (Drp)}
\NormalTok{         estimate      cilow       ciupp   log.est  }
\NormalTok{ Bmsyd }\FloatTok{94.7594486} \FloatTok{76.8898035} \FloatTok{116.7821050}  \FloatTok{4.551342}  
\NormalTok{ Fmsyd  }\FloatTok{0.1884774}  \FloatTok{0.1447707}   \FloatTok{0.2453793} \SpecialCharTok{{-}}\FloatTok{1.668777}  
\NormalTok{ MSYd  }\FloatTok{17.8600178} \FloatTok{16.2681570}  \FloatTok{19.6076444}  \FloatTok{2.882565}  
\NormalTok{Stochastic reference }\FunctionTok{points}\NormalTok{ (Srp)}
\NormalTok{        estimate      cilow       ciupp   log.est  rel.diff.Drp  }
\NormalTok{ Bmsys }\FloatTok{94.710229} \FloatTok{76.8402135} \FloatTok{116.7361074}  \FloatTok{4.550822} \SpecialCharTok{{-}}\FloatTok{0.0005196907}  
\NormalTok{ Fmsys  }\FloatTok{0.188398}  \FloatTok{0.1447201}   \FloatTok{0.2452583} \SpecialCharTok{{-}}\FloatTok{1.669199} \SpecialCharTok{{-}}\FloatTok{0.0004216341}  
\NormalTok{ MSYs  }\FloatTok{17.843213} \FloatTok{16.2542730}  \FloatTok{19.5874793}  \FloatTok{2.881623} \SpecialCharTok{{-}}\FloatTok{0.0009418271}  

\NormalTok{States w }\DecValTok{95}\NormalTok{\% }\FunctionTok{CI}\NormalTok{ (inp}\SpecialCharTok{$}\NormalTok{msytype}\SpecialCharTok{:}\NormalTok{ s)}
\NormalTok{                   estimate      cilow       ciupp    log.est  }
\NormalTok{ B\_2024}\FloatTok{.94}      \FloatTok{118.8335053} \FloatTok{96.9318694} \FloatTok{145.6837887}  \FloatTok{4.7777234}  
\NormalTok{ F\_2024}\FloatTok{.94}        \FloatTok{0.1031939}  \FloatTok{0.0646828}   \FloatTok{0.1646340} \SpecialCharTok{{-}}\FloatTok{2.2711455}  
\NormalTok{ B\_2024}\FloatTok{.94}\SpecialCharTok{/}\NormalTok{Bmsy   }\FloatTok{1.2547061}  \FloatTok{1.0845329}   \FloatTok{1.4515812}  \FloatTok{0.2269014}  
\NormalTok{ F\_2024}\FloatTok{.94}\SpecialCharTok{/}\NormalTok{Fmsy   }\FloatTok{0.5477442}  \FloatTok{0.3417896}   \FloatTok{0.8778024} \SpecialCharTok{{-}}\FloatTok{0.6019469}  

\NormalTok{Predictions w }\DecValTok{95}\NormalTok{\% }\FunctionTok{CI}\NormalTok{ (inp}\SpecialCharTok{$}\NormalTok{msytype}\SpecialCharTok{:}\NormalTok{ s)}
\NormalTok{                 prediction       cilow       ciupp    log.est  }
\NormalTok{ B\_2026}\FloatTok{.00}      \FloatTok{123.0873098} \FloatTok{100.3197792} \FloatTok{151.0219217}  \FloatTok{4.8128939}  
\NormalTok{ F\_2026}\FloatTok{.00}        \FloatTok{0.1031941}   \FloatTok{0.0464383}   \FloatTok{0.2293155} \SpecialCharTok{{-}}\FloatTok{2.2711437}  
\NormalTok{ B\_2026}\FloatTok{.00}\SpecialCharTok{/}\NormalTok{Bmsy   }\FloatTok{1.2996200}   \FloatTok{1.1150104}   \FloatTok{1.5147950}  \FloatTok{0.2620719}  
\NormalTok{ F\_2026}\FloatTok{.00}\SpecialCharTok{/}\NormalTok{Fmsy   }\FloatTok{0.5477452}   \FloatTok{0.2458405}   \FloatTok{1.2204040} \SpecialCharTok{{-}}\FloatTok{0.6019451}  
\NormalTok{ Catch\_2025}\FloatTok{.00}   \FloatTok{12.4909882}   \FloatTok{7.3033391}  \FloatTok{21.3634865}  \FloatTok{2.5250074}  
 \FunctionTok{E}\NormalTok{(B\_inf)       }\FloatTok{137.4532783}          \ConstantTok{NA}          \ConstantTok{NA}  \FloatTok{4.9232841}  
\SpecialCharTok{\textgreater{}} 
\end{Highlighting}
\end{Shaded}

\textbf{Анализ результатов модели SPiCT}

\textbf{1. Сходимость модели}

\textbf{\texttt{Convergence:\ 0\ MSG:\ relative\ convergence\ (4)}}\strut \\
\emph{Интерпретация:} Код 0 указывает на успешную сходимость
оптимизации. Сообщение ``relative convergence'' подтверждает, что
алгоритм достиг локального минимума с заданной точностью. Результаты
могут считаться валидными.

\textbf{2. Целевая функция}

\textbf{\texttt{Objective\ function\ at\ optimum:\ -4.8303552}}\strut \\
\emph{Интерпретация:} Значение логарифмической апостериорной плотности
(с учетом априорных распределений) в точке оптимума. Более высокие
значения (менее отрицательные) указывают на лучшее соответствие модели
данным.

\textbf{3. Дискретизация времени}

\textbf{\texttt{Euler\ time\ step\ (years):\ 1/16\ or\ 0.0625}}\strut \\
\emph{Интерпретация:} Для решения дифференциальных уравнений использован
шаг Эйлера 0.0625 года (\textasciitilde23 дня), что обеспечивает высокую
точность расчетов.

\textbf{4. Данные наблюдений}

\begin{verbatim}
Nobs C: 20,  Nobs I1: 20,  Nobs I2: 19
\end{verbatim}

\emph{Интерпретация:}

\begin{itemize}
\tightlist
\item
  C: 20 точек данных по вылову (2005-2024 гг.)
\item
  I1: 20 значений индекса CPUE
\item
  I2: 19 значений индекса BESS (отсутствует первое наблюдение)
\end{itemize}

\textbf{5. Диагностика остатков}

\begin{verbatim}
Residual diagnostics (p-values)
shapiro   bias    acf   LBox
C   0.1268 0.4386 0.0562 0.1348
I1  0.9554 0.7813 0.3360 0.6800
I2  0.9825 0.9472 0.1390 0.3602  
\end{verbatim}

\emph{Ключевые тесты:}

\begin{itemize}
\tightlist
\item
  \textbf{Shapiro-Wilk:} Нормальность остатков (p \textgreater{} 0.05 →
  нормальность не отвергается)
\item
  \textbf{Bias test:} Систематическая ошибка (p \textgreater{} 0.05 →
  смещение отсутствует)
\item
  \textbf{ACF/Ljung-Box:} Автокорреляция (p \textless{} 0.1 для вылова →
  слабая автокорреляция)\\
  \emph{Заключение:} Остатки удовлетворительны, кроме возможной слабой
  автокорреляции в данных по вылову.
\end{itemize}

\textbf{6. Априорные распределения}

\begin{verbatim}
Priors
logn  ~  dnorm[log(2), 0.1^2]       # Фиксирован n = 2 (модель Шефера)   
logK  ~  dnorm[log(148.413), 0.7^2]  # K ~ 148.4 тыс. тонн (CV=70%)   
logbkfrac ~ dnorm[log(0.75), 0.25^2] # Начальная эксплуатация B/K = 0.75
\end{verbatim}

\emph{Интерпретация:} Использованы информативные априорные распределения
для ключевых параметров, что характерно для data-limited подходов.

\textbf{7. Оценки параметров модели}

\begin{verbatim}
Model parameter estimates w 95% CI
      estimate       cilow       ciupp  
K      189.5 [153.8 - 233.6]  # Емкость среды (тыс. тонн)  
r        0.38 [0.29 - 0.49]   # Внутренняя скорость роста 
q1       0.14 [0.11 - 0.19]   # Catchability CPUE 
q2       0.11 [0.09 - 0.14]   # Catchability BESS  
sdf      0.32 [0.22 - 0.47]   # SD процесса для F
\end{verbatim}

\emph{Ключевые выводы:} - Высокая неопределенность оценки K (дов.
интервал ±40\%) - Умеренная скорость восстановления запаса (r ≈ 38\% в
год) - Индекс BESS имеет более высокую catchability, чем CPUE

\textbf{8. Ориентиры управления}

\begin{verbatim}
Deterministic reference points (Drp)
estimate      95% CI  
Bmsyd   94.8 [76.9 - 116.8]  # Биомасса при MSY  
Fmsyd    0.19 [0.14 - 0.25]  # Смертность при MSY  
MSYd    17.9 [16.3 - 19.6]   # Макс. устойчивый вылов  

Stochastic reference points (Srp)          
estimate      rel.diff.Drp    
Bmsys   94.7         -0.05%        # Незначительные отличия от детерм. модели  
Fmsys    0.19        -0.04%
\end{verbatim}

\emph{Интерпретация:} Результаты устойчивы к стохастичности модели. MSY
≈ 18 тыс. тонн.

\textbf{9. Состояние запаса в 2024 г.}

\begin{verbatim}
States w 95% CI (inp$msytype: s)                   
estimate      95% CI  
B_2024.94        118.8 [96.9 - 145.7]  # Абсолютная биомасса (тыс. т) 
F_2024.94          0.10 [0.06 - 0.16]  # Смертность 
B/Bmsy             1.25 [1.08 - 1.45]  # Биомасса выше Bmsy
F/Fmsy             0.55 [0.34 - 0.88]  # Эксплуатация ниже Fmsy
\end{verbatim}

\emph{Оценка состояния:} Запас находится в благополучном состоянии (B
\textgreater{} Bmsy, F \textless{} Fmsy), но с высокой
неопределенностью.

\textbf{10. Прогнозы}

\begin{verbatim}
Predictions w 95% CI
B_2026.00      123.1 [100.3 - 151.0]  # Прогноз биомассы
Catch_2025.00   12.5 [7.3 - 21.4]     # Прогноз вылова на 2025 г.  
E(B_inf)       137.5                   # Ожидаемая равновесная биомасса
\end{verbatim}

\emph{Прогнозные показатели:} - Биомасса продолжит умеренный рост -
Рекомендуемый вылов на 2025 г. ≈ 12.5 тыс. тонн (дов. интервал ±57\%) -
Потенциальная равновесная биомасса на 16\% выше текущей

\textbf{Ключевые выводы:}

\begin{enumerate}
\def\labelenumi{\arabic{enumi}.}
\tightlist
\item
  Модель успешно сошлась с удовлетворительными остатками
\item
  Запас оценивается выше целевого уровня (B/Bmsy \textgreater{} 1)
\item
  Эксплуатация находится на безопасном уровне (F/Fmsy \textless{} 1)
\item
  Рекомендуемый вылов на 2025 г. --- 12.5 {[}7.3 - 21.4{]} тыс. тонн
\item
  Основные источники неопределенности: оценка K и прогноз вылова
\end{enumerate}

\section{Ретроспективный
анализ}\label{ux440ux435ux442ux440ux43eux441ux43fux435ux43aux442ux438ux432ux43dux44bux439-ux430ux43dux430ux43bux438ux437}

\begin{Shaded}
\begin{Highlighting}[]
\CommentTok{\# {-}{-}{-}{-}{-}{-}{-}{-}{-}{-}{-}{-}{-}{-} 11. РЕТРОСПЕКТИВНЫЙ АНАЛИЗ {-}{-}{-}{-}{-}{-}{-}{-}{-}{-}{-}{-}{-}{-}}

\DocumentationTok{\#\# 11.1 Запуск ретроспективного анализа}
\NormalTok{fit }\OtherTok{\textless{}{-}} \FunctionTok{retro}\NormalTok{(fit)}

\DocumentationTok{\#\# 11.2 Визуализация ретроспективы}
\FunctionTok{plotspict.retro}\NormalTok{(fit, }\AttributeTok{add.mohn =} \ConstantTok{TRUE}\NormalTok{, }\AttributeTok{CI =} \FloatTok{0.95}\NormalTok{)}

\DocumentationTok{\#\# Интерпретация коэффициента Мона (Mohn\textquotesingle{}s rho):}
\DocumentationTok{\#\# Долгоживущие виды: |rho| \textgreater{} 0.2 значимо}
\DocumentationTok{\#\# Короткоживущие виды: |rho| \textgreater{} 0.3 значимо}
\end{Highlighting}
\end{Shaded}

\begin{figure}[H]

{\centering \includegraphics[width=0.8\linewidth,height=\textheight,keepaspectratio]{images/SPICT13.PNG}

}

\caption{Рис. 13.: Визуализация ретроспективного анализа}

\end{figure}%

\textbf{Суть ретроспективного анализа (Retrospective Analysis)}

\textbf{Цель:} Оценка устойчивости модели и выявление систематических
смещений (ретроспективного сдвига) в оценках состояния запаса при
добавлении новых данных.

\textbf{Метод:}

\begin{enumerate}
\def\labelenumi{\arabic{enumi}.}
\item
  Модель последовательно переоценивается с исключением по 1 последнему
  году данных (например: 2005-2023, 2005-2022 и т.д.)
\item
  Для каждого урезанного периода рассчитываются показатели (B/Bmsy,
  F/Fmsy) в \textbf{перекрывающиеся годы}
\item
  Оценки сравниваются с ``базовой'' моделью (со всеми данными)
\end{enumerate}

\textbf{Коэффициент Мона (Mohn's rho)}

\textbf{Формула расчета:}

\begin{verbatim}
ρ = 1/N * Σ [ (X_retro,i - X_base,i) / X_base,i ]
\end{verbatim}

где:

\begin{itemize}
\item
  \textbf{\texttt{N}} -- число исключенных лет
\item
  \textbf{\texttt{X\_retro,i}} -- оценка параметра (напр. B/Bmsy) в году
  i по урезанным данным
\item
  \textbf{\texttt{X\_base,i}} -- оценка того же параметра в году i по
  полным данным
\end{itemize}

\textbf{Интерпретация результатов в вашем случае:}

\begin{verbatim}
        FFmsy         BBmsy   0.0028358361 -0.0002021046 
\end{verbatim}

\begin{enumerate}
\def\labelenumi{\arabic{enumi}.}
\item
  \textbf{Для F/Fmsy:} ρ = 0.0028 (0.28\%)

  \begin{itemize}
  \item
    Положительное значение: текущие оценки F/Fmsy \textbf{слегка
    завышены} по сравнению с ретроспективой
  \item
    Величина \textless{} 0.3\% -- незначима
  \end{itemize}
\item
  \textbf{Для B/Bmsy:} ρ = -0.0002 (-0.02\%)

  \begin{itemize}
  \item
    Отрицательное значение: текущие оценки B/Bmsy \textbf{слегка
    занижены}
  \item
    Величина \textless{} 0.1\% -- пренебрежимо мала
  \end{itemize}
\end{enumerate}

\textbf{Вывод для модели:}

\begin{itemize}
\item
  Коэффициенты Мона близки к нулю (\textbar ρ\textbar{} \textless{}
  0.005)
\item
  \textbf{Отсутствует статистически значимый ретроспективный сдвиг}
\item
  Модель демонстрирует высокую устойчивость к добавлению новых данных
\item
  Результаты можно считать надежными
\end{itemize}

\begin{quote}
\textbf{Важно!} Значимый сдвиг (\textbar ρ\textbar{} \textgreater{}
0.2-0.3) указывает на:

\begin{itemize}
\item
  Недостаточность данных
\item
  Проблемы со спецификацией модели
\item
  Систематические ошибки в данных
\item
  Необходимость пересмотра модели
\end{itemize}
\end{quote}

\section{Прогнозирование и сценарии
управления}\label{ux43fux440ux43eux433ux43dux43eux437ux438ux440ux43eux432ux430ux43dux438ux435-ux438-ux441ux446ux435ux43dux430ux440ux438ux438-ux443ux43fux440ux430ux432ux43bux435ux43dux438ux44f}

\begin{Shaded}
\begin{Highlighting}[]
\CommentTok{\# {-}{-}{-}{-}{-}{-}{-}{-}{-}{-}{-} 12. ПРОГНОЗИРОВАНИЕ И СЦЕНАРИИ УПРАВЛЕНИЯ {-}{-}{-}{-}{-}{-}{-}{-}{-}{-}{-}}

\DocumentationTok{\#\# 12.1 Установка интервала управления}
\NormalTok{inp}\SpecialCharTok{$}\NormalTok{maninterval }\OtherTok{\textless{}{-}} \FunctionTok{c}\NormalTok{(}\DecValTok{2025}\NormalTok{, }\DecValTok{2026}\NormalTok{) }\CommentTok{\# Годы прогноза}

\DocumentationTok{\#\# 12.2 Базовые сценарии управления}
\NormalTok{fit }\OtherTok{\textless{}{-}} \FunctionTok{manage}\NormalTok{(fit)}

\DocumentationTok{\#\# 12.3 Пользовательские сценарии (постоянный вылов)}
\NormalTok{catchvals }\OtherTok{=} \FunctionTok{c}\NormalTok{(}\DecValTok{10}\NormalTok{, }\DecValTok{12}\NormalTok{, }\DecValTok{15}\NormalTok{, }\DecValTok{17}\NormalTok{) }\CommentTok{\# Варианты вылова в тыс.тонн}

\ControlFlowTok{for}\NormalTok{(i }\ControlFlowTok{in} \FunctionTok{seq\_along}\NormalTok{(catchvals))\{}
\NormalTok{  fit }\OtherTok{\textless{}{-}} \FunctionTok{add.man.scenario}\NormalTok{(}
\NormalTok{    fit,}
    \AttributeTok{scenarioTitle =} \FunctionTok{paste0}\NormalTok{(}\StringTok{"Постоянный вылов "}\NormalTok{, catchvals[i], }\StringTok{" тыс.т"}\NormalTok{),}
    \AttributeTok{cabs =}\NormalTok{ catchvals[i]  }\CommentTok{\# Абсолютный вылов}
\NormalTok{  )}
\NormalTok{\}}

\DocumentationTok{\#\# 12.4 Сводка по сценариям управления}
\FunctionTok{sumspict.manage}\NormalTok{(fit, }\AttributeTok{include.unc =} \ConstantTok{TRUE}\NormalTok{) }\CommentTok{\# С учетом неопределенности}
\end{Highlighting}
\end{Shaded}

Получаем:

\begin{Shaded}
\begin{Highlighting}[]
\NormalTok{SPiCT timeline}\SpecialCharTok{:}
                                                  
\NormalTok{      Observations              Management        }
    \FloatTok{2005.00} \SpecialCharTok{{-}} \FloatTok{2025.00}        \FloatTok{2025.00} \SpecialCharTok{{-}} \FloatTok{2026.00}    
 \SpecialCharTok{|{-}{-}{-}{-}{-}{-}{-}{-}{-}{-}{-}{-}{-}{-}{-}{-}{-}{-}{-}{-}{-}{-}{-}}\ErrorTok{|} \SpecialCharTok{{-}{-}{-}{-}{-}{-}{-}{-}{-}{-}{-}{-}{-}{-}{-}{-}{-}{-}{-}{-}{-}{-}}\ErrorTok{|}

\NormalTok{Management evaluation}\SpecialCharTok{:} \FloatTok{2026.00}

\NormalTok{Predicted catch }\ControlFlowTok{for}\NormalTok{ management period and states at management evaluation time}\SpecialCharTok{:}

\NormalTok{                                 C B}\SpecialCharTok{/}\NormalTok{Bmsy F}\SpecialCharTok{/}\NormalTok{Fmsy}
\FloatTok{1.}\NormalTok{ Keep current catch         }\FloatTok{11.8}   \FloatTok{1.31}   \FloatTok{0.52}
\FloatTok{2.}\NormalTok{ Keep current F             }\FloatTok{12.5}   \FloatTok{1.30}   \FloatTok{0.55}
\FloatTok{3.}\NormalTok{ Fish at Fmsy               }\FloatTok{22.0}   \FloatTok{1.20}   \FloatTok{1.00}
\FloatTok{4.}\NormalTok{ No fishing                  }\FloatTok{0.0}   \FloatTok{1.42}   \FloatTok{0.00}
\FloatTok{5.}\NormalTok{ Reduce F by }\DecValTok{25}\NormalTok{\%             }\FloatTok{9.5}   \FloatTok{1.33}   \FloatTok{0.41}
\FloatTok{6.}\NormalTok{ Increase F by }\DecValTok{25}\NormalTok{\%          }\FloatTok{15.4}   \FloatTok{1.27}   \FloatTok{0.68}
\FloatTok{7.}\NormalTok{ MSY hockey}\SpecialCharTok{{-}}\NormalTok{stick rule      }\FloatTok{22.0}   \FloatTok{1.20}   \FloatTok{1.00}
\FloatTok{8.}\NormalTok{ ICES advice rule           }\FloatTok{19.9}   \FloatTok{1.23}   \FloatTok{0.90}
\FloatTok{9.}\NormalTok{ Постоянный вылов }\DecValTok{10}\NormalTok{ тыс.т  }\FloatTok{10.0}   \FloatTok{1.32}   \FloatTok{0.43}
\FloatTok{10.}\NormalTok{ Постоянный вылов }\DecValTok{12}\NormalTok{ тыс.т }\FloatTok{12.0}   \FloatTok{1.30}   \FloatTok{0.53}
\FloatTok{11.}\NormalTok{ Постоянный вылов }\DecValTok{15}\NormalTok{ тыс.т }\FloatTok{15.0}   \FloatTok{1.27}   \FloatTok{0.66}
\FloatTok{12.}\NormalTok{ Постоянный вылов }\DecValTok{17}\NormalTok{ тыс.т }\FloatTok{17.0}   \FloatTok{1.25}   \FloatTok{0.76}

\DecValTok{95}\NormalTok{\% confidence intervals }\ControlFlowTok{for}\NormalTok{ states}\SpecialCharTok{:}

\NormalTok{                              B}\SpecialCharTok{/}\NormalTok{Bmsy.lo B}\SpecialCharTok{/}\NormalTok{Bmsy.hi F}\SpecialCharTok{/}\NormalTok{Fmsy.lo F}\SpecialCharTok{/}\NormalTok{Fmsy.hi}
\FloatTok{1.}\NormalTok{ Keep current catch              }\FloatTok{1.12}      \FloatTok{1.52}      \FloatTok{0.23}      \FloatTok{1.15}
\FloatTok{2.}\NormalTok{ Keep current F                  }\FloatTok{1.12}      \FloatTok{1.51}      \FloatTok{0.25}      \FloatTok{1.22}
\FloatTok{3.}\NormalTok{ Fish at Fmsy                    }\FloatTok{1.00}      \FloatTok{1.44}      \FloatTok{0.45}      \FloatTok{2.23}
\FloatTok{4.}\NormalTok{ No fishing                      }\FloatTok{1.25}      \FloatTok{1.63}      \FloatTok{0.00}      \FloatTok{0.00}
\FloatTok{5.}\NormalTok{ Reduce F by }\DecValTok{25}\NormalTok{\%                 }\FloatTok{1.15}      \FloatTok{1.54}      \FloatTok{0.18}      \FloatTok{0.92}
\FloatTok{6.}\NormalTok{ Increase F by }\DecValTok{25}\NormalTok{\%               }\FloatTok{1.08}      \FloatTok{1.49}      \FloatTok{0.31}      \FloatTok{1.53}
\FloatTok{7.}\NormalTok{ MSY hockey}\SpecialCharTok{{-}}\NormalTok{stick rule           }\FloatTok{1.00}      \FloatTok{1.44}      \FloatTok{0.45}      \FloatTok{2.23}
\FloatTok{8.}\NormalTok{ ICES advice rule                }\FloatTok{1.03}      \FloatTok{1.46}      \FloatTok{0.40}      \FloatTok{2.00}
\FloatTok{9.}\NormalTok{ Постоянный вылов }\DecValTok{10}\NormalTok{ тыс.т       }\FloatTok{1.14}      \FloatTok{1.54}      \FloatTok{0.19}      \FloatTok{0.97}
\FloatTok{10.}\NormalTok{ Постоянный вылов }\DecValTok{12}\NormalTok{ тыс.т      }\FloatTok{1.12}      \FloatTok{1.52}      \FloatTok{0.24}      \FloatTok{1.17}
\FloatTok{11.}\NormalTok{ Постоянный вылов }\DecValTok{15}\NormalTok{ тыс.т      }\FloatTok{1.09}      \FloatTok{1.49}      \FloatTok{0.30}      \FloatTok{1.48}
\FloatTok{12.}\NormalTok{ Постоянный вылов }\DecValTok{17}\NormalTok{ тыс.т      }\FloatTok{1.06}      \FloatTok{1.48}      \FloatTok{0.34}      \FloatTok{1.69}
\end{Highlighting}
\end{Shaded}

\begin{figure}[H]

{\centering \includegraphics[width=0.8\linewidth,height=\textheight,keepaspectratio]{images/SPICT14.PNG}

}

\caption{Рис. 14.: Правило хоккейной клюшки и ICES}

\end{figure}%

\textbf{Процесс прогнозирования и оценки сценариев управления}

\textbf{1. Установка горизонта прогнозирования}

\begin{verbatim}
inp$maninterval <- c(2025, 2026)
\end{verbatim}

\begin{itemize}
\tightlist
\item
  \textbf{Цель:} Определить период, для которого делаются прогнозы
  (2025-2026 гг.)
\item
  \textbf{Механика:} Модель будет рассчитывать состояние запаса и
  возможный вылов на эти годы
\end{itemize}

\textbf{2. Базовые сценарии управления}

\begin{verbatim}
fit <- manage(fit)
\end{verbatim}

Автоматически генерируются стандартные сценарии:

\begin{enumerate}
\def\labelenumi{\arabic{enumi}.}
\tightlist
\item
  \textbf{\texttt{currentCatch}}: Сохранение текущего вылова (среднее за
  последние 3 года)
\item
  \textbf{\texttt{currentF}}: Сохранение текущего уровня смертности (F)
\item
  \textbf{\texttt{Fmsy}}: Эксплуатация на уровне F\textsubscript{MSY}
\item
  \textbf{\texttt{noF}}: Полное прекращение промысла
\item
  \textbf{\texttt{reduceF25}}: Снижение F на 25\%
\item
  \textbf{\texttt{increaseF25}}: Увеличение F на 25\%
\item
  \textbf{\texttt{msyHockeyStick}}: Правило ``хоккейной клюшки'' (F=0
  при B\textgreater B\textsubscript{MSY}, F=F\textsubscript{MSY} при
  B≥B\textsubscript{MSY})
\item
  \textbf{\texttt{ices}}: Правило ICES (F пропорционален уровню
  биомассы)
\end{enumerate}

\textbf{3. Пользовательские сценарии}

\begin{verbatim}
catchvals = c(10, 12, 15, 17) for(i in seq_along(catchvals)){   fit <- add.man.scenario(     fit,     scenarioTitle = paste0("Постоянный вылов ", catchvals[i], " тыс.т"),     cabs = catchvals[i]   ) }
\end{verbatim}

\begin{itemize}
\tightlist
\item
  \textbf{Стратегия:} Фиксированный вылов указанного объема в 2025-2026
  гг.
\item
  \textbf{Диапазон:} От консервативного (10 тыс.т) до рискованного (17
  тыс.т)
\end{itemize}

\textbf{4. Сводка результатов}

\begin{verbatim}
sumspict.manage(fit, include.unc = TRUE)
\end{verbatim}

\textbf{Ключевые выводы из результатов (на 2026 г.)}

\textbf{1. Прогноз состояния запаса}

\begin{longtable}[]{@{}
  >{\raggedright\arraybackslash}p{(\linewidth - 4\tabcolsep) * \real{0.2222}}
  >{\raggedright\arraybackslash}p{(\linewidth - 4\tabcolsep) * \real{0.1944}}
  >{\raggedright\arraybackslash}p{(\linewidth - 4\tabcolsep) * \real{0.5833}}@{}}
\toprule\noalign{}
\begin{minipage}[b]{\linewidth}\raggedright
\textbf{Показатель}
\end{minipage} & \begin{minipage}[b]{\linewidth}\raggedright
\textbf{Значение}
\end{minipage} & \begin{minipage}[b]{\linewidth}\raggedright
\textbf{Интерпретация}
\end{minipage} \\
\midrule\noalign{}
\endhead
\bottomrule\noalign{}
\endlastfoot
B/B\textsubscript{MSY} & 1.25-1.31 & Запас выше целевого уровня
(B\textgreater B\textsubscript{MSY}) \\
F/F\textsubscript{MSY} & 0.52-0.55 & Эксплуатация ниже предельной
(F\textless F\textsubscript{MSY}) \\
\end{longtable}

\textbf{2. Сравнение сценариев}

\begin{longtable}[]{@{}
  >{\raggedright\arraybackslash}p{(\linewidth - 8\tabcolsep) * \real{0.2000}}
  >{\raggedright\arraybackslash}p{(\linewidth - 8\tabcolsep) * \real{0.2000}}
  >{\raggedright\arraybackslash}p{(\linewidth - 8\tabcolsep) * \real{0.2000}}
  >{\raggedright\arraybackslash}p{(\linewidth - 8\tabcolsep) * \real{0.2000}}
  >{\raggedright\arraybackslash}p{(\linewidth - 8\tabcolsep) * \real{0.2000}}@{}}
\toprule\noalign{}
\begin{minipage}[b]{\linewidth}\raggedright
\textbf{Сценарий}
\end{minipage} & \begin{minipage}[b]{\linewidth}\raggedright
\textbf{Вылов (тыс.т)}
\end{minipage} & \begin{minipage}[b]{\linewidth}\raggedright
\textbf{B/}B\textsubscript{MSY}
\end{minipage} & \begin{minipage}[b]{\linewidth}\raggedright
\textbf{F/}F\textsubscript{MSY}
\end{minipage} & \begin{minipage}[b]{\linewidth}\raggedright
\textbf{Риск перелова}
\end{minipage} \\
\midrule\noalign{}
\endhead
\bottomrule\noalign{}
\endlastfoot
\textbf{Безопасные:} & & & & \\
\textbf{\texttt{noF}} (нет промысла) & 0.0 & 1.42 & 0.00 & Нет \\
\textbf{\texttt{reduceF25}} & 9.5 & 1.33 & 0.41 & Низкий \\
Вылов 10 тыс.т & 10.0 & 1.32 & 0.43 & Низкий \\
\textbf{Оптимальные:} & & & & \\
\textbf{\texttt{currentCatch}} & 11.8 & 1.31 & 0.52 & Низкий \\
Вылов 12 тыс.т & 12.0 & 1.30 & 0.53 & Низкий \\
\textbf{Рискованные:} & & & & \\
\textbf{\texttt{increaseF25}} & 15.4 & 1.27 & 0.68 & Умеренный \\
Вылов 15 тыс.т & 15.0 & 1.27 & 0.66 & Умеренный \\
\textbf{Опасные:} & & & & \\
F\textsubscript{MSY} & 22.0 & 1.20 & 1.00 & Высокий \\
Вылов 17 тыс.т & 17.0 & 1.25 & 0.76 & Высокий \\
\end{longtable}

\textbf{3. Анализ неопределенности}

Для ключевых сценариев:

\begin{itemize}
\item
  \textbf{Вылов 12 тыс.т:}\\
  F/F\textsubscript{MSY}\textgreater{} = 0.53 {[}0.24-1.17{]} → 10\%
  вероятность превышения F\textsubscript{MSY}
\item
  \textbf{Вылов 15 тыс.т:}\\
  F/F\textsubscript{MSY} = 0.66 {[}0.30-1.48{]} → 30\% вероятность
  превышения F\textsubscript{MSY}
\item
  \textbf{Вылов 17 тыс.т:}\\
  F/F\textsubscript{MSY} = 0.76 {[}0.34-1.69{]} → 45\% вероятность
  превышения F\textsubscript{MSY}
\end{itemize}

\textbf{Рекомендации по управлению}

\begin{enumerate}
\def\labelenumi{\arabic{enumi}.}
\tightlist
\item
  \textbf{Оптимальный вылов:} 12 тыс. тонн

  \begin{itemize}
  \tightlist
  \item
    Сохраняет запас в безопасной зоне (B/B\textsubscript{MSY}
    \textgreater{} 1.3)
  \item
    Минимизирует риск перелова (F/F\textsubscript{MSY} \textless{} 0.55)
  \item
    Учитывает неопределенность модельных оценок
  \end{itemize}
\item
  \textbf{Предельно допустимый вылов:} 15 тыс. тонн

  \begin{itemize}
  \tightlist
  \item
    Требует усиленного мониторинга
  \item
    Необходим ежегодный пересмотр квот
  \end{itemize}
\item
  \textbf{Не рекомендуются:}

  \begin{itemize}
  \tightlist
  \item
    Сценарии с F≥F\textsubscript{MSY} (22 тыс.т)
  \item
    Фиксированный вылов \textgreater15 тыс.т
  \item
    Стратегии, приводящие к снижению B/B\textsubscript{MSY} \textless{}
    1.25
  \end{itemize}
\end{enumerate}

\begin{quote}
\textbf{Критический фактор:} Высокая неопределенность прогноза вылова
(дов. интервал 7.3-21.4 тыс.т для текущего сценария) требует осторожного
подхода.
\end{quote}

\bookmarksetup{startatroot}

\chapter{Продукционная модель
JABBA}\label{ux43fux440ux43eux434ux443ux43aux446ux438ux43eux43dux43dux430ux44f-ux43cux43eux434ux435ux43bux44c-jabba}

\section{Введение}\label{ux432ux432ux435ux434ux435ux43dux438ux435-6}

Начнём с ловушки. Когда у нас в руках есть ``игрушечный руль'' ---
что‑нибудь вроде dB/dt = r·B·(1 − B/K) − C(t) --- страшно хочется
почувствовать рычаг управления: подберём \emph{r} и \emph{K}, оценим
текущую B, поставим правильный \emph{C\textsubscript{t+1}} и поведём
систему по траектории к \emph{MSY}. Иллюзия контроля рождается там, где
математика выглядит законченной, а данные --- нет. В промысловой
реальности индексы обилия шумны, выборка смещена, «уловистость» \emph{q}
дрейфует, а режимы среды меняются быстрее, чем сходятся наши
апостериорные распределения. Модель даёт ощущение руля, но часть океана
--- это шторм, который рулит нами. Даниэль Канеман сказал бы, что в
такие моменты «Система 1» радостно дорисовывает уверенность, а «Система
2» обязана включить тормоза и проверить допущения, чувствительность и
ретроспективную устойчивость. Биологические системы редко ведут себя «по
учебнику», и именно поэтому дисциплина проверки важнее изящества
формулы.

В этом занятии мы работаем с
\href{https://github.com/jabbamodel/JABBA}{JABBA} --- байесовской
стохастической продукционной моделью, близкой по философии к SPiCT. Оба
инструмента из одной семьи: они ставят биологическую динамику в основу,
честно разделяют ошибки наблюдения и процесса, и главное --- выводят всю
неопределённость в явный вид. Мы будем строить оценку из уловов и пары
индексов (CPUE и BESS), задавать слабые прайеры, запускать MCMC, и не
верить числам до тех пор, пока не проверим сходимость, остатки,
ретроспективу (Mohn's ρ) и предсказательную способность на «срезах»
(MASE). Однако не будем впадать в цинизм: рациональность,
систематический сбор данных и прозрачные модели --- это всё ещё лучший
путь к более разумным решениям; просто оптимизм должен быть «проверенным
на бордюре» ошибок и интервалов.

Зачем именно \href{https://github.com/jabbamodel/JABBA}{JABBA} в курсе.
Во‑первых, реплицируемость: код, входные таблицы, фиксированная
директория результатов --- вы сможете восстановить каждый шаг и
проверить каждое предположение. Во‑вторых, явная работа с
неопределённостью: априорные распределения параметров \emph{r},
\emph{K}, \emph{ψ} и \emph{q}, апостериоры параметров и траекторий,
доверительные интервалы для \emph{B/B\textsubscript{MSY}} и
\emph{F/F\textsubscript{MSY}}, вероятностные выводы, а не одинокие
точечные «оценки». В‑третьих, диагностика «по умолчанию»: от
Geweke/Heidel до фазовых графиков Кобэ, от остатков до process
deviations. В‑четвёртых, прогноз --- не как гадание, а как веер
сценариев с аккуратным учётом параметрической и процессной
вариабельности. Хорошая модель --- это карта, удобная для навигации, а
не фотография местности. Мы будем постоянно помнить, что наши карты
полезны ровно настолько, насколько мы отслеживаем их масштаб,
погрешность и зоны плохой видимости.

С методологической стороны всё просто и сложно одновременно. Просто ---
потому что базовая биология логистична: при малой биомассе прирост почти
пропорционален \emph{B}, близко к \emph{K} --- прирост затухает; вылов
вычитает «сверху». Сложно --- потому что \emph{r} и \emph{K} плохо
идентифицируются без информативных периодов (высоких \emph{B} или резких
спадов), \emph{q} плавает между флотами и годами, а процессная ошибка
\emph{σ²} смешивается с наблюдательной \emph{τ²}. Байесовский подход
помогает не только «усреднить» неопределённость, но и сделать её
объектом управления: мы можем принимать решения, которые устойчивы к
диапазону правдоподобных миров, а не к одному единственному. Мы не ищем
«замысел» в данных, мы подбираем механизм, который лучше других
воспроизводит наблюдаемую адаптивную динамику, и признаём эволюционную
торгуемость параметров --- их компромисс между точностью и
устойчивостью.

Управление запасами --- это всегда история, которую общество
рассказывает себе о будущем: «мы извлечём столько‑то, и запас останется
устойчивым». Математическая часть истории --- необходима, но
недостаточна; у неё есть герои (\emph{B/B\textsubscript{MSY}} и
\emph{F/F\textsubscript{MSY}}), есть антагонисты (перелов, неучтённый
вылов, сдвиги среды), и есть мораль: если вы не тестируете собственный
нарратив ретроспективой и внешней валидацией, он превращается в миф.
Поэтому в этом практикуме мы делаем упор на три вещи. Во‑первых, читаем
таблицы параметров сквозь призму сходимости: PPMR и PPVR для многомерной
стабильности, Geweke/Heidel для цепей. Во‑вторых, различаем «точность» и
«калибровку»: модель может хорошо ранжировать годы (низкий MASE), но
давать завышенную амплитуду (остатки и process deviations подскажут, где
именно). В‑третьих, смотрим на устойчивость решений: Mohn's ρ близок к
нулю --- хорошо; «веер» ретроспективы не расползается --- ещё лучше;
прогноз под реалистичными сценариями не пересекает опасные квадранты
Кобэ --- то, что нужно для рекомендаций.

Важное предупреждение про иллюзию контроля. JABBA и SPiCT не волшебные
палочки, а инструменты, которые позволяют количественно выразить вашу
неопределённость. Если прайеры на \emph{r} чрезмерно широки, индексы
плохо калиброваны по \emph{q}, а периоды высокой биомассы отсутствуют,
«красивые» графики всё равно останутся красивыми --- но управленческая
уверенность будет мнимой. Здесь полезна привычка думать об
``антикхрупкости'': формулируйте решения, которые переживут «плохие»
годы без катастрофы --- например, коридоры вылова, адаптивные пороги,
ежегодную перекалибровку модели с новыми данными, стресс‑тесты при
\emph{r} на нижней границе ДИ. И полезна доля оптимизма:
последовательное накопление данных (хороших, прозрачных,
воспроизводимых) сделает модель лучше --- и это не вера, а эмпирический
факт в дисциплинах, где стандарты учёта повышались.

Наконец, почему мы показываем на одном материале JABBA и проводим
параллели со SPiCT. Потому что полезно видеть одну и ту же задачу через
родственные, но не идентичные инструменты: отличия в реализации
процессной ошибки, спецификации наблюдательных дисперсий, в настроечных
«рычагах» MCMC и диагностике. Консенсус между инструментами --- это не
гарантия истины, но хорошая проверка на то, что вы не «подогрели»
результат особенностями одной конкретной реализации. Если же инструменты
расходятся --- это повод вернуться к данным: к шкалам, пропускам,
структурным разрывам, к возможной нестационарности \emph{q} и к тем
самым чёрным лебедям, которые любят объявляться в самый неудобный
момент.

С этим набором интеллектуальной «защиты от иллюзий» мы и идём дальше:
загрузим уловы и индексы, зададим прайеры, соберём модель, проверим, как
она дышит на диагностике, посмотрим ретроспективу и только после этого
--- аккуратно поговорим про прогнозы. Не потому, что модели плохи, а
потому что океан велик, а мы --- скромны и внимательны. Именно так
появляются решения, которые приносят пользу в реальном управлении, а не
только на красивых слайдах.

И так, библиотека JABBA \url{https://github.com/jabbamodel/JABBA} -
оценка запаса с помощью стохастической версии продукционной модели и
байесовского подхода. \textbf{JABBA} и \textbf{SPiCT} -- наиболее
распространенные в международной практике инструменты, реализующие
продукционный подход к оценке запасов гидробионтов при нехватки данных.

Помимо \href{https://cran.r-project.org/}{R} JABBA требует дополнительно
установки \href{https://sourceforge.net/projects/mcmc-jags/}{JAGS}. Для
демонстрации основных функций пакета в настоящем скрипте используются
входные данные из примера
\href{https://mombus.github.io/cRab/chapter\%205.html}{SPiCT}, поэтому в
скрипте демонстрации модели JABBA сценарий моделирования назван
``SPiCT\_adapted''.

Скрипт этого практического занятия можно скачать по
\href{https://mombus.github.io/cRab/data/JABBA.R}{ссылке}.

\section{Подготовка среды и загрузка
данных}\label{ux43fux43eux434ux433ux43eux442ux43eux432ux43aux430-ux441ux440ux435ux434ux44b-ux438-ux437ux430ux433ux440ux443ux437ux43aux430-ux434ux430ux43dux43dux44bux445}

В данном разделе выполняется базовая настройка среды R для работы с
пакетом JABBA. Инициируется загрузка двух необходимых пакетов:
\textbf{JABBA} (основной инструмент оценки запасов) и \textbf{reshape2}
(для преобразования структур данных). Создается целевая директория ``NEW
JABBA'' для автоматического сохранения всех результатов анализа, после
чего рабочая среда переключается на эту папку.

Формируются три обязательных компонента входных данных:

\begin{enumerate}
\def\labelenumi{\arabic{enumi}.}
\item
  \textbf{Данные по вылову} (catch): Годовые значения уловов за
  20-летний период (2005-2024 гг.), представленные в виде вектора из 20
  числовых значений.
\item
  \textbf{Два индекса обилия}:
\end{enumerate}

\begin{itemize}
\item
  \textbf{CPUE} (улов на единицу усилия): 20 наблюдений
\item
  \textbf{BESS} (альтернативный индекс, например, данные съемок): 19
  наблюдений (первое значение отсутствует, обозначено как NA)
\end{itemize}

\begin{enumerate}
\def\labelenumi{\arabic{enumi}.}
\setcounter{enumi}{2}
\tightlist
\item
  \textbf{Стандартные ошибки} (SE) для индексов: Для упрощения примера
  задаются фиксированным коэффициентом вариации (CV=20\%). Это означает,
  что для каждого ненулевого значения индекса SE рассчитывается как 20\%
  от его величины. Пропуски в индексах (NA) автоматически сохраняются
  как NA в таблице SE.
\end{enumerate}

Все данные структурируются в три таблицы с единой временной осью
(2005-2024 гг.): отдельно для уловов, значений индексов и их стандартных
ошибок. Эта подготовка обеспечивает корректный формат входных данных,
необходимых для последующего построения продукционной модели в JABBA.

\begin{Shaded}
\begin{Highlighting}[]
\CommentTok{\# {-}{-}{-}{-}{-}{-}{-}{-}{-}{-}{-}{-}{-}{-}{-}{-}{-}{-}{-}{-}{-}{-}{-}{-}{-} 1. ПОДГОТОВКА СРЕДЫ {-}{-}{-}{-}{-}{-}{-}{-}{-}{-}{-}{-}{-}{-}{-}{-}{-}{-}{-}{-}{-}{-}{-}{-}{-}{-}{-}}

\CommentTok{\# Загрузка необходимых пакетов}
\FunctionTok{library}\NormalTok{(JABBA) }\CommentTok{\# Основной пакет для оценки запасов}
\FunctionTok{library}\NormalTok{(reshape2) }\CommentTok{\# Для преобразования данных (функция dcast)}

\CommentTok{\# Установка рабочей директории (папки, где будут храниться результаты)}
\FunctionTok{setwd}\NormalTok{(}\StringTok{"C:/BAKANEV/JABBA"}\NormalTok{)}

\CommentTok{\# Создание папки для результатов анализа}
\NormalTok{assessment }\OtherTok{\textless{}{-}} \StringTok{"NEW JABBA"} \CommentTok{\# Название оценки}
\NormalTok{output.dir }\OtherTok{\textless{}{-}} \FunctionTok{file.path}\NormalTok{(}\FunctionTok{getwd}\NormalTok{(), assessment) }\CommentTok{\# Создание пути к папке}
\FunctionTok{dir.create}\NormalTok{(output.dir, }\AttributeTok{showWarnings =} \ConstantTok{FALSE}\NormalTok{) }\CommentTok{\# Создание папки (если не существует)}
\FunctionTok{setwd}\NormalTok{(output.dir) }\CommentTok{\# Переход в созданную папку}

\CommentTok{\# {-}{-}{-}{-}{-}{-}{-}{-}{-}{-}{-}{-}{-}{-}{-}{-}{-}{-}{-}{-}{-}{-}{-}{-}{-} 2. ЗАГРУЗКА И ПОДГОТОВКА ДАННЫХ {-}{-}{-}{-}{-}{-}{-}{-}{-}{-}{-}{-}{-}{-}{-}{-}{-}{-}{-}{-}{-}{-}{-}{-}{-}{-}{-}}

\CommentTok{\# Создание вектора лет анализа}
\NormalTok{Year }\OtherTok{\textless{}{-}} \DecValTok{2005}\SpecialCharTok{:}\DecValTok{2024} \CommentTok{\# Последовательность лет от 2005 до 2024}

\CommentTok{\# Вектор данных об уловах (catch)}
\NormalTok{Catch }\OtherTok{\textless{}{-}} \FunctionTok{c}\NormalTok{(}\DecValTok{5}\NormalTok{,}\DecValTok{7}\NormalTok{,}\DecValTok{6}\NormalTok{,}\DecValTok{10}\NormalTok{,}\DecValTok{14}\NormalTok{,}\DecValTok{25}\NormalTok{,}\DecValTok{28}\NormalTok{,}\DecValTok{30}\NormalTok{,}\DecValTok{32}\NormalTok{,}\DecValTok{35}\NormalTok{,}\DecValTok{25}\NormalTok{,}\DecValTok{20}\NormalTok{,}\DecValTok{15}\NormalTok{,}\DecValTok{12}\NormalTok{,}\DecValTok{10}\NormalTok{,}\DecValTok{12}\NormalTok{,}\DecValTok{10}\NormalTok{,}\DecValTok{13}\NormalTok{,}\DecValTok{11}\NormalTok{,}\DecValTok{12}\NormalTok{)}

\CommentTok{\# Вектор данных индекса обилия CPUE (catch per unit effort)}
\NormalTok{CPUE }\OtherTok{\textless{}{-}} \FunctionTok{c}\NormalTok{(}\FloatTok{27.4}\NormalTok{,}\FloatTok{26.8}\NormalTok{,}\FloatTok{16.8}\NormalTok{,}\FloatTok{23.0}\NormalTok{,}\FloatTok{29.0}\NormalTok{,}\FloatTok{30.0}\NormalTok{,}\FloatTok{16.5}\NormalTok{,}\FloatTok{17.2}\NormalTok{,}\FloatTok{10.5}\NormalTok{,}\FloatTok{14.6}\NormalTok{,}\FloatTok{8.3}\NormalTok{,}\FloatTok{11.4}\NormalTok{,}\FloatTok{15.5}\NormalTok{,}\FloatTok{13.8}\NormalTok{,}\FloatTok{11.5}\NormalTok{,}\FloatTok{15.3}\NormalTok{,}\FloatTok{12.2}\NormalTok{,}\FloatTok{15.6}\NormalTok{,}\FloatTok{16.2}\NormalTok{,}\FloatTok{13.4}\NormalTok{)}
\CommentTok{\# Вектор данных индекса обилия BESS}
\NormalTok{BESS }\OtherTok{\textless{}{-}} \FunctionTok{c}\NormalTok{(}\ConstantTok{NA}\NormalTok{,}\FloatTok{16.3}\NormalTok{,}\FloatTok{20.7}\NormalTok{,}\FloatTok{15.1}\NormalTok{,}\FloatTok{18.6}\NormalTok{,}\FloatTok{16.0}\NormalTok{,}\FloatTok{13.8}\NormalTok{,}\FloatTok{13.3}\NormalTok{,}\FloatTok{11.7}\NormalTok{,}\FloatTok{11.8}\NormalTok{,}\FloatTok{9.3}\NormalTok{,}\FloatTok{7.1}\NormalTok{,}\FloatTok{8.0}\NormalTok{,}\FloatTok{9.2}\NormalTok{,}\FloatTok{10.3}\NormalTok{,}\FloatTok{9.8}\NormalTok{,}\FloatTok{10.3}\NormalTok{,}\FloatTok{11.7}\NormalTok{,}\FloatTok{13.7}\NormalTok{,}\FloatTok{13.4}\NormalTok{)}

\CommentTok{\# Форматирование данных в таблицы для JABBA}
\NormalTok{catch\_data }\OtherTok{\textless{}{-}} \FunctionTok{data.frame}\NormalTok{(}\AttributeTok{year =}\NormalTok{ Year, }\AttributeTok{catch =}\NormalTok{ Catch) }\CommentTok{\# Таблица уловов}
\NormalTok{cpue\_data }\OtherTok{\textless{}{-}} \FunctionTok{data.frame}\NormalTok{(}\AttributeTok{year =}\NormalTok{ Year, }\AttributeTok{CPUE =}\NormalTok{ CPUE, }\AttributeTok{BESS =}\NormalTok{ BESS) }\CommentTok{\# Таблица индексов}

\CommentTok{\# Расчет стандартных ошибок (SE) для индексов}
\CommentTok{\# Используем коэффициент вариации (CV) = 20\% (0.2)}

\NormalTok{se\_data }\OtherTok{\textless{}{-}} \FunctionTok{data.frame}\NormalTok{(}
 \AttributeTok{year =}\NormalTok{ Year,}
 \AttributeTok{CPUE =} \FunctionTok{ifelse}\NormalTok{(}\FunctionTok{is.na}\NormalTok{(CPUE), }\ConstantTok{NA}\NormalTok{, }\FloatTok{0.2}\NormalTok{),}
 \AttributeTok{BESS =} \FunctionTok{ifelse}\NormalTok{(}\FunctionTok{is.na}\NormalTok{(BESS), }\ConstantTok{NA}\NormalTok{, }\FloatTok{0.2}\NormalTok{))}
\end{Highlighting}
\end{Shaded}

\section{Настройка и запуск модели
JABBA}\label{ux43dux430ux441ux442ux440ux43eux439ux43aux430-ux438-ux437ux430ux43fux443ux441ux43a-ux43cux43eux434ux435ux43bux438-jabba}

В этом разделе выполняется конфигурация и запуск JABBA. Сначала с
помощью функции \texttt{build\_jabba} формируется структура входных
данных для модели, куда передаются подготовленные таблицы уловов,
индексов CPUE/BESS и их стандартных ошибок. Указывается название оценки
(``NEW JABBA'') и сценарий моделирования (``SPiCT\_adapted''). В
качестве биологической основы выбрана модель Шефера (логистический
рост). Настраиваются ключевые априорные распределения: для темпа роста
популяции (r) задано нормальное распределение N(0.2±0.5), для емкости
среды (\emph{K}) - логнормальное LN(189.6±0.795), для начальной
заполненности запаса (\emph{ψ}) - бета-распределение Beta(0.75±0.25).
Установлена оценка процессной ошибки (sigma.est=TRUE) и
слабоинформативные априоры для дисперсии наблюдений.

Непосредственный запуск Байесовской оценки выполняется функцией
\texttt{fit\_jabba}, которая использует MCMC-алгоритм. Конфигурация MCMC
включает 50,000 итераций с отбрасыванием первых 10,000 (фаза
``burn-in''), прореживанием цепей в 5 раз и запуском 2 независимых цепей
для проверки сходимости. Результатом работы является объект
\texttt{fit}, содержащий апостериорные распределения параметров, оценки
биомассы и диагностику модели, которые будут использоваться для
последующего анализа состояния запаса.

\begin{Shaded}
\begin{Highlighting}[]
\CommentTok{\# {-}{-}{-}{-}{-}{-}{-}{-}{-}{-}{-}{-}{-}{-}{-}{-}{-}{-}{-} 3. НАСТРОЙКА И ЗАПУСК МОДЕЛИ JABBA {-}{-}{-}{-}{-}{-}{-}{-}{-}{-}{-}{-}{-}{-}{-}{-}{-}{-}{-}{-}}


\CommentTok{\# Создание входных данных для модели}

\NormalTok{jbinput }\OtherTok{\textless{}{-}} \FunctionTok{build\_jabba}\NormalTok{(}
\AttributeTok{catch =}\NormalTok{ catch\_data,}\CommentTok{\# Данные об уловах}
\AttributeTok{cpue =}\NormalTok{ cpue\_data,}\CommentTok{\# Данные индексов обилия}
\AttributeTok{se =}\NormalTok{ se\_data,}\CommentTok{\# Стандартные ошибки}
\AttributeTok{assessment =}\NormalTok{ assessment, }\CommentTok{\# Название оценки}
\AttributeTok{scenario =} \StringTok{"SPiCT\_adapted"}\NormalTok{, }\CommentTok{\# Сценарий модели}
\AttributeTok{model.type =} \StringTok{"Schaefer"}\NormalTok{, }\CommentTok{\# Тип модели (Шефера)}
\AttributeTok{sigma.est =} \ConstantTok{TRUE}\NormalTok{, }\CommentTok{\# Оценивать изменчивость процесса?}
\AttributeTok{r.prior =} \FunctionTok{c}\NormalTok{(}\FloatTok{0.2}\NormalTok{, }\FloatTok{0.5}\NormalTok{),}\CommentTok{\# Априорное распределение для r (среднее, SD)}
\AttributeTok{K.prior =} \FunctionTok{c}\NormalTok{(}\FloatTok{189.6}\NormalTok{, }\FloatTok{0.795}\NormalTok{),}\CommentTok{\# Априорное для K (среднее, SD)}
\AttributeTok{psi.prior =} \FunctionTok{c}\NormalTok{(}\FloatTok{0.75}\NormalTok{, }\FloatTok{0.25}\NormalTok{),}\CommentTok{\# Априорное для начального заполнения}
\AttributeTok{igamma =} \FunctionTok{c}\NormalTok{(}\FloatTok{0.001}\NormalTok{, }\FloatTok{0.001}\NormalTok{), }\CommentTok{\# Параметры для дисперсии наблюдений}
\AttributeTok{verbose =} \ConstantTok{FALSE} \CommentTok{\# Отключить подробный вывод}
\NormalTok{)}

\CommentTok{\# Запуск Байесовской модели (MCMC)}

\NormalTok{fit }\OtherTok{\textless{}{-}} \FunctionTok{fit\_jabba}\NormalTok{(}
\NormalTok{jbinput,}\CommentTok{\# Входные данные}
\AttributeTok{ni =} \DecValTok{50000}\NormalTok{, }\CommentTok{\# Общее количество итераций}
\AttributeTok{nb =} \DecValTok{10000}\NormalTok{, }\CommentTok{\# Количество "выжигаемых" итераций (burn{-}in)}
\AttributeTok{nt =} \DecValTok{5}\NormalTok{,}\CommentTok{\# Частота прореживания (thinning)}
\AttributeTok{nc =} \DecValTok{2} \CommentTok{\# Количество цепей MCMC}
\NormalTok{)}
\end{Highlighting}
\end{Shaded}

\section{Анализ параметров модели JABBA
(fit\$pars)}\label{ux430ux43dux430ux43bux438ux437-ux43fux430ux440ux430ux43cux435ux442ux440ux43eux432-ux43cux43eux434ux435ux43bux438-jabba-fitpars}

Команда \texttt{fit\$pars} выведет таблицу, содержит медианные значения,
95\% доверительные интервалы и результаты тестов сходимости MCMC для
ключевых характеристик модели.

\begin{Shaded}
\begin{Highlighting}[]
\SpecialCharTok{\textgreater{}}\NormalTok{ fit}\SpecialCharTok{$}\NormalTok{pars}

\NormalTok{        Median        LCI           UCI          Geweke.p Heidel.p}
\NormalTok{K      }\FloatTok{257.137636616} \FloatTok{178.4613660245} \FloatTok{442.25933810} \FloatTok{0.533}    \FloatTok{0.401}
\NormalTok{r      }\FloatTok{0.268924223}   \FloatTok{0.1517493450}   \FloatTok{0.42756749}   \FloatTok{0.324}    \FloatTok{0.864}
\NormalTok{q}\FloatTok{.1}    \FloatTok{0.102043488}   \FloatTok{0.0565010978}   \FloatTok{0.15490243}   \FloatTok{0.920}    \FloatTok{0.967}
\NormalTok{q}\FloatTok{.2}    \FloatTok{0.078076035}   \FloatTok{0.0433099875}   \FloatTok{0.11858679}   \FloatTok{0.986}    \FloatTok{0.954}
\NormalTok{psi    }\FloatTok{0.875023951}   \FloatTok{0.5909908905}   \FloatTok{1.23643434}   \FloatTok{0.970}    \FloatTok{0.470}
\NormalTok{sigma2 }\FloatTok{0.003477230}   \FloatTok{0.0005645686}   \FloatTok{0.02110338}   \FloatTok{0.369}    \FloatTok{0.563}
\NormalTok{tau2}\FloatTok{.1} \FloatTok{0.006817871}   \FloatTok{0.0006978604}   \FloatTok{0.05031429}   \FloatTok{0.845}    \FloatTok{0.392}
\NormalTok{tau2}\FloatTok{.2} \FloatTok{0.002775946}   \FloatTok{0.0004835058}   \FloatTok{0.01978467}   \FloatTok{0.522}    \FloatTok{0.445}
\NormalTok{m      }\FloatTok{2.000000000}   \FloatTok{2.0000000000}   \FloatTok{2.00000000}   \ConstantTok{NaN}      \ConstantTok{NA}
\SpecialCharTok{\textgreater{}}
\end{Highlighting}
\end{Shaded}

\textbf{Биологические параметры}:

\begin{itemize}
\tightlist
\item
  \emph{K}(емкость среды): Медиана 257.14 единиц биомассы, например
  тыс.тонн (95\% ДИ: 178.46--442.26). Широкий доверительный интервал
  отражает типичную неопределенность при отсутствии данных о периоде,
  когда запас приближался к нетронутому состоянию.
\item
  \emph{r}(темп роста): Медиана 0.269 год⁻¹ (95\% ДИ: 0.152--0.428)
  соответствует биологическим ожиданиям для многих промысловых видов
  рыб.
\item
  \emph{ψ}(начальная биомасса): Медиана 0.875 (95\% ДИ: 0.591--1.236)
  указывает, что в 2005 году биомасса составляла \textasciitilde87.5\%
  от K.
\end{itemize}

\textbf{Параметры наблюдений}:

\begin{itemize}
\item
  \emph{q.1}(коэффициент уловистости (улавливаемости) CPUE): Медиана
  0.102 (95\% ДИ: 0.056--0.155)
\item
  \emph{q.2}(коэффициент уловистости (улавливаемости) BESS): Медиана
  0.078 (95\% ДИ: 0.043--0.119) Широкие доверительные интервалы
  (\textgreater50\% от медианы) подтверждают недостаточную
  информативность данных для точной оценки этих параметров.
\end{itemize}

\textbf{Оценки ошибок}:

\begin{longtable}[]{@{}
  >{\raggedright\arraybackslash}p{(\linewidth - 16\tabcolsep) * \real{0.1111}}
  >{\raggedright\arraybackslash}p{(\linewidth - 16\tabcolsep) * \real{0.1111}}
  >{\raggedright\arraybackslash}p{(\linewidth - 16\tabcolsep) * \real{0.1111}}
  >{\raggedright\arraybackslash}p{(\linewidth - 16\tabcolsep) * \real{0.1111}}
  >{\raggedright\arraybackslash}p{(\linewidth - 16\tabcolsep) * \real{0.1111}}
  >{\raggedright\arraybackslash}p{(\linewidth - 16\tabcolsep) * \real{0.1111}}
  >{\raggedright\arraybackslash}p{(\linewidth - 16\tabcolsep) * \real{0.1111}}
  >{\raggedright\arraybackslash}p{(\linewidth - 16\tabcolsep) * \real{0.1111}}
  >{\raggedright\arraybackslash}p{(\linewidth - 16\tabcolsep) * \real{0.1111}}@{}}
\toprule\noalign{}
\endhead
\bottomrule\noalign{}
\endlastfoot
& & & & & & & & \\
& \textbf{Параметр} & & \textbf{Медиана} & & \textbf{95\% ДИ} & &
\textbf{Назначение} & \\
& & & & & & & & \\
& sigma2 & & 0.0035 & & {[}0.0006, 0.021{]} & & Дисперсия ошибки
процесса & \\
& & & & & & & & \\
& tau2.1 & & 0.0068 & & {[}0.0007, 0.050{]} & & Дисперсия ошибки CPUE
& \\
& & & & & & & & \\
& tau2.2 & & 0.0028 & & {[}0.0005, 0.020{]} & & Дисперсия ошибки BESS
& \\
& & & & & & & & \\
& Соотношение σ²/τ² показывает: & & Для CPUE: 0.0035/0.0068 ≈ 0.51 & &
Для BESS: 0.0035/0.0028 ≈ 1.25 & & & \\
\end{longtable}

Это свидетельствует, что неопределенность в большей степени обусловлена
погрешностью данных, чем стохастичностью биологической динамики.

\textbf{Диагностика сходимости MCMC}: Все p-значения тестов (Geweke,
Heidel) превышают 0.05, что подтверждает сходимость цепей:

\textbf{Управленческие выводы}:

1.Из-за неопределенности в оценках K и q интерпретацию следует
фокусировать на относительных показателях
(\emph{B/B\textsubscript{msy}}, \emph{F/F\textsubscript{msy}}).

2.Для повышения точности модели требуется:

\begin{itemize}
\item
  Дополнительные данные за периоды высокой биомассы (для уточнения
  \emph{K})
\item
  Калибровка индексов обилия (для снижения \emph{τ²})
\end{itemize}

3.Высокие значения \emph{τ²} указывают на необходимость улучшения
качества данных CPUE и BESS.

\textbf{Диагностика и визуализация результатов}

После выполнения байесовской оценки в JABBA проводится комплексная
диагностика и визуализация результатов. Автоматически генерируется набор
стандартных графиков через функцию jabba\_plots, которые сохраняются в
рабочую директорию. Для углубленного анализа последовательно строятся
специализированные графики: динамика уловов с модельными значениями,
согласованность наблюдаемых и предсказанных индексов CPUE/BESS,
сравнение априорных и апостериорных распределений ключевых параметров
(\emph{r}, \emph{K}, \emph{q}), а также диагностика остатков модели.
Особое внимание уделяется сходимости MCMC-цепей для исключения
вычислительных ошибок.

Важнейшая часть анализа - визуализация временных трендов: абсолютной
биомассы (\emph{B}), промысловой смертности (\emph{F}) и их соотношений
с целевыми уровнями (\emph{B/B\textsubscript{msy}},
\emph{F/F\textsubscript{msy}}). Фазовый портрет и Кобэ-график
интегрируют эти показатели, наглядно отображая историческую и текущую
позицию запаса относительно ориентиров управления (перелов/недолов).
Дополнительно выполняются тест на случайность остатков, логарифмические
аппроксимации и анализ отклонений процесса, что обеспечивает
всестороннюю проверку адекватности модели. Финал этапа - экспорт
рассчитанных временных рядов (биомасса, \emph{F},
\emph{B/B\textsubscript{msy}} и др.) в CSV-файл для дальнейшего
использования.

\begin{Shaded}
\begin{Highlighting}[]
\CommentTok{\# {-}{-}{-}{-}{-}{-}{-}{-}{-}{-}{-}{-}{-}{-}{-}{-}{-}{-}{-} 4. ДИАГНОСТИКА И ВИЗУАЛИЗАЦИЯ РЕЗУЛЬТАТОВ {-}{-}{-}{-}{-}{-}{-}{-}{-}{-}{-}{-}{-}{-}{-}{-}{-}{-}{-}{-}}

\CommentTok{\# Генерация стандартных диагностических графиков}

\FunctionTok{jbplot\_ensemble}\NormalTok{(fit)}
\FunctionTok{jabba\_plots}\NormalTok{(fit, }\AttributeTok{output.dir =}\NormalTok{ output.dir)}

\CommentTok{\# Индивидуальные графики для детального анализа:}

\FunctionTok{jbplot\_catch}\NormalTok{(fit)}\CommentTok{\# График уловов}
\FunctionTok{jbplot\_cpuefits}\NormalTok{(fit)}\CommentTok{\# Сравнение модельных и наблюдаемых индексов}
\FunctionTok{jbplot\_ppdist}\NormalTok{(fit)}\CommentTok{\# Распределения априорных и апостериорных параметров}
\FunctionTok{jbplot\_residuals}\NormalTok{(fit)}\CommentTok{\# Остатки модели}
\FunctionTok{jbplot\_mcmc}\NormalTok{(fit)}\CommentTok{\# Диагностика сходимости MCMC}
\FunctionTok{jbplot\_trj}\NormalTok{(fit, }\AttributeTok{type =} \StringTok{"B"}\NormalTok{)}\CommentTok{\# Динамика биомассы}
\FunctionTok{jbplot\_trj}\NormalTok{(fit, }\AttributeTok{type =} \StringTok{"F"}\NormalTok{)}\CommentTok{\# Динамика промысловой смертности}
\FunctionTok{jbplot\_trj}\NormalTok{(fit, }\AttributeTok{type =} \StringTok{"BBmsy"}\NormalTok{)}\CommentTok{\# Отношение *B/B\textasciitilde{}msy\textasciitilde{}*}
\FunctionTok{jbplot\_trj}\NormalTok{(fit, }\AttributeTok{type =} \StringTok{"FFmsy"}\NormalTok{)}\CommentTok{\# Отношение *F/F\textasciitilde{}msy\textasciitilde{}*}
\FunctionTok{jbplot\_spphase}\NormalTok{(fit)}\CommentTok{\# **График продуктивности запаса с разметкой фаз Kobe**}
\FunctionTok{jbplot\_kobe}\NormalTok{(fit)}\CommentTok{\# Кобэ{-}график (*B/B\textasciitilde{}msy\textasciitilde{}* vs *F/F\textasciitilde{}msy\textasciitilde{}*)}

\CommentTok{\# Дополнительные диагностики:}

\FunctionTok{jbplot\_runstest}\NormalTok{(fit)}\CommentTok{\# Тест на случайность остатков}
\FunctionTok{jbplot\_logfits}\NormalTok{(fit)}\CommentTok{\# Графики в логарифмической шкале}

\FunctionTok{jbplot\_procdev}\NormalTok{(fit)}\CommentTok{\# Отклонения процесса}

\CommentTok{\# Сохранение временных рядов результатов}

\FunctionTok{write.csv}\NormalTok{(fit}\SpecialCharTok{$}\NormalTok{timeseries, }\AttributeTok{file =} \StringTok{"results.csv"}\NormalTok{, }\AttributeTok{row.names =} \ConstantTok{FALSE}\NormalTok{)}
\end{Highlighting}
\end{Shaded}

\begin{figure}[H]

{\centering \includegraphics[width=0.8\linewidth,height=\textheight,keepaspectratio]{images/JABBA1.PNG}

}

\caption{Рис. 1.: Генерация стандартных графиков: относительной и
абсолютной промысловой биомассы, а также смертности, ошибки процесса и
динамики вылова.}

\end{figure}%

\begin{figure}[H]

{\centering \includegraphics[width=0.6\linewidth,height=\textheight,keepaspectratio]{images/JABBA2.png}

}

\caption{Рис. 2.: Сравнение модельных и наблюдаемых индексов}

\end{figure}%

График сравнения позволяет визуально оценить \textbf{адекватность модели
реальным данным} и выявить систематические расхождения. Он отображает:
1.\textbf{Наблюдаемые значения}индексов (например, CPUE и BESS) в виде
точек с вертикальными отрезками, отражающими доверительные интервалы на
основе стандартных ошибок. 2.\textbf{Модельные предсказания} в виде
сплошной линии с затененной областью (50 и 95\% доверительные интервалы
апостериорного распределения).

\textbf{Ключевые аспекты интерпретации}:

\textbf{Согласованность}: Если модельная кривая проходит в пределах
доверительных интервалов наблюдаемых точек, это свидетельствует о
хорошем описании трендов. \textbf{Смещения}: Систематическое
занижение/завышение предсказаний для определенных периодов указывает на
недостатки модели (например, недоучет факторов среды или нелегального
вылова). \textbf{Чувствительность индексов}: Различия в точности
аппроксимации CPUE и BESS помогают оценить, какой индекс информативнее
отражает динамику биомассы. \textbf{Аномалии}: Резкие выбросы точек за
пределы доверительной зоны модели сигнализируют о годах с нетипичными
условиями (ошибки данных, природные катаклизмы). \textbf{Практическое
значение}: График отвечает на вопрос --- способна ли выбранная
продукционная модель (в данном случае Шефера) достоверно воспроизводить
историческую динамику запаса, что является основой для корректных
прогнозов.

\begin{figure}[H]

{\centering \includegraphics[width=0.6\linewidth,height=\textheight,keepaspectratio]{images/JABBA3.png}

}

\caption{Рис. 3.: Распределения априорных и апостериорных параметров}

\end{figure}%

Этот график предоставляет \textbf{инструмент для анализа влияния данных
на исходные предположения модели}. Он визуализирует:

1.\textbf{Априорные распределения}(темные области): - Заданные до
анализа (например:r \textasciitilde{} N(0.2, 0.5),K \textasciitilde{}
LN(189.6, 0.795)). - Отражают экспертные гипотезы или литературные
данные о параметрах.

2.\textbf{Апостериорные распределения}(светлые области): - Рассчитанные
в ходе Байесовского вывода (MCMC) после учета данных (уловы, индексы). -
Показывают,\emph{как фактическая информация модифицировала
первоначальные предположения}.

\textbf{Аспекты интерпретации}:

\textbf{Сдвиг пиков}: Если апостериор смещен относительно априора
(напр., пик для \emph{r} сдвинулся от 0.2 к 0.3) -- данные
``перевесили'' априорную гипотезу. \textbf{Сужение кривой}: Резкое
сокращение дисперсии апостериора (напр., для \emph{K}) свидетельствует о
высокой информативности данных по этому параметру. \textbf{Конфликт}:
Если апостериорный пик находится в ``хвосте'' априора (напр., априор для
\emph{q} задан N(1,0.2), а апостериор с пиком при 2.5) -- сигнал о
несоответствии данных или модели.

\textbf{Параметры-индикаторы}:

\begin{itemize}
\tightlist
\item
  \emph{r} (темп роста): Четкий апостериор указывает на надежную оценку
  продуктивности запаса.
\item
  \emph{K} (емкость среды): Узкий апостериор -- уверенность в оценке
  исторической биомассы.
\item
  \emph{ψ} (начальная заполненность): Расхождение с априором может
  указывать на ошибку в задании начальных условий.
\end{itemize}

\textbf{Диагностическое значение}:

Если апостериоры близки к априорам -- данные не внесли новую информацию
(требует пересмотра индексов). Если апостериоры
асимметричны/многомодальны -- возможны проблемы идентификации параметров
(необходимы дополнительные диагностики).

\textbf{Вывод}: График отвечает на вопрос -- \emph{насколько исходные
биологические гипотезы подтвердились реальными данными}, что критично
для обоснованности оценки.

Касательно графиков коэффициентов пропорциональности или улавливаемости
\emph{q}. Здесь, для простоты примера, вручную не задаются. JABBA по
умолчанию (автоматически) назначает слабый или малоинформативный априор:
\texttt{q\ \textasciitilde{}\ LogNormal(mean\ =\ 1,\ SD\ =\ 1000)} Это
практически \textbf{равномерное распределение} в широком диапазоне (от
\textasciitilde0 до +∞). Поэтому на графиках не видны темные области
априорных распределений \emph{q1} и \emph{q2}. \emph{Физический смысл}:
Мы не знаем, во сколько раз индекс отличается от абсолютной биомассы.

\textbf{PPMR и PPVR: диагностические показатели в JABBA}

В байесовском анализе, который лежит в основе работы JABBA, ключевым
этапом является проверка сходимости MCMC-цепей. Именно для этой цели
используются диагностические показатели PPMR (Potential Scale Reduction
Factor for Multivariate Monitoring) и PPVR (Potential Scale Reduction
Factor for Predictive Variance). Эти статистики позволяют оценить,
насколько надежны полученные оценки параметров модели. PPMR представляет
собой многомерный аналог классического R-hat критерия и оценивает
сходимость сразу для всех параметров модели, учитывая их взаимосвязи. Он
рассчитывается как корень из отношения дисперсии между цепями к
дисперсии внутри цепей. Идеальное значение PPMR должно быть близко к 1,
а значения выше 1.1 указывают на серьезные проблемы со сходимостью. PPVR
же фокусируется конкретно на сходимости дисперсии апостериорных
предсказаний, что особенно важно для оценки надежности доверительных
интервалов прогнозируемых величин, таких как
\emph{B/B\textsubscript{msy}} или \emph{F/F\textsubscript{msy}}.
Значения PPVR также должны стремиться к 1.

В вашем конкретном случае анализ этих показателей дает неоднозначную
картину. Для параметра K (несущая способность) мы видим практически
идеальную сходимость: PPMR равен 0.97, что даже немного меньше 1, что
указывает на превосходную стабильность оценок между цепями, а крайне
низкое значение PPVR (0.109) говорит о высокой согласованности в оценке
неопределенности. Это позволяет с уверенностью доверять полученным
значениям \emph{K}. Однако ситуация с параметром \emph{r} (темп роста
популяции) вызывает серьезные опасения. Значение PPMR 1.471 существенно
превышает критический порог, что свидетельствует о явных проблемах со
сходимостью цепей. Хотя PPVR 0.322 выглядит лучше, это не компенсирует
высокий PPMR. Такая ситуация часто возникает при слишком широких
априорных распределениях или недостатке информативных данных, особенно в
периоды роста популяции. Для параметра \emph{ψ} (начальная заполненность
запаса) ситуация промежуточная: PPMR 1.174 находится на границе
допустимого, а PPVR 0.501 указывает на умеренную согласованность в
оценке неопределенности.

Для улучшения сходимости рекомендуется предпринять несколько шагов.
Во-первых, стоит значительно увеличить количество итераций MCMC -
например, до 100000 для ni и 20000 для фазы ``burn-in'' (nb). Во-вторых,
необходимо пересмотреть априорные распределения, особенно для параметра
r: возможно, стоит уменьшить стандартное отклонение с 0.5 до 0.2-0.3,
если есть экспертные основания для такого ужесточения. В-третьих, стоит
проверить качество входных данных - достаточно ли репрезентативны
имеющиеся индексы обилия, особенно в ключевые периоды динамики
популяции. Важно понимать, что при сохраняющихся проблемах со
сходимостью абсолютные оценки биомассы и темпа роста могут оставаться
ненадежными, однако относительные показатели, такие как
\emph{B/B\textsubscript{msy}}, часто оказываются более устойчивыми к
подобным проблемам. Это связано с тем, что они в меньшей степени зависят
от абсолютных значений проблемных параметров. Таким образом, несмотря на
выявленные сложности, модель может оставаться полезной для принятия
управленческих решений, особенно если фокусироваться на относительных
показателях состояния запаса.

\begin{figure}[H]

{\centering \includegraphics[width=0.6\linewidth,height=\textheight,keepaspectratio]{images/JABBA4.png}

}

\caption{Рис. 4.: Остатки модели}

\end{figure}%

График остатков в JABBA (jbplot\_residuals) представляет собой важный
диагностический инструмент, позволяющий оценить качество соответствия
модели реальным данным. На этом графике отображаются остатки - разницы
между наблюдаемыми значениями индексов обилия (CPUE и BESS) и их
модельными оценками. В вашем случае график строится на основе конкретных
значений остатков, где для CPUE они варьируются от -0.38 до +0.31, а для
BESS - от -0.20 до +0.17, с отсутствующим значением (NA) для BESS в 2005
году.

График имеет несколько ключевых элементов. Во-первых, это боксплоты,
которые визуализируют распределение остатков для каждого индекса в
целом, показывая медиану, квартили и возможные выбросы. Во-вторых, на
график накладывается сглаженная линия (loess), которая помогает выявить
систематические тренды в остатках. Например, если остатки демонстрируют
явную тенденцию к увеличению или уменьшению со временем, это может
указывать на неучтенные факторы в модели. В ваших данных остатки CPUE
показывают некоторую изменчивость, но без явного тренда, в то время как
BESS имеет более стабильные остатки с меньшим разбросом.

Интерпретация остатков имеет решающее значение. В идеале остатки должны
быть случайно распределены вокруг нуля без видимых закономерностей.
Наличие кластеров положительных или отрицательных остатков в
определенные периоды может указывать на систематические ошибки модели. В
нашем случае отсутствие явных трендов на графике - хороший знак, хотя
отдельные выбросы, такие как отрицательный остаток CPUE в 2013 году
(-0.38), заслуживают внимания. Размеры остатков также важны: значения,
превышающие по модулю 0.5, считаются значительными и могут указывать на
проблемы с данными или спецификацией модели.

График остатков особенно полезен для сравнения разных индексов. В нашем
анализе видно, что остатки CPUE имеют больший разброс по сравнению с
BESS, что может говорить либо о более высокой вариабельности данных
CPUE, либо о том, что модель хуже описывает эту компоненту. Отсутствие
данных BESS для 2005 года (NA) корректно обрабатывается графиком. Важно
отметить, что систематические смещения остатков вверх или вниз могут
указывать на проблемы с калибровкой коэффициентов уловистости (q), что
согласуется с ранее выявленными сложностями в их оценке.

\begin{figure}[H]

{\centering \includegraphics[width=0.6\linewidth,height=\textheight,keepaspectratio]{images/JABBA5.png}

}

\caption{Рис. 5.: График продуктивности запаса с разметкой фаз Кобэ с
динамикой вылова}

\end{figure}%

График представляет собой комплексную визуализацию, объединяющую три
ключевых аспекта анализа продукционной модели: 1) продукционную функцию
(дуга), 2) динамику вылова относительно биомассы и 3) фазовые переходы
состояния запаса в координатах Кобэ-графика

На графике по оси X отображается \textbf{биомасса} (в абсолютных
единицах или относительно \emph{B\textsubscript{msy}}), а по оси Y ---
\textbf{вылов} и \textbf{чистая продукция} (surplus production). Кривая
производственной функции (обычно параболическая для модели Шефера)
показывает зависимость между биомассой и устойчивым выловом. Точки на
графике представляют фактические годовые значения биомассы и вылова,
соединенные линиями в хронологическом порядке. Цвет точек кодирует
фазовое состояние запаса согласно классификации Kobe: зеленый ---
устойчивое состояние (\emph{B/B\textsubscript{msy}},
\emph{F/F\textsubscript{msy}}), желтый/оранжевый --- перелов
(\emph{B/B\textsubscript{msy}}, \emph{F/F\textsubscript{msy}} или
\emph{B/B\textsubscript{msy}}, \emph{F/F\textsubscript{msy}}), красный
--- коллапс (\emph{B/B\textsubscript{msy}},
\emph{F/F\textsubscript{msy}}).

\begin{figure}[H]

{\centering \includegraphics[width=0.6\linewidth,height=\textheight,keepaspectratio]{images/JABBA6.png}

}

\caption{Рис. 6.: Кобэ-график (\emph{B/B\textsubscript{msy}} vs
\emph{F/F\textsubscript{msy}}) с динамикой вылова}

\end{figure}%

График Кобэ, генерируемый функцией \texttt{jbplot\_kobe(fit)},
представляет собой фазовый портрет, где по оси X отложено отношение
биомассы к целевому уровню (\emph{B/B\textsubscript{msy}}), а по оси Y
--- отношение промысловой смертности к устойчивому уровню
(\emph{F/F\textsubscript{msy}}). Этот график разделен на четыре
квадранта, определяющих статус запаса. Зеленый квадрант
(\emph{B/B\textsubscript{msy}} \textgreater{} 1,
\emph{F/F\textsubscript{msy}} \textless{} 1) соответствует устойчивому
состоянию без перелова. Желтый квадрант (\emph{B/B\textsubscript{msy}}
\textgreater{} 1, \emph{F/F\textsubscript{msy}} \textgreater{} 1)
сигнализирует о риске перелова. Оранжевый квадрант
(\emph{B/B\textsubscript{msy}} \textless{} 1,
\emph{F/F\textsubscript{msy}} \textgreater{} 1) указывает на активный
перелов, а желтый (\emph{B/B\textsubscript{msy}} \textless{} 1,
\emph{F/F\textsubscript{msy}} \textless{} 1) --- на фазу восстановления
запаса.

В нашем анализе целевые ориентиры рассчитаны на основе параметров
модели: \emph{B\textsubscript{msy}} = \emph{K}/2 = 257.14/2 ≈ 128.57, а
\emph{MSY} = \emph{r}×\emph{K}/4 ≈ 0.269×257.14/4 ≈ 17.29. На графике
отображена траектория запаса с 2005 по 2024 годы, где каждая точка
соответствует медианным оценкам за год, а соединяющие их линии
показывают хронологическую динамику. В 2005 году запас находился в
идеальном состоянии: \emph{B/B\textsubscript{msy}} = 1.83 (95\% ДИ:
1.25--2.52), \emph{F/F\textsubscript{msy}} = 0.15 (0.09--0.26), что
помещает его глубоко в зеленый квадрант.

Период 2010-2015 годов демонстрирует критическое ухудшение состояния: к
2013 году \emph{B/B\textsubscript{msy}} снижается до 0.89 (0.64--1.15),
а \emph{F/F\textsubscript{msy}} достигает пика 1.74 (1.16--2.86) в 2011
году, что соответствует оранжевому квадранту активного перелова. Это
совпадает с историческими данными о максимальных уловах (28-35 единиц в
2009-2013 гг.), превышающих расчетный \emph{MSY} (17.29). Последующий
период (2016-2024) показывает восстановление: к 2024 году
\emph{B/B\textsubscript{msy}} возрастает до 1.29 (0.90--1.64), а
\emph{F/F\textsubscript{msy}} снижается до 0.51 (0.31--0.93), возвращая
запас в зеленый квадрант.

Текущее положение (2024 год) указывает на восстановление запаса, однако
горизонтально вытянутое облако неопределенности для
\emph{B/B\textsubscript{msy}} отражает чувствительность к оценке
параметра K, чей широкий доверительный интервал (178--442) обусловлен
исторической нехваткой данных о периоде высокой биомассы. Вертикальная
компактность \emph{F/F\textsubscript{msy}} подтверждает относительно
точную оценку промысловой смертности. Управленческая рекомендация
основывается на медианных значениях: поддержание
\emph{F/F\textsubscript{msy}} на уровне ≈0.5 позволит сохранить запас в
устойчивом состоянии. Однако из-за неопределенности в оценке
\emph{B/B\textsubscript{msy}} (риск попадания в желтый квадрант при
нижней границе ДИ 0.90) необходим ежегодный мониторинг с обновлением
модели по новым данным. Исторический пример перелова 2011-2013 годов
демонстрирует последствия превышения \emph{F/F\textsubscript{msy}}
\textgreater1.5, что должно учитываться при установке лимитов вылова.

\subsection{Дополнительные
диагностики}\label{ux434ux43eux43fux43eux43bux43dux438ux442ux435ux43bux44cux43dux44bux435-ux434ux438ux430ux433ux43dux43eux441ux442ux438ux43aux438}

\begin{Shaded}
\begin{Highlighting}[]
\CommentTok{\# Дополнительные диагностики:}
\FunctionTok{jbplot\_runstest}\NormalTok{(fit)    }\CommentTok{\# Тест на случайность остатков}
\FunctionTok{jbplot\_logfits}\NormalTok{(fit)     }\CommentTok{\# Графики в логарифмической шкале}
\FunctionTok{jbplot\_procdev}\NormalTok{(fit)     }\CommentTok{\# Отклонения процесса}

\CommentTok{\# Сохранение временных рядов результатов}
\FunctionTok{write.csv}\NormalTok{(fit}\SpecialCharTok{$}\NormalTok{timeseries, }\AttributeTok{file =} \StringTok{"results.csv"}\NormalTok{, }\AttributeTok{row.names =} \ConstantTok{FALSE}\NormalTok{)}
\end{Highlighting}
\end{Shaded}

\begin{figure}[H]

{\centering \includegraphics[width=0.6\linewidth,height=\textheight,keepaspectratio]{images/JABBA7.png}

}

\caption{Рис. 7.: График теста серий (runs test) для диагностики
остатков}

\end{figure}%

Дублирование графика остатков, но для каждого индекса выводится
отдельный график. На графике отображаются остатки модели (разницы между
наблюдаемыми и предсказанными значениями индексов обилия) в виде
последовательности точек, упорядоченных по времени.Тест серий (runs
test) анализирует последовательность чередований положительных и
отрицательных остатков. Случайное распределение остатков (что является
желаемым результатом) будет проявляться в частом чередовании
положительных и отрицательных значений. Напротив, наличие длинных серий
(несколько положительных или отрицательных остатков подряд) может
указывать на:

\begin{itemize}
\item
  Неучтенные временные зависимости в данных
\item
  Неадекватность структуры модели (например, пропущенные важные
  переменные)
\item
  Систематические ошибки в данных
\item
  Неправильную спецификацию функциональной формы модели
\end{itemize}

\textbf{График отклонений процесса (Process Deviations) в JABBA}

\begin{figure}[H]

{\centering \includegraphics[width=0.6\linewidth,height=\textheight,keepaspectratio]{images/JABBA8.png}

}

\caption{Рис. 8.: Отклонения процесса}

\end{figure}%

График отклонений процесса (\textbf{\texttt{jbplot\_procdev(fit)}})
отображает различия между фактической динамикой биомассы и
теоретическими предсказаниями продукционной модели. Эти отклонения
(Bdev) количественно выражают влияние неучтенных моделью факторов на
популяцию --- от климатических аномалий до изменений в кормовой базе. В
нашем анализе медианные значения Bdev колеблются в узком диапазоне от
-0.024 (2011) до +0.021 (2017), при этом все 95\% доверительные
интервалы включают ноль. Это свидетельствует об отсутствии статистически
значимых отклонений, что подтверждает адекватность базовой модели
Шефера. Биологически положительные Bdev указывают на неожиданный рост
биомассы (например, за счет улучшения условий воспроизводства), тогда
как отрицательные --- на незапланированные потери (эпизоотии,
незарегистрированную смертность).

Особый интерес представляют два периода: 2011 год с минимальным
\emph{Bdev} (-0.024) совпадает с пиком промысловой нагрузки
(\emph{F/F\textsubscript{msy}} ≈ 1.74), что может отражать
дополнительную естественную смертность, вызванную переловом. В 2017 году
положительное отклонение (+0.021) соответствует фазе активного
восстановления запаса. Важно, что 80\% лет показывают абсолютные
значения \emph{Bdev} \textless{} 0.01 --- исключительно высокий
показатель, подчеркивающий надежность модели. Приемлемым диапазоном
считаются отклонения в пределах ±0.05 при сохранении доверительных
интервалов, пересекающих ноль; наши данные существенно строже этих
критериев.

Существенное влияние на отклонения процесса оказывает неучтенный вылов.
Если такой вылов присутствует, модель, не получая данных о реальном
изъятии, будет систематически переоценивать биомассу. Это проявляется
как стабильно отрицательные \emph{Bdev}, особенно выраженные в периоды
интенсивного промысла. Например, при ежегодном неучтенном изъятии в 20\%
от официального улова, медианные отклонения сместились бы в зону
-0.05\ldots-0.10, а доверительные интервалы перестали бы включать ноль.
В нашем случае отсутствие таких систематических сдвигов (разброс Bdev
симметричен относительно нуля) позволяет заключить, что неучтенный вылов
не является критическим фактором для данной популяции. Однако для
окончательных выводов требуется анализ ретроспективных данных по
промысловому усилию и независимая верификация учетных методик.

Помимо визуализации, используя нижеприведенный скрипт, можно получить
фактические значения, например, с 90\%-ным доверительным интервалом:
years \textless- fit\$yr

\begin{Shaded}
\begin{Highlighting}[]
\CommentTok{\# Агрегировать Bdev по годам (медиана и 90\% интервал)}

\NormalTok{proc\_dev \textbackslash{}}\OtherTok{\textless{}{-}} \FunctionTok{aggregate}\NormalTok{(Bdev \textbackslash{}}\SpecialCharTok{\textasciitilde{}}\NormalTok{ year, }\AttributeTok{data =}\NormalTok{ fit\textbackslash{}}\SpecialCharTok{$}\NormalTok{kbtrj, }\AttributeTok{FUN =} \ControlFlowTok{function}\NormalTok{(x) }\FunctionTok{c}\NormalTok{(}\AttributeTok{median =} \FunctionTok{median}\NormalTok{(x), }\AttributeTok{lci =} \FunctionTok{quantile}\NormalTok{(x, }\FloatTok{0.05}\NormalTok{), }\AttributeTok{uci =} \FunctionTok{quantile}\NormalTok{(x, }\FloatTok{0.95}\NormalTok{)))}

\FunctionTok{print}\NormalTok{(proc\_dev)}
\end{Highlighting}
\end{Shaded}

\section{Ретроспективный
анализ}\label{ux440ux435ux442ux440ux43eux441ux43fux435ux43aux442ux438ux432ux43dux44bux439-ux430ux43dux430ux43bux438ux437-1}

Мы переходим к ретроспективному анализу (hindcasting), который является
важным инструментом для оценки устойчивости модели и ее чувствительности
к новым данным. В ретроспективном анализе модель последовательно
переоценивается с исключением последних лет данных (по одному году за
раз, в данном случае от 1 до 5 лет). Это позволяет проверить, насколько
сильно меняются оценки ключевых параметров и статуса запаса при
поступлении новых данных.

В вашем скрипте ретроспективный анализ запускается функцией
`hindcast\_jabba()`, которая использует исходные настройки модели
(`jbinput`) и результаты базовой оценки (`fit`). Аргумент `peels = 1:5`
указывает, что нужно последовательно удалять от 1 до 5 последних лет
данных. Результаты сохраняются в объект `hc`.

Затем с помощью функции `jbplot\_retro()` визуализируются результаты
ретроспективного анализа. График показывает, как меняется оценка
биомассы (или B/Bmsy) при исключении данных за последние годы. На
графике будет изображена траектория базовой оценки (со всеми данными) и
траектории, полученные при удалении 1, 2, 3, 4 и 5 лет. Также
рассчитывается и отображается статистика Мона (Mohn's rho), которая
количественно оценивает смещение ретроспективных оценок относительно
базовой.

\begin{Shaded}
\begin{Highlighting}[]
\CommentTok{\# {-}{-}{-}{-}{-}{-}{-}{-}{-}{-}{-}{-}{-}{-}{-}{-}{-}{-}{-} 5. РЕТРОСПЕКТИВНЫЙ АНАЛИЗ {-}{-}{-}{-}{-}{-}{-}{-}{-}{-}{-}{-}{-}{-}{-}{-}{-}{-}{-}{-}}

\CommentTok{\# Создание папки для результатов ретроспективы}
\NormalTok{retro.dir }\OtherTok{\textless{}{-}} \FunctionTok{file.path}\NormalTok{(output.dir, }\StringTok{"retro"}\NormalTok{)}
\FunctionTok{dir.create}\NormalTok{(retro.dir, }\AttributeTok{showWarnings =} \ConstantTok{FALSE}\NormalTok{)}

\CommentTok{\# Запуск ретроспективного анализа (убираем по 1{-}5 лет)}
\NormalTok{hc }\OtherTok{\textless{}{-}} \FunctionTok{hindcast\_jabba}\NormalTok{(}\AttributeTok{jbinput =}\NormalTok{ jbinput, }\AttributeTok{fit =}\NormalTok{ fit, }\AttributeTok{peels =} \DecValTok{1}\SpecialCharTok{:}\DecValTok{5}\NormalTok{)}

\CommentTok{\# Визуализация ретроспективного анализа}
\NormalTok{mohnsrho }\OtherTok{\textless{}{-}} \FunctionTok{jbplot\_retro}\NormalTok{(}
\NormalTok{  hc, }
  \AttributeTok{as.png =} \ConstantTok{FALSE}\NormalTok{,         }\CommentTok{\# Чтобы сохранить как PNG{-}файл установите TRUE}
  \AttributeTok{output.dir =}\NormalTok{ retro.dir,}
  \AttributeTok{xlim =} \FunctionTok{c}\NormalTok{(}\DecValTok{2007}\NormalTok{, }\DecValTok{2022}\NormalTok{)   }\CommentTok{\# Ограничения по годам на графике}
\NormalTok{)}

\CommentTok{\# Кросс{-}валидация}
\NormalTok{mase }\OtherTok{\textless{}{-}} \FunctionTok{jbplot\_hcxval}\NormalTok{(hc, }\AttributeTok{as.png =} \ConstantTok{FALSE}\NormalTok{, }\AttributeTok{output.dir =}\NormalTok{ retro.dir)}
\end{Highlighting}
\end{Shaded}

\textbf{Статистика Мона (Mohn's rho)}:\\
Ключевой показатель смещения, рассчитываемый как относительная разница
между ретроспективной и базовой оценкой в год исключения:

\[
\rho = \frac{X_{\text{ретро}} - X_{\text{база}}}{X_{\text{база}}}
\]

\begin{Shaded}
\begin{Highlighting}[]
\SpecialCharTok{\textgreater{}}\NormalTok{ mohnsrho}
\NormalTok{                 B           F         Bmsy        Fmsy         procB}
\DecValTok{2024}   \SpecialCharTok{{-}}\FloatTok{0.01949857}  \FloatTok{0.01635894}  \FloatTok{0.031098563} \SpecialCharTok{{-}}\FloatTok{0.05870471}  \FloatTok{0.0019184659}
\DecValTok{2023}    \FloatTok{0.06336652} \SpecialCharTok{{-}}\FloatTok{0.05341885} \SpecialCharTok{{-}}\FloatTok{0.009782004}  \FloatTok{0.03945419} \SpecialCharTok{{-}}\FloatTok{0.0004440846}
\DecValTok{2022}   \SpecialCharTok{{-}}\FloatTok{0.04605748}  \FloatTok{0.04811313} \SpecialCharTok{{-}}\FloatTok{0.043482407}  \FloatTok{0.10142418} \SpecialCharTok{{-}}\FloatTok{0.0029895725}
\DecValTok{2021}   \SpecialCharTok{{-}}\FloatTok{0.04274631}  \FloatTok{0.03790912}  \FloatTok{0.014020696} \SpecialCharTok{{-}}\FloatTok{0.03150997}  \FloatTok{0.0011936097}
\DecValTok{2020}   \SpecialCharTok{{-}}\FloatTok{0.04268405}  \FloatTok{0.04748624} \SpecialCharTok{{-}}\FloatTok{0.034149276}  \FloatTok{0.04684297} \SpecialCharTok{{-}}\FloatTok{0.0006575862}
\NormalTok{rho.mu }\SpecialCharTok{{-}}\FloatTok{0.01752398}  \FloatTok{0.01928972} \SpecialCharTok{{-}}\FloatTok{0.008458886}  \FloatTok{0.01950133} \SpecialCharTok{{-}}\FloatTok{0.0001958335}
\NormalTok{               MSY}
\DecValTok{2024}    \FloatTok{0.02871203}
\DecValTok{2023}   \SpecialCharTok{{-}}\FloatTok{0.02266077}
\DecValTok{2022}   \SpecialCharTok{{-}}\FloatTok{0.03747819}
\DecValTok{2021}    \FloatTok{0.02474720}
\DecValTok{2020}    \FloatTok{0.00148479}
\NormalTok{rho.mu }\SpecialCharTok{{-}}\FloatTok{0.00103899}
\SpecialCharTok{\textgreater{}} 
\end{Highlighting}
\end{Shaded}

\begin{figure}[H]

{\centering \includegraphics[width=0.8\linewidth,height=\textheight,keepaspectratio]{images/JABBA9.png}

}

\caption{Рис. 9.: Графики ретроспективного анализа}

\end{figure}%

\begin{enumerate}
\def\labelenumi{\arabic{enumi}.}
\item
  \textbf{Наши результаты:}

  \begin{itemize}
  \item
    Биомасса (*Bм): ρ=−0.018 (слабое отрицательное смещение)
  \item
    Смертность (\emph{F}): ρ=+0.019 (слабое положительное смещение)
  \item
    \emph{MSY}: ρ=−0.001 (незначимое смещение) Критерий: ∣ρ∣\textless0.2
    приемлемо наши значения ≪0.2.
  \end{itemize}
\item
  \textbf{Визуализация (\texttt{jbplot\_retro})} показывает:

  \begin{itemize}
  \item
    Как меняется траектория биомассы при исключении данных
  \item
    ``Веер'' расходящихся линий: чем сильнее расхождение, тем выше
    нестабильность модели
  \item
    В вашем случае линии остаются близкими --- \textbf{модель
    устойчива}.
  \end{itemize}
\end{enumerate}

\textbf{Метрика MASE} (Mean Absolute Scaled Error).

Кросс-валидация с помощью \texttt{jbplot\_hcxval()} оценивает
предсказательную способность модели. Для каждого ``среза'' (peel) модель
предсказывает индекс обилия для удаленных лет, а затем эти предсказания
сравниваются с фактическими наблюдениями. Рассчитывается MASE (Mean
Absolute Scaled Error) --- средняя абсолютная ошибка прогноза,
нормированная на ошибку наивного прогноза (который предполагает, что
будущее значение равно последнему наблюдённому).

\[
\mathrm{MASE} = \frac{\text{Средняя ошибка прогноза}}{\text{Средняя ошибка наивного прогноза}}
\]

\begin{itemize}
\item
  MASE \textless{} 1: Модель лучше наивного метода (предсказывающего
  ``завтра=сегодня'')
\item
  MASE \textgreater{} 1: Модель работает хуже наивного метода
\end{itemize}

\begin{figure}[H]

{\centering \includegraphics[width=0.8\linewidth,height=\textheight,keepaspectratio]{images/JABBA10.png}

}

\caption{Рис. 10.: Метрика MASE}

\end{figure}%

\begin{Shaded}
\begin{Highlighting}[]
\SpecialCharTok{\textgreater{}}\NormalTok{ mase}
\NormalTok{  Index      MASE  MASE.adj     MAE.PR   MAE.base n.eval}
\DecValTok{1}\NormalTok{  CPUE }\FloatTok{0.6806899} \FloatTok{0.6402326} \FloatTok{0.13413055} \FloatTok{0.19705090}      \DecValTok{5}
\DecValTok{2}\NormalTok{  BESS }\FloatTok{0.9539647} \FloatTok{0.6632760} \FloatTok{0.07763672} \FloatTok{0.08138322}      \DecValTok{5}
\DecValTok{3}\NormalTok{ joint }\FloatTok{0.7605651} \FloatTok{0.7605651} \FloatTok{0.10588363} \FloatTok{0.13921706}     \DecValTok{10}
\end{Highlighting}
\end{Shaded}

Наши результаты показывают, что для CPUE модель предсказывает лучше, чем
наивный метод (0.68\textless1), а для BESS --- немного хуже (0.95≈1).
Совместный MASE (0.76) указывает на удовлетворительную общую прогнозную
способность. Однако для индекса BESS стоит обратить внимание на
возможные улучшения.

\begin{longtable}[]{@{}
  >{\raggedright\arraybackslash}p{(\linewidth - 4\tabcolsep) * \real{0.2639}}
  >{\raggedright\arraybackslash}p{(\linewidth - 4\tabcolsep) * \real{0.2639}}
  >{\raggedright\arraybackslash}p{(\linewidth - 4\tabcolsep) * \real{0.4722}}@{}}
\toprule\noalign{}
\begin{minipage}[b]{\linewidth}\raggedright
\textbf{Индекс}
\end{minipage} & \begin{minipage}[b]{\linewidth}\raggedright
\textbf{MASE}
\end{minipage} & \begin{minipage}[b]{\linewidth}\raggedright
\textbf{Интерпретация}
\end{minipage} \\
\midrule\noalign{}
\endhead
\bottomrule\noalign{}
\endlastfoot
CPUE & 0.68 & \textbf{Хорошо}: Прогноз на 32\% точнее наивного \\
BESS & 0.95 & \textbf{Удовлетворительно}: Почти эквивалентен наивному
методу \\
Совместно & 0.76 & Приемлемая общая точность \\
\end{longtable}

\textbf{Почему это важно?}

\begin{enumerate}
\def\labelenumi{\arabic{enumi}.}
\tightlist
\item
  \textbf{Для управления запасами}:
\end{enumerate}

\begin{itemize}
\tightlist
\item
  Стабильность Mohn's rho (ρ≈0) означает, что текущие рекомендации по
  вылову не изменятся радикально при получении новых данных.
\item
  Низкий MASE подтверждает надежность краткосрочных прогнозов.
\end{itemize}

\begin{enumerate}
\def\labelenumi{\arabic{enumi}.}
\setcounter{enumi}{1}
\tightlist
\item
  \textbf{Для диагностики модели}:
\end{enumerate}

\begin{itemize}
\tightlist
\item
  Систематическое смещение ρB\textgreater0.3 могло бы указывать на
  переоценку запаса (риск перелова).
\item
  MASE \textgreater{} 1 для BESS требует улучшения описания этого
  индекса (например, через калибровку \emph{q}).
\end{itemize}

\begin{enumerate}
\def\labelenumi{\arabic{enumi}.}
\setcounter{enumi}{2}
\tightlist
\item
  \textbf{Исторический контекст}:\\
  В наших данных слабое смещение для \emph{MSY} (ρ=−0.001) подтверждает,
  что модель корректно определяет максимальный устойчивый вылов (17.29),
  несмотря на проблемы с оценкой \emph{r} (PPMR=1.47).
\end{enumerate}

\textbf{Выводы для нашего случая:}

Ретроспективный анализ показал:

\begin{enumerate}
\def\labelenumi{\arabic{enumi}.}
\tightlist
\item
  \textbf{Высокую стабильность}: Смещения параметров статистически
  незначимы.
\item
  \textbf{Удовлетворительную прогнозную силу}: Особенно для CPUE
  (MASE=0.68).
\item
  \textbf{Область улучшения}: Индекс BESS требует внимания (MASE=0.95),
  возможно, за счет включения ковариат или пересмотра ошибок наблюдений.
\item
  \textbf{Надежность управленческих выводов}: Текущие оценки статуса
  запаса (B/Bmsy = 1.29, F/Fmsy = 0.51) устойчивы к добавлению новых
  данных.
\end{enumerate}

Этот этап завершает валидацию модели, подтверждая, что она пригодна для
разработки рекомендаций по управлению промыслом.

\section{Прогнозирование}\label{ux43fux440ux43eux433ux43dux43eux437ux438ux440ux43eux432ux430ux43dux438ux435}

Прогнозирование в JABBA представляет собой заключительный этап оценки
запасов, позволяющий смоделировать будущую динамику популяции при
различных сценариях управления промыслом. В нашем случае был выполнен
10-летний стохастический прогноз (2025-2034 гг.), основанный на текущем
состоянии запаса, где биомасса в 2024 году оценивается в 148.6 тыс. т
при соотношении B/Bmsy = 1.29 и промысловой нагрузке F/Fmsy = 0.51. Были
рассмотрены четыре сценария годового изъятия: консервативный (10 тыс.
т), умеренные (12 и 14 тыс. т) и интенсивный (16 тыс. т), что
соответствует 58-93\% от расчетного максимального устойчивого улова (MSY
= 17.29).

\begin{Shaded}
\begin{Highlighting}[]
\CommentTok{\# {-}{-}{-}{-}{-}{-}{-}{-}{-}{-}{-}{-}{-}{-}{-}{-}{-}{-}{-} 6. ПРОГНОЗИРОВАНИЕ {-}{-}{-}{-}{-}{-}{-}{-}{-}{-}{-}{-}{-}{-}{-}{-}{-}{-}{-}{-}}

\CommentTok{\# Прогноз на основе F (ловушечное усилие)}
\NormalTok{fw1 }\OtherTok{\textless{}{-}} \FunctionTok{fw\_jabba}\NormalTok{(}
\NormalTok{  fit,}
  \AttributeTok{nyears =} \DecValTok{10}\NormalTok{,       }\CommentTok{\# Длина прогноза (лет)}
  \AttributeTok{imp.yr =} \DecValTok{1}\NormalTok{,        }\CommentTok{\# Год внедрения новых правил}
  \AttributeTok{imp.values =} \FunctionTok{seq}\NormalTok{(}\DecValTok{10}\NormalTok{, }\DecValTok{16}\NormalTok{, }\DecValTok{2}\NormalTok{), }\CommentTok{\# Варианты управления (уровни улова)}
  \AttributeTok{quant =} \StringTok{"Catch"}\NormalTok{,   }\CommentTok{\# Прогнозировать по уловам}
  \AttributeTok{type =} \StringTok{"abs"}\NormalTok{,      }\CommentTok{\# Абсолютные значения}
  \AttributeTok{stochastic =} \ConstantTok{TRUE}  \CommentTok{\# Стохастический прогноз}
\NormalTok{)}

\CommentTok{\# Графики ансамбля прогнозов}
\FunctionTok{jbpar}\NormalTok{(}\AttributeTok{mfrow =} \FunctionTok{c}\NormalTok{(}\DecValTok{3}\NormalTok{, }\DecValTok{2}\NormalTok{)) }\CommentTok{\# Настройка макета графиков (3 строки, 2 столбца)}
\FunctionTok{jbplot\_ensemble}\NormalTok{(fw1)    }\CommentTok{\# Основной график прогнозов}
\end{Highlighting}
\end{Shaded}

\begin{figure}[H]

{\centering \includegraphics[width=0.8\linewidth,height=\textheight,keepaspectratio]{images/JABBA11.png}

}

\caption{Рис. 11.: Прогностические графики}

\end{figure}%

Ключевой особенностью прогноза является его стохастическая природа ---
каждый сценарий учитывает неопределенность параметров модели, включая
вариабельность темпа роста популяции (r) и ёмкости среды (K), а также
ошибку процесса (σ² = 0.0035). Это позволяет получить не точечные
предсказания, а вероятностные распределения будущих состояний.
Результаты показывают дифференцированную динамику: при вылове 10-12 тыс.
т биомасса демонстрирует устойчивый рост (до 178 и 168 тыс. т к 2034
году), сценарий с 14 тыс. т стабилизирует запас на уровне около 158 тыс.
т, тогда как интенсивный вылов (16 тыс. т) приводит к постепенному
снижению биомассы до 147 единиц.

Биологическая интерпретация этих траекторий основывается на соотношении
прогнозируемых показателей с целевыми ориентирами. Для сценария C16 к
2034 году ожидается приближение B/Bmsy к 1.14, что хотя и остается выше
единицы, но указывает на сокращение ``буферного'' запаса. Особое
внимание следует уделить чувствительности модели к оценке параметра r,
чья высокая неопределенность (PPMR = 1.47) может существенно влиять на
долгосрочные прогнозы --- например, если реальный темп роста окажется
ближе к нижней границе доверительного интервала (0.15), сценарий C16
может привести к переходу в зону перелова уже к 2030 году.

Управленческие рекомендации, вытекающие из анализа, предлагают
компромисс между экономической эффективностью и предосторожностью.
Оптимальным признается диапазон вылова 12-14 единиц, обеспечивающий
70-80\% от потенциального прироста продукции без риска снижения запаса
ниже целевого уровня. Сценарий C16 может рассматриваться как временная
мера только при наличии подтверждающих данных о высоком продуктивном
потенциале популяции, но требует ежегодного мониторинга с коррекцией
лимитов. Визуализация результатов через
\textbf{\texttt{jbplot\_ensemble()}} наглядно демонстрирует ``веер''
траекторий, где расхождение доверительных интервалов усиливается к концу
периода прогноза --- это прямое отражение кумулятивного эффекта
неопределенности параметров и случайных факторов среды.

Важным аспектом является интеграция прогноза в адаптивную систему
управления: установив начальный лимит на уровне 14 тыс. т, следует
планировать повторные оценки по данным ежегодных съемок, что позволит
корректировать вылов в зависимости от фактического состояния запаса.
Такой подход минимизирует риски, связанные с ограниченной точностью
продукционных моделей при работе с данными низкой разрешающей
способности. Исторический урок нашего анализа --- пример перелова
2011-2013 годов --- напоминает, что превышение F/Fmsy \textgreater{} 1.5
способно за несколько лет подорвать даже запас, находившийся в
благополучном состоянии.

При работе с JABBA есть трудности в получении различных графиков и
фактических значений, например прогнозных. Ниже приводятся скрипты
получения отдельных прогностических графиков и таблицы прогнозных
значений выловов и биомасс.

\begin{Shaded}
\begin{Highlighting}[]
\CommentTok{\# График для B/Bmsy с кастомизацией}
\FunctionTok{jbplot\_ensemble}\NormalTok{(}
\NormalTok{  fw1,}
  \AttributeTok{subplots =} \FunctionTok{c}\NormalTok{(}\DecValTok{1}\NormalTok{),        }\CommentTok{\# Только B/Bmsy}
  \AttributeTok{add =} \ConstantTok{TRUE}\NormalTok{,             }\CommentTok{\# Добавить к текущему графику}
  \AttributeTok{xlim =} \FunctionTok{c}\NormalTok{(}\DecValTok{2020}\NormalTok{, }\DecValTok{2035}\NormalTok{),   }\CommentTok{\# Ограничение по годам}
  \AttributeTok{legend.loc =} \StringTok{"topleft"}  \CommentTok{\# Позиция легенды}
\NormalTok{)}
\end{Highlighting}
\end{Shaded}

\begin{figure}[H]

{\centering \includegraphics[width=0.8\linewidth,height=\textheight,keepaspectratio]{images/JABBA12.png}

}

\caption{Рис. 12.: Отдельный график прогноза}

\end{figure}%

Извлечение прогостических данных:

\begin{Shaded}
\begin{Highlighting}[]
\CommentTok{\# Фильтрация данных прогноза (2025{-}2034) для выбранных сценариев}
\NormalTok{forecast\_data }\OtherTok{\textless{}{-}} \FunctionTok{subset}\NormalTok{(}
\NormalTok{  fw1, }
\NormalTok{  year }\SpecialCharTok{\%in\%} \DecValTok{2025}\SpecialCharTok{:}\DecValTok{2034} \SpecialCharTok{\&}    \CommentTok{\# Годы прогноза}
\NormalTok{    run }\SpecialCharTok{\%in\%} \FunctionTok{c}\NormalTok{(}\StringTok{"C10"}\NormalTok{, }\StringTok{"C12"}\NormalTok{, }\StringTok{"C14"}\NormalTok{, }\StringTok{"C16"}\NormalTok{) }\SpecialCharTok{\&} \CommentTok{\# Сценарии управления}
\NormalTok{    type }\SpecialCharTok{==} \StringTok{"prj"}           \CommentTok{\# Только прогнозные значения}
\NormalTok{)}

\CommentTok{\# Расчет медиан биомассы (B) по годам и сценариям}
\NormalTok{median\_B }\OtherTok{\textless{}{-}} \FunctionTok{aggregate}\NormalTok{(}
\NormalTok{  B }\SpecialCharTok{\textasciitilde{}}\NormalTok{ year }\SpecialCharTok{+}\NormalTok{ run,          }\CommentTok{\# Формула: группировка по году и сценарию}
  \AttributeTok{data =}\NormalTok{ forecast\_data, }
  \AttributeTok{FUN =}\NormalTok{ median             }\CommentTok{\# Функция агрегации}
\NormalTok{)}

\CommentTok{\# Расчет медиан улова (Catch) по годам и сценариям}
\NormalTok{median\_Catch }\OtherTok{\textless{}{-}} \FunctionTok{aggregate}\NormalTok{(}
\NormalTok{  Catch }\SpecialCharTok{\textasciitilde{}}\NormalTok{ year }\SpecialCharTok{+}\NormalTok{ run, }
  \AttributeTok{data =}\NormalTok{ forecast\_data, }
  \AttributeTok{FUN =}\NormalTok{ median}
\NormalTok{)}

\CommentTok{\# Преобразование в широкий формат (годы по строкам, сценарии по столбцам)}
\NormalTok{b\_table }\OtherTok{\textless{}{-}} \FunctionTok{dcast}\NormalTok{(median\_B, year }\SpecialCharTok{\textasciitilde{}}\NormalTok{ run, }\AttributeTok{value.var =} \StringTok{"B"}\NormalTok{)}
\NormalTok{catch\_table }\OtherTok{\textless{}{-}} \FunctionTok{dcast}\NormalTok{(median\_Catch, year }\SpecialCharTok{\textasciitilde{}}\NormalTok{ run, }\AttributeTok{value.var =} \StringTok{"Catch"}\NormalTok{)}

\CommentTok{\# Вывод таблиц}
\FunctionTok{print}\NormalTok{(}\StringTok{"Медианная биомасса:"}\NormalTok{)}
\FunctionTok{print}\NormalTok{(b\_table)}

\FunctionTok{print}\NormalTok{(}\StringTok{"Медианные уловы:"}\NormalTok{)}
\FunctionTok{print}\NormalTok{(catch\_table)}

\CommentTok{\# Сохранение таблиц}
\FunctionTok{write.csv}\NormalTok{(b\_table, }\StringTok{"biomass\_forecast.csv"}\NormalTok{, }\AttributeTok{row.names =} \ConstantTok{TRUE}\NormalTok{)}
\FunctionTok{write.csv}\NormalTok{(catch\_table, }\StringTok{"catch\_forecast.csv"}\NormalTok{, }\AttributeTok{row.names =} \ConstantTok{TRUE}\NormalTok{)}
\end{Highlighting}
\end{Shaded}

\bookmarksetup{startatroot}

\chapter{Прогноз пополнения: от факторов до
ансамбля}\label{ux43fux440ux43eux433ux43dux43eux437-ux43fux43eux43fux43eux43bux43dux435ux43dux438ux44f-ux43eux442-ux444ux430ux43aux442ux43eux440ux43eux432-ux434ux43e-ux430ux43dux441ux430ux43cux431ux43bux44f}

\section{Введение}\label{ux432ux432ux435ux434ux435ux43dux438ux435-7}

Начнём с ловушки, в которую легко попасть всем нам. Когда перед глазами
аккуратные таблицы предикторов и графики «запас--пополнение», возникает
приятное ощущение, что, выбрав правильную модель и пару красивых
параметров, мы можем управлять будущим --- будто поставив регулятор на
приборной панели, повысим пополнение в нужный год. Это и есть иллюзия
контроля: данные неполны, среда нестационарна, а часть закономерностей
--- просто шум, который мозг охотно принимает за сигнал. Наша задача ---
осознанно тормозить, проверять допущения, избегать утечки информации из
будущего в прошлое и признавать, что длинные хвосты --- не исключение, а
рабочая среда экологии. При этом, повторюсь, скатываться в цинизм не
нужно: рациональный, аккуратно спланированный анализ действительно
улучшает предсказания и решения --- в смысле медленного, но реального
прогресса, если мы дисциплинируем энтузиазм диагностикой и
воспроизводимостью. Биология редко ведёт себя как учебник: нелинейности,
пороги, запаздывания и переменная «уловистость» индикаторов --- всё это
часть истории, и именно поэтому важно строить не одну «идеальную»
модель, а цикл проверки конкурирующих объяснений, опираясь и на
предметную логику, и на статистическую строгость.

Эта практическая работа --- про такой цикл. Мы изучаем зависимость
пополнения R3haddock (возможно это пикша) от среды и нерестового запаса,
идём от подготовки данных до прогноза с доверительными интервалами,
намеренно сопоставляя разные семейства моделей. Сначала приводим
предикторы к численному виду, осмысленно обходимся с пропусками и
сокращаем мультиколлинеарность, чтобы не строить интерпретацию на «двух
зеркальных термометрах». Затем запускаем автоматический отбор признаков
двумя методами --- Boruta (в духе нелинейной, «лесной» логики) и LASSO
(строгая сжатая линейная постановка). Их пересечение, скорректированное
биологическим смыслом (например, сохранением нерестового запаса
haddock68), даёт устойчивый стартовый набор факторов: температуры и
океанография (T\ldots, O\ldots), биотические индикаторы (например,
codTSB) и нерестовый запас. Далее мы намеренно не женимся на одном
«красивом» уравнении: ставим рядом механистические модели
«запас--пополнение» (Рикер и Бивертон--Холт) и статистические
LM/GLM/GAM, чтобы увидеть, где данные действительно поддерживают горб
плотностной зависимости, а где кривая честно выходит на плато; где
линейная аппроксимация достаточна, а где гладкие функции выявляют
оптимумы, пороги и нелинейные эффекты. Такой параллельный взгляд --- это
не прихоть, а способ не перепутать удобную историю с реальной динамикой.

Критический момент --- валидация во времени. Обычная случайная
кросс‑валидация льстит нам, подмешивая будущее в прошлое; мы избегаем
этого через time‑slice с расширяющимся окном и горизонтом прогноза, а
затем фиксируем результат внешним хронологическим тестом. Так мы
проверяем не только «как хорошо объяснили вчера», но и «как не
обманулись насчёт завтра». По итогам сравнения берём не «победителя по
AIC/Р² любой ценой», а устойчивую к хронологии схему --- иногда это GAM,
иногда более простая GLM, а порой и ансамбль, который признаёт, что
комбинирование разных источников знания (простого и гибкого) часто
надёжнее любого одиночного героя. Прогноз на 2022--2024 мы даём не как
цифры отлитые в граните, а как веер с 50\% и 95\% интервалами --- потому
что у природы длинные хвосты, и задача аналитика --- показать диапазон
правдоподобного, а не притворяться владельцем хрустального шара. И
наконец --- про нарратив: любой вывод модели --- это маленькая история о
будущем запаса. Она приемлема для управления только тогда, когда прошла
проверку данными, диагностикой и попыткой опровержения альтернативами. В
этом практикуме мы именно так и поступаем: даём данным говорить, а себе
--- сомневаться, сравнивать и проверять. Именно так рождаются прогнозы,
которые выдерживают столкновение с реальностью.

И так в сухом остатке, в этой практической работе представлен цикл
прикладного анализа зависимости пополнения запаса гидробионта от
факторов среды (в том числе нерестового запаса): от подготовки данных и
отбора предикторов до сравнения нескольких семейств моделей, выбора
устойчивой к хронологии прогностической схемы и построения прогноза с
доверительными интервалами. Подход ориентирован на начинающих, но
использует современные приёмы: автоматический отбор признаков (Boruta,
LASSO), сопоставление линейных/нелинейных моделей, time-slice валидацию
и ансамблевый прогноз. Целевая переменная: R3haddock --- пополнение
запаса. Кандидатные предикторы: гидрометеорология (температуры T\ldots),
океанография (O\ldots), биотические показатели (например, codTSB) и
нерестовый запас (haddock68). Цель анализа: понять, какие факторы и в
каких формах оказываются значимыми, отобрать рабочий набор моделей и
получить прогноз на 2022--2024 с оценкой неопределенности.

Настоящий анализ разделен на несколько этапов:

\begin{enumerate}
\def\labelenumi{\arabic{enumi})}
\item
  Выбор предикторов. Скрипт можно скачать по
  \href{https://mombus.github.io/cRab/data/RECRUITMENT_PREDICTORS.R}{ссылке}.
\item
  Построение биологически мотивированных (механистических) нелинейных
  классических моделей «запас-пополнение» Рикера и Бивертона-Холта.
  Анализ их значимости и сравнение с моделями LM/GLM/GAM. Скрипт можно
  скачать по
  \href{https://mombus.github.io/cRab/data/RECRUITMENT_CLASSIC.R}{ссылке}.
\item
  Построение классических статистических моделей LM/GLM/GAM. Анализ их
  прогностических способностей и выполнение прогноза. Скрипт можно
  скачать по
  \href{https://mombus.github.io/cRab/data/RECRUITMENT_LM_GLM_GAM.R}{ссылке}.
\item
  Полный цикл (
  \href{https://mombus.github.io/cRab/data/RECRUITMENT_MAIN.R}{СКРИПТ})
  прикладного анализа зависимости пополнения запаса гидробионта от
  факторов среды, включающий классические модели, машинное обучение, а
  также этапы:
\end{enumerate}

\begin{itemize}
\item
  а) выбор предикторов;
\item
  б) базовое сравнение различных моделей;
\item
  в) выбор лучшей прогностической модели;
\item
  г) ансамблевый прогноз.
\end{itemize}

Входные данные для работы скрипта:
\href{https://mombus.github.io/cRab/data/RECRUITMENЕNT.xlsx}{RECRUITMENT.xlsx},
а также промежуточный файл с готовым набором предикторов:
\href{https://mombus.github.io/cRab/data/selected_predictors_dataset.csv}{selected\_predictors\_dataset.csv}.

\section{Выбор
предикторов}\label{ux432ux44bux431ux43eux440-ux43fux440ux435ux434ux438ux43aux442ux43eux440ux43eux432}

В процессе анализа факторов, влияющих на пополнение запасов
гидробионтов, важным этапом является тщательная подготовка данных и
отбор наиболее информативных предикторов, поскольку качество последующих
моделей напрямую зависит от качества входных данных. Начиная с первичной
обработки, мы приводим все потенциальные предикторы к числовому формату,
так как большинство статистических и машинно-обучаемых моделей требуют
именно такого представления данных, при этом заменяем строковые
обозначения пропущенных значений «NA» на стандартные NA, что позволяет
системе R корректно обрабатывать отсутствующие наблюдения. Для
заполнения пропусков мы применяем медианную импутацию, которая
представляет собой простой и устойчивый к выбросам метод, поскольку
медиана менее чувствительна к экстремальным значениям по сравнению со
средним. Хотя существуют и более сложные альтернативы, такие как
множественная импутация с использованием пакета mice, KNN-импутация
через recipes::step\_impute\_knn или даже методы, специально
разработанные для временных рядов, например, фильтр Калмана или
ARIMA-модели, медианная импутация остается практичным выбором для
начального этапа анализа, особенно когда объем данных ограничен или
временные зависимости не являются доминирующими. Следующим важным этапом
является анализ корреляционной структуры данных, поскольку высокая
мультиколлинеарность между предикторами может серьезно ухудшить
интерпретацию моделей и завысить дисперсию оценок параметров, особенно в
линейных моделях. Для автоматического выявления и устранения сильно
коррелированных переменных мы используем функцию findCorrelation с
пороговым значением коэффициента корреляции 0.8, что позволяет сохранить
лишь один представитель из каждой группы высококоррелированных
переменных. Хотя альтернативными подходами могут служить диагностика по
значениям VIF или применение методов снижения размерности, таких как PLS
или PCA, удаление явно коррелированных предикторов оказывается наиболее
прямолинейным решением для обеспечения стабильности последующих моделей.
Для автоматического отбора наиболее значимых предикторов мы применяем
два дополнительных метода, которые по-разному подходят к этой задаче и
тем самым обеспечивают взаимную проверку результатов. Boruta
представляет собой обертку алгоритма Random Forest, которая генерирует
«теневые» переменные, полученные путем случайного перемешивания исходных
признаков, и сравнивает важность реальных предикторов с этими теневыми
копиями, сохраняя только те переменные, чья важность статистически
превосходит уровень шума. Этот метод особенно эффективен при наличии
нелинейных зависимостей и взаимодействий между переменными, демонстрируя
высокую устойчивость к шуму, хотя и требует больше вычислительных
ресурсов и может излишне благоволить к группам коррелированных
признаков. Параллельно мы применяем LASSO-регрессию из пакета glmnet,
которая использует L1-регуляризацию для зануления коэффициентов слабо
влияющих предикторов, тем самым выполняет отбор признаков в процессе
оценки модели. При выборе оптимального значения параметра регуляризации
lambda мы сознательно предпочитаем значение lambda.1se, которое
соответствует более простой модели, но при этом находится в пределах
одной стандартной ошибки от минимального значения ошибки, так как этот
консервативный подход часто обеспечивает лучшую обобщающую способность
на небольших выборках, характерных для экологических данных. Однако
LASSO имеет свои ограничения: он чувствителен к масштабу переменных, что
делает центрирование и стандартизацию обязательными предварительными
шагами, и предполагает линейную форму зависимости между предикторами и
откликом, что может не соответствовать реальной биологической природе
процессов. Финальный набор предикторов формируется как объединение
результатов Boruta и LASSO с учетом биологической логики, что повышает
устойчивость отбора к случайным флуктуациям, присущим каждому отдельному
методу, и гарантирует включение ключевых переменных, таких как
нерестовый запас (haddock68), который биологически должен влиять на
пополнение запаса. Для предварительной проверки значимости отобранных
предикторов мы строим простую линейную модель, которая не предназначена
для окончательного прогноза, но служит в качестве sanity-check, позволяя
оценить порядок величины эффектов и выявить явно незначимые или
противоречащие биологической логике переменные. Важно отметить несколько
нюансов и потенциальных подводных камней, с которыми можно столкнуться
на этом этапе: если распределение целевой переменной R3haddock сильно
скошено, может потребоваться лог-трансформация или использование
моделей, специально разработанных для положительных откликов, таких как
Gamma GLM; корреляция между переменными не обязательно отражает
причинно-следственные связи, и при удалении высококоррелированных
предикторов мы можем потерять полезную информацию, поэтому в некоторых
случаях лучше применять методы, сохраняющие информацию из всех
переменных, например, PLS или GAM; наконец, медианная импутация, хотя и
проста в применении, может быть недостаточно точной для временных рядов,
где хронологически осмысленная импутация, такая как скользящая медиана
или интерполяция, часто дает более реалистичные результаты, учитывающие
естественную динамику экологических процессов. Таким образом, этап
подготовки данных и отбора предикторов представляет собой критически
важный фундамент для последующего построения качественных моделей
прогнозирования пополнения рыбных запасов, где баланс между
статистической строгостью и биологической интерпретируемостью определяет
успех всего анализа.

Скрипт целиком можно скачать по
\href{https://mombus.github.io/cRab/data/RECRUITMENT_PREDICTORS.R}{ссылке}

\begin{Shaded}
\begin{Highlighting}[]
\CommentTok{\# ==============================================================================}
\CommentTok{\# 1) ВЫБОР ПРЕДИКТОРОВ}
\CommentTok{\# {-}{-}{-}{-}{-}{-}{-}{-}{-}{-}{-}{-}{-}{-}{-}{-}{-}{-}{-}{-}{-}{-}{-}{-}{-}{-}{-}{-}{-}{-}{-}{-}{-}{-}{-}{-}{-}{-}{-}{-}{-}{-}{-}{-}{-}{-}{-}{-}{-}{-}{-}{-}{-}{-}{-}{-}{-}{-}{-}{-}{-}{-}{-}{-}{-}{-}{-}{-}{-}{-}{-}{-}{-}{-}{-}{-}{-}{-}}
\CommentTok{\# Цель блока: привести данные к числовому виду, обработать пропуски, сократить}
\CommentTok{\# мультиколлинеарность (сильные корреляции), а затем автоматически выделить}
\CommentTok{\# кандидатов{-}предикторов двумя методами (Boruta, LASSO). В конце сформируем}
\CommentTok{\# финальный пул признаков и проверим их значимость в простой LM.}
\CommentTok{\# ==============================================================================}

\CommentTok{\# Установка и подключение необходимых библиотек}
\CommentTok{\# Для автоматического отбора предикторов нам понадобятся дополнительные пакеты}
\ControlFlowTok{if}\NormalTok{ (}\SpecialCharTok{!}\FunctionTok{require}\NormalTok{(}\StringTok{"pacman"}\NormalTok{)) }\FunctionTok{install.packages}\NormalTok{(}\StringTok{"pacman"}\NormalTok{)}
\end{Highlighting}
\end{Shaded}

\begin{verbatim}
Загрузка требуемого пакета: pacman
\end{verbatim}

\begin{Shaded}
\begin{Highlighting}[]
\NormalTok{pacman}\SpecialCharTok{::}\FunctionTok{p\_load}\NormalTok{(}
\NormalTok{  readxl, tidyverse, caret, corrplot, mgcv, randomForest, xgboost,}
\NormalTok{  Boruta,GGally, FactoMineR, glmnet, recipes, rsample  }\CommentTok{\# Новые библиотеки для автоматического отбора}
\NormalTok{)}

\CommentTok{\# Очистка среды и установка рабочей директории}
\CommentTok{\# Совет: rm(list=ls()) очищает все объекты в памяти R; setwd задаёт папку,}
\CommentTok{\# где искать/сохранять файлы. Убедитесь, что путь корректен на вашей машине.}
\FunctionTok{rm}\NormalTok{(}\AttributeTok{list =} \FunctionTok{ls}\NormalTok{())}
\FunctionTok{setwd}\NormalTok{(}\StringTok{"C:/RECRUITMENT/"}\NormalTok{)}

\CommentTok{\# Пакеты для расширенного отбора предикторов}
\CommentTok{\# Boruta — обёртка над Random Forest для отбора признаков;}
\CommentTok{\# glmnet — регуляризация (LASSO/ElasticNet) для отбора/усиления обобщающей способности;}
\CommentTok{\# FactoMineR — PCA и другие многомерные методы (используем как утилиту).}
\FunctionTok{library}\NormalTok{(Boruta)   }\CommentTok{\# Алгоритм обертки для отбора признаков}
\FunctionTok{library}\NormalTok{(glmnet)   }\CommentTok{\# LASSO{-}регрессия}
\FunctionTok{library}\NormalTok{(FactoMineR) }\CommentTok{\# PCA анализ}


\CommentTok{\# Загрузка и первичная обработка данных}
\CommentTok{\# Шаги: фильтруем годы, приводим типы к числовому, заменяем строковые "NA" на NA.}
\NormalTok{DATA }\OtherTok{\textless{}{-}}\NormalTok{ readxl}\SpecialCharTok{::}\FunctionTok{read\_excel}\NormalTok{(}\StringTok{"RECRUITMENT.xlsx"}\NormalTok{, }\AttributeTok{sheet =} \StringTok{"RECRUITMENT"}\NormalTok{) }\SpecialCharTok{\%\textgreater{}\%}
  \FunctionTok{filter}\NormalTok{(YEAR }\SpecialCharTok{\textgreater{}} \DecValTok{1989} \SpecialCharTok{\&}\NormalTok{ YEAR }\SpecialCharTok{\textless{}} \DecValTok{2022}\NormalTok{) }\SpecialCharTok{\%\textgreater{}\%}
  \CommentTok{\# Преобразуем необходимые столбцы в числовой формат}
  \FunctionTok{mutate}\NormalTok{(}
    \FunctionTok{across}\NormalTok{(}\FunctionTok{starts\_with}\NormalTok{(}\StringTok{"T"}\NormalTok{), as.numeric),}
    \FunctionTok{across}\NormalTok{(}\FunctionTok{starts\_with}\NormalTok{(}\StringTok{"I"}\NormalTok{), as.numeric),}
    \FunctionTok{across}\NormalTok{(}\FunctionTok{starts\_with}\NormalTok{(}\StringTok{"O"}\NormalTok{), as.numeric),}
\NormalTok{  ) }\SpecialCharTok{\%\textgreater{}\%}
  \CommentTok{\# Обработка пропущенных значений (заменяем строку "NA" на NA)}
  \FunctionTok{mutate}\NormalTok{(}\FunctionTok{across}\NormalTok{(}\FunctionTok{where}\NormalTok{(is.character), }\SpecialCharTok{\textasciitilde{}}\FunctionTok{na\_if}\NormalTok{(., }\StringTok{"NA"}\NormalTok{)))}

\CommentTok{\# 1. Подготовка данных {-}{-}{-}{-}{-}{-}{-}{-}{-}{-}{-}{-}{-}{-}{-}{-}{-}{-}{-}{-}{-}{-}{-}{-}{-}{-}{-}{-}{-}{-}{-}{-}{-}{-}{-}{-}{-}{-}{-}{-}{-}{-}{-}{-}{-}{-}{-}{-}{-}{-}{-}{-}{-}{-}{-}}
\CommentTok{\# Выделим все возможные предикторы, включая географию и индексы трески}
\CommentTok{\# Примечание: оставляем только числовые переменные, т.к. большинство моделей}
\CommentTok{\# требует числовой вход без категориальных уровней.}
\NormalTok{predictors }\OtherTok{\textless{}{-}}\NormalTok{ DATA }\SpecialCharTok{\%\textgreater{}\%} 
  \FunctionTok{select}\NormalTok{(}\SpecialCharTok{{-}}\NormalTok{YEAR, }\SpecialCharTok{{-}}\NormalTok{R3haddock) }\SpecialCharTok{\%\textgreater{}\%} 
  \FunctionTok{select\_if}\NormalTok{(is.numeric) }\CommentTok{\# Только числовые переменные}

\CommentTok{\# Целевая переменная}
\NormalTok{response }\OtherTok{\textless{}{-}}\NormalTok{ DATA}\SpecialCharTok{$}\NormalTok{R3haddock}

\CommentTok{\# В статистическом анализе мы различаем:}
\CommentTok{\# {-} Отклик (response/target variable) {-} то, что мы пытаемся предсказать (в нашем случае R3haddock)}
\CommentTok{\# {-} Предикторы (predictors/features) {-} переменные, которые могут объяснять изменения отклика}
\CommentTok{\# Для корректного анализа важно, чтобы предикторы были числовыми или преобразованы в числовой формат.}

\CommentTok{\# 2. Обработка пропусков {-}{-}{-}{-}{-}{-}{-}{-}{-}{-}{-}{-}{-}{-}{-}{-}{-}{-}{-}{-}{-}{-}{-}{-}{-}{-}{-}{-}{-}{-}{-}{-}{-}{-}{-}{-}{-}{-}{-}{-}{-}{-}{-}{-}{-}{-}{-}{-}{-}{-}{-}{-}{-}}
\CommentTok{\# Заполнение медианными значениями — простой и устойчивый способ справиться с NA.}
\CommentTok{\# Альтернативы: множественная иммутация (mice), KNN{-}impute и др.}
\NormalTok{predictors\_filled }\OtherTok{\textless{}{-}}\NormalTok{ predictors }\SpecialCharTok{\%\textgreater{}\%}
  \FunctionTok{mutate}\NormalTok{(}\FunctionTok{across}\NormalTok{(}\FunctionTok{everything}\NormalTok{(), }\SpecialCharTok{\textasciitilde{}}\FunctionTok{ifelse}\NormalTok{(}\FunctionTok{is.na}\NormalTok{(.), }\FunctionTok{median}\NormalTok{(., }\AttributeTok{na.rm =} \ConstantTok{TRUE}\NormalTok{), .)))}

\CommentTok{\# Заполнение медианой {-} простой и устойчивый метод обработки пропусков для числовых переменных.}
\CommentTok{\# Медиана предпочтительнее среднего, так как менее чувствительна к выбросам.}

\CommentTok{\# 3. Предварительный анализ корреляций {-}{-}{-}{-}{-}{-}{-}{-}{-}{-}{-}{-}{-}{-}{-}{-}{-}{-}{-}{-}{-}{-}{-}{-}{-}{-}{-}{-}{-}{-}{-}{-}{-}{-}{-}{-}{-}{-}{-}}
\CommentTok{\# Зачем: высокие корреляции затрудняют интерпретацию и могут вредить ряду моделей.}
\NormalTok{cor\_matrix }\OtherTok{\textless{}{-}} \FunctionTok{cor}\NormalTok{(predictors\_filled, }\AttributeTok{use =} \StringTok{"complete.obs"}\NormalTok{)}
\FunctionTok{corrplot}\NormalTok{(cor\_matrix, }\AttributeTok{method =} \StringTok{"circle"}\NormalTok{, }\AttributeTok{type =} \StringTok{"upper"}\NormalTok{, }\AttributeTok{tl.cex =} \FloatTok{0.7}\NormalTok{)}
\end{Highlighting}
\end{Shaded}

\pandocbounded{\includegraphics[keepaspectratio]{chapter7_files/figure-pdf/unnamed-chunk-1-1.pdf}}

\begin{Shaded}
\begin{Highlighting}[]
\CommentTok{\# Удаляем высокоскоррелированные предикторы (r \textgreater{} 0.8)}
\CommentTok{\# Это механическое сокращение мультиколлинеарности до этапа отбора.}
\NormalTok{high\_cor }\OtherTok{\textless{}{-}} \FunctionTok{findCorrelation}\NormalTok{(cor\_matrix, }\AttributeTok{cutoff =} \FloatTok{0.8}\NormalTok{)}
\NormalTok{predictors\_filtered }\OtherTok{\textless{}{-}}\NormalTok{ predictors\_filled[, }\SpecialCharTok{{-}}\NormalTok{high\_cor]}

\CommentTok{\# Высокая корреляция между предикторами (мультиколлинеарность) может привести к нестабильности моделей.}
\CommentTok{\# Например, если два предиктора почти идентичны, модель может неустойчиво распределять их влияние на отклик.}
\CommentTok{\# Удаление сильно коррелированных переменных (r \textgreater{} 0.8) помогает улучшить интерпретируемость и стабильность моделей.}


\CommentTok{\# 4. Автоматизированный отбор Boruta (обертка Random Forest) {-}{-}{-}{-}{-}{-}{-}{-}{-}{-}{-}{-}{-}{-}{-}{-}{-}}
\CommentTok{\# Идея: определить признаки, которые важнее, чем случайный шум (shadow features).}
\end{Highlighting}
\end{Shaded}

\begin{Shaded}
\begin{Highlighting}[]
\CommentTok{\# Визуализация результатов}
\FunctionTok{plot}\NormalTok{(boruta\_output, }\AttributeTok{cex.axis =} \FloatTok{0.7}\NormalTok{, }\AttributeTok{las =} \DecValTok{2}\NormalTok{)}
\end{Highlighting}
\end{Shaded}

\pandocbounded{\includegraphics[keepaspectratio]{chapter7_files/figure-pdf/unnamed-chunk-2-1.pdf}}

\begin{Shaded}
\begin{Highlighting}[]
\NormalTok{boruta\_stats }\OtherTok{\textless{}{-}} \FunctionTok{attStats}\NormalTok{(boruta\_output)}
\NormalTok{selected\_vars }\OtherTok{\textless{}{-}} \FunctionTok{getSelectedAttributes}\NormalTok{(boruta\_output, }\AttributeTok{withTentative =} \ConstantTok{TRUE}\NormalTok{)}

\CommentTok{\# Boruta {-} это алгоритм отбора признаков, основанный на методе случайного леса.}
\CommentTok{\# Он сравнивает важность реальных переменных с "теневыми" переменными (случайными копиями),}
\CommentTok{\# чтобы определить, действительно ли переменная информативна.}
\CommentTok{\# Результаты Boruta показывают: }
\CommentTok{\#   {-} Confirmed (зеленые) {-} значимые предикторы}
\CommentTok{\#   {-} Tentative (желтые) {-} предикторы, близкие к порогу значимости}
\CommentTok{\#   {-} Rejected (красные) {-} незначимые предикторы}


\CommentTok{\# 5. LASSO с более строгим критерием {-}{-}{-}{-}{-}{-}{-}{-}{-}{-}{-}{-}{-}{-}{-}{-}{-}{-}{-}{-}{-}{-}{-}{-}{-}{-}{-}{-}{-}{-}{-}{-}{-}{-}{-}{-}{-}{-}{-}{-}{-}{-}}
\CommentTok{\# Идея: L1{-}регуляризация зануляет коэффициенты «слабых» предикторов.}
\CommentTok{\# Выбор lambda.1se вместо lambda.min — более консервативный (простая модель).}
\NormalTok{x }\OtherTok{\textless{}{-}} \FunctionTok{as.matrix}\NormalTok{(predictors\_filtered)}
\NormalTok{y }\OtherTok{\textless{}{-}}\NormalTok{ response}

\CommentTok{\# LASSO (Least Absolute Shrinkage and Selection Operator) {-} метод регрессии с L1{-}регуляризацией,}
\CommentTok{\# который одновременно выполняет отбор признаков и оценку коэффициентов. [[8]]}
\CommentTok{\# Параметр lambda контролирует силу регуляризации:}
\CommentTok{\#   {-} lambda.min дает наименьшую ошибку, но может включать шумовые переменные}
\CommentTok{\#   {-} lambda.1se (на 1 стандартную ошибку больше) дает более простую модель с меньшим риском переобучения}
\CommentTok{\# Для прогнозирования мы предпочитаем более строгий критерий (lambda.1se), чтобы модель была устойчивее. [[1]]}

\CommentTok{\# Кросс{-}валидация}
\NormalTok{cv\_fit }\OtherTok{\textless{}{-}} \FunctionTok{cv.glmnet}\NormalTok{(x, y, }\AttributeTok{alpha =} \DecValTok{1}\NormalTok{, }\AttributeTok{nfolds =} \DecValTok{10}\NormalTok{)}
\FunctionTok{plot}\NormalTok{(cv\_fit)}
\end{Highlighting}
\end{Shaded}

\pandocbounded{\includegraphics[keepaspectratio]{chapter7_files/figure-pdf/unnamed-chunk-2-2.pdf}}

\begin{Shaded}
\begin{Highlighting}[]
\CommentTok{\# ИСПОЛЬЗУЕМ lambda.1se вместо lambda.min — СТРОЖЕ!}
\NormalTok{lasso\_coef }\OtherTok{\textless{}{-}} \FunctionTok{coef}\NormalTok{(cv\_fit, }\AttributeTok{s =} \StringTok{"lambda.1se"}\NormalTok{)  }\CommentTok{\# \textless{}{-}{-} Ключевое изменение!}
\NormalTok{lasso\_vars }\OtherTok{\textless{}{-}} \FunctionTok{rownames}\NormalTok{(lasso\_coef)[lasso\_coef[,}\DecValTok{1}\NormalTok{] }\SpecialCharTok{!=} \DecValTok{0}\NormalTok{][}\SpecialCharTok{{-}}\DecValTok{1}\NormalTok{]  }\CommentTok{\# исключаем (Intercept)}


\CommentTok{\# 6. Сравнение отобранных предикторов {-}{-}{-}{-}{-}{-}{-}{-}{-}{-}{-}{-}{-}{-}{-}{-}{-}{-}{-}{-}{-}{-}{-}{-}{-}{-}{-}{-}{-}{-}{-}{-}{-}{-}{-}{-}{-}{-}{-}{-}}
\CommentTok{\# Полезно видеть, какие признаки отмечают оба метода (устойчивые кандидаты).}
\FunctionTok{cat}\NormalTok{(}\StringTok{"Boruta selected:"}\NormalTok{, }\FunctionTok{length}\NormalTok{(selected\_vars), }\StringTok{"variables}\SpecialCharTok{\textbackslash{}n}\StringTok{"}\NormalTok{)}
\end{Highlighting}
\end{Shaded}

\begin{verbatim}
Boruta selected: 3 variables
\end{verbatim}

\begin{Shaded}
\begin{Highlighting}[]
\FunctionTok{print}\NormalTok{(selected\_vars)}
\end{Highlighting}
\end{Shaded}

\begin{verbatim}
[1] "codTSB" "T12"    "I5"    
\end{verbatim}

\begin{Shaded}
\begin{Highlighting}[]
\FunctionTok{cat}\NormalTok{(}\StringTok{"}\SpecialCharTok{\textbackslash{}n}\StringTok{LASSO selected:"}\NormalTok{, }\FunctionTok{length}\NormalTok{(lasso\_vars), }\StringTok{"variables}\SpecialCharTok{\textbackslash{}n}\StringTok{"}\NormalTok{)}
\end{Highlighting}
\end{Shaded}

\begin{verbatim}

LASSO selected: 5 variables
\end{verbatim}

\begin{Shaded}
\begin{Highlighting}[]
\FunctionTok{print}\NormalTok{(lasso\_vars)}
\end{Highlighting}
\end{Shaded}

\begin{verbatim}
[1] "codTSB" "T12"    "NAO3"   "NAO4"   "NAO5"  
\end{verbatim}

\begin{Shaded}
\begin{Highlighting}[]
\CommentTok{\# 7. Финальный набор предикторов (объединение результатов) {-}{-}{-}{-}{-}{-}{-}{-}{-}{-}{-}{-}{-}{-}{-}{-}{-}{-}{-}}
\CommentTok{\# Логика: объединяем списки, добавляем биологически важные переменные вручную.}
\NormalTok{final\_vars }\OtherTok{\textless{}{-}} \FunctionTok{union}\NormalTok{(selected\_vars, lasso\_vars) }

\CommentTok{\# Добавляем обязательные переменные по биологической логике}
\NormalTok{mandatory }\OtherTok{\textless{}{-}} \FunctionTok{c}\NormalTok{(}\StringTok{"haddock68"}\NormalTok{)}
\NormalTok{final\_vars }\OtherTok{\textless{}{-}} \FunctionTok{union}\NormalTok{(final\_vars, mandatory) }\SpecialCharTok{\%\textgreater{}\%} \FunctionTok{unique}\NormalTok{()}

\CommentTok{\# Мы объединяем результаты двух методов отбора признаков для большей надежности.}
\CommentTok{\# Также добавляем переменную haddock68 (нерестовый запас), так как биологически }
\CommentTok{\# логично, что пополнение запаса напрямую зависит от численности производителей. }
\CommentTok{\# Это пример интеграции экспертных знаний в статистический анализ {-} важный принцип }
\CommentTok{\# при работе с данными в биологических науках.}

\CommentTok{\# 8. Проверка значимости {-}{-}{-}{-}{-}{-}{-}{-}{-}{-}{-}{-}{-}{-}{-}{-}{-}{-}{-}{-}{-}{-}{-}{-}{-}{-}{-}{-}{-}{-}{-}{-}{-}{-}{-}{-}{-}{-}{-}{-}{-}{-}{-}{-}{-}{-}{-}{-}{-}{-}{-}{-}{-}}
\CommentTok{\# Быстрая оценка значимости с LM: не как окончательный вывод, а как sanity{-}check.}
\NormalTok{final\_model }\OtherTok{\textless{}{-}} \FunctionTok{lm}\NormalTok{(response }\SpecialCharTok{\textasciitilde{}} \FunctionTok{as.matrix}\NormalTok{(predictors\_filled[, final\_vars]))}
\FunctionTok{summary}\NormalTok{(final\_model)}
\end{Highlighting}
\end{Shaded}

\begin{verbatim}

Call:
lm(formula = response ~ as.matrix(predictors_filled[, final_vars]))

Residuals:
    Min      1Q  Median      3Q     Max 
-270986  -82376   -1037   98086  276129 

Coefficients:
                                                      Estimate Std. Error
(Intercept)                                         -1.082e+06  3.943e+05
as.matrix(predictors_filled[, final_vars])codTSB    -2.346e-01  5.536e-02
as.matrix(predictors_filled[, final_vars])T12        3.864e+05  7.198e+04
as.matrix(predictors_filled[, final_vars])I5        -1.825e+02  2.572e+03
as.matrix(predictors_filled[, final_vars])NAO3      -5.801e+04  3.129e+04
as.matrix(predictors_filled[, final_vars])NAO4       8.345e+04  3.035e+04
as.matrix(predictors_filled[, final_vars])NAO5      -7.278e+04  2.488e+04
as.matrix(predictors_filled[, final_vars])haddock68  1.232e-01  4.515e-01
                                                    t value Pr(>|t|)    
(Intercept)                                          -2.744 0.011305 *  
as.matrix(predictors_filled[, final_vars])codTSB     -4.238 0.000288 ***
as.matrix(predictors_filled[, final_vars])T12         5.368 1.64e-05 ***
as.matrix(predictors_filled[, final_vars])I5         -0.071 0.944028    
as.matrix(predictors_filled[, final_vars])NAO3       -1.854 0.076118 .  
as.matrix(predictors_filled[, final_vars])NAO4        2.750 0.011146 *  
as.matrix(predictors_filled[, final_vars])NAO5       -2.925 0.007412 ** 
as.matrix(predictors_filled[, final_vars])haddock68   0.273 0.787227    
---
Signif. codes:  0 '***' 0.001 '**' 0.01 '*' 0.05 '.' 0.1 ' ' 1

Residual standard error: 158600 on 24 degrees of freedom
Multiple R-squared:  0.7173,    Adjusted R-squared:  0.6348 
F-statistic: 8.698 on 7 and 24 DF,  p-value: 2.58e-05
\end{verbatim}

\begin{Shaded}
\begin{Highlighting}[]
\CommentTok{\# 9. Формирование финального датасета {-}{-}{-}{-}{-}{-}{-}{-}{-}{-}{-}{-}{-}{-}{-}{-}{-}{-}{-}{-}{-}{-}{-}{-}{-}{-}{-}{-}{-}{-}{-}{-}{-}{-}{-}{-}{-}{-}{-}{-}}
\CommentTok{\# Собираем набор с откликом и выбранными предикторами; удалим строки с NA.}
\NormalTok{model\_data }\OtherTok{\textless{}{-}}\NormalTok{ DATA }\SpecialCharTok{\%\textgreater{}\%}
  \FunctionTok{select}\NormalTok{(R3haddock, }\FunctionTok{all\_of}\NormalTok{(final\_vars)) }\SpecialCharTok{\%\textgreater{}\%}
  \FunctionTok{drop\_na}\NormalTok{()}

\CommentTok{\# Просмотр структуры финальных данных}
\FunctionTok{glimpse}\NormalTok{(model\_data)}
\end{Highlighting}
\end{Shaded}

\begin{verbatim}
Rows: 32
Columns: 8
$ R3haddock <dbl> 812363, 389416, 99474, 98946, 118812, 63028, 147657, 83270, ~
$ codTSB    <dbl> 913000, 1347064, 1687381, 2197863, 2112773, 1849957, 1697388~
$ T12       <dbl> 4.72, 4.66, 4.24, 3.90, 3.96, 4.27, 4.16, 4.07, 4.23, 5.08, ~
$ I5        <dbl> 43, 55, 26, 49, 56, 28, 52, 51, 69, 68, 41, 48, 50, 63, 40, ~
$ NAO3      <dbl> 1.46, -0.20, 0.87, 0.67, 1.26, 1.25, -0.24, 1.46, 0.87, 0.23~
$ NAO4      <dbl> 2.00, 0.29, 1.86, 0.97, 1.14, -0.85, -0.17, -1.02, -0.68, -0~
$ NAO5      <dbl> -1.53, 0.08, 2.63, -0.78, -0.57, -1.49, -1.06, -0.28, -1.32,~
$ haddock68 <dbl> 74586, 79205, 53195, 36337, 49122, 81514, 172177, 160886, 96~
\end{verbatim}

\begin{Shaded}
\begin{Highlighting}[]
\CommentTok{\# Визуализация важности переменных}
\CommentTok{\# Внимание: важности от RF — относительные; сопоставляйте с предметной логикой.}
\NormalTok{var\_importance }\OtherTok{\textless{}{-}} \FunctionTok{randomForest}\NormalTok{(R3haddock }\SpecialCharTok{\textasciitilde{}}\NormalTok{ ., }\AttributeTok{data =}\NormalTok{ model\_data, }\AttributeTok{importance =} \ConstantTok{TRUE}\NormalTok{)}
\FunctionTok{varImpPlot}\NormalTok{(var\_importance, }\AttributeTok{main =} \StringTok{"Важность предикторов"}\NormalTok{)}
\end{Highlighting}
\end{Shaded}

\begin{verbatim}
Warning in mtext(outer = TRUE, side = 3, text = main, cex = 1.2): неизвестна
ширина символа 0xc2 в кодировке CP1251
\end{verbatim}

\begin{verbatim}
Warning in mtext(outer = TRUE, side = 3, text = main, cex = 1.2): неизвестна
ширина символа 0xe0 в кодировке CP1251
\end{verbatim}

\begin{verbatim}
Warning in mtext(outer = TRUE, side = 3, text = main, cex = 1.2): неизвестна
ширина символа 0xe6 в кодировке CP1251
\end{verbatim}

\begin{verbatim}
Warning in mtext(outer = TRUE, side = 3, text = main, cex = 1.2): неизвестна
ширина символа 0xed в кодировке CP1251
\end{verbatim}

\begin{verbatim}
Warning in mtext(outer = TRUE, side = 3, text = main, cex = 1.2): неизвестна
ширина символа 0xee в кодировке CP1251
\end{verbatim}

\begin{verbatim}
Warning in mtext(outer = TRUE, side = 3, text = main, cex = 1.2): неизвестна
ширина символа 0xf1 в кодировке CP1251
\end{verbatim}

\begin{verbatim}
Warning in mtext(outer = TRUE, side = 3, text = main, cex = 1.2): неизвестна
ширина символа 0xf2 в кодировке CP1251
\end{verbatim}

\begin{verbatim}
Warning in mtext(outer = TRUE, side = 3, text = main, cex = 1.2): неизвестна
ширина символа 0xfc в кодировке CP1251
\end{verbatim}

\begin{verbatim}
Warning in mtext(outer = TRUE, side = 3, text = main, cex = 1.2): неизвестна
ширина символа 0xef в кодировке CP1251
\end{verbatim}

\begin{verbatim}
Warning in mtext(outer = TRUE, side = 3, text = main, cex = 1.2): неизвестна
ширина символа 0xf0 в кодировке CP1251
\end{verbatim}

\begin{verbatim}
Warning in mtext(outer = TRUE, side = 3, text = main, cex = 1.2): неизвестна
ширина символа 0xe5 в кодировке CP1251
\end{verbatim}

\begin{verbatim}
Warning in mtext(outer = TRUE, side = 3, text = main, cex = 1.2): неизвестна
ширина символа 0xe4 в кодировке CP1251
\end{verbatim}

\begin{verbatim}
Warning in mtext(outer = TRUE, side = 3, text = main, cex = 1.2): неизвестна
ширина символа 0xe8 в кодировке CP1251
\end{verbatim}

\begin{verbatim}
Warning in mtext(outer = TRUE, side = 3, text = main, cex = 1.2): неизвестна
ширина символа 0xea в кодировке CP1251
\end{verbatim}

\begin{verbatim}
Warning in mtext(outer = TRUE, side = 3, text = main, cex = 1.2): неизвестна
ширина символа 0xf2 в кодировке CP1251
\end{verbatim}

\begin{verbatim}
Warning in mtext(outer = TRUE, side = 3, text = main, cex = 1.2): неизвестна
ширина символа 0xee в кодировке CP1251
\end{verbatim}

\begin{verbatim}
Warning in mtext(outer = TRUE, side = 3, text = main, cex = 1.2): неизвестна
ширина символа 0xf0 в кодировке CP1251
\end{verbatim}

\begin{verbatim}
Warning in mtext(outer = TRUE, side = 3, text = main, cex = 1.2): неизвестна
ширина символа 0xee в кодировке CP1251
\end{verbatim}

\begin{verbatim}
Warning in mtext(outer = TRUE, side = 3, text = main, cex = 1.2): неизвестна
ширина символа 0xe2 в кодировке CP1251
\end{verbatim}

\pandocbounded{\includegraphics[keepaspectratio]{chapter7_files/figure-pdf/unnamed-chunk-2-3.pdf}}

\begin{Shaded}
\begin{Highlighting}[]
\CommentTok{\# Перед окончательным выбором модели мы проверяем значимость предикторов с помощью линейной регрессии.}
\CommentTok{\# Функция summary() показывает p{-}значения коэффициентов {-} если p \textless{} 0.05, переменная считается статистически значимой. }
\CommentTok{\# Визуализация важности переменных с помощью случайного леса дает дополнительную перспективу,}
\CommentTok{\# показывая, какие переменные наиболее информативны для предсказания без предположений о линейности.}

\CommentTok{\# ==============================================================================}
\CommentTok{\#  ПОДГОТОВКА ДАННЫХ}
\CommentTok{\# Создаём NAOspring, фиксируем финальный набор признаков, сохраняем CSV.}
\CommentTok{\# {-}{-}{-}{-}{-}{-}{-}{-}{-}{-}{-}{-}{-}{-}{-}{-}{-}{-}{-}{-}{-}{-}{-}{-}{-}{-}{-}{-}{-}{-}{-}{-}{-}{-}{-}{-}{-}{-}{-}{-}{-}{-}{-}{-}{-}{-}{-}{-}{-}{-}{-}{-}{-}{-}{-}{-}{-}{-}{-}{-}{-}{-}{-}{-}{-}{-}{-}{-}{-}{-}{-}{-}{-}{-}{-}{-}{-}{-}}
\CommentTok{\# Цель блока: стандартизировать набор признаков для дальнейшего сравнения}
\CommentTok{\# моделей и обеспечить воспроизводимость (фиксированный CSV с нужными полями).}
\CommentTok{\# ==============================================================================}

\CommentTok{\# 1.1 Пакеты и окружение}
\CommentTok{\# Примечание: блок повторяет базовую инициализацию для автономного запуска.}
\ControlFlowTok{if}\NormalTok{ (}\SpecialCharTok{!}\FunctionTok{require}\NormalTok{(}\StringTok{"pacman"}\NormalTok{)) }\FunctionTok{install.packages}\NormalTok{(}\StringTok{"pacman"}\NormalTok{)}
\NormalTok{pacman}\SpecialCharTok{::}\FunctionTok{p\_load}\NormalTok{(readxl, tidyverse, caret, corrplot)}

\FunctionTok{rm}\NormalTok{(}\AttributeTok{list =} \FunctionTok{ls}\NormalTok{())}
\FunctionTok{set.seed}\NormalTok{(}\DecValTok{123}\NormalTok{)}
\FunctionTok{setwd}\NormalTok{(}\StringTok{"C:/RECRUITMENT/"}\NormalTok{)}

\CommentTok{\# 1.2 Загрузка исходных данных и приведение типов}
\NormalTok{DATA }\OtherTok{\textless{}{-}}\NormalTok{ readxl}\SpecialCharTok{::}\FunctionTok{read\_excel}\NormalTok{(}\StringTok{"RECRUITMENT.xlsx"}\NormalTok{, }\AttributeTok{sheet =} \StringTok{"RECRUITMENT"}\NormalTok{) }\SpecialCharTok{\%\textgreater{}\%}
  \FunctionTok{filter}\NormalTok{(YEAR }\SpecialCharTok{\textgreater{}} \DecValTok{1989} \SpecialCharTok{\&}\NormalTok{ YEAR }\SpecialCharTok{\textless{}} \DecValTok{2022}\NormalTok{) }\SpecialCharTok{\%\textgreater{}\%}
  \FunctionTok{mutate}\NormalTok{(}
    \FunctionTok{across}\NormalTok{(}\FunctionTok{starts\_with}\NormalTok{(}\StringTok{"T"}\NormalTok{), as.numeric),}
    \FunctionTok{across}\NormalTok{(}\FunctionTok{starts\_with}\NormalTok{(}\StringTok{"I"}\NormalTok{), as.numeric),}
    \FunctionTok{across}\NormalTok{(}\FunctionTok{starts\_with}\NormalTok{(}\StringTok{"O"}\NormalTok{), as.numeric),}
    \FunctionTok{across}\NormalTok{(}\FunctionTok{where}\NormalTok{(is.character), }\SpecialCharTok{\textasciitilde{}}\FunctionTok{na\_if}\NormalTok{(., }\StringTok{"NA"}\NormalTok{))}
\NormalTok{  )}

\CommentTok{\# 1.3 Создаём NAOspring (если есть NAO3, NAO4, NAO5)}
\CommentTok{\# Идея: агрегируем весенний индекс NAO как среднее за месяцы 3–5.}
\ControlFlowTok{if}\NormalTok{ (}\FunctionTok{all}\NormalTok{(}\FunctionTok{c}\NormalTok{(}\StringTok{"NAO3"}\NormalTok{,}\StringTok{"NAO4"}\NormalTok{,}\StringTok{"NAO5"}\NormalTok{) }\SpecialCharTok{\%in\%} \FunctionTok{names}\NormalTok{(DATA))) \{}
\NormalTok{  DATA }\OtherTok{\textless{}{-}}\NormalTok{ DATA }\SpecialCharTok{\%\textgreater{}\%}
    \FunctionTok{mutate}\NormalTok{(}\AttributeTok{NAOspring =} \FunctionTok{rowMeans}\NormalTok{(}\FunctionTok{pick}\NormalTok{(NAO3, NAO4, NAO5), }\AttributeTok{na.rm =} \ConstantTok{TRUE}\NormalTok{)) }\SpecialCharTok{\%\textgreater{}\%}
    \FunctionTok{select}\NormalTok{(}\SpecialCharTok{{-}}\NormalTok{NAO3, }\SpecialCharTok{{-}}\NormalTok{NAO4, }\SpecialCharTok{{-}}\NormalTok{NAO5)}
\NormalTok{\}}

\CommentTok{\# NAO (North Atlantic Oscillation) {-} важный климатический индекс, влияющий описывающий изменения атмосферного давления}
\CommentTok{\# над Северной Атлантикой. В частности, он отражает разницу в атмосферном давлении между Исландской депрессией и}
\CommentTok{\# Азорским максимумом. NAO влияет на силу и направление западных ветров, а также на траектории штормов в Северной Атлантике. }
\CommentTok{\# Мы создаем NAOspring как среднее значение за весенние месяцы (марта, апреля, мая),}
\CommentTok{\# так как именно в этот период происходят ключевые процессы, влияющие на нерест трески. }
\CommentTok{\# Создание составных переменных на основе экспертных знаний часто улучшает качество моделей.}

\CommentTok{\# 1.4 Финальный учебный набор предикторов (фиксируем)}
\CommentTok{\# Важно: проверяем присутствие нужных колонок и формируем компактный датасет.}
\NormalTok{needed }\OtherTok{\textless{}{-}} \FunctionTok{c}\NormalTok{(}\StringTok{"codTSB"}\NormalTok{, }\StringTok{"T12"}\NormalTok{, }\StringTok{"I5"}\NormalTok{, }\StringTok{"NAOspring"}\NormalTok{, }\StringTok{"haddock68"}\NormalTok{)}
\FunctionTok{stopifnot}\NormalTok{(}\FunctionTok{all}\NormalTok{(needed }\SpecialCharTok{\%in\%} \FunctionTok{names}\NormalTok{(DATA)))}

\CommentTok{\# Сохраняем YEAR в CSV (ниже он будет отброшен при обучении, но нужен для графика)}
\NormalTok{model\_data }\OtherTok{\textless{}{-}}\NormalTok{ DATA }\SpecialCharTok{\%\textgreater{}\%}
  \FunctionTok{select}\NormalTok{(YEAR, }\FunctionTok{all\_of}\NormalTok{(needed), R3haddock) }\SpecialCharTok{\%\textgreater{}\%}
  \FunctionTok{drop\_na}\NormalTok{()}

\FunctionTok{write.csv}\NormalTok{(model\_data, }\StringTok{"selected\_predictors\_dataset.csv"}\NormalTok{, }\AttributeTok{row.names =} \ConstantTok{FALSE}\NormalTok{)}
\FunctionTok{glimpse}\NormalTok{(model\_data)}
\end{Highlighting}
\end{Shaded}

\begin{verbatim}
Rows: 32
Columns: 7
$ YEAR      <dbl> 1990, 1991, 1992, 1993, 1994, 1995, 1996, 1997, 1998, 1999, ~
$ codTSB    <dbl> 913000, 1347064, 1687381, 2197863, 2112773, 1849957, 1697388~
$ T12       <dbl> 4.72, 4.66, 4.24, 3.90, 3.96, 4.27, 4.16, 4.07, 4.23, 5.08, ~
$ I5        <dbl> 43, 55, 26, 49, 56, 28, 52, 51, 69, 68, 41, 48, 50, 63, 40, ~
$ NAOspring <dbl> 0.64333333, 0.05666667, 1.78666667, 0.28666667, 0.61000000, ~
$ haddock68 <dbl> 74586, 79205, 53195, 36337, 49122, 81514, 172177, 160886, 96~
$ R3haddock <dbl> 812363, 389416, 99474, 98946, 118812, 63028, 147657, 83270, ~
\end{verbatim}

\section{Модели «запас-пополнение» Рикера и
Бивертона-Холта}\label{ux43cux43eux434ux435ux43bux438-ux437ux430ux43fux430ux441-ux43fux43eux43fux43eux43bux43dux435ux43dux438ux435-ux440ux438ux43aux435ux440ux430-ux438-ux431ux438ux432ux435ux440ux442ux43eux43dux430-ux445ux43eux43bux442ux430}

Модели запас-пополнение представляют собой фундаментальный инструмент в
оценке водных биоресурсов, которые гораздо больше, чем просто
математические кривые, --- это формализованные выражения фундаментальных
биологических представлений о том, как численность родительского стада
определяет успех следующего поколения. Среди классических моделей этого
типа наиболее широко используются модель Рикера и модель
Бивертона-Холта, каждая из которых отражает различные гипотезы о
биологических процессах, происходящих в популяции. Модель Рикера,
предложенная Уильямом Рикером в 1954 году и имеющая характерный горб на
графике, выражается уравнением R = a*\emph{S}*exp(-b*S), где
\textbf{\emph{R}} обозначает пополнение, \textbf{\emph{S}} ---
нерестовый запас, а параметры \textbf{\emph{a}} и \textbf{\emph{b}}
имеют четкую биологическую интерпретацию: \textbf{\emph{a}}
соответствует максимальной продуктивности на единицу запаса при очень
низких плотностях, фактически отражая максимальное пополнение
(количество рекрутов) на одного производителя, а \textbf{\emph{b}}
характеризует степень плотностной зависимости, определяющей точку, после
которой начинается снижение из-за внутривидовой конкуренции. Эта модель
предсказывает, что с ростом нерестового запаса пополнение сначала
увеличивается, достигает максимума, а затем снижается, что отражает
явление перенаселенности, когда чрезмерная плотность производителей
приводит к конкуренции за ресурсы, нехватке корма для личинок, усилению
каннибализма или даже эпидемиям, что в итоге снижает выход молоди --- мы
буквально видим, как чрезмерный успех закладывает семена будущего
коллапса пополнения. В отличие от нее, модель Бивертона-Холта,
разработанная в 1957 году, имеет вид R = a\emph{S/(1+b}*S) и
предполагает, что пополнение асимптотически приближается к предельному
значению \textbf{\emph{a/b}}, называемому \textbf{\emph{Rmax}}, с
увеличением нерестового запаса, без последующего снижения, что
соответствует ситуации, когда основной лимитирующий фактор --- это не
внутривидовая конкуренция, а внешние условия: ограниченное количество
нерестовых площадок, хищничество, которое не зависит от плотности, или
просто конечная пропускная способность экосистемы для молоди. Эта модель
идеально описывает сценарий, когда кривая плавно выходит на плато,
символизируя насыщение, и представляет собой альтернативную логику, где
главным ограничивающим фактором являются внешние, а не внутривидовые
процессы.

При оценке параметров этих нелинейных моделей мы сталкиваемся с
необходимостью применения специализированных методов, поскольку обычный
метод наименьших квадратов не справляется с их сложной структурой; в
нашем анализе мы используем улучшенный алгоритм nlsLM из пакета
minpack.lm, который сочетает метод Левенберга-Марквардта с возможностью
наложения ограничений на параметры, что важно для обеспечения
биологической правдоподобности результатов, так как параметры
\textbf{\emph{a}} и \textbf{\emph{b}} должны оставаться положительными.
Для получения надежных начальных оценок параметров в модели Рикера мы
применяем функцию srStarts из пакета FSA, которая автоматически
определяет разумные стартовые значения на основе анализа данных, тогда
как для модели Бивертона-Холта мы используем комбинацию автоматических и
ручных подходов, оценивая a как среднее отношение \textbf{\emph{R/S}}
при низких значениях запаса и устанавливая разумные начальные значения
для \textbf{\emph{b}} с последующей защитой от некорректных значений.
Однако подбор модели --- это только полдела, и критически важно провести
тщательную диагностику, поскольку самая большая ошибка --- слепо
применять эти модели, не задумываясь об их предпосылках. Мы строим
график остатков, потому что любая закономерность в их распределении ---
это сигнал о том, что модель не уловила какой-то важный процесс в
данных. Мы смотрим на доверительные интервалы параметров; если они
невероятно широки, значит, наша модель перепараметризована для имеющихся
данных, и её прогностическая сила будет сомнительной. Модель Рикера не
будет работать, если в вашей системе нет механизма перенаселения, а
модель Бивертона-Холта окажется бесполезной, если пополнение продолжает
расти или, наоборот, обрушивается после достижения пика. Именно поэтому
мы всегда начинаем с простого графика «запас-пополнение» --- его форма
сама подскажет, какая из концепций более адекватна для конкретной
популяции.

Но реальный мир часто бывает сложнее этих двух идеализированных
сценариев. Что если система ведёт себя по-рикеровски при высокой
численности, но при низкой --- работает иначе? Здесь на помощь приходит
модификация --- модель Рикера с порогом, или hockey-stick модель,
которая сочетает в себе линейный рост при малых запасах и плато или спад
при высоких, что может быть биологически более оправдано для многих
запасов, находящихся под прессом промысла. И здесь мы подходим к самому
главному --- интеграции классики и современности. Эти модели не являются
застывшими реликтами, а служат мощным инструментом для создания гипотез.
Если модель Рикера плохо описывает данные, особенно в области низких
значений запаса, это прямой сигнал о том, что возможно, существует
какой-то дополнительный лимитирующий фактор, не учтенный в модели.
Возможно, это температура воды на ключевой стадии развития икры, наличие
хищников или доступность корма. Таким образом, классические модели
становятся трамплином для более сложного анализа, включающего средовые
предикторы. Мы можем включить параметры модели Рикера в качестве
фиксированных эффектов в GAM или использовать предсказания классической
модели в качестве одного из входных признаков для Random Forest. Этот
синтез позволяет нам сохранить биологическую интерпретируемость
классических моделей и добавить к ним гибкость и прогностическую силу
машинного обучения для учета сложных, нелинейных влияний окружающей
среды. В сущности, мы строим мост между глубоким, но узким знанием,
заключенным в одной кривой, и широким, но зачастую ``черно-ящичным''
прогнозом сложного алгоритма, пытаясь получить лучшее из двух миров.
Среди распространенных подводных камней при работе с моделями
запас-пополнение следует отметить высокую чувствительность к начальным
значениям параметров, что может приводить к сходимости к локальным
минимумам, необходимость учета неоднородности дисперсии ошибок, особенно
при работе с данными, охватывающими широкий диапазон значений запаса, и
влияние временных лагов, поскольку пополнение в текущем году может
зависеть не только от нерестового запаса в том же году, но и от условий
прошлых лет. Кроме того, чистые модели запас-пополнение часто
оказываются недостаточными для точного прогнозирования, так как
пополнение зависит не только от размера нерестового запаса, но и от
множества экологических факторов, что делает целесообразным развитие
этих моделей в направлении включения дополнительных предикторов, как это
продемонстрировано в последующих разделах нашего анализа. Тем не менее,
классические модели Рикера и Бивертона-Холта остаются важной отправной
точкой в анализе динамики рыбных популяций, предоставляя
интерпретируемую основу для понимания механизмов регулирования
численности и служа эталоном для оценки добавленной ценности более
сложных моделей, что особенно важно в условиях ограниченных данных,
характерных для многих водных экосистем.

\begin{Shaded}
\begin{Highlighting}[]
\CommentTok{\# ==============================================================================}
\CommentTok{\# ПРАКТИЧЕСКОЕ ЗАНЯТИЕ: АНАЛИЗ ФАКТОРОВ, ВЛИЯЮЩИХ НА ПОПОЛНЕНИЕ }
\CommentTok{\# (КЛАССИЧЕСКИЕ МОДЕЛИ ЗАПАС{-}ПОПОЛНЕНИЕ)}
\CommentTok{\# Курс: "Оценка водных биоресурсов в среде R (для начинающих)"}
\CommentTok{\# ==============================================================================}

\CommentTok{\# Установка и подключение ТОЛЬКО необходимых библиотек}
\ControlFlowTok{if}\NormalTok{ (}\SpecialCharTok{!}\FunctionTok{require}\NormalTok{(}\StringTok{"pacman"}\NormalTok{)) }\FunctionTok{install.packages}\NormalTok{(}\StringTok{"pacman"}\NormalTok{)}
\NormalTok{pacman}\SpecialCharTok{::}\FunctionTok{p\_load}\NormalTok{(}
\NormalTok{  tidyverse,    }\CommentTok{\# Манипуляции с данными и визуализация}
\NormalTok{  FSA,          }\CommentTok{\# Начальные оценки для моделей запас{-}пополнение}
\NormalTok{  minpack.lm,   }\CommentTok{\# Улучшенный алгоритм нелинейной регрессии (nlsLM)}
\NormalTok{  car,          }\CommentTok{\# Проверка допущений моделей}
\NormalTok{  mgcv,         }\CommentTok{\# Построение GAM{-}моделей}
\NormalTok{  investr,      }\CommentTok{\# Доверительные интервалы для нелинейных моделей}
\NormalTok{   caret)       }\CommentTok{\# Расчет RMSE}

\CommentTok{\# Очистка среды и установка рабочей директории}
\FunctionTok{rm}\NormalTok{(}\AttributeTok{list =} \FunctionTok{ls}\NormalTok{())}
\FunctionTok{setwd}\NormalTok{(}\StringTok{"C:/RECRUITMENT/"}\NormalTok{)}

\CommentTok{\# 1. ЗАГРУЗКА ДАННЫХ {-}{-}{-}{-}{-}{-}{-}{-}{-}{-}{-}{-}{-}{-}{-}{-}{-}{-}{-}{-}{-}{-}{-}{-}{-}{-}{-}{-}{-}{-}{-}{-}{-}{-}{-}{-}{-}{-}{-}{-}{-}{-}{-}{-}{-}{-}{-}{-}{-}{-}{-}{-}{-}{-}{-}{-}{-}{-}{-}}
\NormalTok{model\_data }\OtherTok{\textless{}{-}} \FunctionTok{read.csv}\NormalTok{(}\StringTok{"selected\_predictors\_dataset.csv"}\NormalTok{, }
                      \AttributeTok{header =} \ConstantTok{TRUE}\NormalTok{, }
                      \AttributeTok{stringsAsFactors =} \ConstantTok{FALSE}\NormalTok{)}

\CommentTok{\# Проверка структуры данных}
\FunctionTok{str}\NormalTok{(model\_data)}
\end{Highlighting}
\end{Shaded}

\begin{verbatim}
'data.frame':   32 obs. of  7 variables:
 $ YEAR     : int  1990 1991 1992 1993 1994 1995 1996 1997 1998 1999 ...
 $ codTSB   : int  913000 1347064 1687381 2197863 2112773 1849957 1697388 1537459 1350918 1199169 ...
 $ T12      : num  4.72 4.66 4.24 3.9 3.96 4.27 4.16 4.07 4.23 5.08 ...
 $ I5       : int  43 55 26 49 56 28 52 51 69 68 ...
 $ NAOspring: num  0.6433 0.0567 1.7867 0.2867 0.61 ...
 $ haddock68: int  74586 79205 53195 36337 49122 81514 172177 160886 96380 37977 ...
 $ R3haddock: int  812363 389416 99474 98946 118812 63028 147657 83270 359701 386866 ...
\end{verbatim}

\begin{Shaded}
\begin{Highlighting}[]
\CommentTok{\# Проверка на пропущенные значения (должно быть 0 после предобработки)}
\FunctionTok{sum}\NormalTok{(}\FunctionTok{is.na}\NormalTok{(model\_data))}
\end{Highlighting}
\end{Shaded}

\begin{verbatim}
[1] 0
\end{verbatim}

\begin{Shaded}
\begin{Highlighting}[]
\CommentTok{\# 2. ПОДГОТОВКА ДАННЫХ ДЛЯ МОДЕЛЕЙ ЗАПАС{-}ПОПОЛНЕНИЕ {-}{-}{-}{-}{-}{-}{-}{-}{-}{-}{-}{-}{-}{-}{-}{-}{-}{-}{-}{-}{-}{-}{-}{-}{-}{-}{-}{-}}
\NormalTok{rec\_data }\OtherTok{\textless{}{-}} \FunctionTok{data.frame}\NormalTok{(}
  \AttributeTok{S =}\NormalTok{ model\_data}\SpecialCharTok{$}\NormalTok{haddock68,  }\CommentTok{\# Нерестовый запас}
  \AttributeTok{R =}\NormalTok{ model\_data}\SpecialCharTok{$}\NormalTok{R3haddock   }\CommentTok{\# Пополнение}
\NormalTok{)}

\CommentTok{\# 3. ПОДГОНКА МОДЕЛИ РИКЕРА {-}{-}{-}{-}{-}{-}{-}{-}{-}{-}{-}{-}{-}{-}{-}{-}{-}{-}{-}{-}{-}{-}{-}{-}{-}{-}{-}{-}{-}{-}{-}{-}{-}{-}{-}{-}{-}{-}{-}{-}{-}{-}{-}{-}{-}{-}{-}{-}{-}{-}{-}{-}}
\NormalTok{ricker\_starts }\OtherTok{\textless{}{-}}\NormalTok{ FSA}\SpecialCharTok{::}\FunctionTok{srStarts}\NormalTok{(R }\SpecialCharTok{\textasciitilde{}}\NormalTok{ S, }\AttributeTok{data =}\NormalTok{ rec\_data, }\AttributeTok{type =} \StringTok{"Ricker"}\NormalTok{)}
\NormalTok{ricker\_model }\OtherTok{\textless{}{-}}\NormalTok{ minpack.lm}\SpecialCharTok{::}\FunctionTok{nlsLM}\NormalTok{(}
\NormalTok{  R }\SpecialCharTok{\textasciitilde{}}\NormalTok{ a }\SpecialCharTok{*}\NormalTok{ S }\SpecialCharTok{*} \FunctionTok{exp}\NormalTok{(}\SpecialCharTok{{-}}\NormalTok{b }\SpecialCharTok{*}\NormalTok{ S),}
  \AttributeTok{data =}\NormalTok{ rec\_data,}
  \AttributeTok{start =}\NormalTok{ ricker\_starts,}
  \AttributeTok{lower =} \FunctionTok{c}\NormalTok{(}\AttributeTok{a =} \DecValTok{0}\NormalTok{, }\AttributeTok{b =} \DecValTok{0}\NormalTok{)}
\NormalTok{)}

\CommentTok{\# 4. ПОДГОНКА МОДЕЛИ БИВЕРТОНА{-}ХОЛТА {-}{-}{-}{-}{-}{-}{-}{-}{-}{-}{-}{-}{-}{-}{-}{-}{-}{-}{-}{-}{-}{-}{-}{-}{-}{-}{-}{-}{-}{-}{-}{-}{-}{-}{-}{-}{-}{-}{-}{-}{-}{-}{-}}
\NormalTok{a\_start }\OtherTok{\textless{}{-}} \FunctionTok{mean}\NormalTok{(rec\_data}\SpecialCharTok{$}\NormalTok{R[rec\_data}\SpecialCharTok{$}\NormalTok{S }\SpecialCharTok{\textless{}} \FunctionTok{quantile}\NormalTok{(rec\_data}\SpecialCharTok{$}\NormalTok{S, }\FloatTok{0.25}\NormalTok{)] }\SpecialCharTok{/} 
\NormalTok{                rec\_data}\SpecialCharTok{$}\NormalTok{S[rec\_data}\SpecialCharTok{$}\NormalTok{S }\SpecialCharTok{\textless{}} \FunctionTok{quantile}\NormalTok{(rec\_data}\SpecialCharTok{$}\NormalTok{S, }\FloatTok{0.25}\NormalTok{)], }\AttributeTok{na.rm =} \ConstantTok{TRUE}\NormalTok{)}

\ControlFlowTok{if}\NormalTok{ (}\FunctionTok{is.na}\NormalTok{(a\_start) }\SpecialCharTok{||}\NormalTok{ a\_start }\SpecialCharTok{\textless{}=} \DecValTok{0}\NormalTok{) a\_start }\OtherTok{\textless{}{-}} \FloatTok{0.001}

\NormalTok{bh\_model }\OtherTok{\textless{}{-}}\NormalTok{ minpack.lm}\SpecialCharTok{::}\FunctionTok{nlsLM}\NormalTok{(}
\NormalTok{  R }\SpecialCharTok{\textasciitilde{}}\NormalTok{ (a }\SpecialCharTok{*}\NormalTok{ S) }\SpecialCharTok{/}\NormalTok{ (}\DecValTok{1} \SpecialCharTok{+}\NormalTok{ b }\SpecialCharTok{*}\NormalTok{ S),}
  \AttributeTok{data =}\NormalTok{ rec\_data,}
  \AttributeTok{start =} \FunctionTok{list}\NormalTok{(}\AttributeTok{a =}\NormalTok{ a\_start, }\AttributeTok{b =} \FloatTok{0.0001}\NormalTok{),}
  \AttributeTok{lower =} \FunctionTok{c}\NormalTok{(}\AttributeTok{a =} \FloatTok{0.0001}\NormalTok{, }\AttributeTok{b =} \FloatTok{0.00001}\NormalTok{),}
  \AttributeTok{control =} \FunctionTok{nls.lm.control}\NormalTok{(}\AttributeTok{maxiter =} \DecValTok{200}\NormalTok{)}
\NormalTok{)}

\CommentTok{\# 5. ОЦЕНКА КАЧЕСТВА МОДЕЛЕЙ {-}{-}{-}{-}{-}{-}{-}{-}{-}{-}{-}{-}{-}{-}{-}{-}{-}{-}{-}{-}{-}{-}{-}{-}{-}{-}{-}{-}{-}{-}{-}{-}{-}{-}{-}{-}{-}{-}{-}{-}{-}{-}{-}{-}{-}{-}{-}{-}{-}{-}{-}}

\NormalTok{calculate\_R2 }\OtherTok{\textless{}{-}} \ControlFlowTok{function}\NormalTok{(model, data) \{}
\NormalTok{  predicted }\OtherTok{\textless{}{-}} \FunctionTok{predict}\NormalTok{(model, }\AttributeTok{newdata =}\NormalTok{ data)}
\NormalTok{  residuals }\OtherTok{\textless{}{-}}\NormalTok{ data}\SpecialCharTok{$}\NormalTok{R }\SpecialCharTok{{-}}\NormalTok{ predicted}
\NormalTok{  SSE }\OtherTok{\textless{}{-}} \FunctionTok{sum}\NormalTok{(residuals}\SpecialCharTok{\^{}}\DecValTok{2}\NormalTok{)}
\NormalTok{  SST }\OtherTok{\textless{}{-}} \FunctionTok{sum}\NormalTok{((data}\SpecialCharTok{$}\NormalTok{R }\SpecialCharTok{{-}} \FunctionTok{mean}\NormalTok{(data}\SpecialCharTok{$}\NormalTok{R))}\SpecialCharTok{\^{}}\DecValTok{2}\NormalTok{)}
\NormalTok{  R2 }\OtherTok{\textless{}{-}} \DecValTok{1} \SpecialCharTok{{-}}\NormalTok{ (SSE }\SpecialCharTok{/}\NormalTok{ SST)}
\NormalTok{  n }\OtherTok{\textless{}{-}} \FunctionTok{nrow}\NormalTok{(data)}
\NormalTok{  p }\OtherTok{\textless{}{-}} \FunctionTok{length}\NormalTok{(}\FunctionTok{coef}\NormalTok{(model))}
\NormalTok{  adj\_R2 }\OtherTok{\textless{}{-}} \DecValTok{1} \SpecialCharTok{{-}}\NormalTok{ ((n }\SpecialCharTok{{-}} \DecValTok{1}\NormalTok{) }\SpecialCharTok{/}\NormalTok{ (n }\SpecialCharTok{{-}}\NormalTok{ p }\SpecialCharTok{{-}} \DecValTok{1}\NormalTok{)) }\SpecialCharTok{*}\NormalTok{ (}\DecValTok{1} \SpecialCharTok{{-}}\NormalTok{ R2)}
  \FunctionTok{return}\NormalTok{(}\FunctionTok{list}\NormalTok{(}\AttributeTok{R2 =}\NormalTok{ R2, }\AttributeTok{adj\_R2 =}\NormalTok{ adj\_R2))}
\NormalTok{\}}

\NormalTok{calculate\_pvalue }\OtherTok{\textless{}{-}} \ControlFlowTok{function}\NormalTok{(model, data) \{}
\NormalTok{  predicted }\OtherTok{\textless{}{-}} \FunctionTok{predict}\NormalTok{(model, }\AttributeTok{newdata =}\NormalTok{ data)}
\NormalTok{  residuals }\OtherTok{\textless{}{-}}\NormalTok{ data}\SpecialCharTok{$}\NormalTok{R }\SpecialCharTok{{-}}\NormalTok{ predicted}
\NormalTok{  SSE }\OtherTok{\textless{}{-}} \FunctionTok{sum}\NormalTok{(residuals}\SpecialCharTok{\^{}}\DecValTok{2}\NormalTok{)}
\NormalTok{  SST }\OtherTok{\textless{}{-}} \FunctionTok{sum}\NormalTok{((data}\SpecialCharTok{$}\NormalTok{R }\SpecialCharTok{{-}} \FunctionTok{mean}\NormalTok{(data}\SpecialCharTok{$}\NormalTok{R))}\SpecialCharTok{\^{}}\DecValTok{2}\NormalTok{)}
\NormalTok{  SSR }\OtherTok{\textless{}{-}}\NormalTok{ SST }\SpecialCharTok{{-}}\NormalTok{ SSE}
\NormalTok{  n }\OtherTok{\textless{}{-}} \FunctionTok{nrow}\NormalTok{(data)}
\NormalTok{  p }\OtherTok{\textless{}{-}} \FunctionTok{length}\NormalTok{(}\FunctionTok{coef}\NormalTok{(model))}
\NormalTok{  F\_stat }\OtherTok{\textless{}{-}}\NormalTok{ (SSR }\SpecialCharTok{/}\NormalTok{ (p }\SpecialCharTok{{-}} \DecValTok{1}\NormalTok{)) }\SpecialCharTok{/}\NormalTok{ (SSE }\SpecialCharTok{/}\NormalTok{ (n }\SpecialCharTok{{-}}\NormalTok{ p))}
\NormalTok{  p\_value }\OtherTok{\textless{}{-}} \FunctionTok{pf}\NormalTok{(F\_stat, }\AttributeTok{df1 =}\NormalTok{ p }\SpecialCharTok{{-}} \DecValTok{1}\NormalTok{, }\AttributeTok{df2 =}\NormalTok{ n }\SpecialCharTok{{-}}\NormalTok{ p, }\AttributeTok{lower.tail =} \ConstantTok{FALSE}\NormalTok{)}
  \FunctionTok{return}\NormalTok{(p\_value)}
\NormalTok{\}}

\NormalTok{ricker\_r2 }\OtherTok{\textless{}{-}} \FunctionTok{calculate\_R2}\NormalTok{(ricker\_model, rec\_data)}
\NormalTok{bh\_r2 }\OtherTok{\textless{}{-}} \FunctionTok{calculate\_R2}\NormalTok{(bh\_model, rec\_data)}

\NormalTok{ricker\_p }\OtherTok{\textless{}{-}} \FunctionTok{calculate\_pvalue}\NormalTok{(ricker\_model, rec\_data)}
\NormalTok{bh\_p }\OtherTok{\textless{}{-}} \FunctionTok{calculate\_pvalue}\NormalTok{(bh\_model, rec\_data)}

\FunctionTok{cat}\NormalTok{(}\StringTok{"AIC Рикера:"}\NormalTok{, }\FunctionTok{AIC}\NormalTok{(ricker\_model), }\StringTok{"}\SpecialCharTok{\textbackslash{}n}\StringTok{"}\NormalTok{)}
\end{Highlighting}
\end{Shaded}

\begin{verbatim}
AIC Рикера: 891.9919 
\end{verbatim}

\begin{Shaded}
\begin{Highlighting}[]
\FunctionTok{cat}\NormalTok{(}\StringTok{"AIC Бивертона{-}Холта:"}\NormalTok{, }\FunctionTok{AIC}\NormalTok{(bh\_model), }\StringTok{"}\SpecialCharTok{\textbackslash{}n}\StringTok{"}\NormalTok{)}
\end{Highlighting}
\end{Shaded}

\begin{verbatim}
AIC Бивертона-Холта: 894.2029 
\end{verbatim}

\begin{Shaded}
\begin{Highlighting}[]
\CommentTok{\# 6. ВИЗУАЛИЗАЦИЯ РЕЗУЛЬТАТОВ {-}{-}{-}{-}{-}{-}{-}{-}{-}{-}{-}{-}{-}{-}{-}{-}{-}{-}{-}{-}{-}{-}{-}{-}{-}{-}{-}{-}{-}{-}{-}{-}{-}{-}{-}{-}{-}{-}{-}{-}{-}{-}{-}{-}{-}{-}{-}{-}{-}{-}}
\NormalTok{new\_data }\OtherTok{\textless{}{-}} \FunctionTok{data.frame}\NormalTok{(}\AttributeTok{S =} \FunctionTok{seq}\NormalTok{(}\FunctionTok{min}\NormalTok{(rec\_data}\SpecialCharTok{$}\NormalTok{S), }\FunctionTok{max}\NormalTok{(rec\_data}\SpecialCharTok{$}\NormalTok{S), }\AttributeTok{length.out =} \DecValTok{100}\NormalTok{))}
\NormalTok{ricker\_ci }\OtherTok{\textless{}{-}}\NormalTok{ investr}\SpecialCharTok{::}\FunctionTok{predFit}\NormalTok{(ricker\_model, }\AttributeTok{newdata =}\NormalTok{ new\_data, }\AttributeTok{interval =} \StringTok{"confidence"}\NormalTok{)}
\NormalTok{bh\_ci }\OtherTok{\textless{}{-}}\NormalTok{ investr}\SpecialCharTok{::}\FunctionTok{predFit}\NormalTok{(bh\_model, }\AttributeTok{newdata =}\NormalTok{ new\_data, }\AttributeTok{interval =} \StringTok{"confidence"}\NormalTok{)}

\NormalTok{plot\_data }\OtherTok{\textless{}{-}}\NormalTok{ new\_data }\SpecialCharTok{\%\textgreater{}\%}
  \FunctionTok{mutate}\NormalTok{(}
    \AttributeTok{ricker\_pred =} \FunctionTok{predict}\NormalTok{(ricker\_model, }\AttributeTok{newdata =}\NormalTok{ .),}
    \AttributeTok{ricker\_lwr =}\NormalTok{ ricker\_ci[, }\StringTok{"lwr"}\NormalTok{],}
    \AttributeTok{ricker\_upr =}\NormalTok{ ricker\_ci[, }\StringTok{"upr"}\NormalTok{],}
    \AttributeTok{bh\_pred =} \FunctionTok{predict}\NormalTok{(bh\_model, }\AttributeTok{newdata =}\NormalTok{ .),}
    \AttributeTok{bh\_lwr =}\NormalTok{ bh\_ci[, }\StringTok{"lwr"}\NormalTok{],}
    \AttributeTok{bh\_upr =}\NormalTok{ bh\_ci[, }\StringTok{"upr"}\NormalTok{]}
\NormalTok{  )}

\FunctionTok{ggplot}\NormalTok{() }\SpecialCharTok{+}
  \FunctionTok{geom\_point}\NormalTok{(}\AttributeTok{data =}\NormalTok{ rec\_data, }\FunctionTok{aes}\NormalTok{(}\AttributeTok{x =}\NormalTok{ S, }\AttributeTok{y =}\NormalTok{ R), }
             \AttributeTok{color =} \StringTok{"darkgray"}\NormalTok{, }\AttributeTok{size =} \DecValTok{3}\NormalTok{, }\AttributeTok{alpha =} \FloatTok{0.7}\NormalTok{) }\SpecialCharTok{+}
  \FunctionTok{geom\_ribbon}\NormalTok{(}\AttributeTok{data =}\NormalTok{ plot\_data, }\FunctionTok{aes}\NormalTok{(}\AttributeTok{x =}\NormalTok{ S, }\AttributeTok{ymin =}\NormalTok{ ricker\_lwr, }\AttributeTok{ymax =}\NormalTok{ ricker\_upr), }
              \AttributeTok{fill =} \StringTok{"red"}\NormalTok{, }\AttributeTok{alpha =} \FloatTok{0.2}\NormalTok{) }\SpecialCharTok{+}
  \FunctionTok{geom\_ribbon}\NormalTok{(}\AttributeTok{data =}\NormalTok{ plot\_data, }\FunctionTok{aes}\NormalTok{(}\AttributeTok{x =}\NormalTok{ S, }\AttributeTok{ymin =}\NormalTok{ bh\_lwr, }\AttributeTok{ymax =}\NormalTok{ bh\_upr), }
              \AttributeTok{fill =} \StringTok{"blue"}\NormalTok{, }\AttributeTok{alpha =} \FloatTok{0.2}\NormalTok{) }\SpecialCharTok{+}
  \FunctionTok{geom\_line}\NormalTok{(}\AttributeTok{data =}\NormalTok{ plot\_data, }\FunctionTok{aes}\NormalTok{(}\AttributeTok{x =}\NormalTok{ S, }\AttributeTok{y =}\NormalTok{ ricker\_pred), }
            \AttributeTok{color =} \StringTok{"red"}\NormalTok{, }\AttributeTok{linewidth =} \FloatTok{1.2}\NormalTok{) }\SpecialCharTok{+}
  \FunctionTok{geom\_line}\NormalTok{(}\AttributeTok{data =}\NormalTok{ plot\_data, }\FunctionTok{aes}\NormalTok{(}\AttributeTok{x =}\NormalTok{ S, }\AttributeTok{y =}\NormalTok{ bh\_pred), }
            \AttributeTok{color =} \StringTok{"blue"}\NormalTok{, }\AttributeTok{linewidth =} \FloatTok{1.2}\NormalTok{, }\AttributeTok{linetype =} \StringTok{"dashed"}\NormalTok{) }\SpecialCharTok{+}
  \FunctionTok{labs}\NormalTok{(}
    \AttributeTok{title =} \StringTok{"Сравнение моделей запас{-}пополнение"}\NormalTok{,}
    \AttributeTok{subtitle =} \FunctionTok{paste0}\NormalTok{(}
      \StringTok{"Рикер: R² = "}\NormalTok{, }\FunctionTok{round}\NormalTok{(ricker\_r2}\SpecialCharTok{$}\NormalTok{R2, }\DecValTok{2}\NormalTok{), }\StringTok{", p = "}\NormalTok{, }\FunctionTok{format.pval}\NormalTok{(ricker\_p, }\AttributeTok{digits =} \DecValTok{3}\NormalTok{),}
      \StringTok{" | Бивертон{-}Холт: R² = "}\NormalTok{, }\FunctionTok{round}\NormalTok{(bh\_r2}\SpecialCharTok{$}\NormalTok{R2, }\DecValTok{2}\NormalTok{), }\StringTok{", p = "}\NormalTok{, }\FunctionTok{format.pval}\NormalTok{(bh\_p, }\AttributeTok{digits =} \DecValTok{3}\NormalTok{)}
\NormalTok{    ),}
    \AttributeTok{x =} \StringTok{"Нерестовый запас"}\NormalTok{,}
    \AttributeTok{y =} \StringTok{"Пополнение"}
\NormalTok{  ) }\SpecialCharTok{+}
  \FunctionTok{theme\_minimal}\NormalTok{(}\AttributeTok{base\_size =} \DecValTok{14}\NormalTok{) }\SpecialCharTok{+}
  \FunctionTok{theme}\NormalTok{(}
    \AttributeTok{plot.title =} \FunctionTok{element\_text}\NormalTok{(}\AttributeTok{face =} \StringTok{"bold"}\NormalTok{),}
    \AttributeTok{legend.position =} \StringTok{"none"}
\NormalTok{  )}
\end{Highlighting}
\end{Shaded}

\begin{verbatim}
Warning in grid.Call(C_textBounds, as.graphicsAnnot(x$label), x$x, x$y, :
неизвестна ширина символа 0xcf в кодировке CP1251
\end{verbatim}

\begin{verbatim}
Warning in grid.Call(C_textBounds, as.graphicsAnnot(x$label), x$x, x$y, :
неизвестна ширина символа 0xee в кодировке CP1251
\end{verbatim}

\begin{verbatim}
Warning in grid.Call(C_textBounds, as.graphicsAnnot(x$label), x$x, x$y, :
неизвестна ширина символа 0xef в кодировке CP1251
\end{verbatim}

\begin{verbatim}
Warning in grid.Call(C_textBounds, as.graphicsAnnot(x$label), x$x, x$y, :
неизвестна ширина символа 0xee в кодировке CP1251
\end{verbatim}

\begin{verbatim}
Warning in grid.Call(C_textBounds, as.graphicsAnnot(x$label), x$x, x$y, :
неизвестна ширина символа 0xeb в кодировке CP1251
\end{verbatim}

\begin{verbatim}
Warning in grid.Call(C_textBounds, as.graphicsAnnot(x$label), x$x, x$y, :
неизвестна ширина символа 0xed в кодировке CP1251
\end{verbatim}

\begin{verbatim}
Warning in grid.Call(C_textBounds, as.graphicsAnnot(x$label), x$x, x$y, :
неизвестна ширина символа 0xe5 в кодировке CP1251
\end{verbatim}

\begin{verbatim}
Warning in grid.Call(C_textBounds, as.graphicsAnnot(x$label), x$x, x$y, :
неизвестна ширина символа 0xed в кодировке CP1251
\end{verbatim}

\begin{verbatim}
Warning in grid.Call(C_textBounds, as.graphicsAnnot(x$label), x$x, x$y, :
неизвестна ширина символа 0xe8 в кодировке CP1251
\end{verbatim}

\begin{verbatim}
Warning in grid.Call(C_textBounds, as.graphicsAnnot(x$label), x$x, x$y, :
неизвестна ширина символа 0xe5 в кодировке CP1251
\end{verbatim}

\begin{verbatim}
Warning in grid.Call(C_textBounds, as.graphicsAnnot(x$label), x$x, x$y, :
неизвестна ширина символа 0xd1 в кодировке CP1251
\end{verbatim}

\begin{verbatim}
Warning in grid.Call(C_textBounds, as.graphicsAnnot(x$label), x$x, x$y, :
неизвестна ширина символа 0xf0 в кодировке CP1251
\end{verbatim}

\begin{verbatim}
Warning in grid.Call(C_textBounds, as.graphicsAnnot(x$label), x$x, x$y, :
неизвестна ширина символа 0xe0 в кодировке CP1251
\end{verbatim}

\begin{verbatim}
Warning in grid.Call(C_textBounds, as.graphicsAnnot(x$label), x$x, x$y, :
неизвестна ширина символа 0xe2 в кодировке CP1251
\end{verbatim}

\begin{verbatim}
Warning in grid.Call(C_textBounds, as.graphicsAnnot(x$label), x$x, x$y, :
неизвестна ширина символа 0xed в кодировке CP1251
\end{verbatim}

\begin{verbatim}
Warning in grid.Call(C_textBounds, as.graphicsAnnot(x$label), x$x, x$y, :
неизвестна ширина символа 0xe5 в кодировке CP1251
\end{verbatim}

\begin{verbatim}
Warning in grid.Call(C_textBounds, as.graphicsAnnot(x$label), x$x, x$y, :
неизвестна ширина символа 0xed в кодировке CP1251
\end{verbatim}

\begin{verbatim}
Warning in grid.Call(C_textBounds, as.graphicsAnnot(x$label), x$x, x$y, :
неизвестна ширина символа 0xe8 в кодировке CP1251
\end{verbatim}

\begin{verbatim}
Warning in grid.Call(C_textBounds, as.graphicsAnnot(x$label), x$x, x$y, :
неизвестна ширина символа 0xe5 в кодировке CP1251
\end{verbatim}

\begin{verbatim}
Warning in grid.Call(C_textBounds, as.graphicsAnnot(x$label), x$x, x$y, :
неизвестна ширина символа 0xec в кодировке CP1251
\end{verbatim}

\begin{verbatim}
Warning in grid.Call(C_textBounds, as.graphicsAnnot(x$label), x$x, x$y, :
неизвестна ширина символа 0xee в кодировке CP1251
\end{verbatim}

\begin{verbatim}
Warning in grid.Call(C_textBounds, as.graphicsAnnot(x$label), x$x, x$y, :
неизвестна ширина символа 0xe4 в кодировке CP1251
\end{verbatim}

\begin{verbatim}
Warning in grid.Call(C_textBounds, as.graphicsAnnot(x$label), x$x, x$y, :
неизвестна ширина символа 0xe5 в кодировке CP1251
\end{verbatim}

\begin{verbatim}
Warning in grid.Call(C_textBounds, as.graphicsAnnot(x$label), x$x, x$y, :
неизвестна ширина символа 0xeb в кодировке CP1251
\end{verbatim}

\begin{verbatim}
Warning in grid.Call(C_textBounds, as.graphicsAnnot(x$label), x$x, x$y, :
неизвестна ширина символа 0xe5 в кодировке CP1251
\end{verbatim}

\begin{verbatim}
Warning in grid.Call(C_textBounds, as.graphicsAnnot(x$label), x$x, x$y, :
неизвестна ширина символа 0xe9 в кодировке CP1251
\end{verbatim}

\begin{verbatim}
Warning in grid.Call(C_textBounds, as.graphicsAnnot(x$label), x$x, x$y, :
неизвестна ширина символа 0xe7 в кодировке CP1251
\end{verbatim}

\begin{verbatim}
Warning in grid.Call(C_textBounds, as.graphicsAnnot(x$label), x$x, x$y, :
неизвестна ширина символа 0xe0 в кодировке CP1251
\end{verbatim}

\begin{verbatim}
Warning in grid.Call(C_textBounds, as.graphicsAnnot(x$label), x$x, x$y, :
неизвестна ширина символа 0xef в кодировке CP1251
\end{verbatim}

\begin{verbatim}
Warning in grid.Call(C_textBounds, as.graphicsAnnot(x$label), x$x, x$y, :
неизвестна ширина символа 0xe0 в кодировке CP1251
\end{verbatim}

\begin{verbatim}
Warning in grid.Call(C_textBounds, as.graphicsAnnot(x$label), x$x, x$y, :
неизвестна ширина символа 0xf1 в кодировке CP1251
\end{verbatim}

\begin{verbatim}
Warning in grid.Call(C_textBounds, as.graphicsAnnot(x$label), x$x, x$y, :
неизвестна ширина символа 0xef в кодировке CP1251
\end{verbatim}

\begin{verbatim}
Warning in grid.Call(C_textBounds, as.graphicsAnnot(x$label), x$x, x$y, :
неизвестна ширина символа 0xee в кодировке CP1251
\end{verbatim}

\begin{verbatim}
Warning in grid.Call(C_textBounds, as.graphicsAnnot(x$label), x$x, x$y, :
неизвестна ширина символа 0xef в кодировке CP1251
\end{verbatim}

\begin{verbatim}
Warning in grid.Call(C_textBounds, as.graphicsAnnot(x$label), x$x, x$y, :
неизвестна ширина символа 0xee в кодировке CP1251
\end{verbatim}

\begin{verbatim}
Warning in grid.Call(C_textBounds, as.graphicsAnnot(x$label), x$x, x$y, :
неизвестна ширина символа 0xeb в кодировке CP1251
\end{verbatim}

\begin{verbatim}
Warning in grid.Call(C_textBounds, as.graphicsAnnot(x$label), x$x, x$y, :
неизвестна ширина символа 0xed в кодировке CP1251
\end{verbatim}

\begin{verbatim}
Warning in grid.Call(C_textBounds, as.graphicsAnnot(x$label), x$x, x$y, :
неизвестна ширина символа 0xe5 в кодировке CP1251
\end{verbatim}

\begin{verbatim}
Warning in grid.Call(C_textBounds, as.graphicsAnnot(x$label), x$x, x$y, :
неизвестна ширина символа 0xed в кодировке CP1251
\end{verbatim}

\begin{verbatim}
Warning in grid.Call(C_textBounds, as.graphicsAnnot(x$label), x$x, x$y, :
неизвестна ширина символа 0xe8 в кодировке CP1251
\end{verbatim}

\begin{verbatim}
Warning in grid.Call(C_textBounds, as.graphicsAnnot(x$label), x$x, x$y, :
неизвестна ширина символа 0xe5 в кодировке CP1251
\end{verbatim}

\begin{verbatim}
Warning in grid.Call(C_textBounds, as.graphicsAnnot(x$label), x$x, x$y, :
неизвестна ширина символа 0xd0 в кодировке CP1251
\end{verbatim}

\begin{verbatim}
Warning in grid.Call(C_textBounds, as.graphicsAnnot(x$label), x$x, x$y, :
неизвестна ширина символа 0xe8 в кодировке CP1251
\end{verbatim}

\begin{verbatim}
Warning in grid.Call(C_textBounds, as.graphicsAnnot(x$label), x$x, x$y, :
неизвестна ширина символа 0xea в кодировке CP1251
\end{verbatim}

\begin{verbatim}
Warning in grid.Call(C_textBounds, as.graphicsAnnot(x$label), x$x, x$y, :
неизвестна ширина символа 0xe5 в кодировке CP1251
\end{verbatim}

\begin{verbatim}
Warning in grid.Call(C_textBounds, as.graphicsAnnot(x$label), x$x, x$y, :
неизвестна ширина символа 0xf0 в кодировке CP1251
\end{verbatim}

\begin{verbatim}
Warning in grid.Call(C_textBounds, as.graphicsAnnot(x$label), x$x, x$y, :
неизвестна ширина символа 0xc1 в кодировке CP1251
\end{verbatim}

\begin{verbatim}
Warning in grid.Call(C_textBounds, as.graphicsAnnot(x$label), x$x, x$y, :
неизвестна ширина символа 0xe8 в кодировке CP1251
\end{verbatim}

\begin{verbatim}
Warning in grid.Call(C_textBounds, as.graphicsAnnot(x$label), x$x, x$y, :
неизвестна ширина символа 0xe2 в кодировке CP1251
\end{verbatim}

\begin{verbatim}
Warning in grid.Call(C_textBounds, as.graphicsAnnot(x$label), x$x, x$y, :
неизвестна ширина символа 0xe5 в кодировке CP1251
\end{verbatim}

\begin{verbatim}
Warning in grid.Call(C_textBounds, as.graphicsAnnot(x$label), x$x, x$y, :
неизвестна ширина символа 0xf0 в кодировке CP1251
\end{verbatim}

\begin{verbatim}
Warning in grid.Call(C_textBounds, as.graphicsAnnot(x$label), x$x, x$y, :
неизвестна ширина символа 0xf2 в кодировке CP1251
\end{verbatim}

\begin{verbatim}
Warning in grid.Call(C_textBounds, as.graphicsAnnot(x$label), x$x, x$y, :
неизвестна ширина символа 0xee в кодировке CP1251
\end{verbatim}

\begin{verbatim}
Warning in grid.Call(C_textBounds, as.graphicsAnnot(x$label), x$x, x$y, :
неизвестна ширина символа 0xed в кодировке CP1251
\end{verbatim}

\begin{verbatim}
Warning in grid.Call(C_textBounds, as.graphicsAnnot(x$label), x$x, x$y, :
неизвестна ширина символа 0xd5 в кодировке CP1251
\end{verbatim}

\begin{verbatim}
Warning in grid.Call(C_textBounds, as.graphicsAnnot(x$label), x$x, x$y, :
неизвестна ширина символа 0xee в кодировке CP1251
\end{verbatim}

\begin{verbatim}
Warning in grid.Call(C_textBounds, as.graphicsAnnot(x$label), x$x, x$y, :
неизвестна ширина символа 0xeb в кодировке CP1251
\end{verbatim}

\begin{verbatim}
Warning in grid.Call(C_textBounds, as.graphicsAnnot(x$label), x$x, x$y, :
неизвестна ширина символа 0xf2 в кодировке CP1251
\end{verbatim}

\begin{verbatim}
Warning in grid.Call(C_textBounds, as.graphicsAnnot(x$label), x$x, x$y, :
неизвестна ширина символа 0xcd в кодировке CP1251
\end{verbatim}

\begin{verbatim}
Warning in grid.Call(C_textBounds, as.graphicsAnnot(x$label), x$x, x$y, :
неизвестна ширина символа 0xe5 в кодировке CP1251
\end{verbatim}

\begin{verbatim}
Warning in grid.Call(C_textBounds, as.graphicsAnnot(x$label), x$x, x$y, :
неизвестна ширина символа 0xf0 в кодировке CP1251
\end{verbatim}

\begin{verbatim}
Warning in grid.Call(C_textBounds, as.graphicsAnnot(x$label), x$x, x$y, :
неизвестна ширина символа 0xe5 в кодировке CP1251
\end{verbatim}

\begin{verbatim}
Warning in grid.Call(C_textBounds, as.graphicsAnnot(x$label), x$x, x$y, :
неизвестна ширина символа 0xf1 в кодировке CP1251
\end{verbatim}

\begin{verbatim}
Warning in grid.Call(C_textBounds, as.graphicsAnnot(x$label), x$x, x$y, :
неизвестна ширина символа 0xf2 в кодировке CP1251
\end{verbatim}

\begin{verbatim}
Warning in grid.Call(C_textBounds, as.graphicsAnnot(x$label), x$x, x$y, :
неизвестна ширина символа 0xee в кодировке CP1251
\end{verbatim}

\begin{verbatim}
Warning in grid.Call(C_textBounds, as.graphicsAnnot(x$label), x$x, x$y, :
неизвестна ширина символа 0xe2 в кодировке CP1251
\end{verbatim}

\begin{verbatim}
Warning in grid.Call(C_textBounds, as.graphicsAnnot(x$label), x$x, x$y, :
неизвестна ширина символа 0xfb в кодировке CP1251
\end{verbatim}

\begin{verbatim}
Warning in grid.Call(C_textBounds, as.graphicsAnnot(x$label), x$x, x$y, :
неизвестна ширина символа 0xe9 в кодировке CP1251
\end{verbatim}

\begin{verbatim}
Warning in grid.Call(C_textBounds, as.graphicsAnnot(x$label), x$x, x$y, :
неизвестна ширина символа 0xe7 в кодировке CP1251
\end{verbatim}

\begin{verbatim}
Warning in grid.Call(C_textBounds, as.graphicsAnnot(x$label), x$x, x$y, :
неизвестна ширина символа 0xe0 в кодировке CP1251
\end{verbatim}

\begin{verbatim}
Warning in grid.Call(C_textBounds, as.graphicsAnnot(x$label), x$x, x$y, :
неизвестна ширина символа 0xef в кодировке CP1251
\end{verbatim}

\begin{verbatim}
Warning in grid.Call(C_textBounds, as.graphicsAnnot(x$label), x$x, x$y, :
неизвестна ширина символа 0xe0 в кодировке CP1251
\end{verbatim}

\begin{verbatim}
Warning in grid.Call(C_textBounds, as.graphicsAnnot(x$label), x$x, x$y, :
неизвестна ширина символа 0xf1 в кодировке CP1251
\end{verbatim}

\begin{verbatim}
Warning in grid.Call.graphics(C_text, as.graphicsAnnot(x$label), x$x, x$y, :
неизвестна ширина символа 0xcd в кодировке CP1251
\end{verbatim}

\begin{verbatim}
Warning in grid.Call.graphics(C_text, as.graphicsAnnot(x$label), x$x, x$y, :
неизвестна ширина символа 0xe5 в кодировке CP1251
\end{verbatim}

\begin{verbatim}
Warning in grid.Call.graphics(C_text, as.graphicsAnnot(x$label), x$x, x$y, :
неизвестна ширина символа 0xf0 в кодировке CP1251
\end{verbatim}

\begin{verbatim}
Warning in grid.Call.graphics(C_text, as.graphicsAnnot(x$label), x$x, x$y, :
неизвестна ширина символа 0xe5 в кодировке CP1251
\end{verbatim}

\begin{verbatim}
Warning in grid.Call.graphics(C_text, as.graphicsAnnot(x$label), x$x, x$y, :
неизвестна ширина символа 0xf1 в кодировке CP1251
\end{verbatim}

\begin{verbatim}
Warning in grid.Call.graphics(C_text, as.graphicsAnnot(x$label), x$x, x$y, :
неизвестна ширина символа 0xf2 в кодировке CP1251
\end{verbatim}

\begin{verbatim}
Warning in grid.Call.graphics(C_text, as.graphicsAnnot(x$label), x$x, x$y, :
неизвестна ширина символа 0xee в кодировке CP1251
\end{verbatim}

\begin{verbatim}
Warning in grid.Call.graphics(C_text, as.graphicsAnnot(x$label), x$x, x$y, :
неизвестна ширина символа 0xe2 в кодировке CP1251
\end{verbatim}

\begin{verbatim}
Warning in grid.Call.graphics(C_text, as.graphicsAnnot(x$label), x$x, x$y, :
неизвестна ширина символа 0xfb в кодировке CP1251
\end{verbatim}

\begin{verbatim}
Warning in grid.Call.graphics(C_text, as.graphicsAnnot(x$label), x$x, x$y, :
неизвестна ширина символа 0xe9 в кодировке CP1251
\end{verbatim}

\begin{verbatim}
Warning in grid.Call.graphics(C_text, as.graphicsAnnot(x$label), x$x, x$y, :
неизвестна ширина символа 0xe7 в кодировке CP1251
\end{verbatim}

\begin{verbatim}
Warning in grid.Call.graphics(C_text, as.graphicsAnnot(x$label), x$x, x$y, :
неизвестна ширина символа 0xe0 в кодировке CP1251
\end{verbatim}

\begin{verbatim}
Warning in grid.Call.graphics(C_text, as.graphicsAnnot(x$label), x$x, x$y, :
неизвестна ширина символа 0xef в кодировке CP1251
\end{verbatim}

\begin{verbatim}
Warning in grid.Call.graphics(C_text, as.graphicsAnnot(x$label), x$x, x$y, :
неизвестна ширина символа 0xe0 в кодировке CP1251
\end{verbatim}

\begin{verbatim}
Warning in grid.Call.graphics(C_text, as.graphicsAnnot(x$label), x$x, x$y, :
неизвестна ширина символа 0xf1 в кодировке CP1251
\end{verbatim}

\begin{verbatim}
Warning in grid.Call.graphics(C_text, as.graphicsAnnot(x$label), x$x, x$y, :
неизвестна ширина символа 0xcf в кодировке CP1251
\end{verbatim}

\begin{verbatim}
Warning in grid.Call.graphics(C_text, as.graphicsAnnot(x$label), x$x, x$y, :
неизвестна ширина символа 0xee в кодировке CP1251
\end{verbatim}

\begin{verbatim}
Warning in grid.Call.graphics(C_text, as.graphicsAnnot(x$label), x$x, x$y, :
неизвестна ширина символа 0xef в кодировке CP1251
\end{verbatim}

\begin{verbatim}
Warning in grid.Call.graphics(C_text, as.graphicsAnnot(x$label), x$x, x$y, :
неизвестна ширина символа 0xee в кодировке CP1251
\end{verbatim}

\begin{verbatim}
Warning in grid.Call.graphics(C_text, as.graphicsAnnot(x$label), x$x, x$y, :
неизвестна ширина символа 0xeb в кодировке CP1251
\end{verbatim}

\begin{verbatim}
Warning in grid.Call.graphics(C_text, as.graphicsAnnot(x$label), x$x, x$y, :
неизвестна ширина символа 0xed в кодировке CP1251
\end{verbatim}

\begin{verbatim}
Warning in grid.Call.graphics(C_text, as.graphicsAnnot(x$label), x$x, x$y, :
неизвестна ширина символа 0xe5 в кодировке CP1251
\end{verbatim}

\begin{verbatim}
Warning in grid.Call.graphics(C_text, as.graphicsAnnot(x$label), x$x, x$y, :
неизвестна ширина символа 0xed в кодировке CP1251
\end{verbatim}

\begin{verbatim}
Warning in grid.Call.graphics(C_text, as.graphicsAnnot(x$label), x$x, x$y, :
неизвестна ширина символа 0xe8 в кодировке CP1251
\end{verbatim}

\begin{verbatim}
Warning in grid.Call.graphics(C_text, as.graphicsAnnot(x$label), x$x, x$y, :
неизвестна ширина символа 0xe5 в кодировке CP1251
\end{verbatim}

\begin{verbatim}
Warning in grid.Call.graphics(C_text, as.graphicsAnnot(x$label), x$x, x$y, :
неизвестна ширина символа 0xd0 в кодировке CP1251
\end{verbatim}

\begin{verbatim}
Warning in grid.Call.graphics(C_text, as.graphicsAnnot(x$label), x$x, x$y, :
неизвестна ширина символа 0xe8 в кодировке CP1251
\end{verbatim}

\begin{verbatim}
Warning in grid.Call.graphics(C_text, as.graphicsAnnot(x$label), x$x, x$y, :
неизвестна ширина символа 0xea в кодировке CP1251
\end{verbatim}

\begin{verbatim}
Warning in grid.Call.graphics(C_text, as.graphicsAnnot(x$label), x$x, x$y, :
неизвестна ширина символа 0xe5 в кодировке CP1251
\end{verbatim}

\begin{verbatim}
Warning in grid.Call.graphics(C_text, as.graphicsAnnot(x$label), x$x, x$y, :
неизвестна ширина символа 0xf0 в кодировке CP1251
\end{verbatim}

\begin{verbatim}
Warning in grid.Call.graphics(C_text, as.graphicsAnnot(x$label), x$x, x$y, :
неизвестна ширина символа 0xc1 в кодировке CP1251
\end{verbatim}

\begin{verbatim}
Warning in grid.Call.graphics(C_text, as.graphicsAnnot(x$label), x$x, x$y, :
неизвестна ширина символа 0xe8 в кодировке CP1251
\end{verbatim}

\begin{verbatim}
Warning in grid.Call.graphics(C_text, as.graphicsAnnot(x$label), x$x, x$y, :
неизвестна ширина символа 0xe2 в кодировке CP1251
\end{verbatim}

\begin{verbatim}
Warning in grid.Call.graphics(C_text, as.graphicsAnnot(x$label), x$x, x$y, :
неизвестна ширина символа 0xe5 в кодировке CP1251
\end{verbatim}

\begin{verbatim}
Warning in grid.Call.graphics(C_text, as.graphicsAnnot(x$label), x$x, x$y, :
неизвестна ширина символа 0xf0 в кодировке CP1251
\end{verbatim}

\begin{verbatim}
Warning in grid.Call.graphics(C_text, as.graphicsAnnot(x$label), x$x, x$y, :
неизвестна ширина символа 0xf2 в кодировке CP1251
\end{verbatim}

\begin{verbatim}
Warning in grid.Call.graphics(C_text, as.graphicsAnnot(x$label), x$x, x$y, :
неизвестна ширина символа 0xee в кодировке CP1251
\end{verbatim}

\begin{verbatim}
Warning in grid.Call.graphics(C_text, as.graphicsAnnot(x$label), x$x, x$y, :
неизвестна ширина символа 0xed в кодировке CP1251
\end{verbatim}

\begin{verbatim}
Warning in grid.Call.graphics(C_text, as.graphicsAnnot(x$label), x$x, x$y, :
неизвестна ширина символа 0xd5 в кодировке CP1251
\end{verbatim}

\begin{verbatim}
Warning in grid.Call.graphics(C_text, as.graphicsAnnot(x$label), x$x, x$y, :
неизвестна ширина символа 0xee в кодировке CP1251
\end{verbatim}

\begin{verbatim}
Warning in grid.Call.graphics(C_text, as.graphicsAnnot(x$label), x$x, x$y, :
неизвестна ширина символа 0xeb в кодировке CP1251
\end{verbatim}

\begin{verbatim}
Warning in grid.Call.graphics(C_text, as.graphicsAnnot(x$label), x$x, x$y, :
неизвестна ширина символа 0xf2 в кодировке CP1251
\end{verbatim}

\begin{verbatim}
Warning in grid.Call.graphics(C_text, as.graphicsAnnot(x$label), x$x, x$y, :
неизвестна ширина символа 0xd1 в кодировке CP1251
\end{verbatim}

\begin{verbatim}
Warning in grid.Call.graphics(C_text, as.graphicsAnnot(x$label), x$x, x$y, :
неизвестна ширина символа 0xf0 в кодировке CP1251
\end{verbatim}

\begin{verbatim}
Warning in grid.Call.graphics(C_text, as.graphicsAnnot(x$label), x$x, x$y, :
неизвестна ширина символа 0xe0 в кодировке CP1251
\end{verbatim}

\begin{verbatim}
Warning in grid.Call.graphics(C_text, as.graphicsAnnot(x$label), x$x, x$y, :
неизвестна ширина символа 0xe2 в кодировке CP1251
\end{verbatim}

\begin{verbatim}
Warning in grid.Call.graphics(C_text, as.graphicsAnnot(x$label), x$x, x$y, :
неизвестна ширина символа 0xed в кодировке CP1251
\end{verbatim}

\begin{verbatim}
Warning in grid.Call.graphics(C_text, as.graphicsAnnot(x$label), x$x, x$y, :
неизвестна ширина символа 0xe5 в кодировке CP1251
\end{verbatim}

\begin{verbatim}
Warning in grid.Call.graphics(C_text, as.graphicsAnnot(x$label), x$x, x$y, :
неизвестна ширина символа 0xed в кодировке CP1251
\end{verbatim}

\begin{verbatim}
Warning in grid.Call.graphics(C_text, as.graphicsAnnot(x$label), x$x, x$y, :
неизвестна ширина символа 0xe8 в кодировке CP1251
\end{verbatim}

\begin{verbatim}
Warning in grid.Call.graphics(C_text, as.graphicsAnnot(x$label), x$x, x$y, :
неизвестна ширина символа 0xe5 в кодировке CP1251
\end{verbatim}

\begin{verbatim}
Warning in grid.Call.graphics(C_text, as.graphicsAnnot(x$label), x$x, x$y, :
неизвестна ширина символа 0xec в кодировке CP1251
\end{verbatim}

\begin{verbatim}
Warning in grid.Call.graphics(C_text, as.graphicsAnnot(x$label), x$x, x$y, :
неизвестна ширина символа 0xee в кодировке CP1251
\end{verbatim}

\begin{verbatim}
Warning in grid.Call.graphics(C_text, as.graphicsAnnot(x$label), x$x, x$y, :
неизвестна ширина символа 0xe4 в кодировке CP1251
\end{verbatim}

\begin{verbatim}
Warning in grid.Call.graphics(C_text, as.graphicsAnnot(x$label), x$x, x$y, :
неизвестна ширина символа 0xe5 в кодировке CP1251
\end{verbatim}

\begin{verbatim}
Warning in grid.Call.graphics(C_text, as.graphicsAnnot(x$label), x$x, x$y, :
неизвестна ширина символа 0xeb в кодировке CP1251
\end{verbatim}

\begin{verbatim}
Warning in grid.Call.graphics(C_text, as.graphicsAnnot(x$label), x$x, x$y, :
неизвестна ширина символа 0xe5 в кодировке CP1251
\end{verbatim}

\begin{verbatim}
Warning in grid.Call.graphics(C_text, as.graphicsAnnot(x$label), x$x, x$y, :
неизвестна ширина символа 0xe9 в кодировке CP1251
\end{verbatim}

\begin{verbatim}
Warning in grid.Call.graphics(C_text, as.graphicsAnnot(x$label), x$x, x$y, :
неизвестна ширина символа 0xe7 в кодировке CP1251
\end{verbatim}

\begin{verbatim}
Warning in grid.Call.graphics(C_text, as.graphicsAnnot(x$label), x$x, x$y, :
неизвестна ширина символа 0xe0 в кодировке CP1251
\end{verbatim}

\begin{verbatim}
Warning in grid.Call.graphics(C_text, as.graphicsAnnot(x$label), x$x, x$y, :
неизвестна ширина символа 0xef в кодировке CP1251
\end{verbatim}

\begin{verbatim}
Warning in grid.Call.graphics(C_text, as.graphicsAnnot(x$label), x$x, x$y, :
неизвестна ширина символа 0xe0 в кодировке CP1251
\end{verbatim}

\begin{verbatim}
Warning in grid.Call.graphics(C_text, as.graphicsAnnot(x$label), x$x, x$y, :
неизвестна ширина символа 0xf1 в кодировке CP1251
\end{verbatim}

\begin{verbatim}
Warning in grid.Call.graphics(C_text, as.graphicsAnnot(x$label), x$x, x$y, :
неизвестна ширина символа 0xef в кодировке CP1251
\end{verbatim}

\begin{verbatim}
Warning in grid.Call.graphics(C_text, as.graphicsAnnot(x$label), x$x, x$y, :
неизвестна ширина символа 0xee в кодировке CP1251
\end{verbatim}

\begin{verbatim}
Warning in grid.Call.graphics(C_text, as.graphicsAnnot(x$label), x$x, x$y, :
неизвестна ширина символа 0xef в кодировке CP1251
\end{verbatim}

\begin{verbatim}
Warning in grid.Call.graphics(C_text, as.graphicsAnnot(x$label), x$x, x$y, :
неизвестна ширина символа 0xee в кодировке CP1251
\end{verbatim}

\begin{verbatim}
Warning in grid.Call.graphics(C_text, as.graphicsAnnot(x$label), x$x, x$y, :
неизвестна ширина символа 0xeb в кодировке CP1251
\end{verbatim}

\begin{verbatim}
Warning in grid.Call.graphics(C_text, as.graphicsAnnot(x$label), x$x, x$y, :
неизвестна ширина символа 0xed в кодировке CP1251
\end{verbatim}

\begin{verbatim}
Warning in grid.Call.graphics(C_text, as.graphicsAnnot(x$label), x$x, x$y, :
неизвестна ширина символа 0xe5 в кодировке CP1251
\end{verbatim}

\begin{verbatim}
Warning in grid.Call.graphics(C_text, as.graphicsAnnot(x$label), x$x, x$y, :
неизвестна ширина символа 0xed в кодировке CP1251
\end{verbatim}

\begin{verbatim}
Warning in grid.Call.graphics(C_text, as.graphicsAnnot(x$label), x$x, x$y, :
неизвестна ширина символа 0xe8 в кодировке CP1251
\end{verbatim}

\begin{verbatim}
Warning in grid.Call.graphics(C_text, as.graphicsAnnot(x$label), x$x, x$y, :
неизвестна ширина символа 0xe5 в кодировке CP1251
\end{verbatim}

\pandocbounded{\includegraphics[keepaspectratio]{chapter7_files/figure-pdf/unnamed-chunk-3-1.pdf}}

\begin{Shaded}
\begin{Highlighting}[]
\CommentTok{\# 7. СРАВНЕНИЕ С ДРУГИМИ ТИПАМИ МОДЕЛЕЙ {-}{-}{-}{-}{-}{-}{-}{-}{-}{-}{-}{-}{-}{-}{-}{-}{-}{-}{-}{-}{-}{-}{-}{-}{-}{-}{-}{-}{-}{-}{-}{-}{-}{-}{-}{-}{-}{-}{-}{-}}

\NormalTok{lm\_model }\OtherTok{\textless{}{-}} \FunctionTok{lm}\NormalTok{(R3haddock }\SpecialCharTok{\textasciitilde{}}\NormalTok{ ., }\AttributeTok{data =}\NormalTok{ model\_data)}
\NormalTok{glm\_model }\OtherTok{\textless{}{-}} \FunctionTok{glm}\NormalTok{(R3haddock }\SpecialCharTok{\textasciitilde{}}\NormalTok{ ., }\AttributeTok{family =} \FunctionTok{Gamma}\NormalTok{(}\AttributeTok{link =} \StringTok{"log"}\NormalTok{), }\AttributeTok{data =}\NormalTok{ model\_data)}
\NormalTok{gam\_model }\OtherTok{\textless{}{-}}\NormalTok{ mgcv}\SpecialCharTok{::}\FunctionTok{gam}\NormalTok{(R3haddock }\SpecialCharTok{\textasciitilde{}} \FunctionTok{s}\NormalTok{(haddock68) }\SpecialCharTok{+} \FunctionTok{s}\NormalTok{(codTSB) }\SpecialCharTok{+} \FunctionTok{s}\NormalTok{(T12) }\SpecialCharTok{+} \FunctionTok{s}\NormalTok{(I5) }\SpecialCharTok{+} \FunctionTok{s}\NormalTok{(NAOspring),}
               \AttributeTok{data =}\NormalTok{ model\_data, }\AttributeTok{method =} \StringTok{"REML"}\NormalTok{)}

\NormalTok{model\_comparison }\OtherTok{\textless{}{-}} \FunctionTok{data.frame}\NormalTok{(}
  \AttributeTok{Model =} \FunctionTok{c}\NormalTok{(}\StringTok{"Рикер"}\NormalTok{, }\StringTok{"Бивертон{-}Холт"}\NormalTok{, }\StringTok{"LM"}\NormalTok{, }\StringTok{"GLM"}\NormalTok{, }\StringTok{"GAM"}\NormalTok{),}
  \AttributeTok{AIC =} \FunctionTok{c}\NormalTok{(}\FunctionTok{AIC}\NormalTok{(ricker\_model), }\FunctionTok{AIC}\NormalTok{(bh\_model), }\FunctionTok{AIC}\NormalTok{(lm\_model), }\FunctionTok{AIC}\NormalTok{(glm\_model), }\FunctionTok{AIC}\NormalTok{(gam\_model)),}
  \AttributeTok{R2 =} \FunctionTok{c}\NormalTok{(}
\NormalTok{    ricker\_r2}\SpecialCharTok{$}\NormalTok{R2, }
\NormalTok{    bh\_r2}\SpecialCharTok{$}\NormalTok{R2, }
    \FunctionTok{summary}\NormalTok{(lm\_model)}\SpecialCharTok{$}\NormalTok{r.squared,}
    \FunctionTok{cor}\NormalTok{(model\_data}\SpecialCharTok{$}\NormalTok{R3haddock, }\FunctionTok{predict}\NormalTok{(glm\_model, }\AttributeTok{type =} \StringTok{"response"}\NormalTok{))}\SpecialCharTok{\^{}}\DecValTok{2}\NormalTok{,}
    \FunctionTok{summary}\NormalTok{(gam\_model)}\SpecialCharTok{$}\NormalTok{r.sq}
\NormalTok{  )}
\NormalTok{)}

\FunctionTok{print}\NormalTok{(model\_comparison)}
\end{Highlighting}
\end{Shaded}

\begin{verbatim}
          Model      AIC          R2
1         Рикер 891.9919 0.072305265
2 Бивертон-Холт 894.2029 0.005938625
3            LM 877.0895 0.573972535
4           GLM 857.0346 0.566389771
5           GAM 862.9064 0.738660678
\end{verbatim}

\begin{Shaded}
\begin{Highlighting}[]
\CommentTok{\# 8. ИНТЕРПРЕТАЦИЯ РЕЗУЛЬТАТОВ {-}{-}{-}{-}{-}{-}{-}{-}{-}{-}{-}{-}{-}{-}{-}{-}{-}{-}{-}{-}{-}{-}{-}{-}{-}{-}{-}{-}{-}{-}{-}{-}{-}{-}{-}{-}{-}{-}{-}{-}{-}{-}{-}{-}{-}{-}{-}{-}{-}}
\FunctionTok{cat}\NormalTok{(}\StringTok{"}\SpecialCharTok{\textbackslash{}n}\StringTok{Параметры модели Рикера:"}\NormalTok{)}
\end{Highlighting}
\end{Shaded}

\begin{verbatim}

Параметры модели Рикера:
\end{verbatim}

\begin{Shaded}
\begin{Highlighting}[]
\FunctionTok{cat}\NormalTok{(}\StringTok{"}\SpecialCharTok{\textbackslash{}n}\StringTok{a ="}\NormalTok{, }\FunctionTok{coef}\NormalTok{(ricker\_model)[}\DecValTok{1}\NormalTok{], }\StringTok{"{-} максимальная продукция потомства"}\NormalTok{)}
\end{Highlighting}
\end{Shaded}

\begin{verbatim}

a = 7.931302 - максимальная продукция потомства
\end{verbatim}

\begin{Shaded}
\begin{Highlighting}[]
\FunctionTok{cat}\NormalTok{(}\StringTok{"}\SpecialCharTok{\textbackslash{}n}\StringTok{b ="}\NormalTok{, }\FunctionTok{coef}\NormalTok{(ricker\_model)[}\DecValTok{2}\NormalTok{], }\StringTok{"{-} коэффициент плотностной зависимости"}\NormalTok{)}
\end{Highlighting}
\end{Shaded}

\begin{verbatim}

b = 7.4007e-06 - коэффициент плотностной зависимости
\end{verbatim}

\begin{Shaded}
\begin{Highlighting}[]
\FunctionTok{cat}\NormalTok{(}\StringTok{"}\SpecialCharTok{\textbackslash{}n\textbackslash{}n}\StringTok{Параметры модели Бивертона{-}Холта:"}\NormalTok{)}
\end{Highlighting}
\end{Shaded}

\begin{verbatim}


Параметры модели Бивертона-Холта:
\end{verbatim}

\begin{Shaded}
\begin{Highlighting}[]
\FunctionTok{cat}\NormalTok{(}\StringTok{"}\SpecialCharTok{\textbackslash{}n}\StringTok{a ="}\NormalTok{, }\FunctionTok{coef}\NormalTok{(bh\_model)[}\DecValTok{1}\NormalTok{], }\StringTok{"{-} максимальное пополнение на особь"}\NormalTok{)}
\end{Highlighting}
\end{Shaded}

\begin{verbatim}

a = 43.77667 - максимальное пополнение на особь
\end{verbatim}

\begin{Shaded}
\begin{Highlighting}[]
\FunctionTok{cat}\NormalTok{(}\StringTok{"}\SpecialCharTok{\textbackslash{}n}\StringTok{b ="}\NormalTok{, }\FunctionTok{coef}\NormalTok{(bh\_model)[}\DecValTok{2}\NormalTok{], }\StringTok{"{-} коэффициент внутривидовой конкуренции"}\NormalTok{)}
\end{Highlighting}
\end{Shaded}

\begin{verbatim}

b = 0.0001247403 - коэффициент внутривидовой конкуренции
\end{verbatim}

\begin{Shaded}
\begin{Highlighting}[]
\CommentTok{\# ==============================================================================}
\CommentTok{\# 7. СРАВНЕНИЕ С ДРУГИМИ ТИПАМИ МОДЕЛЕЙ {-}{-}{-}{-}{-}{-}{-}{-}{-}{-}{-}{-}{-}{-}{-}{-}{-}{-}{-}{-}{-}{-}{-}{-}{-}{-}{-}{-}{-}{-}{-}{-}{-}{-}{-}{-}{-}{-}{-}{-}}

\CommentTok{\# Построение линейной модели LM}
\NormalTok{lm\_model }\OtherTok{\textless{}{-}} \FunctionTok{lm}\NormalTok{(R3haddock }\SpecialCharTok{\textasciitilde{}}\NormalTok{ ., }\AttributeTok{data =}\NormalTok{ model\_data)}

\CommentTok{\# Диагностика}
\FunctionTok{par}\NormalTok{(}\AttributeTok{mfrow =} \FunctionTok{c}\NormalTok{(}\DecValTok{2}\NormalTok{, }\DecValTok{2}\NormalTok{))}
\FunctionTok{plot}\NormalTok{(lm\_model)}
\end{Highlighting}
\end{Shaded}

\pandocbounded{\includegraphics[keepaspectratio]{chapter7_files/figure-pdf/unnamed-chunk-3-2.pdf}}

\begin{Shaded}
\begin{Highlighting}[]
\FunctionTok{vif}\NormalTok{(lm\_model)  }\CommentTok{\# Проверка мультиколлинеарности}
\end{Highlighting}
\end{Shaded}

\begin{verbatim}
     YEAR    codTSB       T12        I5 NAOspring haddock68 
 2.925549  3.439340  1.970023  1.332529  1.181897  2.522569 
\end{verbatim}

\begin{Shaded}
\begin{Highlighting}[]
\CommentTok{\# Интерпретация}
\FunctionTok{summary}\NormalTok{(lm\_model)}
\end{Highlighting}
\end{Shaded}

\begin{verbatim}

Call:
lm(formula = R3haddock ~ ., data = model_data)

Residuals:
    Min      1Q  Median      3Q     Max 
-279162 -111056  -35757  141083  324173 

Coefficients:
              Estimate Std. Error t value Pr(>|t|)    
(Intercept)  1.448e+07  1.224e+07   1.183 0.248123    
YEAR        -7.863e+03  6.248e+03  -1.258 0.219869    
codTSB      -1.888e-01  7.817e-02  -2.416 0.023338 *  
T12          4.339e+05  9.738e+04   4.456 0.000153 ***
I5          -2.568e+03  3.058e+03  -0.840 0.409036    
NAOspring   -7.666e+04  5.735e+04  -1.337 0.193325    
haddock68    2.782e-01  5.485e-01   0.507 0.616444    
---
Signif. codes:  0 '***' 0.001 '**' 0.01 '*' 0.05 '.' 0.1 ' ' 1

Residual standard error: 190800 on 25 degrees of freedom
Multiple R-squared:  0.574, Adjusted R-squared:  0.4717 
F-statistic: 5.614 on 6 and 25 DF,  p-value: 0.0008393
\end{verbatim}

\begin{Shaded}
\begin{Highlighting}[]
\CommentTok{\# Построение обобщенной линейной модели GLM}
\NormalTok{glm\_model }\OtherTok{\textless{}{-}} \FunctionTok{glm}\NormalTok{(R3haddock }\SpecialCharTok{\textasciitilde{}}\NormalTok{ ., }
                \AttributeTok{family =} \FunctionTok{Gamma}\NormalTok{(}\AttributeTok{link =} \StringTok{"log"}\NormalTok{), }
                \AttributeTok{data =}\NormalTok{ model\_data)}
\FunctionTok{summary}\NormalTok{(glm\_model)}
\end{Highlighting}
\end{Shaded}

\begin{verbatim}

Call:
glm(formula = R3haddock ~ ., family = Gamma(link = "log"), data = model_data)

Coefficients:
              Estimate Std. Error t value Pr(>|t|)    
(Intercept)  2.375e+01  3.667e+01   0.648   0.5231    
YEAR        -8.601e-03  1.872e-02  -0.459   0.6499    
codTSB      -5.945e-07  2.342e-07  -2.539   0.0177 *  
T12          1.411e+00  2.917e-01   4.837 5.68e-05 ***
I5           7.430e-03  9.161e-03   0.811   0.4250    
NAOspring    1.508e-02  1.718e-01   0.088   0.9307    
haddock68    9.422e-07  1.643e-06   0.573   0.5715    
---
Signif. codes:  0 '***' 0.001 '**' 0.01 '*' 0.05 '.' 0.1 ' ' 1

(Dispersion parameter for Gamma family taken to be 0.326724)

    Null deviance: 19.8880  on 31  degrees of freedom
Residual deviance:  8.4535  on 25  degrees of freedom
AIC: 857.03

Number of Fisher Scoring iterations: 9
\end{verbatim}

\begin{Shaded}
\begin{Highlighting}[]
\CommentTok{\# Построение обобщенной аддитивной модели GАM}
\FunctionTok{library}\NormalTok{(mgcv)}
\NormalTok{gam\_model }\OtherTok{\textless{}{-}} \FunctionTok{gam}\NormalTok{(R3haddock }\SpecialCharTok{\textasciitilde{}} 
                 \FunctionTok{s}\NormalTok{(codTSB) }\SpecialCharTok{+} 
                 \FunctionTok{s}\NormalTok{(T12) }\SpecialCharTok{+} 
                 \FunctionTok{s}\NormalTok{(I5) }\SpecialCharTok{+} 
                 \FunctionTok{s}\NormalTok{(NAOspring) }\SpecialCharTok{+} 
                 \FunctionTok{s}\NormalTok{(haddock68),}
               \AttributeTok{data =}\NormalTok{ model\_data,}
               \AttributeTok{method =} \StringTok{"REML"}\NormalTok{)}
\FunctionTok{summary}\NormalTok{(gam\_model)}
\end{Highlighting}
\end{Shaded}

\begin{verbatim}

Family: gaussian 
Link function: identity 

Formula:
R3haddock ~ s(codTSB) + s(T12) + s(I5) + s(NAOspring) + s(haddock68)

Parametric coefficients:
            Estimate Std. Error t value Pr(>|t|)    
(Intercept)   320163      23724   13.49 4.37e-11 ***
---
Signif. codes:  0 '***' 0.001 '**' 0.01 '*' 0.05 '.' 0.1 ' ' 1

Approximate significance of smooth terms:
               edf Ref.df     F p-value   
s(codTSB)    2.354  2.899 1.684 0.17581   
s(T12)       2.190  2.676 5.908 0.00357 **
s(I5)        4.642  5.539 1.518 0.15004   
s(NAOspring) 1.293  1.503 1.154 0.22854   
s(haddock68) 1.824  2.200 0.629 0.65124   
---
Signif. codes:  0 '***' 0.001 '**' 0.01 '*' 0.05 '.' 0.1 ' ' 1

R-sq.(adj) =  0.739   Deviance explained = 84.2%
-REML = 361.62  Scale est. = 1.8011e+10  n = 32
\end{verbatim}

\begin{Shaded}
\begin{Highlighting}[]
\FunctionTok{plot}\NormalTok{(gam\_model, }\AttributeTok{pages =} \DecValTok{1}\NormalTok{, }\AttributeTok{residuals =} \ConstantTok{TRUE}\NormalTok{)}
\end{Highlighting}
\end{Shaded}

\pandocbounded{\includegraphics[keepaspectratio]{chapter7_files/figure-pdf/unnamed-chunk-3-3.pdf}}

\begin{Shaded}
\begin{Highlighting}[]
\CommentTok{\# Таблица сравнения моделей}
\CommentTok{\# Сравнение моделей}
\NormalTok{model\_comparison }\OtherTok{\textless{}{-}} \FunctionTok{data.frame}\NormalTok{(}
  \AttributeTok{Model =} \FunctionTok{c}\NormalTok{(}\StringTok{"Рикер"}\NormalTok{, }\StringTok{"Бивертон{-}Холт"}\NormalTok{, }\StringTok{"LM"}\NormalTok{, }\StringTok{"GLM"}\NormalTok{, }\StringTok{"GAM"}\NormalTok{),}
  \AttributeTok{AIC =} \FunctionTok{c}\NormalTok{(}\FunctionTok{AIC}\NormalTok{(ricker\_model), }\FunctionTok{AIC}\NormalTok{(bh\_model), }\FunctionTok{AIC}\NormalTok{(lm\_model), }\FunctionTok{AIC}\NormalTok{(glm\_model), }\FunctionTok{AIC}\NormalTok{(gam\_model)),}
  \AttributeTok{R2 =} \FunctionTok{c}\NormalTok{(ricker\_r2}\SpecialCharTok{$}\NormalTok{R2, bh\_r2}\SpecialCharTok{$}\NormalTok{R2, }\FunctionTok{summary}\NormalTok{(lm\_model)}\SpecialCharTok{$}\NormalTok{r.squared, }
         \FunctionTok{cor}\NormalTok{(model\_data}\SpecialCharTok{$}\NormalTok{R3haddock, }\FunctionTok{predict}\NormalTok{(glm\_model))}\SpecialCharTok{\^{}}\DecValTok{2}\NormalTok{, }
         \FunctionTok{summary}\NormalTok{(gam\_model)}\SpecialCharTok{$}\NormalTok{r.sq),  }\CommentTok{\# Используем summary(gam\_model)$r.sq для R\^{}2}
  \AttributeTok{Adj\_R2 =} \FunctionTok{c}\NormalTok{(ricker\_r2}\SpecialCharTok{$}\NormalTok{adj\_R2, bh\_r2}\SpecialCharTok{$}\NormalTok{adj\_R2, }\FunctionTok{summary}\NormalTok{(lm\_model)}\SpecialCharTok{$}\NormalTok{adj.r.squared, }\ConstantTok{NA}\NormalTok{, }
             \FunctionTok{summary}\NormalTok{(gam\_model)}\SpecialCharTok{$}\NormalTok{r.sq),  }\CommentTok{\# Используем summary(gam\_model)$r.sq для Adjusted R\^{}2}
  \AttributeTok{RMSE =} \FunctionTok{c}\NormalTok{(}\FunctionTok{RMSE}\NormalTok{(}\FunctionTok{predict}\NormalTok{(ricker\_model), rec\_data}\SpecialCharTok{$}\NormalTok{R), }
           \FunctionTok{RMSE}\NormalTok{(}\FunctionTok{predict}\NormalTok{(bh\_model), rec\_data}\SpecialCharTok{$}\NormalTok{R),}
           \FunctionTok{RMSE}\NormalTok{(}\FunctionTok{predict}\NormalTok{(lm\_model), model\_data}\SpecialCharTok{$}\NormalTok{R3haddock),}
           \FunctionTok{RMSE}\NormalTok{(}\FunctionTok{predict}\NormalTok{(glm\_model, }\AttributeTok{type =} \StringTok{"response"}\NormalTok{), model\_data}\SpecialCharTok{$}\NormalTok{R3haddock),}
           \FunctionTok{RMSE}\NormalTok{(}\FunctionTok{predict}\NormalTok{(gam\_model, }\AttributeTok{type =} \StringTok{"response"}\NormalTok{), model\_data}\SpecialCharTok{$}\NormalTok{R3haddock))}
\NormalTok{)}

\CommentTok{\# Вывод таблицы}
\FunctionTok{print}\NormalTok{(model\_comparison)}
\end{Highlighting}
\end{Shaded}

\begin{verbatim}
          Model      AIC          R2       Adj_R2     RMSE
1         Рикер 891.9919 0.072305265  0.008326318 248869.6
2 Бивертон-Холт 894.2029 0.005938625 -0.062617332 257617.8
3            LM 877.0895 0.573972535  0.471725944 168650.7
4           GLM 857.0346 0.502625978           NA 170386.6
5           GAM 862.9064 0.738660678  0.738660678 102584.0
\end{verbatim}

\begin{Shaded}
\begin{Highlighting}[]
\CommentTok{\# ==============================================================================}
\CommentTok{\# ВИЗУАЛИЗАЦИЯ ВСЕХ МОДЕЛЕЙ НА ОДНОМ ГРАФИКЕ }
\CommentTok{\# ==============================================================================}

\CommentTok{\# Фиксируем другие предикторы на их средних значениях (исключая haddock68)}
\NormalTok{mean\_values }\OtherTok{\textless{}{-}}\NormalTok{ model\_data }\SpecialCharTok{\%\textgreater{}\%}
  \FunctionTok{select}\NormalTok{(}\SpecialCharTok{{-}}\NormalTok{R3haddock, }\SpecialCharTok{{-}}\NormalTok{haddock68) }\SpecialCharTok{\%\textgreater{}\%}
  \FunctionTok{summarise}\NormalTok{(}\FunctionTok{across}\NormalTok{(}\FunctionTok{everything}\NormalTok{(), }\SpecialCharTok{\textasciitilde{}} \FunctionTok{mean}\NormalTok{(.x, }\AttributeTok{na.rm =} \ConstantTok{TRUE}\NormalTok{)))}

\CommentTok{\# Расширяем new\_data, добавляя средние значения других предикторов}
\NormalTok{new\_data\_full }\OtherTok{\textless{}{-}}\NormalTok{ new\_data }\SpecialCharTok{\%\textgreater{}\%}
  \FunctionTok{bind\_cols}\NormalTok{(mean\_values[}\FunctionTok{rep}\NormalTok{(}\DecValTok{1}\NormalTok{, }\FunctionTok{nrow}\NormalTok{(new\_data)), ]) }\SpecialCharTok{\%\textgreater{}\%}
  \FunctionTok{rename}\NormalTok{(}\AttributeTok{haddock68 =}\NormalTok{ S)  }\CommentTok{\# Переименовываем S в haddock68 для совместимости}

\CommentTok{\# Получаем предсказания для всех моделей}
\NormalTok{new\_data\_full }\OtherTok{\textless{}{-}}\NormalTok{ new\_data\_full }\SpecialCharTok{\%\textgreater{}\%}
  \FunctionTok{mutate}\NormalTok{(}
    \CommentTok{\# Предсказания для моделей запаса{-}пополнения}
    \AttributeTok{ricker\_pred =} \FunctionTok{predict}\NormalTok{(ricker\_model, }\AttributeTok{newdata =} \FunctionTok{data.frame}\NormalTok{(}\AttributeTok{S =}\NormalTok{ haddock68)),}
    \AttributeTok{bh\_pred =} \FunctionTok{predict}\NormalTok{(bh\_model, }\AttributeTok{newdata =} \FunctionTok{data.frame}\NormalTok{(}\AttributeTok{S =}\NormalTok{ haddock68)),}
    
    \CommentTok{\# Предсказания для линейной модели (LM)}
    \AttributeTok{lm\_pred =} \FunctionTok{predict}\NormalTok{(lm\_model, }\AttributeTok{newdata =}\NormalTok{ .),}
    
    \CommentTok{\# Предсказания для обобщенной линейной модели (GLM)}
    \AttributeTok{glm\_pred =} \FunctionTok{predict}\NormalTok{(glm\_model, }\AttributeTok{newdata =}\NormalTok{ ., }\AttributeTok{type =} \StringTok{"response"}\NormalTok{),}
    
    \CommentTok{\# Предсказания для обобщенной аддитивной модели (GAM)}
    \AttributeTok{gam\_pred =} \FunctionTok{predict}\NormalTok{(gam\_model, }\AttributeTok{newdata =}\NormalTok{ ., }\AttributeTok{type =} \StringTok{"response"}\NormalTok{)}
\NormalTok{  )}

\CommentTok{\# Создаем длинный формат данных для ggplot}
\NormalTok{plot\_data }\OtherTok{\textless{}{-}}\NormalTok{ new\_data\_full }\SpecialCharTok{\%\textgreater{}\%}
  \FunctionTok{select}\NormalTok{(haddock68, ricker\_pred, bh\_pred, lm\_pred, glm\_pred, gam\_pred) }\SpecialCharTok{\%\textgreater{}\%}
  \FunctionTok{pivot\_longer}\NormalTok{(}
    \AttributeTok{cols =} \SpecialCharTok{{-}}\NormalTok{haddock68,}
    \AttributeTok{names\_to =} \StringTok{"model"}\NormalTok{,}
    \AttributeTok{values\_to =} \StringTok{"prediction"}
\NormalTok{  ) }\SpecialCharTok{\%\textgreater{}\%}
  \FunctionTok{mutate}\NormalTok{(}
    \AttributeTok{model =} \FunctionTok{case\_when}\NormalTok{(}
\NormalTok{      model }\SpecialCharTok{==} \StringTok{"ricker\_pred"} \SpecialCharTok{\textasciitilde{}} \StringTok{"Рикер"}\NormalTok{,}
\NormalTok{      model }\SpecialCharTok{==} \StringTok{"bh\_pred"} \SpecialCharTok{\textasciitilde{}} \StringTok{"Бивертон{-}Холт"}\NormalTok{,}
\NormalTok{      model }\SpecialCharTok{==} \StringTok{"lm\_pred"} \SpecialCharTok{\textasciitilde{}} \StringTok{"Линейная (LM)"}\NormalTok{,}
\NormalTok{      model }\SpecialCharTok{==} \StringTok{"glm\_pred"} \SpecialCharTok{\textasciitilde{}} \StringTok{"Обобщенная линейная (GLM)"}\NormalTok{,}
\NormalTok{      model }\SpecialCharTok{==} \StringTok{"gam\_pred"} \SpecialCharTok{\textasciitilde{}} \StringTok{"Обобщенная аддитивная (GAM)"}\NormalTok{,}
      \ConstantTok{TRUE} \SpecialCharTok{\textasciitilde{}}\NormalTok{ model}
\NormalTok{    )}
\NormalTok{  )}

\CommentTok{\# Создаем палитру цветов для моделей}
\NormalTok{model\_colors }\OtherTok{\textless{}{-}} \FunctionTok{c}\NormalTok{(}
  \StringTok{"Рикер"} \OtherTok{=} \StringTok{"\#E41A1C"}\NormalTok{,          }\CommentTok{\# Красный}
  \StringTok{"Бивертон{-}Холт"} \OtherTok{=} \StringTok{"\#377EB8"}\NormalTok{,  }\CommentTok{\# Синий}
  \StringTok{"Линейная (LM)"} \OtherTok{=} \StringTok{"\#4DAF4A"}\NormalTok{,  }\CommentTok{\# Зеленый}
  \StringTok{"Обобщенная линейная (GLM)"} \OtherTok{=} \StringTok{"\#984EA3"}\NormalTok{, }\CommentTok{\# Фиолетовый}
  \StringTok{"Обобщенная аддитивная (GAM)"} \OtherTok{=} \StringTok{"\#FF7F00"} \CommentTok{\# Оранжевый}
\NormalTok{)}

\CommentTok{\# Создаем график}
\FunctionTok{ggplot}\NormalTok{() }\SpecialCharTok{+}
  \CommentTok{\# Точки исходных данных}
  \FunctionTok{geom\_point}\NormalTok{(}\AttributeTok{data =}\NormalTok{ rec\_data, }\FunctionTok{aes}\NormalTok{(}\AttributeTok{x =}\NormalTok{ S, }\AttributeTok{y =}\NormalTok{ R), }
             \AttributeTok{color =} \StringTok{"darkgray"}\NormalTok{, }\AttributeTok{size =} \FloatTok{2.5}\NormalTok{, }\AttributeTok{alpha =} \FloatTok{0.7}\NormalTok{) }\SpecialCharTok{+}
  
  \CommentTok{\# Линии предсказаний моделей}
  \FunctionTok{geom\_line}\NormalTok{(}\AttributeTok{data =}\NormalTok{ plot\_data, }
            \FunctionTok{aes}\NormalTok{(}\AttributeTok{x =}\NormalTok{ haddock68, }\AttributeTok{y =}\NormalTok{ prediction, }\AttributeTok{color =}\NormalTok{ model, }\AttributeTok{linetype =}\NormalTok{ model),}
            \AttributeTok{linewidth =} \FloatTok{1.2}\NormalTok{) }\SpecialCharTok{+}
  
  \CommentTok{\# Настройка цветов и типов линий}
  \FunctionTok{scale\_color\_manual}\NormalTok{(}\AttributeTok{values =}\NormalTok{ model\_colors) }\SpecialCharTok{+}
  \FunctionTok{scale\_linetype\_manual}\NormalTok{(}\AttributeTok{values =} \FunctionTok{c}\NormalTok{(}
    \StringTok{"Рикер"} \OtherTok{=} \StringTok{"solid"}\NormalTok{,}
    \StringTok{"Бивертон{-}Холт"} \OtherTok{=} \StringTok{"dashed"}\NormalTok{,}
    \StringTok{"Линейная (LM)"} \OtherTok{=} \StringTok{"dotdash"}\NormalTok{,}
    \StringTok{"Обобщенная линейная (GLM)"} \OtherTok{=} \StringTok{"longdash"}\NormalTok{,}
    \StringTok{"Обобщенная аддитивная (GAM)"} \OtherTok{=} \StringTok{"twodash"}
\NormalTok{  )) }\SpecialCharTok{+}
  
  \CommentTok{\# Подписи и темы}
  \FunctionTok{labs}\NormalTok{(}
    \AttributeTok{title =} \StringTok{"Сравнение моделей зависимости пополнения от нерестового запаса"}\NormalTok{,}
    \AttributeTok{subtitle =} \StringTok{"Фиксация других предикторов на средних значениях"}\NormalTok{,}
    \AttributeTok{x =} \StringTok{"Нерестовый запас (тыс. тонн)"}\NormalTok{,}
    \AttributeTok{y =} \StringTok{"Пополнение (млн особей)"}\NormalTok{,}
    \AttributeTok{color =} \StringTok{"Модель"}\NormalTok{,}
    \AttributeTok{linetype =} \StringTok{"Модель"}
\NormalTok{  ) }\SpecialCharTok{+}
  \FunctionTok{theme\_minimal}\NormalTok{(}\AttributeTok{base\_size =} \DecValTok{14}\NormalTok{) }\SpecialCharTok{+}
  \FunctionTok{theme}\NormalTok{(}
    \AttributeTok{plot.title =} \FunctionTok{element\_text}\NormalTok{(}\AttributeTok{face =} \StringTok{"bold"}\NormalTok{, }\AttributeTok{size =} \DecValTok{16}\NormalTok{, }\AttributeTok{hjust =} \FloatTok{0.5}\NormalTok{),}
    \AttributeTok{plot.subtitle =} \FunctionTok{element\_text}\NormalTok{(}\AttributeTok{size =} \DecValTok{12}\NormalTok{, }\AttributeTok{hjust =} \FloatTok{0.5}\NormalTok{, }\AttributeTok{color =} \StringTok{"gray30"}\NormalTok{),}
    \AttributeTok{axis.title =} \FunctionTok{element\_text}\NormalTok{(}\AttributeTok{size =} \DecValTok{12}\NormalTok{),}
    \AttributeTok{legend.position =} \StringTok{"bottom"}\NormalTok{,}
    \AttributeTok{legend.box =} \StringTok{"horizontal"}\NormalTok{,}
    \AttributeTok{legend.title =} \FunctionTok{element\_text}\NormalTok{(}\AttributeTok{face =} \StringTok{"bold"}\NormalTok{),}
    \AttributeTok{panel.grid.minor =} \FunctionTok{element\_blank}\NormalTok{(),}
    \AttributeTok{panel.border =} \FunctionTok{element\_rect}\NormalTok{(}\AttributeTok{color =} \StringTok{"gray80"}\NormalTok{, }\AttributeTok{fill =} \ConstantTok{NA}\NormalTok{, }\AttributeTok{linewidth =} \FloatTok{0.5}\NormalTok{)}
\NormalTok{  ) }\SpecialCharTok{+}
  \FunctionTok{guides}\NormalTok{(}
    \AttributeTok{color =} \FunctionTok{guide\_legend}\NormalTok{(}\AttributeTok{nrow =} \DecValTok{2}\NormalTok{, }\AttributeTok{byrow =} \ConstantTok{TRUE}\NormalTok{),}
    \AttributeTok{linetype =} \FunctionTok{guide\_legend}\NormalTok{(}\AttributeTok{nrow =} \DecValTok{2}\NormalTok{, }\AttributeTok{byrow =} \ConstantTok{TRUE}\NormalTok{)}
\NormalTok{  )}
\end{Highlighting}
\end{Shaded}

\begin{verbatim}
Warning in grid.Call(C_textBounds, as.graphicsAnnot(x$label), x$x, x$y, :
неизвестна ширина символа 0xc1 в кодировке CP1251
\end{verbatim}

\begin{verbatim}
Warning in grid.Call(C_textBounds, as.graphicsAnnot(x$label), x$x, x$y, :
неизвестна ширина символа 0xe8 в кодировке CP1251
\end{verbatim}

\begin{verbatim}
Warning in grid.Call(C_textBounds, as.graphicsAnnot(x$label), x$x, x$y, :
неизвестна ширина символа 0xe2 в кодировке CP1251
\end{verbatim}

\begin{verbatim}
Warning in grid.Call(C_textBounds, as.graphicsAnnot(x$label), x$x, x$y, :
неизвестна ширина символа 0xe5 в кодировке CP1251
\end{verbatim}

\begin{verbatim}
Warning in grid.Call(C_textBounds, as.graphicsAnnot(x$label), x$x, x$y, :
неизвестна ширина символа 0xf0 в кодировке CP1251
\end{verbatim}

\begin{verbatim}
Warning in grid.Call(C_textBounds, as.graphicsAnnot(x$label), x$x, x$y, :
неизвестна ширина символа 0xf2 в кодировке CP1251
\end{verbatim}

\begin{verbatim}
Warning in grid.Call(C_textBounds, as.graphicsAnnot(x$label), x$x, x$y, :
неизвестна ширина символа 0xee в кодировке CP1251
\end{verbatim}

\begin{verbatim}
Warning in grid.Call(C_textBounds, as.graphicsAnnot(x$label), x$x, x$y, :
неизвестна ширина символа 0xed в кодировке CP1251
\end{verbatim}

\begin{verbatim}
Warning in grid.Call(C_textBounds, as.graphicsAnnot(x$label), x$x, x$y, :
неизвестна ширина символа 0xd5 в кодировке CP1251
\end{verbatim}

\begin{verbatim}
Warning in grid.Call(C_textBounds, as.graphicsAnnot(x$label), x$x, x$y, :
неизвестна ширина символа 0xee в кодировке CP1251
\end{verbatim}

\begin{verbatim}
Warning in grid.Call(C_textBounds, as.graphicsAnnot(x$label), x$x, x$y, :
неизвестна ширина символа 0xeb в кодировке CP1251
\end{verbatim}

\begin{verbatim}
Warning in grid.Call(C_textBounds, as.graphicsAnnot(x$label), x$x, x$y, :
неизвестна ширина символа 0xf2 в кодировке CP1251
\end{verbatim}

\begin{verbatim}
Warning in grid.Call(C_textBounds, as.graphicsAnnot(x$label), x$x, x$y, :
неизвестна ширина символа 0xcb в кодировке CP1251
\end{verbatim}

\begin{verbatim}
Warning in grid.Call(C_textBounds, as.graphicsAnnot(x$label), x$x, x$y, :
неизвестна ширина символа 0xe8 в кодировке CP1251
\end{verbatim}

\begin{verbatim}
Warning in grid.Call(C_textBounds, as.graphicsAnnot(x$label), x$x, x$y, :
неизвестна ширина символа 0xed в кодировке CP1251
\end{verbatim}

\begin{verbatim}
Warning in grid.Call(C_textBounds, as.graphicsAnnot(x$label), x$x, x$y, :
неизвестна ширина символа 0xe5 в кодировке CP1251
\end{verbatim}

\begin{verbatim}
Warning in grid.Call(C_textBounds, as.graphicsAnnot(x$label), x$x, x$y, :
неизвестна ширина символа 0xe9 в кодировке CP1251
\end{verbatim}

\begin{verbatim}
Warning in grid.Call(C_textBounds, as.graphicsAnnot(x$label), x$x, x$y, :
неизвестна ширина символа 0xed в кодировке CP1251
\end{verbatim}

\begin{verbatim}
Warning in grid.Call(C_textBounds, as.graphicsAnnot(x$label), x$x, x$y, :
неизвестна ширина символа 0xe0 в кодировке CP1251
\end{verbatim}

\begin{verbatim}
Warning in grid.Call(C_textBounds, as.graphicsAnnot(x$label), x$x, x$y, :
неизвестна ширина символа 0xff в кодировке CP1251
\end{verbatim}

\begin{verbatim}
Warning in grid.Call(C_textBounds, as.graphicsAnnot(x$label), x$x, x$y, :
неизвестна ширина символа 0xce в кодировке CP1251
\end{verbatim}

\begin{verbatim}
Warning in grid.Call(C_textBounds, as.graphicsAnnot(x$label), x$x, x$y, :
неизвестна ширина символа 0xe1 в кодировке CP1251
\end{verbatim}

\begin{verbatim}
Warning in grid.Call(C_textBounds, as.graphicsAnnot(x$label), x$x, x$y, :
неизвестна ширина символа 0xee в кодировке CP1251
\end{verbatim}

\begin{verbatim}
Warning in grid.Call(C_textBounds, as.graphicsAnnot(x$label), x$x, x$y, :
неизвестна ширина символа 0xe1 в кодировке CP1251
\end{verbatim}

\begin{verbatim}
Warning in grid.Call(C_textBounds, as.graphicsAnnot(x$label), x$x, x$y, :
неизвестна ширина символа 0xf9 в кодировке CP1251
\end{verbatim}

\begin{verbatim}
Warning in grid.Call(C_textBounds, as.graphicsAnnot(x$label), x$x, x$y, :
неизвестна ширина символа 0xe5 в кодировке CP1251
\end{verbatim}

\begin{verbatim}
Warning in grid.Call(C_textBounds, as.graphicsAnnot(x$label), x$x, x$y, :
неизвестна ширина символа 0xed в кодировке CP1251
Warning in grid.Call(C_textBounds, as.graphicsAnnot(x$label), x$x, x$y, :
неизвестна ширина символа 0xed в кодировке CP1251
\end{verbatim}

\begin{verbatim}
Warning in grid.Call(C_textBounds, as.graphicsAnnot(x$label), x$x, x$y, :
неизвестна ширина символа 0xe0 в кодировке CP1251
\end{verbatim}

\begin{verbatim}
Warning in grid.Call(C_textBounds, as.graphicsAnnot(x$label), x$x, x$y, :
неизвестна ширина символа 0xff в кодировке CP1251
\end{verbatim}

\begin{verbatim}
Warning in grid.Call(C_textBounds, as.graphicsAnnot(x$label), x$x, x$y, :
неизвестна ширина символа 0xe0 в кодировке CP1251
\end{verbatim}

\begin{verbatim}
Warning in grid.Call(C_textBounds, as.graphicsAnnot(x$label), x$x, x$y, :
неизвестна ширина символа 0xe4 в кодировке CP1251
Warning in grid.Call(C_textBounds, as.graphicsAnnot(x$label), x$x, x$y, :
неизвестна ширина символа 0xe4 в кодировке CP1251
\end{verbatim}

\begin{verbatim}
Warning in grid.Call(C_textBounds, as.graphicsAnnot(x$label), x$x, x$y, :
неизвестна ширина символа 0xe8 в кодировке CP1251
\end{verbatim}

\begin{verbatim}
Warning in grid.Call(C_textBounds, as.graphicsAnnot(x$label), x$x, x$y, :
неизвестна ширина символа 0xf2 в кодировке CP1251
\end{verbatim}

\begin{verbatim}
Warning in grid.Call(C_textBounds, as.graphicsAnnot(x$label), x$x, x$y, :
неизвестна ширина символа 0xe8 в кодировке CP1251
\end{verbatim}

\begin{verbatim}
Warning in grid.Call(C_textBounds, as.graphicsAnnot(x$label), x$x, x$y, :
неизвестна ширина символа 0xe2 в кодировке CP1251
\end{verbatim}

\begin{verbatim}
Warning in grid.Call(C_textBounds, as.graphicsAnnot(x$label), x$x, x$y, :
неизвестна ширина символа 0xed в кодировке CP1251
\end{verbatim}

\begin{verbatim}
Warning in grid.Call(C_textBounds, as.graphicsAnnot(x$label), x$x, x$y, :
неизвестна ширина символа 0xe0 в кодировке CP1251
\end{verbatim}

\begin{verbatim}
Warning in grid.Call(C_textBounds, as.graphicsAnnot(x$label), x$x, x$y, :
неизвестна ширина символа 0xff в кодировке CP1251
\end{verbatim}

\begin{verbatim}
Warning in grid.Call(C_textBounds, as.graphicsAnnot(x$label), x$x, x$y, :
неизвестна ширина символа 0xce в кодировке CP1251
\end{verbatim}

\begin{verbatim}
Warning in grid.Call(C_textBounds, as.graphicsAnnot(x$label), x$x, x$y, :
неизвестна ширина символа 0xe1 в кодировке CP1251
\end{verbatim}

\begin{verbatim}
Warning in grid.Call(C_textBounds, as.graphicsAnnot(x$label), x$x, x$y, :
неизвестна ширина символа 0xee в кодировке CP1251
\end{verbatim}

\begin{verbatim}
Warning in grid.Call(C_textBounds, as.graphicsAnnot(x$label), x$x, x$y, :
неизвестна ширина символа 0xe1 в кодировке CP1251
\end{verbatim}

\begin{verbatim}
Warning in grid.Call(C_textBounds, as.graphicsAnnot(x$label), x$x, x$y, :
неизвестна ширина символа 0xf9 в кодировке CP1251
\end{verbatim}

\begin{verbatim}
Warning in grid.Call(C_textBounds, as.graphicsAnnot(x$label), x$x, x$y, :
неизвестна ширина символа 0xe5 в кодировке CP1251
\end{verbatim}

\begin{verbatim}
Warning in grid.Call(C_textBounds, as.graphicsAnnot(x$label), x$x, x$y, :
неизвестна ширина символа 0xed в кодировке CP1251
Warning in grid.Call(C_textBounds, as.graphicsAnnot(x$label), x$x, x$y, :
неизвестна ширина символа 0xed в кодировке CP1251
\end{verbatim}

\begin{verbatim}
Warning in grid.Call(C_textBounds, as.graphicsAnnot(x$label), x$x, x$y, :
неизвестна ширина символа 0xe0 в кодировке CP1251
\end{verbatim}

\begin{verbatim}
Warning in grid.Call(C_textBounds, as.graphicsAnnot(x$label), x$x, x$y, :
неизвестна ширина символа 0xff в кодировке CP1251
\end{verbatim}

\begin{verbatim}
Warning in grid.Call(C_textBounds, as.graphicsAnnot(x$label), x$x, x$y, :
неизвестна ширина символа 0xeb в кодировке CP1251
\end{verbatim}

\begin{verbatim}
Warning in grid.Call(C_textBounds, as.graphicsAnnot(x$label), x$x, x$y, :
неизвестна ширина символа 0xe8 в кодировке CP1251
\end{verbatim}

\begin{verbatim}
Warning in grid.Call(C_textBounds, as.graphicsAnnot(x$label), x$x, x$y, :
неизвестна ширина символа 0xed в кодировке CP1251
\end{verbatim}

\begin{verbatim}
Warning in grid.Call(C_textBounds, as.graphicsAnnot(x$label), x$x, x$y, :
неизвестна ширина символа 0xe5 в кодировке CP1251
\end{verbatim}

\begin{verbatim}
Warning in grid.Call(C_textBounds, as.graphicsAnnot(x$label), x$x, x$y, :
неизвестна ширина символа 0xe9 в кодировке CP1251
\end{verbatim}

\begin{verbatim}
Warning in grid.Call(C_textBounds, as.graphicsAnnot(x$label), x$x, x$y, :
неизвестна ширина символа 0xed в кодировке CP1251
\end{verbatim}

\begin{verbatim}
Warning in grid.Call(C_textBounds, as.graphicsAnnot(x$label), x$x, x$y, :
неизвестна ширина символа 0xe0 в кодировке CP1251
\end{verbatim}

\begin{verbatim}
Warning in grid.Call(C_textBounds, as.graphicsAnnot(x$label), x$x, x$y, :
неизвестна ширина символа 0xff в кодировке CP1251
\end{verbatim}

\begin{verbatim}
Warning in grid.Call(C_textBounds, as.graphicsAnnot(x$label), x$x, x$y, :
неизвестна ширина символа 0xd0 в кодировке CP1251
\end{verbatim}

\begin{verbatim}
Warning in grid.Call(C_textBounds, as.graphicsAnnot(x$label), x$x, x$y, :
неизвестна ширина символа 0xe8 в кодировке CP1251
\end{verbatim}

\begin{verbatim}
Warning in grid.Call(C_textBounds, as.graphicsAnnot(x$label), x$x, x$y, :
неизвестна ширина символа 0xea в кодировке CP1251
\end{verbatim}

\begin{verbatim}
Warning in grid.Call(C_textBounds, as.graphicsAnnot(x$label), x$x, x$y, :
неизвестна ширина символа 0xe5 в кодировке CP1251
\end{verbatim}

\begin{verbatim}
Warning in grid.Call(C_textBounds, as.graphicsAnnot(x$label), x$x, x$y, :
неизвестна ширина символа 0xf0 в кодировке CP1251
\end{verbatim}

\begin{verbatim}
Warning in grid.Call(C_textBounds, as.graphicsAnnot(x$label), x$x, x$y, :
неизвестна ширина символа 0xc1 в кодировке CP1251
\end{verbatim}

\begin{verbatim}
Warning in grid.Call(C_textBounds, as.graphicsAnnot(x$label), x$x, x$y, :
неизвестна ширина символа 0xe8 в кодировке CP1251
\end{verbatim}

\begin{verbatim}
Warning in grid.Call(C_textBounds, as.graphicsAnnot(x$label), x$x, x$y, :
неизвестна ширина символа 0xe2 в кодировке CP1251
\end{verbatim}

\begin{verbatim}
Warning in grid.Call(C_textBounds, as.graphicsAnnot(x$label), x$x, x$y, :
неизвестна ширина символа 0xe5 в кодировке CP1251
\end{verbatim}

\begin{verbatim}
Warning in grid.Call(C_textBounds, as.graphicsAnnot(x$label), x$x, x$y, :
неизвестна ширина символа 0xf0 в кодировке CP1251
\end{verbatim}

\begin{verbatim}
Warning in grid.Call(C_textBounds, as.graphicsAnnot(x$label), x$x, x$y, :
неизвестна ширина символа 0xf2 в кодировке CP1251
\end{verbatim}

\begin{verbatim}
Warning in grid.Call(C_textBounds, as.graphicsAnnot(x$label), x$x, x$y, :
неизвестна ширина символа 0xee в кодировке CP1251
\end{verbatim}

\begin{verbatim}
Warning in grid.Call(C_textBounds, as.graphicsAnnot(x$label), x$x, x$y, :
неизвестна ширина символа 0xed в кодировке CP1251
\end{verbatim}

\begin{verbatim}
Warning in grid.Call(C_textBounds, as.graphicsAnnot(x$label), x$x, x$y, :
неизвестна ширина символа 0xd5 в кодировке CP1251
\end{verbatim}

\begin{verbatim}
Warning in grid.Call(C_textBounds, as.graphicsAnnot(x$label), x$x, x$y, :
неизвестна ширина символа 0xee в кодировке CP1251
\end{verbatim}

\begin{verbatim}
Warning in grid.Call(C_textBounds, as.graphicsAnnot(x$label), x$x, x$y, :
неизвестна ширина символа 0xeb в кодировке CP1251
\end{verbatim}

\begin{verbatim}
Warning in grid.Call(C_textBounds, as.graphicsAnnot(x$label), x$x, x$y, :
неизвестна ширина символа 0xf2 в кодировке CP1251
\end{verbatim}

\begin{verbatim}
Warning in grid.Call(C_textBounds, as.graphicsAnnot(x$label), x$x, x$y, :
неизвестна ширина символа 0xcb в кодировке CP1251
\end{verbatim}

\begin{verbatim}
Warning in grid.Call(C_textBounds, as.graphicsAnnot(x$label), x$x, x$y, :
неизвестна ширина символа 0xe8 в кодировке CP1251
\end{verbatim}

\begin{verbatim}
Warning in grid.Call(C_textBounds, as.graphicsAnnot(x$label), x$x, x$y, :
неизвестна ширина символа 0xed в кодировке CP1251
\end{verbatim}

\begin{verbatim}
Warning in grid.Call(C_textBounds, as.graphicsAnnot(x$label), x$x, x$y, :
неизвестна ширина символа 0xe5 в кодировке CP1251
\end{verbatim}

\begin{verbatim}
Warning in grid.Call(C_textBounds, as.graphicsAnnot(x$label), x$x, x$y, :
неизвестна ширина символа 0xe9 в кодировке CP1251
\end{verbatim}

\begin{verbatim}
Warning in grid.Call(C_textBounds, as.graphicsAnnot(x$label), x$x, x$y, :
неизвестна ширина символа 0xed в кодировке CP1251
\end{verbatim}

\begin{verbatim}
Warning in grid.Call(C_textBounds, as.graphicsAnnot(x$label), x$x, x$y, :
неизвестна ширина символа 0xe0 в кодировке CP1251
\end{verbatim}

\begin{verbatim}
Warning in grid.Call(C_textBounds, as.graphicsAnnot(x$label), x$x, x$y, :
неизвестна ширина символа 0xff в кодировке CP1251
\end{verbatim}

\begin{verbatim}
Warning in grid.Call(C_textBounds, as.graphicsAnnot(x$label), x$x, x$y, :
неизвестна ширина символа 0xce в кодировке CP1251
\end{verbatim}

\begin{verbatim}
Warning in grid.Call(C_textBounds, as.graphicsAnnot(x$label), x$x, x$y, :
неизвестна ширина символа 0xe1 в кодировке CP1251
\end{verbatim}

\begin{verbatim}
Warning in grid.Call(C_textBounds, as.graphicsAnnot(x$label), x$x, x$y, :
неизвестна ширина символа 0xee в кодировке CP1251
\end{verbatim}

\begin{verbatim}
Warning in grid.Call(C_textBounds, as.graphicsAnnot(x$label), x$x, x$y, :
неизвестна ширина символа 0xe1 в кодировке CP1251
\end{verbatim}

\begin{verbatim}
Warning in grid.Call(C_textBounds, as.graphicsAnnot(x$label), x$x, x$y, :
неизвестна ширина символа 0xf9 в кодировке CP1251
\end{verbatim}

\begin{verbatim}
Warning in grid.Call(C_textBounds, as.graphicsAnnot(x$label), x$x, x$y, :
неизвестна ширина символа 0xe5 в кодировке CP1251
\end{verbatim}

\begin{verbatim}
Warning in grid.Call(C_textBounds, as.graphicsAnnot(x$label), x$x, x$y, :
неизвестна ширина символа 0xed в кодировке CP1251
Warning in grid.Call(C_textBounds, as.graphicsAnnot(x$label), x$x, x$y, :
неизвестна ширина символа 0xed в кодировке CP1251
\end{verbatim}

\begin{verbatim}
Warning in grid.Call(C_textBounds, as.graphicsAnnot(x$label), x$x, x$y, :
неизвестна ширина символа 0xe0 в кодировке CP1251
\end{verbatim}

\begin{verbatim}
Warning in grid.Call(C_textBounds, as.graphicsAnnot(x$label), x$x, x$y, :
неизвестна ширина символа 0xff в кодировке CP1251
\end{verbatim}

\begin{verbatim}
Warning in grid.Call(C_textBounds, as.graphicsAnnot(x$label), x$x, x$y, :
неизвестна ширина символа 0xe0 в кодировке CP1251
\end{verbatim}

\begin{verbatim}
Warning in grid.Call(C_textBounds, as.graphicsAnnot(x$label), x$x, x$y, :
неизвестна ширина символа 0xe4 в кодировке CP1251
Warning in grid.Call(C_textBounds, as.graphicsAnnot(x$label), x$x, x$y, :
неизвестна ширина символа 0xe4 в кодировке CP1251
\end{verbatim}

\begin{verbatim}
Warning in grid.Call(C_textBounds, as.graphicsAnnot(x$label), x$x, x$y, :
неизвестна ширина символа 0xe8 в кодировке CP1251
\end{verbatim}

\begin{verbatim}
Warning in grid.Call(C_textBounds, as.graphicsAnnot(x$label), x$x, x$y, :
неизвестна ширина символа 0xf2 в кодировке CP1251
\end{verbatim}

\begin{verbatim}
Warning in grid.Call(C_textBounds, as.graphicsAnnot(x$label), x$x, x$y, :
неизвестна ширина символа 0xe8 в кодировке CP1251
\end{verbatim}

\begin{verbatim}
Warning in grid.Call(C_textBounds, as.graphicsAnnot(x$label), x$x, x$y, :
неизвестна ширина символа 0xe2 в кодировке CP1251
\end{verbatim}

\begin{verbatim}
Warning in grid.Call(C_textBounds, as.graphicsAnnot(x$label), x$x, x$y, :
неизвестна ширина символа 0xed в кодировке CP1251
\end{verbatim}

\begin{verbatim}
Warning in grid.Call(C_textBounds, as.graphicsAnnot(x$label), x$x, x$y, :
неизвестна ширина символа 0xe0 в кодировке CP1251
\end{verbatim}

\begin{verbatim}
Warning in grid.Call(C_textBounds, as.graphicsAnnot(x$label), x$x, x$y, :
неизвестна ширина символа 0xff в кодировке CP1251
\end{verbatim}

\begin{verbatim}
Warning in grid.Call(C_textBounds, as.graphicsAnnot(x$label), x$x, x$y, :
неизвестна ширина символа 0xce в кодировке CP1251
\end{verbatim}

\begin{verbatim}
Warning in grid.Call(C_textBounds, as.graphicsAnnot(x$label), x$x, x$y, :
неизвестна ширина символа 0xe1 в кодировке CP1251
\end{verbatim}

\begin{verbatim}
Warning in grid.Call(C_textBounds, as.graphicsAnnot(x$label), x$x, x$y, :
неизвестна ширина символа 0xee в кодировке CP1251
\end{verbatim}

\begin{verbatim}
Warning in grid.Call(C_textBounds, as.graphicsAnnot(x$label), x$x, x$y, :
неизвестна ширина символа 0xe1 в кодировке CP1251
\end{verbatim}

\begin{verbatim}
Warning in grid.Call(C_textBounds, as.graphicsAnnot(x$label), x$x, x$y, :
неизвестна ширина символа 0xf9 в кодировке CP1251
\end{verbatim}

\begin{verbatim}
Warning in grid.Call(C_textBounds, as.graphicsAnnot(x$label), x$x, x$y, :
неизвестна ширина символа 0xe5 в кодировке CP1251
\end{verbatim}

\begin{verbatim}
Warning in grid.Call(C_textBounds, as.graphicsAnnot(x$label), x$x, x$y, :
неизвестна ширина символа 0xed в кодировке CP1251
Warning in grid.Call(C_textBounds, as.graphicsAnnot(x$label), x$x, x$y, :
неизвестна ширина символа 0xed в кодировке CP1251
\end{verbatim}

\begin{verbatim}
Warning in grid.Call(C_textBounds, as.graphicsAnnot(x$label), x$x, x$y, :
неизвестна ширина символа 0xe0 в кодировке CP1251
\end{verbatim}

\begin{verbatim}
Warning in grid.Call(C_textBounds, as.graphicsAnnot(x$label), x$x, x$y, :
неизвестна ширина символа 0xff в кодировке CP1251
\end{verbatim}

\begin{verbatim}
Warning in grid.Call(C_textBounds, as.graphicsAnnot(x$label), x$x, x$y, :
неизвестна ширина символа 0xeb в кодировке CP1251
\end{verbatim}

\begin{verbatim}
Warning in grid.Call(C_textBounds, as.graphicsAnnot(x$label), x$x, x$y, :
неизвестна ширина символа 0xe8 в кодировке CP1251
\end{verbatim}

\begin{verbatim}
Warning in grid.Call(C_textBounds, as.graphicsAnnot(x$label), x$x, x$y, :
неизвестна ширина символа 0xed в кодировке CP1251
\end{verbatim}

\begin{verbatim}
Warning in grid.Call(C_textBounds, as.graphicsAnnot(x$label), x$x, x$y, :
неизвестна ширина символа 0xe5 в кодировке CP1251
\end{verbatim}

\begin{verbatim}
Warning in grid.Call(C_textBounds, as.graphicsAnnot(x$label), x$x, x$y, :
неизвестна ширина символа 0xe9 в кодировке CP1251
\end{verbatim}

\begin{verbatim}
Warning in grid.Call(C_textBounds, as.graphicsAnnot(x$label), x$x, x$y, :
неизвестна ширина символа 0xed в кодировке CP1251
\end{verbatim}

\begin{verbatim}
Warning in grid.Call(C_textBounds, as.graphicsAnnot(x$label), x$x, x$y, :
неизвестна ширина символа 0xe0 в кодировке CP1251
\end{verbatim}

\begin{verbatim}
Warning in grid.Call(C_textBounds, as.graphicsAnnot(x$label), x$x, x$y, :
неизвестна ширина символа 0xff в кодировке CP1251
\end{verbatim}

\begin{verbatim}
Warning in grid.Call(C_textBounds, as.graphicsAnnot(x$label), x$x, x$y, :
неизвестна ширина символа 0xd0 в кодировке CP1251
\end{verbatim}

\begin{verbatim}
Warning in grid.Call(C_textBounds, as.graphicsAnnot(x$label), x$x, x$y, :
неизвестна ширина символа 0xe8 в кодировке CP1251
\end{verbatim}

\begin{verbatim}
Warning in grid.Call(C_textBounds, as.graphicsAnnot(x$label), x$x, x$y, :
неизвестна ширина символа 0xea в кодировке CP1251
\end{verbatim}

\begin{verbatim}
Warning in grid.Call(C_textBounds, as.graphicsAnnot(x$label), x$x, x$y, :
неизвестна ширина символа 0xe5 в кодировке CP1251
\end{verbatim}

\begin{verbatim}
Warning in grid.Call(C_textBounds, as.graphicsAnnot(x$label), x$x, x$y, :
неизвестна ширина символа 0xf0 в кодировке CP1251
\end{verbatim}

\begin{verbatim}
Warning in grid.Call(C_textBounds, as.graphicsAnnot(x$label), x$x, x$y, :
неизвестна ширина символа 0xcc в кодировке CP1251
\end{verbatim}

\begin{verbatim}
Warning in grid.Call(C_textBounds, as.graphicsAnnot(x$label), x$x, x$y, :
неизвестна ширина символа 0xee в кодировке CP1251
\end{verbatim}

\begin{verbatim}
Warning in grid.Call(C_textBounds, as.graphicsAnnot(x$label), x$x, x$y, :
неизвестна ширина символа 0xe4 в кодировке CP1251
\end{verbatim}

\begin{verbatim}
Warning in grid.Call(C_textBounds, as.graphicsAnnot(x$label), x$x, x$y, :
неизвестна ширина символа 0xe5 в кодировке CP1251
\end{verbatim}

\begin{verbatim}
Warning in grid.Call(C_textBounds, as.graphicsAnnot(x$label), x$x, x$y, :
неизвестна ширина символа 0xeb в кодировке CP1251
\end{verbatim}

\begin{verbatim}
Warning in grid.Call(C_textBounds, as.graphicsAnnot(x$label), x$x, x$y, :
неизвестна ширина символа 0xfc в кодировке CP1251
\end{verbatim}

\begin{verbatim}
Warning in grid.Call(C_textBounds, as.graphicsAnnot(x$label), x$x, x$y, :
неизвестна ширина символа 0xcc в кодировке CP1251
\end{verbatim}

\begin{verbatim}
Warning in grid.Call(C_textBounds, as.graphicsAnnot(x$label), x$x, x$y, :
неизвестна ширина символа 0xee в кодировке CP1251
\end{verbatim}

\begin{verbatim}
Warning in grid.Call(C_textBounds, as.graphicsAnnot(x$label), x$x, x$y, :
неизвестна ширина символа 0xe4 в кодировке CP1251
\end{verbatim}

\begin{verbatim}
Warning in grid.Call(C_textBounds, as.graphicsAnnot(x$label), x$x, x$y, :
неизвестна ширина символа 0xe5 в кодировке CP1251
\end{verbatim}

\begin{verbatim}
Warning in grid.Call(C_textBounds, as.graphicsAnnot(x$label), x$x, x$y, :
неизвестна ширина символа 0xeb в кодировке CP1251
\end{verbatim}

\begin{verbatim}
Warning in grid.Call(C_textBounds, as.graphicsAnnot(x$label), x$x, x$y, :
неизвестна ширина символа 0xfc в кодировке CP1251
\end{verbatim}

\begin{verbatim}
Warning in grid.Call(C_textBounds, as.graphicsAnnot(x$label), x$x, x$y, :
неизвестна ширина символа 0xcf в кодировке CP1251
\end{verbatim}

\begin{verbatim}
Warning in grid.Call(C_textBounds, as.graphicsAnnot(x$label), x$x, x$y, :
неизвестна ширина символа 0xee в кодировке CP1251
\end{verbatim}

\begin{verbatim}
Warning in grid.Call(C_textBounds, as.graphicsAnnot(x$label), x$x, x$y, :
неизвестна ширина символа 0xef в кодировке CP1251
\end{verbatim}

\begin{verbatim}
Warning in grid.Call(C_textBounds, as.graphicsAnnot(x$label), x$x, x$y, :
неизвестна ширина символа 0xee в кодировке CP1251
\end{verbatim}

\begin{verbatim}
Warning in grid.Call(C_textBounds, as.graphicsAnnot(x$label), x$x, x$y, :
неизвестна ширина символа 0xeb в кодировке CP1251
\end{verbatim}

\begin{verbatim}
Warning in grid.Call(C_textBounds, as.graphicsAnnot(x$label), x$x, x$y, :
неизвестна ширина символа 0xed в кодировке CP1251
\end{verbatim}

\begin{verbatim}
Warning in grid.Call(C_textBounds, as.graphicsAnnot(x$label), x$x, x$y, :
неизвестна ширина символа 0xe5 в кодировке CP1251
\end{verbatim}

\begin{verbatim}
Warning in grid.Call(C_textBounds, as.graphicsAnnot(x$label), x$x, x$y, :
неизвестна ширина символа 0xed в кодировке CP1251
\end{verbatim}

\begin{verbatim}
Warning in grid.Call(C_textBounds, as.graphicsAnnot(x$label), x$x, x$y, :
неизвестна ширина символа 0xe8 в кодировке CP1251
\end{verbatim}

\begin{verbatim}
Warning in grid.Call(C_textBounds, as.graphicsAnnot(x$label), x$x, x$y, :
неизвестна ширина символа 0xe5 в кодировке CP1251
\end{verbatim}

\begin{verbatim}
Warning in grid.Call(C_textBounds, as.graphicsAnnot(x$label), x$x, x$y, :
неизвестна ширина символа 0xec в кодировке CP1251
\end{verbatim}

\begin{verbatim}
Warning in grid.Call(C_textBounds, as.graphicsAnnot(x$label), x$x, x$y, :
неизвестна ширина символа 0xeb в кодировке CP1251
\end{verbatim}

\begin{verbatim}
Warning in grid.Call(C_textBounds, as.graphicsAnnot(x$label), x$x, x$y, :
неизвестна ширина символа 0xed в кодировке CP1251
\end{verbatim}

\begin{verbatim}
Warning in grid.Call(C_textBounds, as.graphicsAnnot(x$label), x$x, x$y, :
неизвестна ширина символа 0xee в кодировке CP1251
\end{verbatim}

\begin{verbatim}
Warning in grid.Call(C_textBounds, as.graphicsAnnot(x$label), x$x, x$y, :
неизвестна ширина символа 0xf1 в кодировке CP1251
\end{verbatim}

\begin{verbatim}
Warning in grid.Call(C_textBounds, as.graphicsAnnot(x$label), x$x, x$y, :
неизвестна ширина символа 0xee в кодировке CP1251
\end{verbatim}

\begin{verbatim}
Warning in grid.Call(C_textBounds, as.graphicsAnnot(x$label), x$x, x$y, :
неизвестна ширина символа 0xe1 в кодировке CP1251
\end{verbatim}

\begin{verbatim}
Warning in grid.Call(C_textBounds, as.graphicsAnnot(x$label), x$x, x$y, :
неизвестна ширина символа 0xe5 в кодировке CP1251
\end{verbatim}

\begin{verbatim}
Warning in grid.Call(C_textBounds, as.graphicsAnnot(x$label), x$x, x$y, :
неизвестна ширина символа 0xe9 в кодировке CP1251
\end{verbatim}

\begin{verbatim}
Warning in grid.Call(C_textBounds, as.graphicsAnnot(x$label), x$x, x$y, :
неизвестна ширина символа 0xd1 в кодировке CP1251
\end{verbatim}

\begin{verbatim}
Warning in grid.Call(C_textBounds, as.graphicsAnnot(x$label), x$x, x$y, :
неизвестна ширина символа 0xf0 в кодировке CP1251
\end{verbatim}

\begin{verbatim}
Warning in grid.Call(C_textBounds, as.graphicsAnnot(x$label), x$x, x$y, :
неизвестна ширина символа 0xe0 в кодировке CP1251
\end{verbatim}

\begin{verbatim}
Warning in grid.Call(C_textBounds, as.graphicsAnnot(x$label), x$x, x$y, :
неизвестна ширина символа 0xe2 в кодировке CP1251
\end{verbatim}

\begin{verbatim}
Warning in grid.Call(C_textBounds, as.graphicsAnnot(x$label), x$x, x$y, :
неизвестна ширина символа 0xed в кодировке CP1251
\end{verbatim}

\begin{verbatim}
Warning in grid.Call(C_textBounds, as.graphicsAnnot(x$label), x$x, x$y, :
неизвестна ширина символа 0xe5 в кодировке CP1251
\end{verbatim}

\begin{verbatim}
Warning in grid.Call(C_textBounds, as.graphicsAnnot(x$label), x$x, x$y, :
неизвестна ширина символа 0xed в кодировке CP1251
\end{verbatim}

\begin{verbatim}
Warning in grid.Call(C_textBounds, as.graphicsAnnot(x$label), x$x, x$y, :
неизвестна ширина символа 0xe8 в кодировке CP1251
\end{verbatim}

\begin{verbatim}
Warning in grid.Call(C_textBounds, as.graphicsAnnot(x$label), x$x, x$y, :
неизвестна ширина символа 0xe5 в кодировке CP1251
\end{verbatim}

\begin{verbatim}
Warning in grid.Call(C_textBounds, as.graphicsAnnot(x$label), x$x, x$y, :
неизвестна ширина символа 0xec в кодировке CP1251
\end{verbatim}

\begin{verbatim}
Warning in grid.Call(C_textBounds, as.graphicsAnnot(x$label), x$x, x$y, :
неизвестна ширина символа 0xee в кодировке CP1251
\end{verbatim}

\begin{verbatim}
Warning in grid.Call(C_textBounds, as.graphicsAnnot(x$label), x$x, x$y, :
неизвестна ширина символа 0xe4 в кодировке CP1251
\end{verbatim}

\begin{verbatim}
Warning in grid.Call(C_textBounds, as.graphicsAnnot(x$label), x$x, x$y, :
неизвестна ширина символа 0xe5 в кодировке CP1251
\end{verbatim}

\begin{verbatim}
Warning in grid.Call(C_textBounds, as.graphicsAnnot(x$label), x$x, x$y, :
неизвестна ширина символа 0xeb в кодировке CP1251
\end{verbatim}

\begin{verbatim}
Warning in grid.Call(C_textBounds, as.graphicsAnnot(x$label), x$x, x$y, :
неизвестна ширина символа 0xe5 в кодировке CP1251
\end{verbatim}

\begin{verbatim}
Warning in grid.Call(C_textBounds, as.graphicsAnnot(x$label), x$x, x$y, :
неизвестна ширина символа 0xe9 в кодировке CP1251
\end{verbatim}

\begin{verbatim}
Warning in grid.Call(C_textBounds, as.graphicsAnnot(x$label), x$x, x$y, :
неизвестна ширина символа 0xe7 в кодировке CP1251
\end{verbatim}

\begin{verbatim}
Warning in grid.Call(C_textBounds, as.graphicsAnnot(x$label), x$x, x$y, :
неизвестна ширина символа 0xe0 в кодировке CP1251
\end{verbatim}

\begin{verbatim}
Warning in grid.Call(C_textBounds, as.graphicsAnnot(x$label), x$x, x$y, :
неизвестна ширина символа 0xe2 в кодировке CP1251
\end{verbatim}

\begin{verbatim}
Warning in grid.Call(C_textBounds, as.graphicsAnnot(x$label), x$x, x$y, :
неизвестна ширина символа 0xe8 в кодировке CP1251
\end{verbatim}

\begin{verbatim}
Warning in grid.Call(C_textBounds, as.graphicsAnnot(x$label), x$x, x$y, :
неизвестна ширина символа 0xf1 в кодировке CP1251
\end{verbatim}

\begin{verbatim}
Warning in grid.Call(C_textBounds, as.graphicsAnnot(x$label), x$x, x$y, :
неизвестна ширина символа 0xe8 в кодировке CP1251
\end{verbatim}

\begin{verbatim}
Warning in grid.Call(C_textBounds, as.graphicsAnnot(x$label), x$x, x$y, :
неизвестна ширина символа 0xec в кодировке CP1251
\end{verbatim}

\begin{verbatim}
Warning in grid.Call(C_textBounds, as.graphicsAnnot(x$label), x$x, x$y, :
неизвестна ширина символа 0xee в кодировке CP1251
\end{verbatim}

\begin{verbatim}
Warning in grid.Call(C_textBounds, as.graphicsAnnot(x$label), x$x, x$y, :
неизвестна ширина символа 0xf1 в кодировке CP1251
\end{verbatim}

\begin{verbatim}
Warning in grid.Call(C_textBounds, as.graphicsAnnot(x$label), x$x, x$y, :
неизвестна ширина символа 0xf2 в кодировке CP1251
\end{verbatim}

\begin{verbatim}
Warning in grid.Call(C_textBounds, as.graphicsAnnot(x$label), x$x, x$y, :
неизвестна ширина символа 0xe8 в кодировке CP1251
\end{verbatim}

\begin{verbatim}
Warning in grid.Call(C_textBounds, as.graphicsAnnot(x$label), x$x, x$y, :
неизвестна ширина символа 0xef в кодировке CP1251
\end{verbatim}

\begin{verbatim}
Warning in grid.Call(C_textBounds, as.graphicsAnnot(x$label), x$x, x$y, :
неизвестна ширина символа 0xee в кодировке CP1251
\end{verbatim}

\begin{verbatim}
Warning in grid.Call(C_textBounds, as.graphicsAnnot(x$label), x$x, x$y, :
неизвестна ширина символа 0xef в кодировке CP1251
\end{verbatim}

\begin{verbatim}
Warning in grid.Call(C_textBounds, as.graphicsAnnot(x$label), x$x, x$y, :
неизвестна ширина символа 0xee в кодировке CP1251
\end{verbatim}

\begin{verbatim}
Warning in grid.Call(C_textBounds, as.graphicsAnnot(x$label), x$x, x$y, :
неизвестна ширина символа 0xeb в кодировке CP1251
\end{verbatim}

\begin{verbatim}
Warning in grid.Call(C_textBounds, as.graphicsAnnot(x$label), x$x, x$y, :
неизвестна ширина символа 0xed в кодировке CP1251
\end{verbatim}

\begin{verbatim}
Warning in grid.Call(C_textBounds, as.graphicsAnnot(x$label), x$x, x$y, :
неизвестна ширина символа 0xe5 в кодировке CP1251
\end{verbatim}

\begin{verbatim}
Warning in grid.Call(C_textBounds, as.graphicsAnnot(x$label), x$x, x$y, :
неизвестна ширина символа 0xed в кодировке CP1251
\end{verbatim}

\begin{verbatim}
Warning in grid.Call(C_textBounds, as.graphicsAnnot(x$label), x$x, x$y, :
неизвестна ширина символа 0xe8 в кодировке CP1251
\end{verbatim}

\begin{verbatim}
Warning in grid.Call(C_textBounds, as.graphicsAnnot(x$label), x$x, x$y, :
неизвестна ширина символа 0xff в кодировке CP1251
\end{verbatim}

\begin{verbatim}
Warning in grid.Call(C_textBounds, as.graphicsAnnot(x$label), x$x, x$y, :
неизвестна ширина символа 0xee в кодировке CP1251
\end{verbatim}

\begin{verbatim}
Warning in grid.Call(C_textBounds, as.graphicsAnnot(x$label), x$x, x$y, :
неизвестна ширина символа 0xf2 в кодировке CP1251
\end{verbatim}

\begin{verbatim}
Warning in grid.Call(C_textBounds, as.graphicsAnnot(x$label), x$x, x$y, :
неизвестна ширина символа 0xed в кодировке CP1251
\end{verbatim}

\begin{verbatim}
Warning in grid.Call(C_textBounds, as.graphicsAnnot(x$label), x$x, x$y, :
неизвестна ширина символа 0xe5 в кодировке CP1251
\end{verbatim}

\begin{verbatim}
Warning in grid.Call(C_textBounds, as.graphicsAnnot(x$label), x$x, x$y, :
неизвестна ширина символа 0xf0 в кодировке CP1251
\end{verbatim}

\begin{verbatim}
Warning in grid.Call(C_textBounds, as.graphicsAnnot(x$label), x$x, x$y, :
неизвестна ширина символа 0xe5 в кодировке CP1251
\end{verbatim}

\begin{verbatim}
Warning in grid.Call(C_textBounds, as.graphicsAnnot(x$label), x$x, x$y, :
неизвестна ширина символа 0xf1 в кодировке CP1251
\end{verbatim}

\begin{verbatim}
Warning in grid.Call(C_textBounds, as.graphicsAnnot(x$label), x$x, x$y, :
неизвестна ширина символа 0xf2 в кодировке CP1251
\end{verbatim}

\begin{verbatim}
Warning in grid.Call(C_textBounds, as.graphicsAnnot(x$label), x$x, x$y, :
неизвестна ширина символа 0xee в кодировке CP1251
\end{verbatim}

\begin{verbatim}
Warning in grid.Call(C_textBounds, as.graphicsAnnot(x$label), x$x, x$y, :
неизвестна ширина символа 0xe2 в кодировке CP1251
\end{verbatim}

\begin{verbatim}
Warning in grid.Call(C_textBounds, as.graphicsAnnot(x$label), x$x, x$y, :
неизвестна ширина символа 0xee в кодировке CP1251
\end{verbatim}

\begin{verbatim}
Warning in grid.Call(C_textBounds, as.graphicsAnnot(x$label), x$x, x$y, :
неизвестна ширина символа 0xe3 в кодировке CP1251
\end{verbatim}

\begin{verbatim}
Warning in grid.Call(C_textBounds, as.graphicsAnnot(x$label), x$x, x$y, :
неизвестна ширина символа 0xee в кодировке CP1251
\end{verbatim}

\begin{verbatim}
Warning in grid.Call(C_textBounds, as.graphicsAnnot(x$label), x$x, x$y, :
неизвестна ширина символа 0xe7 в кодировке CP1251
\end{verbatim}

\begin{verbatim}
Warning in grid.Call(C_textBounds, as.graphicsAnnot(x$label), x$x, x$y, :
неизвестна ширина символа 0xe0 в кодировке CP1251
\end{verbatim}

\begin{verbatim}
Warning in grid.Call(C_textBounds, as.graphicsAnnot(x$label), x$x, x$y, :
неизвестна ширина символа 0xef в кодировке CP1251
\end{verbatim}

\begin{verbatim}
Warning in grid.Call(C_textBounds, as.graphicsAnnot(x$label), x$x, x$y, :
неизвестна ширина символа 0xe0 в кодировке CP1251
\end{verbatim}

\begin{verbatim}
Warning in grid.Call(C_textBounds, as.graphicsAnnot(x$label), x$x, x$y, :
неизвестна ширина символа 0xf1 в кодировке CP1251
\end{verbatim}

\begin{verbatim}
Warning in grid.Call(C_textBounds, as.graphicsAnnot(x$label), x$x, x$y, :
неизвестна ширина символа 0xe0 в кодировке CP1251
\end{verbatim}

\begin{verbatim}
Warning in grid.Call(C_textBounds, as.graphicsAnnot(x$label), x$x, x$y, :
неизвестна ширина символа 0xd4 в кодировке CP1251
\end{verbatim}

\begin{verbatim}
Warning in grid.Call(C_textBounds, as.graphicsAnnot(x$label), x$x, x$y, :
неизвестна ширина символа 0xe8 в кодировке CP1251
\end{verbatim}

\begin{verbatim}
Warning in grid.Call(C_textBounds, as.graphicsAnnot(x$label), x$x, x$y, :
неизвестна ширина символа 0xea в кодировке CP1251
\end{verbatim}

\begin{verbatim}
Warning in grid.Call(C_textBounds, as.graphicsAnnot(x$label), x$x, x$y, :
неизвестна ширина символа 0xf1 в кодировке CP1251
\end{verbatim}

\begin{verbatim}
Warning in grid.Call(C_textBounds, as.graphicsAnnot(x$label), x$x, x$y, :
неизвестна ширина символа 0xe0 в кодировке CP1251
\end{verbatim}

\begin{verbatim}
Warning in grid.Call(C_textBounds, as.graphicsAnnot(x$label), x$x, x$y, :
неизвестна ширина символа 0xf6 в кодировке CP1251
\end{verbatim}

\begin{verbatim}
Warning in grid.Call(C_textBounds, as.graphicsAnnot(x$label), x$x, x$y, :
неизвестна ширина символа 0xe8 в кодировке CP1251
\end{verbatim}

\begin{verbatim}
Warning in grid.Call(C_textBounds, as.graphicsAnnot(x$label), x$x, x$y, :
неизвестна ширина символа 0xff в кодировке CP1251
\end{verbatim}

\begin{verbatim}
Warning in grid.Call(C_textBounds, as.graphicsAnnot(x$label), x$x, x$y, :
неизвестна ширина символа 0xe4 в кодировке CP1251
\end{verbatim}

\begin{verbatim}
Warning in grid.Call(C_textBounds, as.graphicsAnnot(x$label), x$x, x$y, :
неизвестна ширина символа 0xf0 в кодировке CP1251
\end{verbatim}

\begin{verbatim}
Warning in grid.Call(C_textBounds, as.graphicsAnnot(x$label), x$x, x$y, :
неизвестна ширина символа 0xf3 в кодировке CP1251
\end{verbatim}

\begin{verbatim}
Warning in grid.Call(C_textBounds, as.graphicsAnnot(x$label), x$x, x$y, :
неизвестна ширина символа 0xe3 в кодировке CP1251
\end{verbatim}

\begin{verbatim}
Warning in grid.Call(C_textBounds, as.graphicsAnnot(x$label), x$x, x$y, :
неизвестна ширина символа 0xe8 в кодировке CP1251
\end{verbatim}

\begin{verbatim}
Warning in grid.Call(C_textBounds, as.graphicsAnnot(x$label), x$x, x$y, :
неизвестна ширина символа 0xf5 в кодировке CP1251
\end{verbatim}

\begin{verbatim}
Warning in grid.Call(C_textBounds, as.graphicsAnnot(x$label), x$x, x$y, :
неизвестна ширина символа 0xef в кодировке CP1251
\end{verbatim}

\begin{verbatim}
Warning in grid.Call(C_textBounds, as.graphicsAnnot(x$label), x$x, x$y, :
неизвестна ширина символа 0xf0 в кодировке CP1251
\end{verbatim}

\begin{verbatim}
Warning in grid.Call(C_textBounds, as.graphicsAnnot(x$label), x$x, x$y, :
неизвестна ширина символа 0xe5 в кодировке CP1251
\end{verbatim}

\begin{verbatim}
Warning in grid.Call(C_textBounds, as.graphicsAnnot(x$label), x$x, x$y, :
неизвестна ширина символа 0xe4 в кодировке CP1251
\end{verbatim}

\begin{verbatim}
Warning in grid.Call(C_textBounds, as.graphicsAnnot(x$label), x$x, x$y, :
неизвестна ширина символа 0xe8 в кодировке CP1251
\end{verbatim}

\begin{verbatim}
Warning in grid.Call(C_textBounds, as.graphicsAnnot(x$label), x$x, x$y, :
неизвестна ширина символа 0xea в кодировке CP1251
\end{verbatim}

\begin{verbatim}
Warning in grid.Call(C_textBounds, as.graphicsAnnot(x$label), x$x, x$y, :
неизвестна ширина символа 0xf2 в кодировке CP1251
\end{verbatim}

\begin{verbatim}
Warning in grid.Call(C_textBounds, as.graphicsAnnot(x$label), x$x, x$y, :
неизвестна ширина символа 0xee в кодировке CP1251
\end{verbatim}

\begin{verbatim}
Warning in grid.Call(C_textBounds, as.graphicsAnnot(x$label), x$x, x$y, :
неизвестна ширина символа 0xf0 в кодировке CP1251
\end{verbatim}

\begin{verbatim}
Warning in grid.Call(C_textBounds, as.graphicsAnnot(x$label), x$x, x$y, :
неизвестна ширина символа 0xee в кодировке CP1251
\end{verbatim}

\begin{verbatim}
Warning in grid.Call(C_textBounds, as.graphicsAnnot(x$label), x$x, x$y, :
неизвестна ширина символа 0xe2 в кодировке CP1251
\end{verbatim}

\begin{verbatim}
Warning in grid.Call(C_textBounds, as.graphicsAnnot(x$label), x$x, x$y, :
неизвестна ширина символа 0xed в кодировке CP1251
\end{verbatim}

\begin{verbatim}
Warning in grid.Call(C_textBounds, as.graphicsAnnot(x$label), x$x, x$y, :
неизвестна ширина символа 0xe0 в кодировке CP1251
\end{verbatim}

\begin{verbatim}
Warning in grid.Call(C_textBounds, as.graphicsAnnot(x$label), x$x, x$y, :
неизвестна ширина символа 0xf1 в кодировке CP1251
\end{verbatim}

\begin{verbatim}
Warning in grid.Call(C_textBounds, as.graphicsAnnot(x$label), x$x, x$y, :
неизвестна ширина символа 0xf0 в кодировке CP1251
\end{verbatim}

\begin{verbatim}
Warning in grid.Call(C_textBounds, as.graphicsAnnot(x$label), x$x, x$y, :
неизвестна ширина символа 0xe5 в кодировке CP1251
\end{verbatim}

\begin{verbatim}
Warning in grid.Call(C_textBounds, as.graphicsAnnot(x$label), x$x, x$y, :
неизвестна ширина символа 0xe4 в кодировке CP1251
\end{verbatim}

\begin{verbatim}
Warning in grid.Call(C_textBounds, as.graphicsAnnot(x$label), x$x, x$y, :
неизвестна ширина символа 0xed в кодировке CP1251
\end{verbatim}

\begin{verbatim}
Warning in grid.Call(C_textBounds, as.graphicsAnnot(x$label), x$x, x$y, :
неизвестна ширина символа 0xe8 в кодировке CP1251
\end{verbatim}

\begin{verbatim}
Warning in grid.Call(C_textBounds, as.graphicsAnnot(x$label), x$x, x$y, :
неизвестна ширина символа 0xf5 в кодировке CP1251
\end{verbatim}

\begin{verbatim}
Warning in grid.Call(C_textBounds, as.graphicsAnnot(x$label), x$x, x$y, :
неизвестна ширина символа 0xe7 в кодировке CP1251
\end{verbatim}

\begin{verbatim}
Warning in grid.Call(C_textBounds, as.graphicsAnnot(x$label), x$x, x$y, :
неизвестна ширина символа 0xed в кодировке CP1251
\end{verbatim}

\begin{verbatim}
Warning in grid.Call(C_textBounds, as.graphicsAnnot(x$label), x$x, x$y, :
неизвестна ширина символа 0xe0 в кодировке CP1251
\end{verbatim}

\begin{verbatim}
Warning in grid.Call(C_textBounds, as.graphicsAnnot(x$label), x$x, x$y, :
неизвестна ширина символа 0xf7 в кодировке CP1251
\end{verbatim}

\begin{verbatim}
Warning in grid.Call(C_textBounds, as.graphicsAnnot(x$label), x$x, x$y, :
неизвестна ширина символа 0xe5 в кодировке CP1251
\end{verbatim}

\begin{verbatim}
Warning in grid.Call(C_textBounds, as.graphicsAnnot(x$label), x$x, x$y, :
неизвестна ширина символа 0xed в кодировке CP1251
\end{verbatim}

\begin{verbatim}
Warning in grid.Call(C_textBounds, as.graphicsAnnot(x$label), x$x, x$y, :
неизвестна ширина символа 0xe8 в кодировке CP1251
\end{verbatim}

\begin{verbatim}
Warning in grid.Call(C_textBounds, as.graphicsAnnot(x$label), x$x, x$y, :
неизвестна ширина символа 0xff в кодировке CP1251
\end{verbatim}

\begin{verbatim}
Warning in grid.Call(C_textBounds, as.graphicsAnnot(x$label), x$x, x$y, :
неизвестна ширина символа 0xf5 в кодировке CP1251
\end{verbatim}

\begin{verbatim}
Warning in grid.Call(C_textBounds, as.graphicsAnnot(x$label), x$x, x$y, :
неизвестна ширина символа 0xcd в кодировке CP1251
\end{verbatim}

\begin{verbatim}
Warning in grid.Call(C_textBounds, as.graphicsAnnot(x$label), x$x, x$y, :
неизвестна ширина символа 0xe5 в кодировке CP1251
\end{verbatim}

\begin{verbatim}
Warning in grid.Call(C_textBounds, as.graphicsAnnot(x$label), x$x, x$y, :
неизвестна ширина символа 0xf0 в кодировке CP1251
\end{verbatim}

\begin{verbatim}
Warning in grid.Call(C_textBounds, as.graphicsAnnot(x$label), x$x, x$y, :
неизвестна ширина символа 0xe5 в кодировке CP1251
\end{verbatim}

\begin{verbatim}
Warning in grid.Call(C_textBounds, as.graphicsAnnot(x$label), x$x, x$y, :
неизвестна ширина символа 0xf1 в кодировке CP1251
\end{verbatim}

\begin{verbatim}
Warning in grid.Call(C_textBounds, as.graphicsAnnot(x$label), x$x, x$y, :
неизвестна ширина символа 0xf2 в кодировке CP1251
\end{verbatim}

\begin{verbatim}
Warning in grid.Call(C_textBounds, as.graphicsAnnot(x$label), x$x, x$y, :
неизвестна ширина символа 0xee в кодировке CP1251
\end{verbatim}

\begin{verbatim}
Warning in grid.Call(C_textBounds, as.graphicsAnnot(x$label), x$x, x$y, :
неизвестна ширина символа 0xe2 в кодировке CP1251
\end{verbatim}

\begin{verbatim}
Warning in grid.Call(C_textBounds, as.graphicsAnnot(x$label), x$x, x$y, :
неизвестна ширина символа 0xfb в кодировке CP1251
\end{verbatim}

\begin{verbatim}
Warning in grid.Call(C_textBounds, as.graphicsAnnot(x$label), x$x, x$y, :
неизвестна ширина символа 0xe9 в кодировке CP1251
\end{verbatim}

\begin{verbatim}
Warning in grid.Call(C_textBounds, as.graphicsAnnot(x$label), x$x, x$y, :
неизвестна ширина символа 0xe7 в кодировке CP1251
\end{verbatim}

\begin{verbatim}
Warning in grid.Call(C_textBounds, as.graphicsAnnot(x$label), x$x, x$y, :
неизвестна ширина символа 0xe0 в кодировке CP1251
\end{verbatim}

\begin{verbatim}
Warning in grid.Call(C_textBounds, as.graphicsAnnot(x$label), x$x, x$y, :
неизвестна ширина символа 0xef в кодировке CP1251
\end{verbatim}

\begin{verbatim}
Warning in grid.Call(C_textBounds, as.graphicsAnnot(x$label), x$x, x$y, :
неизвестна ширина символа 0xe0 в кодировке CP1251
\end{verbatim}

\begin{verbatim}
Warning in grid.Call(C_textBounds, as.graphicsAnnot(x$label), x$x, x$y, :
неизвестна ширина символа 0xf1 в кодировке CP1251
\end{verbatim}

\begin{verbatim}
Warning in grid.Call(C_textBounds, as.graphicsAnnot(x$label), x$x, x$y, :
неизвестна ширина символа 0xf2 в кодировке CP1251
\end{verbatim}

\begin{verbatim}
Warning in grid.Call(C_textBounds, as.graphicsAnnot(x$label), x$x, x$y, :
неизвестна ширина символа 0xfb в кодировке CP1251
\end{verbatim}

\begin{verbatim}
Warning in grid.Call(C_textBounds, as.graphicsAnnot(x$label), x$x, x$y, :
неизвестна ширина символа 0xf1 в кодировке CP1251
\end{verbatim}

\begin{verbatim}
Warning in grid.Call(C_textBounds, as.graphicsAnnot(x$label), x$x, x$y, :
неизвестна ширина символа 0xf2 в кодировке CP1251
\end{verbatim}

\begin{verbatim}
Warning in grid.Call(C_textBounds, as.graphicsAnnot(x$label), x$x, x$y, :
неизвестна ширина символа 0xee в кодировке CP1251
\end{verbatim}

\begin{verbatim}
Warning in grid.Call(C_textBounds, as.graphicsAnnot(x$label), x$x, x$y, :
неизвестна ширина символа 0xed в кодировке CP1251
Warning in grid.Call(C_textBounds, as.graphicsAnnot(x$label), x$x, x$y, :
неизвестна ширина символа 0xed в кодировке CP1251
\end{verbatim}

\begin{verbatim}
Warning in grid.Call.graphics(C_text, as.graphicsAnnot(x$label), x$x, x$y, :
неизвестна ширина символа 0xcd в кодировке CP1251
\end{verbatim}

\begin{verbatim}
Warning in grid.Call.graphics(C_text, as.graphicsAnnot(x$label), x$x, x$y, :
неизвестна ширина символа 0xe5 в кодировке CP1251
\end{verbatim}

\begin{verbatim}
Warning in grid.Call.graphics(C_text, as.graphicsAnnot(x$label), x$x, x$y, :
неизвестна ширина символа 0xf0 в кодировке CP1251
\end{verbatim}

\begin{verbatim}
Warning in grid.Call.graphics(C_text, as.graphicsAnnot(x$label), x$x, x$y, :
неизвестна ширина символа 0xe5 в кодировке CP1251
\end{verbatim}

\begin{verbatim}
Warning in grid.Call.graphics(C_text, as.graphicsAnnot(x$label), x$x, x$y, :
неизвестна ширина символа 0xf1 в кодировке CP1251
\end{verbatim}

\begin{verbatim}
Warning in grid.Call.graphics(C_text, as.graphicsAnnot(x$label), x$x, x$y, :
неизвестна ширина символа 0xf2 в кодировке CP1251
\end{verbatim}

\begin{verbatim}
Warning in grid.Call.graphics(C_text, as.graphicsAnnot(x$label), x$x, x$y, :
неизвестна ширина символа 0xee в кодировке CP1251
\end{verbatim}

\begin{verbatim}
Warning in grid.Call.graphics(C_text, as.graphicsAnnot(x$label), x$x, x$y, :
неизвестна ширина символа 0xe2 в кодировке CP1251
\end{verbatim}

\begin{verbatim}
Warning in grid.Call.graphics(C_text, as.graphicsAnnot(x$label), x$x, x$y, :
неизвестна ширина символа 0xfb в кодировке CP1251
\end{verbatim}

\begin{verbatim}
Warning in grid.Call.graphics(C_text, as.graphicsAnnot(x$label), x$x, x$y, :
неизвестна ширина символа 0xe9 в кодировке CP1251
\end{verbatim}

\begin{verbatim}
Warning in grid.Call.graphics(C_text, as.graphicsAnnot(x$label), x$x, x$y, :
неизвестна ширина символа 0xe7 в кодировке CP1251
\end{verbatim}

\begin{verbatim}
Warning in grid.Call.graphics(C_text, as.graphicsAnnot(x$label), x$x, x$y, :
неизвестна ширина символа 0xe0 в кодировке CP1251
\end{verbatim}

\begin{verbatim}
Warning in grid.Call.graphics(C_text, as.graphicsAnnot(x$label), x$x, x$y, :
неизвестна ширина символа 0xef в кодировке CP1251
\end{verbatim}

\begin{verbatim}
Warning in grid.Call.graphics(C_text, as.graphicsAnnot(x$label), x$x, x$y, :
неизвестна ширина символа 0xe0 в кодировке CP1251
\end{verbatim}

\begin{verbatim}
Warning in grid.Call.graphics(C_text, as.graphicsAnnot(x$label), x$x, x$y, :
неизвестна ширина символа 0xf1 в кодировке CP1251
\end{verbatim}

\begin{verbatim}
Warning in grid.Call.graphics(C_text, as.graphicsAnnot(x$label), x$x, x$y, :
неизвестна ширина символа 0xf2 в кодировке CP1251
\end{verbatim}

\begin{verbatim}
Warning in grid.Call.graphics(C_text, as.graphicsAnnot(x$label), x$x, x$y, :
неизвестна ширина символа 0xfb в кодировке CP1251
\end{verbatim}

\begin{verbatim}
Warning in grid.Call.graphics(C_text, as.graphicsAnnot(x$label), x$x, x$y, :
неизвестна ширина символа 0xf1 в кодировке CP1251
\end{verbatim}

\begin{verbatim}
Warning in grid.Call.graphics(C_text, as.graphicsAnnot(x$label), x$x, x$y, :
неизвестна ширина символа 0xf2 в кодировке CP1251
\end{verbatim}

\begin{verbatim}
Warning in grid.Call.graphics(C_text, as.graphicsAnnot(x$label), x$x, x$y, :
неизвестна ширина символа 0xee в кодировке CP1251
\end{verbatim}

\begin{verbatim}
Warning in grid.Call.graphics(C_text, as.graphicsAnnot(x$label), x$x, x$y, :
неизвестна ширина символа 0xed в кодировке CP1251
Warning in grid.Call.graphics(C_text, as.graphicsAnnot(x$label), x$x, x$y, :
неизвестна ширина символа 0xed в кодировке CP1251
\end{verbatim}

\begin{verbatim}
Warning in grid.Call.graphics(C_text, as.graphicsAnnot(x$label), x$x, x$y, :
неизвестна ширина символа 0xcf в кодировке CP1251
\end{verbatim}

\begin{verbatim}
Warning in grid.Call.graphics(C_text, as.graphicsAnnot(x$label), x$x, x$y, :
неизвестна ширина символа 0xee в кодировке CP1251
\end{verbatim}

\begin{verbatim}
Warning in grid.Call.graphics(C_text, as.graphicsAnnot(x$label), x$x, x$y, :
неизвестна ширина символа 0xef в кодировке CP1251
\end{verbatim}

\begin{verbatim}
Warning in grid.Call.graphics(C_text, as.graphicsAnnot(x$label), x$x, x$y, :
неизвестна ширина символа 0xee в кодировке CP1251
\end{verbatim}

\begin{verbatim}
Warning in grid.Call.graphics(C_text, as.graphicsAnnot(x$label), x$x, x$y, :
неизвестна ширина символа 0xeb в кодировке CP1251
\end{verbatim}

\begin{verbatim}
Warning in grid.Call.graphics(C_text, as.graphicsAnnot(x$label), x$x, x$y, :
неизвестна ширина символа 0xed в кодировке CP1251
\end{verbatim}

\begin{verbatim}
Warning in grid.Call.graphics(C_text, as.graphicsAnnot(x$label), x$x, x$y, :
неизвестна ширина символа 0xe5 в кодировке CP1251
\end{verbatim}

\begin{verbatim}
Warning in grid.Call.graphics(C_text, as.graphicsAnnot(x$label), x$x, x$y, :
неизвестна ширина символа 0xed в кодировке CP1251
\end{verbatim}

\begin{verbatim}
Warning in grid.Call.graphics(C_text, as.graphicsAnnot(x$label), x$x, x$y, :
неизвестна ширина символа 0xe8 в кодировке CP1251
\end{verbatim}

\begin{verbatim}
Warning in grid.Call.graphics(C_text, as.graphicsAnnot(x$label), x$x, x$y, :
неизвестна ширина символа 0xe5 в кодировке CP1251
\end{verbatim}

\begin{verbatim}
Warning in grid.Call.graphics(C_text, as.graphicsAnnot(x$label), x$x, x$y, :
неизвестна ширина символа 0xec в кодировке CP1251
\end{verbatim}

\begin{verbatim}
Warning in grid.Call.graphics(C_text, as.graphicsAnnot(x$label), x$x, x$y, :
неизвестна ширина символа 0xeb в кодировке CP1251
\end{verbatim}

\begin{verbatim}
Warning in grid.Call.graphics(C_text, as.graphicsAnnot(x$label), x$x, x$y, :
неизвестна ширина символа 0xed в кодировке CP1251
\end{verbatim}

\begin{verbatim}
Warning in grid.Call.graphics(C_text, as.graphicsAnnot(x$label), x$x, x$y, :
неизвестна ширина символа 0xee в кодировке CP1251
\end{verbatim}

\begin{verbatim}
Warning in grid.Call.graphics(C_text, as.graphicsAnnot(x$label), x$x, x$y, :
неизвестна ширина символа 0xf1 в кодировке CP1251
\end{verbatim}

\begin{verbatim}
Warning in grid.Call.graphics(C_text, as.graphicsAnnot(x$label), x$x, x$y, :
неизвестна ширина символа 0xee в кодировке CP1251
\end{verbatim}

\begin{verbatim}
Warning in grid.Call.graphics(C_text, as.graphicsAnnot(x$label), x$x, x$y, :
неизвестна ширина символа 0xe1 в кодировке CP1251
\end{verbatim}

\begin{verbatim}
Warning in grid.Call.graphics(C_text, as.graphicsAnnot(x$label), x$x, x$y, :
неизвестна ширина символа 0xe5 в кодировке CP1251
\end{verbatim}

\begin{verbatim}
Warning in grid.Call.graphics(C_text, as.graphicsAnnot(x$label), x$x, x$y, :
неизвестна ширина символа 0xe9 в кодировке CP1251
\end{verbatim}

\begin{verbatim}
Warning in grid.Call.graphics(C_text, as.graphicsAnnot(x$label), x$x, x$y, :
неизвестна ширина символа 0xcc в кодировке CP1251
\end{verbatim}

\begin{verbatim}
Warning in grid.Call.graphics(C_text, as.graphicsAnnot(x$label), x$x, x$y, :
неизвестна ширина символа 0xee в кодировке CP1251
\end{verbatim}

\begin{verbatim}
Warning in grid.Call.graphics(C_text, as.graphicsAnnot(x$label), x$x, x$y, :
неизвестна ширина символа 0xe4 в кодировке CP1251
\end{verbatim}

\begin{verbatim}
Warning in grid.Call.graphics(C_text, as.graphicsAnnot(x$label), x$x, x$y, :
неизвестна ширина символа 0xe5 в кодировке CP1251
\end{verbatim}

\begin{verbatim}
Warning in grid.Call.graphics(C_text, as.graphicsAnnot(x$label), x$x, x$y, :
неизвестна ширина символа 0xeb в кодировке CP1251
\end{verbatim}

\begin{verbatim}
Warning in grid.Call.graphics(C_text, as.graphicsAnnot(x$label), x$x, x$y, :
неизвестна ширина символа 0xfc в кодировке CP1251
\end{verbatim}

\begin{verbatim}
Warning in grid.Call.graphics(C_text, as.graphicsAnnot(x$label), x$x, x$y, :
неизвестна ширина символа 0xc1 в кодировке CP1251
\end{verbatim}

\begin{verbatim}
Warning in grid.Call.graphics(C_text, as.graphicsAnnot(x$label), x$x, x$y, :
неизвестна ширина символа 0xe8 в кодировке CP1251
\end{verbatim}

\begin{verbatim}
Warning in grid.Call.graphics(C_text, as.graphicsAnnot(x$label), x$x, x$y, :
неизвестна ширина символа 0xe2 в кодировке CP1251
\end{verbatim}

\begin{verbatim}
Warning in grid.Call.graphics(C_text, as.graphicsAnnot(x$label), x$x, x$y, :
неизвестна ширина символа 0xe5 в кодировке CP1251
\end{verbatim}

\begin{verbatim}
Warning in grid.Call.graphics(C_text, as.graphicsAnnot(x$label), x$x, x$y, :
неизвестна ширина символа 0xf0 в кодировке CP1251
\end{verbatim}

\begin{verbatim}
Warning in grid.Call.graphics(C_text, as.graphicsAnnot(x$label), x$x, x$y, :
неизвестна ширина символа 0xf2 в кодировке CP1251
\end{verbatim}

\begin{verbatim}
Warning in grid.Call.graphics(C_text, as.graphicsAnnot(x$label), x$x, x$y, :
неизвестна ширина символа 0xee в кодировке CP1251
\end{verbatim}

\begin{verbatim}
Warning in grid.Call.graphics(C_text, as.graphicsAnnot(x$label), x$x, x$y, :
неизвестна ширина символа 0xed в кодировке CP1251
\end{verbatim}

\begin{verbatim}
Warning in grid.Call.graphics(C_text, as.graphicsAnnot(x$label), x$x, x$y, :
неизвестна ширина символа 0xd5 в кодировке CP1251
\end{verbatim}

\begin{verbatim}
Warning in grid.Call.graphics(C_text, as.graphicsAnnot(x$label), x$x, x$y, :
неизвестна ширина символа 0xee в кодировке CP1251
\end{verbatim}

\begin{verbatim}
Warning in grid.Call.graphics(C_text, as.graphicsAnnot(x$label), x$x, x$y, :
неизвестна ширина символа 0xeb в кодировке CP1251
\end{verbatim}

\begin{verbatim}
Warning in grid.Call.graphics(C_text, as.graphicsAnnot(x$label), x$x, x$y, :
неизвестна ширина символа 0xf2 в кодировке CP1251
\end{verbatim}

\begin{verbatim}
Warning in grid.Call.graphics(C_text, as.graphicsAnnot(x$label), x$x, x$y, :
неизвестна ширина символа 0xcb в кодировке CP1251
\end{verbatim}

\begin{verbatim}
Warning in grid.Call.graphics(C_text, as.graphicsAnnot(x$label), x$x, x$y, :
неизвестна ширина символа 0xe8 в кодировке CP1251
\end{verbatim}

\begin{verbatim}
Warning in grid.Call.graphics(C_text, as.graphicsAnnot(x$label), x$x, x$y, :
неизвестна ширина символа 0xed в кодировке CP1251
\end{verbatim}

\begin{verbatim}
Warning in grid.Call.graphics(C_text, as.graphicsAnnot(x$label), x$x, x$y, :
неизвестна ширина символа 0xe5 в кодировке CP1251
\end{verbatim}

\begin{verbatim}
Warning in grid.Call.graphics(C_text, as.graphicsAnnot(x$label), x$x, x$y, :
неизвестна ширина символа 0xe9 в кодировке CP1251
\end{verbatim}

\begin{verbatim}
Warning in grid.Call.graphics(C_text, as.graphicsAnnot(x$label), x$x, x$y, :
неизвестна ширина символа 0xed в кодировке CP1251
\end{verbatim}

\begin{verbatim}
Warning in grid.Call.graphics(C_text, as.graphicsAnnot(x$label), x$x, x$y, :
неизвестна ширина символа 0xe0 в кодировке CP1251
\end{verbatim}

\begin{verbatim}
Warning in grid.Call.graphics(C_text, as.graphicsAnnot(x$label), x$x, x$y, :
неизвестна ширина символа 0xff в кодировке CP1251
\end{verbatim}

\begin{verbatim}
Warning in grid.Call.graphics(C_text, as.graphicsAnnot(x$label), x$x, x$y, :
неизвестна ширина символа 0xce в кодировке CP1251
\end{verbatim}

\begin{verbatim}
Warning in grid.Call.graphics(C_text, as.graphicsAnnot(x$label), x$x, x$y, :
неизвестна ширина символа 0xe1 в кодировке CP1251
\end{verbatim}

\begin{verbatim}
Warning in grid.Call.graphics(C_text, as.graphicsAnnot(x$label), x$x, x$y, :
неизвестна ширина символа 0xee в кодировке CP1251
\end{verbatim}

\begin{verbatim}
Warning in grid.Call.graphics(C_text, as.graphicsAnnot(x$label), x$x, x$y, :
неизвестна ширина символа 0xe1 в кодировке CP1251
\end{verbatim}

\begin{verbatim}
Warning in grid.Call.graphics(C_text, as.graphicsAnnot(x$label), x$x, x$y, :
неизвестна ширина символа 0xf9 в кодировке CP1251
\end{verbatim}

\begin{verbatim}
Warning in grid.Call.graphics(C_text, as.graphicsAnnot(x$label), x$x, x$y, :
неизвестна ширина символа 0xe5 в кодировке CP1251
\end{verbatim}

\begin{verbatim}
Warning in grid.Call.graphics(C_text, as.graphicsAnnot(x$label), x$x, x$y, :
неизвестна ширина символа 0xed в кодировке CP1251
Warning in grid.Call.graphics(C_text, as.graphicsAnnot(x$label), x$x, x$y, :
неизвестна ширина символа 0xed в кодировке CP1251
\end{verbatim}

\begin{verbatim}
Warning in grid.Call.graphics(C_text, as.graphicsAnnot(x$label), x$x, x$y, :
неизвестна ширина символа 0xe0 в кодировке CP1251
\end{verbatim}

\begin{verbatim}
Warning in grid.Call.graphics(C_text, as.graphicsAnnot(x$label), x$x, x$y, :
неизвестна ширина символа 0xff в кодировке CP1251
\end{verbatim}

\begin{verbatim}
Warning in grid.Call.graphics(C_text, as.graphicsAnnot(x$label), x$x, x$y, :
неизвестна ширина символа 0xe0 в кодировке CP1251
\end{verbatim}

\begin{verbatim}
Warning in grid.Call.graphics(C_text, as.graphicsAnnot(x$label), x$x, x$y, :
неизвестна ширина символа 0xe4 в кодировке CP1251
Warning in grid.Call.graphics(C_text, as.graphicsAnnot(x$label), x$x, x$y, :
неизвестна ширина символа 0xe4 в кодировке CP1251
\end{verbatim}

\begin{verbatim}
Warning in grid.Call.graphics(C_text, as.graphicsAnnot(x$label), x$x, x$y, :
неизвестна ширина символа 0xe8 в кодировке CP1251
\end{verbatim}

\begin{verbatim}
Warning in grid.Call.graphics(C_text, as.graphicsAnnot(x$label), x$x, x$y, :
неизвестна ширина символа 0xf2 в кодировке CP1251
\end{verbatim}

\begin{verbatim}
Warning in grid.Call.graphics(C_text, as.graphicsAnnot(x$label), x$x, x$y, :
неизвестна ширина символа 0xe8 в кодировке CP1251
\end{verbatim}

\begin{verbatim}
Warning in grid.Call.graphics(C_text, as.graphicsAnnot(x$label), x$x, x$y, :
неизвестна ширина символа 0xe2 в кодировке CP1251
\end{verbatim}

\begin{verbatim}
Warning in grid.Call.graphics(C_text, as.graphicsAnnot(x$label), x$x, x$y, :
неизвестна ширина символа 0xed в кодировке CP1251
\end{verbatim}

\begin{verbatim}
Warning in grid.Call.graphics(C_text, as.graphicsAnnot(x$label), x$x, x$y, :
неизвестна ширина символа 0xe0 в кодировке CP1251
\end{verbatim}

\begin{verbatim}
Warning in grid.Call.graphics(C_text, as.graphicsAnnot(x$label), x$x, x$y, :
неизвестна ширина символа 0xff в кодировке CP1251
\end{verbatim}

\begin{verbatim}
Warning in grid.Call.graphics(C_text, as.graphicsAnnot(x$label), x$x, x$y, :
неизвестна ширина символа 0xce в кодировке CP1251
\end{verbatim}

\begin{verbatim}
Warning in grid.Call.graphics(C_text, as.graphicsAnnot(x$label), x$x, x$y, :
неизвестна ширина символа 0xe1 в кодировке CP1251
\end{verbatim}

\begin{verbatim}
Warning in grid.Call.graphics(C_text, as.graphicsAnnot(x$label), x$x, x$y, :
неизвестна ширина символа 0xee в кодировке CP1251
\end{verbatim}

\begin{verbatim}
Warning in grid.Call.graphics(C_text, as.graphicsAnnot(x$label), x$x, x$y, :
неизвестна ширина символа 0xe1 в кодировке CP1251
\end{verbatim}

\begin{verbatim}
Warning in grid.Call.graphics(C_text, as.graphicsAnnot(x$label), x$x, x$y, :
неизвестна ширина символа 0xf9 в кодировке CP1251
\end{verbatim}

\begin{verbatim}
Warning in grid.Call.graphics(C_text, as.graphicsAnnot(x$label), x$x, x$y, :
неизвестна ширина символа 0xe5 в кодировке CP1251
\end{verbatim}

\begin{verbatim}
Warning in grid.Call.graphics(C_text, as.graphicsAnnot(x$label), x$x, x$y, :
неизвестна ширина символа 0xed в кодировке CP1251
Warning in grid.Call.graphics(C_text, as.graphicsAnnot(x$label), x$x, x$y, :
неизвестна ширина символа 0xed в кодировке CP1251
\end{verbatim}

\begin{verbatim}
Warning in grid.Call.graphics(C_text, as.graphicsAnnot(x$label), x$x, x$y, :
неизвестна ширина символа 0xe0 в кодировке CP1251
\end{verbatim}

\begin{verbatim}
Warning in grid.Call.graphics(C_text, as.graphicsAnnot(x$label), x$x, x$y, :
неизвестна ширина символа 0xff в кодировке CP1251
\end{verbatim}

\begin{verbatim}
Warning in grid.Call.graphics(C_text, as.graphicsAnnot(x$label), x$x, x$y, :
неизвестна ширина символа 0xeb в кодировке CP1251
\end{verbatim}

\begin{verbatim}
Warning in grid.Call.graphics(C_text, as.graphicsAnnot(x$label), x$x, x$y, :
неизвестна ширина символа 0xe8 в кодировке CP1251
\end{verbatim}

\begin{verbatim}
Warning in grid.Call.graphics(C_text, as.graphicsAnnot(x$label), x$x, x$y, :
неизвестна ширина символа 0xed в кодировке CP1251
\end{verbatim}

\begin{verbatim}
Warning in grid.Call.graphics(C_text, as.graphicsAnnot(x$label), x$x, x$y, :
неизвестна ширина символа 0xe5 в кодировке CP1251
\end{verbatim}

\begin{verbatim}
Warning in grid.Call.graphics(C_text, as.graphicsAnnot(x$label), x$x, x$y, :
неизвестна ширина символа 0xe9 в кодировке CP1251
\end{verbatim}

\begin{verbatim}
Warning in grid.Call.graphics(C_text, as.graphicsAnnot(x$label), x$x, x$y, :
неизвестна ширина символа 0xed в кодировке CP1251
\end{verbatim}

\begin{verbatim}
Warning in grid.Call.graphics(C_text, as.graphicsAnnot(x$label), x$x, x$y, :
неизвестна ширина символа 0xe0 в кодировке CP1251
\end{verbatim}

\begin{verbatim}
Warning in grid.Call.graphics(C_text, as.graphicsAnnot(x$label), x$x, x$y, :
неизвестна ширина символа 0xff в кодировке CP1251
\end{verbatim}

\begin{verbatim}
Warning in grid.Call.graphics(C_text, as.graphicsAnnot(x$label), x$x, x$y, :
неизвестна ширина символа 0xd0 в кодировке CP1251
\end{verbatim}

\begin{verbatim}
Warning in grid.Call.graphics(C_text, as.graphicsAnnot(x$label), x$x, x$y, :
неизвестна ширина символа 0xe8 в кодировке CP1251
\end{verbatim}

\begin{verbatim}
Warning in grid.Call.graphics(C_text, as.graphicsAnnot(x$label), x$x, x$y, :
неизвестна ширина символа 0xea в кодировке CP1251
\end{verbatim}

\begin{verbatim}
Warning in grid.Call.graphics(C_text, as.graphicsAnnot(x$label), x$x, x$y, :
неизвестна ширина символа 0xe5 в кодировке CP1251
\end{verbatim}

\begin{verbatim}
Warning in grid.Call.graphics(C_text, as.graphicsAnnot(x$label), x$x, x$y, :
неизвестна ширина символа 0xf0 в кодировке CP1251
\end{verbatim}

\begin{verbatim}
Warning in grid.Call.graphics(C_text, as.graphicsAnnot(x$label), x$x, x$y, :
неизвестна ширина символа 0xd4 в кодировке CP1251
\end{verbatim}

\begin{verbatim}
Warning in grid.Call.graphics(C_text, as.graphicsAnnot(x$label), x$x, x$y, :
неизвестна ширина символа 0xe8 в кодировке CP1251
\end{verbatim}

\begin{verbatim}
Warning in grid.Call.graphics(C_text, as.graphicsAnnot(x$label), x$x, x$y, :
неизвестна ширина символа 0xea в кодировке CP1251
\end{verbatim}

\begin{verbatim}
Warning in grid.Call.graphics(C_text, as.graphicsAnnot(x$label), x$x, x$y, :
неизвестна ширина символа 0xf1 в кодировке CP1251
\end{verbatim}

\begin{verbatim}
Warning in grid.Call.graphics(C_text, as.graphicsAnnot(x$label), x$x, x$y, :
неизвестна ширина символа 0xe0 в кодировке CP1251
\end{verbatim}

\begin{verbatim}
Warning in grid.Call.graphics(C_text, as.graphicsAnnot(x$label), x$x, x$y, :
неизвестна ширина символа 0xf6 в кодировке CP1251
\end{verbatim}

\begin{verbatim}
Warning in grid.Call.graphics(C_text, as.graphicsAnnot(x$label), x$x, x$y, :
неизвестна ширина символа 0xe8 в кодировке CP1251
\end{verbatim}

\begin{verbatim}
Warning in grid.Call.graphics(C_text, as.graphicsAnnot(x$label), x$x, x$y, :
неизвестна ширина символа 0xff в кодировке CP1251
\end{verbatim}

\begin{verbatim}
Warning in grid.Call.graphics(C_text, as.graphicsAnnot(x$label), x$x, x$y, :
неизвестна ширина символа 0xe4 в кодировке CP1251
\end{verbatim}

\begin{verbatim}
Warning in grid.Call.graphics(C_text, as.graphicsAnnot(x$label), x$x, x$y, :
неизвестна ширина символа 0xf0 в кодировке CP1251
\end{verbatim}

\begin{verbatim}
Warning in grid.Call.graphics(C_text, as.graphicsAnnot(x$label), x$x, x$y, :
неизвестна ширина символа 0xf3 в кодировке CP1251
\end{verbatim}

\begin{verbatim}
Warning in grid.Call.graphics(C_text, as.graphicsAnnot(x$label), x$x, x$y, :
неизвестна ширина символа 0xe3 в кодировке CP1251
\end{verbatim}

\begin{verbatim}
Warning in grid.Call.graphics(C_text, as.graphicsAnnot(x$label), x$x, x$y, :
неизвестна ширина символа 0xe8 в кодировке CP1251
\end{verbatim}

\begin{verbatim}
Warning in grid.Call.graphics(C_text, as.graphicsAnnot(x$label), x$x, x$y, :
неизвестна ширина символа 0xf5 в кодировке CP1251
\end{verbatim}

\begin{verbatim}
Warning in grid.Call.graphics(C_text, as.graphicsAnnot(x$label), x$x, x$y, :
неизвестна ширина символа 0xef в кодировке CP1251
\end{verbatim}

\begin{verbatim}
Warning in grid.Call.graphics(C_text, as.graphicsAnnot(x$label), x$x, x$y, :
неизвестна ширина символа 0xf0 в кодировке CP1251
\end{verbatim}

\begin{verbatim}
Warning in grid.Call.graphics(C_text, as.graphicsAnnot(x$label), x$x, x$y, :
неизвестна ширина символа 0xe5 в кодировке CP1251
\end{verbatim}

\begin{verbatim}
Warning in grid.Call.graphics(C_text, as.graphicsAnnot(x$label), x$x, x$y, :
неизвестна ширина символа 0xe4 в кодировке CP1251
\end{verbatim}

\begin{verbatim}
Warning in grid.Call.graphics(C_text, as.graphicsAnnot(x$label), x$x, x$y, :
неизвестна ширина символа 0xe8 в кодировке CP1251
\end{verbatim}

\begin{verbatim}
Warning in grid.Call.graphics(C_text, as.graphicsAnnot(x$label), x$x, x$y, :
неизвестна ширина символа 0xea в кодировке CP1251
\end{verbatim}

\begin{verbatim}
Warning in grid.Call.graphics(C_text, as.graphicsAnnot(x$label), x$x, x$y, :
неизвестна ширина символа 0xf2 в кодировке CP1251
\end{verbatim}

\begin{verbatim}
Warning in grid.Call.graphics(C_text, as.graphicsAnnot(x$label), x$x, x$y, :
неизвестна ширина символа 0xee в кодировке CP1251
\end{verbatim}

\begin{verbatim}
Warning in grid.Call.graphics(C_text, as.graphicsAnnot(x$label), x$x, x$y, :
неизвестна ширина символа 0xf0 в кодировке CP1251
\end{verbatim}

\begin{verbatim}
Warning in grid.Call.graphics(C_text, as.graphicsAnnot(x$label), x$x, x$y, :
неизвестна ширина символа 0xee в кодировке CP1251
\end{verbatim}

\begin{verbatim}
Warning in grid.Call.graphics(C_text, as.graphicsAnnot(x$label), x$x, x$y, :
неизвестна ширина символа 0xe2 в кодировке CP1251
\end{verbatim}

\begin{verbatim}
Warning in grid.Call.graphics(C_text, as.graphicsAnnot(x$label), x$x, x$y, :
неизвестна ширина символа 0xed в кодировке CP1251
\end{verbatim}

\begin{verbatim}
Warning in grid.Call.graphics(C_text, as.graphicsAnnot(x$label), x$x, x$y, :
неизвестна ширина символа 0xe0 в кодировке CP1251
\end{verbatim}

\begin{verbatim}
Warning in grid.Call.graphics(C_text, as.graphicsAnnot(x$label), x$x, x$y, :
неизвестна ширина символа 0xf1 в кодировке CP1251
\end{verbatim}

\begin{verbatim}
Warning in grid.Call.graphics(C_text, as.graphicsAnnot(x$label), x$x, x$y, :
неизвестна ширина символа 0xf0 в кодировке CP1251
\end{verbatim}

\begin{verbatim}
Warning in grid.Call.graphics(C_text, as.graphicsAnnot(x$label), x$x, x$y, :
неизвестна ширина символа 0xe5 в кодировке CP1251
\end{verbatim}

\begin{verbatim}
Warning in grid.Call.graphics(C_text, as.graphicsAnnot(x$label), x$x, x$y, :
неизвестна ширина символа 0xe4 в кодировке CP1251
\end{verbatim}

\begin{verbatim}
Warning in grid.Call.graphics(C_text, as.graphicsAnnot(x$label), x$x, x$y, :
неизвестна ширина символа 0xed в кодировке CP1251
\end{verbatim}

\begin{verbatim}
Warning in grid.Call.graphics(C_text, as.graphicsAnnot(x$label), x$x, x$y, :
неизвестна ширина символа 0xe8 в кодировке CP1251
\end{verbatim}

\begin{verbatim}
Warning in grid.Call.graphics(C_text, as.graphicsAnnot(x$label), x$x, x$y, :
неизвестна ширина символа 0xf5 в кодировке CP1251
\end{verbatim}

\begin{verbatim}
Warning in grid.Call.graphics(C_text, as.graphicsAnnot(x$label), x$x, x$y, :
неизвестна ширина символа 0xe7 в кодировке CP1251
\end{verbatim}

\begin{verbatim}
Warning in grid.Call.graphics(C_text, as.graphicsAnnot(x$label), x$x, x$y, :
неизвестна ширина символа 0xed в кодировке CP1251
\end{verbatim}

\begin{verbatim}
Warning in grid.Call.graphics(C_text, as.graphicsAnnot(x$label), x$x, x$y, :
неизвестна ширина символа 0xe0 в кодировке CP1251
\end{verbatim}

\begin{verbatim}
Warning in grid.Call.graphics(C_text, as.graphicsAnnot(x$label), x$x, x$y, :
неизвестна ширина символа 0xf7 в кодировке CP1251
\end{verbatim}

\begin{verbatim}
Warning in grid.Call.graphics(C_text, as.graphicsAnnot(x$label), x$x, x$y, :
неизвестна ширина символа 0xe5 в кодировке CP1251
\end{verbatim}

\begin{verbatim}
Warning in grid.Call.graphics(C_text, as.graphicsAnnot(x$label), x$x, x$y, :
неизвестна ширина символа 0xed в кодировке CP1251
\end{verbatim}

\begin{verbatim}
Warning in grid.Call.graphics(C_text, as.graphicsAnnot(x$label), x$x, x$y, :
неизвестна ширина символа 0xe8 в кодировке CP1251
\end{verbatim}

\begin{verbatim}
Warning in grid.Call.graphics(C_text, as.graphicsAnnot(x$label), x$x, x$y, :
неизвестна ширина символа 0xff в кодировке CP1251
\end{verbatim}

\begin{verbatim}
Warning in grid.Call.graphics(C_text, as.graphicsAnnot(x$label), x$x, x$y, :
неизвестна ширина символа 0xf5 в кодировке CP1251
\end{verbatim}

\begin{verbatim}
Warning in grid.Call.graphics(C_text, as.graphicsAnnot(x$label), x$x, x$y, :
неизвестна ширина символа 0xd1 в кодировке CP1251
\end{verbatim}

\begin{verbatim}
Warning in grid.Call.graphics(C_text, as.graphicsAnnot(x$label), x$x, x$y, :
неизвестна ширина символа 0xf0 в кодировке CP1251
\end{verbatim}

\begin{verbatim}
Warning in grid.Call.graphics(C_text, as.graphicsAnnot(x$label), x$x, x$y, :
неизвестна ширина символа 0xe0 в кодировке CP1251
\end{verbatim}

\begin{verbatim}
Warning in grid.Call.graphics(C_text, as.graphicsAnnot(x$label), x$x, x$y, :
неизвестна ширина символа 0xe2 в кодировке CP1251
\end{verbatim}

\begin{verbatim}
Warning in grid.Call.graphics(C_text, as.graphicsAnnot(x$label), x$x, x$y, :
неизвестна ширина символа 0xed в кодировке CP1251
\end{verbatim}

\begin{verbatim}
Warning in grid.Call.graphics(C_text, as.graphicsAnnot(x$label), x$x, x$y, :
неизвестна ширина символа 0xe5 в кодировке CP1251
\end{verbatim}

\begin{verbatim}
Warning in grid.Call.graphics(C_text, as.graphicsAnnot(x$label), x$x, x$y, :
неизвестна ширина символа 0xed в кодировке CP1251
\end{verbatim}

\begin{verbatim}
Warning in grid.Call.graphics(C_text, as.graphicsAnnot(x$label), x$x, x$y, :
неизвестна ширина символа 0xe8 в кодировке CP1251
\end{verbatim}

\begin{verbatim}
Warning in grid.Call.graphics(C_text, as.graphicsAnnot(x$label), x$x, x$y, :
неизвестна ширина символа 0xe5 в кодировке CP1251
\end{verbatim}

\begin{verbatim}
Warning in grid.Call.graphics(C_text, as.graphicsAnnot(x$label), x$x, x$y, :
неизвестна ширина символа 0xec в кодировке CP1251
\end{verbatim}

\begin{verbatim}
Warning in grid.Call.graphics(C_text, as.graphicsAnnot(x$label), x$x, x$y, :
неизвестна ширина символа 0xee в кодировке CP1251
\end{verbatim}

\begin{verbatim}
Warning in grid.Call.graphics(C_text, as.graphicsAnnot(x$label), x$x, x$y, :
неизвестна ширина символа 0xe4 в кодировке CP1251
\end{verbatim}

\begin{verbatim}
Warning in grid.Call.graphics(C_text, as.graphicsAnnot(x$label), x$x, x$y, :
неизвестна ширина символа 0xe5 в кодировке CP1251
\end{verbatim}

\begin{verbatim}
Warning in grid.Call.graphics(C_text, as.graphicsAnnot(x$label), x$x, x$y, :
неизвестна ширина символа 0xeb в кодировке CP1251
\end{verbatim}

\begin{verbatim}
Warning in grid.Call.graphics(C_text, as.graphicsAnnot(x$label), x$x, x$y, :
неизвестна ширина символа 0xe5 в кодировке CP1251
\end{verbatim}

\begin{verbatim}
Warning in grid.Call.graphics(C_text, as.graphicsAnnot(x$label), x$x, x$y, :
неизвестна ширина символа 0xe9 в кодировке CP1251
\end{verbatim}

\begin{verbatim}
Warning in grid.Call.graphics(C_text, as.graphicsAnnot(x$label), x$x, x$y, :
неизвестна ширина символа 0xe7 в кодировке CP1251
\end{verbatim}

\begin{verbatim}
Warning in grid.Call.graphics(C_text, as.graphicsAnnot(x$label), x$x, x$y, :
неизвестна ширина символа 0xe0 в кодировке CP1251
\end{verbatim}

\begin{verbatim}
Warning in grid.Call.graphics(C_text, as.graphicsAnnot(x$label), x$x, x$y, :
неизвестна ширина символа 0xe2 в кодировке CP1251
\end{verbatim}

\begin{verbatim}
Warning in grid.Call.graphics(C_text, as.graphicsAnnot(x$label), x$x, x$y, :
неизвестна ширина символа 0xe8 в кодировке CP1251
\end{verbatim}

\begin{verbatim}
Warning in grid.Call.graphics(C_text, as.graphicsAnnot(x$label), x$x, x$y, :
неизвестна ширина символа 0xf1 в кодировке CP1251
\end{verbatim}

\begin{verbatim}
Warning in grid.Call.graphics(C_text, as.graphicsAnnot(x$label), x$x, x$y, :
неизвестна ширина символа 0xe8 в кодировке CP1251
\end{verbatim}

\begin{verbatim}
Warning in grid.Call.graphics(C_text, as.graphicsAnnot(x$label), x$x, x$y, :
неизвестна ширина символа 0xec в кодировке CP1251
\end{verbatim}

\begin{verbatim}
Warning in grid.Call.graphics(C_text, as.graphicsAnnot(x$label), x$x, x$y, :
неизвестна ширина символа 0xee в кодировке CP1251
\end{verbatim}

\begin{verbatim}
Warning in grid.Call.graphics(C_text, as.graphicsAnnot(x$label), x$x, x$y, :
неизвестна ширина символа 0xf1 в кодировке CP1251
\end{verbatim}

\begin{verbatim}
Warning in grid.Call.graphics(C_text, as.graphicsAnnot(x$label), x$x, x$y, :
неизвестна ширина символа 0xf2 в кодировке CP1251
\end{verbatim}

\begin{verbatim}
Warning in grid.Call.graphics(C_text, as.graphicsAnnot(x$label), x$x, x$y, :
неизвестна ширина символа 0xe8 в кодировке CP1251
\end{verbatim}

\begin{verbatim}
Warning in grid.Call.graphics(C_text, as.graphicsAnnot(x$label), x$x, x$y, :
неизвестна ширина символа 0xef в кодировке CP1251
\end{verbatim}

\begin{verbatim}
Warning in grid.Call.graphics(C_text, as.graphicsAnnot(x$label), x$x, x$y, :
неизвестна ширина символа 0xee в кодировке CP1251
\end{verbatim}

\begin{verbatim}
Warning in grid.Call.graphics(C_text, as.graphicsAnnot(x$label), x$x, x$y, :
неизвестна ширина символа 0xef в кодировке CP1251
\end{verbatim}

\begin{verbatim}
Warning in grid.Call.graphics(C_text, as.graphicsAnnot(x$label), x$x, x$y, :
неизвестна ширина символа 0xee в кодировке CP1251
\end{verbatim}

\begin{verbatim}
Warning in grid.Call.graphics(C_text, as.graphicsAnnot(x$label), x$x, x$y, :
неизвестна ширина символа 0xeb в кодировке CP1251
\end{verbatim}

\begin{verbatim}
Warning in grid.Call.graphics(C_text, as.graphicsAnnot(x$label), x$x, x$y, :
неизвестна ширина символа 0xed в кодировке CP1251
\end{verbatim}

\begin{verbatim}
Warning in grid.Call.graphics(C_text, as.graphicsAnnot(x$label), x$x, x$y, :
неизвестна ширина символа 0xe5 в кодировке CP1251
\end{verbatim}

\begin{verbatim}
Warning in grid.Call.graphics(C_text, as.graphicsAnnot(x$label), x$x, x$y, :
неизвестна ширина символа 0xed в кодировке CP1251
\end{verbatim}

\begin{verbatim}
Warning in grid.Call.graphics(C_text, as.graphicsAnnot(x$label), x$x, x$y, :
неизвестна ширина символа 0xe8 в кодировке CP1251
\end{verbatim}

\begin{verbatim}
Warning in grid.Call.graphics(C_text, as.graphicsAnnot(x$label), x$x, x$y, :
неизвестна ширина символа 0xff в кодировке CP1251
\end{verbatim}

\begin{verbatim}
Warning in grid.Call.graphics(C_text, as.graphicsAnnot(x$label), x$x, x$y, :
неизвестна ширина символа 0xee в кодировке CP1251
\end{verbatim}

\begin{verbatim}
Warning in grid.Call.graphics(C_text, as.graphicsAnnot(x$label), x$x, x$y, :
неизвестна ширина символа 0xf2 в кодировке CP1251
\end{verbatim}

\begin{verbatim}
Warning in grid.Call.graphics(C_text, as.graphicsAnnot(x$label), x$x, x$y, :
неизвестна ширина символа 0xed в кодировке CP1251
\end{verbatim}

\begin{verbatim}
Warning in grid.Call.graphics(C_text, as.graphicsAnnot(x$label), x$x, x$y, :
неизвестна ширина символа 0xe5 в кодировке CP1251
\end{verbatim}

\begin{verbatim}
Warning in grid.Call.graphics(C_text, as.graphicsAnnot(x$label), x$x, x$y, :
неизвестна ширина символа 0xf0 в кодировке CP1251
\end{verbatim}

\begin{verbatim}
Warning in grid.Call.graphics(C_text, as.graphicsAnnot(x$label), x$x, x$y, :
неизвестна ширина символа 0xe5 в кодировке CP1251
\end{verbatim}

\begin{verbatim}
Warning in grid.Call.graphics(C_text, as.graphicsAnnot(x$label), x$x, x$y, :
неизвестна ширина символа 0xf1 в кодировке CP1251
\end{verbatim}

\begin{verbatim}
Warning in grid.Call.graphics(C_text, as.graphicsAnnot(x$label), x$x, x$y, :
неизвестна ширина символа 0xf2 в кодировке CP1251
\end{verbatim}

\begin{verbatim}
Warning in grid.Call.graphics(C_text, as.graphicsAnnot(x$label), x$x, x$y, :
неизвестна ширина символа 0xee в кодировке CP1251
\end{verbatim}

\begin{verbatim}
Warning in grid.Call.graphics(C_text, as.graphicsAnnot(x$label), x$x, x$y, :
неизвестна ширина символа 0xe2 в кодировке CP1251
\end{verbatim}

\begin{verbatim}
Warning in grid.Call.graphics(C_text, as.graphicsAnnot(x$label), x$x, x$y, :
неизвестна ширина символа 0xee в кодировке CP1251
\end{verbatim}

\begin{verbatim}
Warning in grid.Call.graphics(C_text, as.graphicsAnnot(x$label), x$x, x$y, :
неизвестна ширина символа 0xe3 в кодировке CP1251
\end{verbatim}

\begin{verbatim}
Warning in grid.Call.graphics(C_text, as.graphicsAnnot(x$label), x$x, x$y, :
неизвестна ширина символа 0xee в кодировке CP1251
\end{verbatim}

\begin{verbatim}
Warning in grid.Call.graphics(C_text, as.graphicsAnnot(x$label), x$x, x$y, :
неизвестна ширина символа 0xe7 в кодировке CP1251
\end{verbatim}

\begin{verbatim}
Warning in grid.Call.graphics(C_text, as.graphicsAnnot(x$label), x$x, x$y, :
неизвестна ширина символа 0xe0 в кодировке CP1251
\end{verbatim}

\begin{verbatim}
Warning in grid.Call.graphics(C_text, as.graphicsAnnot(x$label), x$x, x$y, :
неизвестна ширина символа 0xef в кодировке CP1251
\end{verbatim}

\begin{verbatim}
Warning in grid.Call.graphics(C_text, as.graphicsAnnot(x$label), x$x, x$y, :
неизвестна ширина символа 0xe0 в кодировке CP1251
\end{verbatim}

\begin{verbatim}
Warning in grid.Call.graphics(C_text, as.graphicsAnnot(x$label), x$x, x$y, :
неизвестна ширина символа 0xf1 в кодировке CP1251
\end{verbatim}

\begin{verbatim}
Warning in grid.Call.graphics(C_text, as.graphicsAnnot(x$label), x$x, x$y, :
неизвестна ширина символа 0xe0 в кодировке CP1251
\end{verbatim}

\pandocbounded{\includegraphics[keepaspectratio]{chapter7_files/figure-pdf/unnamed-chunk-3-4.pdf}}

\section{Статистические модели
LM/GLM/GAM}\label{ux441ux442ux430ux442ux438ux441ux442ux438ux447ux435ux441ux43aux438ux435-ux43cux43eux434ux435ux43bux438-lmglmgam}

Статистические модели линейной регрессии (LM), обобщенной линейной
регрессии (GLM) и обобщенной аддитивной регрессии (GAM) представляют
собой мощный и взаимодополняющий набор инструментов для анализа водных
биоресурсов, позволяющий исследователям от простых линейных зависимостей
переходить к сложным нелинейным взаимодействиям, сохраняя при этом
интерпретируемость результатов. Линейная модель (LM) служит фундаментом
для всего статистического анализа в гидробиологии, основываясь на
предположении, что зависимость между предикторами и откликом является
линейной, а остатки распределены нормально с постоянной дисперсией. Эта
модель предоставляет простую интерпретацию коэффициентов как величины
изменения отклика при единичном изменении предиктора, что особенно ценно
при работе с такими биологическими показателями, как пополнение запаса
или нерестовая биомасса. Однако при анализе водных биоресурсов мы часто
сталкиваемся с данными, которые нарушают ключевые предположения LM:
пополнение рыбы или беспозвоночного не может быть отрицательным, его
распределение обычно сильно скошено вправо, а дисперсия часто
увеличивается с ростом среднего значения. Именно здесь на помощь
приходит обобщенная линейная модель (GLM), расширяющая возможности LM за
счет введения двух ключевых компонентов --- экспоненциального семейства
распределений и связующей функции (link-function). Для данных о рыбных
запасах особенно полезно Gamma-распределение с логарифмической связкой,
которое учитывает положительность отклика и мультипликативную природу
ошибок, характерную для биологических данных. В отличие от LM, где мы
интерпретируем коэффициенты как абсолютные изменения, в GLM с
лог-связкой коэффициенты отражают относительные изменения: увеличение
предиктора на единицу приводит к умножению ожидаемого отклика на
exp(коэффициент), что соответствует биологической реальности, где
эффекты часто действуют мультипликативно, а не аддитивно. Но даже GLM
сохраняет ограничение на линейность в преобразованном пространстве, что
может быть недостаточным для описания сложных экологических
зависимостей, таких как оптимальный диапазон температуры для нереста или
пороговые эффекты средовых факторов. Здесь в игру вступают обобщенные
аддитивные модели (GAM), которые заменяют линейные комбинации
предикторов на гладкие функции, оцениваемые с помощью сплайнов, что
позволяет моделировать практически любые нелинейные зависимости без
предварительного задания их формы. GAM сохраняет интерпретируемость
линейных моделей, так как каждая гладкая функция может быть
визуализирована и проанализирована отдельно, показывая, как именно
каждый фактор влияет на пополнение запаса, будь то монотонный рост,
оптимум с максимумом или сложная колебательная зависимость. При работе с
GAM особое внимание уделяется выбору степени гладкости, так как
чрезмерно гибкие функции могут переобучиться на шум в данных, тогда как
недостаточно гибкие не уловят реальные биологические закономерности; в
пакете mgcv это решается автоматически через метод максимального
правдоподобия с штрафом (REML), который балансирует качество подгонки и
гладкость функций. Сравнивая эти модели с классическими моделями
запас-пополнение, мы видим, что GAM может рассматриваться как их
естественное обобщение: вместо фиксированной формы кривой Рикера или
Бивертона-Холта GAM позволяет данным ``говорить за себя'', выявляя
оптимальную форму зависимости без предварительных гипотез, при этом
сохраняя возможность включить нерестовый запас как один из гладких
членов в модель, дополненный другими экологическими факторами. Однако
при всей своей гибкости, GAM, как и LM с GLM, требует тщательной
проверки предположений: мы анализируем графики остатков против
предсказанных значений, чтобы убедиться в отсутствии систематических
отклонений, проверяем нормальность остатков (для LM) или соответствие
выбранному распределению (для GLM/GAM), и исследуем влияние влиятельных
точек, которые могут исказить результаты, особенно в условиях
ограниченных данных, характерных для гидробиологических исследований.
Выбор между LM, GLM и GAM должен основываться не только на
статистических критериях, таких как AIC или кросс-валидация, но и на
биологической интерпретируемости результатов: иногда более простая
модель с меньшей точностью предпочтительнее сложного ``черного ящика'',
особенно когда результаты должны быть понятны начальникам и менеджерам
рыболовства. Практический подход, который обычно рекомендуется
начинающим ихтиологам/гидробиологам, состоит в последовательном
усложнении модели: начните с классической модели запас-пополнение, затем
добавьте средовые факторы через LM/GLM, и только если зависимости явно
нелинейны, перейдите к GAM, всегда проверяя, действительно ли усложнение
модели приводит к биологически значимому улучшению понимания процесса.
Важно помнить, что статистическая модель --- это не самоцель, а
инструмент для понимания биологических процессов, и даже самая
изощренная модель бесполезна, если её результаты нельзя перевести на
язык биологии и применить для устойчивого управления водными ресурсами.
В конечном счете, сочетание классических представлений об экосистемах с
современными статистическими методами позволяет нам строить мост между
фундаментальной биологией и прикладной оценкой запасов, где каждая
модель, от простой линейной регрессии до сложного GAM, вносит свой вклад
в формирование целостного понимания динамики водных биоресурсов.

\begin{Shaded}
\begin{Highlighting}[]
\CommentTok{\# ==============================================================================}
\CommentTok{\# Версия: только LM / GLM(Gamma) / GAM}
\CommentTok{\# Без caret/train: стандартная оценка параметров lm/glm/gam, собственная time{-}slice CV,}
\CommentTok{\# выбор лучшей модели, прогноз 2022–2024, эмпирические интервалы и график.}
\CommentTok{\# ==============================================================================}

\CommentTok{\# 0) Пакеты и окружение {-}{-}{-}{-}{-}{-}{-}{-}{-}{-}{-}{-}{-}{-}{-}{-}{-}{-}{-}{-}{-}{-}{-}{-}{-}{-}{-}{-}{-}{-}{-}{-}{-}{-}{-}{-}{-}{-}{-}{-}{-}{-}{-}{-}{-}{-}{-}{-}{-}{-}{-}{-}{-}{-}{-}{-}}
\ControlFlowTok{if}\NormalTok{ (}\SpecialCharTok{!}\FunctionTok{require}\NormalTok{(}\StringTok{"pacman"}\NormalTok{)) }\FunctionTok{install.packages}\NormalTok{(}\StringTok{"pacman"}\NormalTok{)}
\NormalTok{pacman}\SpecialCharTok{::}\FunctionTok{p\_load}\NormalTok{(}
\NormalTok{  readxl, tidyverse, mgcv, lmtest, car, ggplot2, corrplot}
\NormalTok{)}

\FunctionTok{rm}\NormalTok{(}\AttributeTok{list =} \FunctionTok{ls}\NormalTok{())}
\FunctionTok{set.seed}\NormalTok{(}\DecValTok{123}\NormalTok{)}
\FunctionTok{setwd}\NormalTok{(}\StringTok{"C:/RECRUITMENT/"}\NormalTok{)  }\CommentTok{\# при необходимости измените путь}


\CommentTok{\# 1) Подготовка данных {-}{-}{-}{-}{-}{-}{-}{-}{-}{-}{-}{-}{-}{-}{-}{-}{-}{-}{-}{-}{-}{-}{-}{-}{-}{-}{-}{-}{-}{-}{-}{-}{-}{-}{-}{-}{-}{-}{-}{-}{-}{-}{-}{-}{-}{-}{-}{-}{-}{-}{-}{-}{-}{-}{-}{-}{-}}
\CommentTok{\# Загрузка, приведение типов, создание NAOspring, фиксируем набор признаков}
\NormalTok{DATA }\OtherTok{\textless{}{-}}\NormalTok{ readxl}\SpecialCharTok{::}\FunctionTok{read\_excel}\NormalTok{(}\StringTok{"RECRUITMENT.xlsx"}\NormalTok{, }\AttributeTok{sheet =} \StringTok{"RECRUITMENT"}\NormalTok{) }\SpecialCharTok{\%\textgreater{}\%}
  \FunctionTok{filter}\NormalTok{(YEAR }\SpecialCharTok{\textgreater{}} \DecValTok{1989} \SpecialCharTok{\&}\NormalTok{ YEAR }\SpecialCharTok{\textless{}} \DecValTok{2022}\NormalTok{) }\SpecialCharTok{\%\textgreater{}\%}
  \FunctionTok{mutate}\NormalTok{(}
    \FunctionTok{across}\NormalTok{(}\FunctionTok{starts\_with}\NormalTok{(}\StringTok{"T"}\NormalTok{), as.numeric),}
    \FunctionTok{across}\NormalTok{(}\FunctionTok{starts\_with}\NormalTok{(}\StringTok{"I"}\NormalTok{), as.numeric),}
    \FunctionTok{across}\NormalTok{(}\FunctionTok{starts\_with}\NormalTok{(}\StringTok{"O"}\NormalTok{), as.numeric),}
    \FunctionTok{across}\NormalTok{(}\FunctionTok{where}\NormalTok{(is.character), }\SpecialCharTok{\textasciitilde{}}\FunctionTok{na\_if}\NormalTok{(., }\StringTok{"NA"}\NormalTok{))}
\NormalTok{  )}

\ControlFlowTok{if}\NormalTok{ (}\FunctionTok{all}\NormalTok{(}\FunctionTok{c}\NormalTok{(}\StringTok{"NAO3"}\NormalTok{,}\StringTok{"NAO4"}\NormalTok{,}\StringTok{"NAO5"}\NormalTok{) }\SpecialCharTok{\%in\%} \FunctionTok{names}\NormalTok{(DATA))) \{}
\NormalTok{  DATA }\OtherTok{\textless{}{-}}\NormalTok{ DATA }\SpecialCharTok{\%\textgreater{}\%}
    \FunctionTok{mutate}\NormalTok{(}\AttributeTok{NAOspring =} \FunctionTok{rowMeans}\NormalTok{(}\FunctionTok{pick}\NormalTok{(NAO3, NAO4, NAO5), }\AttributeTok{na.rm =} \ConstantTok{TRUE}\NormalTok{)) }\SpecialCharTok{\%\textgreater{}\%}
    \FunctionTok{select}\NormalTok{(}\SpecialCharTok{{-}}\NormalTok{NAO3, }\SpecialCharTok{{-}}\NormalTok{NAO4, }\SpecialCharTok{{-}}\NormalTok{NAO5)}
\NormalTok{\}}

\NormalTok{needed }\OtherTok{\textless{}{-}} \FunctionTok{c}\NormalTok{(}\StringTok{"codTSB"}\NormalTok{, }\StringTok{"T12"}\NormalTok{, }\StringTok{"I5"}\NormalTok{, }\StringTok{"NAOspring"}\NormalTok{, }\StringTok{"haddock68"}\NormalTok{)}
\FunctionTok{stopifnot}\NormalTok{(}\FunctionTok{all}\NormalTok{(needed }\SpecialCharTok{\%in\%} \FunctionTok{names}\NormalTok{(DATA)))}

\NormalTok{model\_data }\OtherTok{\textless{}{-}}\NormalTok{ DATA }\SpecialCharTok{\%\textgreater{}\%}
  \FunctionTok{select}\NormalTok{(YEAR, }\FunctionTok{all\_of}\NormalTok{(needed), R3haddock) }\SpecialCharTok{\%\textgreater{}\%}
  \FunctionTok{drop\_na}\NormalTok{() }\SpecialCharTok{\%\textgreater{}\%}
  \FunctionTok{arrange}\NormalTok{(YEAR)}

\FunctionTok{write.csv}\NormalTok{(model\_data, }\StringTok{"selected\_predictors\_dataset.csv"}\NormalTok{, }\AttributeTok{row.names =} \ConstantTok{FALSE}\NormalTok{)}
\FunctionTok{glimpse}\NormalTok{(model\_data)}
\end{Highlighting}
\end{Shaded}

\begin{verbatim}
Rows: 32
Columns: 7
$ YEAR      <dbl> 1990, 1991, 1992, 1993, 1994, 1995, 1996, 1997, 1998, 1999, ~
$ codTSB    <dbl> 913000, 1347064, 1687381, 2197863, 2112773, 1849957, 1697388~
$ T12       <dbl> 4.72, 4.66, 4.24, 3.90, 3.96, 4.27, 4.16, 4.07, 4.23, 5.08, ~
$ I5        <dbl> 43, 55, 26, 49, 56, 28, 52, 51, 69, 68, 41, 48, 50, 63, 40, ~
$ NAOspring <dbl> 0.64333333, 0.05666667, 1.78666667, 0.28666667, 0.61000000, ~
$ haddock68 <dbl> 74586, 79205, 53195, 36337, 49122, 81514, 172177, 160886, 96~
$ R3haddock <dbl> 812363, 389416, 99474, 98946, 118812, 63028, 147657, 83270, ~
\end{verbatim}

\begin{Shaded}
\begin{Highlighting}[]
\CommentTok{\# 2) Формулы моделей и вспомогательные функции {-}{-}{-}{-}{-}{-}{-}{-}{-}{-}{-}{-}{-}{-}{-}{-}{-}{-}{-}{-}{-}{-}{-}{-}{-}{-}{-}{-}{-}{-}{-}{-}}
\NormalTok{f\_lm  }\OtherTok{\textless{}{-}} \FunctionTok{as.formula}\NormalTok{(}\StringTok{"R3haddock \textasciitilde{} codTSB + T12 + I5 + NAOspring + haddock68"}\NormalTok{)}
\NormalTok{f\_gam }\OtherTok{\textless{}{-}} \FunctionTok{as.formula}\NormalTok{(}\StringTok{"R3haddock \textasciitilde{} s(codTSB,bs=\textquotesingle{}tp\textquotesingle{},k=5) + s(T12,bs=\textquotesingle{}tp\textquotesingle{},k=5) + s(I5,bs=\textquotesingle{}tp\textquotesingle{},k=5) + s(NAOspring,bs=\textquotesingle{}tp\textquotesingle{},k=5) + s(haddock68,bs=\textquotesingle{}tp\textquotesingle{},k=5)"}\NormalTok{)}

\NormalTok{rmse }\OtherTok{\textless{}{-}} \ControlFlowTok{function}\NormalTok{(a, p) }\FunctionTok{sqrt}\NormalTok{(}\FunctionTok{mean}\NormalTok{((a }\SpecialCharTok{{-}}\NormalTok{ p)}\SpecialCharTok{\^{}}\DecValTok{2}\NormalTok{, }\AttributeTok{na.rm =} \ConstantTok{TRUE}\NormalTok{))}
\NormalTok{mae  }\OtherTok{\textless{}{-}} \ControlFlowTok{function}\NormalTok{(a, p) }\FunctionTok{mean}\NormalTok{(}\FunctionTok{abs}\NormalTok{(a }\SpecialCharTok{{-}}\NormalTok{ p), }\AttributeTok{na.rm =} \ConstantTok{TRUE}\NormalTok{)}
\NormalTok{r2   }\OtherTok{\textless{}{-}} \ControlFlowTok{function}\NormalTok{(a, p) }\DecValTok{1} \SpecialCharTok{{-}} \FunctionTok{sum}\NormalTok{((a }\SpecialCharTok{{-}}\NormalTok{ p)}\SpecialCharTok{\^{}}\DecValTok{2}\NormalTok{, }\AttributeTok{na.rm =} \ConstantTok{TRUE}\NormalTok{) }\SpecialCharTok{/} \FunctionTok{sum}\NormalTok{((a }\SpecialCharTok{{-}} \FunctionTok{mean}\NormalTok{(a))}\SpecialCharTok{\^{}}\DecValTok{2}\NormalTok{, }\AttributeTok{na.rm =} \ConstantTok{TRUE}\NormalTok{)}

\NormalTok{safe\_fit }\OtherTok{\textless{}{-}} \ControlFlowTok{function}\NormalTok{(expr) \{}
\NormalTok{  out }\OtherTok{\textless{}{-}} \FunctionTok{try}\NormalTok{(}\FunctionTok{eval}\NormalTok{(expr), }\AttributeTok{silent =} \ConstantTok{TRUE}\NormalTok{)}
  \ControlFlowTok{if}\NormalTok{ (}\FunctionTok{inherits}\NormalTok{(out, }\StringTok{"try{-}error"}\NormalTok{)) }\ConstantTok{NULL} \ControlFlowTok{else}\NormalTok{ out}
\NormalTok{\}}


\CommentTok{\# 3) Time{-}slice CV (expanding window, h=3) + хронологический тест {-}{-}{-}{-}{-}{-}{-}{-}{-}{-}{-}{-}{-}}
\NormalTok{md }\OtherTok{\textless{}{-}}\NormalTok{ model\_data}
\NormalTok{md\_for\_fit }\OtherTok{\textless{}{-}}\NormalTok{ md }\SpecialCharTok{\%\textgreater{}\%} \FunctionTok{select}\NormalTok{(codTSB, T12, I5, NAOspring, haddock68, R3haddock)}

\NormalTok{n }\OtherTok{\textless{}{-}} \FunctionTok{nrow}\NormalTok{(md\_for\_fit)}
\NormalTok{holdout\_frac }\OtherTok{\textless{}{-}} \FloatTok{0.2}
\NormalTok{n\_test }\OtherTok{\textless{}{-}} \FunctionTok{max}\NormalTok{(}\DecValTok{4}\NormalTok{, }\FunctionTok{ceiling}\NormalTok{(n }\SpecialCharTok{*}\NormalTok{ holdout\_frac))}
\NormalTok{train\_ts }\OtherTok{\textless{}{-}} \FunctionTok{head}\NormalTok{(md\_for\_fit, n }\SpecialCharTok{{-}}\NormalTok{ n\_test)}
\NormalTok{test\_ts  }\OtherTok{\textless{}{-}} \FunctionTok{tail}\NormalTok{(md\_for\_fit, n\_test)}

\NormalTok{n\_train }\OtherTok{\textless{}{-}} \FunctionTok{nrow}\NormalTok{(train\_ts)}
\NormalTok{initial\_frac }\OtherTok{\textless{}{-}} \FloatTok{0.6}
\NormalTok{horizon      }\OtherTok{\textless{}{-}} \DecValTok{3}
\NormalTok{initialWindow }\OtherTok{\textless{}{-}} \FunctionTok{max}\NormalTok{(}\DecValTok{10}\NormalTok{, }\FunctionTok{floor}\NormalTok{(initial\_frac }\SpecialCharTok{*}\NormalTok{ n\_train))}
\ControlFlowTok{if}\NormalTok{ (initialWindow }\SpecialCharTok{+}\NormalTok{ horizon }\SpecialCharTok{\textgreater{}}\NormalTok{ n\_train) initialWindow }\OtherTok{\textless{}{-}}\NormalTok{ n\_train }\SpecialCharTok{{-}}\NormalTok{ horizon}

\CommentTok{\# Аккумулируем метрики и остатки по срезам}
\NormalTok{cv\_summ }\OtherTok{\textless{}{-}} \FunctionTok{tibble}\NormalTok{(}\AttributeTok{Model =} \FunctionTok{character}\NormalTok{(), }\AttributeTok{RMSE =} \FunctionTok{double}\NormalTok{(), }\AttributeTok{MAE =} \FunctionTok{double}\NormalTok{())}
\NormalTok{resids\_cv }\OtherTok{\textless{}{-}} \FunctionTok{list}\NormalTok{(}\AttributeTok{LM =} \FunctionTok{c}\NormalTok{(), }\AttributeTok{GLM =} \FunctionTok{c}\NormalTok{(), }\AttributeTok{GAM =} \FunctionTok{c}\NormalTok{())}

\NormalTok{slice\_id }\OtherTok{\textless{}{-}} \DecValTok{0}
\ControlFlowTok{for}\NormalTok{ (i }\ControlFlowTok{in} \FunctionTok{seq}\NormalTok{(initialWindow, n\_train }\SpecialCharTok{{-}}\NormalTok{ horizon)) \{}
\NormalTok{  slice\_id }\OtherTok{\textless{}{-}}\NormalTok{ slice\_id }\SpecialCharTok{+} \DecValTok{1}
\NormalTok{  idx\_tr }\OtherTok{\textless{}{-}} \DecValTok{1}\SpecialCharTok{:}\NormalTok{i}
\NormalTok{  idx\_te }\OtherTok{\textless{}{-}}\NormalTok{ (i}\SpecialCharTok{+}\DecValTok{1}\NormalTok{)}\SpecialCharTok{:}\NormalTok{(i}\SpecialCharTok{+}\NormalTok{horizon)}
\NormalTok{  dtr }\OtherTok{\textless{}{-}}\NormalTok{ train\_ts[idx\_tr, ]}
\NormalTok{  dte }\OtherTok{\textless{}{-}}\NormalTok{ train\_ts[idx\_te, ]}

  \CommentTok{\# LM}
\NormalTok{  lm\_fit }\OtherTok{\textless{}{-}} \FunctionTok{safe\_fit}\NormalTok{(}\FunctionTok{quote}\NormalTok{(}\FunctionTok{lm}\NormalTok{(f\_lm, }\AttributeTok{data =}\NormalTok{ dtr)))}
  \ControlFlowTok{if}\NormalTok{ (}\SpecialCharTok{!}\FunctionTok{is.null}\NormalTok{(lm\_fit)) \{}
\NormalTok{    pr }\OtherTok{\textless{}{-}} \FunctionTok{try}\NormalTok{(}\FunctionTok{predict}\NormalTok{(lm\_fit, }\AttributeTok{newdata =}\NormalTok{ dte), }\AttributeTok{silent =} \ConstantTok{TRUE}\NormalTok{)}
    \ControlFlowTok{if}\NormalTok{ (}\SpecialCharTok{!}\FunctionTok{inherits}\NormalTok{(pr, }\StringTok{"try{-}error"}\NormalTok{)) \{}
\NormalTok{      cv\_summ }\OtherTok{\textless{}{-}} \FunctionTok{add\_row}\NormalTok{(cv\_summ, }\AttributeTok{Model =} \StringTok{"LM"}\NormalTok{, }\AttributeTok{RMSE =} \FunctionTok{rmse}\NormalTok{(dte}\SpecialCharTok{$}\NormalTok{R3haddock, pr), }\AttributeTok{MAE =} \FunctionTok{mae}\NormalTok{(dte}\SpecialCharTok{$}\NormalTok{R3haddock, pr))}
\NormalTok{      resids\_cv}\SpecialCharTok{$}\NormalTok{LM }\OtherTok{\textless{}{-}} \FunctionTok{c}\NormalTok{(resids\_cv}\SpecialCharTok{$}\NormalTok{LM, dte}\SpecialCharTok{$}\NormalTok{R3haddock }\SpecialCharTok{{-}}\NormalTok{ pr)}
\NormalTok{    \}}
\NormalTok{  \}}

  \CommentTok{\# GLM (Gamma)}
\NormalTok{  glm\_fit }\OtherTok{\textless{}{-}} \FunctionTok{safe\_fit}\NormalTok{(}\FunctionTok{quote}\NormalTok{(}\FunctionTok{glm}\NormalTok{(f\_lm, }\AttributeTok{data =}\NormalTok{ dtr, }\AttributeTok{family =} \FunctionTok{Gamma}\NormalTok{(}\AttributeTok{link =} \StringTok{"log"}\NormalTok{))))}
  \ControlFlowTok{if}\NormalTok{ (}\SpecialCharTok{!}\FunctionTok{is.null}\NormalTok{(glm\_fit)) \{}
\NormalTok{    pr }\OtherTok{\textless{}{-}} \FunctionTok{try}\NormalTok{(}\FunctionTok{predict}\NormalTok{(glm\_fit, }\AttributeTok{newdata =}\NormalTok{ dte, }\AttributeTok{type =} \StringTok{"response"}\NormalTok{), }\AttributeTok{silent =} \ConstantTok{TRUE}\NormalTok{)}
    \ControlFlowTok{if}\NormalTok{ (}\SpecialCharTok{!}\FunctionTok{inherits}\NormalTok{(pr, }\StringTok{"try{-}error"}\NormalTok{)) \{}
\NormalTok{      cv\_summ }\OtherTok{\textless{}{-}} \FunctionTok{add\_row}\NormalTok{(cv\_summ, }\AttributeTok{Model =} \StringTok{"GLM"}\NormalTok{, }\AttributeTok{RMSE =} \FunctionTok{rmse}\NormalTok{(dte}\SpecialCharTok{$}\NormalTok{R3haddock, pr), }\AttributeTok{MAE =} \FunctionTok{mae}\NormalTok{(dte}\SpecialCharTok{$}\NormalTok{R3haddock, pr))}
\NormalTok{      resids\_cv}\SpecialCharTok{$}\NormalTok{GLM }\OtherTok{\textless{}{-}} \FunctionTok{c}\NormalTok{(resids\_cv}\SpecialCharTok{$}\NormalTok{GLM, dte}\SpecialCharTok{$}\NormalTok{R3haddock }\SpecialCharTok{{-}}\NormalTok{ pr)}
\NormalTok{    \}}
\NormalTok{  \}}

  \CommentTok{\# GAM (Gamma log), ограничиваем сложность k для стабильности на малом n}
\NormalTok{  gam\_fit }\OtherTok{\textless{}{-}} \FunctionTok{safe\_fit}\NormalTok{(}\FunctionTok{quote}\NormalTok{(mgcv}\SpecialCharTok{::}\FunctionTok{gam}\NormalTok{(f\_gam, }\AttributeTok{data =}\NormalTok{ dtr, }\AttributeTok{family =} \FunctionTok{Gamma}\NormalTok{(}\AttributeTok{link =} \StringTok{"log"}\NormalTok{), }\AttributeTok{method =} \StringTok{"REML"}\NormalTok{, }\AttributeTok{select =} \ConstantTok{TRUE}\NormalTok{)))}
  \ControlFlowTok{if}\NormalTok{ (}\SpecialCharTok{!}\FunctionTok{is.null}\NormalTok{(gam\_fit)) \{}
\NormalTok{    pr }\OtherTok{\textless{}{-}} \FunctionTok{try}\NormalTok{(}\FunctionTok{predict}\NormalTok{(gam\_fit, }\AttributeTok{newdata =}\NormalTok{ dte, }\AttributeTok{type =} \StringTok{"response"}\NormalTok{), }\AttributeTok{silent =} \ConstantTok{TRUE}\NormalTok{)}
    \ControlFlowTok{if}\NormalTok{ (}\SpecialCharTok{!}\FunctionTok{inherits}\NormalTok{(pr, }\StringTok{"try{-}error"}\NormalTok{)) \{}
\NormalTok{      cv\_summ }\OtherTok{\textless{}{-}} \FunctionTok{add\_row}\NormalTok{(cv\_summ, }\AttributeTok{Model =} \StringTok{"GAM"}\NormalTok{, }\AttributeTok{RMSE =} \FunctionTok{rmse}\NormalTok{(dte}\SpecialCharTok{$}\NormalTok{R3haddock, pr), }\AttributeTok{MAE =} \FunctionTok{mae}\NormalTok{(dte}\SpecialCharTok{$}\NormalTok{R3haddock, pr))}
\NormalTok{      resids\_cv}\SpecialCharTok{$}\NormalTok{GAM }\OtherTok{\textless{}{-}} \FunctionTok{c}\NormalTok{(resids\_cv}\SpecialCharTok{$}\NormalTok{GAM, dte}\SpecialCharTok{$}\NormalTok{R3haddock }\SpecialCharTok{{-}}\NormalTok{ pr)}
\NormalTok{    \}}
\NormalTok{  \}}
\NormalTok{\}}

\CommentTok{\# Средние метрики по моделям}
\NormalTok{cv\_rank }\OtherTok{\textless{}{-}}\NormalTok{ cv\_summ }\SpecialCharTok{\%\textgreater{}\%} \FunctionTok{group\_by}\NormalTok{(Model) }\SpecialCharTok{\%\textgreater{}\%} \FunctionTok{summarise}\NormalTok{(}\AttributeTok{RMSE =} \FunctionTok{mean}\NormalTok{(RMSE, }\AttributeTok{na.rm =} \ConstantTok{TRUE}\NormalTok{), }\AttributeTok{MAE =} \FunctionTok{mean}\NormalTok{(MAE, }\AttributeTok{na.rm =} \ConstantTok{TRUE}\NormalTok{), }\AttributeTok{.groups =} \StringTok{"drop"}\NormalTok{) }\SpecialCharTok{\%\textgreater{}\%} \FunctionTok{arrange}\NormalTok{(RMSE, MAE)}
\FunctionTok{print}\NormalTok{(cv\_rank)}
\end{Highlighting}
\end{Shaded}

\begin{verbatim}
# A tibble: 3 x 3
  Model    RMSE     MAE
  <chr>   <dbl>   <dbl>
1 GLM   280259. 237187.
2 LM    370298. 340884.
3 GAM   504613. 432062.
\end{verbatim}

\begin{Shaded}
\begin{Highlighting}[]
\NormalTok{best\_model\_name }\OtherTok{\textless{}{-}}\NormalTok{ cv\_rank}\SpecialCharTok{$}\NormalTok{Model[}\DecValTok{1}\NormalTok{]}
\FunctionTok{cat}\NormalTok{(}\FunctionTok{sprintf}\NormalTok{(}\StringTok{"}\SpecialCharTok{\textbackslash{}n}\StringTok{Лучшая модель по time{-}slice CV: \%s}\SpecialCharTok{\textbackslash{}n}\StringTok{"}\NormalTok{, best\_model\_name))}
\end{Highlighting}
\end{Shaded}

\begin{verbatim}

Лучшая модель по time-slice CV: GLM
\end{verbatim}

\begin{Shaded}
\begin{Highlighting}[]
\CommentTok{\# Хронологический тест: обучаем на всём train\_ts, прогнозируем на test\_ts}
\NormalTok{fit\_on }\OtherTok{\textless{}{-}} \ControlFlowTok{function}\NormalTok{(model\_name, data) \{}
  \ControlFlowTok{if}\NormalTok{ (model\_name }\SpecialCharTok{==} \StringTok{"LM"}\NormalTok{) }\FunctionTok{return}\NormalTok{(}\FunctionTok{lm}\NormalTok{(f\_lm, }\AttributeTok{data =}\NormalTok{ data))}
  \ControlFlowTok{if}\NormalTok{ (model\_name }\SpecialCharTok{==} \StringTok{"GLM"}\NormalTok{) }\FunctionTok{return}\NormalTok{(}\FunctionTok{glm}\NormalTok{(f\_lm, }\AttributeTok{data =}\NormalTok{ data, }\AttributeTok{family =} \FunctionTok{Gamma}\NormalTok{(}\AttributeTok{link =} \StringTok{"log"}\NormalTok{)))}
\NormalTok{  mgcv}\SpecialCharTok{::}\FunctionTok{gam}\NormalTok{(f\_gam, }\AttributeTok{data =}\NormalTok{ data, }\AttributeTok{family =} \FunctionTok{Gamma}\NormalTok{(}\AttributeTok{link =} \StringTok{"log"}\NormalTok{), }\AttributeTok{method =} \StringTok{"REML"}\NormalTok{, }\AttributeTok{select =} \ConstantTok{TRUE}\NormalTok{)}
\NormalTok{\}}

\NormalTok{predict\_on }\OtherTok{\textless{}{-}} \ControlFlowTok{function}\NormalTok{(fit, newdata, model\_name) \{}
  \ControlFlowTok{if}\NormalTok{ (model\_name }\SpecialCharTok{==} \StringTok{"GLM"}\NormalTok{) }\FunctionTok{return}\NormalTok{(}\FunctionTok{predict}\NormalTok{(fit, }\AttributeTok{newdata =}\NormalTok{ newdata, }\AttributeTok{type =} \StringTok{"response"}\NormalTok{))}
  \ControlFlowTok{if}\NormalTok{ (}\FunctionTok{inherits}\NormalTok{(fit, }\StringTok{"gam"}\NormalTok{)) }\FunctionTok{return}\NormalTok{(}\FunctionTok{predict}\NormalTok{(fit, }\AttributeTok{newdata =}\NormalTok{ newdata, }\AttributeTok{type =} \StringTok{"response"}\NormalTok{))}
  \FunctionTok{predict}\NormalTok{(fit, }\AttributeTok{newdata =}\NormalTok{ newdata)}
\NormalTok{\}}

\NormalTok{fit\_train }\OtherTok{\textless{}{-}} \FunctionTok{fit\_on}\NormalTok{(best\_model\_name, train\_ts)}
\NormalTok{pred\_te   }\OtherTok{\textless{}{-}} \FunctionTok{predict\_on}\NormalTok{(fit\_train, test\_ts, best\_model\_name)}
\NormalTok{test\_metrics }\OtherTok{\textless{}{-}} \FunctionTok{tibble}\NormalTok{(}
  \AttributeTok{Model =}\NormalTok{ best\_model\_name,}
  \AttributeTok{RMSE  =} \FunctionTok{rmse}\NormalTok{(test\_ts}\SpecialCharTok{$}\NormalTok{R3haddock, pred\_te),}
  \AttributeTok{MAE   =} \FunctionTok{mae}\NormalTok{ (test\_ts}\SpecialCharTok{$}\NormalTok{R3haddock, pred\_te),}
  \AttributeTok{R2    =} \FunctionTok{r2}\NormalTok{  (test\_ts}\SpecialCharTok{$}\NormalTok{R3haddock, pred\_te)}
\NormalTok{)}
\FunctionTok{print}\NormalTok{(test\_metrics)}
\end{Highlighting}
\end{Shaded}

\begin{verbatim}
# A tibble: 1 x 4
  Model    RMSE     MAE    R2
  <chr>   <dbl>   <dbl> <dbl>
1 GLM   182048. 141692. 0.363
\end{verbatim}

\begin{Shaded}
\begin{Highlighting}[]
\CommentTok{\# 4) Диагностика моделей (подгонка на всех данных до 2021) {-}{-}{-}{-}{-}{-}{-}{-}{-}{-}{-}{-}{-}{-}{-}{-}{-}{-}{-}}
\NormalTok{full\_fit\_df }\OtherTok{\textless{}{-}}\NormalTok{ md\_for\_fit}

\NormalTok{lm\_full  }\OtherTok{\textless{}{-}} \FunctionTok{lm}\NormalTok{(f\_lm,  }\AttributeTok{data =}\NormalTok{ full\_fit\_df)}
\NormalTok{glm\_full }\OtherTok{\textless{}{-}} \FunctionTok{glm}\NormalTok{(f\_lm, }\AttributeTok{data =}\NormalTok{ full\_fit\_df, }\AttributeTok{family =} \FunctionTok{Gamma}\NormalTok{(}\AttributeTok{link =} \StringTok{"log"}\NormalTok{))}
\NormalTok{gam\_full }\OtherTok{\textless{}{-}}\NormalTok{ mgcv}\SpecialCharTok{::}\FunctionTok{gam}\NormalTok{(f\_gam, }\AttributeTok{data =}\NormalTok{ full\_fit\_df, }\AttributeTok{family =} \FunctionTok{Gamma}\NormalTok{(}\AttributeTok{link =} \StringTok{"log"}\NormalTok{), }\AttributeTok{method =} \StringTok{"REML"}\NormalTok{, }\AttributeTok{select =} \ConstantTok{TRUE}\NormalTok{)}

\FunctionTok{cat}\NormalTok{(}\StringTok{"}\SpecialCharTok{\textbackslash{}n}\StringTok{[LM] Сводка:}\SpecialCharTok{\textbackslash{}n}\StringTok{"}\NormalTok{); }\FunctionTok{print}\NormalTok{(}\FunctionTok{summary}\NormalTok{(lm\_full))}
\end{Highlighting}
\end{Shaded}

\begin{verbatim}

[LM] Сводка:
\end{verbatim}

\begin{verbatim}

Call:
lm(formula = f_lm, data = full_fit_df)

Residuals:
    Min      1Q  Median      3Q     Max 
-257877 -155326  -18935  101135  326940 

Coefficients:
              Estimate Std. Error t value Pr(>|t|)    
(Intercept) -9.189e+05  4.459e+05  -2.061 0.049455 *  
codTSB      -2.406e-01  6.722e-02  -3.579 0.001386 ** 
T12          3.679e+05  8.296e+04   4.435 0.000149 ***
I5          -1.770e+03  3.025e+03  -0.585 0.563536    
NAOspring   -5.125e+04  5.427e+04  -0.944 0.353710    
haddock68    4.385e-01  5.395e-01   0.813 0.423698    
---
Signif. codes:  0 '***' 0.001 '**' 0.01 '*' 0.05 '.' 0.1 ' ' 1

Residual standard error: 192900 on 26 degrees of freedom
Multiple R-squared:  0.547, Adjusted R-squared:  0.4599 
F-statistic: 6.279 on 5 and 26 DF,  p-value: 0.0006042
\end{verbatim}

\begin{Shaded}
\begin{Highlighting}[]
\FunctionTok{cat}\NormalTok{(}\StringTok{"}\SpecialCharTok{\textbackslash{}n}\StringTok{[LM] VIF:}\SpecialCharTok{\textbackslash{}n}\StringTok{"}\NormalTok{); }\FunctionTok{print}\NormalTok{(car}\SpecialCharTok{::}\FunctionTok{vif}\NormalTok{(lm\_full))}
\end{Highlighting}
\end{Shaded}

\begin{verbatim}

[LM] VIF:
\end{verbatim}

\begin{verbatim}
   codTSB       T12        I5 NAOspring haddock68 
 2.487391  1.398254  1.275233  1.035349  2.386554 
\end{verbatim}

\begin{Shaded}
\begin{Highlighting}[]
\FunctionTok{cat}\NormalTok{(}\StringTok{"}\SpecialCharTok{\textbackslash{}n}\StringTok{[LM] Breusch–Pagan:}\SpecialCharTok{\textbackslash{}n}\StringTok{"}\NormalTok{); }\FunctionTok{print}\NormalTok{(lmtest}\SpecialCharTok{::}\FunctionTok{bptest}\NormalTok{(lm\_full))}
\end{Highlighting}
\end{Shaded}

\begin{verbatim}

[LM] Breusch–Pagan:
\end{verbatim}

\begin{verbatim}

    studentized Breusch-Pagan test

data:  lm_full
BP = 5.1481, df = 5, p-value = 0.3981
\end{verbatim}

\begin{Shaded}
\begin{Highlighting}[]
\FunctionTok{cat}\NormalTok{(}\StringTok{"}\SpecialCharTok{\textbackslash{}n}\StringTok{[LM] Durbin–Watson:}\SpecialCharTok{\textbackslash{}n}\StringTok{"}\NormalTok{); }\FunctionTok{print}\NormalTok{(lmtest}\SpecialCharTok{::}\FunctionTok{dwtest}\NormalTok{(lm\_full))}
\end{Highlighting}
\end{Shaded}

\begin{verbatim}

[LM] Durbin–Watson:
\end{verbatim}

\begin{verbatim}

    Durbin-Watson test

data:  lm_full
DW = 1.7745, p-value = 0.1264
alternative hypothesis: true autocorrelation is greater than 0
\end{verbatim}

\begin{Shaded}
\begin{Highlighting}[]
\NormalTok{glm\_resid }\OtherTok{\textless{}{-}} \FunctionTok{residuals}\NormalTok{(glm\_full, }\AttributeTok{type =} \StringTok{"pearson"}\NormalTok{)}
\FunctionTok{cat}\NormalTok{(}\StringTok{"}\SpecialCharTok{\textbackslash{}n}\StringTok{[GLM{-}Gamma] Сводка:}\SpecialCharTok{\textbackslash{}n}\StringTok{"}\NormalTok{); }\FunctionTok{print}\NormalTok{(}\FunctionTok{summary}\NormalTok{(glm\_full))}
\end{Highlighting}
\end{Shaded}

\begin{verbatim}

[GLM-Gamma] Сводка:
\end{verbatim}

\begin{verbatim}

Call:
glm(formula = f_lm, family = Gamma(link = "log"), data = full_fit_df)

Coefficients:
              Estimate Std. Error t value Pr(>|t|)    
(Intercept)  6.907e+00  1.288e+00   5.361 1.30e-05 ***
codTSB      -6.525e-07  1.942e-07  -3.359  0.00242 ** 
T12          1.341e+00  2.397e-01   5.593 7.08e-06 ***
I5           8.270e-03  8.740e-03   0.946  0.35278    
NAOspring    4.898e-02  1.568e-01   0.312  0.75725    
haddock68    1.117e-06  1.559e-06   0.717  0.48000    
---
Signif. codes:  0 '***' 0.001 '**' 0.01 '*' 0.05 '.' 0.1 ' ' 1

(Dispersion parameter for Gamma family taken to be 0.3107759)

    Null deviance: 19.8880  on 31  degrees of freedom
Residual deviance:  8.5197  on 26  degrees of freedom
AIC: 855.29

Number of Fisher Scoring iterations: 8
\end{verbatim}

\begin{Shaded}
\begin{Highlighting}[]
\FunctionTok{cat}\NormalTok{(}\FunctionTok{sprintf}\NormalTok{(}\StringTok{"[GLM{-}Gamma] Pearson dispersion: \%.3f}\SpecialCharTok{\textbackslash{}n}\StringTok{"}\NormalTok{, }\FunctionTok{sum}\NormalTok{(glm\_resid}\SpecialCharTok{\^{}}\DecValTok{2}\NormalTok{, }\AttributeTok{na.rm =} \ConstantTok{TRUE}\NormalTok{) }\SpecialCharTok{/}\NormalTok{ glm\_full}\SpecialCharTok{$}\NormalTok{df.residual))}
\end{Highlighting}
\end{Shaded}

\begin{verbatim}
[GLM-Gamma] Pearson dispersion: 0.311
\end{verbatim}

\begin{Shaded}
\begin{Highlighting}[]
\FunctionTok{cat}\NormalTok{(}\StringTok{"}\SpecialCharTok{\textbackslash{}n}\StringTok{[GAM] Сводка:}\SpecialCharTok{\textbackslash{}n}\StringTok{"}\NormalTok{); }\FunctionTok{print}\NormalTok{(}\FunctionTok{summary}\NormalTok{(gam\_full))}
\end{Highlighting}
\end{Shaded}

\begin{verbatim}

[GAM] Сводка:
\end{verbatim}

\begin{verbatim}

Family: Gamma 
Link function: log 

Formula:
R3haddock ~ s(codTSB, bs = "tp", k = 5) + s(T12, bs = "tp", k = 5) + 
    s(I5, bs = "tp", k = 5) + s(NAOspring, bs = "tp", k = 5) + 
    s(haddock68, bs = "tp", k = 5)

Parametric coefficients:
            Estimate Std. Error t value Pr(>|t|)    
(Intercept) 12.48948    0.08886   140.5   <2e-16 ***
---
Signif. codes:  0 '***' 0.001 '**' 0.01 '*' 0.05 '.' 0.1 ' ' 1

Approximate significance of smooth terms:
                   edf Ref.df     F  p-value    
s(codTSB)    1.7083407      4 4.917 7.43e-05 ***
s(T12)       0.9669993      4 7.752 3.96e-06 ***
s(I5)        0.0001638      4 0.000    0.616    
s(NAOspring) 0.0001092      4 0.000    0.980    
s(haddock68) 0.4529453      4 0.145    0.252    
---
Signif. codes:  0 '***' 0.001 '**' 0.01 '*' 0.05 '.' 0.1 ' ' 1

R-sq.(adj) =  0.592   Deviance explained = 60.2%
-REML = 426.83  Scale est. = 0.2527    n = 32
\end{verbatim}

\begin{Shaded}
\begin{Highlighting}[]
\FunctionTok{cat}\NormalTok{(}\StringTok{"}\SpecialCharTok{\textbackslash{}n}\StringTok{[GAM] Concurvity (коротко):}\SpecialCharTok{\textbackslash{}n}\StringTok{"}\NormalTok{)}
\end{Highlighting}
\end{Shaded}

\begin{verbatim}

[GAM] Concurvity (коротко):
\end{verbatim}

\begin{Shaded}
\begin{Highlighting}[]
\NormalTok{ccv }\OtherTok{\textless{}{-}} \FunctionTok{try}\NormalTok{(mgcv}\SpecialCharTok{::}\FunctionTok{concurvity}\NormalTok{(gam\_full, }\AttributeTok{full =} \ConstantTok{FALSE}\NormalTok{), }\AttributeTok{silent =} \ConstantTok{TRUE}\NormalTok{)}
\ControlFlowTok{if}\NormalTok{ (}\SpecialCharTok{!}\FunctionTok{inherits}\NormalTok{(ccv, }\StringTok{"try{-}error"}\NormalTok{) }\SpecialCharTok{\&\&} \FunctionTok{is.list}\NormalTok{(ccv)) \{}
  \CommentTok{\# Выведем усечённо и безопасно}
  \FunctionTok{print}\NormalTok{(}\FunctionTok{lapply}\NormalTok{(ccv, }\ControlFlowTok{function}\NormalTok{(m) }\ControlFlowTok{if}\NormalTok{ (}\FunctionTok{is.null}\NormalTok{(m)) }\ConstantTok{NULL} \ControlFlowTok{else} \FunctionTok{round}\NormalTok{(m, }\DecValTok{3}\NormalTok{)))}
\NormalTok{\} }\ControlFlowTok{else}\NormalTok{ \{}
  \FunctionTok{cat}\NormalTok{(}\StringTok{"не удалось оценить concurvity}\SpecialCharTok{\textbackslash{}n}\StringTok{"}\NormalTok{)}
\NormalTok{\}}
\end{Highlighting}
\end{Shaded}

\begin{verbatim}
$worst
             para s(codTSB) s(T12) s(I5) s(NAOspring) s(haddock68)
para            1     0.000  0.000 0.000        0.000        0.000
s(codTSB)       0     1.000  0.554 0.298        0.178        0.821
s(T12)          0     0.554  1.000 0.361        0.362        0.347
s(I5)           0     0.298  0.361 1.000        0.182        0.132
s(NAOspring)    0     0.178  0.362 0.182        1.000        0.216
s(haddock68)    0     0.821  0.347 0.132        0.216        1.000

$observed
             para s(codTSB) s(T12) s(I5) s(NAOspring) s(haddock68)
para            1     0.000  0.000 0.000        0.000        0.000
s(codTSB)       0     1.000  0.392 0.166        0.045        0.385
s(T12)          0     0.273  1.000 0.217        0.064        0.218
s(I5)           0     0.166  0.283 1.000        0.047        0.044
s(NAOspring)    0     0.031  0.128 0.114        1.000        0.049
s(haddock68)    0     0.441  0.204 0.113        0.116        1.000

$estimate
             para s(codTSB) s(T12) s(I5) s(NAOspring) s(haddock68)
para            1     0.000  0.000 0.000        0.000        0.000
s(codTSB)       0     1.000  0.320 0.140        0.041        0.747
s(T12)          0     0.281  1.000 0.201        0.068        0.243
s(I5)           0     0.121  0.278 1.000        0.053        0.096
s(NAOspring)    0     0.055  0.128 0.113        1.000        0.089
s(haddock68)    0     0.544  0.205 0.099        0.136        1.000
\end{verbatim}

\begin{Shaded}
\begin{Highlighting}[]
\FunctionTok{invisible}\NormalTok{(}\FunctionTok{try}\NormalTok{(mgcv}\SpecialCharTok{::}\FunctionTok{gam.check}\NormalTok{(gam\_full), }\AttributeTok{silent =} \ConstantTok{TRUE}\NormalTok{))}
\end{Highlighting}
\end{Shaded}

\pandocbounded{\includegraphics[keepaspectratio]{chapter7_files/figure-pdf/unnamed-chunk-4-1.pdf}}

\begin{verbatim}

Method: REML   Optimizer: outer newton
full convergence after 13 iterations.
Gradient range [-4.544099e-05,0.000320414]
(score 426.834 & scale 0.2526969).
Hessian positive definite, eigenvalue range [5.946259e-06,16.95877].
Model rank =  21 / 21 

Basis dimension (k) checking results. Low p-value (k-index<1) may
indicate that k is too low, especially if edf is close to k'.

                   k'      edf k-index p-value
s(codTSB)    4.000000 1.708341    1.12    0.69
s(T12)       4.000000 0.966999    1.29    0.95
s(I5)        4.000000 0.000164    0.86    0.20
s(NAOspring) 4.000000 0.000109    0.99    0.50
s(haddock68) 4.000000 0.452945    1.10    0.73
\end{verbatim}

\begin{Shaded}
\begin{Highlighting}[]
\CommentTok{\# 5) Прогноз 2022–2024 и эмпирические интервалы {-}{-}{-}{-}{-}{-}{-}{-}{-}{-}{-}{-}{-}{-}{-}{-}{-}{-}{-}{-}{-}{-}{-}{-}{-}{-}{-}{-}{-}{-}}
\NormalTok{best\_full }\OtherTok{\textless{}{-}} \ControlFlowTok{switch}\NormalTok{(best\_model\_name,}
  \AttributeTok{LM  =}\NormalTok{ lm\_full,}
  \AttributeTok{GLM =}\NormalTok{ glm\_full,}
  \AttributeTok{GAM =}\NormalTok{ gam\_full}
\NormalTok{)}

\CommentTok{\# Остатки для PI: из CV выбранной модели, иначе из полного фита}
\NormalTok{resids }\OtherTok{\textless{}{-}} \ControlFlowTok{if}\NormalTok{ (}\FunctionTok{length}\NormalTok{(resids\_cv[[best\_model\_name]]) }\SpecialCharTok{\textgreater{}} \DecValTok{5}\NormalTok{) resids\_cv[[best\_model\_name]] }\ControlFlowTok{else} \FunctionTok{residuals}\NormalTok{(best\_full)}

\NormalTok{q025 }\OtherTok{\textless{}{-}} \FunctionTok{as.numeric}\NormalTok{(}\FunctionTok{quantile}\NormalTok{(resids, }\FloatTok{0.025}\NormalTok{, }\AttributeTok{na.rm =} \ConstantTok{TRUE}\NormalTok{))}
\NormalTok{q250 }\OtherTok{\textless{}{-}} \FunctionTok{as.numeric}\NormalTok{(}\FunctionTok{quantile}\NormalTok{(resids, }\FloatTok{0.250}\NormalTok{, }\AttributeTok{na.rm =} \ConstantTok{TRUE}\NormalTok{))}
\NormalTok{q750 }\OtherTok{\textless{}{-}} \FunctionTok{as.numeric}\NormalTok{(}\FunctionTok{quantile}\NormalTok{(resids, }\FloatTok{0.750}\NormalTok{, }\AttributeTok{na.rm =} \ConstantTok{TRUE}\NormalTok{))}
\NormalTok{q975 }\OtherTok{\textless{}{-}} \FunctionTok{as.numeric}\NormalTok{(}\FunctionTok{quantile}\NormalTok{(resids, }\FloatTok{0.975}\NormalTok{, }\AttributeTok{na.rm =} \ConstantTok{TRUE}\NormalTok{))}

\NormalTok{fc\_start }\OtherTok{\textless{}{-}} \DecValTok{2022}
\NormalTok{pred\_cols }\OtherTok{\textless{}{-}} \FunctionTok{c}\NormalTok{(}\StringTok{"codTSB"}\NormalTok{,}\StringTok{"T12"}\NormalTok{,}\StringTok{"I5"}\NormalTok{,}\StringTok{"NAOspring"}\NormalTok{,}\StringTok{"haddock68"}\NormalTok{)}
\NormalTok{mu }\OtherTok{\textless{}{-}}\NormalTok{ md }\SpecialCharTok{\%\textgreater{}\%} \FunctionTok{filter}\NormalTok{(YEAR }\SpecialCharTok{\textgreater{}} \DecValTok{1989} \SpecialCharTok{\&}\NormalTok{ YEAR }\SpecialCharTok{\textless{}}\NormalTok{ fc\_start) }\SpecialCharTok{\%\textgreater{}\%} \FunctionTok{summarise}\NormalTok{(}\FunctionTok{across}\NormalTok{(}\FunctionTok{all\_of}\NormalTok{(pred\_cols), }\SpecialCharTok{\textasciitilde{}}\FunctionTok{mean}\NormalTok{(.x, }\AttributeTok{na.rm =} \ConstantTok{TRUE}\NormalTok{))) }\SpecialCharTok{\%\textgreater{}\%} \FunctionTok{as.list}\NormalTok{()}

\ControlFlowTok{if}\NormalTok{ (}\SpecialCharTok{!}\FunctionTok{exists}\NormalTok{(}\StringTok{"user\_future"}\NormalTok{)) user\_future }\OtherTok{\textless{}{-}} \ConstantTok{NULL}

\NormalTok{build\_future }\OtherTok{\textless{}{-}} \ControlFlowTok{function}\NormalTok{(years, mu, }\AttributeTok{user\_df =} \ConstantTok{NULL}\NormalTok{) \{}
\NormalTok{  df }\OtherTok{\textless{}{-}}\NormalTok{ tibble}\SpecialCharTok{::}\FunctionTok{tibble}\NormalTok{(}\AttributeTok{YEAR =}\NormalTok{ years)}
  \ControlFlowTok{for}\NormalTok{ (v }\ControlFlowTok{in}\NormalTok{ pred\_cols) df[[v]] }\OtherTok{\textless{}{-}}\NormalTok{ mu[[v]]}
  \ControlFlowTok{if}\NormalTok{ (}\SpecialCharTok{!}\FunctionTok{is.null}\NormalTok{(user\_df)) \{}
    \ControlFlowTok{for}\NormalTok{ (i }\ControlFlowTok{in} \FunctionTok{seq\_len}\NormalTok{(}\FunctionTok{nrow}\NormalTok{(user\_df))) \{}
\NormalTok{      yr }\OtherTok{\textless{}{-}}\NormalTok{ user\_df}\SpecialCharTok{$}\NormalTok{YEAR[i]}
      \ControlFlowTok{if}\NormalTok{ (yr }\SpecialCharTok{\%in\%}\NormalTok{ years) \{}
\NormalTok{        idx }\OtherTok{\textless{}{-}} \FunctionTok{which}\NormalTok{(df}\SpecialCharTok{$}\NormalTok{YEAR }\SpecialCharTok{==}\NormalTok{ yr)}
        \ControlFlowTok{for}\NormalTok{ (v }\ControlFlowTok{in} \FunctionTok{intersect}\NormalTok{(pred\_cols, }\FunctionTok{names}\NormalTok{(user\_df))) \{}
\NormalTok{          val }\OtherTok{\textless{}{-}}\NormalTok{ user\_df[[v]][i]}
          \ControlFlowTok{if}\NormalTok{ (}\SpecialCharTok{!}\FunctionTok{is.na}\NormalTok{(val)) df[[v]][idx] }\OtherTok{\textless{}{-}}\NormalTok{ val}
\NormalTok{        \}}
\NormalTok{      \}}
\NormalTok{    \}}
\NormalTok{  \}}
\NormalTok{  df}
\NormalTok{\}}

\NormalTok{future\_years }\OtherTok{\textless{}{-}}\NormalTok{ fc\_start}\SpecialCharTok{:}\DecValTok{2024}
\NormalTok{scenario\_future }\OtherTok{\textless{}{-}} \FunctionTok{build\_future}\NormalTok{(future\_years, mu, user\_future)}

\NormalTok{predict\_best }\OtherTok{\textless{}{-}} \ControlFlowTok{function}\NormalTok{(fit, newdata, model\_name) \{}
  \ControlFlowTok{if}\NormalTok{ (model\_name }\SpecialCharTok{==} \StringTok{"GLM"}\NormalTok{) }\FunctionTok{return}\NormalTok{(}\FunctionTok{predict}\NormalTok{(fit, }\AttributeTok{newdata =}\NormalTok{ newdata, }\AttributeTok{type =} \StringTok{"response"}\NormalTok{))}
  \FunctionTok{predict}\NormalTok{(fit, }\AttributeTok{newdata =}\NormalTok{ newdata)}
\NormalTok{\}}

\NormalTok{pred\_future }\OtherTok{\textless{}{-}} \FunctionTok{predict\_best}\NormalTok{(best\_full, scenario\_future, best\_model\_name)}

\NormalTok{forecast\_tbl }\OtherTok{\textless{}{-}}\NormalTok{ tibble}\SpecialCharTok{::}\FunctionTok{tibble}\NormalTok{(}
  \AttributeTok{YEAR      =}\NormalTok{ scenario\_future}\SpecialCharTok{$}\NormalTok{YEAR,}
  \AttributeTok{Model     =}\NormalTok{ best\_model\_name,}
  \AttributeTok{pred\_mean =} \FunctionTok{as.numeric}\NormalTok{(pred\_future),}
  \AttributeTok{PI50\_low  =}\NormalTok{ pred\_future }\SpecialCharTok{+}\NormalTok{ q250, }\AttributeTok{PI50\_high =}\NormalTok{ pred\_future }\SpecialCharTok{+}\NormalTok{ q750,}
  \AttributeTok{PI95\_low  =}\NormalTok{ pred\_future }\SpecialCharTok{+}\NormalTok{ q025, }\AttributeTok{PI95\_high =}\NormalTok{ pred\_future }\SpecialCharTok{+}\NormalTok{ q975}
\NormalTok{)}


\NormalTok{knitr}\SpecialCharTok{::}\FunctionTok{kable}\NormalTok{(}
\NormalTok{  forecast\_tbl }\SpecialCharTok{\%\textgreater{}\%}\NormalTok{ dplyr}\SpecialCharTok{::}\FunctionTok{mutate}\NormalTok{(dplyr}\SpecialCharTok{::}\FunctionTok{across}\NormalTok{(}\FunctionTok{where}\NormalTok{(is.numeric), }\SpecialCharTok{\textasciitilde{}}\FunctionTok{round}\NormalTok{(.x, }\DecValTok{2}\NormalTok{))),}
  \AttributeTok{caption =} \StringTok{"Holdout{-}метрики (округлено до 2 знаков)"}
\NormalTok{)}
\end{Highlighting}
\end{Shaded}

\begin{longtable}[]{@{}rlrrrrr@{}}
\caption{Holdout-метрики (округлено до 2 знаков)}\tabularnewline
\toprule\noalign{}
YEAR & Model & pred\_mean & PI50\_low & PI50\_high & PI95\_low &
PI95\_high \\
\midrule\noalign{}
\endfirsthead
\toprule\noalign{}
YEAR & Model & pred\_mean & PI50\_low & PI50\_high & PI95\_low &
PI95\_high \\
\midrule\noalign{}
\endhead
\bottomrule\noalign{}
\endlastfoot
2022 & GLM & 268057.6 & 172383 & 509783.3 & -203196.1 & 1051570 \\
2023 & GLM & 268057.6 & 172383 & 509783.3 & -203196.1 & 1051570 \\
2024 & GLM & 268057.6 & 172383 & 509783.3 & -203196.1 & 1051570 \\
\end{longtable}

\begin{Shaded}
\begin{Highlighting}[]
\CommentTok{\# 6) Визуализация 1990–2024 {-}{-}{-}{-}{-}{-}{-}{-}{-}{-}{-}{-}{-}{-}{-}{-}{-}{-}{-}{-}{-}{-}{-}{-}{-}{-}{-}{-}{-}{-}{-}{-}{-}{-}{-}{-}{-}{-}{-}{-}{-}{-}{-}{-}{-}{-}{-}{-}{-}{-}{-}}
\NormalTok{pred\_df }\OtherTok{\textless{}{-}} \FunctionTok{bind\_rows}\NormalTok{(}
\NormalTok{  md }\SpecialCharTok{\%\textgreater{}\%} \FunctionTok{select}\NormalTok{(YEAR, }\FunctionTok{all\_of}\NormalTok{(pred\_cols)),}
\NormalTok{  scenario\_future}
\NormalTok{) }\SpecialCharTok{\%\textgreater{}\%} \FunctionTok{distinct}\NormalTok{(YEAR, }\AttributeTok{.keep\_all =} \ConstantTok{TRUE}\NormalTok{) }\SpecialCharTok{\%\textgreater{}\%} \FunctionTok{arrange}\NormalTok{(YEAR)}

\NormalTok{pred\_df}\SpecialCharTok{$}\NormalTok{Pred      }\OtherTok{\textless{}{-}} \FunctionTok{as.numeric}\NormalTok{(}\FunctionTok{predict\_best}\NormalTok{(best\_full, pred\_df, best\_model\_name))}
\NormalTok{pred\_df}\SpecialCharTok{$}\NormalTok{PI50\_low  }\OtherTok{\textless{}{-}}\NormalTok{ pred\_df}\SpecialCharTok{$}\NormalTok{Pred }\SpecialCharTok{+}\NormalTok{ q250}
\NormalTok{pred\_df}\SpecialCharTok{$}\NormalTok{PI50\_high }\OtherTok{\textless{}{-}}\NormalTok{ pred\_df}\SpecialCharTok{$}\NormalTok{Pred }\SpecialCharTok{+}\NormalTok{ q750}
\NormalTok{pred\_df}\SpecialCharTok{$}\NormalTok{PI95\_low  }\OtherTok{\textless{}{-}}\NormalTok{ pred\_df}\SpecialCharTok{$}\NormalTok{Pred }\SpecialCharTok{+}\NormalTok{ q025}
\NormalTok{pred\_df}\SpecialCharTok{$}\NormalTok{PI95\_high }\OtherTok{\textless{}{-}}\NormalTok{ pred\_df}\SpecialCharTok{$}\NormalTok{Pred }\SpecialCharTok{+}\NormalTok{ q975}

\NormalTok{hist\_df }\OtherTok{\textless{}{-}}\NormalTok{ md }\SpecialCharTok{\%\textgreater{}\%} \FunctionTok{select}\NormalTok{(YEAR, R3haddock)}

\FunctionTok{ggplot}\NormalTok{() }\SpecialCharTok{+}
  \FunctionTok{geom\_ribbon}\NormalTok{(}\AttributeTok{data =}\NormalTok{ pred\_df, }\FunctionTok{aes}\NormalTok{(}\AttributeTok{x =}\NormalTok{ YEAR, }\AttributeTok{ymin =}\NormalTok{ PI95\_low, }\AttributeTok{ymax =}\NormalTok{ PI95\_high), }\AttributeTok{fill =} \StringTok{"grey80"}\NormalTok{, }\AttributeTok{alpha =} \FloatTok{0.25}\NormalTok{) }\SpecialCharTok{+}
  \FunctionTok{geom\_ribbon}\NormalTok{(}\AttributeTok{data =}\NormalTok{ pred\_df, }\FunctionTok{aes}\NormalTok{(}\AttributeTok{x =}\NormalTok{ YEAR, }\AttributeTok{ymin =}\NormalTok{ PI50\_low, }\AttributeTok{ymax =}\NormalTok{ PI50\_high), }\AttributeTok{fill =} \StringTok{"grey60"}\NormalTok{, }\AttributeTok{alpha =} \FloatTok{0.35}\NormalTok{) }\SpecialCharTok{+}
  \FunctionTok{geom\_line}\NormalTok{(}\AttributeTok{data =} \FunctionTok{subset}\NormalTok{(pred\_df, YEAR }\SpecialCharTok{\textless{}}\NormalTok{ fc\_start), }\FunctionTok{aes}\NormalTok{(}\AttributeTok{x =}\NormalTok{ YEAR, }\AttributeTok{y =}\NormalTok{ Pred), }\AttributeTok{color =} \StringTok{"steelblue4"}\NormalTok{, }\AttributeTok{linewidth =} \DecValTok{1}\NormalTok{) }\SpecialCharTok{+}
  \FunctionTok{geom\_line}\NormalTok{(}\AttributeTok{data =} \FunctionTok{subset}\NormalTok{(pred\_df, YEAR }\SpecialCharTok{\textgreater{}=}\NormalTok{ fc\_start}\DecValTok{{-}1}\NormalTok{), }\FunctionTok{aes}\NormalTok{(}\AttributeTok{x =}\NormalTok{ YEAR, }\AttributeTok{y =}\NormalTok{ Pred), }\AttributeTok{color =} \StringTok{"steelblue4"}\NormalTok{, }\AttributeTok{linewidth =} \DecValTok{1}\NormalTok{, }\AttributeTok{linetype =} \StringTok{"dashed"}\NormalTok{) }\SpecialCharTok{+}
  \FunctionTok{geom\_point}\NormalTok{(}\AttributeTok{data =}\NormalTok{ hist\_df, }\FunctionTok{aes}\NormalTok{(}\AttributeTok{x =}\NormalTok{ YEAR, }\AttributeTok{y =}\NormalTok{ R3haddock), }\AttributeTok{color =} \StringTok{"black"}\NormalTok{, }\AttributeTok{size =} \DecValTok{2}\NormalTok{, }\AttributeTok{alpha =} \FloatTok{0.9}\NormalTok{) }\SpecialCharTok{+}
  \FunctionTok{scale\_x\_continuous}\NormalTok{(}\AttributeTok{expand =} \FunctionTok{expansion}\NormalTok{(}\AttributeTok{mult =} \FunctionTok{c}\NormalTok{(}\DecValTok{0}\NormalTok{, }\DecValTok{0}\NormalTok{))) }\SpecialCharTok{+}
  \FunctionTok{labs}\NormalTok{(}\AttributeTok{title =} \FunctionTok{paste0}\NormalTok{(}\StringTok{"Пополнение R3haddock: факт (1990–2021) и прогноз (2022–2024) — "}\NormalTok{, best\_model\_name),}
       \AttributeTok{subtitle =} \StringTok{"Прогноз — пунктир, интервалы — эмпирические из остатков"}\NormalTok{,}
       \AttributeTok{x =} \StringTok{"Год"}\NormalTok{, }\AttributeTok{y =} \StringTok{"R3haddock"}\NormalTok{) }\SpecialCharTok{+}
  \FunctionTok{theme\_minimal}\NormalTok{(}\AttributeTok{base\_size =} \DecValTok{12}\NormalTok{) }\SpecialCharTok{+}
  \FunctionTok{theme}\NormalTok{(}\AttributeTok{legend.position =} \StringTok{"none"}\NormalTok{)}
\end{Highlighting}
\end{Shaded}

\begin{verbatim}
Warning in grid.Call(C_textBounds, as.graphicsAnnot(x$label), x$x, x$y, :
неизвестна ширина символа 0xcf в кодировке CP1251
\end{verbatim}

\begin{verbatim}
Warning in grid.Call(C_textBounds, as.graphicsAnnot(x$label), x$x, x$y, :
неизвестна ширина символа 0xee в кодировке CP1251
\end{verbatim}

\begin{verbatim}
Warning in grid.Call(C_textBounds, as.graphicsAnnot(x$label), x$x, x$y, :
неизвестна ширина символа 0xef в кодировке CP1251
\end{verbatim}

\begin{verbatim}
Warning in grid.Call(C_textBounds, as.graphicsAnnot(x$label), x$x, x$y, :
неизвестна ширина символа 0xee в кодировке CP1251
\end{verbatim}

\begin{verbatim}
Warning in grid.Call(C_textBounds, as.graphicsAnnot(x$label), x$x, x$y, :
неизвестна ширина символа 0xeb в кодировке CP1251
\end{verbatim}

\begin{verbatim}
Warning in grid.Call(C_textBounds, as.graphicsAnnot(x$label), x$x, x$y, :
неизвестна ширина символа 0xed в кодировке CP1251
\end{verbatim}

\begin{verbatim}
Warning in grid.Call(C_textBounds, as.graphicsAnnot(x$label), x$x, x$y, :
неизвестна ширина символа 0xe5 в кодировке CP1251
\end{verbatim}

\begin{verbatim}
Warning in grid.Call(C_textBounds, as.graphicsAnnot(x$label), x$x, x$y, :
неизвестна ширина символа 0xed в кодировке CP1251
\end{verbatim}

\begin{verbatim}
Warning in grid.Call(C_textBounds, as.graphicsAnnot(x$label), x$x, x$y, :
неизвестна ширина символа 0xe8 в кодировке CP1251
\end{verbatim}

\begin{verbatim}
Warning in grid.Call(C_textBounds, as.graphicsAnnot(x$label), x$x, x$y, :
неизвестна ширина символа 0xe5 в кодировке CP1251
\end{verbatim}

\begin{verbatim}
Warning in grid.Call(C_textBounds, as.graphicsAnnot(x$label), x$x, x$y, :
неизвестна ширина символа 0xf4 в кодировке CP1251
\end{verbatim}

\begin{verbatim}
Warning in grid.Call(C_textBounds, as.graphicsAnnot(x$label), x$x, x$y, :
неизвестна ширина символа 0xe0 в кодировке CP1251
\end{verbatim}

\begin{verbatim}
Warning in grid.Call(C_textBounds, as.graphicsAnnot(x$label), x$x, x$y, :
неизвестна ширина символа 0xea в кодировке CP1251
\end{verbatim}

\begin{verbatim}
Warning in grid.Call(C_textBounds, as.graphicsAnnot(x$label), x$x, x$y, :
неизвестна ширина символа 0xf2 в кодировке CP1251
\end{verbatim}

\begin{verbatim}
Warning in grid.Call(C_textBounds, as.graphicsAnnot(x$label), x$x, x$y, :
неизвестна ширина символа 0xe8 в кодировке CP1251
\end{verbatim}

\begin{verbatim}
Warning in grid.Call(C_textBounds, as.graphicsAnnot(x$label), x$x, x$y, :
неизвестна ширина символа 0xef в кодировке CP1251
\end{verbatim}

\begin{verbatim}
Warning in grid.Call(C_textBounds, as.graphicsAnnot(x$label), x$x, x$y, :
неизвестна ширина символа 0xf0 в кодировке CP1251
\end{verbatim}

\begin{verbatim}
Warning in grid.Call(C_textBounds, as.graphicsAnnot(x$label), x$x, x$y, :
неизвестна ширина символа 0xee в кодировке CP1251
\end{verbatim}

\begin{verbatim}
Warning in grid.Call(C_textBounds, as.graphicsAnnot(x$label), x$x, x$y, :
неизвестна ширина символа 0xe3 в кодировке CP1251
\end{verbatim}

\begin{verbatim}
Warning in grid.Call(C_textBounds, as.graphicsAnnot(x$label), x$x, x$y, :
неизвестна ширина символа 0xed в кодировке CP1251
\end{verbatim}

\begin{verbatim}
Warning in grid.Call(C_textBounds, as.graphicsAnnot(x$label), x$x, x$y, :
неизвестна ширина символа 0xee в кодировке CP1251
\end{verbatim}

\begin{verbatim}
Warning in grid.Call(C_textBounds, as.graphicsAnnot(x$label), x$x, x$y, :
неизвестна ширина символа 0xe7 в кодировке CP1251
\end{verbatim}

\begin{verbatim}
Warning in grid.Call(C_textBounds, as.graphicsAnnot(x$label), x$x, x$y, :
неизвестна ширина символа 0xcf в кодировке CP1251
\end{verbatim}

\begin{verbatim}
Warning in grid.Call(C_textBounds, as.graphicsAnnot(x$label), x$x, x$y, :
неизвестна ширина символа 0xf0 в кодировке CP1251
\end{verbatim}

\begin{verbatim}
Warning in grid.Call(C_textBounds, as.graphicsAnnot(x$label), x$x, x$y, :
неизвестна ширина символа 0xee в кодировке CP1251
\end{verbatim}

\begin{verbatim}
Warning in grid.Call(C_textBounds, as.graphicsAnnot(x$label), x$x, x$y, :
неизвестна ширина символа 0xe3 в кодировке CP1251
\end{verbatim}

\begin{verbatim}
Warning in grid.Call(C_textBounds, as.graphicsAnnot(x$label), x$x, x$y, :
неизвестна ширина символа 0xed в кодировке CP1251
\end{verbatim}

\begin{verbatim}
Warning in grid.Call(C_textBounds, as.graphicsAnnot(x$label), x$x, x$y, :
неизвестна ширина символа 0xee в кодировке CP1251
\end{verbatim}

\begin{verbatim}
Warning in grid.Call(C_textBounds, as.graphicsAnnot(x$label), x$x, x$y, :
неизвестна ширина символа 0xe7 в кодировке CP1251
\end{verbatim}

\begin{verbatim}
Warning in grid.Call(C_textBounds, as.graphicsAnnot(x$label), x$x, x$y, :
неизвестна ширина символа 0xef в кодировке CP1251
\end{verbatim}

\begin{verbatim}
Warning in grid.Call(C_textBounds, as.graphicsAnnot(x$label), x$x, x$y, :
неизвестна ширина символа 0xf3 в кодировке CP1251
\end{verbatim}

\begin{verbatim}
Warning in grid.Call(C_textBounds, as.graphicsAnnot(x$label), x$x, x$y, :
неизвестна ширина символа 0xed в кодировке CP1251
\end{verbatim}

\begin{verbatim}
Warning in grid.Call(C_textBounds, as.graphicsAnnot(x$label), x$x, x$y, :
неизвестна ширина символа 0xea в кодировке CP1251
\end{verbatim}

\begin{verbatim}
Warning in grid.Call(C_textBounds, as.graphicsAnnot(x$label), x$x, x$y, :
неизвестна ширина символа 0xf2 в кодировке CP1251
\end{verbatim}

\begin{verbatim}
Warning in grid.Call(C_textBounds, as.graphicsAnnot(x$label), x$x, x$y, :
неизвестна ширина символа 0xe8 в кодировке CP1251
\end{verbatim}

\begin{verbatim}
Warning in grid.Call(C_textBounds, as.graphicsAnnot(x$label), x$x, x$y, :
неизвестна ширина символа 0xf0 в кодировке CP1251
\end{verbatim}

\begin{verbatim}
Warning in grid.Call(C_textBounds, as.graphicsAnnot(x$label), x$x, x$y, :
неизвестна ширина символа 0xe8 в кодировке CP1251
\end{verbatim}

\begin{verbatim}
Warning in grid.Call(C_textBounds, as.graphicsAnnot(x$label), x$x, x$y, :
неизвестна ширина символа 0xed в кодировке CP1251
\end{verbatim}

\begin{verbatim}
Warning in grid.Call(C_textBounds, as.graphicsAnnot(x$label), x$x, x$y, :
неизвестна ширина символа 0xf2 в кодировке CP1251
\end{verbatim}

\begin{verbatim}
Warning in grid.Call(C_textBounds, as.graphicsAnnot(x$label), x$x, x$y, :
неизвестна ширина символа 0xe5 в кодировке CP1251
\end{verbatim}

\begin{verbatim}
Warning in grid.Call(C_textBounds, as.graphicsAnnot(x$label), x$x, x$y, :
неизвестна ширина символа 0xf0 в кодировке CP1251
\end{verbatim}

\begin{verbatim}
Warning in grid.Call(C_textBounds, as.graphicsAnnot(x$label), x$x, x$y, :
неизвестна ширина символа 0xe2 в кодировке CP1251
\end{verbatim}

\begin{verbatim}
Warning in grid.Call(C_textBounds, as.graphicsAnnot(x$label), x$x, x$y, :
неизвестна ширина символа 0xe0 в кодировке CP1251
\end{verbatim}

\begin{verbatim}
Warning in grid.Call(C_textBounds, as.graphicsAnnot(x$label), x$x, x$y, :
неизвестна ширина символа 0xeb в кодировке CP1251
\end{verbatim}

\begin{verbatim}
Warning in grid.Call(C_textBounds, as.graphicsAnnot(x$label), x$x, x$y, :
неизвестна ширина символа 0xfb в кодировке CP1251
\end{verbatim}

\begin{verbatim}
Warning in grid.Call(C_textBounds, as.graphicsAnnot(x$label), x$x, x$y, :
неизвестна ширина символа 0xfd в кодировке CP1251
\end{verbatim}

\begin{verbatim}
Warning in grid.Call(C_textBounds, as.graphicsAnnot(x$label), x$x, x$y, :
неизвестна ширина символа 0xec в кодировке CP1251
\end{verbatim}

\begin{verbatim}
Warning in grid.Call(C_textBounds, as.graphicsAnnot(x$label), x$x, x$y, :
неизвестна ширина символа 0xef в кодировке CP1251
\end{verbatim}

\begin{verbatim}
Warning in grid.Call(C_textBounds, as.graphicsAnnot(x$label), x$x, x$y, :
неизвестна ширина символа 0xe8 в кодировке CP1251
\end{verbatim}

\begin{verbatim}
Warning in grid.Call(C_textBounds, as.graphicsAnnot(x$label), x$x, x$y, :
неизвестна ширина символа 0xf0 в кодировке CP1251
\end{verbatim}

\begin{verbatim}
Warning in grid.Call(C_textBounds, as.graphicsAnnot(x$label), x$x, x$y, :
неизвестна ширина символа 0xe8 в кодировке CP1251
\end{verbatim}

\begin{verbatim}
Warning in grid.Call(C_textBounds, as.graphicsAnnot(x$label), x$x, x$y, :
неизвестна ширина символа 0xf7 в кодировке CP1251
\end{verbatim}

\begin{verbatim}
Warning in grid.Call(C_textBounds, as.graphicsAnnot(x$label), x$x, x$y, :
неизвестна ширина символа 0xe5 в кодировке CP1251
\end{verbatim}

\begin{verbatim}
Warning in grid.Call(C_textBounds, as.graphicsAnnot(x$label), x$x, x$y, :
неизвестна ширина символа 0xf1 в кодировке CP1251
\end{verbatim}

\begin{verbatim}
Warning in grid.Call(C_textBounds, as.graphicsAnnot(x$label), x$x, x$y, :
неизвестна ширина символа 0xea в кодировке CP1251
\end{verbatim}

\begin{verbatim}
Warning in grid.Call(C_textBounds, as.graphicsAnnot(x$label), x$x, x$y, :
неизвестна ширина символа 0xe8 в кодировке CP1251
\end{verbatim}

\begin{verbatim}
Warning in grid.Call(C_textBounds, as.graphicsAnnot(x$label), x$x, x$y, :
неизвестна ширина символа 0xe5 в кодировке CP1251
\end{verbatim}

\begin{verbatim}
Warning in grid.Call(C_textBounds, as.graphicsAnnot(x$label), x$x, x$y, :
неизвестна ширина символа 0xe8 в кодировке CP1251
\end{verbatim}

\begin{verbatim}
Warning in grid.Call(C_textBounds, as.graphicsAnnot(x$label), x$x, x$y, :
неизвестна ширина символа 0xe7 в кодировке CP1251
\end{verbatim}

\begin{verbatim}
Warning in grid.Call(C_textBounds, as.graphicsAnnot(x$label), x$x, x$y, :
неизвестна ширина символа 0xee в кодировке CP1251
\end{verbatim}

\begin{verbatim}
Warning in grid.Call(C_textBounds, as.graphicsAnnot(x$label), x$x, x$y, :
неизвестна ширина символа 0xf1 в кодировке CP1251
\end{verbatim}

\begin{verbatim}
Warning in grid.Call(C_textBounds, as.graphicsAnnot(x$label), x$x, x$y, :
неизвестна ширина символа 0xf2 в кодировке CP1251
\end{verbatim}

\begin{verbatim}
Warning in grid.Call(C_textBounds, as.graphicsAnnot(x$label), x$x, x$y, :
неизвестна ширина символа 0xe0 в кодировке CP1251
\end{verbatim}

\begin{verbatim}
Warning in grid.Call(C_textBounds, as.graphicsAnnot(x$label), x$x, x$y, :
неизвестна ширина символа 0xf2 в кодировке CP1251
\end{verbatim}

\begin{verbatim}
Warning in grid.Call(C_textBounds, as.graphicsAnnot(x$label), x$x, x$y, :
неизвестна ширина символа 0xea в кодировке CP1251
\end{verbatim}

\begin{verbatim}
Warning in grid.Call(C_textBounds, as.graphicsAnnot(x$label), x$x, x$y, :
неизвестна ширина символа 0xee в кодировке CP1251
\end{verbatim}

\begin{verbatim}
Warning in grid.Call(C_textBounds, as.graphicsAnnot(x$label), x$x, x$y, :
неизвестна ширина символа 0xe2 в кодировке CP1251
\end{verbatim}

\begin{verbatim}
Warning in grid.Call(C_textBounds, as.graphicsAnnot(x$label), x$x, x$y, :
неизвестна ширина символа 0xc3 в кодировке CP1251
\end{verbatim}

\begin{verbatim}
Warning in grid.Call(C_textBounds, as.graphicsAnnot(x$label), x$x, x$y, :
неизвестна ширина символа 0xee в кодировке CP1251
\end{verbatim}

\begin{verbatim}
Warning in grid.Call(C_textBounds, as.graphicsAnnot(x$label), x$x, x$y, :
неизвестна ширина символа 0xe4 в кодировке CP1251
\end{verbatim}

\begin{verbatim}
Warning in grid.Call.graphics(C_text, as.graphicsAnnot(x$label), x$x, x$y, :
неизвестна ширина символа 0xc3 в кодировке CP1251
\end{verbatim}

\begin{verbatim}
Warning in grid.Call.graphics(C_text, as.graphicsAnnot(x$label), x$x, x$y, :
неизвестна ширина символа 0xee в кодировке CP1251
\end{verbatim}

\begin{verbatim}
Warning in grid.Call.graphics(C_text, as.graphicsAnnot(x$label), x$x, x$y, :
неизвестна ширина символа 0xe4 в кодировке CP1251
\end{verbatim}

\begin{verbatim}
Warning in grid.Call.graphics(C_text, as.graphicsAnnot(x$label), x$x, x$y, :
неизвестна ширина символа 0xcf в кодировке CP1251
\end{verbatim}

\begin{verbatim}
Warning in grid.Call.graphics(C_text, as.graphicsAnnot(x$label), x$x, x$y, :
неизвестна ширина символа 0xf0 в кодировке CP1251
\end{verbatim}

\begin{verbatim}
Warning in grid.Call.graphics(C_text, as.graphicsAnnot(x$label), x$x, x$y, :
неизвестна ширина символа 0xee в кодировке CP1251
\end{verbatim}

\begin{verbatim}
Warning in grid.Call.graphics(C_text, as.graphicsAnnot(x$label), x$x, x$y, :
неизвестна ширина символа 0xe3 в кодировке CP1251
\end{verbatim}

\begin{verbatim}
Warning in grid.Call.graphics(C_text, as.graphicsAnnot(x$label), x$x, x$y, :
неизвестна ширина символа 0xed в кодировке CP1251
\end{verbatim}

\begin{verbatim}
Warning in grid.Call.graphics(C_text, as.graphicsAnnot(x$label), x$x, x$y, :
неизвестна ширина символа 0xee в кодировке CP1251
\end{verbatim}

\begin{verbatim}
Warning in grid.Call.graphics(C_text, as.graphicsAnnot(x$label), x$x, x$y, :
неизвестна ширина символа 0xe7 в кодировке CP1251
\end{verbatim}

\begin{verbatim}
Warning in grid.Call.graphics(C_text, as.graphicsAnnot(x$label), x$x, x$y, :
неизвестна ширина символа 0xef в кодировке CP1251
\end{verbatim}

\begin{verbatim}
Warning in grid.Call.graphics(C_text, as.graphicsAnnot(x$label), x$x, x$y, :
неизвестна ширина символа 0xf3 в кодировке CP1251
\end{verbatim}

\begin{verbatim}
Warning in grid.Call.graphics(C_text, as.graphicsAnnot(x$label), x$x, x$y, :
неизвестна ширина символа 0xed в кодировке CP1251
\end{verbatim}

\begin{verbatim}
Warning in grid.Call.graphics(C_text, as.graphicsAnnot(x$label), x$x, x$y, :
неизвестна ширина символа 0xea в кодировке CP1251
\end{verbatim}

\begin{verbatim}
Warning in grid.Call.graphics(C_text, as.graphicsAnnot(x$label), x$x, x$y, :
неизвестна ширина символа 0xf2 в кодировке CP1251
\end{verbatim}

\begin{verbatim}
Warning in grid.Call.graphics(C_text, as.graphicsAnnot(x$label), x$x, x$y, :
неизвестна ширина символа 0xe8 в кодировке CP1251
\end{verbatim}

\begin{verbatim}
Warning in grid.Call.graphics(C_text, as.graphicsAnnot(x$label), x$x, x$y, :
неизвестна ширина символа 0xf0 в кодировке CP1251
\end{verbatim}

\begin{verbatim}
Warning in grid.Call.graphics(C_text, as.graphicsAnnot(x$label), x$x, x$y, :
неизвестна ширина символа 0xe8 в кодировке CP1251
\end{verbatim}

\begin{verbatim}
Warning in grid.Call.graphics(C_text, as.graphicsAnnot(x$label), x$x, x$y, :
неизвестна ширина символа 0xed в кодировке CP1251
\end{verbatim}

\begin{verbatim}
Warning in grid.Call.graphics(C_text, as.graphicsAnnot(x$label), x$x, x$y, :
неизвестна ширина символа 0xf2 в кодировке CP1251
\end{verbatim}

\begin{verbatim}
Warning in grid.Call.graphics(C_text, as.graphicsAnnot(x$label), x$x, x$y, :
неизвестна ширина символа 0xe5 в кодировке CP1251
\end{verbatim}

\begin{verbatim}
Warning in grid.Call.graphics(C_text, as.graphicsAnnot(x$label), x$x, x$y, :
неизвестна ширина символа 0xf0 в кодировке CP1251
\end{verbatim}

\begin{verbatim}
Warning in grid.Call.graphics(C_text, as.graphicsAnnot(x$label), x$x, x$y, :
неизвестна ширина символа 0xe2 в кодировке CP1251
\end{verbatim}

\begin{verbatim}
Warning in grid.Call.graphics(C_text, as.graphicsAnnot(x$label), x$x, x$y, :
неизвестна ширина символа 0xe0 в кодировке CP1251
\end{verbatim}

\begin{verbatim}
Warning in grid.Call.graphics(C_text, as.graphicsAnnot(x$label), x$x, x$y, :
неизвестна ширина символа 0xeb в кодировке CP1251
\end{verbatim}

\begin{verbatim}
Warning in grid.Call.graphics(C_text, as.graphicsAnnot(x$label), x$x, x$y, :
неизвестна ширина символа 0xfb в кодировке CP1251
\end{verbatim}

\begin{verbatim}
Warning in grid.Call.graphics(C_text, as.graphicsAnnot(x$label), x$x, x$y, :
неизвестна ширина символа 0xfd в кодировке CP1251
\end{verbatim}

\begin{verbatim}
Warning in grid.Call.graphics(C_text, as.graphicsAnnot(x$label), x$x, x$y, :
неизвестна ширина символа 0xec в кодировке CP1251
\end{verbatim}

\begin{verbatim}
Warning in grid.Call.graphics(C_text, as.graphicsAnnot(x$label), x$x, x$y, :
неизвестна ширина символа 0xef в кодировке CP1251
\end{verbatim}

\begin{verbatim}
Warning in grid.Call.graphics(C_text, as.graphicsAnnot(x$label), x$x, x$y, :
неизвестна ширина символа 0xe8 в кодировке CP1251
\end{verbatim}

\begin{verbatim}
Warning in grid.Call.graphics(C_text, as.graphicsAnnot(x$label), x$x, x$y, :
неизвестна ширина символа 0xf0 в кодировке CP1251
\end{verbatim}

\begin{verbatim}
Warning in grid.Call.graphics(C_text, as.graphicsAnnot(x$label), x$x, x$y, :
неизвестна ширина символа 0xe8 в кодировке CP1251
\end{verbatim}

\begin{verbatim}
Warning in grid.Call.graphics(C_text, as.graphicsAnnot(x$label), x$x, x$y, :
неизвестна ширина символа 0xf7 в кодировке CP1251
\end{verbatim}

\begin{verbatim}
Warning in grid.Call.graphics(C_text, as.graphicsAnnot(x$label), x$x, x$y, :
неизвестна ширина символа 0xe5 в кодировке CP1251
\end{verbatim}

\begin{verbatim}
Warning in grid.Call.graphics(C_text, as.graphicsAnnot(x$label), x$x, x$y, :
неизвестна ширина символа 0xf1 в кодировке CP1251
\end{verbatim}

\begin{verbatim}
Warning in grid.Call.graphics(C_text, as.graphicsAnnot(x$label), x$x, x$y, :
неизвестна ширина символа 0xea в кодировке CP1251
\end{verbatim}

\begin{verbatim}
Warning in grid.Call.graphics(C_text, as.graphicsAnnot(x$label), x$x, x$y, :
неизвестна ширина символа 0xe8 в кодировке CP1251
\end{verbatim}

\begin{verbatim}
Warning in grid.Call.graphics(C_text, as.graphicsAnnot(x$label), x$x, x$y, :
неизвестна ширина символа 0xe5 в кодировке CP1251
\end{verbatim}

\begin{verbatim}
Warning in grid.Call.graphics(C_text, as.graphicsAnnot(x$label), x$x, x$y, :
неизвестна ширина символа 0xe8 в кодировке CP1251
\end{verbatim}

\begin{verbatim}
Warning in grid.Call.graphics(C_text, as.graphicsAnnot(x$label), x$x, x$y, :
неизвестна ширина символа 0xe7 в кодировке CP1251
\end{verbatim}

\begin{verbatim}
Warning in grid.Call.graphics(C_text, as.graphicsAnnot(x$label), x$x, x$y, :
неизвестна ширина символа 0xee в кодировке CP1251
\end{verbatim}

\begin{verbatim}
Warning in grid.Call.graphics(C_text, as.graphicsAnnot(x$label), x$x, x$y, :
неизвестна ширина символа 0xf1 в кодировке CP1251
\end{verbatim}

\begin{verbatim}
Warning in grid.Call.graphics(C_text, as.graphicsAnnot(x$label), x$x, x$y, :
неизвестна ширина символа 0xf2 в кодировке CP1251
\end{verbatim}

\begin{verbatim}
Warning in grid.Call.graphics(C_text, as.graphicsAnnot(x$label), x$x, x$y, :
неизвестна ширина символа 0xe0 в кодировке CP1251
\end{verbatim}

\begin{verbatim}
Warning in grid.Call.graphics(C_text, as.graphicsAnnot(x$label), x$x, x$y, :
неизвестна ширина символа 0xf2 в кодировке CP1251
\end{verbatim}

\begin{verbatim}
Warning in grid.Call.graphics(C_text, as.graphicsAnnot(x$label), x$x, x$y, :
неизвестна ширина символа 0xea в кодировке CP1251
\end{verbatim}

\begin{verbatim}
Warning in grid.Call.graphics(C_text, as.graphicsAnnot(x$label), x$x, x$y, :
неизвестна ширина символа 0xee в кодировке CP1251
\end{verbatim}

\begin{verbatim}
Warning in grid.Call.graphics(C_text, as.graphicsAnnot(x$label), x$x, x$y, :
неизвестна ширина символа 0xe2 в кодировке CP1251
\end{verbatim}

\begin{verbatim}
Warning in grid.Call.graphics(C_text, as.graphicsAnnot(x$label), x$x, x$y, :
неизвестна ширина символа 0xcf в кодировке CP1251
\end{verbatim}

\begin{verbatim}
Warning in grid.Call.graphics(C_text, as.graphicsAnnot(x$label), x$x, x$y, :
неизвестна ширина символа 0xee в кодировке CP1251
\end{verbatim}

\begin{verbatim}
Warning in grid.Call.graphics(C_text, as.graphicsAnnot(x$label), x$x, x$y, :
неизвестна ширина символа 0xef в кодировке CP1251
\end{verbatim}

\begin{verbatim}
Warning in grid.Call.graphics(C_text, as.graphicsAnnot(x$label), x$x, x$y, :
неизвестна ширина символа 0xee в кодировке CP1251
\end{verbatim}

\begin{verbatim}
Warning in grid.Call.graphics(C_text, as.graphicsAnnot(x$label), x$x, x$y, :
неизвестна ширина символа 0xeb в кодировке CP1251
\end{verbatim}

\begin{verbatim}
Warning in grid.Call.graphics(C_text, as.graphicsAnnot(x$label), x$x, x$y, :
неизвестна ширина символа 0xed в кодировке CP1251
\end{verbatim}

\begin{verbatim}
Warning in grid.Call.graphics(C_text, as.graphicsAnnot(x$label), x$x, x$y, :
неизвестна ширина символа 0xe5 в кодировке CP1251
\end{verbatim}

\begin{verbatim}
Warning in grid.Call.graphics(C_text, as.graphicsAnnot(x$label), x$x, x$y, :
неизвестна ширина символа 0xed в кодировке CP1251
\end{verbatim}

\begin{verbatim}
Warning in grid.Call.graphics(C_text, as.graphicsAnnot(x$label), x$x, x$y, :
неизвестна ширина символа 0xe8 в кодировке CP1251
\end{verbatim}

\begin{verbatim}
Warning in grid.Call.graphics(C_text, as.graphicsAnnot(x$label), x$x, x$y, :
неизвестна ширина символа 0xe5 в кодировке CP1251
\end{verbatim}

\begin{verbatim}
Warning in grid.Call.graphics(C_text, as.graphicsAnnot(x$label), x$x, x$y, :
неизвестна ширина символа 0xf4 в кодировке CP1251
\end{verbatim}

\begin{verbatim}
Warning in grid.Call.graphics(C_text, as.graphicsAnnot(x$label), x$x, x$y, :
неизвестна ширина символа 0xe0 в кодировке CP1251
\end{verbatim}

\begin{verbatim}
Warning in grid.Call.graphics(C_text, as.graphicsAnnot(x$label), x$x, x$y, :
неизвестна ширина символа 0xea в кодировке CP1251
\end{verbatim}

\begin{verbatim}
Warning in grid.Call.graphics(C_text, as.graphicsAnnot(x$label), x$x, x$y, :
неизвестна ширина символа 0xf2 в кодировке CP1251
\end{verbatim}

\begin{verbatim}
Warning in grid.Call.graphics(C_text, as.graphicsAnnot(x$label), x$x, x$y, :
неизвестна ширина символа 0xe8 в кодировке CP1251
\end{verbatim}

\begin{verbatim}
Warning in grid.Call.graphics(C_text, as.graphicsAnnot(x$label), x$x, x$y, :
неизвестна ширина символа 0xef в кодировке CP1251
\end{verbatim}

\begin{verbatim}
Warning in grid.Call.graphics(C_text, as.graphicsAnnot(x$label), x$x, x$y, :
неизвестна ширина символа 0xf0 в кодировке CP1251
\end{verbatim}

\begin{verbatim}
Warning in grid.Call.graphics(C_text, as.graphicsAnnot(x$label), x$x, x$y, :
неизвестна ширина символа 0xee в кодировке CP1251
\end{verbatim}

\begin{verbatim}
Warning in grid.Call.graphics(C_text, as.graphicsAnnot(x$label), x$x, x$y, :
неизвестна ширина символа 0xe3 в кодировке CP1251
\end{verbatim}

\begin{verbatim}
Warning in grid.Call.graphics(C_text, as.graphicsAnnot(x$label), x$x, x$y, :
неизвестна ширина символа 0xed в кодировке CP1251
\end{verbatim}

\begin{verbatim}
Warning in grid.Call.graphics(C_text, as.graphicsAnnot(x$label), x$x, x$y, :
неизвестна ширина символа 0xee в кодировке CP1251
\end{verbatim}

\begin{verbatim}
Warning in grid.Call.graphics(C_text, as.graphicsAnnot(x$label), x$x, x$y, :
неизвестна ширина символа 0xe7 в кодировке CP1251
\end{verbatim}

\pandocbounded{\includegraphics[keepaspectratio]{chapter7_files/figure-pdf/unnamed-chunk-4-2.pdf}}

\begin{Shaded}
\begin{Highlighting}[]
\CommentTok{\# AIC{-}таблица (LM/GLM сопоставимы напрямую; для GAM также показываем ML)}
\FunctionTok{cat}\NormalTok{(}\StringTok{"}\SpecialCharTok{\textbackslash{}n}\StringTok{AIC (LM): "}\NormalTok{,  }\FunctionTok{AIC}\NormalTok{(lm\_full),  }\StringTok{"}\SpecialCharTok{\textbackslash{}n}\StringTok{"}\NormalTok{, }\AttributeTok{sep =} \StringTok{""}\NormalTok{)}
\end{Highlighting}
\end{Shaded}

\begin{verbatim}

AIC (LM): 877.0549
\end{verbatim}

\begin{Shaded}
\begin{Highlighting}[]
\FunctionTok{cat}\NormalTok{(}\StringTok{"AIC (GLM): "}\NormalTok{, }\FunctionTok{AIC}\NormalTok{(glm\_full), }\StringTok{"}\SpecialCharTok{\textbackslash{}n}\StringTok{"}\NormalTok{, }\AttributeTok{sep =} \StringTok{""}\NormalTok{)}
\end{Highlighting}
\end{Shaded}

\begin{verbatim}
AIC (GLM): 855.2949
\end{verbatim}

\begin{Shaded}
\begin{Highlighting}[]
\NormalTok{gam\_full\_ml }\OtherTok{\textless{}{-}}\NormalTok{ mgcv}\SpecialCharTok{::}\FunctionTok{gam}\NormalTok{(f\_gam, }\AttributeTok{data =}\NormalTok{ full\_fit\_df, }\AttributeTok{family =} \FunctionTok{Gamma}\NormalTok{(}\AttributeTok{link =} \StringTok{"log"}\NormalTok{), }\AttributeTok{method =} \StringTok{"ML"}\NormalTok{, }\AttributeTok{select =} \ConstantTok{TRUE}\NormalTok{)}
\FunctionTok{cat}\NormalTok{(}\StringTok{"AIC (GAM, REML): "}\NormalTok{, }\FunctionTok{AIC}\NormalTok{(gam\_full),    }\StringTok{"}\SpecialCharTok{\textbackslash{}n}\StringTok{"}\NormalTok{, }\AttributeTok{sep =} \StringTok{""}\NormalTok{)}
\end{Highlighting}
\end{Shaded}

\begin{verbatim}
AIC (GAM, REML): 850.8686
\end{verbatim}

\begin{Shaded}
\begin{Highlighting}[]
\FunctionTok{cat}\NormalTok{(}\StringTok{"AIC (GAM, ML):   "}\NormalTok{, }\FunctionTok{AIC}\NormalTok{(gam\_full\_ml), }\StringTok{"}\SpecialCharTok{\textbackslash{}n}\StringTok{"}\NormalTok{, }\AttributeTok{sep =} \StringTok{""}\NormalTok{)}
\end{Highlighting}
\end{Shaded}

\begin{verbatim}
AIC (GAM, ML):   850.7948
\end{verbatim}

\begin{Shaded}
\begin{Highlighting}[]
\CommentTok{\# ============================================================================}
\CommentTok{\# Конец}
\CommentTok{\# ============================================================================}
\end{Highlighting}
\end{Shaded}

\section{Полный цикл от факторов до ансамблевого
прогноза}\label{ux43fux43eux43bux43dux44bux439-ux446ux438ux43aux43b-ux43eux442-ux444ux430ux43aux442ux43eux440ux43eux432-ux434ux43e-ux430ux43dux441ux430ux43cux431ux43bux435ux432ux43eux433ux43e-ux43fux440ux43eux433ux43dux43eux437ux430}

Полный цикл анализа от идентификации ключевых факторов до создания
надежного ансамблевого прогноза пополнения рыбных запасов представляет
собой сложный, но систематизированный процесс, требующий как глубокого
понимания биологических процессов, так и владения современными методами
анализа данных. Начиная с формирования исходного набора предикторов,
включающего как биологические переменные (нерестовый запас, биомасса
хищников), так и комплексные океанографические показатели (температура,
соленость, климатические индексы), мы проходим через строгую
последовательность этапов, каждый из которых важен для конечного
результата. На этапе подготовки данных мы не просто приводим информацию
к числовому формату и заменяем строковые обозначения пропущенных
значений «NA» на стандартные NA, но и проводим глубокий анализ
корреляционной структуры, устраняя мультиколлинеарность через анализ
корреляций и VIF-диагностику, что важно для корректной интерпретации
последующих моделей. Для обработки пропусков мы применяем медианную
импутацию, которая представляет собой простой и устойчивый к выбросам
метод, хотя в некоторых случаях могут быть использованы и более сложные
методы, такие как KNN-импутация или множественная импутация с
использованием пакета MICE, особенно когда данные имеют сложную
структуру или временные зависимости.

Затем следует этап отбора предикторов, где мы применяем два
комплементарных метода: Boruta на основе Random Forest для выявления
нелинейных зависимостей и LASSO-регрессию для линейного отбора с
регуляризацией. Их объединение позволяет получить устойчивый набор
предикторов, дополненный биологически значимыми переменными по
экспертной оценке, что создает баланс между статистической значимостью и
содержательной интерпретируемостью. Этот этап является мостом между
классической ихтиологией и современными методами анализа, где экспертные
знания биолога взаимодействуют с алгоритмической строгостью статистики,
гарантируя включение ключевых факторов, таких как нерестовый запас,
который должен присутствовать в модели по самой своей природе процесса
пополнения.

После подготовки данных мы переходим к сравнению различных семейств
моделей через единую кросс-валидационную процедуру (5-fold CV) с
последующим хронологическим тестированием на отложенной выборке. Помимо
линейных и обобщенных линейных моделей (LM, GLM), обобщенных аддитивных
моделей (GAM), мы тестируем современные алгоритмы машинного обучения:
Random Forest для улавливания сложных нелинейных зависимостей и
взаимодействий между факторами, будучи при этом устойчивым к шуму и
выбросам; XGBoost, с его градиентным бустингом над деревьями решений,
часто дающий высочайшую точность прогноза; SVM с радиальным ядром для
сложных разделяющих поверхностей; и нейронные сети для автоматического
извлечения признаков. Каждая модель оценивается по комплексу метрик:
RMSE, MAE, R² и MAPE, что позволяет сравнивать их прогностическую силу
на разных участках данных и выявлять модели, которые лучше всего
справляются с конкретными аспектами прогнозирования.

Особое внимание уделяется временным характеристикам данных, поскольку
при анализе водных биоресурсов мы имеем дело с временными рядами, где
случайное перемешивание данных приведет к утечке информации из будущего
в прошлое, искусственно завысив качество прогноза. Для решения этой
проблемы мы применяем специализированную time-slice кросс-валидацию с
расширяющимся окном и горизонтом прогноза 3 года, которая имитирует
реальные условия прогнозирования, обучаясь только на данных из прошлого
и проверяя на последующих периодах. Это позволяет оценить устойчивость
моделей к временным сдвигам и их способность к экстраполяции, что
критически важно для практических задач управления рыбными запасами.

Выбор окончательной модели --- это не просто вопрос максимальной
точности на кросс-валидации, а сложный компромисс между точностью,
интерпретируемостью и биологической правдоподобностью. Кульминацией
цикла становится построение ансамблевой модели, комбинирующей сильные
стороны отдельных алгоритмов. В нашем анализе оптимальный ансамбль
(CUBIST + LM) строится через взвешенное усреднение предсказаний, где
веса определяются на основе кросс-валидационной ошибки --- например,
75\% веса приходится на мощную нелинейную модель Cubist, а 25\% --- на
простую и устойчивую линейную регрессию. Такой подход позволяет
нивелировать индивидуальные недостатки моделей, сохранить
интерпретируемость линейных моделей, где биолог может понять, как именно
каждый фактор влияет на прогноз, и при этом использовать гибкость
методов машинного обучения для захвата сложных нелинейных паттернов,
которые могут ускользнуть от классических статистических методов.

Важнейшим компонентом становится оценка неопределенности через
эмпирические доверительные интервалы, построенные на основе
распределения остатков ансамблевой модели. Мы используем квантили
остатков из кросс-валидации для построения 50\% и 95\% доверительных
интервалов, что позволяет получить не только точечный прогноз, но и меру
его надежности, важную для принятия управленческих решений. Это дает
возможность визуализировать не только ожидаемое значение пополнения, но
и диапазон возможных сценариев, что особенно важно в условиях высокой
экологической неопределенности.

Финальная визуализация представляет собой совмещение исторических данных
с прогнозом на 3 года вперед, где исторические данные отображаются
сплошной линией, прогноз --- пунктиром, а 50\% и 95\% доверительные
интервалы --- серыми лентами различной интенсивности. Такой график не
только демонстрирует результат, но и позволяет визуально оценить
точность модели на исторических данных и неопределенность будущих
предсказаний, делая результаты доступными не только для статистиков, но
и для управленцев и политиков, принимающих решения на основе этих
прогнозов.

Представленный цикл является итеративным процессом: прогнозная точность
ансамбля может быть улучшена через включение новых предикторов, изучение
влияния предикторов с задержкой (лагами), тонкую настройку
гиперпараметров моделей и обновление данных по мере их поступления. Этот
подход представляет собой практический компромисс между статистической
строгостью, вычислительной эффективностью и биологической
интерпретируемостью, делая его мощным инструментом для решения
прикладных задач оценки водных биоресурсов, где каждый этап, от
первичной обработки данных до финального прогноза, подчинен одной цели
--- обеспечению устойчивого управления рыбными запасами на основе
надежного научного анализа.

Скрипт лучше скачать
\href{https://mombus.github.io/cRab/data/RECRUITMENT_MAIN.R}{целиком}).

\begin{Shaded}
\begin{Highlighting}[]
\CommentTok{\# ==============================================================================}
\CommentTok{\# ПРАКТИЧЕСКОЕ ЗАНЯТИЕ: АНАЛИЗ ФАКТОРОВ И ПРОГНОЗ ПОПОЛНЕНИЯ ЗАПАСА}
\CommentTok{\# Курс: "Оценка водных биоресурсов в среде R (для начинающих)"}
\CommentTok{\# Автор: Баканев С. В. Дата:20.08.2025}
\CommentTok{\# Структура:}
\CommentTok{\# 1) Подготовка данных и выбор предикторов}
\CommentTok{\# 2) Базовое сравнение моделей (5{-}fold CV + holdout)}
\CommentTok{\# 3) Выбор лучшей прогностической модели (time{-}slice CV на 3 года + хронологический тест)}
\CommentTok{\# 4) Прогноз 2022–2024 (ансамбль CUBIST+LM) и график 1990–2024 с ДИ}
\CommentTok{\# {-}{-}{-}{-}{-}{-}{-}{-}{-}{-}{-}{-}{-}{-}{-}{-}{-}{-}{-}{-}{-}{-}{-}{-}{-}{-}{-}{-}{-}{-}{-}{-}{-}{-}{-}{-}{-}{-}{-}{-}{-}{-}{-}{-}{-}{-}{-}{-}{-}{-}{-}{-}{-}{-}{-}{-}{-}{-}{-}{-}{-}{-}{-}{-}{-}{-}{-}{-}{-}{-}{-}{-}{-}{-}{-}{-}{-}{-}}
\CommentTok{\# Пояснения к занятию (для начинающих):}
\CommentTok{\# {-} Мы работаем с временным рядом пополнения запаса R3haddock и набором факторов}
\CommentTok{\#   среды/биомассы. Цель — построить понятные и проверяемые модели прогноза.}
\CommentTok{\# {-} Сначала отберём информативные предикторы (Boruta и LASSO), затем сравним}
\CommentTok{\#   разные модели машинного обучения на кросс{-}валидации (CV), после чего выберем}
\CommentTok{\#   лучшую схему по time{-}slice CV (учитывая хронологию), и сделаем прогноз.}
\CommentTok{\# ==============================================================================}


\CommentTok{\# ==============================================================================}
\CommentTok{\# 1) ВЫБОР ПРЕДИКТОРОВ}
\CommentTok{\# {-}{-}{-}{-}{-}{-}{-}{-}{-}{-}{-}{-}{-}{-}{-}{-}{-}{-}{-}{-}{-}{-}{-}{-}{-}{-}{-}{-}{-}{-}{-}{-}{-}{-}{-}{-}{-}{-}{-}{-}{-}{-}{-}{-}{-}{-}{-}{-}{-}{-}{-}{-}{-}{-}{-}{-}{-}{-}{-}{-}{-}{-}{-}{-}{-}{-}{-}{-}{-}{-}{-}{-}{-}{-}{-}{-}{-}{-}}
\CommentTok{\# Цель блока: привести данные к числовому виду, обработать пропуски, сократить}
\CommentTok{\# мультиколлинеарность (сильные корреляции), а затем автоматически выделить}
\CommentTok{\# кандидатов{-}предикторов двумя методами (Boruta, LASSO). В конце сформируем}
\CommentTok{\# финальный пул признаков и проверим их значимость в простой LM.}
\CommentTok{\# ==============================================================================}

\CommentTok{\# Установка и подключение необходимых библиотек}
\CommentTok{\# Для автоматического отбора предикторов нам понадобятся дополнительные пакеты}
\ControlFlowTok{if}\NormalTok{ (}\SpecialCharTok{!}\FunctionTok{require}\NormalTok{(}\StringTok{"pacman"}\NormalTok{)) }\FunctionTok{install.packages}\NormalTok{(}\StringTok{"pacman"}\NormalTok{)}
\NormalTok{pacman}\SpecialCharTok{::}\FunctionTok{p\_load}\NormalTok{(}
\NormalTok{  readxl, tidyverse, caret, corrplot, mgcv, randomForest, xgboost,}
\NormalTok{  Boruta,GGally, FactoMineR, glmnet, recipes, rsample  }\CommentTok{\# Новые библиотеки для автоматического отбора}
\NormalTok{)}

\CommentTok{\# Очистка среды и установка рабочей директории}
\CommentTok{\# Совет: rm(list=ls()) очищает все объекты в памяти R; setwd задаёт папку,}
\CommentTok{\# где искать/сохранять файлы. Убедитесь, что путь корректен на вашей машине.}
\FunctionTok{rm}\NormalTok{(}\AttributeTok{list =} \FunctionTok{ls}\NormalTok{())}
\FunctionTok{setwd}\NormalTok{(}\StringTok{"C:/RECRUITMENT/"}\NormalTok{)}

\CommentTok{\# Пакеты для расширенного отбора предикторов}
\CommentTok{\# Boruta — обёртка над Random Forest для отбора признаков;}
\CommentTok{\# glmnet — регуляризация (LASSO/ElasticNet) для отбора/усиления обобщающей способности;}
\CommentTok{\# FactoMineR — PCA и другие многомерные методы (используем как утилиту).}
\FunctionTok{library}\NormalTok{(Boruta)   }\CommentTok{\# Алгоритм обертки для отбора признаков}
\FunctionTok{library}\NormalTok{(glmnet)   }\CommentTok{\# LASSO{-}регрессия}
\FunctionTok{library}\NormalTok{(FactoMineR) }\CommentTok{\# PCA анализ}


\CommentTok{\# Загрузка и первичная обработка данных}
\CommentTok{\# Шаги: фильтруем годы, приводим типы к числовому, заменяем строковые "NA" на NA.}
\NormalTok{DATA }\OtherTok{\textless{}{-}}\NormalTok{ readxl}\SpecialCharTok{::}\FunctionTok{read\_excel}\NormalTok{(}\StringTok{"RECRUITMENT.xlsx"}\NormalTok{, }\AttributeTok{sheet =} \StringTok{"RECRUITMENT"}\NormalTok{) }\SpecialCharTok{\%\textgreater{}\%}
  \FunctionTok{filter}\NormalTok{(YEAR }\SpecialCharTok{\textgreater{}} \DecValTok{1989} \SpecialCharTok{\&}\NormalTok{ YEAR }\SpecialCharTok{\textless{}} \DecValTok{2022}\NormalTok{) }\SpecialCharTok{\%\textgreater{}\%}
  \CommentTok{\# Преобразуем необходимые столбцы в числовой формат}
  \FunctionTok{mutate}\NormalTok{(}
    \FunctionTok{across}\NormalTok{(}\FunctionTok{starts\_with}\NormalTok{(}\StringTok{"T"}\NormalTok{), as.numeric),}
    \FunctionTok{across}\NormalTok{(}\FunctionTok{starts\_with}\NormalTok{(}\StringTok{"I"}\NormalTok{), as.numeric),}
    \FunctionTok{across}\NormalTok{(}\FunctionTok{starts\_with}\NormalTok{(}\StringTok{"O"}\NormalTok{), as.numeric),}
\NormalTok{  ) }\SpecialCharTok{\%\textgreater{}\%}
  \CommentTok{\# Обработка пропущенных значений (заменяем строку "NA" на NA)}
  \FunctionTok{mutate}\NormalTok{(}\FunctionTok{across}\NormalTok{(}\FunctionTok{where}\NormalTok{(is.character), }\SpecialCharTok{\textasciitilde{}}\FunctionTok{na\_if}\NormalTok{(., }\StringTok{"NA"}\NormalTok{)))}

\CommentTok{\# 1. Подготовка данных {-}{-}{-}{-}{-}{-}{-}{-}{-}{-}{-}{-}{-}{-}{-}{-}{-}{-}{-}{-}{-}{-}{-}{-}{-}{-}{-}{-}{-}{-}{-}{-}{-}{-}{-}{-}{-}{-}{-}{-}{-}{-}{-}{-}{-}{-}{-}{-}{-}{-}{-}{-}{-}{-}{-}}
\CommentTok{\# Выделим все возможные предикторы, включая географию и индексы трески}
\CommentTok{\# Примечание: оставляем только числовые переменные, т.к. большинство моделей}
\CommentTok{\# требует числовой вход без категориальных уровней.}
\NormalTok{predictors }\OtherTok{\textless{}{-}}\NormalTok{ DATA }\SpecialCharTok{\%\textgreater{}\%} 
  \FunctionTok{select}\NormalTok{(}\SpecialCharTok{{-}}\NormalTok{YEAR, }\SpecialCharTok{{-}}\NormalTok{R3haddock) }\SpecialCharTok{\%\textgreater{}\%} 
  \FunctionTok{select\_if}\NormalTok{(is.numeric) }\CommentTok{\# Только числовые переменные}

\CommentTok{\# Целевая переменная}
\NormalTok{response }\OtherTok{\textless{}{-}}\NormalTok{ DATA}\SpecialCharTok{$}\NormalTok{R3haddock}

\CommentTok{\# В статистическом анализе мы различаем:}
\CommentTok{\# {-} Отклик (response/target variable) {-} то, что мы пытаемся предсказать (в нашем случае R3haddock)}
\CommentTok{\# {-} Предикторы (predictors/features) {-} переменные, которые могут объяснять изменения отклика}
\CommentTok{\# Для корректного анализа важно, чтобы предикторы были числовыми или преобразованы в числовой формат.}

\CommentTok{\# 2. Обработка пропусков {-}{-}{-}{-}{-}{-}{-}{-}{-}{-}{-}{-}{-}{-}{-}{-}{-}{-}{-}{-}{-}{-}{-}{-}{-}{-}{-}{-}{-}{-}{-}{-}{-}{-}{-}{-}{-}{-}{-}{-}{-}{-}{-}{-}{-}{-}{-}{-}{-}{-}{-}{-}{-}}
\CommentTok{\# Заполнение медианными значениями — простой и устойчивый способ справиться с NA.}
\CommentTok{\# Альтернативы: множественная иммутация (mice), KNN{-}impute и др.}
\NormalTok{predictors\_filled }\OtherTok{\textless{}{-}}\NormalTok{ predictors }\SpecialCharTok{\%\textgreater{}\%}
  \FunctionTok{mutate}\NormalTok{(}\FunctionTok{across}\NormalTok{(}\FunctionTok{everything}\NormalTok{(), }\SpecialCharTok{\textasciitilde{}}\FunctionTok{ifelse}\NormalTok{(}\FunctionTok{is.na}\NormalTok{(.), }\FunctionTok{median}\NormalTok{(., }\AttributeTok{na.rm =} \ConstantTok{TRUE}\NormalTok{), .)))}

\CommentTok{\# Заполнение медианой {-} простой и устойчивый метод обработки пропусков для числовых переменных.}
\CommentTok{\# Медиана предпочтительнее среднего, так как менее чувствительна к выбросам.}

\CommentTok{\# 3. Предварительный анализ корреляций {-}{-}{-}{-}{-}{-}{-}{-}{-}{-}{-}{-}{-}{-}{-}{-}{-}{-}{-}{-}{-}{-}{-}{-}{-}{-}{-}{-}{-}{-}{-}{-}{-}{-}{-}{-}{-}{-}{-}}
\CommentTok{\# Зачем: высокие корреляции затрудняют интерпретацию и могут вредить ряду моделей.}
\NormalTok{cor\_matrix }\OtherTok{\textless{}{-}} \FunctionTok{cor}\NormalTok{(predictors\_filled, }\AttributeTok{use =} \StringTok{"complete.obs"}\NormalTok{)}
\FunctionTok{corrplot}\NormalTok{(cor\_matrix, }\AttributeTok{method =} \StringTok{"circle"}\NormalTok{, }\AttributeTok{type =} \StringTok{"upper"}\NormalTok{, }\AttributeTok{tl.cex =} \FloatTok{0.7}\NormalTok{)}
\end{Highlighting}
\end{Shaded}

\pandocbounded{\includegraphics[keepaspectratio]{chapter7_files/figure-pdf/unnamed-chunk-5-1.pdf}}

\begin{Shaded}
\begin{Highlighting}[]
\CommentTok{\# Удаляем высокоскоррелированные предикторы (r \textgreater{} 0.8)}
\CommentTok{\# Это механическое сокращение мультиколлинеарности до этапа отбора.}
\NormalTok{high\_cor }\OtherTok{\textless{}{-}} \FunctionTok{findCorrelation}\NormalTok{(cor\_matrix, }\AttributeTok{cutoff =} \FloatTok{0.8}\NormalTok{)}
\NormalTok{predictors\_filtered }\OtherTok{\textless{}{-}}\NormalTok{ predictors\_filled[, }\SpecialCharTok{{-}}\NormalTok{high\_cor]}

\CommentTok{\# Высокая корреляция между предикторами (мультиколлинеарность) может привести к нестабильности моделей.}
\CommentTok{\# Например, если два предиктора почти идентичны, модель может неустойчиво распределять их влияние на отклик.}
\CommentTok{\# Удаление сильно коррелированных переменных (r \textgreater{} 0.8) помогает улучшить интерпретируемость и стабильность моделей.}


\CommentTok{\# 4. Автоматизированный отбор Boruta (обертка Random Forest) {-}{-}{-}{-}{-}{-}{-}{-}{-}{-}{-}{-}{-}{-}{-}{-}{-}}
\CommentTok{\# Идея: определить признаки, которые важнее, чем случайный шум (shadow features).}
\end{Highlighting}
\end{Shaded}

\begin{Shaded}
\begin{Highlighting}[]
\CommentTok{\# Визуализация результатов}
\FunctionTok{plot}\NormalTok{(boruta\_output, }\AttributeTok{cex.axis =} \FloatTok{0.7}\NormalTok{, }\AttributeTok{las =} \DecValTok{2}\NormalTok{)}
\end{Highlighting}
\end{Shaded}

\pandocbounded{\includegraphics[keepaspectratio]{chapter7_files/figure-pdf/unnamed-chunk-6-1.pdf}}

\begin{Shaded}
\begin{Highlighting}[]
\NormalTok{boruta\_stats }\OtherTok{\textless{}{-}} \FunctionTok{attStats}\NormalTok{(boruta\_output)}
\NormalTok{selected\_vars }\OtherTok{\textless{}{-}} \FunctionTok{getSelectedAttributes}\NormalTok{(boruta\_output, }\AttributeTok{withTentative =} \ConstantTok{TRUE}\NormalTok{)}

\CommentTok{\# Boruta {-} это алгоритм отбора признаков, основанный на методе случайного леса.}
\CommentTok{\# Он сравнивает важность реальных переменных с "теневыми" переменными (случайными копиями),}
\CommentTok{\# чтобы определить, действительно ли переменная информативна. }
\CommentTok{\# Результаты Boruta показывают: }
\CommentTok{\#   {-} Confirmed (зеленые) {-} значимые предикторы}
\CommentTok{\#   {-} Tentative (желтые) {-} предикторы, близкие к порогу значимости}
\CommentTok{\#   {-} Rejected (красные) {-} незначимые предикторы}


\CommentTok{\# 5. LASSO с более строгим критерием {-}{-}{-}{-}{-}{-}{-}{-}{-}{-}{-}{-}{-}{-}{-}{-}{-}{-}{-}{-}{-}{-}{-}{-}{-}{-}{-}{-}{-}{-}{-}{-}{-}{-}{-}{-}{-}{-}{-}{-}{-}{-}}
\CommentTok{\# Идея: L1{-}регуляризация зануляет коэффициенты «слабых» предикторов.}
\CommentTok{\# Выбор lambda.1se вместо lambda.min — более консервативный (простая модель).}
\NormalTok{x }\OtherTok{\textless{}{-}} \FunctionTok{as.matrix}\NormalTok{(predictors\_filtered)}
\NormalTok{y }\OtherTok{\textless{}{-}}\NormalTok{ response}

\CommentTok{\# LASSO (Least Absolute Shrinkage and Selection Operator) {-} метод регрессии с L1{-}регуляризацией,}
\CommentTok{\# который одновременно выполняет отбор признаков и оценку коэффициентов. }
\CommentTok{\# Параметр lambda контролирует силу регуляризации:}
\CommentTok{\#   {-} lambda.min дает наименьшую ошибку, но может включать шумовые переменные}
\CommentTok{\#   {-} lambda.1se (на 1 стандартную ошибку больше) дает более простую модель с меньшим риском переобучения}
\CommentTok{\# Для прогнозирования мы предпочитаем более строгий критерий (lambda.1se), чтобы модель была устойчивее. }

\CommentTok{\# Кросс{-}валидация}
\NormalTok{cv\_fit }\OtherTok{\textless{}{-}} \FunctionTok{cv.glmnet}\NormalTok{(x, y, }\AttributeTok{alpha =} \DecValTok{1}\NormalTok{, }\AttributeTok{nfolds =} \DecValTok{10}\NormalTok{)}
\FunctionTok{plot}\NormalTok{(cv\_fit)}
\end{Highlighting}
\end{Shaded}

\pandocbounded{\includegraphics[keepaspectratio]{chapter7_files/figure-pdf/unnamed-chunk-6-2.pdf}}

\begin{Shaded}
\begin{Highlighting}[]
\CommentTok{\# ИСПОЛЬЗУЕМ lambda.1se вместо lambda.min — СТРОЖЕ!}
\NormalTok{lasso\_coef }\OtherTok{\textless{}{-}} \FunctionTok{coef}\NormalTok{(cv\_fit, }\AttributeTok{s =} \StringTok{"lambda.1se"}\NormalTok{)  }\CommentTok{\# \textless{}{-}{-} Ключевое изменение!}
\NormalTok{lasso\_vars }\OtherTok{\textless{}{-}} \FunctionTok{rownames}\NormalTok{(lasso\_coef)[lasso\_coef[,}\DecValTok{1}\NormalTok{] }\SpecialCharTok{!=} \DecValTok{0}\NormalTok{][}\SpecialCharTok{{-}}\DecValTok{1}\NormalTok{]  }\CommentTok{\# исключаем (Intercept)}


\CommentTok{\# 6. Сравнение отобранных предикторов {-}{-}{-}{-}{-}{-}{-}{-}{-}{-}{-}{-}{-}{-}{-}{-}{-}{-}{-}{-}{-}{-}{-}{-}{-}{-}{-}{-}{-}{-}{-}{-}{-}{-}{-}{-}{-}{-}{-}{-}}
\CommentTok{\# Полезно видеть, какие признаки отмечают оба метода (устойчивые кандидаты).}
\FunctionTok{cat}\NormalTok{(}\StringTok{"Boruta selected:"}\NormalTok{, }\FunctionTok{length}\NormalTok{(selected\_vars), }\StringTok{"variables}\SpecialCharTok{\textbackslash{}n}\StringTok{"}\NormalTok{)}
\end{Highlighting}
\end{Shaded}

\begin{verbatim}
Boruta selected: 3 variables
\end{verbatim}

\begin{Shaded}
\begin{Highlighting}[]
\FunctionTok{print}\NormalTok{(selected\_vars)}
\end{Highlighting}
\end{Shaded}

\begin{verbatim}
[1] "codTSB" "T12"    "I5"    
\end{verbatim}

\begin{Shaded}
\begin{Highlighting}[]
\FunctionTok{cat}\NormalTok{(}\StringTok{"}\SpecialCharTok{\textbackslash{}n}\StringTok{LASSO selected:"}\NormalTok{, }\FunctionTok{length}\NormalTok{(lasso\_vars), }\StringTok{"variables}\SpecialCharTok{\textbackslash{}n}\StringTok{"}\NormalTok{)}
\end{Highlighting}
\end{Shaded}

\begin{verbatim}

LASSO selected: 5 variables
\end{verbatim}

\begin{Shaded}
\begin{Highlighting}[]
\FunctionTok{print}\NormalTok{(lasso\_vars)}
\end{Highlighting}
\end{Shaded}

\begin{verbatim}
[1] "codTSB" "T12"    "NAO3"   "NAO4"   "NAO5"  
\end{verbatim}

\begin{Shaded}
\begin{Highlighting}[]
\CommentTok{\# 7. Финальный набор предикторов (объединение результатов) {-}{-}{-}{-}{-}{-}{-}{-}{-}{-}{-}{-}{-}{-}{-}{-}{-}{-}{-}}
\CommentTok{\# Логика: объединяем списки, добавляем биологически важные переменные вручную.}
\NormalTok{final\_vars }\OtherTok{\textless{}{-}} \FunctionTok{union}\NormalTok{(selected\_vars, lasso\_vars) }

\CommentTok{\# Добавляем обязательные переменные по биологической логике}
\NormalTok{mandatory }\OtherTok{\textless{}{-}} \FunctionTok{c}\NormalTok{(}\StringTok{"haddock68"}\NormalTok{)}
\NormalTok{final\_vars }\OtherTok{\textless{}{-}} \FunctionTok{union}\NormalTok{(final\_vars, mandatory) }\SpecialCharTok{\%\textgreater{}\%} \FunctionTok{unique}\NormalTok{()}

\CommentTok{\# Мы объединяем результаты двух методов отбора признаков для большей надежности.}
\CommentTok{\# Также добавляем переменную haddock68 (нерестовый запас), так как биологически }
\CommentTok{\# логично, что пополнение запаса напрямую зависит от численности производителей. }
\CommentTok{\# Это пример интеграции экспертных знаний в статистический анализ {-} важный принцип }
\CommentTok{\# при работе с данными в биологических науках.}

\CommentTok{\# 8. Проверка значимости {-}{-}{-}{-}{-}{-}{-}{-}{-}{-}{-}{-}{-}{-}{-}{-}{-}{-}{-}{-}{-}{-}{-}{-}{-}{-}{-}{-}{-}{-}{-}{-}{-}{-}{-}{-}{-}{-}{-}{-}{-}{-}{-}{-}{-}{-}{-}{-}{-}{-}{-}{-}{-}}
\CommentTok{\# Быстрая оценка значимости с LM: не как окончательный вывод, а как sanity{-}check.}
\NormalTok{final\_model }\OtherTok{\textless{}{-}} \FunctionTok{lm}\NormalTok{(response }\SpecialCharTok{\textasciitilde{}} \FunctionTok{as.matrix}\NormalTok{(predictors\_filled[, final\_vars]))}
\FunctionTok{summary}\NormalTok{(final\_model)}
\end{Highlighting}
\end{Shaded}

\begin{verbatim}

Call:
lm(formula = response ~ as.matrix(predictors_filled[, final_vars]))

Residuals:
    Min      1Q  Median      3Q     Max 
-270986  -82376   -1037   98086  276129 

Coefficients:
                                                      Estimate Std. Error
(Intercept)                                         -1.082e+06  3.943e+05
as.matrix(predictors_filled[, final_vars])codTSB    -2.346e-01  5.536e-02
as.matrix(predictors_filled[, final_vars])T12        3.864e+05  7.198e+04
as.matrix(predictors_filled[, final_vars])I5        -1.825e+02  2.572e+03
as.matrix(predictors_filled[, final_vars])NAO3      -5.801e+04  3.129e+04
as.matrix(predictors_filled[, final_vars])NAO4       8.345e+04  3.035e+04
as.matrix(predictors_filled[, final_vars])NAO5      -7.278e+04  2.488e+04
as.matrix(predictors_filled[, final_vars])haddock68  1.232e-01  4.515e-01
                                                    t value Pr(>|t|)    
(Intercept)                                          -2.744 0.011305 *  
as.matrix(predictors_filled[, final_vars])codTSB     -4.238 0.000288 ***
as.matrix(predictors_filled[, final_vars])T12         5.368 1.64e-05 ***
as.matrix(predictors_filled[, final_vars])I5         -0.071 0.944028    
as.matrix(predictors_filled[, final_vars])NAO3       -1.854 0.076118 .  
as.matrix(predictors_filled[, final_vars])NAO4        2.750 0.011146 *  
as.matrix(predictors_filled[, final_vars])NAO5       -2.925 0.007412 ** 
as.matrix(predictors_filled[, final_vars])haddock68   0.273 0.787227    
---
Signif. codes:  0 '***' 0.001 '**' 0.01 '*' 0.05 '.' 0.1 ' ' 1

Residual standard error: 158600 on 24 degrees of freedom
Multiple R-squared:  0.7173,    Adjusted R-squared:  0.6348 
F-statistic: 8.698 on 7 and 24 DF,  p-value: 2.58e-05
\end{verbatim}

\begin{Shaded}
\begin{Highlighting}[]
\CommentTok{\# 9. Формирование финального датасета {-}{-}{-}{-}{-}{-}{-}{-}{-}{-}{-}{-}{-}{-}{-}{-}{-}{-}{-}{-}{-}{-}{-}{-}{-}{-}{-}{-}{-}{-}{-}{-}{-}{-}{-}{-}{-}{-}{-}{-}}
\CommentTok{\# Собираем набор с откликом и выбранными предикторами; удалим строки с NA.}
\NormalTok{model\_data }\OtherTok{\textless{}{-}}\NormalTok{ DATA }\SpecialCharTok{\%\textgreater{}\%}
  \FunctionTok{select}\NormalTok{(R3haddock, }\FunctionTok{all\_of}\NormalTok{(final\_vars)) }\SpecialCharTok{\%\textgreater{}\%}
  \FunctionTok{drop\_na}\NormalTok{()}

\CommentTok{\# Просмотр структуры финальных данных}
\FunctionTok{glimpse}\NormalTok{(model\_data)}
\end{Highlighting}
\end{Shaded}

\begin{verbatim}
Rows: 32
Columns: 8
$ R3haddock <dbl> 812363, 389416, 99474, 98946, 118812, 63028, 147657, 83270, ~
$ codTSB    <dbl> 913000, 1347064, 1687381, 2197863, 2112773, 1849957, 1697388~
$ T12       <dbl> 4.72, 4.66, 4.24, 3.90, 3.96, 4.27, 4.16, 4.07, 4.23, 5.08, ~
$ I5        <dbl> 43, 55, 26, 49, 56, 28, 52, 51, 69, 68, 41, 48, 50, 63, 40, ~
$ NAO3      <dbl> 1.46, -0.20, 0.87, 0.67, 1.26, 1.25, -0.24, 1.46, 0.87, 0.23~
$ NAO4      <dbl> 2.00, 0.29, 1.86, 0.97, 1.14, -0.85, -0.17, -1.02, -0.68, -0~
$ NAO5      <dbl> -1.53, 0.08, 2.63, -0.78, -0.57, -1.49, -1.06, -0.28, -1.32,~
$ haddock68 <dbl> 74586, 79205, 53195, 36337, 49122, 81514, 172177, 160886, 96~
\end{verbatim}

\begin{Shaded}
\begin{Highlighting}[]
\CommentTok{\# Визуализация важности переменных}
\CommentTok{\# Внимание: важности от RF — относительные; сопоставляйте с предметной логикой.}
\NormalTok{var\_importance }\OtherTok{\textless{}{-}} \FunctionTok{randomForest}\NormalTok{(R3haddock }\SpecialCharTok{\textasciitilde{}}\NormalTok{ ., }\AttributeTok{data =}\NormalTok{ model\_data, }\AttributeTok{importance =} \ConstantTok{TRUE}\NormalTok{)}
\FunctionTok{varImpPlot}\NormalTok{(var\_importance, }\AttributeTok{main =} \StringTok{"Важность предикторов"}\NormalTok{)}
\end{Highlighting}
\end{Shaded}

\begin{verbatim}
Warning in mtext(outer = TRUE, side = 3, text = main, cex = 1.2): неизвестна
ширина символа 0xc2 в кодировке CP1251
\end{verbatim}

\begin{verbatim}
Warning in mtext(outer = TRUE, side = 3, text = main, cex = 1.2): неизвестна
ширина символа 0xe0 в кодировке CP1251
\end{verbatim}

\begin{verbatim}
Warning in mtext(outer = TRUE, side = 3, text = main, cex = 1.2): неизвестна
ширина символа 0xe6 в кодировке CP1251
\end{verbatim}

\begin{verbatim}
Warning in mtext(outer = TRUE, side = 3, text = main, cex = 1.2): неизвестна
ширина символа 0xed в кодировке CP1251
\end{verbatim}

\begin{verbatim}
Warning in mtext(outer = TRUE, side = 3, text = main, cex = 1.2): неизвестна
ширина символа 0xee в кодировке CP1251
\end{verbatim}

\begin{verbatim}
Warning in mtext(outer = TRUE, side = 3, text = main, cex = 1.2): неизвестна
ширина символа 0xf1 в кодировке CP1251
\end{verbatim}

\begin{verbatim}
Warning in mtext(outer = TRUE, side = 3, text = main, cex = 1.2): неизвестна
ширина символа 0xf2 в кодировке CP1251
\end{verbatim}

\begin{verbatim}
Warning in mtext(outer = TRUE, side = 3, text = main, cex = 1.2): неизвестна
ширина символа 0xfc в кодировке CP1251
\end{verbatim}

\begin{verbatim}
Warning in mtext(outer = TRUE, side = 3, text = main, cex = 1.2): неизвестна
ширина символа 0xef в кодировке CP1251
\end{verbatim}

\begin{verbatim}
Warning in mtext(outer = TRUE, side = 3, text = main, cex = 1.2): неизвестна
ширина символа 0xf0 в кодировке CP1251
\end{verbatim}

\begin{verbatim}
Warning in mtext(outer = TRUE, side = 3, text = main, cex = 1.2): неизвестна
ширина символа 0xe5 в кодировке CP1251
\end{verbatim}

\begin{verbatim}
Warning in mtext(outer = TRUE, side = 3, text = main, cex = 1.2): неизвестна
ширина символа 0xe4 в кодировке CP1251
\end{verbatim}

\begin{verbatim}
Warning in mtext(outer = TRUE, side = 3, text = main, cex = 1.2): неизвестна
ширина символа 0xe8 в кодировке CP1251
\end{verbatim}

\begin{verbatim}
Warning in mtext(outer = TRUE, side = 3, text = main, cex = 1.2): неизвестна
ширина символа 0xea в кодировке CP1251
\end{verbatim}

\begin{verbatim}
Warning in mtext(outer = TRUE, side = 3, text = main, cex = 1.2): неизвестна
ширина символа 0xf2 в кодировке CP1251
\end{verbatim}

\begin{verbatim}
Warning in mtext(outer = TRUE, side = 3, text = main, cex = 1.2): неизвестна
ширина символа 0xee в кодировке CP1251
\end{verbatim}

\begin{verbatim}
Warning in mtext(outer = TRUE, side = 3, text = main, cex = 1.2): неизвестна
ширина символа 0xf0 в кодировке CP1251
\end{verbatim}

\begin{verbatim}
Warning in mtext(outer = TRUE, side = 3, text = main, cex = 1.2): неизвестна
ширина символа 0xee в кодировке CP1251
\end{verbatim}

\begin{verbatim}
Warning in mtext(outer = TRUE, side = 3, text = main, cex = 1.2): неизвестна
ширина символа 0xe2 в кодировке CP1251
\end{verbatim}

\pandocbounded{\includegraphics[keepaspectratio]{chapter7_files/figure-pdf/unnamed-chunk-6-3.pdf}}

\begin{Shaded}
\begin{Highlighting}[]
\CommentTok{\# Перед окончательным выбором модели мы проверяем значимость предикторов с помощью линейной регрессии.}
\CommentTok{\# Функция summary() показывает p{-}значения коэффициентов {-} если p \textless{} 0.05, переменная считается статистически значимой. }
\CommentTok{\# Визуализация важности переменных с помощью случайного леса дает дополнительную перспективу,}
\CommentTok{\# показывая, какие переменные наиболее информативны для предсказания без предположений о линейности.}

\CommentTok{\# ==============================================================================}
\CommentTok{\#  ПОДГОТОВКА ДАННЫХ}
\CommentTok{\# Создаём NAOspring, фиксируем финальный набор признаков, сохраняем CSV.}
\CommentTok{\# {-}{-}{-}{-}{-}{-}{-}{-}{-}{-}{-}{-}{-}{-}{-}{-}{-}{-}{-}{-}{-}{-}{-}{-}{-}{-}{-}{-}{-}{-}{-}{-}{-}{-}{-}{-}{-}{-}{-}{-}{-}{-}{-}{-}{-}{-}{-}{-}{-}{-}{-}{-}{-}{-}{-}{-}{-}{-}{-}{-}{-}{-}{-}{-}{-}{-}{-}{-}{-}{-}{-}{-}{-}{-}{-}{-}{-}{-}}
\CommentTok{\# Цель блока: стандартизировать набор признаков для дальнейшего сравнения}
\CommentTok{\# моделей и обеспечить воспроизводимость (фиксированный CSV с нужными полями).}
\CommentTok{\# ==============================================================================}

\CommentTok{\# 1.1 Пакеты и окружение}
\CommentTok{\# Примечание: блок повторяет базовую инициализацию для автономного запуска.}
\ControlFlowTok{if}\NormalTok{ (}\SpecialCharTok{!}\FunctionTok{require}\NormalTok{(}\StringTok{"pacman"}\NormalTok{)) }\FunctionTok{install.packages}\NormalTok{(}\StringTok{"pacman"}\NormalTok{)}
\NormalTok{pacman}\SpecialCharTok{::}\FunctionTok{p\_load}\NormalTok{(readxl, tidyverse, caret, corrplot)}

\FunctionTok{rm}\NormalTok{(}\AttributeTok{list =} \FunctionTok{ls}\NormalTok{())}
\FunctionTok{set.seed}\NormalTok{(}\DecValTok{123}\NormalTok{)}
\FunctionTok{setwd}\NormalTok{(}\StringTok{"C:/RECRUITMENT/"}\NormalTok{)}

\CommentTok{\# 1.2 Загрузка исходных данных и приведение типов}
\NormalTok{DATA }\OtherTok{\textless{}{-}}\NormalTok{ readxl}\SpecialCharTok{::}\FunctionTok{read\_excel}\NormalTok{(}\StringTok{"RECRUITMENT.xlsx"}\NormalTok{, }\AttributeTok{sheet =} \StringTok{"RECRUITMENT"}\NormalTok{) }\SpecialCharTok{\%\textgreater{}\%}
  \FunctionTok{filter}\NormalTok{(YEAR }\SpecialCharTok{\textgreater{}} \DecValTok{1989} \SpecialCharTok{\&}\NormalTok{ YEAR }\SpecialCharTok{\textless{}} \DecValTok{2022}\NormalTok{) }\SpecialCharTok{\%\textgreater{}\%}
  \FunctionTok{mutate}\NormalTok{(}
    \FunctionTok{across}\NormalTok{(}\FunctionTok{starts\_with}\NormalTok{(}\StringTok{"T"}\NormalTok{), as.numeric),}
    \FunctionTok{across}\NormalTok{(}\FunctionTok{starts\_with}\NormalTok{(}\StringTok{"I"}\NormalTok{), as.numeric),}
    \FunctionTok{across}\NormalTok{(}\FunctionTok{starts\_with}\NormalTok{(}\StringTok{"O"}\NormalTok{), as.numeric),}
    \FunctionTok{across}\NormalTok{(}\FunctionTok{where}\NormalTok{(is.character), }\SpecialCharTok{\textasciitilde{}}\FunctionTok{na\_if}\NormalTok{(., }\StringTok{"NA"}\NormalTok{))}
\NormalTok{  )}

\CommentTok{\# 1.3 Создаём NAOspring (если есть NAO3, NAO4, NAO5)}
\CommentTok{\# Идея: агрегируем весенний индекс NAO как среднее за месяцы 3–5.}
\ControlFlowTok{if}\NormalTok{ (}\FunctionTok{all}\NormalTok{(}\FunctionTok{c}\NormalTok{(}\StringTok{"NAO3"}\NormalTok{,}\StringTok{"NAO4"}\NormalTok{,}\StringTok{"NAO5"}\NormalTok{) }\SpecialCharTok{\%in\%} \FunctionTok{names}\NormalTok{(DATA))) \{}
\NormalTok{  DATA }\OtherTok{\textless{}{-}}\NormalTok{ DATA }\SpecialCharTok{\%\textgreater{}\%}
    \FunctionTok{mutate}\NormalTok{(}\AttributeTok{NAOspring =} \FunctionTok{rowMeans}\NormalTok{(}\FunctionTok{pick}\NormalTok{(NAO3, NAO4, NAO5), }\AttributeTok{na.rm =} \ConstantTok{TRUE}\NormalTok{)) }\SpecialCharTok{\%\textgreater{}\%}
    \FunctionTok{select}\NormalTok{(}\SpecialCharTok{{-}}\NormalTok{NAO3, }\SpecialCharTok{{-}}\NormalTok{NAO4, }\SpecialCharTok{{-}}\NormalTok{NAO5)}
\NormalTok{\}}

\CommentTok{\# NAO (North Atlantic Oscillation) {-} важный климатический индекс, влияющий описывающий изменения атмосферного давления}
\CommentTok{\# над Северной Атлантикой. В частности, он отражает разницу в атмосферном давлении между Исландской депрессией и}
\CommentTok{\# Азорским максимумом. NAO влияет на силу и направление западных ветров, а также на траектории штормов в Северной Атлантике. }
\CommentTok{\# Мы создаем NAOspring как среднее значение за весенние месяцы (марта, апреля, мая),}
\CommentTok{\# так как именно в этот период происходят ключевые процессы, влияющие на нерест трески. }
\CommentTok{\# Создание составных переменных на основе экспертных знаний часто улучшает качество моделей.}

\CommentTok{\# 1.4 Финальный учебный набор предикторов (фиксируем)}
\CommentTok{\# Важно: проверяем присутствие нужных колонок и формируем компактный датасет.}
\NormalTok{needed }\OtherTok{\textless{}{-}} \FunctionTok{c}\NormalTok{(}\StringTok{"codTSB"}\NormalTok{, }\StringTok{"T12"}\NormalTok{, }\StringTok{"I5"}\NormalTok{, }\StringTok{"NAOspring"}\NormalTok{, }\StringTok{"haddock68"}\NormalTok{)}
\FunctionTok{stopifnot}\NormalTok{(}\FunctionTok{all}\NormalTok{(needed }\SpecialCharTok{\%in\%} \FunctionTok{names}\NormalTok{(DATA)))}

\CommentTok{\# Сохраняем YEAR в CSV (ниже он будет отброшен при обучении, но нужен для графика)}
\NormalTok{model\_data }\OtherTok{\textless{}{-}}\NormalTok{ DATA }\SpecialCharTok{\%\textgreater{}\%}
  \FunctionTok{select}\NormalTok{(YEAR, }\FunctionTok{all\_of}\NormalTok{(needed), R3haddock) }\SpecialCharTok{\%\textgreater{}\%}
  \FunctionTok{drop\_na}\NormalTok{()}

\FunctionTok{write.csv}\NormalTok{(model\_data, }\StringTok{"selected\_predictors\_dataset.csv"}\NormalTok{, }\AttributeTok{row.names =} \ConstantTok{FALSE}\NormalTok{)}
\FunctionTok{glimpse}\NormalTok{(model\_data)}
\end{Highlighting}
\end{Shaded}

\begin{verbatim}
Rows: 32
Columns: 7
$ YEAR      <dbl> 1990, 1991, 1992, 1993, 1994, 1995, 1996, 1997, 1998, 1999, ~
$ codTSB    <dbl> 913000, 1347064, 1687381, 2197863, 2112773, 1849957, 1697388~
$ T12       <dbl> 4.72, 4.66, 4.24, 3.90, 3.96, 4.27, 4.16, 4.07, 4.23, 5.08, ~
$ I5        <dbl> 43, 55, 26, 49, 56, 28, 52, 51, 69, 68, 41, 48, 50, 63, 40, ~
$ NAOspring <dbl> 0.64333333, 0.05666667, 1.78666667, 0.28666667, 0.61000000, ~
$ haddock68 <dbl> 74586, 79205, 53195, 36337, 49122, 81514, 172177, 160886, 96~
$ R3haddock <dbl> 812363, 389416, 99474, 98946, 118812, 63028, 147657, 83270, ~
\end{verbatim}

\begin{Shaded}
\begin{Highlighting}[]
\CommentTok{\# (необязательно) Глянуть попарные связи и корреляции}
\CommentTok{\# ggpairs может быть медленным, оставим по желанию}
 \FunctionTok{ggpairs}\NormalTok{(model\_data, }\AttributeTok{columns =} \DecValTok{2}\SpecialCharTok{:}\DecValTok{7}\NormalTok{,}
         \AttributeTok{lower =} \FunctionTok{list}\NormalTok{(}\AttributeTok{continuous =} \FunctionTok{wrap}\NormalTok{(}\StringTok{"smooth"}\NormalTok{, }\AttributeTok{alpha =} \FloatTok{0.3}\NormalTok{, }\AttributeTok{size =} \FloatTok{0.5}\NormalTok{)),}
         \AttributeTok{upper =} \FunctionTok{list}\NormalTok{(}\AttributeTok{cor =} \FunctionTok{wrap}\NormalTok{(}\StringTok{"cor"}\NormalTok{, }\AttributeTok{size =} \DecValTok{3}\NormalTok{)))}
\end{Highlighting}
\end{Shaded}

\pandocbounded{\includegraphics[keepaspectratio]{chapter7_files/figure-pdf/unnamed-chunk-6-4.pdf}}

\begin{Shaded}
\begin{Highlighting}[]
\CommentTok{\# ==============================================================================}
\CommentTok{\# 2) БАЗОВОЕ СРАВНЕНИЕ МОДЕЛЕЙ (5{-}FOLD CV + HOLDOUT)}
\CommentTok{\# Единые фолды CV, тренировочно{-}тестовое разбиение, сводка метрик.}
\CommentTok{\# {-}{-}{-}{-}{-}{-}{-}{-}{-}{-}{-}{-}{-}{-}{-}{-}{-}{-}{-}{-}{-}{-}{-}{-}{-}{-}{-}{-}{-}{-}{-}{-}{-}{-}{-}{-}{-}{-}{-}{-}{-}{-}{-}{-}{-}{-}{-}{-}{-}{-}{-}{-}{-}{-}{-}{-}{-}{-}{-}{-}{-}{-}{-}{-}{-}{-}{-}{-}{-}{-}{-}{-}{-}{-}{-}{-}{-}{-}}
\CommentTok{\# Идея блока: быстрая «панель» сравнения разных семейств моделей на одинаковых}
\CommentTok{\# условиях (одинаковые фолды CV) и внешний тест (holdout). Это помогает увидеть}
\CommentTok{\# уровни ошибок и выбрать несколько лидеров для более строгой проверки далее.}
\CommentTok{\# ==============================================================================}

\CommentTok{\# 2.1 Пакеты и данные}
\NormalTok{pacman}\SpecialCharTok{::}\FunctionTok{p\_load}\NormalTok{(mgcv, randomForest, xgboost, nnet, earth, kernlab, pls, Cubist, ranger, gbm, lattice)}

\NormalTok{model\_data }\OtherTok{\textless{}{-}} \FunctionTok{read.csv}\NormalTok{(}\StringTok{"selected\_predictors\_dataset.csv"}\NormalTok{, }\AttributeTok{header =} \ConstantTok{TRUE}\NormalTok{, }\AttributeTok{stringsAsFactors =} \ConstantTok{FALSE}\NormalTok{)}
\CommentTok{\# Если YEAR отсутствует (на всякий случай), создадим}
\ControlFlowTok{if}\NormalTok{ (}\SpecialCharTok{!}\StringTok{"YEAR"} \SpecialCharTok{\%in\%} \FunctionTok{names}\NormalTok{(model\_data)) \{}
\NormalTok{  model\_data}\SpecialCharTok{$}\NormalTok{YEAR }\OtherTok{\textless{}{-}} \FunctionTok{seq}\NormalTok{(}\DecValTok{1990}\NormalTok{, }\AttributeTok{by =} \DecValTok{1}\NormalTok{, }\AttributeTok{length.out =} \FunctionTok{nrow}\NormalTok{(model\_data))}
\NormalTok{\}}

\CommentTok{\# Используем только предикторы и отклик (YEAR исключаем)}
\NormalTok{model\_data }\OtherTok{\textless{}{-}}\NormalTok{ model\_data }\SpecialCharTok{\%\textgreater{}\%}
  \FunctionTok{select}\NormalTok{(codTSB, T12, I5, NAOspring, haddock68, R3haddock) }\SpecialCharTok{\%\textgreater{}\%}
  \FunctionTok{na.omit}\NormalTok{()}

\CommentTok{\# 2.2 Holdout и CV{-}контроллер}
\CommentTok{\# Пропорция 80/20 обеспечивает внешний тест; внутри train — 5{-}fold CV для}
\CommentTok{\# корректной настройки моделей и оценки средней ошибки.}
\NormalTok{train\_idx }\OtherTok{\textless{}{-}}\NormalTok{ caret}\SpecialCharTok{::}\FunctionTok{createDataPartition}\NormalTok{(model\_data}\SpecialCharTok{$}\NormalTok{R3haddock, }\AttributeTok{p =} \FloatTok{0.8}\NormalTok{, }\AttributeTok{list =} \ConstantTok{FALSE}\NormalTok{)}
\NormalTok{train }\OtherTok{\textless{}{-}}\NormalTok{ model\_data[train\_idx, ]}
\NormalTok{test  }\OtherTok{\textless{}{-}}\NormalTok{ model\_data[}\SpecialCharTok{{-}}\NormalTok{train\_idx, ]}

\NormalTok{ctrl }\OtherTok{\textless{}{-}}\NormalTok{ caret}\SpecialCharTok{::}\FunctionTok{trainControl}\NormalTok{(}\AttributeTok{method =} \StringTok{"cv"}\NormalTok{, }\AttributeTok{number =} \DecValTok{5}\NormalTok{, }\AttributeTok{savePredictions =} \StringTok{"final"}\NormalTok{)}

\CommentTok{\# Holdout{-}метод: мы делим данные на обучающую (80\%) и тестовую (20\%) выборки.}
\CommentTok{\# Кросс{-}валидация (5{-}fold CV): данные разбиваются на 5 частей, модель обучается на 4 частях и тестируется на 5{-}й, }
\CommentTok{\# и этот процесс повторяется 5 раз. Это дает более надежную оценку качества модели, чем одно разбиение. }


\CommentTok{\# 2.3 Кастомный GAM (mgcv) для caret (bs="tp", REML, select=TRUE)}
\CommentTok{\# GAM даёт гладкие нелинейности по каждому признаку; REML стабилизирует оценку.}
\NormalTok{gam\_spec }\OtherTok{\textless{}{-}} \FunctionTok{list}\NormalTok{(}
  \AttributeTok{type =} \StringTok{"Regression"}\NormalTok{, }\AttributeTok{library =} \StringTok{"mgcv"}\NormalTok{, }\AttributeTok{loop =} \ConstantTok{NULL}\NormalTok{,}
  \AttributeTok{parameters =} \FunctionTok{data.frame}\NormalTok{(}\AttributeTok{parameter =} \StringTok{"none"}\NormalTok{, }\AttributeTok{class =} \StringTok{"character"}\NormalTok{, }\AttributeTok{label =} \StringTok{"none"}\NormalTok{),}
  \AttributeTok{grid =} \ControlFlowTok{function}\NormalTok{(x,y,}\AttributeTok{len=}\ConstantTok{NULL}\NormalTok{,}\AttributeTok{search=}\StringTok{"grid"}\NormalTok{) }\FunctionTok{data.frame}\NormalTok{(}\AttributeTok{none =} \ConstantTok{NA}\NormalTok{),}
  \AttributeTok{fit =} \ControlFlowTok{function}\NormalTok{(x,y,...) \{}
\NormalTok{    df }\OtherTok{\textless{}{-}}\NormalTok{ x; df}\SpecialCharTok{$}\NormalTok{R3haddock }\OtherTok{\textless{}{-}}\NormalTok{ y}
\NormalTok{    mgcv}\SpecialCharTok{::}\FunctionTok{gam}\NormalTok{(}
\NormalTok{      R3haddock }\SpecialCharTok{\textasciitilde{}} \FunctionTok{s}\NormalTok{(codTSB,}\AttributeTok{bs=}\StringTok{"tp"}\NormalTok{) }\SpecialCharTok{+} \FunctionTok{s}\NormalTok{(T12,}\AttributeTok{bs=}\StringTok{"tp"}\NormalTok{) }\SpecialCharTok{+} \FunctionTok{s}\NormalTok{(I5,}\AttributeTok{bs=}\StringTok{"tp"}\NormalTok{) }\SpecialCharTok{+}
                  \FunctionTok{s}\NormalTok{(NAOspring,}\AttributeTok{bs=}\StringTok{"tp"}\NormalTok{) }\SpecialCharTok{+} \FunctionTok{s}\NormalTok{(haddock68,}\AttributeTok{bs=}\StringTok{"tp"}\NormalTok{),}
      \AttributeTok{data=}\NormalTok{df, }\AttributeTok{method=}\StringTok{"REML"}\NormalTok{, }\AttributeTok{select=}\ConstantTok{TRUE}\NormalTok{, ...}
\NormalTok{    )}
\NormalTok{  \},}
  \AttributeTok{predict =} \ControlFlowTok{function}\NormalTok{(modelFit, newdata, }\AttributeTok{submodels =} \ConstantTok{NULL}\NormalTok{) \{}
    \FunctionTok{predict}\NormalTok{(modelFit, }\AttributeTok{newdata =}\NormalTok{ newdata, }\AttributeTok{type =} \StringTok{"response"}\NormalTok{)}
\NormalTok{  \},}
  \AttributeTok{prob =} \ConstantTok{NULL}\NormalTok{, }\AttributeTok{sort =} \ControlFlowTok{function}\NormalTok{(x) x}
\NormalTok{)}

\CommentTok{\# 2.4 Обучение моделей}
\CommentTok{\# Подсказка: разные методы по{-}разному чувствительны к масштабу, числу признаков}
\CommentTok{\# и мультиколлинеарности. Мы применяем одинаковые фолды CV для честного сравнения.}

\CommentTok{\# {-}{-}{-} 1. Линейная регрессия (LM)}
\CommentTok{\# Учебный смысл: базовая линейная модель; ориентир для сравнения.}
\CommentTok{\# ПОЯСНЕНИЕ: LM предполагает линейную зависимость между предикторами и откликом.}
\CommentTok{\# Это простая модель, которая служит "нижней планкой" {-} более сложные модели могут быть лучше LM. }
\NormalTok{lm\_model    }\OtherTok{\textless{}{-}}\NormalTok{ caret}\SpecialCharTok{::}\FunctionTok{train}\NormalTok{(R3haddock }\SpecialCharTok{\textasciitilde{}}\NormalTok{ ., }\AttributeTok{data =}\NormalTok{ train, }\AttributeTok{method =} \StringTok{"lm"}\NormalTok{, }\AttributeTok{trControl =}\NormalTok{ ctrl)}

\CommentTok{\# {-}{-}{-} 2. Обобщённая линейная модель (GLM: Gamma с лог{-}ссылкой)}
\CommentTok{\# Учебный смысл: модель для положительных откликов; допускает нелинейность в шкале log.}
\CommentTok{\# ПОЯСНЕНИЕ: GLM с Gamma{-}распределением подходит для положительных непрерывных данных }
\CommentTok{\# (как размер популяции), где дисперсия зависит от среднего значения.}
\NormalTok{glm\_model   }\OtherTok{\textless{}{-}}\NormalTok{ caret}\SpecialCharTok{::}\FunctionTok{train}\NormalTok{(R3haddock }\SpecialCharTok{\textasciitilde{}}\NormalTok{ ., }\AttributeTok{data =}\NormalTok{ train, }\AttributeTok{method =} \StringTok{"glm"}\NormalTok{,}
                            \AttributeTok{family =} \FunctionTok{Gamma}\NormalTok{(}\AttributeTok{link =} \StringTok{"log"}\NormalTok{), }\AttributeTok{trControl =}\NormalTok{ ctrl)}

\CommentTok{\# {-}{-}{-} 3. Обобщённая аддитивная модель (GAM, mgcv: bs="tp", REML, select=TRUE)}
\CommentTok{\# Учебный смысл: гибкие гладкие нелинейности по каждому предиктору.}
\CommentTok{\# ПОЯСНЕНИЕ: GAM позволяет моделировать нелинейные зависимости с помощью гладких функций (splines),}
\CommentTok{\# сохраняя интерпретируемость отдельных эффектов. Это компромисс между простотой LM и сложностью ML.}
\NormalTok{gam\_model   }\OtherTok{\textless{}{-}}\NormalTok{ caret}\SpecialCharTok{::}\FunctionTok{train}\NormalTok{(}\AttributeTok{x =}\NormalTok{ train[, }\SpecialCharTok{{-}}\FunctionTok{which}\NormalTok{(}\FunctionTok{names}\NormalTok{(train)}\SpecialCharTok{==}\StringTok{"R3haddock"}\NormalTok{)],}
                            \AttributeTok{y =}\NormalTok{ train}\SpecialCharTok{$}\NormalTok{R3haddock, }\AttributeTok{method =}\NormalTok{ gam\_spec, }\AttributeTok{trControl =}\NormalTok{ ctrl)}
\end{Highlighting}
\end{Shaded}

\begin{verbatim}
Warning in nominalTrainWorkflow(x = x, y = y, wts = weights, info = trainInfo,
: There were missing values in resampled performance measures.
\end{verbatim}

\begin{Shaded}
\begin{Highlighting}[]
\CommentTok{\# {-}{-}{-} 4. Random Forest (rf: ntree=1000, mtry=1)}
\CommentTok{\# Учебный смысл: ансамбль деревьев; устойчив к шуму; нелинейности/взаимодействия "из коробки".}
\CommentTok{\# ПОЯСНЕНИЕ: Random Forest строит множество деревьев решений и усредняет их результаты.}
\CommentTok{\# Это мощный метод, который автоматически улавливает нелинейные зависимости и взаимодействия. }
\NormalTok{rf\_model    }\OtherTok{\textless{}{-}}\NormalTok{ caret}\SpecialCharTok{::}\FunctionTok{train}\NormalTok{(R3haddock }\SpecialCharTok{\textasciitilde{}}\NormalTok{ ., }\AttributeTok{data =}\NormalTok{ train, }\AttributeTok{method =} \StringTok{"rf"}\NormalTok{, }\AttributeTok{trControl =}\NormalTok{ ctrl,}
                            \AttributeTok{ntree =} \DecValTok{1000}\NormalTok{, }\AttributeTok{tuneGrid =} \FunctionTok{data.frame}\NormalTok{(}\AttributeTok{mtry =} \DecValTok{1}\NormalTok{), }\AttributeTok{importance =} \ConstantTok{TRUE}\NormalTok{)}

\CommentTok{\# {-}{-}{-} 5. XGBoost (xgbTree) }
\CommentTok{\# Учебный смысл: бустинг деревьев; сильная ML{-}модель, легко переобучается без валидации.}
\CommentTok{\# ПОЯСНЕНИЕ: XGBoost {-} это градиентный бустинг над деревьями решений, который последовательно }
\CommentTok{\# строит деревья, исправляя ошибки предыдущих. Требует тщательной настройки параметров.}
\NormalTok{xgb\_grid    }\OtherTok{\textless{}{-}} \FunctionTok{expand.grid}\NormalTok{(}\AttributeTok{nrounds=}\DecValTok{100}\NormalTok{, }\AttributeTok{max\_depth=}\DecValTok{4}\NormalTok{, }\AttributeTok{eta=}\FloatTok{0.1}\NormalTok{, }\AttributeTok{gamma=}\DecValTok{0}\NormalTok{,}
                           \AttributeTok{colsample\_bytree=}\FloatTok{0.8}\NormalTok{, }\AttributeTok{min\_child\_weight=}\DecValTok{1}\NormalTok{, }\AttributeTok{subsample=}\FloatTok{0.8}\NormalTok{)}
\NormalTok{xgb\_model   }\OtherTok{\textless{}{-}}\NormalTok{ caret}\SpecialCharTok{::}\FunctionTok{train}\NormalTok{(R3haddock }\SpecialCharTok{\textasciitilde{}}\NormalTok{ ., }\AttributeTok{data =}\NormalTok{ train, }\AttributeTok{method =} \StringTok{"xgbTree"}\NormalTok{,}
                            \AttributeTok{trControl =}\NormalTok{ ctrl, }\AttributeTok{tuneGrid =}\NormalTok{ xgb\_grid, }\AttributeTok{verbose =} \DecValTok{0}\NormalTok{)}

\CommentTok{\# {-}{-}{-} 6. Нейросеть (MLP, nnet: линейный выход, стандартизация)}
\CommentTok{\# Учебный смысл: универсальный аппроксиматор; чувствителен к масштабу; требует регуляризации.}
\CommentTok{\# ПОЯСНЕНИЕ: Нейронные сети могут моделировать сложные нелинейные отношения. }
\CommentTok{\# Используемая архитектура (1 скрытый слой) {-} компромисс между гибкостью и риском переобучения.}
\CommentTok{\# Линейный выходной слой подходит для регрессии.}
\NormalTok{nnet\_model  }\OtherTok{\textless{}{-}}\NormalTok{ caret}\SpecialCharTok{::}\FunctionTok{train}\NormalTok{(R3haddock }\SpecialCharTok{\textasciitilde{}}\NormalTok{ ., }\AttributeTok{data =}\NormalTok{ train, }\AttributeTok{method =} \StringTok{"nnet"}\NormalTok{,}
                            \AttributeTok{trControl =}\NormalTok{ ctrl, }\AttributeTok{preProcess =} \FunctionTok{c}\NormalTok{(}\StringTok{"center"}\NormalTok{,}\StringTok{"scale"}\NormalTok{),}
                            \AttributeTok{tuneGrid =} \FunctionTok{expand.grid}\NormalTok{(}\AttributeTok{size =} \DecValTok{5}\NormalTok{, }\AttributeTok{decay =} \FloatTok{0.1}\NormalTok{),}
                            \AttributeTok{linout =} \ConstantTok{TRUE}\NormalTok{, }\AttributeTok{trace =} \ConstantTok{FALSE}\NormalTok{, }\AttributeTok{MaxNWts =} \DecValTok{5000}\NormalTok{)}

\CommentTok{\# {-}{-}{-} 7. Elastic Net (glmnet)}
\CommentTok{\# Учебный смысл: регуляризация (L1/L2), борьба с мультиколлинеарностью, частичный отбор признаков.}
\CommentTok{\# ПОЯСНЕНИЕ: Комбинирует L1 (лассо) и L2 (ридж) регуляризации. Автоматически отбирает признаки }
\CommentTok{\# и уменьшает влияние мультиколлинеарности. Параметр alpha балансирует между лассо и риджем.}
\NormalTok{glmnet\_model}\OtherTok{\textless{}{-}}\NormalTok{ caret}\SpecialCharTok{::}\FunctionTok{train}\NormalTok{(R3haddock }\SpecialCharTok{\textasciitilde{}}\NormalTok{ ., }\AttributeTok{data =}\NormalTok{ train, }\AttributeTok{method =} \StringTok{"glmnet"}\NormalTok{,}
                            \AttributeTok{trControl =}\NormalTok{ ctrl, }\AttributeTok{preProcess =} \FunctionTok{c}\NormalTok{(}\StringTok{"center"}\NormalTok{,}\StringTok{"scale"}\NormalTok{), }\AttributeTok{tuneLength =} \DecValTok{10}\NormalTok{)}

\CommentTok{\# {-}{-}{-} 8. MARS (earth)}
\CommentTok{\# Учебный смысл: кусочно{-}линейные сплайны + простые взаимодействия; гибкая интерпретация.}
\CommentTok{\# ПОЯСНЕНИЕ: Многомерные адаптивные регрессионные сплайны (MARS) строят кусочно{-}линейные модели }
\CommentTok{\# с автоматическим выбором точек излома. Поддерживает взаимодействия ограниченного порядка.}
\NormalTok{earth\_model }\OtherTok{\textless{}{-}}\NormalTok{ caret}\SpecialCharTok{::}\FunctionTok{train}\NormalTok{(R3haddock }\SpecialCharTok{\textasciitilde{}}\NormalTok{ ., }\AttributeTok{data =}\NormalTok{ train, }\AttributeTok{method =} \StringTok{"earth"}\NormalTok{,}
                            \AttributeTok{trControl =}\NormalTok{ ctrl, }\AttributeTok{tuneLength =} \DecValTok{10}\NormalTok{)}
\end{Highlighting}
\end{Shaded}

\begin{verbatim}
Warning in nominalTrainWorkflow(x = x, y = y, wts = weights, info = trainInfo,
: There were missing values in resampled performance measures.
\end{verbatim}

\begin{Shaded}
\begin{Highlighting}[]
\CommentTok{\# {-}{-}{-} 9. SVM с радиальным ядром (svmRadial)}
\CommentTok{\# Учебный смысл: ядровой метод; улавливает сложные нелинейности; важна стандартизация.}
\CommentTok{\# ПОЯСНЕНИЕ: Метод опорных векторов с радиальным ядром проецирует данные в пространство }
\CommentTok{\# высокой размерности, где становится возможным линейное разделение. Параметр gamma управляет }
\CommentTok{\# гибкостью границы решения.}
\NormalTok{svm\_model   }\OtherTok{\textless{}{-}}\NormalTok{ caret}\SpecialCharTok{::}\FunctionTok{train}\NormalTok{(R3haddock }\SpecialCharTok{\textasciitilde{}}\NormalTok{ ., }\AttributeTok{data =}\NormalTok{ train, }\AttributeTok{method =} \StringTok{"svmRadial"}\NormalTok{,}
                            \AttributeTok{trControl =}\NormalTok{ ctrl, }\AttributeTok{preProcess =} \FunctionTok{c}\NormalTok{(}\StringTok{"center"}\NormalTok{,}\StringTok{"scale"}\NormalTok{), }\AttributeTok{tuneLength =} \DecValTok{8}\NormalTok{)}

\CommentTok{\# {-}{-}{-} 10. k{-}ближайших соседей (kNN)}
\CommentTok{\# Учебный смысл: простая интуитивная нелинейная модель на расстояниях; чувствительна к масштабу.}
\CommentTok{\# ПОЯСНЕНИЕ: Предсказание основано на усреднении значений k ближайших наблюдений. }
\CommentTok{\# Требует вычисления попарных расстояний, что может быть ресурсоемким при больших данных.}
\NormalTok{knn\_model   }\OtherTok{\textless{}{-}}\NormalTok{ caret}\SpecialCharTok{::}\FunctionTok{train}\NormalTok{(R3haddock }\SpecialCharTok{\textasciitilde{}}\NormalTok{ ., }\AttributeTok{data =}\NormalTok{ train, }\AttributeTok{method =} \StringTok{"knn"}\NormalTok{,}
                            \AttributeTok{trControl =}\NormalTok{ ctrl, }\AttributeTok{preProcess =} \FunctionTok{c}\NormalTok{(}\StringTok{"center"}\NormalTok{,}\StringTok{"scale"}\NormalTok{), }\AttributeTok{tuneLength =} \DecValTok{15}\NormalTok{)}
\end{Highlighting}
\end{Shaded}

\begin{verbatim}
Warning in knnregTrain(train = structure(c(-1.54402860027016,
-1.04267060653844, : k = 23 exceeds number 22 of patterns
\end{verbatim}

\begin{verbatim}
Warning in knnregTrain(train = structure(c(-1.54402860027016,
-1.04267060653844, : k = 25 exceeds number 22 of patterns
\end{verbatim}

\begin{verbatim}
Warning in knnregTrain(train = structure(c(-1.54402860027016,
-1.04267060653844, : k = 27 exceeds number 22 of patterns
\end{verbatim}

\begin{verbatim}
Warning in knnregTrain(train = structure(c(-1.54402860027016,
-1.04267060653844, : k = 29 exceeds number 22 of patterns
\end{verbatim}

\begin{verbatim}
Warning in knnregTrain(train = structure(c(-1.54402860027016,
-1.04267060653844, : k = 31 exceeds number 22 of patterns
\end{verbatim}

\begin{verbatim}
Warning in knnregTrain(train = structure(c(-1.54402860027016,
-1.04267060653844, : k = 33 exceeds number 22 of patterns
\end{verbatim}

\begin{verbatim}
Warning in knnregTrain(train = structure(c(-1.32034464856588,
-0.795974449614684, : k = 23 exceeds number 22 of patterns
\end{verbatim}

\begin{verbatim}
Warning in knnregTrain(train = structure(c(-1.32034464856588,
-0.795974449614684, : k = 25 exceeds number 22 of patterns
\end{verbatim}

\begin{verbatim}
Warning in knnregTrain(train = structure(c(-1.32034464856588,
-0.795974449614684, : k = 27 exceeds number 22 of patterns
\end{verbatim}

\begin{verbatim}
Warning in knnregTrain(train = structure(c(-1.32034464856588,
-0.795974449614684, : k = 29 exceeds number 22 of patterns
\end{verbatim}

\begin{verbatim}
Warning in knnregTrain(train = structure(c(-1.32034464856588,
-0.795974449614684, : k = 31 exceeds number 22 of patterns
\end{verbatim}

\begin{verbatim}
Warning in knnregTrain(train = structure(c(-1.32034464856588,
-0.795974449614684, : k = 33 exceeds number 22 of patterns
\end{verbatim}

\begin{verbatim}
Warning in knnregTrain(train = structure(c(-1.59980040948893,
-0.134264066935858, : k = 25 exceeds number 23 of patterns
\end{verbatim}

\begin{verbatim}
Warning in knnregTrain(train = structure(c(-1.59980040948893,
-0.134264066935858, : k = 27 exceeds number 23 of patterns
\end{verbatim}

\begin{verbatim}
Warning in knnregTrain(train = structure(c(-1.59980040948893,
-0.134264066935858, : k = 29 exceeds number 23 of patterns
\end{verbatim}

\begin{verbatim}
Warning in knnregTrain(train = structure(c(-1.59980040948893,
-0.134264066935858, : k = 31 exceeds number 23 of patterns
\end{verbatim}

\begin{verbatim}
Warning in knnregTrain(train = structure(c(-1.59980040948893,
-0.134264066935858, : k = 33 exceeds number 23 of patterns
\end{verbatim}

\begin{verbatim}
Warning in knnregTrain(train = structure(c(-1.47821802102901,
-0.982937987281781, : k = 23 exceeds number 22 of patterns
\end{verbatim}

\begin{verbatim}
Warning in knnregTrain(train = structure(c(-1.47821802102901,
-0.982937987281781, : k = 25 exceeds number 22 of patterns
\end{verbatim}

\begin{verbatim}
Warning in knnregTrain(train = structure(c(-1.47821802102901,
-0.982937987281781, : k = 27 exceeds number 22 of patterns
\end{verbatim}

\begin{verbatim}
Warning in knnregTrain(train = structure(c(-1.47821802102901,
-0.982937987281781, : k = 29 exceeds number 22 of patterns
\end{verbatim}

\begin{verbatim}
Warning in knnregTrain(train = structure(c(-1.47821802102901,
-0.982937987281781, : k = 31 exceeds number 22 of patterns
\end{verbatim}

\begin{verbatim}
Warning in knnregTrain(train = structure(c(-1.47821802102901,
-0.982937987281781, : k = 33 exceeds number 22 of patterns
\end{verbatim}

\begin{verbatim}
Warning in knnregTrain(train = structure(c(-1.07995722209223,
-0.0328010741655768, : k = 25 exceeds number 23 of patterns
\end{verbatim}

\begin{verbatim}
Warning in knnregTrain(train = structure(c(-1.07995722209223,
-0.0328010741655768, : k = 27 exceeds number 23 of patterns
\end{verbatim}

\begin{verbatim}
Warning in knnregTrain(train = structure(c(-1.07995722209223,
-0.0328010741655768, : k = 29 exceeds number 23 of patterns
\end{verbatim}

\begin{verbatim}
Warning in knnregTrain(train = structure(c(-1.07995722209223,
-0.0328010741655768, : k = 31 exceeds number 23 of patterns
\end{verbatim}

\begin{verbatim}
Warning in knnregTrain(train = structure(c(-1.07995722209223,
-0.0328010741655768, : k = 33 exceeds number 23 of patterns
\end{verbatim}

\begin{verbatim}
Warning in nominalTrainWorkflow(x = x, y = y, wts = weights, info = trainInfo,
: There were missing values in resampled performance measures.
\end{verbatim}

\begin{Shaded}
\begin{Highlighting}[]
\CommentTok{\# {-}{-}{-} 11. Ranger (быстрый Random Forest)}
\CommentTok{\# Учебный смысл: альтернативная/быстрая реализация леса; сравнить с randomForest.}
\CommentTok{\# ПОЯСНЕНИЕ: Оптимизированная реализация Random Forest на C++. Поддерживает распараллеливание }
\CommentTok{\# и эффективную работу с категориальными переменными. Важен параметр mtry (число признаков в узле).}
\NormalTok{ranger\_model}\OtherTok{\textless{}{-}}\NormalTok{ caret}\SpecialCharTok{::}\FunctionTok{train}\NormalTok{(R3haddock }\SpecialCharTok{\textasciitilde{}}\NormalTok{ ., }\AttributeTok{data =}\NormalTok{ train, }\AttributeTok{method =} \StringTok{"ranger"}\NormalTok{,}
                            \AttributeTok{trControl =}\NormalTok{ ctrl, }\AttributeTok{tuneLength =} \DecValTok{3}\NormalTok{, }\AttributeTok{importance =} \StringTok{"impurity"}\NormalTok{)}

\CommentTok{\# {-}{-}{-} 12. GBM (классический градиентный бустинг)}
\CommentTok{\# Учебный смысл: другой бустинг деревьев; полезно сравнить с XGBoost.}
\CommentTok{\# ПОЯСНЕНИЕ: Градиентный бустинг строит деревья последовательно, где каждое новое дерево }
\CommentTok{\# корректирует ошибки предыдущих. Параметр shrinkage (темп обучения) контролирует скорость обучения.}
\NormalTok{gbm\_model   }\OtherTok{\textless{}{-}}\NormalTok{ caret}\SpecialCharTok{::}\FunctionTok{train}\NormalTok{(R3haddock }\SpecialCharTok{\textasciitilde{}}\NormalTok{ ., }\AttributeTok{data =}\NormalTok{ train, }\AttributeTok{method =} \StringTok{"gbm"}\NormalTok{,}
                            \AttributeTok{trControl =}\NormalTok{ ctrl,}
                            \AttributeTok{tuneGrid =} \FunctionTok{expand.grid}\NormalTok{(}\AttributeTok{n.trees=}\DecValTok{100}\NormalTok{, }\AttributeTok{interaction.depth=}\DecValTok{1}\NormalTok{,}
                                                   \AttributeTok{shrinkage=}\FloatTok{0.1}\NormalTok{, }\AttributeTok{n.minobsinnode=}\DecValTok{2}\NormalTok{),}
                            \AttributeTok{distribution =} \StringTok{"gaussian"}\NormalTok{, }\AttributeTok{bag.fraction =} \DecValTok{1}\NormalTok{, }\AttributeTok{verbose =} \ConstantTok{FALSE}\NormalTok{)}

\CommentTok{\# {-}{-}{-} 13. PLS (Partial Least Squares)}
\CommentTok{\# Учебный смысл: проекция на скрытые компоненты с учетом отклика; решает мультиколлинеарность.}
\CommentTok{\# ПОЯСНЕНИЕ: Частные наименьшие квадраты (PLS) проецируют предикторы в латентное пространство, }
\CommentTok{\# максимизируя ковариацию с откликом. Эффективен при высокой корреляции признаков.}
\NormalTok{pls\_model   }\OtherTok{\textless{}{-}}\NormalTok{ caret}\SpecialCharTok{::}\FunctionTok{train}\NormalTok{(R3haddock }\SpecialCharTok{\textasciitilde{}}\NormalTok{ ., }\AttributeTok{data =}\NormalTok{ train, }\AttributeTok{method =} \StringTok{"pls"}\NormalTok{,}
                            \AttributeTok{trControl =}\NormalTok{ ctrl, }\AttributeTok{preProcess =} \FunctionTok{c}\NormalTok{(}\StringTok{"center"}\NormalTok{,}\StringTok{"scale"}\NormalTok{), }\AttributeTok{tuneLength =} \DecValTok{10}\NormalTok{)}

\CommentTok{\# {-}{-}{-} 14. Cubist (правила + деревья)}
\CommentTok{\# Учебный смысл: интерпретируемые правила с комитетами; часто силен на табличных данных.}
\CommentTok{\# ПОЯСНЕНИЕ: Cubist объединяет деревья решений с линейными моделями в листьях. Генерирует }
\CommentTok{\# набор правил "если{-}то", что улучшает интерпретируемость. Комитеты (комитеты) уменьшают дисперсию.}
\NormalTok{cubist\_model}\OtherTok{\textless{}{-}}\NormalTok{ caret}\SpecialCharTok{::}\FunctionTok{train}\NormalTok{(R3haddock }\SpecialCharTok{\textasciitilde{}}\NormalTok{ ., }\AttributeTok{data =}\NormalTok{ train, }\AttributeTok{method =} \StringTok{"cubist"}\NormalTok{,}
                            \AttributeTok{trControl =}\NormalTok{ ctrl, }\AttributeTok{tuneLength =} \DecValTok{5}\NormalTok{)}
\end{Highlighting}
\end{Shaded}

\begin{verbatim}
Warning in nominalTrainWorkflow(x = x, y = y, wts = weights, info = trainInfo,
: There were missing values in resampled performance measures.
\end{verbatim}

\begin{Shaded}
\begin{Highlighting}[]
\CommentTok{\# 2.5 Метрики и оценка на тесте}
\CommentTok{\# Замечание: RMSE/MAE — абсолютные ошибки; R2 — доля объяснённой вариации;}
\CommentTok{\# MAPE/sMAPE — относительные ошибки (осторожно при малых значениях отклика).}
\NormalTok{rmse  }\OtherTok{\textless{}{-}} \ControlFlowTok{function}\NormalTok{(a, p) }\FunctionTok{sqrt}\NormalTok{(}\FunctionTok{mean}\NormalTok{((a }\SpecialCharTok{{-}}\NormalTok{ p)}\SpecialCharTok{\^{}}\DecValTok{2}\NormalTok{, }\AttributeTok{na.rm =} \ConstantTok{TRUE}\NormalTok{))}
\NormalTok{mae   }\OtherTok{\textless{}{-}} \ControlFlowTok{function}\NormalTok{(a, p) }\FunctionTok{mean}\NormalTok{(}\FunctionTok{abs}\NormalTok{(a }\SpecialCharTok{{-}}\NormalTok{ p), }\AttributeTok{na.rm =} \ConstantTok{TRUE}\NormalTok{)}
\NormalTok{r2    }\OtherTok{\textless{}{-}} \ControlFlowTok{function}\NormalTok{(a, p) }\DecValTok{1} \SpecialCharTok{{-}} \FunctionTok{sum}\NormalTok{((a }\SpecialCharTok{{-}}\NormalTok{ p)}\SpecialCharTok{\^{}}\DecValTok{2}\NormalTok{, }\AttributeTok{na.rm =} \ConstantTok{TRUE}\NormalTok{) }\SpecialCharTok{/} \FunctionTok{sum}\NormalTok{((a }\SpecialCharTok{{-}} \FunctionTok{mean}\NormalTok{(a))}\SpecialCharTok{\^{}}\DecValTok{2}\NormalTok{, }\AttributeTok{na.rm =} \ConstantTok{TRUE}\NormalTok{)}
\NormalTok{mape  }\OtherTok{\textless{}{-}} \ControlFlowTok{function}\NormalTok{(a, p) }\FunctionTok{mean}\NormalTok{(}\FunctionTok{abs}\NormalTok{((a }\SpecialCharTok{{-}}\NormalTok{ p) }\SpecialCharTok{/}\NormalTok{ a), }\AttributeTok{na.rm =} \ConstantTok{TRUE}\NormalTok{) }\SpecialCharTok{*} \DecValTok{100}
\NormalTok{smape }\OtherTok{\textless{}{-}} \ControlFlowTok{function}\NormalTok{(a, p) }\FunctionTok{mean}\NormalTok{(}\DecValTok{2} \SpecialCharTok{*} \FunctionTok{abs}\NormalTok{(p }\SpecialCharTok{{-}}\NormalTok{ a) }\SpecialCharTok{/}\NormalTok{ (}\FunctionTok{abs}\NormalTok{(a) }\SpecialCharTok{+} \FunctionTok{abs}\NormalTok{(p)), }\AttributeTok{na.rm =} \ConstantTok{TRUE}\NormalTok{) }\SpecialCharTok{*} \DecValTok{100}
\NormalTok{metrics\_vec }\OtherTok{\textless{}{-}} \ControlFlowTok{function}\NormalTok{(y, pred) }\FunctionTok{c}\NormalTok{(}\AttributeTok{RMSE=}\FunctionTok{rmse}\NormalTok{(y,pred), }\AttributeTok{MAE=}\FunctionTok{mae}\NormalTok{(y,pred), }\AttributeTok{R2=}\FunctionTok{r2}\NormalTok{(y,pred),}
                                   \AttributeTok{MAPE=}\FunctionTok{mape}\NormalTok{(y,pred), }\AttributeTok{sMAPE=}\FunctionTok{smape}\NormalTok{(y,pred))}

\CommentTok{\# Для оценки качества моделей мы используем несколько метрик:}
\CommentTok{\#   {-} RMSE (Root Mean Square Error): среднеквадратичная ошибка (чувствительна к выбросам)}
\CommentTok{\#   {-} MAE (Mean Absolute Error): средняя абсолютная ошибка (более интерпретируема)}
\CommentTok{\#   {-} R²: коэффициент детерминации (доля объясненной дисперсии)}
\CommentTok{\#   {-} MAPE: средняя абсолютная процентная ошибка (в процентах от фактического значения)}
\CommentTok{\#   {-} sMAPE: симметричная MAPE (устраняет проблему деления на ноль) }

\NormalTok{y\_test }\OtherTok{\textless{}{-}}\NormalTok{ test}\SpecialCharTok{$}\NormalTok{R3haddock}
\NormalTok{preds\_test }\OtherTok{\textless{}{-}} \FunctionTok{list}\NormalTok{(}
  \AttributeTok{LM=}\FunctionTok{predict}\NormalTok{(lm\_model,test), }\AttributeTok{GLM=}\FunctionTok{predict}\NormalTok{(glm\_model,test), }\AttributeTok{GAM=}\FunctionTok{predict}\NormalTok{(gam\_model,test),}
  \AttributeTok{RF=}\FunctionTok{predict}\NormalTok{(rf\_model,test), }\AttributeTok{XGB=}\FunctionTok{predict}\NormalTok{(xgb\_model,test), }\AttributeTok{NNET=}\FunctionTok{predict}\NormalTok{(nnet\_model,test),}
  \AttributeTok{ENet=}\FunctionTok{predict}\NormalTok{(glmnet\_model,test), }\AttributeTok{MARS=}\FunctionTok{predict}\NormalTok{(earth\_model,test), }\AttributeTok{SVM=}\FunctionTok{predict}\NormalTok{(svm\_model,test),}
  \AttributeTok{kNN=}\FunctionTok{predict}\NormalTok{(knn\_model,test), }\AttributeTok{RANGER=}\FunctionTok{predict}\NormalTok{(ranger\_model,test), }\AttributeTok{GBM=}\FunctionTok{predict}\NormalTok{(gbm\_model,test),}
  \AttributeTok{PLS=}\FunctionTok{predict}\NormalTok{(pls\_model,test), }\AttributeTok{CUBIST=}\FunctionTok{predict}\NormalTok{(cubist\_model,test)}
\NormalTok{)}
\NormalTok{metrics\_table }\OtherTok{\textless{}{-}} \FunctionTok{do.call}\NormalTok{(rbind, }\FunctionTok{lapply}\NormalTok{(}\FunctionTok{names}\NormalTok{(preds\_test), }\ControlFlowTok{function}\NormalTok{(nm)\{}
  \FunctionTok{data.frame}\NormalTok{(}\AttributeTok{Model =}\NormalTok{ nm, }\FunctionTok{t}\NormalTok{(}\FunctionTok{metrics\_vec}\NormalTok{(y\_test, preds\_test[[nm]])), }\AttributeTok{row.names =} \ConstantTok{NULL}\NormalTok{)}
\NormalTok{\})) }\SpecialCharTok{\%\textgreater{}\%} \FunctionTok{arrange}\NormalTok{(RMSE, MAE)}

\CommentTok{\# Создаем копию таблицы для округления}
\NormalTok{metrics\_table\_rounded }\OtherTok{\textless{}{-}}\NormalTok{ metrics\_table}

\CommentTok{\# Находим индексы числовых столбцов (исключая первый столбец "Model")}
\NormalTok{numeric\_cols }\OtherTok{\textless{}{-}} \FunctionTok{sapply}\NormalTok{(metrics\_table\_rounded, is.numeric)}

\CommentTok{\# Округляем только числовые столбцы до 2 знаков}
\NormalTok{metrics\_table\_rounded[numeric\_cols] }\OtherTok{\textless{}{-}} \FunctionTok{round}\NormalTok{(metrics\_table\_rounded[numeric\_cols], }\DecValTok{2}\NormalTok{)}

\CommentTok{\# Выводим округленную таблицу}
\FunctionTok{print}\NormalTok{(metrics\_table\_rounded)}
\end{Highlighting}
\end{Shaded}

\begin{verbatim}
    Model      RMSE       MAE    R2   MAPE sMAPE
1     GBM  65112.49  55805.52  0.76  34.41 27.07
2    MARS  95150.47  78452.71  0.49  51.69 36.18
3      RF 125056.40  97862.47  0.12  74.27 44.45
4     PLS 130411.86 111694.80  0.04  62.51 47.69
5      LM 131592.23 113063.92  0.02  63.62 48.29
6     GAM 140393.68 121236.06 -0.11  70.56 48.99
7  CUBIST 147031.92 123320.33 -0.22  50.52 46.56
8    ENet 147470.05 124147.91 -0.23  81.27 52.96
9  RANGER 148623.73 127762.15 -0.25  87.21 54.25
10    kNN 149082.19 128657.28 -0.26  88.53 53.55
11    GLM 151391.83 129510.77 -0.30  74.40 55.01
12    SVM 167864.37 147269.48 -0.59 104.73 57.83
13    XGB 208719.19 194346.48 -1.46  88.97 65.24
14   NNET 216172.26 199594.44 -1.64  93.12 87.93
\end{verbatim}

\begin{Shaded}
\begin{Highlighting}[]
\NormalTok{knitr}\SpecialCharTok{::}\FunctionTok{kable}\NormalTok{(}
\NormalTok{  metrics\_table }\SpecialCharTok{\%\textgreater{}\%}\NormalTok{ dplyr}\SpecialCharTok{::}\FunctionTok{mutate}\NormalTok{(dplyr}\SpecialCharTok{::}\FunctionTok{across}\NormalTok{(}\FunctionTok{where}\NormalTok{(is.numeric), }\SpecialCharTok{\textasciitilde{}}\FunctionTok{round}\NormalTok{(.x, }\DecValTok{2}\NormalTok{))),}
  \AttributeTok{caption =} \StringTok{"Holdout{-}метрики (округлено до 2 знаков)"}
\NormalTok{)}
\end{Highlighting}
\end{Shaded}

\begin{longtable}[]{@{}lrrrrr@{}}
\caption{Holdout-метрики (округлено до 2 знаков)}\tabularnewline
\toprule\noalign{}
Model & RMSE & MAE & R2 & MAPE & sMAPE \\
\midrule\noalign{}
\endfirsthead
\toprule\noalign{}
Model & RMSE & MAE & R2 & MAPE & sMAPE \\
\midrule\noalign{}
\endhead
\bottomrule\noalign{}
\endlastfoot
GBM & 65112.49 & 55805.52 & 0.76 & 34.41 & 27.07 \\
MARS & 95150.47 & 78452.71 & 0.49 & 51.69 & 36.18 \\
RF & 125056.40 & 97862.47 & 0.12 & 74.27 & 44.45 \\
PLS & 130411.86 & 111694.80 & 0.04 & 62.51 & 47.69 \\
LM & 131592.23 & 113063.92 & 0.02 & 63.62 & 48.29 \\
GAM & 140393.68 & 121236.06 & -0.11 & 70.56 & 48.99 \\
CUBIST & 147031.92 & 123320.33 & -0.22 & 50.52 & 46.56 \\
ENet & 147470.05 & 124147.91 & -0.23 & 81.27 & 52.96 \\
RANGER & 148623.73 & 127762.15 & -0.25 & 87.21 & 54.25 \\
kNN & 149082.19 & 128657.28 & -0.26 & 88.53 & 53.55 \\
GLM & 151391.83 & 129510.77 & -0.30 & 74.40 & 55.01 \\
SVM & 167864.37 & 147269.48 & -0.59 & 104.73 & 57.83 \\
XGB & 208719.19 & 194346.48 & -1.46 & 88.97 & 65.24 \\
NNET & 216172.26 & 199594.44 & -1.64 & 93.12 & 87.93 \\
\end{longtable}

\begin{Shaded}
\begin{Highlighting}[]
\CommentTok{\# 2.6 CV{-}резюме}
\CommentTok{\# Сводим результаты CV по всем моделям и смотрим распределения ошибок.}
\NormalTok{results }\OtherTok{\textless{}{-}}\NormalTok{ caret}\SpecialCharTok{::}\FunctionTok{resamples}\NormalTok{(}\FunctionTok{list}\NormalTok{(}
  \AttributeTok{LM=}\NormalTok{lm\_model, }\AttributeTok{GLM=}\NormalTok{glm\_model, }\AttributeTok{GAM=}\NormalTok{gam\_model, }\AttributeTok{RF=}\NormalTok{rf\_model, }\AttributeTok{XGB=}\NormalTok{xgb\_model, }\AttributeTok{NNET=}\NormalTok{nnet\_model,}
  \AttributeTok{ENet=}\NormalTok{glmnet\_model, }\AttributeTok{MARS=}\NormalTok{earth\_model, }\AttributeTok{SVM=}\NormalTok{svm\_model, }\AttributeTok{kNN=}\NormalTok{knn\_model, }\AttributeTok{RANGER=}\NormalTok{ranger\_model,}
  \AttributeTok{GBM=}\NormalTok{gbm\_model, }\AttributeTok{PLS=}\NormalTok{pls\_model, }\AttributeTok{CUBIST=}\NormalTok{cubist\_model}
\NormalTok{))}
\FunctionTok{summary}\NormalTok{(results)}
\end{Highlighting}
\end{Shaded}

\begin{verbatim}

Call:
summary.resamples(object = results)

Models: LM, GLM, GAM, RF, XGB, NNET, ENet, MARS, SVM, kNN, RANGER, GBM, PLS, CUBIST 
Number of resamples: 5 

MAE 
            Min.  1st Qu.   Median     Mean  3rd Qu.     Max. NA's
LM      57204.50 173765.7 178162.9 167545.5 210890.4 217704.1    0
GLM    121162.89 123189.9 126872.5 161564.3 215648.4 220947.6    0
GAM    113581.17 186001.4 194396.8 216924.0 243663.3 346977.3    0
RF     140917.34 173192.3 183106.9 211428.5 199504.4 360421.6    0
XGB    191739.57 196545.5 204224.0 220586.8 211582.9 298842.3    0
NNET   210800.50 230696.3 231946.6 255228.9 268869.5 333831.8    0
ENet    95118.26 114161.5 186843.2 175901.3 203787.5 279596.0    0
MARS   100715.25 160470.7 226674.7 224830.7 281753.1 354539.8    0
SVM    137655.75 168734.2 186772.6 218295.0 270224.5 328087.8    0
kNN    153725.61 173448.8 204644.5 201592.1 210974.4 265167.1    0
RANGER 134039.92 173499.6 173657.3 188631.8 213348.0 248614.3    0
GBM    142143.64 174377.4 182594.4 195116.5 210961.4 265505.7    0
PLS    140180.64 169826.6 174374.4 174539.4 177221.3 211093.9    0
CUBIST  40943.57 166731.9 172415.9 151410.2 183157.2 193802.4    0

RMSE 
            Min.  1st Qu.   Median     Mean  3rd Qu.     Max. NA's
LM      83394.05 204460.2 213949.1 198791.4 234397.7 257756.2    0
GLM    144920.86 174777.3 199683.5 211248.8 268416.8 268445.7    0
GAM    126481.10 212799.6 231166.1 251450.9 265213.5 421594.3    0
RF     152505.15 221538.9 237863.1 264290.3 246285.2 463259.3    0
XGB    224482.10 230814.5 274803.3 282033.4 324393.6 355673.5    0
NNET   262262.48 292982.1 298735.5 322398.1 312361.7 445649.0    0
ENet    96405.72 194703.2 212260.2 215307.8 227408.1 345761.6    0
MARS   133708.92 174187.4 275486.7 264992.8 296803.1 444778.0    0
SVM    175110.30 208575.8 211026.8 261495.0 330657.7 382104.5    0
kNN    191674.11 194743.2 293783.4 261451.2 307657.4 319397.8    0
RANGER 179305.47 214964.7 252239.7 245471.8 256030.3 324818.7    0
GBM    155235.18 236112.1 265637.9 269271.4 342304.1 347067.5    0
PLS    192456.23 204052.4 206556.2 210244.7 207702.6 240455.9    0
CUBIST  51000.20 217680.6 220028.3 190972.0 226862.3 239288.7    0

Rsquared 
              Min.    1st Qu.     Median      Mean   3rd Qu.      Max. NA's
LM     0.038191154 0.34151126 0.59570117 0.5146318 0.6446897 0.9530659    0
GLM    0.036094303 0.25361489 0.52140959 0.5247907 0.8499520 0.9628829    0
GAM    0.078247285 0.20151523 0.39934478 0.3553497 0.5269644 0.5706765    0
RF     0.007930602 0.01517051 0.11630358 0.1925039 0.1207826 0.7023323    0
XGB    0.041587807 0.09841163 0.11696588 0.2060251 0.2118923 0.5612680    0
NNET   0.018389559 0.02872886 0.07737108 0.1597232 0.1157484 0.5583779    0
ENet   0.170405604 0.55004140 0.62684865 0.5443111 0.6741520 0.7001081    0
MARS   0.007289893 0.04026192 0.24769274 0.3088361 0.3755802 0.8733557    0
SVM    0.044855520 0.06565037 0.24061923 0.2637711 0.4593310 0.5083994    0
kNN    0.001714988 0.10251708 0.41997970 0.3013771 0.4201459 0.5625279    0
RANGER 0.006761073 0.20688755 0.22338943 0.2919063 0.4553028 0.5671905    0
GBM    0.056729819 0.15259735 0.27867315 0.2779509 0.3246546 0.5770994    0
PLS    0.255560646 0.39058052 0.43207618 0.5269205 0.6127945 0.9435907    0
CUBIST 0.004399571 0.20141307 0.52557866 0.5053060 0.8675585 0.9275803    0
\end{verbatim}

\begin{Shaded}
\begin{Highlighting}[]
\NormalTok{lattice}\SpecialCharTok{::}\FunctionTok{dotplot}\NormalTok{(results, }\AttributeTok{metric =} \StringTok{"RMSE"}\NormalTok{)}
\end{Highlighting}
\end{Shaded}

\pandocbounded{\includegraphics[keepaspectratio]{chapter7_files/figure-pdf/unnamed-chunk-6-5.pdf}}

\begin{Shaded}
\begin{Highlighting}[]
\CommentTok{\# ==============================================================================}
\CommentTok{\# 3) ВЫБОР ЛУЧШЕЙ ПРОГНОСТИЧЕСКОЙ МОДЕЛИ (TIME{-}SLICE CV НА 3 ГОДА + ХРОНО{-}ТЕСТ)}
\CommentTok{\# Делим последние годы в тест, внутри train — скользящее окно, h=3.}
\CommentTok{\# {-}{-}{-}{-}{-}{-}{-}{-}{-}{-}{-}{-}{-}{-}{-}{-}{-}{-}{-}{-}{-}{-}{-}{-}{-}{-}{-}{-}{-}{-}{-}{-}{-}{-}{-}{-}{-}{-}{-}{-}{-}{-}{-}{-}{-}{-}{-}{-}{-}{-}{-}{-}{-}{-}{-}{-}{-}{-}{-}{-}{-}{-}{-}{-}{-}{-}{-}{-}{-}{-}{-}{-}{-}{-}{-}{-}{-}{-}}
\CommentTok{\# Почему time{-}slice: временные данные нельзя случайно перемешивать, иначе мы}
\CommentTok{\# «подсматриваем в будущее». Создаём серии обучающих/валидационных окон,}
\CommentTok{\# увеличивая тренировочный период, и тестируем на ближайшем горизонте (3 года).}
\CommentTok{\# ==============================================================================}

\CommentTok{\# 3.1 Данные для time{-}slice (с YEAR)}
\NormalTok{model\_data }\OtherTok{\textless{}{-}} \FunctionTok{read.csv}\NormalTok{(}\StringTok{"selected\_predictors\_dataset.csv"}\NormalTok{, }\AttributeTok{header =} \ConstantTok{TRUE}\NormalTok{, }\AttributeTok{stringsAsFactors =} \ConstantTok{FALSE}\NormalTok{)}
\ControlFlowTok{if}\NormalTok{ (}\SpecialCharTok{!}\StringTok{"YEAR"} \SpecialCharTok{\%in\%} \FunctionTok{names}\NormalTok{(model\_data)) \{}
\NormalTok{  model\_data}\SpecialCharTok{$}\NormalTok{YEAR }\OtherTok{\textless{}{-}} \FunctionTok{seq}\NormalTok{(}\DecValTok{1990}\NormalTok{, }\AttributeTok{by =} \DecValTok{1}\NormalTok{, }\AttributeTok{length.out =} \FunctionTok{nrow}\NormalTok{(model\_data))}
\NormalTok{\}}
\CommentTok{\# Хронологический порядок}
\NormalTok{model\_data }\OtherTok{\textless{}{-}}\NormalTok{ model\_data }\SpecialCharTok{\%\textgreater{}\%} \FunctionTok{arrange}\NormalTok{(YEAR)}

\CommentTok{\# Для временных рядов обычные методы кросс{-}валидации (случайное разбиение) неприменимы,}
\CommentTok{\# так как это приведет к утечке информации из будущего в прошлое. [[1]]}
\CommentTok{\# Time{-}slice CV (скользящее окно) имитирует реальную ситуацию прогнозирования:}
\CommentTok{\#   {-} Мы обучаемся на данных из прошлого}
\CommentTok{\#   {-} Прогнозируем на несколько шагов вперед}
\CommentTok{\#   {-} Последовательно сдвигаем окно обучения вперед}

\CommentTok{\# Исходные фичи (исключаем YEAR)}
\NormalTok{md\_for\_fit }\OtherTok{\textless{}{-}}\NormalTok{ model\_data }\SpecialCharTok{\%\textgreater{}\%} \FunctionTok{select}\NormalTok{(codTSB, T12, I5, NAOspring, haddock68, R3haddock)}

\CommentTok{\# 3.2 Хронологический holdout (последние годы)}
\CommentTok{\# Идея: отложим \textasciitilde{}20\% последних лет как полностью внешний тест будущего качества.}
\NormalTok{n }\OtherTok{\textless{}{-}} \FunctionTok{nrow}\NormalTok{(md\_for\_fit)}
\NormalTok{holdout\_frac }\OtherTok{\textless{}{-}} \FloatTok{0.2}
\NormalTok{n\_test }\OtherTok{\textless{}{-}} \FunctionTok{max}\NormalTok{(}\DecValTok{4}\NormalTok{, }\FunctionTok{ceiling}\NormalTok{(n }\SpecialCharTok{*}\NormalTok{ holdout\_frac))}
\NormalTok{train\_ts }\OtherTok{\textless{}{-}} \FunctionTok{head}\NormalTok{(md\_for\_fit, n }\SpecialCharTok{{-}}\NormalTok{ n\_test)}
\NormalTok{test\_ts  }\OtherTok{\textless{}{-}} \FunctionTok{tail}\NormalTok{(md\_for\_fit, n\_test)}

\CommentTok{\# 3.3 Time{-}slice CV (h=3, expanding window рекомендован: fixedWindow=FALSE)}
\CommentTok{\# initialWindow — размер первого «обучающего» фрагмента; horizon — горизонт}
\CommentTok{\# валидации (здесь 3 года). Далее окно расширяется.}
\NormalTok{n\_train }\OtherTok{\textless{}{-}} \FunctionTok{nrow}\NormalTok{(train\_ts)}
\NormalTok{initial\_frac }\OtherTok{\textless{}{-}} \FloatTok{0.6}
\NormalTok{horizon      }\OtherTok{\textless{}{-}} \DecValTok{3}
\NormalTok{initialWindow }\OtherTok{\textless{}{-}} \FunctionTok{max}\NormalTok{(}\DecValTok{10}\NormalTok{, }\FunctionTok{floor}\NormalTok{(initial\_frac }\SpecialCharTok{*}\NormalTok{ n\_train))}
\ControlFlowTok{if}\NormalTok{ (initialWindow }\SpecialCharTok{+}\NormalTok{ horizon }\SpecialCharTok{\textgreater{}}\NormalTok{ n\_train) initialWindow }\OtherTok{\textless{}{-}}\NormalTok{ n\_train }\SpecialCharTok{{-}}\NormalTok{ horizon}

\NormalTok{slices }\OtherTok{\textless{}{-}}\NormalTok{ caret}\SpecialCharTok{::}\FunctionTok{createTimeSlices}\NormalTok{(}\DecValTok{1}\SpecialCharTok{:}\NormalTok{n\_train, }\AttributeTok{initialWindow =}\NormalTok{ initialWindow,}
                                  \AttributeTok{horizon =}\NormalTok{ horizon, }\AttributeTok{fixedWindow =} \ConstantTok{FALSE}\NormalTok{)}
\NormalTok{ctrl\_ts }\OtherTok{\textless{}{-}}\NormalTok{ caret}\SpecialCharTok{::}\FunctionTok{trainControl}\NormalTok{(}\AttributeTok{method =} \StringTok{"cv"}\NormalTok{, }\AttributeTok{index =}\NormalTok{ slices}\SpecialCharTok{$}\NormalTok{train, }\AttributeTok{indexOut =}\NormalTok{ slices}\SpecialCharTok{$}\NormalTok{test,}
                               \AttributeTok{savePredictions =} \StringTok{"final"}\NormalTok{)}

\CommentTok{\# В нашем случае:}
\CommentTok{\#   {-} horizon = 3: прогнозируем на 3 года вперед}
\CommentTok{\#   {-} expanding window: размер обучающей выборки увеличивается с каждым шагом}
\CommentTok{\#   {-} initialWindow: начальный размер обучающей выборки (60\% от данных)}
\CommentTok{\# Этот подход наиболее реалистичен для задач прогнозирования временных рядов в гидробиологии.}


\CommentTok{\# 3.4 Обучение (ядро набора, без GBM — он нестабилен на малом n в timeslice)}
\CommentTok{\# Примечание: используем ту же рецептуру, что и в базовом сравнении, но с}
\CommentTok{\# хронологическими срезами.}

\NormalTok{fit\_ts }\OtherTok{\textless{}{-}} \ControlFlowTok{function}\NormalTok{(method, form, data, ctrl, ...) \{}
\NormalTok{  out }\OtherTok{\textless{}{-}} \FunctionTok{try}\NormalTok{(caret}\SpecialCharTok{::}\FunctionTok{train}\NormalTok{(form, }\AttributeTok{data =}\NormalTok{ data, }\AttributeTok{method =}\NormalTok{ method, }\AttributeTok{trControl =}\NormalTok{ ctrl, ...), }\ConstantTok{TRUE}\NormalTok{)}
  \ControlFlowTok{if}\NormalTok{ (}\FunctionTok{inherits}\NormalTok{(out,}\StringTok{"try{-}error"}\NormalTok{)) }\ConstantTok{NULL} \ControlFlowTok{else}\NormalTok{ out}
\NormalTok{\}}
\NormalTok{lm\_ts   }\OtherTok{\textless{}{-}} \FunctionTok{fit\_ts}\NormalTok{(}\StringTok{"lm"}\NormalTok{,        R3haddock }\SpecialCharTok{\textasciitilde{}}\NormalTok{ ., train\_ts, ctrl\_ts)}
\NormalTok{glm\_ts  }\OtherTok{\textless{}{-}} \FunctionTok{fit\_ts}\NormalTok{(}\StringTok{"glm"}\NormalTok{,       R3haddock }\SpecialCharTok{\textasciitilde{}}\NormalTok{ ., train\_ts, ctrl\_ts, }\AttributeTok{family =} \FunctionTok{Gamma}\NormalTok{(}\AttributeTok{link=}\StringTok{"log"}\NormalTok{))}
\NormalTok{gam\_ts  }\OtherTok{\textless{}{-}}\NormalTok{ caret}\SpecialCharTok{::}\FunctionTok{train}\NormalTok{(}\AttributeTok{x =}\NormalTok{ train\_ts[, }\SpecialCharTok{{-}}\FunctionTok{which}\NormalTok{(}\FunctionTok{names}\NormalTok{(train\_ts)}\SpecialCharTok{==}\StringTok{"R3haddock"}\NormalTok{)],}
                        \AttributeTok{y =}\NormalTok{ train\_ts}\SpecialCharTok{$}\NormalTok{R3haddock, }\AttributeTok{method =}\NormalTok{ gam\_spec, }\AttributeTok{trControl =}\NormalTok{ ctrl\_ts)}
\end{Highlighting}
\end{Shaded}

\begin{verbatim}
Warning in nominalTrainWorkflow(x = x, y = y, wts = weights, info = trainInfo,
: There were missing values in resampled performance measures.
\end{verbatim}

\begin{Shaded}
\begin{Highlighting}[]
\NormalTok{rf\_ts   }\OtherTok{\textless{}{-}} \FunctionTok{fit\_ts}\NormalTok{(}\StringTok{"rf"}\NormalTok{,        R3haddock }\SpecialCharTok{\textasciitilde{}}\NormalTok{ ., train\_ts, ctrl\_ts, }\AttributeTok{ntree=}\DecValTok{1000}\NormalTok{, }\AttributeTok{tuneGrid=}\FunctionTok{data.frame}\NormalTok{(}\AttributeTok{mtry=}\DecValTok{1}\NormalTok{))}
\NormalTok{xgb\_ts  }\OtherTok{\textless{}{-}} \FunctionTok{fit\_ts}\NormalTok{(}\StringTok{"xgbTree"}\NormalTok{,   R3haddock }\SpecialCharTok{\textasciitilde{}}\NormalTok{ ., train\_ts, ctrl\_ts, }\AttributeTok{tuneGrid =}\NormalTok{ xgb\_grid, }\AttributeTok{verbose =} \DecValTok{0}\NormalTok{)}
\NormalTok{rgr\_ts  }\OtherTok{\textless{}{-}} \FunctionTok{fit\_ts}\NormalTok{(}\StringTok{"ranger"}\NormalTok{,    R3haddock }\SpecialCharTok{\textasciitilde{}}\NormalTok{ ., train\_ts, ctrl\_ts, }\AttributeTok{tuneLength=}\DecValTok{3}\NormalTok{)}
\NormalTok{nnet\_ts }\OtherTok{\textless{}{-}} \FunctionTok{fit\_ts}\NormalTok{(}\StringTok{"nnet"}\NormalTok{,      R3haddock }\SpecialCharTok{\textasciitilde{}}\NormalTok{ ., train\_ts, ctrl\_ts,}
                  \AttributeTok{preProcess=}\FunctionTok{c}\NormalTok{(}\StringTok{"center"}\NormalTok{,}\StringTok{"scale"}\NormalTok{),}
                  \AttributeTok{tuneGrid=}\FunctionTok{expand.grid}\NormalTok{(}\AttributeTok{size=}\DecValTok{5}\NormalTok{,}\AttributeTok{decay=}\FloatTok{0.1}\NormalTok{), }\AttributeTok{linout=}\ConstantTok{TRUE}\NormalTok{, }\AttributeTok{trace=}\ConstantTok{FALSE}\NormalTok{, }\AttributeTok{MaxNWts=}\DecValTok{5000}\NormalTok{)}
\end{Highlighting}
\end{Shaded}

\begin{verbatim}
Warning in nominalTrainWorkflow(x = x, y = y, wts = weights, info = trainInfo,
: There were missing values in resampled performance measures.
\end{verbatim}

\begin{Shaded}
\begin{Highlighting}[]
\NormalTok{svm\_ts  }\OtherTok{\textless{}{-}} \FunctionTok{fit\_ts}\NormalTok{(}\StringTok{"svmRadial"}\NormalTok{, R3haddock }\SpecialCharTok{\textasciitilde{}}\NormalTok{ ., train\_ts, ctrl\_ts, }\AttributeTok{preProcess=}\FunctionTok{c}\NormalTok{(}\StringTok{"center"}\NormalTok{,}\StringTok{"scale"}\NormalTok{), }\AttributeTok{tuneLength=}\DecValTok{8}\NormalTok{)}
\NormalTok{knn\_ts  }\OtherTok{\textless{}{-}} \FunctionTok{fit\_ts}\NormalTok{(}\StringTok{"knn"}\NormalTok{,       R3haddock }\SpecialCharTok{\textasciitilde{}}\NormalTok{ ., train\_ts, ctrl\_ts, }\AttributeTok{preProcess=}\FunctionTok{c}\NormalTok{(}\StringTok{"center"}\NormalTok{,}\StringTok{"scale"}\NormalTok{), }\AttributeTok{tuneLength=}\DecValTok{15}\NormalTok{)}
\end{Highlighting}
\end{Shaded}

\begin{verbatim}
Warning in knnregTrain(train = structure(c(-1.91776861098288,
-0.63635090426016, : k = 17 exceeds number 15 of patterns
\end{verbatim}

\begin{verbatim}
Warning in knnregTrain(train = structure(c(-1.91776861098288,
-0.63635090426016, : k = 19 exceeds number 15 of patterns
\end{verbatim}

\begin{verbatim}
Warning in knnregTrain(train = structure(c(-1.91776861098288,
-0.63635090426016, : k = 21 exceeds number 15 of patterns
\end{verbatim}

\begin{verbatim}
Warning in knnregTrain(train = structure(c(-1.91776861098288,
-0.63635090426016, : k = 23 exceeds number 15 of patterns
\end{verbatim}

\begin{verbatim}
Warning in knnregTrain(train = structure(c(-1.91776861098288,
-0.63635090426016, : k = 25 exceeds number 15 of patterns
\end{verbatim}

\begin{verbatim}
Warning in knnregTrain(train = structure(c(-1.91776861098288,
-0.63635090426016, : k = 27 exceeds number 15 of patterns
\end{verbatim}

\begin{verbatim}
Warning in knnregTrain(train = structure(c(-1.91776861098288,
-0.63635090426016, : k = 29 exceeds number 15 of patterns
\end{verbatim}

\begin{verbatim}
Warning in knnregTrain(train = structure(c(-1.91776861098288,
-0.63635090426016, : k = 31 exceeds number 15 of patterns
\end{verbatim}

\begin{verbatim}
Warning in knnregTrain(train = structure(c(-1.91776861098288,
-0.63635090426016, : k = 33 exceeds number 15 of patterns
\end{verbatim}

\begin{verbatim}
Warning in knnregTrain(train = structure(c(-1.97509275968546,
-0.649515010512528, : k = 17 exceeds number 16 of patterns
\end{verbatim}

\begin{verbatim}
Warning in knnregTrain(train = structure(c(-1.97509275968546,
-0.649515010512528, : k = 19 exceeds number 16 of patterns
\end{verbatim}

\begin{verbatim}
Warning in knnregTrain(train = structure(c(-1.97509275968546,
-0.649515010512528, : k = 21 exceeds number 16 of patterns
\end{verbatim}

\begin{verbatim}
Warning in knnregTrain(train = structure(c(-1.97509275968546,
-0.649515010512528, : k = 23 exceeds number 16 of patterns
\end{verbatim}

\begin{verbatim}
Warning in knnregTrain(train = structure(c(-1.97509275968546,
-0.649515010512528, : k = 25 exceeds number 16 of patterns
\end{verbatim}

\begin{verbatim}
Warning in knnregTrain(train = structure(c(-1.97509275968546,
-0.649515010512528, : k = 27 exceeds number 16 of patterns
\end{verbatim}

\begin{verbatim}
Warning in knnregTrain(train = structure(c(-1.97509275968546,
-0.649515010512528, : k = 29 exceeds number 16 of patterns
\end{verbatim}

\begin{verbatim}
Warning in knnregTrain(train = structure(c(-1.97509275968546,
-0.649515010512528, : k = 31 exceeds number 16 of patterns
\end{verbatim}

\begin{verbatim}
Warning in knnregTrain(train = structure(c(-1.97509275968546,
-0.649515010512528, : k = 33 exceeds number 16 of patterns
\end{verbatim}

\begin{verbatim}
Warning in knnregTrain(train = structure(c(-2.03591114922526,
-0.667021070399982, : k = 19 exceeds number 17 of patterns
\end{verbatim}

\begin{verbatim}
Warning in knnregTrain(train = structure(c(-2.03591114922526,
-0.667021070399982, : k = 21 exceeds number 17 of patterns
\end{verbatim}

\begin{verbatim}
Warning in knnregTrain(train = structure(c(-2.03591114922526,
-0.667021070399982, : k = 23 exceeds number 17 of patterns
\end{verbatim}

\begin{verbatim}
Warning in knnregTrain(train = structure(c(-2.03591114922526,
-0.667021070399982, : k = 25 exceeds number 17 of patterns
\end{verbatim}

\begin{verbatim}
Warning in knnregTrain(train = structure(c(-2.03591114922526,
-0.667021070399982, : k = 27 exceeds number 17 of patterns
\end{verbatim}

\begin{verbatim}
Warning in knnregTrain(train = structure(c(-2.03591114922526,
-0.667021070399982, : k = 29 exceeds number 17 of patterns
\end{verbatim}

\begin{verbatim}
Warning in knnregTrain(train = structure(c(-2.03591114922526,
-0.667021070399982, : k = 31 exceeds number 17 of patterns
\end{verbatim}

\begin{verbatim}
Warning in knnregTrain(train = structure(c(-2.03591114922526,
-0.667021070399982, : k = 33 exceeds number 17 of patterns
\end{verbatim}

\begin{verbatim}
Warning in knnregTrain(train = structure(c(-2.09685135650479,
-0.723233808010845, : k = 19 exceeds number 18 of patterns
\end{verbatim}

\begin{verbatim}
Warning in knnregTrain(train = structure(c(-2.09685135650479,
-0.723233808010845, : k = 21 exceeds number 18 of patterns
\end{verbatim}

\begin{verbatim}
Warning in knnregTrain(train = structure(c(-2.09685135650479,
-0.723233808010845, : k = 23 exceeds number 18 of patterns
\end{verbatim}

\begin{verbatim}
Warning in knnregTrain(train = structure(c(-2.09685135650479,
-0.723233808010845, : k = 25 exceeds number 18 of patterns
\end{verbatim}

\begin{verbatim}
Warning in knnregTrain(train = structure(c(-2.09685135650479,
-0.723233808010845, : k = 27 exceeds number 18 of patterns
\end{verbatim}

\begin{verbatim}
Warning in knnregTrain(train = structure(c(-2.09685135650479,
-0.723233808010845, : k = 29 exceeds number 18 of patterns
\end{verbatim}

\begin{verbatim}
Warning in knnregTrain(train = structure(c(-2.09685135650479,
-0.723233808010845, : k = 31 exceeds number 18 of patterns
\end{verbatim}

\begin{verbatim}
Warning in knnregTrain(train = structure(c(-2.09685135650479,
-0.723233808010845, : k = 33 exceeds number 18 of patterns
\end{verbatim}

\begin{verbatim}
Warning in knnregTrain(train = structure(c(-1.87781906605955,
-0.736313775515053, : k = 21 exceeds number 19 of patterns
\end{verbatim}

\begin{verbatim}
Warning in knnregTrain(train = structure(c(-1.87781906605955,
-0.736313775515053, : k = 23 exceeds number 19 of patterns
\end{verbatim}

\begin{verbatim}
Warning in knnregTrain(train = structure(c(-1.87781906605955,
-0.736313775515053, : k = 25 exceeds number 19 of patterns
\end{verbatim}

\begin{verbatim}
Warning in knnregTrain(train = structure(c(-1.87781906605955,
-0.736313775515053, : k = 27 exceeds number 19 of patterns
\end{verbatim}

\begin{verbatim}
Warning in knnregTrain(train = structure(c(-1.87781906605955,
-0.736313775515053, : k = 29 exceeds number 19 of patterns
\end{verbatim}

\begin{verbatim}
Warning in knnregTrain(train = structure(c(-1.87781906605955,
-0.736313775515053, : k = 31 exceeds number 19 of patterns
\end{verbatim}

\begin{verbatim}
Warning in knnregTrain(train = structure(c(-1.87781906605955,
-0.736313775515053, : k = 33 exceeds number 19 of patterns
\end{verbatim}

\begin{verbatim}
Warning in knnregTrain(train = structure(c(-1.59299667707382,
-0.714710653867683, : k = 21 exceeds number 20 of patterns
\end{verbatim}

\begin{verbatim}
Warning in knnregTrain(train = structure(c(-1.59299667707382,
-0.714710653867683, : k = 23 exceeds number 20 of patterns
\end{verbatim}

\begin{verbatim}
Warning in knnregTrain(train = structure(c(-1.59299667707382,
-0.714710653867683, : k = 25 exceeds number 20 of patterns
\end{verbatim}

\begin{verbatim}
Warning in knnregTrain(train = structure(c(-1.59299667707382,
-0.714710653867683, : k = 27 exceeds number 20 of patterns
\end{verbatim}

\begin{verbatim}
Warning in knnregTrain(train = structure(c(-1.59299667707382,
-0.714710653867683, : k = 29 exceeds number 20 of patterns
\end{verbatim}

\begin{verbatim}
Warning in knnregTrain(train = structure(c(-1.59299667707382,
-0.714710653867683, : k = 31 exceeds number 20 of patterns
\end{verbatim}

\begin{verbatim}
Warning in knnregTrain(train = structure(c(-1.59299667707382,
-0.714710653867683, : k = 33 exceeds number 20 of patterns
\end{verbatim}

\begin{verbatim}
Warning in knnregTrain(train = structure(c(-1.44471091878981,
-0.719611760680745, : k = 23 exceeds number 21 of patterns
\end{verbatim}

\begin{verbatim}
Warning in knnregTrain(train = structure(c(-1.44471091878981,
-0.719611760680745, : k = 25 exceeds number 21 of patterns
\end{verbatim}

\begin{verbatim}
Warning in knnregTrain(train = structure(c(-1.44471091878981,
-0.719611760680745, : k = 27 exceeds number 21 of patterns
\end{verbatim}

\begin{verbatim}
Warning in knnregTrain(train = structure(c(-1.44471091878981,
-0.719611760680745, : k = 29 exceeds number 21 of patterns
\end{verbatim}

\begin{verbatim}
Warning in knnregTrain(train = structure(c(-1.44471091878981,
-0.719611760680745, : k = 31 exceeds number 21 of patterns
\end{verbatim}

\begin{verbatim}
Warning in knnregTrain(train = structure(c(-1.44471091878981,
-0.719611760680745, : k = 33 exceeds number 21 of patterns
\end{verbatim}

\begin{verbatim}
Warning in knnregTrain(train = structure(c(-1.35845826709587,
-0.734816317064085, : k = 23 exceeds number 22 of patterns
\end{verbatim}

\begin{verbatim}
Warning in knnregTrain(train = structure(c(-1.35845826709587,
-0.734816317064085, : k = 25 exceeds number 22 of patterns
\end{verbatim}

\begin{verbatim}
Warning in knnregTrain(train = structure(c(-1.35845826709587,
-0.734816317064085, : k = 27 exceeds number 22 of patterns
\end{verbatim}

\begin{verbatim}
Warning in knnregTrain(train = structure(c(-1.35845826709587,
-0.734816317064085, : k = 29 exceeds number 22 of patterns
\end{verbatim}

\begin{verbatim}
Warning in knnregTrain(train = structure(c(-1.35845826709587,
-0.734816317064085, : k = 31 exceeds number 22 of patterns
\end{verbatim}

\begin{verbatim}
Warning in knnregTrain(train = structure(c(-1.35845826709587,
-0.734816317064085, : k = 33 exceeds number 22 of patterns
\end{verbatim}

\begin{verbatim}
Warning in nominalTrainWorkflow(x = x, y = y, wts = weights, info = trainInfo,
: There were missing values in resampled performance measures.
\end{verbatim}

\begin{Shaded}
\begin{Highlighting}[]
\NormalTok{enet\_ts }\OtherTok{\textless{}{-}} \FunctionTok{fit\_ts}\NormalTok{(}\StringTok{"glmnet"}\NormalTok{,    R3haddock }\SpecialCharTok{\textasciitilde{}}\NormalTok{ ., train\_ts, ctrl\_ts, }\AttributeTok{preProcess=}\FunctionTok{c}\NormalTok{(}\StringTok{"center"}\NormalTok{,}\StringTok{"scale"}\NormalTok{), }\AttributeTok{tuneLength=}\DecValTok{10}\NormalTok{)}
\NormalTok{mars\_ts }\OtherTok{\textless{}{-}} \FunctionTok{fit\_ts}\NormalTok{(}\StringTok{"earth"}\NormalTok{,     R3haddock }\SpecialCharTok{\textasciitilde{}}\NormalTok{ ., train\_ts, ctrl\_ts, }\AttributeTok{tuneLength=}\DecValTok{10}\NormalTok{)}
\NormalTok{pls\_ts  }\OtherTok{\textless{}{-}} \FunctionTok{fit\_ts}\NormalTok{(}\StringTok{"pls"}\NormalTok{,       R3haddock }\SpecialCharTok{\textasciitilde{}}\NormalTok{ ., train\_ts, ctrl\_ts, }\AttributeTok{preProcess=}\FunctionTok{c}\NormalTok{(}\StringTok{"center"}\NormalTok{,}\StringTok{"scale"}\NormalTok{), }\AttributeTok{tuneLength=}\DecValTok{10}\NormalTok{)}
\NormalTok{cub\_ts  }\OtherTok{\textless{}{-}} \FunctionTok{fit\_ts}\NormalTok{(}\StringTok{"cubist"}\NormalTok{,    R3haddock }\SpecialCharTok{\textasciitilde{}}\NormalTok{ ., train\_ts, ctrl\_ts, }\AttributeTok{tuneLength=}\DecValTok{5}\NormalTok{)}

\NormalTok{models\_ts }\OtherTok{\textless{}{-}} \FunctionTok{list}\NormalTok{(}\AttributeTok{LM=}\NormalTok{lm\_ts, }\AttributeTok{GLM=}\NormalTok{glm\_ts, }\AttributeTok{GAM=}\NormalTok{gam\_ts, }\AttributeTok{RF=}\NormalTok{rf\_ts, }\AttributeTok{XGB=}\NormalTok{xgb\_ts, }\AttributeTok{RANGER=}\NormalTok{rgr\_ts,}
                  \AttributeTok{NNET=}\NormalTok{nnet\_ts, }\AttributeTok{SVM=}\NormalTok{svm\_ts, }\AttributeTok{kNN=}\NormalTok{knn\_ts, }\AttributeTok{ENet=}\NormalTok{enet\_ts, }\AttributeTok{MARS=}\NormalTok{mars\_ts, }\AttributeTok{PLS=}\NormalTok{pls\_ts, }\AttributeTok{CUBIST=}\NormalTok{cub\_ts)}
\NormalTok{models\_ts }\OtherTok{\textless{}{-}}\NormalTok{ models\_ts[}\SpecialCharTok{!}\FunctionTok{vapply}\NormalTok{(models\_ts, is.null, }\FunctionTok{logical}\NormalTok{(}\DecValTok{1}\NormalTok{))]}

\CommentTok{\# 3.5 Ранжирование по time{-}slice CV и по хронологическому тесту}
\CommentTok{\# Сначала ранжируем по средним ошибкам на валидационных срезах, затем — по внешнему тесту.}
\NormalTok{cv\_metrics }\OtherTok{\textless{}{-}} \ControlFlowTok{function}\NormalTok{(m) \{}
  \ControlFlowTok{if}\NormalTok{ (}\FunctionTok{is.null}\NormalTok{(m}\SpecialCharTok{$}\NormalTok{pred) }\SpecialCharTok{||} \SpecialCharTok{!}\StringTok{"Resample"} \SpecialCharTok{\%in\%} \FunctionTok{names}\NormalTok{(m}\SpecialCharTok{$}\NormalTok{pred)) }\FunctionTok{return}\NormalTok{(}\FunctionTok{c}\NormalTok{(}\AttributeTok{RMSE=}\ConstantTok{NA}\NormalTok{, }\AttributeTok{MAE=}\ConstantTok{NA}\NormalTok{))}
\NormalTok{  by\_slice }\OtherTok{\textless{}{-}}\NormalTok{ m}\SpecialCharTok{$}\NormalTok{pred }\SpecialCharTok{\%\textgreater{}\%} \FunctionTok{group\_by}\NormalTok{(Resample) }\SpecialCharTok{\%\textgreater{}\%}
    \FunctionTok{summarise}\NormalTok{(}\AttributeTok{RMSE=}\FunctionTok{rmse}\NormalTok{(obs,pred), }\AttributeTok{MAE=}\FunctionTok{mae}\NormalTok{(obs,pred), }\AttributeTok{.groups=}\StringTok{"drop"}\NormalTok{)}
  \FunctionTok{c}\NormalTok{(}\AttributeTok{RMSE =} \FunctionTok{mean}\NormalTok{(by\_slice}\SpecialCharTok{$}\NormalTok{RMSE, }\AttributeTok{na.rm =} \ConstantTok{TRUE}\NormalTok{), }\AttributeTok{MAE =} \FunctionTok{mean}\NormalTok{(by\_slice}\SpecialCharTok{$}\NormalTok{MAE, }\AttributeTok{na.rm =} \ConstantTok{TRUE}\NormalTok{))}
\NormalTok{\}}
\NormalTok{cv\_rank }\OtherTok{\textless{}{-}} \FunctionTok{do.call}\NormalTok{(rbind, }\FunctionTok{lapply}\NormalTok{(models\_ts, cv\_metrics)) }\SpecialCharTok{\%\textgreater{}\%} \FunctionTok{as.data.frame}\NormalTok{()}
\NormalTok{cv\_rank}\SpecialCharTok{$}\NormalTok{Model }\OtherTok{\textless{}{-}} \FunctionTok{rownames}\NormalTok{(cv\_rank)}
\NormalTok{cv\_rank }\OtherTok{\textless{}{-}}\NormalTok{ cv\_rank[}\FunctionTok{is.finite}\NormalTok{(cv\_rank}\SpecialCharTok{$}\NormalTok{RMSE), ] }\SpecialCharTok{\%\textgreater{}\%} \FunctionTok{relocate}\NormalTok{(Model) }\SpecialCharTok{\%\textgreater{}\%} \FunctionTok{arrange}\NormalTok{(RMSE, MAE)}
\FunctionTok{cat}\NormalTok{(}\StringTok{"}\SpecialCharTok{\textbackslash{}n}\StringTok{Time{-}slice CV (h=3), средние RMSE/MAE:}\SpecialCharTok{\textbackslash{}n}\StringTok{"}\NormalTok{); }\FunctionTok{print}\NormalTok{(cv\_rank)}
\end{Highlighting}
\end{Shaded}

\begin{verbatim}

Time-slice CV (h=3), средние RMSE/MAE:
\end{verbatim}

\begin{verbatim}
        Model     RMSE      MAE
SVM       SVM 227929.6 191921.6
kNN       kNN 234897.8 197135.0
ENet     ENet 250989.0 214248.7
XGB       XGB 277153.4 248532.6
RANGER RANGER 280255.1 249992.1
GLM       GLM 280259.5 237186.9
NNET     NNET 296856.1 264368.2
PLS       PLS 302968.3 274707.6
RF         RF 303710.0 263019.5
CUBIST CUBIST 314443.4 281437.9
LM         LM 370298.1 340883.8
MARS     MARS 427624.1 378476.5
GAM       GAM 714951.0 625520.3
\end{verbatim}

\begin{Shaded}
\begin{Highlighting}[]
\NormalTok{preds\_ts }\OtherTok{\textless{}{-}} \FunctionTok{lapply}\NormalTok{(models\_ts, }\ControlFlowTok{function}\NormalTok{(m) }\FunctionTok{try}\NormalTok{(}\FunctionTok{predict}\NormalTok{(m, }\AttributeTok{newdata =}\NormalTok{ test\_ts), }\ConstantTok{TRUE}\NormalTok{))}
\NormalTok{keep }\OtherTok{\textless{}{-}} \FunctionTok{vapply}\NormalTok{(preds\_ts, }\ControlFlowTok{function}\NormalTok{(p) }\FunctionTok{is.numeric}\NormalTok{(p) }\SpecialCharTok{\&\&} \FunctionTok{length}\NormalTok{(p)}\SpecialCharTok{==}\FunctionTok{nrow}\NormalTok{(test\_ts) }\SpecialCharTok{\&\&} \FunctionTok{all}\NormalTok{(}\FunctionTok{is.finite}\NormalTok{(p)), }\FunctionTok{logical}\NormalTok{(}\DecValTok{1}\NormalTok{))}
\NormalTok{preds\_ts }\OtherTok{\textless{}{-}}\NormalTok{ preds\_ts[keep]}
\NormalTok{test\_rank }\OtherTok{\textless{}{-}} \FunctionTok{do.call}\NormalTok{(rbind, }\FunctionTok{lapply}\NormalTok{(}\FunctionTok{names}\NormalTok{(preds\_ts), }\ControlFlowTok{function}\NormalTok{(nm)\{}
  \FunctionTok{data.frame}\NormalTok{(}\AttributeTok{Model=}\NormalTok{nm, }\FunctionTok{t}\NormalTok{(}\FunctionTok{metrics\_vec}\NormalTok{(test\_ts}\SpecialCharTok{$}\NormalTok{R3haddock, preds\_ts[[nm]])), }\AttributeTok{row.names =} \ConstantTok{NULL}\NormalTok{)}
\NormalTok{\})) }\SpecialCharTok{\%\textgreater{}\%} \FunctionTok{arrange}\NormalTok{(RMSE, MAE)}
\FunctionTok{cat}\NormalTok{(}\StringTok{"}\SpecialCharTok{\textbackslash{}n}\StringTok{Хронологический тест (последние годы), RMSE/MAE/R2:}\SpecialCharTok{\textbackslash{}n}\StringTok{"}\NormalTok{); }\FunctionTok{print}\NormalTok{(test\_rank)}
\end{Highlighting}
\end{Shaded}

\begin{verbatim}

Хронологический тест (последние годы), RMSE/MAE/R2:
\end{verbatim}

\begin{verbatim}
    Model     RMSE      MAE         R2      MAPE     sMAPE
1  CUBIST 148248.4 107629.3  0.5774780  54.93724  38.06342
2      LM 156940.0 129604.7  0.5264822  97.30313  49.10292
3     GAM 158131.0 125038.0  0.5192677  51.33030  42.12742
4     PLS 176786.2 138249.0  0.3991499 123.21995  50.93728
5     GLM 182047.7 141692.4  0.3628531  73.71724  50.32932
6      RF 185485.8 148117.2  0.3385600  96.29497  52.15688
7     kNN 187966.9 143164.4  0.3207460  71.77551  50.80097
8    ENet 195260.5 161191.7  0.2670102 115.55953  56.27135
9  RANGER 208161.4 171717.0  0.1669526 115.93679  58.76935
10    SVM 250994.6 197801.6 -0.2111500  84.52010  67.59732
11   MARS 273186.2 208579.6 -0.4347849  85.29788  71.46314
12    XGB 288167.4 233693.4 -0.5964639 102.19238  78.99024
13   NNET 345227.3 291463.2 -1.2912878 233.62022 102.82204
\end{verbatim}

\begin{Shaded}
\begin{Highlighting}[]
\CommentTok{\# ==============================================================================}
\CommentTok{\# 4) ПРОГНОЗ 2022–2024 (АНСАМБЛЬ CUBIST+LM) И ГРАФИК 1990–2024 С ДИ}
\CommentTok{\# Прогнозные линии (медиана и ДИ) — пунктир; исторические — сплошные.}
\CommentTok{\# Можно задать свои сценарии предикторов (user\_future); по умолчанию — средние.}
\CommentTok{\# {-}{-}{-}{-}{-}{-}{-}{-}{-}{-}{-}{-}{-}{-}{-}{-}{-}{-}{-}{-}{-}{-}{-}{-}{-}{-}{-}{-}{-}{-}{-}{-}{-}{-}{-}{-}{-}{-}{-}{-}{-}{-}{-}{-}{-}{-}{-}{-}{-}{-}{-}{-}{-}{-}{-}{-}{-}{-}{-}{-}{-}{-}{-}{-}{-}{-}{-}{-}{-}{-}{-}{-}{-}{-}{-}{-}{-}{-}}
\CommentTok{\# Логика ансамбля: комбинируем сильную нелинейную модель (Cubist) с простой и}
\CommentTok{\# устойчивой линейной (LM). Веса можно настраивать. Доверительные интервалы}
\CommentTok{\# получаем эмпирически из распределения остатков (простая и наглядная эвристика).}
\CommentTok{\# ==============================================================================}

\CommentTok{\# 4.1 Полные модели для прогноза (на всех данных) и вес ансамбля}
\NormalTok{model\_data }\OtherTok{\textless{}{-}} \FunctionTok{read.csv}\NormalTok{(}\StringTok{"selected\_predictors\_dataset.csv"}\NormalTok{, }\AttributeTok{header =} \ConstantTok{TRUE}\NormalTok{, }\AttributeTok{stringsAsFactors =} \ConstantTok{FALSE}\NormalTok{)}
\ControlFlowTok{if}\NormalTok{ (}\SpecialCharTok{!}\StringTok{"YEAR"} \SpecialCharTok{\%in\%} \FunctionTok{names}\NormalTok{(model\_data)) \{}
\NormalTok{  model\_data}\SpecialCharTok{$}\NormalTok{YEAR }\OtherTok{\textless{}{-}} \FunctionTok{seq}\NormalTok{(}\DecValTok{1990}\NormalTok{, }\AttributeTok{by =} \DecValTok{1}\NormalTok{, }\AttributeTok{length.out =} \FunctionTok{nrow}\NormalTok{(model\_data))}
\NormalTok{\}}
\NormalTok{model\_data }\OtherTok{\textless{}{-}}\NormalTok{ model\_data }\SpecialCharTok{\%\textgreater{}\%} \FunctionTok{arrange}\NormalTok{(YEAR)}

\NormalTok{cubist\_full }\OtherTok{\textless{}{-}}\NormalTok{ caret}\SpecialCharTok{::}\FunctionTok{train}\NormalTok{(R3haddock }\SpecialCharTok{\textasciitilde{}}\NormalTok{ codTSB }\SpecialCharTok{+}\NormalTok{ T12 }\SpecialCharTok{+}\NormalTok{ I5 }\SpecialCharTok{+}\NormalTok{ NAOspring }\SpecialCharTok{+}\NormalTok{ haddock68,}
                            \AttributeTok{data =}\NormalTok{ model\_data, }\AttributeTok{method =} \StringTok{"cubist"}\NormalTok{,}
                            \AttributeTok{trControl =}\NormalTok{ caret}\SpecialCharTok{::}\FunctionTok{trainControl}\NormalTok{(}\AttributeTok{method=}\StringTok{"none"}\NormalTok{),}
                            \AttributeTok{tuneGrid =} \ControlFlowTok{if}\NormalTok{ (}\FunctionTok{exists}\NormalTok{(}\StringTok{"cubist\_model"}\NormalTok{)) cubist\_model}\SpecialCharTok{$}\NormalTok{bestTune }\ControlFlowTok{else} \ConstantTok{NULL}\NormalTok{,}
                            \AttributeTok{tuneLength =} \ControlFlowTok{if}\NormalTok{ (}\FunctionTok{exists}\NormalTok{(}\StringTok{"cubist\_model"}\NormalTok{)) }\DecValTok{1} \ControlFlowTok{else} \DecValTok{5}\NormalTok{)}

\NormalTok{lm\_full }\OtherTok{\textless{}{-}}\NormalTok{ caret}\SpecialCharTok{::}\FunctionTok{train}\NormalTok{(R3haddock }\SpecialCharTok{\textasciitilde{}}\NormalTok{ codTSB }\SpecialCharTok{+}\NormalTok{ T12 }\SpecialCharTok{+}\NormalTok{ I5 }\SpecialCharTok{+}\NormalTok{ NAOspring }\SpecialCharTok{+}\NormalTok{ haddock68,}
                        \AttributeTok{data =}\NormalTok{ model\_data, }\AttributeTok{method =} \StringTok{"lm"}\NormalTok{,}
                        \AttributeTok{trControl =}\NormalTok{ caret}\SpecialCharTok{::}\FunctionTok{trainControl}\NormalTok{(}\AttributeTok{method=}\StringTok{"none"}\NormalTok{))}

\NormalTok{alpha\_opt }\OtherTok{\textless{}{-}} \ControlFlowTok{if}\NormalTok{ (}\FunctionTok{exists}\NormalTok{(}\StringTok{"alpha\_opt"}\NormalTok{)) alpha\_opt }\ControlFlowTok{else} \FloatTok{0.75}
\NormalTok{predict\_ensemble }\OtherTok{\textless{}{-}} \ControlFlowTok{function}\NormalTok{(newdata, }\AttributeTok{alpha =}\NormalTok{ alpha\_opt) \{}
\NormalTok{  alpha }\SpecialCharTok{*} \FunctionTok{predict}\NormalTok{(cubist\_full, newdata) }\SpecialCharTok{+}\NormalTok{ (}\DecValTok{1} \SpecialCharTok{{-}}\NormalTok{ alpha) }\SpecialCharTok{*} \FunctionTok{predict}\NormalTok{(lm\_full, newdata)}
\NormalTok{\}}

\CommentTok{\# Ансамбль моделей часто дает более точные и устойчивые прогнозы, чем отдельные модели. [[8]]}
\CommentTok{\# В нашем случае:}
\CommentTok{\#   {-} CUBIST: мощная модель, основанная на правилах, хорошо работающая с табличными данными}
\CommentTok{\#   {-} LM: простая интерпретируемая модель, устойчивая к шуму}
\CommentTok{\#   {-} alpha\_opt = 0.75: веса ансамбля (75\% CUBIST, 25\% LM), оптимизированные ранее (см. скрипт "ENS\_WEIGHT.R")}
\CommentTok{\# Комбинирование моделей с разными сильными сторонами снижает риск систематических ошибок.}

\CommentTok{\# 4.2 Остатки для ДИ (из CV, если есть; иначе — по фитам)}
\CommentTok{\# Эмпирические квантилы остатков дают «практические» интервалы прогноза без}
\CommentTok{\# предположения нормальности ошибок (хотя строгий PI требует аккуратности).}
\NormalTok{get\_residuals\_for\_pi }\OtherTok{\textless{}{-}} \ControlFlowTok{function}\NormalTok{() \{}
  \ControlFlowTok{if}\NormalTok{ (}\FunctionTok{exists}\NormalTok{(}\StringTok{"lm\_model"}\NormalTok{) }\SpecialCharTok{\&\&} \FunctionTok{exists}\NormalTok{(}\StringTok{"cubist\_model"}\NormalTok{) }\SpecialCharTok{\&\&}
      \SpecialCharTok{!}\FunctionTok{is.null}\NormalTok{(lm\_model}\SpecialCharTok{$}\NormalTok{pred) }\SpecialCharTok{\&\&} \SpecialCharTok{!}\FunctionTok{is.null}\NormalTok{(cubist\_model}\SpecialCharTok{$}\NormalTok{pred)) \{}
\NormalTok{    pl }\OtherTok{\textless{}{-}}\NormalTok{ lm\_model}\SpecialCharTok{$}\NormalTok{pred }\SpecialCharTok{\%\textgreater{}\%} \FunctionTok{select}\NormalTok{(Resample,rowIndex,obs,}\AttributeTok{p\_lm=}\NormalTok{pred)}
\NormalTok{    pc }\OtherTok{\textless{}{-}}\NormalTok{ cubist\_model}\SpecialCharTok{$}\NormalTok{pred }\SpecialCharTok{\%\textgreater{}\%} \FunctionTok{select}\NormalTok{(Resample,rowIndex,}\AttributeTok{p\_cu=}\NormalTok{pred)}
    \FunctionTok{inner\_join}\NormalTok{(pl, pc, }\AttributeTok{by=}\FunctionTok{c}\NormalTok{(}\StringTok{"Resample"}\NormalTok{,}\StringTok{"rowIndex"}\NormalTok{)) }\SpecialCharTok{\%\textgreater{}\%}
      \FunctionTok{mutate}\NormalTok{(}\AttributeTok{p\_ens =}\NormalTok{ alpha\_opt }\SpecialCharTok{*}\NormalTok{ p\_cu }\SpecialCharTok{+}\NormalTok{ (}\DecValTok{1} \SpecialCharTok{{-}}\NormalTok{ alpha\_opt) }\SpecialCharTok{*}\NormalTok{ p\_lm,}
             \AttributeTok{resid =}\NormalTok{ obs }\SpecialCharTok{{-}}\NormalTok{ p\_ens) }\SpecialCharTok{\%\textgreater{}\%}
      \FunctionTok{pull}\NormalTok{(resid) }\SpecialCharTok{\%\textgreater{}\%}\NormalTok{ .[}\FunctionTok{is.finite}\NormalTok{(.)]}
\NormalTok{  \} }\ControlFlowTok{else}\NormalTok{ \{}
\NormalTok{    model\_data}\SpecialCharTok{$}\NormalTok{R3haddock }\SpecialCharTok{{-}} \FunctionTok{predict\_ensemble}\NormalTok{(model\_data)}
\NormalTok{  \}}
\NormalTok{\}}
\NormalTok{resids }\OtherTok{\textless{}{-}} \FunctionTok{get\_residuals\_for\_pi}\NormalTok{()}
\NormalTok{q025 }\OtherTok{\textless{}{-}} \FunctionTok{as.numeric}\NormalTok{(}\FunctionTok{quantile}\NormalTok{(resids, }\FloatTok{0.025}\NormalTok{, }\AttributeTok{na.rm =} \ConstantTok{TRUE}\NormalTok{))}
\NormalTok{q250 }\OtherTok{\textless{}{-}} \FunctionTok{as.numeric}\NormalTok{(}\FunctionTok{quantile}\NormalTok{(resids, }\FloatTok{0.250}\NormalTok{, }\AttributeTok{na.rm =} \ConstantTok{TRUE}\NormalTok{))}
\NormalTok{q750 }\OtherTok{\textless{}{-}} \FunctionTok{as.numeric}\NormalTok{(}\FunctionTok{quantile}\NormalTok{(resids, }\FloatTok{0.750}\NormalTok{, }\AttributeTok{na.rm =} \ConstantTok{TRUE}\NormalTok{))}
\NormalTok{q975 }\OtherTok{\textless{}{-}} \FunctionTok{as.numeric}\NormalTok{(}\FunctionTok{quantile}\NormalTok{(resids, }\FloatTok{0.975}\NormalTok{, }\AttributeTok{na.rm =} \ConstantTok{TRUE}\NormalTok{))}

\CommentTok{\# Доверительные интервалы (ДИ) показывают неопределенность прогноза.}
\CommentTok{\# Мы используем квантили остатков из кросс{-}валидации для построения ДИ:}
\CommentTok{\#   {-} PI50 (50\% интервал): между 25{-}м и 75{-}м процентилями}
\CommentTok{\#   {-} PI95 (95\% интервал): между 2.5{-}м и 97.5{-}м процентилями}
\CommentTok{\# Это непараметрический подход, не требующий предположений о нормальности ошибок.}

\CommentTok{\# 4.3 Сценарии будущего (по умолчанию — средние; можно переопределить user\_future)}
\NormalTok{fc\_start }\OtherTok{\textless{}{-}} \DecValTok{2022}
\NormalTok{pred\_cols }\OtherTok{\textless{}{-}} \FunctionTok{c}\NormalTok{(}\StringTok{"codTSB"}\NormalTok{,}\StringTok{"T12"}\NormalTok{,}\StringTok{"I5"}\NormalTok{,}\StringTok{"NAOspring"}\NormalTok{,}\StringTok{"haddock68"}\NormalTok{)}
\NormalTok{train\_period }\OtherTok{\textless{}{-}}\NormalTok{ model\_data }\SpecialCharTok{\%\textgreater{}\%} \FunctionTok{filter}\NormalTok{(YEAR }\SpecialCharTok{\textgreater{}} \DecValTok{1989} \SpecialCharTok{\&}\NormalTok{ YEAR }\SpecialCharTok{\textless{}}\NormalTok{ fc\_start)}
\NormalTok{mu }\OtherTok{\textless{}{-}}\NormalTok{ train\_period }\SpecialCharTok{\%\textgreater{}\%} \FunctionTok{summarise}\NormalTok{(}\FunctionTok{across}\NormalTok{(}\FunctionTok{all\_of}\NormalTok{(pred\_cols), }\SpecialCharTok{\textasciitilde{}}\FunctionTok{mean}\NormalTok{(.x, }\AttributeTok{na.rm =} \ConstantTok{TRUE}\NormalTok{))) }\SpecialCharTok{\%\textgreater{}\%} \FunctionTok{as.list}\NormalTok{()}

\CommentTok{\# Пример пользовательского сценария:}
\CommentTok{\# user\_future \textless{}{-} tibble::tribble(}
\CommentTok{\#   \textasciitilde{}YEAR, \textasciitilde{}codTSB, \textasciitilde{}T12, \textasciitilde{}I5, \textasciitilde{}NAOspring, \textasciitilde{}haddock68,}
\CommentTok{\#   2022, 2100000, 5.1, 48,  0.3, 120000,}
\CommentTok{\#   2023, 2050000, 4.8, 50, {-}0.1, 115000,}
\CommentTok{\#   2024, 2150000, 5.0, 47,  0.2, 118000}
\CommentTok{\# )}
\ControlFlowTok{if}\NormalTok{ (}\SpecialCharTok{!}\FunctionTok{exists}\NormalTok{(}\StringTok{"user\_future"}\NormalTok{)) user\_future }\OtherTok{\textless{}{-}} \ConstantTok{NULL}

\NormalTok{build\_future }\OtherTok{\textless{}{-}} \ControlFlowTok{function}\NormalTok{(years, mu, }\AttributeTok{user\_df=}\ConstantTok{NULL}\NormalTok{) \{}
\NormalTok{  df }\OtherTok{\textless{}{-}}\NormalTok{ tibble}\SpecialCharTok{::}\FunctionTok{tibble}\NormalTok{(}\AttributeTok{YEAR =}\NormalTok{ years)}
  \ControlFlowTok{for}\NormalTok{ (v }\ControlFlowTok{in}\NormalTok{ pred\_cols) df[[v]] }\OtherTok{\textless{}{-}}\NormalTok{ mu[[v]]}
  \ControlFlowTok{if}\NormalTok{ (}\SpecialCharTok{!}\FunctionTok{is.null}\NormalTok{(user\_df)) \{}
    \ControlFlowTok{for}\NormalTok{ (i }\ControlFlowTok{in} \FunctionTok{seq\_len}\NormalTok{(}\FunctionTok{nrow}\NormalTok{(user\_df))) \{}
\NormalTok{      yr }\OtherTok{\textless{}{-}}\NormalTok{ user\_df}\SpecialCharTok{$}\NormalTok{YEAR[i]}
      \ControlFlowTok{if}\NormalTok{ (yr }\SpecialCharTok{\%in\%}\NormalTok{ years) \{}
\NormalTok{        idx }\OtherTok{\textless{}{-}} \FunctionTok{which}\NormalTok{(df}\SpecialCharTok{$}\NormalTok{YEAR }\SpecialCharTok{==}\NormalTok{ yr)}
        \ControlFlowTok{for}\NormalTok{ (v }\ControlFlowTok{in} \FunctionTok{intersect}\NormalTok{(pred\_cols, }\FunctionTok{names}\NormalTok{(user\_df))) \{}
\NormalTok{          val }\OtherTok{\textless{}{-}}\NormalTok{ user\_df[[v]][i]}
          \ControlFlowTok{if}\NormalTok{ (}\SpecialCharTok{!}\FunctionTok{is.na}\NormalTok{(val)) df[[v]][idx] }\OtherTok{\textless{}{-}}\NormalTok{ val}
\NormalTok{        \}}
\NormalTok{      \}}
\NormalTok{    \}}
\NormalTok{  \}}
\NormalTok{  df}
\NormalTok{\}}
\NormalTok{future\_years }\OtherTok{\textless{}{-}}\NormalTok{ fc\_start}\SpecialCharTok{:}\DecValTok{2024}
\NormalTok{scenario\_future }\OtherTok{\textless{}{-}} \FunctionTok{build\_future}\NormalTok{(future\_years, mu, user\_future)}

\CommentTok{\# 4.4 Прогноз и таблица ДИ}
\NormalTok{pred\_future }\OtherTok{\textless{}{-}} \FunctionTok{predict\_ensemble}\NormalTok{(scenario\_future)}
\NormalTok{forecast\_tbl }\OtherTok{\textless{}{-}}\NormalTok{ tibble}\SpecialCharTok{::}\FunctionTok{tibble}\NormalTok{(}
  \AttributeTok{YEAR      =}\NormalTok{ scenario\_future}\SpecialCharTok{$}\NormalTok{YEAR,}
  \AttributeTok{pred\_mean =} \FunctionTok{as.numeric}\NormalTok{(pred\_future),}
  \AttributeTok{PI50\_low  =}\NormalTok{ pred\_future }\SpecialCharTok{+}\NormalTok{ q250, }\AttributeTok{PI50\_high =}\NormalTok{ pred\_future }\SpecialCharTok{+}\NormalTok{ q750,}
  \AttributeTok{PI95\_low  =}\NormalTok{ pred\_future }\SpecialCharTok{+}\NormalTok{ q025, }\AttributeTok{PI95\_high =}\NormalTok{ pred\_future }\SpecialCharTok{+}\NormalTok{ q975}
\NormalTok{)}

\DocumentationTok{\#\#\#\# Таблица прогноза 2022–2024}
\FunctionTok{print}\NormalTok{(forecast\_tbl)}
\end{Highlighting}
\end{Shaded}

\begin{verbatim}
# A tibble: 3 x 6
   YEAR pred_mean PI50_low PI50_high PI95_low PI95_high
  <int>     <dbl>    <dbl>     <dbl>    <dbl>     <dbl>
1  2022   253815.   63865.   536150.  -46219.   668189.
2  2023   253815.   63865.   536150.  -46219.   668189.
3  2024   253815.   63865.   536150.  -46219.   668189.
\end{verbatim}

\begin{Shaded}
\begin{Highlighting}[]
\CommentTok{\# По умолчанию мы используем средние значения предикторов для прогноза.}
\CommentTok{\# Однако вы можете определить собственный сценарий (user\_future), указав конкретные значения}
\CommentTok{\# для каждого года и каждого предиктора. Это позволяет моделировать различные экологические сценарии. }

\CommentTok{\# 4.5 Непрерывный ряд 1990–2024 и график: ленты сплошные; линии медианы/ДИ — сплошные до 2021, пунктир с 2022}
\NormalTok{pred\_df }\OtherTok{\textless{}{-}} \FunctionTok{bind\_rows}\NormalTok{(}
\NormalTok{  model\_data }\SpecialCharTok{\%\textgreater{}\%} \FunctionTok{select}\NormalTok{(YEAR, }\FunctionTok{all\_of}\NormalTok{(pred\_cols)),}
\NormalTok{  scenario\_future}
\NormalTok{) }\SpecialCharTok{\%\textgreater{}\%} \FunctionTok{distinct}\NormalTok{(YEAR, }\AttributeTok{.keep\_all =} \ConstantTok{TRUE}\NormalTok{) }\SpecialCharTok{\%\textgreater{}\%} \FunctionTok{arrange}\NormalTok{(YEAR)}

\NormalTok{pred\_df}\SpecialCharTok{$}\NormalTok{Pred      }\OtherTok{\textless{}{-}} \FunctionTok{as.numeric}\NormalTok{(}\FunctionTok{predict\_ensemble}\NormalTok{(pred\_df))}
\NormalTok{pred\_df}\SpecialCharTok{$}\NormalTok{PI50\_low  }\OtherTok{\textless{}{-}}\NormalTok{ pred\_df}\SpecialCharTok{$}\NormalTok{Pred }\SpecialCharTok{+}\NormalTok{ q250}
\NormalTok{pred\_df}\SpecialCharTok{$}\NormalTok{PI50\_high }\OtherTok{\textless{}{-}}\NormalTok{ pred\_df}\SpecialCharTok{$}\NormalTok{Pred }\SpecialCharTok{+}\NormalTok{ q750}
\NormalTok{pred\_df}\SpecialCharTok{$}\NormalTok{PI95\_low  }\OtherTok{\textless{}{-}}\NormalTok{ pred\_df}\SpecialCharTok{$}\NormalTok{Pred }\SpecialCharTok{+}\NormalTok{ q025}
\NormalTok{pred\_df}\SpecialCharTok{$}\NormalTok{PI95\_high }\OtherTok{\textless{}{-}}\NormalTok{ pred\_df}\SpecialCharTok{$}\NormalTok{Pred }\SpecialCharTok{+}\NormalTok{ q975}

\NormalTok{hist\_df }\OtherTok{\textless{}{-}}\NormalTok{ model\_data }\SpecialCharTok{\%\textgreater{}\%} \FunctionTok{select}\NormalTok{(YEAR, R3haddock)}

\FunctionTok{ggplot}\NormalTok{() }\SpecialCharTok{+}
  \FunctionTok{geom\_ribbon}\NormalTok{(}\AttributeTok{data =}\NormalTok{ pred\_df, }\FunctionTok{aes}\NormalTok{(}\AttributeTok{x =}\NormalTok{ YEAR, }\AttributeTok{ymin =}\NormalTok{ PI95\_low, }\AttributeTok{ymax =}\NormalTok{ PI95\_high),}
              \AttributeTok{fill =} \StringTok{"grey80"}\NormalTok{, }\AttributeTok{alpha =} \FloatTok{0.25}\NormalTok{) }\SpecialCharTok{+}
  \FunctionTok{geom\_ribbon}\NormalTok{(}\AttributeTok{data =}\NormalTok{ pred\_df, }\FunctionTok{aes}\NormalTok{(}\AttributeTok{x =}\NormalTok{ YEAR, }\AttributeTok{ymin =}\NormalTok{ PI50\_low, }\AttributeTok{ymax =}\NormalTok{ PI50\_high),}
              \AttributeTok{fill =} \StringTok{"grey60"}\NormalTok{, }\AttributeTok{alpha =} \FloatTok{0.35}\NormalTok{) }\SpecialCharTok{+}
  \FunctionTok{geom\_line}\NormalTok{(}\AttributeTok{data =} \FunctionTok{subset}\NormalTok{(pred\_df, YEAR }\SpecialCharTok{\textless{}}\NormalTok{ fc\_start), }\FunctionTok{aes}\NormalTok{(}\AttributeTok{x =}\NormalTok{ YEAR, }\AttributeTok{y =}\NormalTok{ PI95\_low),}
            \AttributeTok{color =} \StringTok{"grey45"}\NormalTok{, }\AttributeTok{linewidth =} \FloatTok{0.6}\NormalTok{) }\SpecialCharTok{+}
  \FunctionTok{geom\_line}\NormalTok{(}\AttributeTok{data =} \FunctionTok{subset}\NormalTok{(pred\_df, YEAR }\SpecialCharTok{\textless{}}\NormalTok{ fc\_start), }\FunctionTok{aes}\NormalTok{(}\AttributeTok{x =}\NormalTok{ YEAR, }\AttributeTok{y =}\NormalTok{ PI95\_high),}
            \AttributeTok{color =} \StringTok{"grey45"}\NormalTok{, }\AttributeTok{linewidth =} \FloatTok{0.6}\NormalTok{) }\SpecialCharTok{+}
  \FunctionTok{geom\_line}\NormalTok{(}\AttributeTok{data =} \FunctionTok{subset}\NormalTok{(pred\_df, YEAR }\SpecialCharTok{\textless{}}\NormalTok{ fc\_start), }\FunctionTok{aes}\NormalTok{(}\AttributeTok{x =}\NormalTok{ YEAR, }\AttributeTok{y =}\NormalTok{ PI50\_low),}
            \AttributeTok{color =} \StringTok{"grey35"}\NormalTok{, }\AttributeTok{linewidth =} \FloatTok{0.6}\NormalTok{) }\SpecialCharTok{+}
  \FunctionTok{geom\_line}\NormalTok{(}\AttributeTok{data =} \FunctionTok{subset}\NormalTok{(pred\_df, YEAR }\SpecialCharTok{\textless{}}\NormalTok{ fc\_start), }\FunctionTok{aes}\NormalTok{(}\AttributeTok{x =}\NormalTok{ YEAR, }\AttributeTok{y =}\NormalTok{ PI50\_high),}
            \AttributeTok{color =} \StringTok{"grey35"}\NormalTok{, }\AttributeTok{linewidth =} \FloatTok{0.6}\NormalTok{) }\SpecialCharTok{+}
  \FunctionTok{geom\_line}\NormalTok{(}\AttributeTok{data =} \FunctionTok{subset}\NormalTok{(pred\_df, YEAR }\SpecialCharTok{\textgreater{}=}\NormalTok{ fc\_start}\DecValTok{{-}1}\NormalTok{), }\FunctionTok{aes}\NormalTok{(}\AttributeTok{x =}\NormalTok{ YEAR, }\AttributeTok{y =}\NormalTok{ PI95\_low),}
            \AttributeTok{color =} \StringTok{"grey45"}\NormalTok{, }\AttributeTok{linewidth =} \FloatTok{0.6}\NormalTok{, }\AttributeTok{linetype =} \StringTok{"dashed"}\NormalTok{) }\SpecialCharTok{+}
  \FunctionTok{geom\_line}\NormalTok{(}\AttributeTok{data =} \FunctionTok{subset}\NormalTok{(pred\_df, YEAR }\SpecialCharTok{\textgreater{}=}\NormalTok{ fc\_start}\DecValTok{{-}1}\NormalTok{), }\FunctionTok{aes}\NormalTok{(}\AttributeTok{x =}\NormalTok{ YEAR, }\AttributeTok{y =}\NormalTok{ PI95\_high),}
            \AttributeTok{color =} \StringTok{"grey45"}\NormalTok{, }\AttributeTok{linewidth =} \FloatTok{0.6}\NormalTok{, }\AttributeTok{linetype =} \StringTok{"dashed"}\NormalTok{) }\SpecialCharTok{+}
  \FunctionTok{geom\_line}\NormalTok{(}\AttributeTok{data =} \FunctionTok{subset}\NormalTok{(pred\_df, YEAR }\SpecialCharTok{\textgreater{}=}\NormalTok{ fc\_start}\DecValTok{{-}1}\NormalTok{), }\FunctionTok{aes}\NormalTok{(}\AttributeTok{x =}\NormalTok{ YEAR, }\AttributeTok{y =}\NormalTok{ PI50\_low),}
            \AttributeTok{color =} \StringTok{"grey35"}\NormalTok{, }\AttributeTok{linewidth =} \FloatTok{0.6}\NormalTok{, }\AttributeTok{linetype =} \StringTok{"dashed"}\NormalTok{) }\SpecialCharTok{+}
  \FunctionTok{geom\_line}\NormalTok{(}\AttributeTok{data =} \FunctionTok{subset}\NormalTok{(pred\_df, YEAR }\SpecialCharTok{\textgreater{}=}\NormalTok{ fc\_start}\DecValTok{{-}1}\NormalTok{), }\FunctionTok{aes}\NormalTok{(}\AttributeTok{x =}\NormalTok{ YEAR, }\AttributeTok{y =}\NormalTok{ PI50\_high),}
            \AttributeTok{color =} \StringTok{"grey35"}\NormalTok{, }\AttributeTok{linewidth =} \FloatTok{0.6}\NormalTok{, }\AttributeTok{linetype =} \StringTok{"dashed"}\NormalTok{) }\SpecialCharTok{+}
  \FunctionTok{geom\_line}\NormalTok{(}\AttributeTok{data =} \FunctionTok{subset}\NormalTok{(pred\_df, YEAR }\SpecialCharTok{\textless{}}\NormalTok{ fc\_start), }\FunctionTok{aes}\NormalTok{(}\AttributeTok{x =}\NormalTok{ YEAR, }\AttributeTok{y =}\NormalTok{ Pred),}
            \AttributeTok{color =} \StringTok{"steelblue4"}\NormalTok{, }\AttributeTok{linewidth =} \DecValTok{1}\NormalTok{) }\SpecialCharTok{+}
  \FunctionTok{geom\_line}\NormalTok{(}\AttributeTok{data =} \FunctionTok{subset}\NormalTok{(pred\_df, YEAR }\SpecialCharTok{\textgreater{}=}\NormalTok{ fc\_start}\DecValTok{{-}1}\NormalTok{), }\FunctionTok{aes}\NormalTok{(}\AttributeTok{x =}\NormalTok{ YEAR, }\AttributeTok{y =}\NormalTok{ Pred),}
            \AttributeTok{color =} \StringTok{"steelblue4"}\NormalTok{, }\AttributeTok{linewidth =} \DecValTok{1}\NormalTok{, }\AttributeTok{linetype =} \StringTok{"dashed"}\NormalTok{) }\SpecialCharTok{+}
  \FunctionTok{geom\_point}\NormalTok{(}\AttributeTok{data =}\NormalTok{ hist\_df, }\FunctionTok{aes}\NormalTok{(}\AttributeTok{x =}\NormalTok{ YEAR, }\AttributeTok{y =}\NormalTok{ R3haddock),}
             \AttributeTok{color =} \StringTok{"black"}\NormalTok{, }\AttributeTok{size =} \DecValTok{2}\NormalTok{, }\AttributeTok{alpha =} \FloatTok{0.9}\NormalTok{) }\SpecialCharTok{+}
  \FunctionTok{scale\_x\_continuous}\NormalTok{(}\AttributeTok{expand =} \FunctionTok{expansion}\NormalTok{(}\AttributeTok{mult =} \FunctionTok{c}\NormalTok{(}\DecValTok{0}\NormalTok{, }\DecValTok{0}\NormalTok{))) }\SpecialCharTok{+}
  \FunctionTok{labs}\NormalTok{(}
    \AttributeTok{title =} \StringTok{"Пополнение R3haddock: факт (1990–2021) и прогноз (2022–2024)}\SpecialCharTok{\textbackslash{}n}\StringTok{Ансамбль CUBIST+LM; непрерывные ДИ, прогноз — пунктир"}\NormalTok{,}
    \AttributeTok{x =} \StringTok{"Год"}\NormalTok{, }\AttributeTok{y =} \StringTok{"R3haddock"}
\NormalTok{  ) }\SpecialCharTok{+}
  \FunctionTok{theme\_minimal}\NormalTok{(}\AttributeTok{base\_size =} \DecValTok{12}\NormalTok{) }\SpecialCharTok{+}
  \FunctionTok{theme}\NormalTok{(}\AttributeTok{legend.position =} \StringTok{"none"}\NormalTok{)}
\end{Highlighting}
\end{Shaded}

\begin{verbatim}
Warning in grid.Call(C_textBounds, as.graphicsAnnot(x$label), x$x, x$y, :
неизвестна ширина символа 0xcf в кодировке CP1251
\end{verbatim}

\begin{verbatim}
Warning in grid.Call(C_textBounds, as.graphicsAnnot(x$label), x$x, x$y, :
неизвестна ширина символа 0xee в кодировке CP1251
\end{verbatim}

\begin{verbatim}
Warning in grid.Call(C_textBounds, as.graphicsAnnot(x$label), x$x, x$y, :
неизвестна ширина символа 0xef в кодировке CP1251
\end{verbatim}

\begin{verbatim}
Warning in grid.Call(C_textBounds, as.graphicsAnnot(x$label), x$x, x$y, :
неизвестна ширина символа 0xee в кодировке CP1251
\end{verbatim}

\begin{verbatim}
Warning in grid.Call(C_textBounds, as.graphicsAnnot(x$label), x$x, x$y, :
неизвестна ширина символа 0xeb в кодировке CP1251
\end{verbatim}

\begin{verbatim}
Warning in grid.Call(C_textBounds, as.graphicsAnnot(x$label), x$x, x$y, :
неизвестна ширина символа 0xed в кодировке CP1251
\end{verbatim}

\begin{verbatim}
Warning in grid.Call(C_textBounds, as.graphicsAnnot(x$label), x$x, x$y, :
неизвестна ширина символа 0xe5 в кодировке CP1251
\end{verbatim}

\begin{verbatim}
Warning in grid.Call(C_textBounds, as.graphicsAnnot(x$label), x$x, x$y, :
неизвестна ширина символа 0xed в кодировке CP1251
\end{verbatim}

\begin{verbatim}
Warning in grid.Call(C_textBounds, as.graphicsAnnot(x$label), x$x, x$y, :
неизвестна ширина символа 0xe8 в кодировке CP1251
\end{verbatim}

\begin{verbatim}
Warning in grid.Call(C_textBounds, as.graphicsAnnot(x$label), x$x, x$y, :
неизвестна ширина символа 0xe5 в кодировке CP1251
\end{verbatim}

\begin{verbatim}
Warning in grid.Call(C_textBounds, as.graphicsAnnot(x$label), x$x, x$y, :
неизвестна ширина символа 0xf4 в кодировке CP1251
\end{verbatim}

\begin{verbatim}
Warning in grid.Call(C_textBounds, as.graphicsAnnot(x$label), x$x, x$y, :
неизвестна ширина символа 0xe0 в кодировке CP1251
\end{verbatim}

\begin{verbatim}
Warning in grid.Call(C_textBounds, as.graphicsAnnot(x$label), x$x, x$y, :
неизвестна ширина символа 0xea в кодировке CP1251
\end{verbatim}

\begin{verbatim}
Warning in grid.Call(C_textBounds, as.graphicsAnnot(x$label), x$x, x$y, :
неизвестна ширина символа 0xf2 в кодировке CP1251
\end{verbatim}

\begin{verbatim}
Warning in grid.Call(C_textBounds, as.graphicsAnnot(x$label), x$x, x$y, :
неизвестна ширина символа 0xe8 в кодировке CP1251
\end{verbatim}

\begin{verbatim}
Warning in grid.Call(C_textBounds, as.graphicsAnnot(x$label), x$x, x$y, :
неизвестна ширина символа 0xef в кодировке CP1251
\end{verbatim}

\begin{verbatim}
Warning in grid.Call(C_textBounds, as.graphicsAnnot(x$label), x$x, x$y, :
неизвестна ширина символа 0xf0 в кодировке CP1251
\end{verbatim}

\begin{verbatim}
Warning in grid.Call(C_textBounds, as.graphicsAnnot(x$label), x$x, x$y, :
неизвестна ширина символа 0xee в кодировке CP1251
\end{verbatim}

\begin{verbatim}
Warning in grid.Call(C_textBounds, as.graphicsAnnot(x$label), x$x, x$y, :
неизвестна ширина символа 0xe3 в кодировке CP1251
\end{verbatim}

\begin{verbatim}
Warning in grid.Call(C_textBounds, as.graphicsAnnot(x$label), x$x, x$y, :
неизвестна ширина символа 0xed в кодировке CP1251
\end{verbatim}

\begin{verbatim}
Warning in grid.Call(C_textBounds, as.graphicsAnnot(x$label), x$x, x$y, :
неизвестна ширина символа 0xee в кодировке CP1251
\end{verbatim}

\begin{verbatim}
Warning in grid.Call(C_textBounds, as.graphicsAnnot(x$label), x$x, x$y, :
неизвестна ширина символа 0xe7 в кодировке CP1251
\end{verbatim}

\begin{verbatim}
Warning in grid.Call(C_textBounds, as.graphicsAnnot(x$label), x$x, x$y, :
неизвестна ширина символа 0xc0 в кодировке CP1251
\end{verbatim}

\begin{verbatim}
Warning in grid.Call(C_textBounds, as.graphicsAnnot(x$label), x$x, x$y, :
неизвестна ширина символа 0xed в кодировке CP1251
\end{verbatim}

\begin{verbatim}
Warning in grid.Call(C_textBounds, as.graphicsAnnot(x$label), x$x, x$y, :
неизвестна ширина символа 0xf1 в кодировке CP1251
\end{verbatim}

\begin{verbatim}
Warning in grid.Call(C_textBounds, as.graphicsAnnot(x$label), x$x, x$y, :
неизвестна ширина символа 0xe0 в кодировке CP1251
\end{verbatim}

\begin{verbatim}
Warning in grid.Call(C_textBounds, as.graphicsAnnot(x$label), x$x, x$y, :
неизвестна ширина символа 0xec в кодировке CP1251
\end{verbatim}

\begin{verbatim}
Warning in grid.Call(C_textBounds, as.graphicsAnnot(x$label), x$x, x$y, :
неизвестна ширина символа 0xe1 в кодировке CP1251
\end{verbatim}

\begin{verbatim}
Warning in grid.Call(C_textBounds, as.graphicsAnnot(x$label), x$x, x$y, :
неизвестна ширина символа 0xeb в кодировке CP1251
\end{verbatim}

\begin{verbatim}
Warning in grid.Call(C_textBounds, as.graphicsAnnot(x$label), x$x, x$y, :
неизвестна ширина символа 0xfc в кодировке CP1251
\end{verbatim}

\begin{verbatim}
Warning in grid.Call(C_textBounds, as.graphicsAnnot(x$label), x$x, x$y, :
неизвестна ширина символа 0xed в кодировке CP1251
\end{verbatim}

\begin{verbatim}
Warning in grid.Call(C_textBounds, as.graphicsAnnot(x$label), x$x, x$y, :
неизвестна ширина символа 0xe5 в кодировке CP1251
\end{verbatim}

\begin{verbatim}
Warning in grid.Call(C_textBounds, as.graphicsAnnot(x$label), x$x, x$y, :
неизвестна ширина символа 0xef в кодировке CP1251
\end{verbatim}

\begin{verbatim}
Warning in grid.Call(C_textBounds, as.graphicsAnnot(x$label), x$x, x$y, :
неизвестна ширина символа 0xf0 в кодировке CP1251
\end{verbatim}

\begin{verbatim}
Warning in grid.Call(C_textBounds, as.graphicsAnnot(x$label), x$x, x$y, :
неизвестна ширина символа 0xe5 в кодировке CP1251
\end{verbatim}

\begin{verbatim}
Warning in grid.Call(C_textBounds, as.graphicsAnnot(x$label), x$x, x$y, :
неизвестна ширина символа 0xf0 в кодировке CP1251
\end{verbatim}

\begin{verbatim}
Warning in grid.Call(C_textBounds, as.graphicsAnnot(x$label), x$x, x$y, :
неизвестна ширина символа 0xfb в кодировке CP1251
\end{verbatim}

\begin{verbatim}
Warning in grid.Call(C_textBounds, as.graphicsAnnot(x$label), x$x, x$y, :
неизвестна ширина символа 0xe2 в кодировке CP1251
\end{verbatim}

\begin{verbatim}
Warning in grid.Call(C_textBounds, as.graphicsAnnot(x$label), x$x, x$y, :
неизвестна ширина символа 0xed в кодировке CP1251
\end{verbatim}

\begin{verbatim}
Warning in grid.Call(C_textBounds, as.graphicsAnnot(x$label), x$x, x$y, :
неизвестна ширина символа 0xfb в кодировке CP1251
\end{verbatim}

\begin{verbatim}
Warning in grid.Call(C_textBounds, as.graphicsAnnot(x$label), x$x, x$y, :
неизвестна ширина символа 0xe5 в кодировке CP1251
\end{verbatim}

\begin{verbatim}
Warning in grid.Call(C_textBounds, as.graphicsAnnot(x$label), x$x, x$y, :
неизвестна ширина символа 0xc4 в кодировке CP1251
\end{verbatim}

\begin{verbatim}
Warning in grid.Call(C_textBounds, as.graphicsAnnot(x$label), x$x, x$y, :
неизвестна ширина символа 0xc8 в кодировке CP1251
\end{verbatim}

\begin{verbatim}
Warning in grid.Call(C_textBounds, as.graphicsAnnot(x$label), x$x, x$y, :
неизвестна ширина символа 0xef в кодировке CP1251
\end{verbatim}

\begin{verbatim}
Warning in grid.Call(C_textBounds, as.graphicsAnnot(x$label), x$x, x$y, :
неизвестна ширина символа 0xf0 в кодировке CP1251
\end{verbatim}

\begin{verbatim}
Warning in grid.Call(C_textBounds, as.graphicsAnnot(x$label), x$x, x$y, :
неизвестна ширина символа 0xee в кодировке CP1251
\end{verbatim}

\begin{verbatim}
Warning in grid.Call(C_textBounds, as.graphicsAnnot(x$label), x$x, x$y, :
неизвестна ширина символа 0xe3 в кодировке CP1251
\end{verbatim}

\begin{verbatim}
Warning in grid.Call(C_textBounds, as.graphicsAnnot(x$label), x$x, x$y, :
неизвестна ширина символа 0xed в кодировке CP1251
\end{verbatim}

\begin{verbatim}
Warning in grid.Call(C_textBounds, as.graphicsAnnot(x$label), x$x, x$y, :
неизвестна ширина символа 0xee в кодировке CP1251
\end{verbatim}

\begin{verbatim}
Warning in grid.Call(C_textBounds, as.graphicsAnnot(x$label), x$x, x$y, :
неизвестна ширина символа 0xe7 в кодировке CP1251
\end{verbatim}

\begin{verbatim}
Warning in grid.Call(C_textBounds, as.graphicsAnnot(x$label), x$x, x$y, :
неизвестна ширина символа 0xef в кодировке CP1251
\end{verbatim}

\begin{verbatim}
Warning in grid.Call(C_textBounds, as.graphicsAnnot(x$label), x$x, x$y, :
неизвестна ширина символа 0xf3 в кодировке CP1251
\end{verbatim}

\begin{verbatim}
Warning in grid.Call(C_textBounds, as.graphicsAnnot(x$label), x$x, x$y, :
неизвестна ширина символа 0xed в кодировке CP1251
\end{verbatim}

\begin{verbatim}
Warning in grid.Call(C_textBounds, as.graphicsAnnot(x$label), x$x, x$y, :
неизвестна ширина символа 0xea в кодировке CP1251
\end{verbatim}

\begin{verbatim}
Warning in grid.Call(C_textBounds, as.graphicsAnnot(x$label), x$x, x$y, :
неизвестна ширина символа 0xf2 в кодировке CP1251
\end{verbatim}

\begin{verbatim}
Warning in grid.Call(C_textBounds, as.graphicsAnnot(x$label), x$x, x$y, :
неизвестна ширина символа 0xe8 в кодировке CP1251
\end{verbatim}

\begin{verbatim}
Warning in grid.Call(C_textBounds, as.graphicsAnnot(x$label), x$x, x$y, :
неизвестна ширина символа 0xf0 в кодировке CP1251
\end{verbatim}

\begin{verbatim}
Warning in grid.Call(C_textBounds, as.graphicsAnnot(x$label), x$x, x$y, :
неизвестна ширина символа 0xc3 в кодировке CP1251
\end{verbatim}

\begin{verbatim}
Warning in grid.Call(C_textBounds, as.graphicsAnnot(x$label), x$x, x$y, :
неизвестна ширина символа 0xee в кодировке CP1251
\end{verbatim}

\begin{verbatim}
Warning in grid.Call(C_textBounds, as.graphicsAnnot(x$label), x$x, x$y, :
неизвестна ширина символа 0xe4 в кодировке CP1251
\end{verbatim}

\begin{verbatim}
Warning in grid.Call.graphics(C_text, as.graphicsAnnot(x$label), x$x, x$y, :
неизвестна ширина символа 0xc3 в кодировке CP1251
\end{verbatim}

\begin{verbatim}
Warning in grid.Call.graphics(C_text, as.graphicsAnnot(x$label), x$x, x$y, :
неизвестна ширина символа 0xee в кодировке CP1251
\end{verbatim}

\begin{verbatim}
Warning in grid.Call.graphics(C_text, as.graphicsAnnot(x$label), x$x, x$y, :
неизвестна ширина символа 0xe4 в кодировке CP1251
\end{verbatim}

\begin{verbatim}
Warning in grid.Call.graphics(C_text, as.graphicsAnnot(x$label), x$x, x$y, :
неизвестна ширина символа 0xcf в кодировке CP1251
\end{verbatim}

\begin{verbatim}
Warning in grid.Call.graphics(C_text, as.graphicsAnnot(x$label), x$x, x$y, :
неизвестна ширина символа 0xee в кодировке CP1251
\end{verbatim}

\begin{verbatim}
Warning in grid.Call.graphics(C_text, as.graphicsAnnot(x$label), x$x, x$y, :
неизвестна ширина символа 0xef в кодировке CP1251
\end{verbatim}

\begin{verbatim}
Warning in grid.Call.graphics(C_text, as.graphicsAnnot(x$label), x$x, x$y, :
неизвестна ширина символа 0xee в кодировке CP1251
\end{verbatim}

\begin{verbatim}
Warning in grid.Call.graphics(C_text, as.graphicsAnnot(x$label), x$x, x$y, :
неизвестна ширина символа 0xeb в кодировке CP1251
\end{verbatim}

\begin{verbatim}
Warning in grid.Call.graphics(C_text, as.graphicsAnnot(x$label), x$x, x$y, :
неизвестна ширина символа 0xed в кодировке CP1251
\end{verbatim}

\begin{verbatim}
Warning in grid.Call.graphics(C_text, as.graphicsAnnot(x$label), x$x, x$y, :
неизвестна ширина символа 0xe5 в кодировке CP1251
\end{verbatim}

\begin{verbatim}
Warning in grid.Call.graphics(C_text, as.graphicsAnnot(x$label), x$x, x$y, :
неизвестна ширина символа 0xed в кодировке CP1251
\end{verbatim}

\begin{verbatim}
Warning in grid.Call.graphics(C_text, as.graphicsAnnot(x$label), x$x, x$y, :
неизвестна ширина символа 0xe8 в кодировке CP1251
\end{verbatim}

\begin{verbatim}
Warning in grid.Call.graphics(C_text, as.graphicsAnnot(x$label), x$x, x$y, :
неизвестна ширина символа 0xe5 в кодировке CP1251
\end{verbatim}

\begin{verbatim}
Warning in grid.Call.graphics(C_text, as.graphicsAnnot(x$label), x$x, x$y, :
неизвестна ширина символа 0xf4 в кодировке CP1251
\end{verbatim}

\begin{verbatim}
Warning in grid.Call.graphics(C_text, as.graphicsAnnot(x$label), x$x, x$y, :
неизвестна ширина символа 0xe0 в кодировке CP1251
\end{verbatim}

\begin{verbatim}
Warning in grid.Call.graphics(C_text, as.graphicsAnnot(x$label), x$x, x$y, :
неизвестна ширина символа 0xea в кодировке CP1251
\end{verbatim}

\begin{verbatim}
Warning in grid.Call.graphics(C_text, as.graphicsAnnot(x$label), x$x, x$y, :
неизвестна ширина символа 0xf2 в кодировке CP1251
\end{verbatim}

\begin{verbatim}
Warning in grid.Call.graphics(C_text, as.graphicsAnnot(x$label), x$x, x$y, :
неизвестна ширина символа 0xe8 в кодировке CP1251
\end{verbatim}

\begin{verbatim}
Warning in grid.Call.graphics(C_text, as.graphicsAnnot(x$label), x$x, x$y, :
неизвестна ширина символа 0xef в кодировке CP1251
\end{verbatim}

\begin{verbatim}
Warning in grid.Call.graphics(C_text, as.graphicsAnnot(x$label), x$x, x$y, :
неизвестна ширина символа 0xf0 в кодировке CP1251
\end{verbatim}

\begin{verbatim}
Warning in grid.Call.graphics(C_text, as.graphicsAnnot(x$label), x$x, x$y, :
неизвестна ширина символа 0xee в кодировке CP1251
\end{verbatim}

\begin{verbatim}
Warning in grid.Call.graphics(C_text, as.graphicsAnnot(x$label), x$x, x$y, :
неизвестна ширина символа 0xe3 в кодировке CP1251
\end{verbatim}

\begin{verbatim}
Warning in grid.Call.graphics(C_text, as.graphicsAnnot(x$label), x$x, x$y, :
неизвестна ширина символа 0xed в кодировке CP1251
\end{verbatim}

\begin{verbatim}
Warning in grid.Call.graphics(C_text, as.graphicsAnnot(x$label), x$x, x$y, :
неизвестна ширина символа 0xee в кодировке CP1251
\end{verbatim}

\begin{verbatim}
Warning in grid.Call.graphics(C_text, as.graphicsAnnot(x$label), x$x, x$y, :
неизвестна ширина символа 0xe7 в кодировке CP1251
\end{verbatim}

\begin{verbatim}
Warning in grid.Call.graphics(C_text, as.graphicsAnnot(x$label), x$x, x$y, :
неизвестна ширина символа 0xc0 в кодировке CP1251
\end{verbatim}

\begin{verbatim}
Warning in grid.Call.graphics(C_text, as.graphicsAnnot(x$label), x$x, x$y, :
неизвестна ширина символа 0xed в кодировке CP1251
\end{verbatim}

\begin{verbatim}
Warning in grid.Call.graphics(C_text, as.graphicsAnnot(x$label), x$x, x$y, :
неизвестна ширина символа 0xf1 в кодировке CP1251
\end{verbatim}

\begin{verbatim}
Warning in grid.Call.graphics(C_text, as.graphicsAnnot(x$label), x$x, x$y, :
неизвестна ширина символа 0xe0 в кодировке CP1251
\end{verbatim}

\begin{verbatim}
Warning in grid.Call.graphics(C_text, as.graphicsAnnot(x$label), x$x, x$y, :
неизвестна ширина символа 0xec в кодировке CP1251
\end{verbatim}

\begin{verbatim}
Warning in grid.Call.graphics(C_text, as.graphicsAnnot(x$label), x$x, x$y, :
неизвестна ширина символа 0xe1 в кодировке CP1251
\end{verbatim}

\begin{verbatim}
Warning in grid.Call.graphics(C_text, as.graphicsAnnot(x$label), x$x, x$y, :
неизвестна ширина символа 0xeb в кодировке CP1251
\end{verbatim}

\begin{verbatim}
Warning in grid.Call.graphics(C_text, as.graphicsAnnot(x$label), x$x, x$y, :
неизвестна ширина символа 0xfc в кодировке CP1251
\end{verbatim}

\begin{verbatim}
Warning in grid.Call.graphics(C_text, as.graphicsAnnot(x$label), x$x, x$y, :
неизвестна ширина символа 0xed в кодировке CP1251
\end{verbatim}

\begin{verbatim}
Warning in grid.Call.graphics(C_text, as.graphicsAnnot(x$label), x$x, x$y, :
неизвестна ширина символа 0xe5 в кодировке CP1251
\end{verbatim}

\begin{verbatim}
Warning in grid.Call.graphics(C_text, as.graphicsAnnot(x$label), x$x, x$y, :
неизвестна ширина символа 0xef в кодировке CP1251
\end{verbatim}

\begin{verbatim}
Warning in grid.Call.graphics(C_text, as.graphicsAnnot(x$label), x$x, x$y, :
неизвестна ширина символа 0xf0 в кодировке CP1251
\end{verbatim}

\begin{verbatim}
Warning in grid.Call.graphics(C_text, as.graphicsAnnot(x$label), x$x, x$y, :
неизвестна ширина символа 0xe5 в кодировке CP1251
\end{verbatim}

\begin{verbatim}
Warning in grid.Call.graphics(C_text, as.graphicsAnnot(x$label), x$x, x$y, :
неизвестна ширина символа 0xf0 в кодировке CP1251
\end{verbatim}

\begin{verbatim}
Warning in grid.Call.graphics(C_text, as.graphicsAnnot(x$label), x$x, x$y, :
неизвестна ширина символа 0xfb в кодировке CP1251
\end{verbatim}

\begin{verbatim}
Warning in grid.Call.graphics(C_text, as.graphicsAnnot(x$label), x$x, x$y, :
неизвестна ширина символа 0xe2 в кодировке CP1251
\end{verbatim}

\begin{verbatim}
Warning in grid.Call.graphics(C_text, as.graphicsAnnot(x$label), x$x, x$y, :
неизвестна ширина символа 0xed в кодировке CP1251
\end{verbatim}

\begin{verbatim}
Warning in grid.Call.graphics(C_text, as.graphicsAnnot(x$label), x$x, x$y, :
неизвестна ширина символа 0xfb в кодировке CP1251
\end{verbatim}

\begin{verbatim}
Warning in grid.Call.graphics(C_text, as.graphicsAnnot(x$label), x$x, x$y, :
неизвестна ширина символа 0xe5 в кодировке CP1251
\end{verbatim}

\begin{verbatim}
Warning in grid.Call.graphics(C_text, as.graphicsAnnot(x$label), x$x, x$y, :
неизвестна ширина символа 0xc4 в кодировке CP1251
\end{verbatim}

\begin{verbatim}
Warning in grid.Call.graphics(C_text, as.graphicsAnnot(x$label), x$x, x$y, :
неизвестна ширина символа 0xc8 в кодировке CP1251
\end{verbatim}

\begin{verbatim}
Warning in grid.Call.graphics(C_text, as.graphicsAnnot(x$label), x$x, x$y, :
неизвестна ширина символа 0xef в кодировке CP1251
\end{verbatim}

\begin{verbatim}
Warning in grid.Call.graphics(C_text, as.graphicsAnnot(x$label), x$x, x$y, :
неизвестна ширина символа 0xf0 в кодировке CP1251
\end{verbatim}

\begin{verbatim}
Warning in grid.Call.graphics(C_text, as.graphicsAnnot(x$label), x$x, x$y, :
неизвестна ширина символа 0xee в кодировке CP1251
\end{verbatim}

\begin{verbatim}
Warning in grid.Call.graphics(C_text, as.graphicsAnnot(x$label), x$x, x$y, :
неизвестна ширина символа 0xe3 в кодировке CP1251
\end{verbatim}

\begin{verbatim}
Warning in grid.Call.graphics(C_text, as.graphicsAnnot(x$label), x$x, x$y, :
неизвестна ширина символа 0xed в кодировке CP1251
\end{verbatim}

\begin{verbatim}
Warning in grid.Call.graphics(C_text, as.graphicsAnnot(x$label), x$x, x$y, :
неизвестна ширина символа 0xee в кодировке CP1251
\end{verbatim}

\begin{verbatim}
Warning in grid.Call.graphics(C_text, as.graphicsAnnot(x$label), x$x, x$y, :
неизвестна ширина символа 0xe7 в кодировке CP1251
\end{verbatim}

\begin{verbatim}
Warning in grid.Call.graphics(C_text, as.graphicsAnnot(x$label), x$x, x$y, :
неизвестна ширина символа 0xef в кодировке CP1251
\end{verbatim}

\begin{verbatim}
Warning in grid.Call.graphics(C_text, as.graphicsAnnot(x$label), x$x, x$y, :
неизвестна ширина символа 0xf3 в кодировке CP1251
\end{verbatim}

\begin{verbatim}
Warning in grid.Call.graphics(C_text, as.graphicsAnnot(x$label), x$x, x$y, :
неизвестна ширина символа 0xed в кодировке CP1251
\end{verbatim}

\begin{verbatim}
Warning in grid.Call.graphics(C_text, as.graphicsAnnot(x$label), x$x, x$y, :
неизвестна ширина символа 0xea в кодировке CP1251
\end{verbatim}

\begin{verbatim}
Warning in grid.Call.graphics(C_text, as.graphicsAnnot(x$label), x$x, x$y, :
неизвестна ширина символа 0xf2 в кодировке CP1251
\end{verbatim}

\begin{verbatim}
Warning in grid.Call.graphics(C_text, as.graphicsAnnot(x$label), x$x, x$y, :
неизвестна ширина символа 0xe8 в кодировке CP1251
\end{verbatim}

\begin{verbatim}
Warning in grid.Call.graphics(C_text, as.graphicsAnnot(x$label), x$x, x$y, :
неизвестна ширина символа 0xf0 в кодировке CP1251
\end{verbatim}

\pandocbounded{\includegraphics[keepaspectratio]{chapter7_files/figure-pdf/unnamed-chunk-6-6.pdf}}

\begin{Shaded}
\begin{Highlighting}[]
\CommentTok{\# На графике:}
\CommentTok{\#   {-} Черные точки: исторические данные (1990{-}2021)}
\CommentTok{\#   {-} Сплошная синяя линия: прогнозные значения (1990{-}2021)}
\CommentTok{\#   {-} Пунктирная синяя линия: прогноз на 2022{-}2024}
\CommentTok{\#   {-} Серые ленты: 50\% и 95\% доверительные интервалы}
\CommentTok{\# Такая визуализация позволяет легко интерпретировать как исторические данные, }
\CommentTok{\# так и будущие прогнозы с учетом неопределенности.}
\end{Highlighting}
\end{Shaded}

\bookmarksetup{startatroot}

\chapter{Модель Catch-Survey Analysis
(CSA)}\label{ux43cux43eux434ux435ux43bux44c-catch-survey-analysis-csa}

\section{Введение}\label{ux432ux432ux435ux434ux435ux43dux438ux435-8}

Продолжим с честного предупреждения. Как только мы делим запас на
осмысленные группы --- пререкруты, рекруты, пострекруты --- и
прописываем переходы между ними, очень легко почувствовать, что мы
«внутри механизма», а значит, управляем им. Это та самая иллюзия
контроля: формула кажется ближе к биологии, чем функция продукции в
продукционной модели, и мозг дорисовывает уверенность, которой в данных
может не хватать. Дисциплина «медленного» режима анализа данных и работы
мозга в том, чтобы отделить структуру от знания: CSA и правда даёт
больше биологического смысла, но достоверность этого смысла определяется
не изяществом уравнений, а качеством наблюдений, идентифицируемостью
параметров и тем, как аккуратно мы проверяем альтернативы. У системы
есть история, пороги, сдвиги и запаздывания; толстые хвосты и редкие
годы «сюрпризов». Наша задача --- встроить всё это в анализ, не потеряв
прозрачности и воспроизводимости.

Практическая ценность CSA в том, что она распаковывает общую динамику по
«каналам»: пополнение, рост (переходы между категориями), естественная
смертность и изъятие промыслом, а затем связывает скрытые состояния с
наблюдаемыми индексами через улавливаемость. Это полезно для объектов с
выборочным промыслом (например, только крупные самцы) и сильно
неоднородной размерной структурой: давление смещается по группам, и
агрегаты легко маскируют реальные сдвиги. Но вместе с детализацией
приходит классическая ловушка идентифицируемости: \emph{M} путается с
\emph{q}, переходные вероятности (\emph{Gp}, \emph{Gr}, \emph{Mp}) ---
между собой, процессная дисперсия --- с наблюдательной. Если не
заякорить параметры априорами и не опереться на внешнюю информацию
(калибровки тралов, биологию линьки, разумные диапазоны смертности),
модель будет «объяснять» всё и сразу, но в каждом запуске по‑разному.
Байесовская постановка --- именно способ честно ввести эти якоря и сразу
показать, где данные «передвинули» прайеры, а где --- нет.

Главные источники смещения здесь не банальны. Улавливаемость редко
постоянна: меняются суда, орудия, глубины, сезонность; индексы могут
отражать доступность, а не истинную численность. Выборочная природа
съёмок порождает нули и неоднородную дисперсию; логнормальная модель
наблюдений удобна, но не всегда робастна к выбросам. Переходы между
группами зависят от роста и линьки, а значит --- от среды; если эти
зависимости «впитаны» в постоянные \emph{Gp}/\emph{Gr}/\emph{Mp}, то
часть динамики будет ложиться в ошибки. Поэтому мы заранее признаём:
фиксированные прайеры --- осознанный компромисс, а чувствительность к
прайерам --- обязательная проверка, а не факультатив.

Диагностика --- не приложение, а часть модели. Мы смотрим траектории
цепей, эффективный размер выборки, MC‑ошибку, проверяем сходимость
независимых цепей и, что не менее важно, сопоставляем прайеры и
постериоры по ключевым параметрам: если плотности почти не разошлись,
значит, данные нас мало чему научили; если разошлись «до хвостов» ---
возможно, мы зашли за границы биологически правдоподобного. Постериорные
проверочные прогонки (posterior predictive) --- простой и мощный тест:
генерируем псевдо‑индексы из модели и убеждаемся, что реальная серия не
выглядит «инородным телом». Бабл‑графики остатков по группам и годам
помогают увидеть систематику: дрейф знака --- сигнал к
времени‑зависимому \emph{q} или пропущенным ковариатам. А вероятность
нарушить управленческий предел (\emph{PR} \textless{}
\emph{PR\textsubscript{lim}}) должна выводиться прямо из постериора, а
не прикидываться на глаз по одной траектории.

Стабильность выводов проверяем во времени --- ретроспективой. Отрезая по
одному--несколько последних лет и переоценивая модель, мы видим, «дышит»
ли оценка прошлых лет от добавления новой информации. Это сохраняет нас
от соблазна «подогнать настоящее» и выдать его за прогнозную силу. Там
же хорошо выявляются скрытые конфликты идентифицируемости: если при
каждом «срезе» меняется баланс \emph{M}--\emph{q} или расползаются
переходы из одной размерной категории в другую, значит, данных не
хватает или прайеры слишком расплывчаты. Такой «проверенный на бордюре»
консерватизм --- это не скепсис, а инструмент против самоуверенности.

Часть неопределённости мы принимаем как данность и переводим в язык
решений. Менеджменту нужны не «числа», а вероятности: какова
P(\emph{PR}\textless{}\emph{PR\textsubscript{lim}}) в текущем году, как
меняется она при сценарии улова, каков шанс сохранить \emph{PR} над
порогом при консервативном и при агрессивном изъятии. Картина «веера»
--- медиана и 50/95\% интервалы --- честнее единственной жирной линии. И
здесь полезна мысль: аккуратно собранные данные и прозрачные процедуры,
повторённые из года в год, делают систему разумнее --- даже если в
каждом конкретном году интервал широк. Прогресс не в том, чтобы угадать
до тонны, а в том, чтобы системно сокращать неопределённость и принимать
решения, устойчивые к её остаткам.

Практическая реализация в этом занятии выдержана в том же ключе. Мы
задаём прайеры, учим модель в JAGS, сохраняем полные постериоры
параметров и состояний, сравниваем прайеры с постериорами, строим
диагностические графики остатков и динамики по группам, считаем
вероятность пересечения порога \emph{PR\textsubscript{lim}} и даём
интерпретацию в управленческих терминах. Там, где это уместно, тестируем
чувствительность к диапазонам прайеров на \emph{M} и \emph{q}, а также к
альтернативам в наблюдательной части (логнормальная дисперсия). И да,
Excel‑симулятор на четыре группы --- не игрушка, а хороший способ
«почувствовать руками» идентифицируемость: как меняется постериор при
фиксации \emph{M}, при расширении дисперсий наблюдений, при «дрейфе»
\emph{q}. Интуиция, подкреплённая такими играми, экономит много времени
в полноценной байесовской оценке.

Наконец, важная оговорка: CSA --- не конечная станция. В данных с явными
климатическими сдвигами стоит рассмотреть время‑зависимую
улавливаемость, ковариаты для переходов и смертности, иерархическую
связку нескольких съёмок. Если в этом блоке мы делаем базовую, учебную
версию, то следующий шаг --- включать «регуляторы» сложности только
после того, как базовая модель пройдёт диагностику. Это тот самый «мост»
между ясностью и гибкостью: сперва минимально достаточная структура,
затем --- аккуратные расширения с прицельной проверкой альтернатив. Так
вводится порядок в систему, где соблазнов «знать больше, чем знаем»
всегда больше, чем данных.

И так, модель ``анализа уловов и съемок'' - Catch-Survey Analysis (CSA)
представляет собой инструмент для оценки состояния запасов, особенно тех
видов, данные по индивидуальному возрасту которых труднодоступны или
отсутствуют, что типично для многих беспозвоночных, таких как крабы,
креветки, а также для некоторых рыб. В отличие от классических
продукционных моделей, которые оперируют агрегированными показателями
всей популяции и требуют строгих допущений о ее равновесном состоянии и
постоянной емкости среды, когортные модели, подобные CSA, позволяют
отслеживать судьбу отдельных функциональных категорий (например,
пререкруты, рекруты, пострекруты). Они явным образом учитывают такие
процессы, как рост, пополнение и естественная смертность, разделяя запас
на дискретные размерные или возрастные группы. Это дает несомненное
преимущество при анализе динамики популяций с выраженной цикличностью
или тех, которые подвергаются интенсивному промысловому прессу,
избирательно воздействующему на определенные размерные или возрастные
категории (например, пререкруты не подвержены прямой прмысловой
смертности в отличие от рекрутов и посрекрутов). Подробнее о модели и ее
реализации можно почитать в статье
\href{https://mombus.github.io/cRab/data/CSA.pdf}{``Результаты
применения стохастической когортной модели CSA для оценки запаса
камчатского краба Paralithodes camtschaticus в Баренцевом море''}. В
статье описывается реализация модели в программе OpenBUGS, которая в
упрощенном виде (без прогноза, риск-анализа и диагностики) и в учебных
целях была переведена в среду R и представлена ниже, а полный срипт
\href{https://mombus.github.io/cRab/data/CSA.R}{здесь}.Также доступна
иммитационная CSA модель для 4 размерных групп, реализованная в MS Excel
по \href{https://mombus.github.io/cRab/data/CSA.xlsx}{ссылке}.

Данная реализация модели представляет собой байесовский подход к оценке
запасов, который позволяет учитывать неопределенности как в процессе
динамики популяции, так и в процессе наблюдений, что особенно важно при
работе с данными, характеризующимися высокой вариабельностью и
неполнотой. В основе модели лежит разделение популяции на три
размерно-возрастные группы: пререкруты (P1), рекруты (R) и пострекруты
(P), что соответствует биологическим особенностям многих видов крабов,
включая камчатского краба. Модель включает два основных компонента:
динамику процесса, описывающую естественные изменения численности
популяции, и модель наблюдений, связывающую ненаблюдаемые ``истинную''
численность запаса с доступными данными съемок (индексами численности
пререкрутов, рекрутов и пострекрутов). Уравнения процессной динамики для
пострекрутов имеют вид:

P{[}i{]} = {[}(P1{[}i-1{]}×Gp×Mp) + R{[}i-1{]} + P{[}i-1{]} -
catch{[}i-1{]}{]} × exp(-M) + εP, где

Gp обозначает вероятность перехода пререкрутов в пострекруты,

Mp - вероятность линьки пререкрутов,

M - коэффициент естественной смертности, а εP представляет собой
процессную ошибку.

Для рекрутов уравнение динамики выглядит как

R{[}i{]} = (P1{[}i-1{]}×Gr×Mp) × exp(-M) + εR, где

Gr - вероятность перехода пререкрутов в рекруты. Динамика пререкрутов
моделируется как лог-случайное блуждание P1{[}i{]} = P1{[}i-1{]} + εP1.
Модель наблюдений предполагает, что данные траловых съемок соответствуют
логнормальному распределению относительно истинной численности,
умноженной на коэффициент улавливаемости:

bioindexP1{[}i{]} \textasciitilde{} lognormal(log(q1×P1{[}i{]}),
precbioindexP1),

аналогично для рекрутов и пострекрутов, где q1, q2, q3 - коэффициенты
улавливаемости для каждой группы, а precbioindex - параметры точности. В
байесовском подходе ключевую роль играют априорные распределения
параметров, которые в данной реализации задаются как равномерные для
коэффициентов улавливаемости (q1, q2, q3 \textasciitilde{}
dunif(0.1,1)), нормальные для вероятностей перехода (Gr
\textasciitilde{} dnorm(0.9,500), Gp \textasciitilde{} dnorm(0.075,500),
Mp \textasciitilde{} dnorm(0.95,500)) и для коэффициента естественной
смертности (M \textasciitilde{} dnorm(0.2,100)). Использование
байесовского подхода позволяет не только получить точечные оценки
параметров, но и оценить полные апостериорные распределения, что дает
возможность проводить риск-анализ различных сценариев управления
запасом. В данном занятии мы реализуем модель CSA в среде R с
использованием пакетов rjags и coda, что позволяет эффективно работать с
байесовскими иерархическими моделями через интерфейс с программой JAGS,
которую также необходимо установить.

Мы рассмотрим полный цикл работы с моделью: от подготовки данных и
задания априорных распределений до обучения модели и анализа
результатов, включая визуализацию априорных и апостериорных
распределений параметров, анализ остатков и сравнение моделируемой и
фактической динамики запаса. Особое внимание будет уделено интерпретации
результатов в контексте управления водными биоресурсами, что является
ключевой целью применения подобных моделей в практической деятельности
гидробиологов и ихтиологов.

\section{Реализация
модели}\label{ux440ux435ux430ux43bux438ux437ux430ux446ux438ux44f-ux43cux43eux434ux435ux43bux438}

\begin{Shaded}
\begin{Highlighting}[]
\CommentTok{\# ========================================================================================================================}
\CommentTok{\# ПРАКТИЧЕСКОЕ ЗАНЯТИЕ: МОДЕЛЬ Catch{-}Survey Analysis (CSA) {-} три категории (пререкруты (P1), рекруты (R), пострекруты (P)}
\CommentTok{\# Курс: "Оценка водных биоресурсов в среде R (для начинающих)"}
\CommentTok{\# Автор: Баканев С. В. Дата: 20.08.2025}
\CommentTok{\# Структура:}
\CommentTok{\# 1) Входные данные}
\CommentTok{\# 2) Модель}
\CommentTok{\# 3) Прайеры}
\CommentTok{\# 4) Обучение модели}
\CommentTok{\# 5) Подготовка выходных данных }
\CommentTok{\# 6) Анализ результатов (визуализация априорных и апостериорных параметров;бабл{-}плоты остатков;  динамика индексов) }
\CommentTok{\# ========================================================================================================================}
\CommentTok{\# Установка рабочей директории}
\FunctionTok{setwd}\NormalTok{(}\StringTok{"C:/CSA"}\NormalTok{)}

\CommentTok{\# Подключение необходимых библиотек}
\CommentTok{\# install.packages(c("rjags", "coda"))  \# Раскомментировать для установки}
\FunctionTok{library}\NormalTok{(rjags)  }\CommentTok{\# Для работы с JAGS}
\end{Highlighting}
\end{Shaded}

\begin{verbatim}
Загрузка требуемого пакета: coda
\end{verbatim}

\begin{verbatim}
Linked to JAGS 4.3.1
\end{verbatim}

\begin{verbatim}
Loaded modules: basemod,bugs
\end{verbatim}

\begin{Shaded}
\begin{Highlighting}[]
\FunctionTok{library}\NormalTok{(coda)   }\CommentTok{\# Для анализа MCMC{-}выхода}
\FunctionTok{library}\NormalTok{(ggplot2)}\CommentTok{\# Рисунки}

\CommentTok{\# ========================================================================================================================}
\CommentTok{\# {-}{-}{-} Входные данные {-}{-}{-}}
\CommentTok{\# ========================================================================================================================}
\NormalTok{data\_list }\OtherTok{\textless{}{-}} \FunctionTok{list}\NormalTok{(}
  \AttributeTok{N =} \DecValTok{16}\NormalTok{,}\CommentTok{\# Количество временных точек}
 \CommentTok{\# Наблюдаемые данные (индексы запаса)}
  \AttributeTok{bioindexP1 =} \FunctionTok{c}\NormalTok{(}\DecValTok{1500}\NormalTok{,}\DecValTok{1028}\NormalTok{,}\DecValTok{554}\NormalTok{,}\DecValTok{887}\NormalTok{,}\DecValTok{1345}\NormalTok{,}\DecValTok{1817}\NormalTok{,}\DecValTok{2291}\NormalTok{,}\DecValTok{1958}\NormalTok{,}\DecValTok{1500}\NormalTok{,}\DecValTok{1028}\NormalTok{,}\DecValTok{554}\NormalTok{,}\DecValTok{887}\NormalTok{,}\DecValTok{1345}\NormalTok{,}\DecValTok{1817}\NormalTok{,}\DecValTok{2291}\NormalTok{,}\DecValTok{1958}\NormalTok{),}
  \AttributeTok{bioindexR  =} \FunctionTok{c}\NormalTok{(}\DecValTok{2531}\NormalTok{,}\DecValTok{1927}\NormalTok{,}\DecValTok{1305}\NormalTok{,}\DecValTok{764}\NormalTok{,}\DecValTok{1216}\NormalTok{,   }\DecValTok{1820}\NormalTok{,}\DecValTok{2442}\NormalTok{,}\DecValTok{2983}\NormalTok{,}\DecValTok{2531}\NormalTok{,}\DecValTok{1927}\NormalTok{,}\DecValTok{1305}\NormalTok{,}\DecValTok{764}\NormalTok{,}\DecValTok{1216}\NormalTok{,}\DecValTok{1820}\NormalTok{,}\DecValTok{2442}\NormalTok{,}\DecValTok{2983}\NormalTok{),}
  \AttributeTok{bioindexP  =} \FunctionTok{c}\NormalTok{(}\DecValTok{13741}\NormalTok{,}\DecValTok{13770}\NormalTok{,}\DecValTok{13060}\NormalTok{,}\DecValTok{11653}\NormalTok{,}\DecValTok{9782}\NormalTok{,}\DecValTok{8634}\NormalTok{,}\DecValTok{8321}\NormalTok{,}\DecValTok{8793}\NormalTok{,}\DecValTok{9809}\NormalTok{,}\DecValTok{10177}\NormalTok{,}\DecValTok{9776}\NormalTok{,}\DecValTok{9566}\NormalTok{,}\DecValTok{8789}\NormalTok{,}\DecValTok{8640}\NormalTok{,}\DecValTok{9240}\NormalTok{,}\DecValTok{10547}\NormalTok{),}
  \AttributeTok{catch      =} \FunctionTok{c}\NormalTok{(}\DecValTok{6}\NormalTok{,}\DecValTok{2}\NormalTok{,}\DecValTok{6}\NormalTok{,}\DecValTok{15}\NormalTok{,}\DecValTok{21}\NormalTok{,}\DecValTok{37}\NormalTok{,}\DecValTok{37}\NormalTok{,}\DecValTok{315}\NormalTok{,}\DecValTok{945}\NormalTok{,}\DecValTok{890}\NormalTok{,}\DecValTok{991}\NormalTok{,}\DecValTok{1060}\NormalTok{,}\DecValTok{1000}\NormalTok{,}\DecValTok{1000}\NormalTok{,}\DecValTok{1600}\NormalTok{,}\DecValTok{1673}\NormalTok{,}\DecValTok{1250}\NormalTok{)}
\NormalTok{)}

\CommentTok{\# Создание вектора лет для подписей}
\NormalTok{YEAR }\OtherTok{\textless{}{-}} \DecValTok{2000} \SpecialCharTok{+} \DecValTok{0}\SpecialCharTok{:}\NormalTok{(data\_list}\SpecialCharTok{$}\NormalTok{N }\SpecialCharTok{{-}} \DecValTok{1}\NormalTok{)}

\CommentTok{\# ========================================================================================================================}
\CommentTok{\# {-}{-}{-} Генерация модели CSA {-}{-}}
\CommentTok{\# ========================================================================================================================}
\NormalTok{model\_string }\OtherTok{\textless{}{-}} \StringTok{"}
\StringTok{model \{}
\StringTok{   for (i in 1:N) \{}
\StringTok{    bioindexP1med[i] \textless{}{-} log(1.0E{-}6 + q1 * P1[i])}
\StringTok{    bioindexP1[i] \textasciitilde{} dlnorm(bioindexP1med[i], precbioindexP1)}
\StringTok{    bioindexRmed[i]  \textless{}{-} log(1.0E{-}6 + q2 * R[i])}
\StringTok{    bioindexR[i] \textasciitilde{} dlnorm(bioindexRmed[i],  precbioindexR)}
\StringTok{    bioindexPmed[i]  \textless{}{-} log(1.0E{-}6 + q3 * P[i])}
\StringTok{    bioindexP[i] \textasciitilde{} dlnorm(bioindexPmed[i],  precbioindexP)}
\StringTok{  \}}

\StringTok{  inv\_surv \textless{}{-} exp({-}M)\# Коэффициент естественной смертности}
\StringTok{  for (i in 2:N) \{}
\StringTok{       tmpPraw[i] \textless{}{-} (P1[i{-}1]*Gp*Mp + R[i{-}1] + P[i{-}1] {-} catch[i{-}1]) * inv\_surv}
\StringTok{    tmpPpos[i] \textless{}{-} tmpPraw[i] * step(tmpPraw[i]) }
\StringTok{    Pmed[i] \textless{}{-} log(1.0E{-}6 + tmpPpos[i])}
\StringTok{    P[i] \textasciitilde{} dlnorm(Pmed[i], precP)}

\StringTok{    tmpRraw[i] \textless{}{-} (P1[i{-}1]*Gr*Mp) * inv\_surv}
\StringTok{    tmpRpos[i] \textless{}{-} tmpRraw[i] * step(tmpRraw[i])}
\StringTok{    Rmed[i] \textless{}{-} log(1.0E{-}6 + tmpRpos[i])}
\StringTok{    R[i] \textasciitilde{} dlnorm(Rmed[i], precR)}

\StringTok{    P1med[i] \textless{}{-} log(1.0E{-}6 + P1[i{-}1])}
\StringTok{    P1[i] \textasciitilde{} dlnorm(P1med[i], precP1)}
\StringTok{  \}}

\StringTok{  for (i in 1:N) \{}
\StringTok{    PR[i] \textless{}{-} P[i] + R[i]}
\StringTok{    p.PRlim[i] \textless{}{-} step(PRlim {-} PR[i])}
\StringTok{  \}}
\StringTok{  PRlim \textless{}{-} 4000}


\StringTok{  P1[1] \textasciitilde{} dunif(200,4000)}
\StringTok{  P[1]  \textasciitilde{} dunif(200,6000)}
\StringTok{  R[1]  \textasciitilde{} dunif(200,25000)}

\StringTok{  Gr \textasciitilde{} dnorm(0.9,  500)}
\StringTok{  Gp \textasciitilde{} dnorm(0.075,500)}
\StringTok{  Mp \textasciitilde{} dnorm(0.95, 500)}

\StringTok{   precbioindexP1 \textasciitilde{} dgamma(12.22, 1.1)}
\StringTok{  precbioindexR  \textasciitilde{} dgamma(12.22, 1.1)}
\StringTok{  precbioindexP  \textasciitilde{} dgamma(12.22, 1.1)}

\StringTok{  q1 \textasciitilde{} dunif(0.1,1)}
\StringTok{  q2 \textasciitilde{} dunif(0.1,1)}
\StringTok{  q3 \textasciitilde{} dunif(0.1,1) }

\StringTok{  precP1 \textasciitilde{} dgamma(12.22, 1.1)}
\StringTok{  precR  \textasciitilde{} dgamma(12.22, 1.1)}
\StringTok{  precP  \textasciitilde{} dgamma(12.22, 1.1)}

\StringTok{  M \textasciitilde{} dnorm(0.2, 100)}
\StringTok{\}}
\StringTok{"}


\CommentTok{\# ========================================================================================================================}
\CommentTok{\# {-}{-}{-} Обучение модели {-}{-}{-}}
\CommentTok{\# ========================================================================================================================}
\FunctionTok{set.seed}\NormalTok{(}\DecValTok{1}\NormalTok{)  }\CommentTok{\# Для воспроизводимости}
\CommentTok{\# Инициализация модели JAGS}
\NormalTok{jm }\OtherTok{\textless{}{-}} \FunctionTok{jags.model}\NormalTok{(}
  \FunctionTok{textConnection}\NormalTok{(model\_string),  }\CommentTok{\# Модель из строки}
  \AttributeTok{data =}\NormalTok{ data\_list,             }\CommentTok{\# Данные}
  \AttributeTok{n.chains =} \DecValTok{3}\NormalTok{,                 }\CommentTok{\# Количество цепей}
  \AttributeTok{n.adapt =} \DecValTok{1500}                \CommentTok{\# Длина адаптационной фазы}
\NormalTok{)}
\end{Highlighting}
\end{Shaded}

\begin{verbatim}
Compiling model graph
   Resolving undeclared variables
   Allocating nodes
Graph information:
   Observed stochastic nodes: 48
   Unobserved stochastic nodes: 61
   Total graph size: 576

Initializing model
\end{verbatim}

\begin{Shaded}
\begin{Highlighting}[]
\CommentTok{\# Обновление модели (burn{-}in)}
\FunctionTok{update}\NormalTok{(jm, }\DecValTok{4000}\NormalTok{)}

\CommentTok{\# Переменные для мониторинга}
\NormalTok{vars\_to\_monitor }\OtherTok{\textless{}{-}} \FunctionTok{c}\NormalTok{(}
  \StringTok{"M"}\NormalTok{,}\StringTok{"Gp"}\NormalTok{,}\StringTok{"Gr"}\NormalTok{,}\StringTok{"Mp"}\NormalTok{,}\StringTok{"q1"}\NormalTok{,}\StringTok{"q2"}\NormalTok{,}\StringTok{"q3"}\NormalTok{,                    }\CommentTok{\# Параметры}
  \StringTok{"precP"}\NormalTok{,}\StringTok{"precP1"}\NormalTok{,}\StringTok{"precR"}\NormalTok{,}\StringTok{"precbioindexP"}\NormalTok{,}\StringTok{"precbioindexP1"}\NormalTok{,}\StringTok{"precbioindexR"}\NormalTok{,  }\CommentTok{\# Точности}
  \StringTok{"P"}\NormalTok{,}\StringTok{"P1"}\NormalTok{,}\StringTok{"R"}\NormalTok{,}\StringTok{"PR"}\NormalTok{,}\StringTok{"p.PRlim"}                           \CommentTok{\# Состояния и производные}
\NormalTok{)}


\CommentTok{\# Генерация MCMC{-}выборок}
\NormalTok{samps }\OtherTok{\textless{}{-}} \FunctionTok{coda.samples}\NormalTok{(}
\NormalTok{  jm, }
  \AttributeTok{variable.names =}\NormalTok{ vars\_to\_monitor,  }\CommentTok{\# Мониторируемые переменные}
  \AttributeTok{n.iter =} \DecValTok{6000}\NormalTok{,                     }\CommentTok{\# Длина выборки}
  \AttributeTok{thin =} \DecValTok{3}                           \CommentTok{\# Прореживание}
\NormalTok{)}
\CommentTok{\# ========================================================================================================================}
\CommentTok{\# {-}{-}{-} Анализ результатов {-}{-}{-}}
\CommentTok{\# ========================================================================================================================}
\CommentTok{\# Стандартная статистика по выборкам}
\NormalTok{sm }\OtherTok{\textless{}{-}} \FunctionTok{summary}\NormalTok{(samps)}
\NormalTok{stats }\OtherTok{\textless{}{-}}\NormalTok{ sm}\SpecialCharTok{$}\NormalTok{statistics   }\CommentTok{\# Средние, SD, стандартные ошибки}
\NormalTok{quants }\OtherTok{\textless{}{-}}\NormalTok{ sm}\SpecialCharTok{$}\NormalTok{quantiles   }\CommentTok{\# Квантили (2.5\%, 25\%, 50\%, 75\%, 97.5\%)}

\CommentTok{\# Матрица всех сэмплов для ручных вычислений}
\NormalTok{draws\_mat }\OtherTok{\textless{}{-}} \FunctionTok{as.matrix}\NormalTok{(samps)}

\CommentTok{\# Функция для расчета MC ошибки через эффективный размер выборки}
\NormalTok{mcse\_from\_ess }\OtherTok{\textless{}{-}} \ControlFlowTok{function}\NormalTok{(vec) \{}
\NormalTok{  ess }\OtherTok{\textless{}{-}} \FunctionTok{effectiveSize}\NormalTok{(}\FunctionTok{as.mcmc}\NormalTok{(vec))  }\CommentTok{\# Эффективный размер выборки}
  \FunctionTok{sd}\NormalTok{(vec) }\SpecialCharTok{/} \FunctionTok{sqrt}\NormalTok{(}\FunctionTok{as.numeric}\NormalTok{(ess))     }\CommentTok{\# MC ошибка}
\NormalTok{\}}

\CommentTok{\# Функция для создания строки результата}
\NormalTok{make\_row }\OtherTok{\textless{}{-}} \ControlFlowTok{function}\NormalTok{(year, mapping, node, mean, sd, mcse, q2}\FloatTok{.5}\NormalTok{, q25, q50, q75, q97}\FloatTok{.5}\NormalTok{) \{}
  \FunctionTok{data.frame}\NormalTok{(}
    \AttributeTok{YEAR =}\NormalTok{ year,}
    \StringTok{\textasciigrave{}}\AttributeTok{\#Vectors to monitor}\StringTok{\textasciigrave{}} \OtherTok{=}\NormalTok{ mapping,}
    \AttributeTok{node =}\NormalTok{ node,}
    \AttributeTok{mean =}\NormalTok{ mean,}
    \AttributeTok{sd =}\NormalTok{ sd,}
    \StringTok{\textasciigrave{}}\AttributeTok{MC error}\StringTok{\textasciigrave{}} \OtherTok{=}\NormalTok{ mcse,}
    \StringTok{\textasciigrave{}}\AttributeTok{2.50\%}\StringTok{\textasciigrave{}} \OtherTok{=}\NormalTok{ q2}\FloatTok{.5}\NormalTok{,}
    \StringTok{\textasciigrave{}}\AttributeTok{25.00\%}\StringTok{\textasciigrave{}} \OtherTok{=}\NormalTok{ q25,}
    \AttributeTok{median =}\NormalTok{ q50,}
    \StringTok{\textasciigrave{}}\AttributeTok{75.00\%}\StringTok{\textasciigrave{}} \OtherTok{=}\NormalTok{ q75,}
    \StringTok{\textasciigrave{}}\AttributeTok{97.50\%}\StringTok{\textasciigrave{}} \OtherTok{=}\NormalTok{ q97}\FloatTok{.5}\NormalTok{,}
    \AttributeTok{check.names =} \ConstantTok{FALSE}
\NormalTok{  )}
\NormalTok{\}}

\CommentTok{\# Список для накопления результатов}
\NormalTok{rows }\OtherTok{\textless{}{-}} \FunctionTok{list}\NormalTok{()}

\CommentTok{\# Функция добавления скалярных параметров}
\NormalTok{add\_scalar }\OtherTok{\textless{}{-}} \ControlFlowTok{function}\NormalTok{(x\_idx, vname) \{}
  \ControlFlowTok{if}\NormalTok{ (vname }\SpecialCharTok{\%in\%} \FunctionTok{rownames}\NormalTok{(stats)) \{}
    \CommentTok{\# Если параметр есть в готовой статистике}
\NormalTok{    m }\OtherTok{\textless{}{-}}\NormalTok{ stats[vname, }\StringTok{"Mean"}\NormalTok{]}
\NormalTok{    s }\OtherTok{\textless{}{-}}\NormalTok{ stats[vname, }\StringTok{"SD"}\NormalTok{]}
\NormalTok{    mcse }\OtherTok{\textless{}{-}} \FunctionTok{mcse\_from\_ess}\NormalTok{(draws\_mat[, vname])}
\NormalTok{    q }\OtherTok{\textless{}{-}}\NormalTok{ quants[vname, }\FunctionTok{c}\NormalTok{(}\StringTok{"2.5\%"}\NormalTok{, }\StringTok{"25\%"}\NormalTok{, }\StringTok{"50\%"}\NormalTok{, }\StringTok{"75\%"}\NormalTok{, }\StringTok{"97.5\%"}\NormalTok{)]}
\NormalTok{    rows[[}\FunctionTok{length}\NormalTok{(rows) }\SpecialCharTok{+} \DecValTok{1}\NormalTok{]] }\OtherTok{\textless{}\textless{}{-}} \FunctionTok{make\_row}\NormalTok{(}\ConstantTok{NA}\NormalTok{, }\FunctionTok{paste0}\NormalTok{(}\StringTok{"x["}\NormalTok{, x\_idx, }\StringTok{"]\textless{}{-}"}\NormalTok{, vname), }\FunctionTok{paste0}\NormalTok{(}\StringTok{"x["}\NormalTok{, x\_idx, }\StringTok{"]"}\NormalTok{),}
\NormalTok{                                          m, s, mcse, q[}\DecValTok{1}\NormalTok{], q[}\DecValTok{2}\NormalTok{], q[}\DecValTok{3}\NormalTok{], q[}\DecValTok{4}\NormalTok{], q[}\DecValTok{5}\NormalTok{])}
\NormalTok{  \} }\ControlFlowTok{else} \ControlFlowTok{if}\NormalTok{ (vname }\SpecialCharTok{\%in\%} \FunctionTok{c}\NormalTok{(}\StringTok{"sigmaP1"}\NormalTok{,}\StringTok{"sigmaR"}\NormalTok{,}\StringTok{"sigmaP"}\NormalTok{)) \{}
    \CommentTok{\# Для стандартных отклонений (преобразуем из точности)}
\NormalTok{    src }\OtherTok{\textless{}{-}} \ControlFlowTok{switch}\NormalTok{(vname,}
                  \AttributeTok{sigmaP1 =} \StringTok{"precP1"}\NormalTok{,}
                  \AttributeTok{sigmaR  =} \StringTok{"precR"}\NormalTok{,}
                  \AttributeTok{sigmaP  =} \StringTok{"precP"}\NormalTok{)}
    \ControlFlowTok{if}\NormalTok{ (src }\SpecialCharTok{\%in\%} \FunctionTok{colnames}\NormalTok{(draws\_mat)) \{}
\NormalTok{      vec }\OtherTok{\textless{}{-}} \FunctionTok{sqrt}\NormalTok{(}\DecValTok{1} \SpecialCharTok{/}\NormalTok{ draws\_mat[, src])  }\CommentTok{\# Преобразование precision {-}\textgreater{} sigma}
\NormalTok{      m }\OtherTok{\textless{}{-}} \FunctionTok{mean}\NormalTok{(vec); s }\OtherTok{\textless{}{-}} \FunctionTok{sd}\NormalTok{(vec); mcse }\OtherTok{\textless{}{-}} \FunctionTok{mcse\_from\_ess}\NormalTok{(vec)}
\NormalTok{      q }\OtherTok{\textless{}{-}} \FunctionTok{quantile}\NormalTok{(vec, }\FunctionTok{c}\NormalTok{(}\FloatTok{0.025}\NormalTok{,}\FloatTok{0.25}\NormalTok{,}\FloatTok{0.5}\NormalTok{,}\FloatTok{0.75}\NormalTok{,}\FloatTok{0.975}\NormalTok{))}
\NormalTok{      rows[[}\FunctionTok{length}\NormalTok{(rows) }\SpecialCharTok{+} \DecValTok{1}\NormalTok{]] }\OtherTok{\textless{}\textless{}{-}} \FunctionTok{make\_row}\NormalTok{(}\ConstantTok{NA}\NormalTok{, }\FunctionTok{paste0}\NormalTok{(}\StringTok{"x["}\NormalTok{, x\_idx, }\StringTok{"]\textless{}{-}"}\NormalTok{, vname), }\FunctionTok{paste0}\NormalTok{(}\StringTok{"x["}\NormalTok{, x\_idx, }\StringTok{"]"}\NormalTok{),}
\NormalTok{                                            m, s, mcse, q[}\DecValTok{1}\NormalTok{], q[}\DecValTok{2}\NormalTok{], q[}\DecValTok{3}\NormalTok{], q[}\DecValTok{4}\NormalTok{], q[}\DecValTok{5}\NormalTok{])}
\NormalTok{    \}}
\NormalTok{  \}}
\NormalTok{\}}

\CommentTok{\# Добавление основных параметров}
\FunctionTok{add\_scalar}\NormalTok{(}\DecValTok{1}\NormalTok{,  }\StringTok{"M"}\NormalTok{)}
\FunctionTok{add\_scalar}\NormalTok{(}\DecValTok{2}\NormalTok{,  }\StringTok{"q1"}\NormalTok{)}
\FunctionTok{add\_scalar}\NormalTok{(}\DecValTok{3}\NormalTok{,  }\StringTok{"q2"}\NormalTok{)}
\FunctionTok{add\_scalar}\NormalTok{(}\DecValTok{4}\NormalTok{,  }\StringTok{"q3"}\NormalTok{)}
\FunctionTok{add\_scalar}\NormalTok{(}\DecValTok{5}\NormalTok{,  }\StringTok{"sigmaP1"}\NormalTok{)}
\FunctionTok{add\_scalar}\NormalTok{(}\DecValTok{6}\NormalTok{,  }\StringTok{"sigmaR"}\NormalTok{)}
\FunctionTok{add\_scalar}\NormalTok{(}\DecValTok{7}\NormalTok{,  }\StringTok{"sigmaP"}\NormalTok{)}
\FunctionTok{add\_scalar}\NormalTok{(}\DecValTok{8}\NormalTok{,  }\StringTok{"precbioindexP1"}\NormalTok{)}
\FunctionTok{add\_scalar}\NormalTok{(}\DecValTok{9}\NormalTok{,  }\StringTok{"precbioindexR"}\NormalTok{)}
\FunctionTok{add\_scalar}\NormalTok{(}\DecValTok{10}\NormalTok{, }\StringTok{"precbioindexP"}\NormalTok{)}
\FunctionTok{add\_scalar}\NormalTok{(}\DecValTok{11}\NormalTok{, }\StringTok{"Gr"}\NormalTok{)}
\FunctionTok{add\_scalar}\NormalTok{(}\DecValTok{12}\NormalTok{, }\StringTok{"Gp"}\NormalTok{)}
\FunctionTok{add\_scalar}\NormalTok{(}\DecValTok{13}\NormalTok{, }\StringTok{"Mp"}\NormalTok{)}

\CommentTok{\# Функция добавления временных рядов}
\NormalTok{add\_series }\OtherTok{\textless{}{-}} \ControlFlowTok{function}\NormalTok{(base\_idx, varname, years) \{}
  \ControlFlowTok{for}\NormalTok{ (i }\ControlFlowTok{in} \FunctionTok{seq\_along}\NormalTok{(years)) \{}
\NormalTok{    rn }\OtherTok{\textless{}{-}} \FunctionTok{paste0}\NormalTok{(varname, }\StringTok{"["}\NormalTok{, i, }\StringTok{"]"}\NormalTok{)  }\CommentTok{\# Имя переменной с индексом}
    \ControlFlowTok{if}\NormalTok{ (}\SpecialCharTok{!}\NormalTok{rn }\SpecialCharTok{\%in\%} \FunctionTok{rownames}\NormalTok{(stats)) }\ControlFlowTok{next}  \CommentTok{\# Пропуск если нет данных}
\NormalTok{    m }\OtherTok{\textless{}{-}}\NormalTok{ stats[rn, }\StringTok{"Mean"}\NormalTok{]}
\NormalTok{    s }\OtherTok{\textless{}{-}}\NormalTok{ stats[rn, }\StringTok{"SD"}\NormalTok{]}
\NormalTok{    mcse }\OtherTok{\textless{}{-}} \FunctionTok{mcse\_from\_ess}\NormalTok{(draws\_mat[, rn])}
\NormalTok{    q }\OtherTok{\textless{}{-}}\NormalTok{ quants[rn, }\FunctionTok{c}\NormalTok{(}\StringTok{"2.5\%"}\NormalTok{, }\StringTok{"25\%"}\NormalTok{, }\StringTok{"50\%"}\NormalTok{, }\StringTok{"75\%"}\NormalTok{, }\StringTok{"97.5\%"}\NormalTok{)]}
\NormalTok{    xi }\OtherTok{\textless{}{-}}\NormalTok{ base\_idx }\SpecialCharTok{+}\NormalTok{ (i }\SpecialCharTok{{-}} \DecValTok{1}\NormalTok{)  }\CommentTok{\# Вычисление индекса в выходной таблице}
\NormalTok{    rows[[}\FunctionTok{length}\NormalTok{(rows) }\SpecialCharTok{+} \DecValTok{1}\NormalTok{]] }\OtherTok{\textless{}\textless{}{-}} \FunctionTok{make\_row}\NormalTok{(years[i], }\FunctionTok{paste0}\NormalTok{(}\StringTok{"x["}\NormalTok{, xi, }\StringTok{"]\textless{}{-}"}\NormalTok{, rn), }\FunctionTok{paste0}\NormalTok{(}\StringTok{"x["}\NormalTok{, xi, }\StringTok{"]"}\NormalTok{),}
\NormalTok{                                          m, s, mcse, q[}\DecValTok{1}\NormalTok{], q[}\DecValTok{2}\NormalTok{], q[}\DecValTok{3}\NormalTok{], q[}\DecValTok{4}\NormalTok{], q[}\DecValTok{5}\NormalTok{])}
\NormalTok{  \}}
\NormalTok{\}}

\CommentTok{\# Добавление временных рядов}
\FunctionTok{add\_series}\NormalTok{(}\DecValTok{100}\NormalTok{, }\StringTok{"P1"}\NormalTok{, YEAR)}
\FunctionTok{add\_series}\NormalTok{(}\DecValTok{200}\NormalTok{, }\StringTok{"R"}\NormalTok{,  YEAR)}
\FunctionTok{add\_series}\NormalTok{(}\DecValTok{300}\NormalTok{, }\StringTok{"P"}\NormalTok{,  YEAR)}

\CommentTok{\# Создание итоговой таблицы}
\NormalTok{out\_df }\OtherTok{\textless{}{-}} \FunctionTok{do.call}\NormalTok{(rbind, rows)}

\CommentTok{\# Создание групп для сортировки}
\NormalTok{out\_df}\SpecialCharTok{$}\NormalTok{group }\OtherTok{\textless{}{-}} \FunctionTok{ifelse}\NormalTok{(}\FunctionTok{is.na}\NormalTok{(out\_df}\SpecialCharTok{$}\NormalTok{YEAR), }\StringTok{"param"}\NormalTok{,}
                \FunctionTok{ifelse}\NormalTok{(}\FunctionTok{grepl}\NormalTok{(}\StringTok{"\textless{}{-}P1}\SpecialCharTok{\textbackslash{}\textbackslash{}}\StringTok{["}\NormalTok{, out\_df}\SpecialCharTok{$}\StringTok{\textasciigrave{}}\AttributeTok{\#Vectors to monitor}\StringTok{\textasciigrave{}}\NormalTok{), }\StringTok{"P1"}\NormalTok{,}
                \FunctionTok{ifelse}\NormalTok{(}\FunctionTok{grepl}\NormalTok{(}\StringTok{"\textless{}{-}R}\SpecialCharTok{\textbackslash{}\textbackslash{}}\StringTok{["}\NormalTok{,  out\_df}\SpecialCharTok{$}\StringTok{\textasciigrave{}}\AttributeTok{\#Vectors to monitor}\StringTok{\textasciigrave{}}\NormalTok{), }\StringTok{"R"}\NormalTok{, }\StringTok{"P"}\NormalTok{)))}

\CommentTok{\# Сортировка параметров по индексу}
\NormalTok{param\_rows }\OtherTok{\textless{}{-}}\NormalTok{ out\_df[out\_df}\SpecialCharTok{$}\NormalTok{group }\SpecialCharTok{==} \StringTok{"param"}\NormalTok{, ]}
\NormalTok{param\_idx  }\OtherTok{\textless{}{-}} \FunctionTok{as.numeric}\NormalTok{(}\FunctionTok{sub}\NormalTok{(}\StringTok{".*}\SpecialCharTok{\textbackslash{}\textbackslash{}}\StringTok{[(}\SpecialCharTok{\textbackslash{}\textbackslash{}}\StringTok{d+)}\SpecialCharTok{\textbackslash{}\textbackslash{}}\StringTok{].*"}\NormalTok{, }\StringTok{"}\SpecialCharTok{\textbackslash{}\textbackslash{}}\StringTok{1"}\NormalTok{, param\_rows}\SpecialCharTok{$}\NormalTok{node))}
\NormalTok{param\_rows }\OtherTok{\textless{}{-}}\NormalTok{ param\_rows[}\FunctionTok{order}\NormalTok{(param\_idx), ]}

\CommentTok{\# Сортировка временных рядов по году}
\NormalTok{p1\_rows }\OtherTok{\textless{}{-}}\NormalTok{ out\_df[out\_df}\SpecialCharTok{$}\NormalTok{group }\SpecialCharTok{==} \StringTok{"P1"}\NormalTok{, ]}
\NormalTok{p1\_rows }\OtherTok{\textless{}{-}}\NormalTok{ p1\_rows[}\FunctionTok{order}\NormalTok{(p1\_rows}\SpecialCharTok{$}\NormalTok{YEAR), ]}

\NormalTok{r\_rows  }\OtherTok{\textless{}{-}}\NormalTok{ out\_df[out\_df}\SpecialCharTok{$}\NormalTok{group }\SpecialCharTok{==} \StringTok{"R"}\NormalTok{, ]}
\NormalTok{r\_rows  }\OtherTok{\textless{}{-}}\NormalTok{ r\_rows[}\FunctionTok{order}\NormalTok{(r\_rows}\SpecialCharTok{$}\NormalTok{YEAR), ]}

\NormalTok{p\_rows  }\OtherTok{\textless{}{-}}\NormalTok{ out\_df[out\_df}\SpecialCharTok{$}\NormalTok{group }\SpecialCharTok{==} \StringTok{"P"}\NormalTok{, ]}
\NormalTok{p\_rows  }\OtherTok{\textless{}{-}}\NormalTok{ p\_rows[}\FunctionTok{order}\NormalTok{(p\_rows}\SpecialCharTok{$}\NormalTok{YEAR), ]}

\CommentTok{\# Компоновка финальной таблицы}
\NormalTok{out\_df }\OtherTok{\textless{}{-}} \FunctionTok{rbind}\NormalTok{(param\_rows, p1\_rows, r\_rows, p\_rows)}
\NormalTok{out\_df}\SpecialCharTok{$}\NormalTok{group }\OtherTok{\textless{}{-}} \ConstantTok{NULL}  \CommentTok{\# Удаление вспомогательной колонки}

\CommentTok{\# Сохранение результатов}
\FunctionTok{write.csv}\NormalTok{(out\_df, }\StringTok{"monitor\_summary.csv"}\NormalTok{, }\AttributeTok{row.names =} \ConstantTok{FALSE}\NormalTok{)}
\FunctionTok{cat}\NormalTok{(}\StringTok{"Saved: monitor\_summary.csv}\SpecialCharTok{\textbackslash{}n}\StringTok{"}\NormalTok{)}
\end{Highlighting}
\end{Shaded}

\begin{verbatim}
Saved: monitor_summary.csv
\end{verbatim}

\begin{Shaded}
\begin{Highlighting}[]
\CommentTok{\# Вывод структуры результатов}
\FunctionTok{str}\NormalTok{(out\_df)}
\end{Highlighting}
\end{Shaded}

\begin{verbatim}
'data.frame':   61 obs. of  11 variables:
 $ YEAR               : num  NA NA NA NA NA NA NA NA NA NA ...
 $ #Vectors to monitor: chr  "x[1]<-M" "x[2]<-q1" "x[3]<-q2" "x[4]<-q3" ...
 $ node               : chr  "x[1]" "x[2]" "x[3]" "x[4]" ...
 $ mean               : num  0.17 0.425 0.745 0.934 0.315 ...
 $ sd                 : num  0.0646 0.0997 0.1412 0.0607 0.041 ...
 $ MC error           : num  0.002425 0.007459 0.010694 0.001508 0.000609 ...
 $ 2.50%              : num  0.0429 0.2586 0.4709 0.7747 0.2451 ...
 $ 25.00%             : num  0.127 0.353 0.641 0.906 0.286 ...
 $ median             : num  0.17 0.415 0.748 0.952 0.31 ...
 $ 75.00%             : num  0.212 0.488 0.857 0.98 0.34 ...
 $ 97.50%             : num  0.296 0.642 0.985 0.999 0.404 ...
\end{verbatim}

\begin{Shaded}
\begin{Highlighting}[]
\CommentTok{\# ========================================================================================================================}
\CommentTok{\# Визуализация априорных и апостериорных параметров}
\CommentTok{\# Параметры: M, Gp, Gr, Mp, q1, q2, q3, precP1, precR, precP, precbioindexP1, precbioindexR, precbioindexP}
\CommentTok{\# И производные: sigmaP1, sigmaR, sigmaP}
\CommentTok{\# ========================================================================================================================}

\CommentTok{\# Сэмплируем приоры прямо из той же JAGS{-}модели (без данных)}
\NormalTok{sample\_priors\_from\_model }\OtherTok{\textless{}{-}} \ControlFlowTok{function}\NormalTok{(model\_string, }\AttributeTok{n\_iter =} \DecValTok{20000}\NormalTok{, }\AttributeTok{n\_adapt =} \DecValTok{500}\NormalTok{) \{}
\NormalTok{  jm\_prior }\OtherTok{\textless{}{-}} \FunctionTok{jags.model}\NormalTok{(}\FunctionTok{textConnection}\NormalTok{(model\_string), }\AttributeTok{data =} \FunctionTok{list}\NormalTok{(}\AttributeTok{N =} \DecValTok{0}\NormalTok{), }\AttributeTok{n.chains =} \DecValTok{1}\NormalTok{, }\AttributeTok{n.adapt =}\NormalTok{ n\_adapt)}
\NormalTok{  vars }\OtherTok{\textless{}{-}} \FunctionTok{c}\NormalTok{(}\StringTok{"M"}\NormalTok{,}\StringTok{"Gp"}\NormalTok{,}\StringTok{"Gr"}\NormalTok{,}\StringTok{"Mp"}\NormalTok{,}\StringTok{"q1"}\NormalTok{,}\StringTok{"q2"}\NormalTok{,}\StringTok{"q3"}\NormalTok{,}
            \StringTok{"precP1"}\NormalTok{,}\StringTok{"precR"}\NormalTok{,}\StringTok{"precP"}\NormalTok{,}\StringTok{"precbioindexP1"}\NormalTok{,}\StringTok{"precbioindexR"}\NormalTok{,}\StringTok{"precbioindexP"}\NormalTok{)}
\NormalTok{  priors }\OtherTok{\textless{}{-}} \FunctionTok{coda.samples}\NormalTok{(jm\_prior, }\AttributeTok{variable.names =}\NormalTok{ vars, }\AttributeTok{n.iter =}\NormalTok{ n\_iter)}
  \FunctionTok{as.matrix}\NormalTok{(priors)}
\NormalTok{\}}

\CommentTok{\# Получаем матрицы приоров и постериоров}
\NormalTok{prior\_mat }\OtherTok{\textless{}{-}} \FunctionTok{sample\_priors\_from\_model}\NormalTok{(model\_string, }\AttributeTok{n\_iter =} \DecValTok{20000}\NormalTok{, }\AttributeTok{n\_adapt =} \DecValTok{500}\NormalTok{)}
\end{Highlighting}
\end{Shaded}

\begin{verbatim}
Compiling model graph
   Resolving undeclared variables
   Allocating nodes
Graph information:
   Observed stochastic nodes: 0
   Unobserved stochastic nodes: 16
   Total graph size: 33

Initializing model
\end{verbatim}

\begin{Shaded}
\begin{Highlighting}[]
\NormalTok{post\_mat  }\OtherTok{\textless{}{-}} \FunctionTok{as.matrix}\NormalTok{(samps)}

\CommentTok{\# Добавляем производные сигмы из прецизионов}
\NormalTok{add\_sigmas }\OtherTok{\textless{}{-}} \ControlFlowTok{function}\NormalTok{(mat) \{}
\NormalTok{  add }\OtherTok{\textless{}{-}} \ControlFlowTok{function}\NormalTok{(dst, src) \{}
    \ControlFlowTok{if}\NormalTok{ (}\FunctionTok{all}\NormalTok{(src }\SpecialCharTok{\%in\%} \FunctionTok{colnames}\NormalTok{(mat))) dst }\OtherTok{\textless{}{-}} \FunctionTok{cbind}\NormalTok{(dst, }\FunctionTok{setNames}\NormalTok{(}\FunctionTok{as.data.frame}\NormalTok{(}\FunctionTok{sqrt}\NormalTok{(}\DecValTok{1}\SpecialCharTok{/}\NormalTok{mat[, src, }\AttributeTok{drop=}\ConstantTok{FALSE}\NormalTok{])), }\FunctionTok{gsub}\NormalTok{(}\StringTok{"\^{}prec"}\NormalTok{,}\StringTok{"sigma"}\NormalTok{, src)))}
\NormalTok{    dst}
\NormalTok{  \}}
\NormalTok{  out }\OtherTok{\textless{}{-}}\NormalTok{ mat}
\NormalTok{  out }\OtherTok{\textless{}{-}} \FunctionTok{add}\NormalTok{(out, }\FunctionTok{c}\NormalTok{(}\StringTok{"precP1"}\NormalTok{))}
\NormalTok{  out }\OtherTok{\textless{}{-}} \FunctionTok{add}\NormalTok{(out, }\FunctionTok{c}\NormalTok{(}\StringTok{"precR"}\NormalTok{))}
\NormalTok{  out }\OtherTok{\textless{}{-}} \FunctionTok{add}\NormalTok{(out, }\FunctionTok{c}\NormalTok{(}\StringTok{"precP"}\NormalTok{))}
\NormalTok{  out}
\NormalTok{\}}
\NormalTok{prior\_mat }\OtherTok{\textless{}{-}} \FunctionTok{add\_sigmas}\NormalTok{(prior\_mat)}
\NormalTok{post\_mat  }\OtherTok{\textless{}{-}} \FunctionTok{add\_sigmas}\NormalTok{(post\_mat)}

\CommentTok{\# Список параметров для визуализации}
\NormalTok{params }\OtherTok{\textless{}{-}} \FunctionTok{intersect}\NormalTok{(}
  \FunctionTok{c}\NormalTok{(}\StringTok{"M"}\NormalTok{,}\StringTok{"Gp"}\NormalTok{,}\StringTok{"Gr"}\NormalTok{,}\StringTok{"Mp"}\NormalTok{,}\StringTok{"q1"}\NormalTok{,}\StringTok{"q2"}\NormalTok{,}\StringTok{"q3"}\NormalTok{,}
    \StringTok{"sigmaP1"}\NormalTok{,}\StringTok{"sigmaR"}\NormalTok{,}\StringTok{"sigmaP"}\NormalTok{,}
    \StringTok{"precbioindexP1"}\NormalTok{,}\StringTok{"precbioindexR"}\NormalTok{,}\StringTok{"precbioindexP"}\NormalTok{),}
  \FunctionTok{union}\NormalTok{(}\FunctionTok{colnames}\NormalTok{(prior\_mat), }\FunctionTok{colnames}\NormalTok{(post\_mat))}
\NormalTok{)}

\CommentTok{\# В long{-}формат}
\NormalTok{mk\_df }\OtherTok{\textless{}{-}} \ControlFlowTok{function}\NormalTok{(mat, label) \{}
  \ControlFlowTok{if}\NormalTok{ (}\FunctionTok{is.null}\NormalTok{(mat) }\SpecialCharTok{||} \FunctionTok{nrow}\NormalTok{(mat) }\SpecialCharTok{==} \DecValTok{0}\NormalTok{) }\FunctionTok{return}\NormalTok{(}\FunctionTok{data.frame}\NormalTok{())}
\NormalTok{  mat }\OtherTok{\textless{}{-}}\NormalTok{ mat[, }\FunctionTok{intersect}\NormalTok{(}\FunctionTok{colnames}\NormalTok{(mat), params), drop }\OtherTok{=} \ConstantTok{FALSE}\NormalTok{]}
  \FunctionTok{reshape}\NormalTok{(}
    \FunctionTok{data.frame}\NormalTok{(}\AttributeTok{iter =} \FunctionTok{seq\_len}\NormalTok{(}\FunctionTok{nrow}\NormalTok{(mat)), mat, }\AttributeTok{check.names =} \ConstantTok{FALSE}\NormalTok{),}
    \AttributeTok{direction =} \StringTok{"long"}\NormalTok{, }\AttributeTok{varying =}\NormalTok{ params, }\AttributeTok{v.names =} \StringTok{"value"}\NormalTok{, }\AttributeTok{timevar =} \StringTok{"param"}\NormalTok{, }\AttributeTok{times =}\NormalTok{ params}
\NormalTok{  )[, }\FunctionTok{c}\NormalTok{(}\StringTok{"param"}\NormalTok{,}\StringTok{"value"}\NormalTok{)]}
\NormalTok{\}}
\NormalTok{prior\_df }\OtherTok{\textless{}{-}} \FunctionTok{mk\_df}\NormalTok{(prior\_mat, }\StringTok{"Prior"}\NormalTok{); prior\_df}\SpecialCharTok{$}\NormalTok{dist }\OtherTok{\textless{}{-}} \StringTok{"Prior"}
\NormalTok{post\_df  }\OtherTok{\textless{}{-}} \FunctionTok{mk\_df}\NormalTok{(post\_mat,  }\StringTok{"Posterior"}\NormalTok{); post\_df}\SpecialCharTok{$}\NormalTok{dist }\OtherTok{\textless{}{-}} \StringTok{"Posterior"}
\NormalTok{plot\_df  }\OtherTok{\textless{}{-}} \FunctionTok{rbind}\NormalTok{(prior\_df, post\_df)}

\CommentTok{\# Подписи}
\NormalTok{param\_labels }\OtherTok{\textless{}{-}} \FunctionTok{c}\NormalTok{(}
  \AttributeTok{M=}\StringTok{"M (mortality)"}\NormalTok{, }\AttributeTok{Gp=}\StringTok{"Gp"}\NormalTok{, }\AttributeTok{Gr=}\StringTok{"Gr"}\NormalTok{, }\AttributeTok{Mp=}\StringTok{"Mp"}\NormalTok{,}
  \AttributeTok{q1=}\StringTok{"q1"}\NormalTok{, }\AttributeTok{q2=}\StringTok{"q2"}\NormalTok{, }\AttributeTok{q3=}\StringTok{"q3"}\NormalTok{,}
  \AttributeTok{sigmaP1=}\StringTok{"sigmaP1"}\NormalTok{, }\AttributeTok{sigmaR=}\StringTok{"sigmaR"}\NormalTok{, }\AttributeTok{sigmaP=}\StringTok{"sigmaP"}\NormalTok{,}
  \AttributeTok{precbioindexP1=}\StringTok{"precbioindexP1"}\NormalTok{, }\AttributeTok{precbioindexR=}\StringTok{"precbioindexR"}\NormalTok{, }\AttributeTok{precbioindexP=}\StringTok{"precbioindexP"}
\NormalTok{)}
\NormalTok{plot\_df}\SpecialCharTok{$}\NormalTok{param\_f }\OtherTok{\textless{}{-}} \FunctionTok{factor}\NormalTok{(plot\_df}\SpecialCharTok{$}\NormalTok{param, }\AttributeTok{levels =}\NormalTok{ params, }\AttributeTok{labels =} \FunctionTok{unname}\NormalTok{(param\_labels[params]))}

\CommentTok{\# График prior vs posterior (берёт priors из модели!)}
\FunctionTok{library}\NormalTok{(ggplot2)}
\FunctionTok{ggplot}\NormalTok{(plot\_df, }\FunctionTok{aes}\NormalTok{(}\AttributeTok{x =}\NormalTok{ value, }\AttributeTok{color =}\NormalTok{ dist, }\AttributeTok{fill =}\NormalTok{ dist)) }\SpecialCharTok{+}
  \FunctionTok{geom\_density}\NormalTok{(}\AttributeTok{alpha =} \FloatTok{0.25}\NormalTok{, }\AttributeTok{linewidth =} \FloatTok{0.7}\NormalTok{) }\SpecialCharTok{+}
  \FunctionTok{facet\_wrap}\NormalTok{(}\SpecialCharTok{\textasciitilde{}}\NormalTok{ param\_f, }\AttributeTok{scales =} \StringTok{"free"}\NormalTok{, }\AttributeTok{ncol =} \DecValTok{4}\NormalTok{) }\SpecialCharTok{+}
  \FunctionTok{scale\_color\_manual}\NormalTok{(}\AttributeTok{values =} \FunctionTok{c}\NormalTok{(}\StringTok{"Prior"} \OtherTok{=} \StringTok{"\#999999"}\NormalTok{, }\StringTok{"Posterior"} \OtherTok{=} \StringTok{"\#1b9e77"}\NormalTok{)) }\SpecialCharTok{+}
  \FunctionTok{scale\_fill\_manual}\NormalTok{(}\AttributeTok{values  =} \FunctionTok{c}\NormalTok{(}\StringTok{"Prior"} \OtherTok{=} \StringTok{"\#bbbbbb"}\NormalTok{, }\StringTok{"Posterior"} \OtherTok{=} \StringTok{"\#1b9e77"}\NormalTok{)) }\SpecialCharTok{+}
  \FunctionTok{labs}\NormalTok{(}\AttributeTok{title =} \StringTok{"Априорные (из модели) vs апостериорные распределения"}\NormalTok{,}
       \AttributeTok{x =} \StringTok{"Значение"}\NormalTok{, }\AttributeTok{y =} \StringTok{"Плотность"}\NormalTok{, }\AttributeTok{color =} \StringTok{""}\NormalTok{, }\AttributeTok{fill =} \StringTok{""}\NormalTok{) }\SpecialCharTok{+}
  \FunctionTok{theme\_minimal}\NormalTok{(}\AttributeTok{base\_size =} \DecValTok{12}\NormalTok{) }\SpecialCharTok{+}
  \FunctionTok{theme}\NormalTok{(}\AttributeTok{legend.position =} \StringTok{"top"}\NormalTok{)}
\end{Highlighting}
\end{Shaded}

\begin{verbatim}
Warning in grid.Call(C_textBounds, as.graphicsAnnot(x$label), x$x, x$y, :
неизвестна ширина символа 0xcf в кодировке CP1251
\end{verbatim}

\begin{verbatim}
Warning in grid.Call(C_textBounds, as.graphicsAnnot(x$label), x$x, x$y, :
неизвестна ширина символа 0xeb в кодировке CP1251
\end{verbatim}

\begin{verbatim}
Warning in grid.Call(C_textBounds, as.graphicsAnnot(x$label), x$x, x$y, :
неизвестна ширина символа 0xee в кодировке CP1251
\end{verbatim}

\begin{verbatim}
Warning in grid.Call(C_textBounds, as.graphicsAnnot(x$label), x$x, x$y, :
неизвестна ширина символа 0xf2 в кодировке CP1251
\end{verbatim}

\begin{verbatim}
Warning in grid.Call(C_textBounds, as.graphicsAnnot(x$label), x$x, x$y, :
неизвестна ширина символа 0xed в кодировке CP1251
\end{verbatim}

\begin{verbatim}
Warning in grid.Call(C_textBounds, as.graphicsAnnot(x$label), x$x, x$y, :
неизвестна ширина символа 0xee в кодировке CP1251
\end{verbatim}

\begin{verbatim}
Warning in grid.Call(C_textBounds, as.graphicsAnnot(x$label), x$x, x$y, :
неизвестна ширина символа 0xf1 в кодировке CP1251
\end{verbatim}

\begin{verbatim}
Warning in grid.Call(C_textBounds, as.graphicsAnnot(x$label), x$x, x$y, :
неизвестна ширина символа 0xf2 в кодировке CP1251
\end{verbatim}

\begin{verbatim}
Warning in grid.Call(C_textBounds, as.graphicsAnnot(x$label), x$x, x$y, :
неизвестна ширина символа 0xfc в кодировке CP1251
\end{verbatim}

\begin{verbatim}
Warning in grid.Call(C_textBounds, as.graphicsAnnot(x$label), x$x, x$y, :
неизвестна ширина символа 0xc0 в кодировке CP1251
\end{verbatim}

\begin{verbatim}
Warning in grid.Call(C_textBounds, as.graphicsAnnot(x$label), x$x, x$y, :
неизвестна ширина символа 0xef в кодировке CP1251
\end{verbatim}

\begin{verbatim}
Warning in grid.Call(C_textBounds, as.graphicsAnnot(x$label), x$x, x$y, :
неизвестна ширина символа 0xf0 в кодировке CP1251
\end{verbatim}

\begin{verbatim}
Warning in grid.Call(C_textBounds, as.graphicsAnnot(x$label), x$x, x$y, :
неизвестна ширина символа 0xe8 в кодировке CP1251
\end{verbatim}

\begin{verbatim}
Warning in grid.Call(C_textBounds, as.graphicsAnnot(x$label), x$x, x$y, :
неизвестна ширина символа 0xee в кодировке CP1251
\end{verbatim}

\begin{verbatim}
Warning in grid.Call(C_textBounds, as.graphicsAnnot(x$label), x$x, x$y, :
неизвестна ширина символа 0xf0 в кодировке CP1251
\end{verbatim}

\begin{verbatim}
Warning in grid.Call(C_textBounds, as.graphicsAnnot(x$label), x$x, x$y, :
неизвестна ширина символа 0xed в кодировке CP1251
\end{verbatim}

\begin{verbatim}
Warning in grid.Call(C_textBounds, as.graphicsAnnot(x$label), x$x, x$y, :
неизвестна ширина символа 0xfb в кодировке CP1251
\end{verbatim}

\begin{verbatim}
Warning in grid.Call(C_textBounds, as.graphicsAnnot(x$label), x$x, x$y, :
неизвестна ширина символа 0xe5 в кодировке CP1251
\end{verbatim}

\begin{verbatim}
Warning in grid.Call(C_textBounds, as.graphicsAnnot(x$label), x$x, x$y, :
неизвестна ширина символа 0xe8 в кодировке CP1251
\end{verbatim}

\begin{verbatim}
Warning in grid.Call(C_textBounds, as.graphicsAnnot(x$label), x$x, x$y, :
неизвестна ширина символа 0xe7 в кодировке CP1251
\end{verbatim}

\begin{verbatim}
Warning in grid.Call(C_textBounds, as.graphicsAnnot(x$label), x$x, x$y, :
неизвестна ширина символа 0xec в кодировке CP1251
\end{verbatim}

\begin{verbatim}
Warning in grid.Call(C_textBounds, as.graphicsAnnot(x$label), x$x, x$y, :
неизвестна ширина символа 0xee в кодировке CP1251
\end{verbatim}

\begin{verbatim}
Warning in grid.Call(C_textBounds, as.graphicsAnnot(x$label), x$x, x$y, :
неизвестна ширина символа 0xe4 в кодировке CP1251
\end{verbatim}

\begin{verbatim}
Warning in grid.Call(C_textBounds, as.graphicsAnnot(x$label), x$x, x$y, :
неизвестна ширина символа 0xe5 в кодировке CP1251
\end{verbatim}

\begin{verbatim}
Warning in grid.Call(C_textBounds, as.graphicsAnnot(x$label), x$x, x$y, :
неизвестна ширина символа 0xeb в кодировке CP1251
\end{verbatim}

\begin{verbatim}
Warning in grid.Call(C_textBounds, as.graphicsAnnot(x$label), x$x, x$y, :
неизвестна ширина символа 0xe8 в кодировке CP1251
\end{verbatim}

\begin{verbatim}
Warning in grid.Call(C_textBounds, as.graphicsAnnot(x$label), x$x, x$y, :
неизвестна ширина символа 0xe0 в кодировке CP1251
\end{verbatim}

\begin{verbatim}
Warning in grid.Call(C_textBounds, as.graphicsAnnot(x$label), x$x, x$y, :
неизвестна ширина символа 0xef в кодировке CP1251
\end{verbatim}

\begin{verbatim}
Warning in grid.Call(C_textBounds, as.graphicsAnnot(x$label), x$x, x$y, :
неизвестна ширина символа 0xee в кодировке CP1251
\end{verbatim}

\begin{verbatim}
Warning in grid.Call(C_textBounds, as.graphicsAnnot(x$label), x$x, x$y, :
неизвестна ширина символа 0xf1 в кодировке CP1251
\end{verbatim}

\begin{verbatim}
Warning in grid.Call(C_textBounds, as.graphicsAnnot(x$label), x$x, x$y, :
неизвестна ширина символа 0xf2 в кодировке CP1251
\end{verbatim}

\begin{verbatim}
Warning in grid.Call(C_textBounds, as.graphicsAnnot(x$label), x$x, x$y, :
неизвестна ширина символа 0xe5 в кодировке CP1251
\end{verbatim}

\begin{verbatim}
Warning in grid.Call(C_textBounds, as.graphicsAnnot(x$label), x$x, x$y, :
неизвестна ширина символа 0xf0 в кодировке CP1251
\end{verbatim}

\begin{verbatim}
Warning in grid.Call(C_textBounds, as.graphicsAnnot(x$label), x$x, x$y, :
неизвестна ширина символа 0xe8 в кодировке CP1251
\end{verbatim}

\begin{verbatim}
Warning in grid.Call(C_textBounds, as.graphicsAnnot(x$label), x$x, x$y, :
неизвестна ширина символа 0xee в кодировке CP1251
\end{verbatim}

\begin{verbatim}
Warning in grid.Call(C_textBounds, as.graphicsAnnot(x$label), x$x, x$y, :
неизвестна ширина символа 0xf0 в кодировке CP1251
\end{verbatim}

\begin{verbatim}
Warning in grid.Call(C_textBounds, as.graphicsAnnot(x$label), x$x, x$y, :
неизвестна ширина символа 0xed в кодировке CP1251
\end{verbatim}

\begin{verbatim}
Warning in grid.Call(C_textBounds, as.graphicsAnnot(x$label), x$x, x$y, :
неизвестна ширина символа 0xfb в кодировке CP1251
\end{verbatim}

\begin{verbatim}
Warning in grid.Call(C_textBounds, as.graphicsAnnot(x$label), x$x, x$y, :
неизвестна ширина символа 0xe5 в кодировке CP1251
\end{verbatim}

\begin{verbatim}
Warning in grid.Call(C_textBounds, as.graphicsAnnot(x$label), x$x, x$y, :
неизвестна ширина символа 0xf0 в кодировке CP1251
\end{verbatim}

\begin{verbatim}
Warning in grid.Call(C_textBounds, as.graphicsAnnot(x$label), x$x, x$y, :
неизвестна ширина символа 0xe0 в кодировке CP1251
\end{verbatim}

\begin{verbatim}
Warning in grid.Call(C_textBounds, as.graphicsAnnot(x$label), x$x, x$y, :
неизвестна ширина символа 0xf1 в кодировке CP1251
\end{verbatim}

\begin{verbatim}
Warning in grid.Call(C_textBounds, as.graphicsAnnot(x$label), x$x, x$y, :
неизвестна ширина символа 0xef в кодировке CP1251
\end{verbatim}

\begin{verbatim}
Warning in grid.Call(C_textBounds, as.graphicsAnnot(x$label), x$x, x$y, :
неизвестна ширина символа 0xf0 в кодировке CP1251
\end{verbatim}

\begin{verbatim}
Warning in grid.Call(C_textBounds, as.graphicsAnnot(x$label), x$x, x$y, :
неизвестна ширина символа 0xe5 в кодировке CP1251
\end{verbatim}

\begin{verbatim}
Warning in grid.Call(C_textBounds, as.graphicsAnnot(x$label), x$x, x$y, :
неизвестна ширина символа 0xe4 в кодировке CP1251
\end{verbatim}

\begin{verbatim}
Warning in grid.Call(C_textBounds, as.graphicsAnnot(x$label), x$x, x$y, :
неизвестна ширина символа 0xe5 в кодировке CP1251
\end{verbatim}

\begin{verbatim}
Warning in grid.Call(C_textBounds, as.graphicsAnnot(x$label), x$x, x$y, :
неизвестна ширина символа 0xeb в кодировке CP1251
\end{verbatim}

\begin{verbatim}
Warning in grid.Call(C_textBounds, as.graphicsAnnot(x$label), x$x, x$y, :
неизвестна ширина символа 0xe5 в кодировке CP1251
\end{verbatim}

\begin{verbatim}
Warning in grid.Call(C_textBounds, as.graphicsAnnot(x$label), x$x, x$y, :
неизвестна ширина символа 0xed в кодировке CP1251
\end{verbatim}

\begin{verbatim}
Warning in grid.Call(C_textBounds, as.graphicsAnnot(x$label), x$x, x$y, :
неизвестна ширина символа 0xe8 в кодировке CP1251
\end{verbatim}

\begin{verbatim}
Warning in grid.Call(C_textBounds, as.graphicsAnnot(x$label), x$x, x$y, :
неизвестна ширина символа 0xff в кодировке CP1251
\end{verbatim}

\begin{verbatim}
Warning in grid.Call(C_textBounds, as.graphicsAnnot(x$label), x$x, x$y, :
неизвестна ширина символа 0xc7 в кодировке CP1251
\end{verbatim}

\begin{verbatim}
Warning in grid.Call(C_textBounds, as.graphicsAnnot(x$label), x$x, x$y, :
неизвестна ширина символа 0xed в кодировке CP1251
\end{verbatim}

\begin{verbatim}
Warning in grid.Call(C_textBounds, as.graphicsAnnot(x$label), x$x, x$y, :
неизвестна ширина символа 0xe0 в кодировке CP1251
\end{verbatim}

\begin{verbatim}
Warning in grid.Call(C_textBounds, as.graphicsAnnot(x$label), x$x, x$y, :
неизвестна ширина символа 0xf7 в кодировке CP1251
\end{verbatim}

\begin{verbatim}
Warning in grid.Call(C_textBounds, as.graphicsAnnot(x$label), x$x, x$y, :
неизвестна ширина символа 0xe5 в кодировке CP1251
\end{verbatim}

\begin{verbatim}
Warning in grid.Call(C_textBounds, as.graphicsAnnot(x$label), x$x, x$y, :
неизвестна ширина символа 0xed в кодировке CP1251
\end{verbatim}

\begin{verbatim}
Warning in grid.Call(C_textBounds, as.graphicsAnnot(x$label), x$x, x$y, :
неизвестна ширина символа 0xe8 в кодировке CP1251
\end{verbatim}

\begin{verbatim}
Warning in grid.Call(C_textBounds, as.graphicsAnnot(x$label), x$x, x$y, :
неизвестна ширина символа 0xe5 в кодировке CP1251
\end{verbatim}

\begin{verbatim}
Warning in grid.Call.graphics(C_text, as.graphicsAnnot(x$label), x$x, x$y, :
неизвестна ширина символа 0xc7 в кодировке CP1251
\end{verbatim}

\begin{verbatim}
Warning in grid.Call.graphics(C_text, as.graphicsAnnot(x$label), x$x, x$y, :
неизвестна ширина символа 0xed в кодировке CP1251
\end{verbatim}

\begin{verbatim}
Warning in grid.Call.graphics(C_text, as.graphicsAnnot(x$label), x$x, x$y, :
неизвестна ширина символа 0xe0 в кодировке CP1251
\end{verbatim}

\begin{verbatim}
Warning in grid.Call.graphics(C_text, as.graphicsAnnot(x$label), x$x, x$y, :
неизвестна ширина символа 0xf7 в кодировке CP1251
\end{verbatim}

\begin{verbatim}
Warning in grid.Call.graphics(C_text, as.graphicsAnnot(x$label), x$x, x$y, :
неизвестна ширина символа 0xe5 в кодировке CP1251
\end{verbatim}

\begin{verbatim}
Warning in grid.Call.graphics(C_text, as.graphicsAnnot(x$label), x$x, x$y, :
неизвестна ширина символа 0xed в кодировке CP1251
\end{verbatim}

\begin{verbatim}
Warning in grid.Call.graphics(C_text, as.graphicsAnnot(x$label), x$x, x$y, :
неизвестна ширина символа 0xe8 в кодировке CP1251
\end{verbatim}

\begin{verbatim}
Warning in grid.Call.graphics(C_text, as.graphicsAnnot(x$label), x$x, x$y, :
неизвестна ширина символа 0xe5 в кодировке CP1251
\end{verbatim}

\begin{verbatim}
Warning in grid.Call.graphics(C_text, as.graphicsAnnot(x$label), x$x, x$y, :
неизвестна ширина символа 0xcf в кодировке CP1251
\end{verbatim}

\begin{verbatim}
Warning in grid.Call.graphics(C_text, as.graphicsAnnot(x$label), x$x, x$y, :
неизвестна ширина символа 0xeb в кодировке CP1251
\end{verbatim}

\begin{verbatim}
Warning in grid.Call.graphics(C_text, as.graphicsAnnot(x$label), x$x, x$y, :
неизвестна ширина символа 0xee в кодировке CP1251
\end{verbatim}

\begin{verbatim}
Warning in grid.Call.graphics(C_text, as.graphicsAnnot(x$label), x$x, x$y, :
неизвестна ширина символа 0xf2 в кодировке CP1251
\end{verbatim}

\begin{verbatim}
Warning in grid.Call.graphics(C_text, as.graphicsAnnot(x$label), x$x, x$y, :
неизвестна ширина символа 0xed в кодировке CP1251
\end{verbatim}

\begin{verbatim}
Warning in grid.Call.graphics(C_text, as.graphicsAnnot(x$label), x$x, x$y, :
неизвестна ширина символа 0xee в кодировке CP1251
\end{verbatim}

\begin{verbatim}
Warning in grid.Call.graphics(C_text, as.graphicsAnnot(x$label), x$x, x$y, :
неизвестна ширина символа 0xf1 в кодировке CP1251
\end{verbatim}

\begin{verbatim}
Warning in grid.Call.graphics(C_text, as.graphicsAnnot(x$label), x$x, x$y, :
неизвестна ширина символа 0xf2 в кодировке CP1251
\end{verbatim}

\begin{verbatim}
Warning in grid.Call.graphics(C_text, as.graphicsAnnot(x$label), x$x, x$y, :
неизвестна ширина символа 0xfc в кодировке CP1251
\end{verbatim}

\begin{verbatim}
Warning in grid.Call.graphics(C_text, as.graphicsAnnot(x$label), x$x, x$y, :
неизвестна ширина символа 0xc0 в кодировке CP1251
\end{verbatim}

\begin{verbatim}
Warning in grid.Call.graphics(C_text, as.graphicsAnnot(x$label), x$x, x$y, :
неизвестна ширина символа 0xef в кодировке CP1251
\end{verbatim}

\begin{verbatim}
Warning in grid.Call.graphics(C_text, as.graphicsAnnot(x$label), x$x, x$y, :
неизвестна ширина символа 0xf0 в кодировке CP1251
\end{verbatim}

\begin{verbatim}
Warning in grid.Call.graphics(C_text, as.graphicsAnnot(x$label), x$x, x$y, :
неизвестна ширина символа 0xe8 в кодировке CP1251
\end{verbatim}

\begin{verbatim}
Warning in grid.Call.graphics(C_text, as.graphicsAnnot(x$label), x$x, x$y, :
неизвестна ширина символа 0xee в кодировке CP1251
\end{verbatim}

\begin{verbatim}
Warning in grid.Call.graphics(C_text, as.graphicsAnnot(x$label), x$x, x$y, :
неизвестна ширина символа 0xf0 в кодировке CP1251
\end{verbatim}

\begin{verbatim}
Warning in grid.Call.graphics(C_text, as.graphicsAnnot(x$label), x$x, x$y, :
неизвестна ширина символа 0xed в кодировке CP1251
\end{verbatim}

\begin{verbatim}
Warning in grid.Call.graphics(C_text, as.graphicsAnnot(x$label), x$x, x$y, :
неизвестна ширина символа 0xfb в кодировке CP1251
\end{verbatim}

\begin{verbatim}
Warning in grid.Call.graphics(C_text, as.graphicsAnnot(x$label), x$x, x$y, :
неизвестна ширина символа 0xe5 в кодировке CP1251
\end{verbatim}

\begin{verbatim}
Warning in grid.Call.graphics(C_text, as.graphicsAnnot(x$label), x$x, x$y, :
неизвестна ширина символа 0xe8 в кодировке CP1251
\end{verbatim}

\begin{verbatim}
Warning in grid.Call.graphics(C_text, as.graphicsAnnot(x$label), x$x, x$y, :
неизвестна ширина символа 0xe7 в кодировке CP1251
\end{verbatim}

\begin{verbatim}
Warning in grid.Call.graphics(C_text, as.graphicsAnnot(x$label), x$x, x$y, :
неизвестна ширина символа 0xec в кодировке CP1251
\end{verbatim}

\begin{verbatim}
Warning in grid.Call.graphics(C_text, as.graphicsAnnot(x$label), x$x, x$y, :
неизвестна ширина символа 0xee в кодировке CP1251
\end{verbatim}

\begin{verbatim}
Warning in grid.Call.graphics(C_text, as.graphicsAnnot(x$label), x$x, x$y, :
неизвестна ширина символа 0xe4 в кодировке CP1251
\end{verbatim}

\begin{verbatim}
Warning in grid.Call.graphics(C_text, as.graphicsAnnot(x$label), x$x, x$y, :
неизвестна ширина символа 0xe5 в кодировке CP1251
\end{verbatim}

\begin{verbatim}
Warning in grid.Call.graphics(C_text, as.graphicsAnnot(x$label), x$x, x$y, :
неизвестна ширина символа 0xeb в кодировке CP1251
\end{verbatim}

\begin{verbatim}
Warning in grid.Call.graphics(C_text, as.graphicsAnnot(x$label), x$x, x$y, :
неизвестна ширина символа 0xe8 в кодировке CP1251
\end{verbatim}

\begin{verbatim}
Warning in grid.Call.graphics(C_text, as.graphicsAnnot(x$label), x$x, x$y, :
неизвестна ширина символа 0xe0 в кодировке CP1251
\end{verbatim}

\begin{verbatim}
Warning in grid.Call.graphics(C_text, as.graphicsAnnot(x$label), x$x, x$y, :
неизвестна ширина символа 0xef в кодировке CP1251
\end{verbatim}

\begin{verbatim}
Warning in grid.Call.graphics(C_text, as.graphicsAnnot(x$label), x$x, x$y, :
неизвестна ширина символа 0xee в кодировке CP1251
\end{verbatim}

\begin{verbatim}
Warning in grid.Call.graphics(C_text, as.graphicsAnnot(x$label), x$x, x$y, :
неизвестна ширина символа 0xf1 в кодировке CP1251
\end{verbatim}

\begin{verbatim}
Warning in grid.Call.graphics(C_text, as.graphicsAnnot(x$label), x$x, x$y, :
неизвестна ширина символа 0xf2 в кодировке CP1251
\end{verbatim}

\begin{verbatim}
Warning in grid.Call.graphics(C_text, as.graphicsAnnot(x$label), x$x, x$y, :
неизвестна ширина символа 0xe5 в кодировке CP1251
\end{verbatim}

\begin{verbatim}
Warning in grid.Call.graphics(C_text, as.graphicsAnnot(x$label), x$x, x$y, :
неизвестна ширина символа 0xf0 в кодировке CP1251
\end{verbatim}

\begin{verbatim}
Warning in grid.Call.graphics(C_text, as.graphicsAnnot(x$label), x$x, x$y, :
неизвестна ширина символа 0xe8 в кодировке CP1251
\end{verbatim}

\begin{verbatim}
Warning in grid.Call.graphics(C_text, as.graphicsAnnot(x$label), x$x, x$y, :
неизвестна ширина символа 0xee в кодировке CP1251
\end{verbatim}

\begin{verbatim}
Warning in grid.Call.graphics(C_text, as.graphicsAnnot(x$label), x$x, x$y, :
неизвестна ширина символа 0xf0 в кодировке CP1251
\end{verbatim}

\begin{verbatim}
Warning in grid.Call.graphics(C_text, as.graphicsAnnot(x$label), x$x, x$y, :
неизвестна ширина символа 0xed в кодировке CP1251
\end{verbatim}

\begin{verbatim}
Warning in grid.Call.graphics(C_text, as.graphicsAnnot(x$label), x$x, x$y, :
неизвестна ширина символа 0xfb в кодировке CP1251
\end{verbatim}

\begin{verbatim}
Warning in grid.Call.graphics(C_text, as.graphicsAnnot(x$label), x$x, x$y, :
неизвестна ширина символа 0xe5 в кодировке CP1251
\end{verbatim}

\begin{verbatim}
Warning in grid.Call.graphics(C_text, as.graphicsAnnot(x$label), x$x, x$y, :
неизвестна ширина символа 0xf0 в кодировке CP1251
\end{verbatim}

\begin{verbatim}
Warning in grid.Call.graphics(C_text, as.graphicsAnnot(x$label), x$x, x$y, :
неизвестна ширина символа 0xe0 в кодировке CP1251
\end{verbatim}

\begin{verbatim}
Warning in grid.Call.graphics(C_text, as.graphicsAnnot(x$label), x$x, x$y, :
неизвестна ширина символа 0xf1 в кодировке CP1251
\end{verbatim}

\begin{verbatim}
Warning in grid.Call.graphics(C_text, as.graphicsAnnot(x$label), x$x, x$y, :
неизвестна ширина символа 0xef в кодировке CP1251
\end{verbatim}

\begin{verbatim}
Warning in grid.Call.graphics(C_text, as.graphicsAnnot(x$label), x$x, x$y, :
неизвестна ширина символа 0xf0 в кодировке CP1251
\end{verbatim}

\begin{verbatim}
Warning in grid.Call.graphics(C_text, as.graphicsAnnot(x$label), x$x, x$y, :
неизвестна ширина символа 0xe5 в кодировке CP1251
\end{verbatim}

\begin{verbatim}
Warning in grid.Call.graphics(C_text, as.graphicsAnnot(x$label), x$x, x$y, :
неизвестна ширина символа 0xe4 в кодировке CP1251
\end{verbatim}

\begin{verbatim}
Warning in grid.Call.graphics(C_text, as.graphicsAnnot(x$label), x$x, x$y, :
неизвестна ширина символа 0xe5 в кодировке CP1251
\end{verbatim}

\begin{verbatim}
Warning in grid.Call.graphics(C_text, as.graphicsAnnot(x$label), x$x, x$y, :
неизвестна ширина символа 0xeb в кодировке CP1251
\end{verbatim}

\begin{verbatim}
Warning in grid.Call.graphics(C_text, as.graphicsAnnot(x$label), x$x, x$y, :
неизвестна ширина символа 0xe5 в кодировке CP1251
\end{verbatim}

\begin{verbatim}
Warning in grid.Call.graphics(C_text, as.graphicsAnnot(x$label), x$x, x$y, :
неизвестна ширина символа 0xed в кодировке CP1251
\end{verbatim}

\begin{verbatim}
Warning in grid.Call.graphics(C_text, as.graphicsAnnot(x$label), x$x, x$y, :
неизвестна ширина символа 0xe8 в кодировке CP1251
\end{verbatim}

\begin{verbatim}
Warning in grid.Call.graphics(C_text, as.graphicsAnnot(x$label), x$x, x$y, :
неизвестна ширина символа 0xff в кодировке CP1251
\end{verbatim}

\pandocbounded{\includegraphics[keepaspectratio]{chapter8_files/figure-pdf/unnamed-chunk-1-1.pdf}}

\begin{Shaded}
\begin{Highlighting}[]
\CommentTok{\# ========================================================================================================================}
\CommentTok{\# Бабл{-}плоты остатков P1, R, P по годам (2000–2015)}
\CommentTok{\# Требуется: объекты samps, data\_list. Если YEAR не создан, создадим.}
\CommentTok{\# ========================================================================================================================}

\ControlFlowTok{if}\NormalTok{ (}\SpecialCharTok{!}\FunctionTok{exists}\NormalTok{(}\StringTok{"YEAR"}\NormalTok{)) YEAR }\OtherTok{\textless{}{-}} \DecValTok{2000} \SpecialCharTok{+} \DecValTok{0}\SpecialCharTok{:}\NormalTok{(data\_list}\SpecialCharTok{$}\NormalTok{N }\SpecialCharTok{{-}} \DecValTok{1}\NormalTok{)}
\NormalTok{draws\_mat }\OtherTok{\textless{}{-}} \FunctionTok{as.matrix}\NormalTok{(samps)}
\NormalTok{eps }\OtherTok{\textless{}{-}} \FloatTok{1.0e{-}6}

\NormalTok{resid\_bubble\_summary }\OtherTok{\textless{}{-}} \ControlFlowTok{function}\NormalTok{(series, obs\_vec, q\_name, state\_name\_prefix) \{}
\NormalTok{  rows }\OtherTok{\textless{}{-}} \FunctionTok{list}\NormalTok{()}
  \ControlFlowTok{for}\NormalTok{ (i }\ControlFlowTok{in} \FunctionTok{seq\_along}\NormalTok{(obs\_vec)) \{}
    \ControlFlowTok{if}\NormalTok{ (}\FunctionTok{is.na}\NormalTok{(obs\_vec[i])) }\ControlFlowTok{next}
\NormalTok{    q\_draws     }\OtherTok{\textless{}{-}}\NormalTok{ draws\_mat[, q\_name]}
\NormalTok{    state\_draws }\OtherTok{\textless{}{-}}\NormalTok{ draws\_mat[, }\FunctionTok{paste0}\NormalTok{(state\_name\_prefix, }\StringTok{"["}\NormalTok{, i, }\StringTok{"]"}\NormalTok{)]}
    \CommentTok{\# residual per draw: log(observed) {-} log(expected)}
\NormalTok{    res\_draws }\OtherTok{\textless{}{-}} \FunctionTok{log}\NormalTok{(obs\_vec[i]) }\SpecialCharTok{{-}} \FunctionTok{log}\NormalTok{(eps }\SpecialCharTok{+}\NormalTok{ q\_draws }\SpecialCharTok{*}\NormalTok{ state\_draws)}
\NormalTok{    r\_mean }\OtherTok{\textless{}{-}} \FunctionTok{mean}\NormalTok{(res\_draws, }\AttributeTok{na.rm =} \ConstantTok{TRUE}\NormalTok{)}
\NormalTok{    rows[[}\FunctionTok{length}\NormalTok{(rows) }\SpecialCharTok{+} \DecValTok{1}\NormalTok{]] }\OtherTok{\textless{}{-}} \FunctionTok{data.frame}\NormalTok{(}
      \AttributeTok{YEAR =}\NormalTok{ YEAR[i],}
      \AttributeTok{series =}\NormalTok{ series,}
      \AttributeTok{resid =}\NormalTok{ r\_mean,}
      \AttributeTok{abs\_resid =} \FunctionTok{abs}\NormalTok{(r\_mean),}
      \AttributeTok{sign =} \FunctionTok{ifelse}\NormalTok{(r\_mean }\SpecialCharTok{\textgreater{}=} \DecValTok{0}\NormalTok{, }\StringTok{"pos"}\NormalTok{, }\StringTok{"neg"}\NormalTok{)}
\NormalTok{    )}
\NormalTok{  \}}
  \FunctionTok{do.call}\NormalTok{(rbind, rows)}
\NormalTok{\}}

\NormalTok{b1 }\OtherTok{\textless{}{-}} \FunctionTok{resid\_bubble\_summary}\NormalTok{(}\StringTok{"P1"}\NormalTok{, data\_list}\SpecialCharTok{$}\NormalTok{bioindexP1, }\StringTok{"q1"}\NormalTok{, }\StringTok{"P1"}\NormalTok{)}
\NormalTok{b2 }\OtherTok{\textless{}{-}} \FunctionTok{resid\_bubble\_summary}\NormalTok{(}\StringTok{"R"}\NormalTok{,  data\_list}\SpecialCharTok{$}\NormalTok{bioindexR,  }\StringTok{"q2"}\NormalTok{, }\StringTok{"R"}\NormalTok{)}
\NormalTok{b3 }\OtherTok{\textless{}{-}} \FunctionTok{resid\_bubble\_summary}\NormalTok{(}\StringTok{"P"}\NormalTok{,  data\_list}\SpecialCharTok{$}\NormalTok{bioindexP,  }\StringTok{"q3"}\NormalTok{, }\StringTok{"P"}\NormalTok{)}
\NormalTok{bubbles }\OtherTok{\textless{}{-}} \FunctionTok{rbind}\NormalTok{(b1, b2, b3)}

\CommentTok{\# Порядок рядов сверху вниз: P1, R, P}
\NormalTok{bubbles}\SpecialCharTok{$}\NormalTok{series }\OtherTok{\textless{}{-}} \FunctionTok{factor}\NormalTok{(bubbles}\SpecialCharTok{$}\NormalTok{series, }\AttributeTok{levels =} \FunctionTok{c}\NormalTok{(}\StringTok{"P1"}\NormalTok{, }\StringTok{"R"}\NormalTok{, }\StringTok{"P"}\NormalTok{))}

\CommentTok{\# Убираем пустое расстояние {-} используем минимальные интервалы}
\NormalTok{lvl }\OtherTok{\textless{}{-}} \FunctionTok{c}\NormalTok{(}\StringTok{"P1"}\NormalTok{,}\StringTok{"R"}\NormalTok{,}\StringTok{"P"}\NormalTok{)}
\NormalTok{y\_map }\OtherTok{\textless{}{-}} \FunctionTok{setNames}\NormalTok{(}\FunctionTok{c}\NormalTok{(}\DecValTok{1}\NormalTok{, }\DecValTok{2}\NormalTok{, }\DecValTok{3}\NormalTok{), lvl)  }\CommentTok{\# Числовые позиции без больших промежутков}

\NormalTok{bubbles}\SpecialCharTok{$}\NormalTok{y\_pos }\OtherTok{\textless{}{-}} \FunctionTok{unname}\NormalTok{(y\_map[}\FunctionTok{as.character}\NormalTok{(bubbles}\SpecialCharTok{$}\NormalTok{series)])}

\CommentTok{\# Создаем вытянутый прямоугольный график}
\FunctionTok{ggplot}\NormalTok{(bubbles, }\FunctionTok{aes}\NormalTok{(}\AttributeTok{x =}\NormalTok{ YEAR, }\AttributeTok{y =}\NormalTok{ y\_pos)) }\SpecialCharTok{+}
  \FunctionTok{geom\_point}\NormalTok{(}\FunctionTok{aes}\NormalTok{(}\AttributeTok{size =}\NormalTok{ abs\_resid, }\AttributeTok{fill =}\NormalTok{ sign), }\AttributeTok{shape =} \DecValTok{21}\NormalTok{, }\AttributeTok{color =} \StringTok{"black"}\NormalTok{, }\AttributeTok{alpha =} \FloatTok{0.9}\NormalTok{) }\SpecialCharTok{+}
  \FunctionTok{scale\_fill\_manual}\NormalTok{(}\AttributeTok{values =} \FunctionTok{c}\NormalTok{(}\AttributeTok{neg =} \StringTok{"black"}\NormalTok{, }\AttributeTok{pos =} \StringTok{"white"}\NormalTok{),}
                    \AttributeTok{breaks =} \FunctionTok{c}\NormalTok{(}\StringTok{"pos"}\NormalTok{,}\StringTok{"neg"}\NormalTok{),}
                    \AttributeTok{labels =} \FunctionTok{c}\NormalTok{(}\StringTok{"положительные"}\NormalTok{,}\StringTok{"отрицательные"}\NormalTok{),}
                    \AttributeTok{name =} \StringTok{""}\NormalTok{) }\SpecialCharTok{+}
  \FunctionTok{scale\_size\_area}\NormalTok{(}\AttributeTok{max\_size =} \DecValTok{12}\NormalTok{, }\AttributeTok{name =} \StringTok{"Остатки"}\NormalTok{) }\SpecialCharTok{+}
  \FunctionTok{scale\_x\_continuous}\NormalTok{(}\AttributeTok{breaks =} \FunctionTok{seq}\NormalTok{(}\DecValTok{2000}\NormalTok{, }\DecValTok{2015}\NormalTok{, }\AttributeTok{by =} \DecValTok{2}\NormalTok{), }\AttributeTok{limits =} \FunctionTok{c}\NormalTok{(}\DecValTok{2000}\NormalTok{, }\DecValTok{2015}\NormalTok{)) }\SpecialCharTok{+}
  \FunctionTok{scale\_y\_continuous}\NormalTok{(}\AttributeTok{breaks =} \FunctionTok{unname}\NormalTok{(y\_map), }
                     \AttributeTok{labels =} \FunctionTok{names}\NormalTok{(y\_map),}
                     \AttributeTok{limits =} \FunctionTok{c}\NormalTok{(}\FloatTok{0.5}\NormalTok{, }\FloatTok{3.5}\NormalTok{),  }\CommentTok{\# Убираем пустое пространство сверху и снизу}
                     \AttributeTok{expand =} \FunctionTok{c}\NormalTok{(}\DecValTok{0}\NormalTok{, }\DecValTok{0}\NormalTok{)) }\SpecialCharTok{+}     \CommentTok{\# Убираем расширение осей}
  \FunctionTok{labs}\NormalTok{(}\AttributeTok{title =} \StringTok{"Пузырьковая диаграмма остатков (лог{-}шкала): P1, R, P"}\NormalTok{, }
       \AttributeTok{x =} \StringTok{"Год"}\NormalTok{, }
       \AttributeTok{y =} \StringTok{""}\NormalTok{) }\SpecialCharTok{+}
  \FunctionTok{theme\_minimal}\NormalTok{(}\AttributeTok{base\_size =} \DecValTok{12}\NormalTok{) }\SpecialCharTok{+}
  \FunctionTok{theme}\NormalTok{(}
    \AttributeTok{legend.position =} \StringTok{"top"}\NormalTok{,}
    \AttributeTok{panel.grid.major.y =} \FunctionTok{element\_blank}\NormalTok{(),}
    \AttributeTok{axis.ticks.y =} \FunctionTok{element\_blank}\NormalTok{(),}
    \AttributeTok{aspect.ratio =} \FloatTok{0.3}\NormalTok{,  }\CommentTok{\# Делаем график вытянутым прямоугольником (ширина \textgreater{} высоты)}
    \AttributeTok{plot.margin =} \FunctionTok{margin}\NormalTok{(}\DecValTok{5}\NormalTok{, }\DecValTok{10}\NormalTok{, }\DecValTok{5}\NormalTok{, }\DecValTok{5}\NormalTok{, }\StringTok{"pt"}\NormalTok{)  }\CommentTok{\# Убираем лишние отступы вокруг графика}
\NormalTok{  )}
\end{Highlighting}
\end{Shaded}

\begin{verbatim}
Warning in grid.Call(C_textBounds, as.graphicsAnnot(x$label), x$x, x$y, :
неизвестна ширина символа 0xce в кодировке CP1251
\end{verbatim}

\begin{verbatim}
Warning in grid.Call(C_textBounds, as.graphicsAnnot(x$label), x$x, x$y, :
неизвестна ширина символа 0xf1 в кодировке CP1251
\end{verbatim}

\begin{verbatim}
Warning in grid.Call(C_textBounds, as.graphicsAnnot(x$label), x$x, x$y, :
неизвестна ширина символа 0xf2 в кодировке CP1251
\end{verbatim}

\begin{verbatim}
Warning in grid.Call(C_textBounds, as.graphicsAnnot(x$label), x$x, x$y, :
неизвестна ширина символа 0xe0 в кодировке CP1251
\end{verbatim}

\begin{verbatim}
Warning in grid.Call(C_textBounds, as.graphicsAnnot(x$label), x$x, x$y, :
неизвестна ширина символа 0xf2 в кодировке CP1251
\end{verbatim}

\begin{verbatim}
Warning in grid.Call(C_textBounds, as.graphicsAnnot(x$label), x$x, x$y, :
неизвестна ширина символа 0xea в кодировке CP1251
\end{verbatim}

\begin{verbatim}
Warning in grid.Call(C_textBounds, as.graphicsAnnot(x$label), x$x, x$y, :
неизвестна ширина символа 0xe8 в кодировке CP1251
\end{verbatim}

\begin{verbatim}
Warning in grid.Call(C_textBounds, as.graphicsAnnot(x$label), x$x, x$y, :
неизвестна ширина символа 0xce в кодировке CP1251
\end{verbatim}

\begin{verbatim}
Warning in grid.Call(C_textBounds, as.graphicsAnnot(x$label), x$x, x$y, :
неизвестна ширина символа 0xf1 в кодировке CP1251
\end{verbatim}

\begin{verbatim}
Warning in grid.Call(C_textBounds, as.graphicsAnnot(x$label), x$x, x$y, :
неизвестна ширина символа 0xf2 в кодировке CP1251
\end{verbatim}

\begin{verbatim}
Warning in grid.Call(C_textBounds, as.graphicsAnnot(x$label), x$x, x$y, :
неизвестна ширина символа 0xe0 в кодировке CP1251
\end{verbatim}

\begin{verbatim}
Warning in grid.Call(C_textBounds, as.graphicsAnnot(x$label), x$x, x$y, :
неизвестна ширина символа 0xf2 в кодировке CP1251
\end{verbatim}

\begin{verbatim}
Warning in grid.Call(C_textBounds, as.graphicsAnnot(x$label), x$x, x$y, :
неизвестна ширина символа 0xea в кодировке CP1251
\end{verbatim}

\begin{verbatim}
Warning in grid.Call(C_textBounds, as.graphicsAnnot(x$label), x$x, x$y, :
неизвестна ширина символа 0xe8 в кодировке CP1251
\end{verbatim}

\begin{verbatim}
Warning in grid.Call(C_textBounds, as.graphicsAnnot(x$label), x$x, x$y, :
неизвестна ширина символа 0xef в кодировке CP1251
\end{verbatim}

\begin{verbatim}
Warning in grid.Call(C_textBounds, as.graphicsAnnot(x$label), x$x, x$y, :
неизвестна ширина символа 0xee в кодировке CP1251
\end{verbatim}

\begin{verbatim}
Warning in grid.Call(C_textBounds, as.graphicsAnnot(x$label), x$x, x$y, :
неизвестна ширина символа 0xeb в кодировке CP1251
\end{verbatim}

\begin{verbatim}
Warning in grid.Call(C_textBounds, as.graphicsAnnot(x$label), x$x, x$y, :
неизвестна ширина символа 0xee в кодировке CP1251
\end{verbatim}

\begin{verbatim}
Warning in grid.Call(C_textBounds, as.graphicsAnnot(x$label), x$x, x$y, :
неизвестна ширина символа 0xe6 в кодировке CP1251
\end{verbatim}

\begin{verbatim}
Warning in grid.Call(C_textBounds, as.graphicsAnnot(x$label), x$x, x$y, :
неизвестна ширина символа 0xe8 в кодировке CP1251
\end{verbatim}

\begin{verbatim}
Warning in grid.Call(C_textBounds, as.graphicsAnnot(x$label), x$x, x$y, :
неизвестна ширина символа 0xf2 в кодировке CP1251
\end{verbatim}

\begin{verbatim}
Warning in grid.Call(C_textBounds, as.graphicsAnnot(x$label), x$x, x$y, :
неизвестна ширина символа 0xe5 в кодировке CP1251
\end{verbatim}

\begin{verbatim}
Warning in grid.Call(C_textBounds, as.graphicsAnnot(x$label), x$x, x$y, :
неизвестна ширина символа 0xeb в кодировке CP1251
\end{verbatim}

\begin{verbatim}
Warning in grid.Call(C_textBounds, as.graphicsAnnot(x$label), x$x, x$y, :
неизвестна ширина символа 0xfc в кодировке CP1251
\end{verbatim}

\begin{verbatim}
Warning in grid.Call(C_textBounds, as.graphicsAnnot(x$label), x$x, x$y, :
неизвестна ширина символа 0xed в кодировке CP1251
\end{verbatim}

\begin{verbatim}
Warning in grid.Call(C_textBounds, as.graphicsAnnot(x$label), x$x, x$y, :
неизвестна ширина символа 0xfb в кодировке CP1251
\end{verbatim}

\begin{verbatim}
Warning in grid.Call(C_textBounds, as.graphicsAnnot(x$label), x$x, x$y, :
неизвестна ширина символа 0xe5 в кодировке CP1251
\end{verbatim}

\begin{verbatim}
Warning in grid.Call(C_textBounds, as.graphicsAnnot(x$label), x$x, x$y, :
неизвестна ширина символа 0xee в кодировке CP1251
\end{verbatim}

\begin{verbatim}
Warning in grid.Call(C_textBounds, as.graphicsAnnot(x$label), x$x, x$y, :
неизвестна ширина символа 0xf2 в кодировке CP1251
\end{verbatim}

\begin{verbatim}
Warning in grid.Call(C_textBounds, as.graphicsAnnot(x$label), x$x, x$y, :
неизвестна ширина символа 0xf0 в кодировке CP1251
\end{verbatim}

\begin{verbatim}
Warning in grid.Call(C_textBounds, as.graphicsAnnot(x$label), x$x, x$y, :
неизвестна ширина символа 0xe8 в кодировке CP1251
\end{verbatim}

\begin{verbatim}
Warning in grid.Call(C_textBounds, as.graphicsAnnot(x$label), x$x, x$y, :
неизвестна ширина символа 0xf6 в кодировке CP1251
\end{verbatim}

\begin{verbatim}
Warning in grid.Call(C_textBounds, as.graphicsAnnot(x$label), x$x, x$y, :
неизвестна ширина символа 0xe0 в кодировке CP1251
\end{verbatim}

\begin{verbatim}
Warning in grid.Call(C_textBounds, as.graphicsAnnot(x$label), x$x, x$y, :
неизвестна ширина символа 0xf2 в кодировке CP1251
\end{verbatim}

\begin{verbatim}
Warning in grid.Call(C_textBounds, as.graphicsAnnot(x$label), x$x, x$y, :
неизвестна ширина символа 0xe5 в кодировке CP1251
\end{verbatim}

\begin{verbatim}
Warning in grid.Call(C_textBounds, as.graphicsAnnot(x$label), x$x, x$y, :
неизвестна ширина символа 0xeb в кодировке CP1251
\end{verbatim}

\begin{verbatim}
Warning in grid.Call(C_textBounds, as.graphicsAnnot(x$label), x$x, x$y, :
неизвестна ширина символа 0xfc в кодировке CP1251
\end{verbatim}

\begin{verbatim}
Warning in grid.Call(C_textBounds, as.graphicsAnnot(x$label), x$x, x$y, :
неизвестна ширина символа 0xed в кодировке CP1251
\end{verbatim}

\begin{verbatim}
Warning in grid.Call(C_textBounds, as.graphicsAnnot(x$label), x$x, x$y, :
неизвестна ширина символа 0xfb в кодировке CP1251
\end{verbatim}

\begin{verbatim}
Warning in grid.Call(C_textBounds, as.graphicsAnnot(x$label), x$x, x$y, :
неизвестна ширина символа 0xe5 в кодировке CP1251
\end{verbatim}

\begin{verbatim}
Warning in grid.Call(C_textBounds, as.graphicsAnnot(x$label), x$x, x$y, :
неизвестна ширина символа 0xef в кодировке CP1251
\end{verbatim}

\begin{verbatim}
Warning in grid.Call(C_textBounds, as.graphicsAnnot(x$label), x$x, x$y, :
неизвестна ширина символа 0xee в кодировке CP1251
\end{verbatim}

\begin{verbatim}
Warning in grid.Call(C_textBounds, as.graphicsAnnot(x$label), x$x, x$y, :
неизвестна ширина символа 0xeb в кодировке CP1251
\end{verbatim}

\begin{verbatim}
Warning in grid.Call(C_textBounds, as.graphicsAnnot(x$label), x$x, x$y, :
неизвестна ширина символа 0xee в кодировке CP1251
\end{verbatim}

\begin{verbatim}
Warning in grid.Call(C_textBounds, as.graphicsAnnot(x$label), x$x, x$y, :
неизвестна ширина символа 0xe6 в кодировке CP1251
\end{verbatim}

\begin{verbatim}
Warning in grid.Call(C_textBounds, as.graphicsAnnot(x$label), x$x, x$y, :
неизвестна ширина символа 0xe8 в кодировке CP1251
\end{verbatim}

\begin{verbatim}
Warning in grid.Call(C_textBounds, as.graphicsAnnot(x$label), x$x, x$y, :
неизвестна ширина символа 0xf2 в кодировке CP1251
\end{verbatim}

\begin{verbatim}
Warning in grid.Call(C_textBounds, as.graphicsAnnot(x$label), x$x, x$y, :
неизвестна ширина символа 0xe5 в кодировке CP1251
\end{verbatim}

\begin{verbatim}
Warning in grid.Call(C_textBounds, as.graphicsAnnot(x$label), x$x, x$y, :
неизвестна ширина символа 0xeb в кодировке CP1251
\end{verbatim}

\begin{verbatim}
Warning in grid.Call(C_textBounds, as.graphicsAnnot(x$label), x$x, x$y, :
неизвестна ширина символа 0xfc в кодировке CP1251
\end{verbatim}

\begin{verbatim}
Warning in grid.Call(C_textBounds, as.graphicsAnnot(x$label), x$x, x$y, :
неизвестна ширина символа 0xed в кодировке CP1251
\end{verbatim}

\begin{verbatim}
Warning in grid.Call(C_textBounds, as.graphicsAnnot(x$label), x$x, x$y, :
неизвестна ширина символа 0xfb в кодировке CP1251
\end{verbatim}

\begin{verbatim}
Warning in grid.Call(C_textBounds, as.graphicsAnnot(x$label), x$x, x$y, :
неизвестна ширина символа 0xe5 в кодировке CP1251
\end{verbatim}

\begin{verbatim}
Warning in grid.Call(C_textBounds, as.graphicsAnnot(x$label), x$x, x$y, :
неизвестна ширина символа 0xee в кодировке CP1251
\end{verbatim}

\begin{verbatim}
Warning in grid.Call(C_textBounds, as.graphicsAnnot(x$label), x$x, x$y, :
неизвестна ширина символа 0xf2 в кодировке CP1251
\end{verbatim}

\begin{verbatim}
Warning in grid.Call(C_textBounds, as.graphicsAnnot(x$label), x$x, x$y, :
неизвестна ширина символа 0xf0 в кодировке CP1251
\end{verbatim}

\begin{verbatim}
Warning in grid.Call(C_textBounds, as.graphicsAnnot(x$label), x$x, x$y, :
неизвестна ширина символа 0xe8 в кодировке CP1251
\end{verbatim}

\begin{verbatim}
Warning in grid.Call(C_textBounds, as.graphicsAnnot(x$label), x$x, x$y, :
неизвестна ширина символа 0xf6 в кодировке CP1251
\end{verbatim}

\begin{verbatim}
Warning in grid.Call(C_textBounds, as.graphicsAnnot(x$label), x$x, x$y, :
неизвестна ширина символа 0xe0 в кодировке CP1251
\end{verbatim}

\begin{verbatim}
Warning in grid.Call(C_textBounds, as.graphicsAnnot(x$label), x$x, x$y, :
неизвестна ширина символа 0xf2 в кодировке CP1251
\end{verbatim}

\begin{verbatim}
Warning in grid.Call(C_textBounds, as.graphicsAnnot(x$label), x$x, x$y, :
неизвестна ширина символа 0xe5 в кодировке CP1251
\end{verbatim}

\begin{verbatim}
Warning in grid.Call(C_textBounds, as.graphicsAnnot(x$label), x$x, x$y, :
неизвестна ширина символа 0xeb в кодировке CP1251
\end{verbatim}

\begin{verbatim}
Warning in grid.Call(C_textBounds, as.graphicsAnnot(x$label), x$x, x$y, :
неизвестна ширина символа 0xfc в кодировке CP1251
\end{verbatim}

\begin{verbatim}
Warning in grid.Call(C_textBounds, as.graphicsAnnot(x$label), x$x, x$y, :
неизвестна ширина символа 0xed в кодировке CP1251
\end{verbatim}

\begin{verbatim}
Warning in grid.Call(C_textBounds, as.graphicsAnnot(x$label), x$x, x$y, :
неизвестна ширина символа 0xfb в кодировке CP1251
\end{verbatim}

\begin{verbatim}
Warning in grid.Call(C_textBounds, as.graphicsAnnot(x$label), x$x, x$y, :
неизвестна ширина символа 0xe5 в кодировке CP1251
\end{verbatim}

\begin{verbatim}
Warning in grid.Call(C_textBounds, as.graphicsAnnot(x$label), x$x, x$y, :
неизвестна ширина символа 0xcf в кодировке CP1251
\end{verbatim}

\begin{verbatim}
Warning in grid.Call(C_textBounds, as.graphicsAnnot(x$label), x$x, x$y, :
неизвестна ширина символа 0xf3 в кодировке CP1251
\end{verbatim}

\begin{verbatim}
Warning in grid.Call(C_textBounds, as.graphicsAnnot(x$label), x$x, x$y, :
неизвестна ширина символа 0xe7 в кодировке CP1251
\end{verbatim}

\begin{verbatim}
Warning in grid.Call(C_textBounds, as.graphicsAnnot(x$label), x$x, x$y, :
неизвестна ширина символа 0xfb в кодировке CP1251
\end{verbatim}

\begin{verbatim}
Warning in grid.Call(C_textBounds, as.graphicsAnnot(x$label), x$x, x$y, :
неизвестна ширина символа 0xf0 в кодировке CP1251
\end{verbatim}

\begin{verbatim}
Warning in grid.Call(C_textBounds, as.graphicsAnnot(x$label), x$x, x$y, :
неизвестна ширина символа 0xfc в кодировке CP1251
\end{verbatim}

\begin{verbatim}
Warning in grid.Call(C_textBounds, as.graphicsAnnot(x$label), x$x, x$y, :
неизвестна ширина символа 0xea в кодировке CP1251
\end{verbatim}

\begin{verbatim}
Warning in grid.Call(C_textBounds, as.graphicsAnnot(x$label), x$x, x$y, :
неизвестна ширина символа 0xee в кодировке CP1251
\end{verbatim}

\begin{verbatim}
Warning in grid.Call(C_textBounds, as.graphicsAnnot(x$label), x$x, x$y, :
неизвестна ширина символа 0xe2 в кодировке CP1251
\end{verbatim}

\begin{verbatim}
Warning in grid.Call(C_textBounds, as.graphicsAnnot(x$label), x$x, x$y, :
неизвестна ширина символа 0xe0 в кодировке CP1251
\end{verbatim}

\begin{verbatim}
Warning in grid.Call(C_textBounds, as.graphicsAnnot(x$label), x$x, x$y, :
неизвестна ширина символа 0xff в кодировке CP1251
\end{verbatim}

\begin{verbatim}
Warning in grid.Call(C_textBounds, as.graphicsAnnot(x$label), x$x, x$y, :
неизвестна ширина символа 0xe4 в кодировке CP1251
\end{verbatim}

\begin{verbatim}
Warning in grid.Call(C_textBounds, as.graphicsAnnot(x$label), x$x, x$y, :
неизвестна ширина символа 0xe8 в кодировке CP1251
\end{verbatim}

\begin{verbatim}
Warning in grid.Call(C_textBounds, as.graphicsAnnot(x$label), x$x, x$y, :
неизвестна ширина символа 0xe0 в кодировке CP1251
\end{verbatim}

\begin{verbatim}
Warning in grid.Call(C_textBounds, as.graphicsAnnot(x$label), x$x, x$y, :
неизвестна ширина символа 0xe3 в кодировке CP1251
\end{verbatim}

\begin{verbatim}
Warning in grid.Call(C_textBounds, as.graphicsAnnot(x$label), x$x, x$y, :
неизвестна ширина символа 0xf0 в кодировке CP1251
\end{verbatim}

\begin{verbatim}
Warning in grid.Call(C_textBounds, as.graphicsAnnot(x$label), x$x, x$y, :
неизвестна ширина символа 0xe0 в кодировке CP1251
\end{verbatim}

\begin{verbatim}
Warning in grid.Call(C_textBounds, as.graphicsAnnot(x$label), x$x, x$y, :
неизвестна ширина символа 0xec в кодировке CP1251
Warning in grid.Call(C_textBounds, as.graphicsAnnot(x$label), x$x, x$y, :
неизвестна ширина символа 0xec в кодировке CP1251
\end{verbatim}

\begin{verbatim}
Warning in grid.Call(C_textBounds, as.graphicsAnnot(x$label), x$x, x$y, :
неизвестна ширина символа 0xe0 в кодировке CP1251
\end{verbatim}

\begin{verbatim}
Warning in grid.Call(C_textBounds, as.graphicsAnnot(x$label), x$x, x$y, :
неизвестна ширина символа 0xee в кодировке CP1251
\end{verbatim}

\begin{verbatim}
Warning in grid.Call(C_textBounds, as.graphicsAnnot(x$label), x$x, x$y, :
неизвестна ширина символа 0xf1 в кодировке CP1251
\end{verbatim}

\begin{verbatim}
Warning in grid.Call(C_textBounds, as.graphicsAnnot(x$label), x$x, x$y, :
неизвестна ширина символа 0xf2 в кодировке CP1251
\end{verbatim}

\begin{verbatim}
Warning in grid.Call(C_textBounds, as.graphicsAnnot(x$label), x$x, x$y, :
неизвестна ширина символа 0xe0 в кодировке CP1251
\end{verbatim}

\begin{verbatim}
Warning in grid.Call(C_textBounds, as.graphicsAnnot(x$label), x$x, x$y, :
неизвестна ширина символа 0xf2 в кодировке CP1251
\end{verbatim}

\begin{verbatim}
Warning in grid.Call(C_textBounds, as.graphicsAnnot(x$label), x$x, x$y, :
неизвестна ширина символа 0xea в кодировке CP1251
\end{verbatim}

\begin{verbatim}
Warning in grid.Call(C_textBounds, as.graphicsAnnot(x$label), x$x, x$y, :
неизвестна ширина символа 0xee в кодировке CP1251
\end{verbatim}

\begin{verbatim}
Warning in grid.Call(C_textBounds, as.graphicsAnnot(x$label), x$x, x$y, :
неизвестна ширина символа 0xe2 в кодировке CP1251
\end{verbatim}

\begin{verbatim}
Warning in grid.Call(C_textBounds, as.graphicsAnnot(x$label), x$x, x$y, :
неизвестна ширина символа 0xeb в кодировке CP1251
\end{verbatim}

\begin{verbatim}
Warning in grid.Call(C_textBounds, as.graphicsAnnot(x$label), x$x, x$y, :
неизвестна ширина символа 0xee в кодировке CP1251
\end{verbatim}

\begin{verbatim}
Warning in grid.Call(C_textBounds, as.graphicsAnnot(x$label), x$x, x$y, :
неизвестна ширина символа 0xe3 в кодировке CP1251
\end{verbatim}

\begin{verbatim}
Warning in grid.Call(C_textBounds, as.graphicsAnnot(x$label), x$x, x$y, :
неизвестна ширина символа 0xf8 в кодировке CP1251
\end{verbatim}

\begin{verbatim}
Warning in grid.Call(C_textBounds, as.graphicsAnnot(x$label), x$x, x$y, :
неизвестна ширина символа 0xea в кодировке CP1251
\end{verbatim}

\begin{verbatim}
Warning in grid.Call(C_textBounds, as.graphicsAnnot(x$label), x$x, x$y, :
неизвестна ширина символа 0xe0 в кодировке CP1251
\end{verbatim}

\begin{verbatim}
Warning in grid.Call(C_textBounds, as.graphicsAnnot(x$label), x$x, x$y, :
неизвестна ширина символа 0xeb в кодировке CP1251
\end{verbatim}

\begin{verbatim}
Warning in grid.Call(C_textBounds, as.graphicsAnnot(x$label), x$x, x$y, :
неизвестна ширина символа 0xe0 в кодировке CP1251
\end{verbatim}

\begin{verbatim}
Warning in grid.Call(C_textBounds, as.graphicsAnnot(x$label), x$x, x$y, :
неизвестна ширина символа 0xc3 в кодировке CP1251
\end{verbatim}

\begin{verbatim}
Warning in grid.Call(C_textBounds, as.graphicsAnnot(x$label), x$x, x$y, :
неизвестна ширина символа 0xee в кодировке CP1251
\end{verbatim}

\begin{verbatim}
Warning in grid.Call(C_textBounds, as.graphicsAnnot(x$label), x$x, x$y, :
неизвестна ширина символа 0xe4 в кодировке CP1251
\end{verbatim}

\begin{verbatim}
Warning in grid.Call.graphics(C_text, as.graphicsAnnot(x$label), x$x, x$y, :
неизвестна ширина символа 0xc3 в кодировке CP1251
\end{verbatim}

\begin{verbatim}
Warning in grid.Call.graphics(C_text, as.graphicsAnnot(x$label), x$x, x$y, :
неизвестна ширина символа 0xee в кодировке CP1251
\end{verbatim}

\begin{verbatim}
Warning in grid.Call.graphics(C_text, as.graphicsAnnot(x$label), x$x, x$y, :
неизвестна ширина символа 0xe4 в кодировке CP1251
\end{verbatim}

\begin{verbatim}
Warning in grid.Call.graphics(C_text, as.graphicsAnnot(x$label), x$x, x$y, :
неизвестна ширина символа 0xce в кодировке CP1251
\end{verbatim}

\begin{verbatim}
Warning in grid.Call.graphics(C_text, as.graphicsAnnot(x$label), x$x, x$y, :
неизвестна ширина символа 0xf1 в кодировке CP1251
\end{verbatim}

\begin{verbatim}
Warning in grid.Call.graphics(C_text, as.graphicsAnnot(x$label), x$x, x$y, :
неизвестна ширина символа 0xf2 в кодировке CP1251
\end{verbatim}

\begin{verbatim}
Warning in grid.Call.graphics(C_text, as.graphicsAnnot(x$label), x$x, x$y, :
неизвестна ширина символа 0xe0 в кодировке CP1251
\end{verbatim}

\begin{verbatim}
Warning in grid.Call.graphics(C_text, as.graphicsAnnot(x$label), x$x, x$y, :
неизвестна ширина символа 0xf2 в кодировке CP1251
\end{verbatim}

\begin{verbatim}
Warning in grid.Call.graphics(C_text, as.graphicsAnnot(x$label), x$x, x$y, :
неизвестна ширина символа 0xea в кодировке CP1251
\end{verbatim}

\begin{verbatim}
Warning in grid.Call.graphics(C_text, as.graphicsAnnot(x$label), x$x, x$y, :
неизвестна ширина символа 0xe8 в кодировке CP1251
\end{verbatim}

\begin{verbatim}
Warning in grid.Call.graphics(C_text, as.graphicsAnnot(x$label), x$x, x$y, :
неизвестна ширина символа 0xef в кодировке CP1251
\end{verbatim}

\begin{verbatim}
Warning in grid.Call.graphics(C_text, as.graphicsAnnot(x$label), x$x, x$y, :
неизвестна ширина символа 0xee в кодировке CP1251
\end{verbatim}

\begin{verbatim}
Warning in grid.Call.graphics(C_text, as.graphicsAnnot(x$label), x$x, x$y, :
неизвестна ширина символа 0xeb в кодировке CP1251
\end{verbatim}

\begin{verbatim}
Warning in grid.Call.graphics(C_text, as.graphicsAnnot(x$label), x$x, x$y, :
неизвестна ширина символа 0xee в кодировке CP1251
\end{verbatim}

\begin{verbatim}
Warning in grid.Call.graphics(C_text, as.graphicsAnnot(x$label), x$x, x$y, :
неизвестна ширина символа 0xe6 в кодировке CP1251
\end{verbatim}

\begin{verbatim}
Warning in grid.Call.graphics(C_text, as.graphicsAnnot(x$label), x$x, x$y, :
неизвестна ширина символа 0xe8 в кодировке CP1251
\end{verbatim}

\begin{verbatim}
Warning in grid.Call.graphics(C_text, as.graphicsAnnot(x$label), x$x, x$y, :
неизвестна ширина символа 0xf2 в кодировке CP1251
\end{verbatim}

\begin{verbatim}
Warning in grid.Call.graphics(C_text, as.graphicsAnnot(x$label), x$x, x$y, :
неизвестна ширина символа 0xe5 в кодировке CP1251
\end{verbatim}

\begin{verbatim}
Warning in grid.Call.graphics(C_text, as.graphicsAnnot(x$label), x$x, x$y, :
неизвестна ширина символа 0xeb в кодировке CP1251
\end{verbatim}

\begin{verbatim}
Warning in grid.Call.graphics(C_text, as.graphicsAnnot(x$label), x$x, x$y, :
неизвестна ширина символа 0xfc в кодировке CP1251
\end{verbatim}

\begin{verbatim}
Warning in grid.Call.graphics(C_text, as.graphicsAnnot(x$label), x$x, x$y, :
неизвестна ширина символа 0xed в кодировке CP1251
\end{verbatim}

\begin{verbatim}
Warning in grid.Call.graphics(C_text, as.graphicsAnnot(x$label), x$x, x$y, :
неизвестна ширина символа 0xfb в кодировке CP1251
\end{verbatim}

\begin{verbatim}
Warning in grid.Call.graphics(C_text, as.graphicsAnnot(x$label), x$x, x$y, :
неизвестна ширина символа 0xe5 в кодировке CP1251
\end{verbatim}

\begin{verbatim}
Warning in grid.Call.graphics(C_text, as.graphicsAnnot(x$label), x$x, x$y, :
неизвестна ширина символа 0xee в кодировке CP1251
\end{verbatim}

\begin{verbatim}
Warning in grid.Call.graphics(C_text, as.graphicsAnnot(x$label), x$x, x$y, :
неизвестна ширина символа 0xf2 в кодировке CP1251
\end{verbatim}

\begin{verbatim}
Warning in grid.Call.graphics(C_text, as.graphicsAnnot(x$label), x$x, x$y, :
неизвестна ширина символа 0xf0 в кодировке CP1251
\end{verbatim}

\begin{verbatim}
Warning in grid.Call.graphics(C_text, as.graphicsAnnot(x$label), x$x, x$y, :
неизвестна ширина символа 0xe8 в кодировке CP1251
\end{verbatim}

\begin{verbatim}
Warning in grid.Call.graphics(C_text, as.graphicsAnnot(x$label), x$x, x$y, :
неизвестна ширина символа 0xf6 в кодировке CP1251
\end{verbatim}

\begin{verbatim}
Warning in grid.Call.graphics(C_text, as.graphicsAnnot(x$label), x$x, x$y, :
неизвестна ширина символа 0xe0 в кодировке CP1251
\end{verbatim}

\begin{verbatim}
Warning in grid.Call.graphics(C_text, as.graphicsAnnot(x$label), x$x, x$y, :
неизвестна ширина символа 0xf2 в кодировке CP1251
\end{verbatim}

\begin{verbatim}
Warning in grid.Call.graphics(C_text, as.graphicsAnnot(x$label), x$x, x$y, :
неизвестна ширина символа 0xe5 в кодировке CP1251
\end{verbatim}

\begin{verbatim}
Warning in grid.Call.graphics(C_text, as.graphicsAnnot(x$label), x$x, x$y, :
неизвестна ширина символа 0xeb в кодировке CP1251
\end{verbatim}

\begin{verbatim}
Warning in grid.Call.graphics(C_text, as.graphicsAnnot(x$label), x$x, x$y, :
неизвестна ширина символа 0xfc в кодировке CP1251
\end{verbatim}

\begin{verbatim}
Warning in grid.Call.graphics(C_text, as.graphicsAnnot(x$label), x$x, x$y, :
неизвестна ширина символа 0xed в кодировке CP1251
\end{verbatim}

\begin{verbatim}
Warning in grid.Call.graphics(C_text, as.graphicsAnnot(x$label), x$x, x$y, :
неизвестна ширина символа 0xfb в кодировке CP1251
\end{verbatim}

\begin{verbatim}
Warning in grid.Call.graphics(C_text, as.graphicsAnnot(x$label), x$x, x$y, :
неизвестна ширина символа 0xe5 в кодировке CP1251
\end{verbatim}

\begin{verbatim}
Warning in grid.Call.graphics(C_text, as.graphicsAnnot(x$label), x$x, x$y, :
неизвестна ширина символа 0xcf в кодировке CP1251
\end{verbatim}

\begin{verbatim}
Warning in grid.Call.graphics(C_text, as.graphicsAnnot(x$label), x$x, x$y, :
неизвестна ширина символа 0xf3 в кодировке CP1251
\end{verbatim}

\begin{verbatim}
Warning in grid.Call.graphics(C_text, as.graphicsAnnot(x$label), x$x, x$y, :
неизвестна ширина символа 0xe7 в кодировке CP1251
\end{verbatim}

\begin{verbatim}
Warning in grid.Call.graphics(C_text, as.graphicsAnnot(x$label), x$x, x$y, :
неизвестна ширина символа 0xfb в кодировке CP1251
\end{verbatim}

\begin{verbatim}
Warning in grid.Call.graphics(C_text, as.graphicsAnnot(x$label), x$x, x$y, :
неизвестна ширина символа 0xf0 в кодировке CP1251
\end{verbatim}

\begin{verbatim}
Warning in grid.Call.graphics(C_text, as.graphicsAnnot(x$label), x$x, x$y, :
неизвестна ширина символа 0xfc в кодировке CP1251
\end{verbatim}

\begin{verbatim}
Warning in grid.Call.graphics(C_text, as.graphicsAnnot(x$label), x$x, x$y, :
неизвестна ширина символа 0xea в кодировке CP1251
\end{verbatim}

\begin{verbatim}
Warning in grid.Call.graphics(C_text, as.graphicsAnnot(x$label), x$x, x$y, :
неизвестна ширина символа 0xee в кодировке CP1251
\end{verbatim}

\begin{verbatim}
Warning in grid.Call.graphics(C_text, as.graphicsAnnot(x$label), x$x, x$y, :
неизвестна ширина символа 0xe2 в кодировке CP1251
\end{verbatim}

\begin{verbatim}
Warning in grid.Call.graphics(C_text, as.graphicsAnnot(x$label), x$x, x$y, :
неизвестна ширина символа 0xe0 в кодировке CP1251
\end{verbatim}

\begin{verbatim}
Warning in grid.Call.graphics(C_text, as.graphicsAnnot(x$label), x$x, x$y, :
неизвестна ширина символа 0xff в кодировке CP1251
\end{verbatim}

\begin{verbatim}
Warning in grid.Call.graphics(C_text, as.graphicsAnnot(x$label), x$x, x$y, :
неизвестна ширина символа 0xe4 в кодировке CP1251
\end{verbatim}

\begin{verbatim}
Warning in grid.Call.graphics(C_text, as.graphicsAnnot(x$label), x$x, x$y, :
неизвестна ширина символа 0xe8 в кодировке CP1251
\end{verbatim}

\begin{verbatim}
Warning in grid.Call.graphics(C_text, as.graphicsAnnot(x$label), x$x, x$y, :
неизвестна ширина символа 0xe0 в кодировке CP1251
\end{verbatim}

\begin{verbatim}
Warning in grid.Call.graphics(C_text, as.graphicsAnnot(x$label), x$x, x$y, :
неизвестна ширина символа 0xe3 в кодировке CP1251
\end{verbatim}

\begin{verbatim}
Warning in grid.Call.graphics(C_text, as.graphicsAnnot(x$label), x$x, x$y, :
неизвестна ширина символа 0xf0 в кодировке CP1251
\end{verbatim}

\begin{verbatim}
Warning in grid.Call.graphics(C_text, as.graphicsAnnot(x$label), x$x, x$y, :
неизвестна ширина символа 0xe0 в кодировке CP1251
\end{verbatim}

\begin{verbatim}
Warning in grid.Call.graphics(C_text, as.graphicsAnnot(x$label), x$x, x$y, :
неизвестна ширина символа 0xec в кодировке CP1251
Warning in grid.Call.graphics(C_text, as.graphicsAnnot(x$label), x$x, x$y, :
неизвестна ширина символа 0xec в кодировке CP1251
\end{verbatim}

\begin{verbatim}
Warning in grid.Call.graphics(C_text, as.graphicsAnnot(x$label), x$x, x$y, :
неизвестна ширина символа 0xe0 в кодировке CP1251
\end{verbatim}

\begin{verbatim}
Warning in grid.Call.graphics(C_text, as.graphicsAnnot(x$label), x$x, x$y, :
неизвестна ширина символа 0xee в кодировке CP1251
\end{verbatim}

\begin{verbatim}
Warning in grid.Call.graphics(C_text, as.graphicsAnnot(x$label), x$x, x$y, :
неизвестна ширина символа 0xf1 в кодировке CP1251
\end{verbatim}

\begin{verbatim}
Warning in grid.Call.graphics(C_text, as.graphicsAnnot(x$label), x$x, x$y, :
неизвестна ширина символа 0xf2 в кодировке CP1251
\end{verbatim}

\begin{verbatim}
Warning in grid.Call.graphics(C_text, as.graphicsAnnot(x$label), x$x, x$y, :
неизвестна ширина символа 0xe0 в кодировке CP1251
\end{verbatim}

\begin{verbatim}
Warning in grid.Call.graphics(C_text, as.graphicsAnnot(x$label), x$x, x$y, :
неизвестна ширина символа 0xf2 в кодировке CP1251
\end{verbatim}

\begin{verbatim}
Warning in grid.Call.graphics(C_text, as.graphicsAnnot(x$label), x$x, x$y, :
неизвестна ширина символа 0xea в кодировке CP1251
\end{verbatim}

\begin{verbatim}
Warning in grid.Call.graphics(C_text, as.graphicsAnnot(x$label), x$x, x$y, :
неизвестна ширина символа 0xee в кодировке CP1251
\end{verbatim}

\begin{verbatim}
Warning in grid.Call.graphics(C_text, as.graphicsAnnot(x$label), x$x, x$y, :
неизвестна ширина символа 0xe2 в кодировке CP1251
\end{verbatim}

\begin{verbatim}
Warning in grid.Call.graphics(C_text, as.graphicsAnnot(x$label), x$x, x$y, :
неизвестна ширина символа 0xeb в кодировке CP1251
\end{verbatim}

\begin{verbatim}
Warning in grid.Call.graphics(C_text, as.graphicsAnnot(x$label), x$x, x$y, :
неизвестна ширина символа 0xee в кодировке CP1251
\end{verbatim}

\begin{verbatim}
Warning in grid.Call.graphics(C_text, as.graphicsAnnot(x$label), x$x, x$y, :
неизвестна ширина символа 0xe3 в кодировке CP1251
\end{verbatim}

\begin{verbatim}
Warning in grid.Call.graphics(C_text, as.graphicsAnnot(x$label), x$x, x$y, :
неизвестна ширина символа 0xf8 в кодировке CP1251
\end{verbatim}

\begin{verbatim}
Warning in grid.Call.graphics(C_text, as.graphicsAnnot(x$label), x$x, x$y, :
неизвестна ширина символа 0xea в кодировке CP1251
\end{verbatim}

\begin{verbatim}
Warning in grid.Call.graphics(C_text, as.graphicsAnnot(x$label), x$x, x$y, :
неизвестна ширина символа 0xe0 в кодировке CP1251
\end{verbatim}

\begin{verbatim}
Warning in grid.Call.graphics(C_text, as.graphicsAnnot(x$label), x$x, x$y, :
неизвестна ширина символа 0xeb в кодировке CP1251
\end{verbatim}

\begin{verbatim}
Warning in grid.Call.graphics(C_text, as.graphicsAnnot(x$label), x$x, x$y, :
неизвестна ширина символа 0xe0 в кодировке CP1251
\end{verbatim}

\pandocbounded{\includegraphics[keepaspectratio]{chapter8_files/figure-pdf/unnamed-chunk-1-2.pdf}}

\begin{Shaded}
\begin{Highlighting}[]
\CommentTok{\# ========================================================================================================================}
\CommentTok{\# ДИНАМИКА ИНДЕКСОВ (ПРЕРЕКРУТЫ, РЕКРУТЫ, ПОСТРЕКРУТЫ) МОДЕЛЬНЫХ И ФАКТИЧЕСКИХ (ТОЧКИ)}
\CommentTok{\# ========================================================================================================================}
\CommentTok{\# Три графика динамики P1, R, P: медиана (линия), 95\% ДИ (лента), точки — наблюдения,}
\CommentTok{\# приведённые к единому масштабу  делением на медиану q (Posterior median).}
\CommentTok{\# ========================================================================================================================}
\ControlFlowTok{if}\NormalTok{ (}\SpecialCharTok{!}\FunctionTok{exists}\NormalTok{(}\StringTok{"YEAR"}\NormalTok{)) YEAR }\OtherTok{\textless{}{-}} \DecValTok{2000} \SpecialCharTok{+} \DecValTok{0}\SpecialCharTok{:}\NormalTok{(data\_list}\SpecialCharTok{$}\NormalTok{N }\SpecialCharTok{{-}} \DecValTok{1}\NormalTok{)}
\NormalTok{draws\_mat }\OtherTok{\textless{}{-}} \FunctionTok{as.matrix}\NormalTok{(samps)}

\NormalTok{series\_summary }\OtherTok{\textless{}{-}} \ControlFlowTok{function}\NormalTok{(varname, obs\_vec, qname, series\_label) \{}
\NormalTok{  med\_q }\OtherTok{\textless{}{-}} \FunctionTok{median}\NormalTok{(draws\_mat[, qname], }\AttributeTok{na.rm =} \ConstantTok{TRUE}\NormalTok{)}
\NormalTok{  rows }\OtherTok{\textless{}{-}} \FunctionTok{vector}\NormalTok{(}\StringTok{"list"}\NormalTok{, }\FunctionTok{length}\NormalTok{(obs\_vec))}
  \ControlFlowTok{for}\NormalTok{ (i }\ControlFlowTok{in} \FunctionTok{seq\_along}\NormalTok{(obs\_vec)) \{}
\NormalTok{    rn }\OtherTok{\textless{}{-}} \FunctionTok{paste0}\NormalTok{(varname, }\StringTok{"["}\NormalTok{, i, }\StringTok{"]"}\NormalTok{)}
    \ControlFlowTok{if}\NormalTok{ (}\SpecialCharTok{!}\NormalTok{rn }\SpecialCharTok{\%in\%} \FunctionTok{colnames}\NormalTok{(draws\_mat)) }\ControlFlowTok{next}
\NormalTok{    v }\OtherTok{\textless{}{-}}\NormalTok{ draws\_mat[, rn]}
\NormalTok{    qs }\OtherTok{\textless{}{-}} \FunctionTok{quantile}\NormalTok{(v, }\FunctionTok{c}\NormalTok{(}\FloatTok{0.025}\NormalTok{, }\FloatTok{0.5}\NormalTok{, }\FloatTok{0.975}\NormalTok{), }\AttributeTok{na.rm =} \ConstantTok{TRUE}\NormalTok{)}
\NormalTok{    obs\_state }\OtherTok{\textless{}{-}} \ControlFlowTok{if}\NormalTok{ (}\SpecialCharTok{!}\FunctionTok{is.na}\NormalTok{(obs\_vec[i])) obs\_vec[i] }\SpecialCharTok{/}\NormalTok{ med\_q }\ControlFlowTok{else} \ConstantTok{NA\_real\_}
\NormalTok{    rows[[i]] }\OtherTok{\textless{}{-}} \FunctionTok{data.frame}\NormalTok{(}
      \AttributeTok{YEAR =}\NormalTok{ YEAR[i],}
      \AttributeTok{series =}\NormalTok{ series\_label,}
      \AttributeTok{median =}\NormalTok{ qs[}\DecValTok{2}\NormalTok{],}
      \AttributeTok{lo =}\NormalTok{ qs[}\DecValTok{1}\NormalTok{],}
      \AttributeTok{hi =}\NormalTok{ qs[}\DecValTok{3}\NormalTok{],}
      \AttributeTok{obs =}\NormalTok{ obs\_state}
\NormalTok{    )}
\NormalTok{  \}}
  \FunctionTok{do.call}\NormalTok{(rbind, rows)}
\NormalTok{\}}

\NormalTok{df\_p1 }\OtherTok{\textless{}{-}} \FunctionTok{series\_summary}\NormalTok{(}\StringTok{"P1"}\NormalTok{, data\_list}\SpecialCharTok{$}\NormalTok{bioindexP1, }\StringTok{"q1"}\NormalTok{, }\StringTok{"P1"}\NormalTok{)}
\NormalTok{df\_r  }\OtherTok{\textless{}{-}} \FunctionTok{series\_summary}\NormalTok{(}\StringTok{"R"}\NormalTok{,  data\_list}\SpecialCharTok{$}\NormalTok{bioindexR,  }\StringTok{"q2"}\NormalTok{, }\StringTok{"R"}\NormalTok{)}
\NormalTok{df\_p  }\OtherTok{\textless{}{-}} \FunctionTok{series\_summary}\NormalTok{(}\StringTok{"P"}\NormalTok{,  data\_list}\SpecialCharTok{$}\NormalTok{bioindexP,  }\StringTok{"q3"}\NormalTok{, }\StringTok{"P"}\NormalTok{)}


\NormalTok{p\_P1 }\OtherTok{\textless{}{-}} \FunctionTok{ggplot}\NormalTok{(df\_p1, }\FunctionTok{aes}\NormalTok{(}\AttributeTok{x =}\NormalTok{ YEAR)) }\SpecialCharTok{+}
  \FunctionTok{geom\_ribbon}\NormalTok{(}\FunctionTok{aes}\NormalTok{(}\AttributeTok{ymin =}\NormalTok{ lo, }\AttributeTok{ymax =}\NormalTok{ hi), }\AttributeTok{fill =} \StringTok{"\#1f77b4"}\NormalTok{, }\AttributeTok{alpha =} \FloatTok{0.2}\NormalTok{) }\SpecialCharTok{+}
  \FunctionTok{geom\_line}\NormalTok{(}\FunctionTok{aes}\NormalTok{(}\AttributeTok{y =}\NormalTok{ median), }\AttributeTok{color =} \StringTok{"\#1f77b4"}\NormalTok{, }\AttributeTok{linewidth =} \DecValTok{1}\NormalTok{) }\SpecialCharTok{+}
  \FunctionTok{geom\_point}\NormalTok{(}\FunctionTok{aes}\NormalTok{(}\AttributeTok{y =}\NormalTok{ obs), }\AttributeTok{shape =} \DecValTok{21}\NormalTok{, }\AttributeTok{size =} \DecValTok{2}\NormalTok{, }\AttributeTok{color =} \StringTok{"black"}\NormalTok{, }\AttributeTok{fill =} \StringTok{"white"}\NormalTok{, }\AttributeTok{na.rm =} \ConstantTok{TRUE}\NormalTok{) }\SpecialCharTok{+}
  \FunctionTok{scale\_x\_continuous}\NormalTok{(}\AttributeTok{breaks =} \FunctionTok{seq}\NormalTok{(}\DecValTok{2000}\NormalTok{, }\DecValTok{2015}\NormalTok{, }\AttributeTok{by =} \DecValTok{2}\NormalTok{), }\AttributeTok{limits =} \FunctionTok{c}\NormalTok{(}\DecValTok{2000}\NormalTok{, }\DecValTok{2015}\NormalTok{)) }\SpecialCharTok{+}
  \FunctionTok{labs}\NormalTok{(}\AttributeTok{title =} \StringTok{"Моделируемая и фактическая (точки) динамика пререкрутов"}\NormalTok{, }\AttributeTok{x =} \StringTok{"Годы"}\NormalTok{, }\AttributeTok{y =} \StringTok{"Пререкруты (экз.)"}\NormalTok{) }\SpecialCharTok{+}
  \FunctionTok{theme\_minimal}\NormalTok{(}\AttributeTok{base\_size =} \DecValTok{12}\NormalTok{)}

\FunctionTok{print}\NormalTok{(p\_P1)}
\end{Highlighting}
\end{Shaded}

\begin{verbatim}
Warning in grid.Call(C_textBounds, as.graphicsAnnot(x$label), x$x, x$y, :
неизвестна ширина символа 0xcf в кодировке CP1251
\end{verbatim}

\begin{verbatim}
Warning in grid.Call(C_textBounds, as.graphicsAnnot(x$label), x$x, x$y, :
неизвестна ширина символа 0xf0 в кодировке CP1251
\end{verbatim}

\begin{verbatim}
Warning in grid.Call(C_textBounds, as.graphicsAnnot(x$label), x$x, x$y, :
неизвестна ширина символа 0xe5 в кодировке CP1251
\end{verbatim}

\begin{verbatim}
Warning in grid.Call(C_textBounds, as.graphicsAnnot(x$label), x$x, x$y, :
неизвестна ширина символа 0xf0 в кодировке CP1251
\end{verbatim}

\begin{verbatim}
Warning in grid.Call(C_textBounds, as.graphicsAnnot(x$label), x$x, x$y, :
неизвестна ширина символа 0xe5 в кодировке CP1251
\end{verbatim}

\begin{verbatim}
Warning in grid.Call(C_textBounds, as.graphicsAnnot(x$label), x$x, x$y, :
неизвестна ширина символа 0xea в кодировке CP1251
\end{verbatim}

\begin{verbatim}
Warning in grid.Call(C_textBounds, as.graphicsAnnot(x$label), x$x, x$y, :
неизвестна ширина символа 0xf0 в кодировке CP1251
\end{verbatim}

\begin{verbatim}
Warning in grid.Call(C_textBounds, as.graphicsAnnot(x$label), x$x, x$y, :
неизвестна ширина символа 0xf3 в кодировке CP1251
\end{verbatim}

\begin{verbatim}
Warning in grid.Call(C_textBounds, as.graphicsAnnot(x$label), x$x, x$y, :
неизвестна ширина символа 0xf2 в кодировке CP1251
\end{verbatim}

\begin{verbatim}
Warning in grid.Call(C_textBounds, as.graphicsAnnot(x$label), x$x, x$y, :
неизвестна ширина символа 0xfb в кодировке CP1251
\end{verbatim}

\begin{verbatim}
Warning in grid.Call(C_textBounds, as.graphicsAnnot(x$label), x$x, x$y, :
неизвестна ширина символа 0xfd в кодировке CP1251
\end{verbatim}

\begin{verbatim}
Warning in grid.Call(C_textBounds, as.graphicsAnnot(x$label), x$x, x$y, :
неизвестна ширина символа 0xea в кодировке CP1251
\end{verbatim}

\begin{verbatim}
Warning in grid.Call(C_textBounds, as.graphicsAnnot(x$label), x$x, x$y, :
неизвестна ширина символа 0xe7 в кодировке CP1251
\end{verbatim}

\begin{verbatim}
Warning in grid.Call(C_textBounds, as.graphicsAnnot(x$label), x$x, x$y, :
неизвестна ширина символа 0xcc в кодировке CP1251
\end{verbatim}

\begin{verbatim}
Warning in grid.Call(C_textBounds, as.graphicsAnnot(x$label), x$x, x$y, :
неизвестна ширина символа 0xee в кодировке CP1251
\end{verbatim}

\begin{verbatim}
Warning in grid.Call(C_textBounds, as.graphicsAnnot(x$label), x$x, x$y, :
неизвестна ширина символа 0xe4 в кодировке CP1251
\end{verbatim}

\begin{verbatim}
Warning in grid.Call(C_textBounds, as.graphicsAnnot(x$label), x$x, x$y, :
неизвестна ширина символа 0xe5 в кодировке CP1251
\end{verbatim}

\begin{verbatim}
Warning in grid.Call(C_textBounds, as.graphicsAnnot(x$label), x$x, x$y, :
неизвестна ширина символа 0xeb в кодировке CP1251
\end{verbatim}

\begin{verbatim}
Warning in grid.Call(C_textBounds, as.graphicsAnnot(x$label), x$x, x$y, :
неизвестна ширина символа 0xe8 в кодировке CP1251
\end{verbatim}

\begin{verbatim}
Warning in grid.Call(C_textBounds, as.graphicsAnnot(x$label), x$x, x$y, :
неизвестна ширина символа 0xf0 в кодировке CP1251
\end{verbatim}

\begin{verbatim}
Warning in grid.Call(C_textBounds, as.graphicsAnnot(x$label), x$x, x$y, :
неизвестна ширина символа 0xf3 в кодировке CP1251
\end{verbatim}

\begin{verbatim}
Warning in grid.Call(C_textBounds, as.graphicsAnnot(x$label), x$x, x$y, :
неизвестна ширина символа 0xe5 в кодировке CP1251
\end{verbatim}

\begin{verbatim}
Warning in grid.Call(C_textBounds, as.graphicsAnnot(x$label), x$x, x$y, :
неизвестна ширина символа 0xec в кодировке CP1251
\end{verbatim}

\begin{verbatim}
Warning in grid.Call(C_textBounds, as.graphicsAnnot(x$label), x$x, x$y, :
неизвестна ширина символа 0xe0 в кодировке CP1251
\end{verbatim}

\begin{verbatim}
Warning in grid.Call(C_textBounds, as.graphicsAnnot(x$label), x$x, x$y, :
неизвестна ширина символа 0xff в кодировке CP1251
\end{verbatim}

\begin{verbatim}
Warning in grid.Call(C_textBounds, as.graphicsAnnot(x$label), x$x, x$y, :
неизвестна ширина символа 0xe8 в кодировке CP1251
\end{verbatim}

\begin{verbatim}
Warning in grid.Call(C_textBounds, as.graphicsAnnot(x$label), x$x, x$y, :
неизвестна ширина символа 0xf4 в кодировке CP1251
\end{verbatim}

\begin{verbatim}
Warning in grid.Call(C_textBounds, as.graphicsAnnot(x$label), x$x, x$y, :
неизвестна ширина символа 0xe0 в кодировке CP1251
\end{verbatim}

\begin{verbatim}
Warning in grid.Call(C_textBounds, as.graphicsAnnot(x$label), x$x, x$y, :
неизвестна ширина символа 0xea в кодировке CP1251
\end{verbatim}

\begin{verbatim}
Warning in grid.Call(C_textBounds, as.graphicsAnnot(x$label), x$x, x$y, :
неизвестна ширина символа 0xf2 в кодировке CP1251
\end{verbatim}

\begin{verbatim}
Warning in grid.Call(C_textBounds, as.graphicsAnnot(x$label), x$x, x$y, :
неизвестна ширина символа 0xe8 в кодировке CP1251
\end{verbatim}

\begin{verbatim}
Warning in grid.Call(C_textBounds, as.graphicsAnnot(x$label), x$x, x$y, :
неизвестна ширина символа 0xf7 в кодировке CP1251
\end{verbatim}

\begin{verbatim}
Warning in grid.Call(C_textBounds, as.graphicsAnnot(x$label), x$x, x$y, :
неизвестна ширина символа 0xe5 в кодировке CP1251
\end{verbatim}

\begin{verbatim}
Warning in grid.Call(C_textBounds, as.graphicsAnnot(x$label), x$x, x$y, :
неизвестна ширина символа 0xf1 в кодировке CP1251
\end{verbatim}

\begin{verbatim}
Warning in grid.Call(C_textBounds, as.graphicsAnnot(x$label), x$x, x$y, :
неизвестна ширина символа 0xea в кодировке CP1251
\end{verbatim}

\begin{verbatim}
Warning in grid.Call(C_textBounds, as.graphicsAnnot(x$label), x$x, x$y, :
неизвестна ширина символа 0xe0 в кодировке CP1251
\end{verbatim}

\begin{verbatim}
Warning in grid.Call(C_textBounds, as.graphicsAnnot(x$label), x$x, x$y, :
неизвестна ширина символа 0xff в кодировке CP1251
\end{verbatim}

\begin{verbatim}
Warning in grid.Call(C_textBounds, as.graphicsAnnot(x$label), x$x, x$y, :
неизвестна ширина символа 0xf2 в кодировке CP1251
\end{verbatim}

\begin{verbatim}
Warning in grid.Call(C_textBounds, as.graphicsAnnot(x$label), x$x, x$y, :
неизвестна ширина символа 0xee в кодировке CP1251
\end{verbatim}

\begin{verbatim}
Warning in grid.Call(C_textBounds, as.graphicsAnnot(x$label), x$x, x$y, :
неизвестна ширина символа 0xf7 в кодировке CP1251
\end{verbatim}

\begin{verbatim}
Warning in grid.Call(C_textBounds, as.graphicsAnnot(x$label), x$x, x$y, :
неизвестна ширина символа 0xea в кодировке CP1251
\end{verbatim}

\begin{verbatim}
Warning in grid.Call(C_textBounds, as.graphicsAnnot(x$label), x$x, x$y, :
неизвестна ширина символа 0xe8 в кодировке CP1251
\end{verbatim}

\begin{verbatim}
Warning in grid.Call(C_textBounds, as.graphicsAnnot(x$label), x$x, x$y, :
неизвестна ширина символа 0xe4 в кодировке CP1251
\end{verbatim}

\begin{verbatim}
Warning in grid.Call(C_textBounds, as.graphicsAnnot(x$label), x$x, x$y, :
неизвестна ширина символа 0xe8 в кодировке CP1251
\end{verbatim}

\begin{verbatim}
Warning in grid.Call(C_textBounds, as.graphicsAnnot(x$label), x$x, x$y, :
неизвестна ширина символа 0xed в кодировке CP1251
\end{verbatim}

\begin{verbatim}
Warning in grid.Call(C_textBounds, as.graphicsAnnot(x$label), x$x, x$y, :
неизвестна ширина символа 0xe0 в кодировке CP1251
\end{verbatim}

\begin{verbatim}
Warning in grid.Call(C_textBounds, as.graphicsAnnot(x$label), x$x, x$y, :
неизвестна ширина символа 0xec в кодировке CP1251
\end{verbatim}

\begin{verbatim}
Warning in grid.Call(C_textBounds, as.graphicsAnnot(x$label), x$x, x$y, :
неизвестна ширина символа 0xe8 в кодировке CP1251
\end{verbatim}

\begin{verbatim}
Warning in grid.Call(C_textBounds, as.graphicsAnnot(x$label), x$x, x$y, :
неизвестна ширина символа 0xea в кодировке CP1251
\end{verbatim}

\begin{verbatim}
Warning in grid.Call(C_textBounds, as.graphicsAnnot(x$label), x$x, x$y, :
неизвестна ширина символа 0xe0 в кодировке CP1251
\end{verbatim}

\begin{verbatim}
Warning in grid.Call(C_textBounds, as.graphicsAnnot(x$label), x$x, x$y, :
неизвестна ширина символа 0xef в кодировке CP1251
\end{verbatim}

\begin{verbatim}
Warning in grid.Call(C_textBounds, as.graphicsAnnot(x$label), x$x, x$y, :
неизвестна ширина символа 0xf0 в кодировке CP1251
\end{verbatim}

\begin{verbatim}
Warning in grid.Call(C_textBounds, as.graphicsAnnot(x$label), x$x, x$y, :
неизвестна ширина символа 0xe5 в кодировке CP1251
\end{verbatim}

\begin{verbatim}
Warning in grid.Call(C_textBounds, as.graphicsAnnot(x$label), x$x, x$y, :
неизвестна ширина символа 0xf0 в кодировке CP1251
\end{verbatim}

\begin{verbatim}
Warning in grid.Call(C_textBounds, as.graphicsAnnot(x$label), x$x, x$y, :
неизвестна ширина символа 0xe5 в кодировке CP1251
\end{verbatim}

\begin{verbatim}
Warning in grid.Call(C_textBounds, as.graphicsAnnot(x$label), x$x, x$y, :
неизвестна ширина символа 0xea в кодировке CP1251
\end{verbatim}

\begin{verbatim}
Warning in grid.Call(C_textBounds, as.graphicsAnnot(x$label), x$x, x$y, :
неизвестна ширина символа 0xf0 в кодировке CP1251
\end{verbatim}

\begin{verbatim}
Warning in grid.Call(C_textBounds, as.graphicsAnnot(x$label), x$x, x$y, :
неизвестна ширина символа 0xf3 в кодировке CP1251
\end{verbatim}

\begin{verbatim}
Warning in grid.Call(C_textBounds, as.graphicsAnnot(x$label), x$x, x$y, :
неизвестна ширина символа 0xf2 в кодировке CP1251
\end{verbatim}

\begin{verbatim}
Warning in grid.Call(C_textBounds, as.graphicsAnnot(x$label), x$x, x$y, :
неизвестна ширина символа 0xee в кодировке CP1251
\end{verbatim}

\begin{verbatim}
Warning in grid.Call(C_textBounds, as.graphicsAnnot(x$label), x$x, x$y, :
неизвестна ширина символа 0xe2 в кодировке CP1251
\end{verbatim}

\begin{verbatim}
Warning in grid.Call(C_textBounds, as.graphicsAnnot(x$label), x$x, x$y, :
неизвестна ширина символа 0xc3 в кодировке CP1251
\end{verbatim}

\begin{verbatim}
Warning in grid.Call(C_textBounds, as.graphicsAnnot(x$label), x$x, x$y, :
неизвестна ширина символа 0xee в кодировке CP1251
\end{verbatim}

\begin{verbatim}
Warning in grid.Call(C_textBounds, as.graphicsAnnot(x$label), x$x, x$y, :
неизвестна ширина символа 0xe4 в кодировке CP1251
\end{verbatim}

\begin{verbatim}
Warning in grid.Call(C_textBounds, as.graphicsAnnot(x$label), x$x, x$y, :
неизвестна ширина символа 0xfb в кодировке CP1251
\end{verbatim}

\begin{verbatim}
Warning in grid.Call.graphics(C_text, as.graphicsAnnot(x$label), x$x, x$y, :
неизвестна ширина символа 0xc3 в кодировке CP1251
\end{verbatim}

\begin{verbatim}
Warning in grid.Call.graphics(C_text, as.graphicsAnnot(x$label), x$x, x$y, :
неизвестна ширина символа 0xee в кодировке CP1251
\end{verbatim}

\begin{verbatim}
Warning in grid.Call.graphics(C_text, as.graphicsAnnot(x$label), x$x, x$y, :
неизвестна ширина символа 0xe4 в кодировке CP1251
\end{verbatim}

\begin{verbatim}
Warning in grid.Call.graphics(C_text, as.graphicsAnnot(x$label), x$x, x$y, :
неизвестна ширина символа 0xfb в кодировке CP1251
\end{verbatim}

\begin{verbatim}
Warning in grid.Call.graphics(C_text, as.graphicsAnnot(x$label), x$x, x$y, :
неизвестна ширина символа 0xcf в кодировке CP1251
\end{verbatim}

\begin{verbatim}
Warning in grid.Call.graphics(C_text, as.graphicsAnnot(x$label), x$x, x$y, :
неизвестна ширина символа 0xf0 в кодировке CP1251
\end{verbatim}

\begin{verbatim}
Warning in grid.Call.graphics(C_text, as.graphicsAnnot(x$label), x$x, x$y, :
неизвестна ширина символа 0xe5 в кодировке CP1251
\end{verbatim}

\begin{verbatim}
Warning in grid.Call.graphics(C_text, as.graphicsAnnot(x$label), x$x, x$y, :
неизвестна ширина символа 0xf0 в кодировке CP1251
\end{verbatim}

\begin{verbatim}
Warning in grid.Call.graphics(C_text, as.graphicsAnnot(x$label), x$x, x$y, :
неизвестна ширина символа 0xe5 в кодировке CP1251
\end{verbatim}

\begin{verbatim}
Warning in grid.Call.graphics(C_text, as.graphicsAnnot(x$label), x$x, x$y, :
неизвестна ширина символа 0xea в кодировке CP1251
\end{verbatim}

\begin{verbatim}
Warning in grid.Call.graphics(C_text, as.graphicsAnnot(x$label), x$x, x$y, :
неизвестна ширина символа 0xf0 в кодировке CP1251
\end{verbatim}

\begin{verbatim}
Warning in grid.Call.graphics(C_text, as.graphicsAnnot(x$label), x$x, x$y, :
неизвестна ширина символа 0xf3 в кодировке CP1251
\end{verbatim}

\begin{verbatim}
Warning in grid.Call.graphics(C_text, as.graphicsAnnot(x$label), x$x, x$y, :
неизвестна ширина символа 0xf2 в кодировке CP1251
\end{verbatim}

\begin{verbatim}
Warning in grid.Call.graphics(C_text, as.graphicsAnnot(x$label), x$x, x$y, :
неизвестна ширина символа 0xfb в кодировке CP1251
\end{verbatim}

\begin{verbatim}
Warning in grid.Call.graphics(C_text, as.graphicsAnnot(x$label), x$x, x$y, :
неизвестна ширина символа 0xfd в кодировке CP1251
\end{verbatim}

\begin{verbatim}
Warning in grid.Call.graphics(C_text, as.graphicsAnnot(x$label), x$x, x$y, :
неизвестна ширина символа 0xea в кодировке CP1251
\end{verbatim}

\begin{verbatim}
Warning in grid.Call.graphics(C_text, as.graphicsAnnot(x$label), x$x, x$y, :
неизвестна ширина символа 0xe7 в кодировке CP1251
\end{verbatim}

\begin{verbatim}
Warning in grid.Call.graphics(C_text, as.graphicsAnnot(x$label), x$x, x$y, :
неизвестна ширина символа 0xcc в кодировке CP1251
\end{verbatim}

\begin{verbatim}
Warning in grid.Call.graphics(C_text, as.graphicsAnnot(x$label), x$x, x$y, :
неизвестна ширина символа 0xee в кодировке CP1251
\end{verbatim}

\begin{verbatim}
Warning in grid.Call.graphics(C_text, as.graphicsAnnot(x$label), x$x, x$y, :
неизвестна ширина символа 0xe4 в кодировке CP1251
\end{verbatim}

\begin{verbatim}
Warning in grid.Call.graphics(C_text, as.graphicsAnnot(x$label), x$x, x$y, :
неизвестна ширина символа 0xe5 в кодировке CP1251
\end{verbatim}

\begin{verbatim}
Warning in grid.Call.graphics(C_text, as.graphicsAnnot(x$label), x$x, x$y, :
неизвестна ширина символа 0xeb в кодировке CP1251
\end{verbatim}

\begin{verbatim}
Warning in grid.Call.graphics(C_text, as.graphicsAnnot(x$label), x$x, x$y, :
неизвестна ширина символа 0xe8 в кодировке CP1251
\end{verbatim}

\begin{verbatim}
Warning in grid.Call.graphics(C_text, as.graphicsAnnot(x$label), x$x, x$y, :
неизвестна ширина символа 0xf0 в кодировке CP1251
\end{verbatim}

\begin{verbatim}
Warning in grid.Call.graphics(C_text, as.graphicsAnnot(x$label), x$x, x$y, :
неизвестна ширина символа 0xf3 в кодировке CP1251
\end{verbatim}

\begin{verbatim}
Warning in grid.Call.graphics(C_text, as.graphicsAnnot(x$label), x$x, x$y, :
неизвестна ширина символа 0xe5 в кодировке CP1251
\end{verbatim}

\begin{verbatim}
Warning in grid.Call.graphics(C_text, as.graphicsAnnot(x$label), x$x, x$y, :
неизвестна ширина символа 0xec в кодировке CP1251
\end{verbatim}

\begin{verbatim}
Warning in grid.Call.graphics(C_text, as.graphicsAnnot(x$label), x$x, x$y, :
неизвестна ширина символа 0xe0 в кодировке CP1251
\end{verbatim}

\begin{verbatim}
Warning in grid.Call.graphics(C_text, as.graphicsAnnot(x$label), x$x, x$y, :
неизвестна ширина символа 0xff в кодировке CP1251
\end{verbatim}

\begin{verbatim}
Warning in grid.Call.graphics(C_text, as.graphicsAnnot(x$label), x$x, x$y, :
неизвестна ширина символа 0xe8 в кодировке CP1251
\end{verbatim}

\begin{verbatim}
Warning in grid.Call.graphics(C_text, as.graphicsAnnot(x$label), x$x, x$y, :
неизвестна ширина символа 0xf4 в кодировке CP1251
\end{verbatim}

\begin{verbatim}
Warning in grid.Call.graphics(C_text, as.graphicsAnnot(x$label), x$x, x$y, :
неизвестна ширина символа 0xe0 в кодировке CP1251
\end{verbatim}

\begin{verbatim}
Warning in grid.Call.graphics(C_text, as.graphicsAnnot(x$label), x$x, x$y, :
неизвестна ширина символа 0xea в кодировке CP1251
\end{verbatim}

\begin{verbatim}
Warning in grid.Call.graphics(C_text, as.graphicsAnnot(x$label), x$x, x$y, :
неизвестна ширина символа 0xf2 в кодировке CP1251
\end{verbatim}

\begin{verbatim}
Warning in grid.Call.graphics(C_text, as.graphicsAnnot(x$label), x$x, x$y, :
неизвестна ширина символа 0xe8 в кодировке CP1251
\end{verbatim}

\begin{verbatim}
Warning in grid.Call.graphics(C_text, as.graphicsAnnot(x$label), x$x, x$y, :
неизвестна ширина символа 0xf7 в кодировке CP1251
\end{verbatim}

\begin{verbatim}
Warning in grid.Call.graphics(C_text, as.graphicsAnnot(x$label), x$x, x$y, :
неизвестна ширина символа 0xe5 в кодировке CP1251
\end{verbatim}

\begin{verbatim}
Warning in grid.Call.graphics(C_text, as.graphicsAnnot(x$label), x$x, x$y, :
неизвестна ширина символа 0xf1 в кодировке CP1251
\end{verbatim}

\begin{verbatim}
Warning in grid.Call.graphics(C_text, as.graphicsAnnot(x$label), x$x, x$y, :
неизвестна ширина символа 0xea в кодировке CP1251
\end{verbatim}

\begin{verbatim}
Warning in grid.Call.graphics(C_text, as.graphicsAnnot(x$label), x$x, x$y, :
неизвестна ширина символа 0xe0 в кодировке CP1251
\end{verbatim}

\begin{verbatim}
Warning in grid.Call.graphics(C_text, as.graphicsAnnot(x$label), x$x, x$y, :
неизвестна ширина символа 0xff в кодировке CP1251
\end{verbatim}

\begin{verbatim}
Warning in grid.Call.graphics(C_text, as.graphicsAnnot(x$label), x$x, x$y, :
неизвестна ширина символа 0xf2 в кодировке CP1251
\end{verbatim}

\begin{verbatim}
Warning in grid.Call.graphics(C_text, as.graphicsAnnot(x$label), x$x, x$y, :
неизвестна ширина символа 0xee в кодировке CP1251
\end{verbatim}

\begin{verbatim}
Warning in grid.Call.graphics(C_text, as.graphicsAnnot(x$label), x$x, x$y, :
неизвестна ширина символа 0xf7 в кодировке CP1251
\end{verbatim}

\begin{verbatim}
Warning in grid.Call.graphics(C_text, as.graphicsAnnot(x$label), x$x, x$y, :
неизвестна ширина символа 0xea в кодировке CP1251
\end{verbatim}

\begin{verbatim}
Warning in grid.Call.graphics(C_text, as.graphicsAnnot(x$label), x$x, x$y, :
неизвестна ширина символа 0xe8 в кодировке CP1251
\end{verbatim}

\begin{verbatim}
Warning in grid.Call.graphics(C_text, as.graphicsAnnot(x$label), x$x, x$y, :
неизвестна ширина символа 0xe4 в кодировке CP1251
\end{verbatim}

\begin{verbatim}
Warning in grid.Call.graphics(C_text, as.graphicsAnnot(x$label), x$x, x$y, :
неизвестна ширина символа 0xe8 в кодировке CP1251
\end{verbatim}

\begin{verbatim}
Warning in grid.Call.graphics(C_text, as.graphicsAnnot(x$label), x$x, x$y, :
неизвестна ширина символа 0xed в кодировке CP1251
\end{verbatim}

\begin{verbatim}
Warning in grid.Call.graphics(C_text, as.graphicsAnnot(x$label), x$x, x$y, :
неизвестна ширина символа 0xe0 в кодировке CP1251
\end{verbatim}

\begin{verbatim}
Warning in grid.Call.graphics(C_text, as.graphicsAnnot(x$label), x$x, x$y, :
неизвестна ширина символа 0xec в кодировке CP1251
\end{verbatim}

\begin{verbatim}
Warning in grid.Call.graphics(C_text, as.graphicsAnnot(x$label), x$x, x$y, :
неизвестна ширина символа 0xe8 в кодировке CP1251
\end{verbatim}

\begin{verbatim}
Warning in grid.Call.graphics(C_text, as.graphicsAnnot(x$label), x$x, x$y, :
неизвестна ширина символа 0xea в кодировке CP1251
\end{verbatim}

\begin{verbatim}
Warning in grid.Call.graphics(C_text, as.graphicsAnnot(x$label), x$x, x$y, :
неизвестна ширина символа 0xe0 в кодировке CP1251
\end{verbatim}

\begin{verbatim}
Warning in grid.Call.graphics(C_text, as.graphicsAnnot(x$label), x$x, x$y, :
неизвестна ширина символа 0xef в кодировке CP1251
\end{verbatim}

\begin{verbatim}
Warning in grid.Call.graphics(C_text, as.graphicsAnnot(x$label), x$x, x$y, :
неизвестна ширина символа 0xf0 в кодировке CP1251
\end{verbatim}

\begin{verbatim}
Warning in grid.Call.graphics(C_text, as.graphicsAnnot(x$label), x$x, x$y, :
неизвестна ширина символа 0xe5 в кодировке CP1251
\end{verbatim}

\begin{verbatim}
Warning in grid.Call.graphics(C_text, as.graphicsAnnot(x$label), x$x, x$y, :
неизвестна ширина символа 0xf0 в кодировке CP1251
\end{verbatim}

\begin{verbatim}
Warning in grid.Call.graphics(C_text, as.graphicsAnnot(x$label), x$x, x$y, :
неизвестна ширина символа 0xe5 в кодировке CP1251
\end{verbatim}

\begin{verbatim}
Warning in grid.Call.graphics(C_text, as.graphicsAnnot(x$label), x$x, x$y, :
неизвестна ширина символа 0xea в кодировке CP1251
\end{verbatim}

\begin{verbatim}
Warning in grid.Call.graphics(C_text, as.graphicsAnnot(x$label), x$x, x$y, :
неизвестна ширина символа 0xf0 в кодировке CP1251
\end{verbatim}

\begin{verbatim}
Warning in grid.Call.graphics(C_text, as.graphicsAnnot(x$label), x$x, x$y, :
неизвестна ширина символа 0xf3 в кодировке CP1251
\end{verbatim}

\begin{verbatim}
Warning in grid.Call.graphics(C_text, as.graphicsAnnot(x$label), x$x, x$y, :
неизвестна ширина символа 0xf2 в кодировке CP1251
\end{verbatim}

\begin{verbatim}
Warning in grid.Call.graphics(C_text, as.graphicsAnnot(x$label), x$x, x$y, :
неизвестна ширина символа 0xee в кодировке CP1251
\end{verbatim}

\begin{verbatim}
Warning in grid.Call.graphics(C_text, as.graphicsAnnot(x$label), x$x, x$y, :
неизвестна ширина символа 0xe2 в кодировке CP1251
\end{verbatim}

\pandocbounded{\includegraphics[keepaspectratio]{chapter8_files/figure-pdf/unnamed-chunk-1-3.pdf}}

\begin{Shaded}
\begin{Highlighting}[]
\NormalTok{p\_R }\OtherTok{\textless{}{-}} \FunctionTok{ggplot}\NormalTok{(df\_r, }\FunctionTok{aes}\NormalTok{(}\AttributeTok{x =}\NormalTok{ YEAR)) }\SpecialCharTok{+}
  \FunctionTok{geom\_ribbon}\NormalTok{(}\FunctionTok{aes}\NormalTok{(}\AttributeTok{ymin =}\NormalTok{ lo, }\AttributeTok{ymax =}\NormalTok{ hi), }\AttributeTok{fill =} \StringTok{"\#2ca02c"}\NormalTok{, }\AttributeTok{alpha =} \FloatTok{0.2}\NormalTok{) }\SpecialCharTok{+}
  \FunctionTok{geom\_line}\NormalTok{(}\FunctionTok{aes}\NormalTok{(}\AttributeTok{y =}\NormalTok{ median), }\AttributeTok{color =} \StringTok{"\#2ca02c"}\NormalTok{, }\AttributeTok{linewidth =} \DecValTok{1}\NormalTok{) }\SpecialCharTok{+}
  \FunctionTok{geom\_point}\NormalTok{(}\FunctionTok{aes}\NormalTok{(}\AttributeTok{y =}\NormalTok{ obs), }\AttributeTok{shape =} \DecValTok{21}\NormalTok{, }\AttributeTok{size =} \DecValTok{2}\NormalTok{, }\AttributeTok{color =} \StringTok{"black"}\NormalTok{, }\AttributeTok{fill =} \StringTok{"white"}\NormalTok{, }\AttributeTok{na.rm =} \ConstantTok{TRUE}\NormalTok{) }\SpecialCharTok{+}
  \FunctionTok{scale\_x\_continuous}\NormalTok{(}\AttributeTok{breaks =} \FunctionTok{seq}\NormalTok{(}\DecValTok{2000}\NormalTok{, }\DecValTok{2015}\NormalTok{, }\AttributeTok{by =} \DecValTok{2}\NormalTok{), }\AttributeTok{limits =} \FunctionTok{c}\NormalTok{(}\DecValTok{2000}\NormalTok{, }\DecValTok{2015}\NormalTok{)) }\SpecialCharTok{+}
  \FunctionTok{labs}\NormalTok{(}\AttributeTok{title =} \StringTok{"Моделируемая и фактическая (точки) динамика рекрутов"}\NormalTok{, }\AttributeTok{x =} \StringTok{"Годы"}\NormalTok{, }\AttributeTok{y =} \StringTok{"Рекруты (экз.)"}\NormalTok{) }\SpecialCharTok{+}
  \FunctionTok{theme\_minimal}\NormalTok{(}\AttributeTok{base\_size =} \DecValTok{12}\NormalTok{)}

\FunctionTok{print}\NormalTok{(p\_R)}
\end{Highlighting}
\end{Shaded}

\begin{verbatim}
Warning in grid.Call(C_textBounds, as.graphicsAnnot(x$label), x$x, x$y, :
неизвестна ширина символа 0xd0 в кодировке CP1251
\end{verbatim}

\begin{verbatim}
Warning in grid.Call(C_textBounds, as.graphicsAnnot(x$label), x$x, x$y, :
неизвестна ширина символа 0xe5 в кодировке CP1251
\end{verbatim}

\begin{verbatim}
Warning in grid.Call(C_textBounds, as.graphicsAnnot(x$label), x$x, x$y, :
неизвестна ширина символа 0xea в кодировке CP1251
\end{verbatim}

\begin{verbatim}
Warning in grid.Call(C_textBounds, as.graphicsAnnot(x$label), x$x, x$y, :
неизвестна ширина символа 0xf0 в кодировке CP1251
\end{verbatim}

\begin{verbatim}
Warning in grid.Call(C_textBounds, as.graphicsAnnot(x$label), x$x, x$y, :
неизвестна ширина символа 0xf3 в кодировке CP1251
\end{verbatim}

\begin{verbatim}
Warning in grid.Call(C_textBounds, as.graphicsAnnot(x$label), x$x, x$y, :
неизвестна ширина символа 0xf2 в кодировке CP1251
\end{verbatim}

\begin{verbatim}
Warning in grid.Call(C_textBounds, as.graphicsAnnot(x$label), x$x, x$y, :
неизвестна ширина символа 0xfb в кодировке CP1251
\end{verbatim}

\begin{verbatim}
Warning in grid.Call(C_textBounds, as.graphicsAnnot(x$label), x$x, x$y, :
неизвестна ширина символа 0xfd в кодировке CP1251
\end{verbatim}

\begin{verbatim}
Warning in grid.Call(C_textBounds, as.graphicsAnnot(x$label), x$x, x$y, :
неизвестна ширина символа 0xea в кодировке CP1251
\end{verbatim}

\begin{verbatim}
Warning in grid.Call(C_textBounds, as.graphicsAnnot(x$label), x$x, x$y, :
неизвестна ширина символа 0xe7 в кодировке CP1251
\end{verbatim}

\begin{verbatim}
Warning in grid.Call(C_textBounds, as.graphicsAnnot(x$label), x$x, x$y, :
неизвестна ширина символа 0xcc в кодировке CP1251
\end{verbatim}

\begin{verbatim}
Warning in grid.Call(C_textBounds, as.graphicsAnnot(x$label), x$x, x$y, :
неизвестна ширина символа 0xee в кодировке CP1251
\end{verbatim}

\begin{verbatim}
Warning in grid.Call(C_textBounds, as.graphicsAnnot(x$label), x$x, x$y, :
неизвестна ширина символа 0xe4 в кодировке CP1251
\end{verbatim}

\begin{verbatim}
Warning in grid.Call(C_textBounds, as.graphicsAnnot(x$label), x$x, x$y, :
неизвестна ширина символа 0xe5 в кодировке CP1251
\end{verbatim}

\begin{verbatim}
Warning in grid.Call(C_textBounds, as.graphicsAnnot(x$label), x$x, x$y, :
неизвестна ширина символа 0xeb в кодировке CP1251
\end{verbatim}

\begin{verbatim}
Warning in grid.Call(C_textBounds, as.graphicsAnnot(x$label), x$x, x$y, :
неизвестна ширина символа 0xe8 в кодировке CP1251
\end{verbatim}

\begin{verbatim}
Warning in grid.Call(C_textBounds, as.graphicsAnnot(x$label), x$x, x$y, :
неизвестна ширина символа 0xf0 в кодировке CP1251
\end{verbatim}

\begin{verbatim}
Warning in grid.Call(C_textBounds, as.graphicsAnnot(x$label), x$x, x$y, :
неизвестна ширина символа 0xf3 в кодировке CP1251
\end{verbatim}

\begin{verbatim}
Warning in grid.Call(C_textBounds, as.graphicsAnnot(x$label), x$x, x$y, :
неизвестна ширина символа 0xe5 в кодировке CP1251
\end{verbatim}

\begin{verbatim}
Warning in grid.Call(C_textBounds, as.graphicsAnnot(x$label), x$x, x$y, :
неизвестна ширина символа 0xec в кодировке CP1251
\end{verbatim}

\begin{verbatim}
Warning in grid.Call(C_textBounds, as.graphicsAnnot(x$label), x$x, x$y, :
неизвестна ширина символа 0xe0 в кодировке CP1251
\end{verbatim}

\begin{verbatim}
Warning in grid.Call(C_textBounds, as.graphicsAnnot(x$label), x$x, x$y, :
неизвестна ширина символа 0xff в кодировке CP1251
\end{verbatim}

\begin{verbatim}
Warning in grid.Call(C_textBounds, as.graphicsAnnot(x$label), x$x, x$y, :
неизвестна ширина символа 0xe8 в кодировке CP1251
\end{verbatim}

\begin{verbatim}
Warning in grid.Call(C_textBounds, as.graphicsAnnot(x$label), x$x, x$y, :
неизвестна ширина символа 0xf4 в кодировке CP1251
\end{verbatim}

\begin{verbatim}
Warning in grid.Call(C_textBounds, as.graphicsAnnot(x$label), x$x, x$y, :
неизвестна ширина символа 0xe0 в кодировке CP1251
\end{verbatim}

\begin{verbatim}
Warning in grid.Call(C_textBounds, as.graphicsAnnot(x$label), x$x, x$y, :
неизвестна ширина символа 0xea в кодировке CP1251
\end{verbatim}

\begin{verbatim}
Warning in grid.Call(C_textBounds, as.graphicsAnnot(x$label), x$x, x$y, :
неизвестна ширина символа 0xf2 в кодировке CP1251
\end{verbatim}

\begin{verbatim}
Warning in grid.Call(C_textBounds, as.graphicsAnnot(x$label), x$x, x$y, :
неизвестна ширина символа 0xe8 в кодировке CP1251
\end{verbatim}

\begin{verbatim}
Warning in grid.Call(C_textBounds, as.graphicsAnnot(x$label), x$x, x$y, :
неизвестна ширина символа 0xf7 в кодировке CP1251
\end{verbatim}

\begin{verbatim}
Warning in grid.Call(C_textBounds, as.graphicsAnnot(x$label), x$x, x$y, :
неизвестна ширина символа 0xe5 в кодировке CP1251
\end{verbatim}

\begin{verbatim}
Warning in grid.Call(C_textBounds, as.graphicsAnnot(x$label), x$x, x$y, :
неизвестна ширина символа 0xf1 в кодировке CP1251
\end{verbatim}

\begin{verbatim}
Warning in grid.Call(C_textBounds, as.graphicsAnnot(x$label), x$x, x$y, :
неизвестна ширина символа 0xea в кодировке CP1251
\end{verbatim}

\begin{verbatim}
Warning in grid.Call(C_textBounds, as.graphicsAnnot(x$label), x$x, x$y, :
неизвестна ширина символа 0xe0 в кодировке CP1251
\end{verbatim}

\begin{verbatim}
Warning in grid.Call(C_textBounds, as.graphicsAnnot(x$label), x$x, x$y, :
неизвестна ширина символа 0xff в кодировке CP1251
\end{verbatim}

\begin{verbatim}
Warning in grid.Call(C_textBounds, as.graphicsAnnot(x$label), x$x, x$y, :
неизвестна ширина символа 0xf2 в кодировке CP1251
\end{verbatim}

\begin{verbatim}
Warning in grid.Call(C_textBounds, as.graphicsAnnot(x$label), x$x, x$y, :
неизвестна ширина символа 0xee в кодировке CP1251
\end{verbatim}

\begin{verbatim}
Warning in grid.Call(C_textBounds, as.graphicsAnnot(x$label), x$x, x$y, :
неизвестна ширина символа 0xf7 в кодировке CP1251
\end{verbatim}

\begin{verbatim}
Warning in grid.Call(C_textBounds, as.graphicsAnnot(x$label), x$x, x$y, :
неизвестна ширина символа 0xea в кодировке CP1251
\end{verbatim}

\begin{verbatim}
Warning in grid.Call(C_textBounds, as.graphicsAnnot(x$label), x$x, x$y, :
неизвестна ширина символа 0xe8 в кодировке CP1251
\end{verbatim}

\begin{verbatim}
Warning in grid.Call(C_textBounds, as.graphicsAnnot(x$label), x$x, x$y, :
неизвестна ширина символа 0xe4 в кодировке CP1251
\end{verbatim}

\begin{verbatim}
Warning in grid.Call(C_textBounds, as.graphicsAnnot(x$label), x$x, x$y, :
неизвестна ширина символа 0xe8 в кодировке CP1251
\end{verbatim}

\begin{verbatim}
Warning in grid.Call(C_textBounds, as.graphicsAnnot(x$label), x$x, x$y, :
неизвестна ширина символа 0xed в кодировке CP1251
\end{verbatim}

\begin{verbatim}
Warning in grid.Call(C_textBounds, as.graphicsAnnot(x$label), x$x, x$y, :
неизвестна ширина символа 0xe0 в кодировке CP1251
\end{verbatim}

\begin{verbatim}
Warning in grid.Call(C_textBounds, as.graphicsAnnot(x$label), x$x, x$y, :
неизвестна ширина символа 0xec в кодировке CP1251
\end{verbatim}

\begin{verbatim}
Warning in grid.Call(C_textBounds, as.graphicsAnnot(x$label), x$x, x$y, :
неизвестна ширина символа 0xe8 в кодировке CP1251
\end{verbatim}

\begin{verbatim}
Warning in grid.Call(C_textBounds, as.graphicsAnnot(x$label), x$x, x$y, :
неизвестна ширина символа 0xea в кодировке CP1251
\end{verbatim}

\begin{verbatim}
Warning in grid.Call(C_textBounds, as.graphicsAnnot(x$label), x$x, x$y, :
неизвестна ширина символа 0xe0 в кодировке CP1251
\end{verbatim}

\begin{verbatim}
Warning in grid.Call(C_textBounds, as.graphicsAnnot(x$label), x$x, x$y, :
неизвестна ширина символа 0xf0 в кодировке CP1251
\end{verbatim}

\begin{verbatim}
Warning in grid.Call(C_textBounds, as.graphicsAnnot(x$label), x$x, x$y, :
неизвестна ширина символа 0xe5 в кодировке CP1251
\end{verbatim}

\begin{verbatim}
Warning in grid.Call(C_textBounds, as.graphicsAnnot(x$label), x$x, x$y, :
неизвестна ширина символа 0xea в кодировке CP1251
\end{verbatim}

\begin{verbatim}
Warning in grid.Call(C_textBounds, as.graphicsAnnot(x$label), x$x, x$y, :
неизвестна ширина символа 0xf0 в кодировке CP1251
\end{verbatim}

\begin{verbatim}
Warning in grid.Call(C_textBounds, as.graphicsAnnot(x$label), x$x, x$y, :
неизвестна ширина символа 0xf3 в кодировке CP1251
\end{verbatim}

\begin{verbatim}
Warning in grid.Call(C_textBounds, as.graphicsAnnot(x$label), x$x, x$y, :
неизвестна ширина символа 0xf2 в кодировке CP1251
\end{verbatim}

\begin{verbatim}
Warning in grid.Call(C_textBounds, as.graphicsAnnot(x$label), x$x, x$y, :
неизвестна ширина символа 0xee в кодировке CP1251
\end{verbatim}

\begin{verbatim}
Warning in grid.Call(C_textBounds, as.graphicsAnnot(x$label), x$x, x$y, :
неизвестна ширина символа 0xe2 в кодировке CP1251
\end{verbatim}

\begin{verbatim}
Warning in grid.Call(C_textBounds, as.graphicsAnnot(x$label), x$x, x$y, :
неизвестна ширина символа 0xc3 в кодировке CP1251
\end{verbatim}

\begin{verbatim}
Warning in grid.Call(C_textBounds, as.graphicsAnnot(x$label), x$x, x$y, :
неизвестна ширина символа 0xee в кодировке CP1251
\end{verbatim}

\begin{verbatim}
Warning in grid.Call(C_textBounds, as.graphicsAnnot(x$label), x$x, x$y, :
неизвестна ширина символа 0xe4 в кодировке CP1251
\end{verbatim}

\begin{verbatim}
Warning in grid.Call(C_textBounds, as.graphicsAnnot(x$label), x$x, x$y, :
неизвестна ширина символа 0xfb в кодировке CP1251
\end{verbatim}

\begin{verbatim}
Warning in grid.Call.graphics(C_text, as.graphicsAnnot(x$label), x$x, x$y, :
неизвестна ширина символа 0xc3 в кодировке CP1251
\end{verbatim}

\begin{verbatim}
Warning in grid.Call.graphics(C_text, as.graphicsAnnot(x$label), x$x, x$y, :
неизвестна ширина символа 0xee в кодировке CP1251
\end{verbatim}

\begin{verbatim}
Warning in grid.Call.graphics(C_text, as.graphicsAnnot(x$label), x$x, x$y, :
неизвестна ширина символа 0xe4 в кодировке CP1251
\end{verbatim}

\begin{verbatim}
Warning in grid.Call.graphics(C_text, as.graphicsAnnot(x$label), x$x, x$y, :
неизвестна ширина символа 0xfb в кодировке CP1251
\end{verbatim}

\begin{verbatim}
Warning in grid.Call.graphics(C_text, as.graphicsAnnot(x$label), x$x, x$y, :
неизвестна ширина символа 0xd0 в кодировке CP1251
\end{verbatim}

\begin{verbatim}
Warning in grid.Call.graphics(C_text, as.graphicsAnnot(x$label), x$x, x$y, :
неизвестна ширина символа 0xe5 в кодировке CP1251
\end{verbatim}

\begin{verbatim}
Warning in grid.Call.graphics(C_text, as.graphicsAnnot(x$label), x$x, x$y, :
неизвестна ширина символа 0xea в кодировке CP1251
\end{verbatim}

\begin{verbatim}
Warning in grid.Call.graphics(C_text, as.graphicsAnnot(x$label), x$x, x$y, :
неизвестна ширина символа 0xf0 в кодировке CP1251
\end{verbatim}

\begin{verbatim}
Warning in grid.Call.graphics(C_text, as.graphicsAnnot(x$label), x$x, x$y, :
неизвестна ширина символа 0xf3 в кодировке CP1251
\end{verbatim}

\begin{verbatim}
Warning in grid.Call.graphics(C_text, as.graphicsAnnot(x$label), x$x, x$y, :
неизвестна ширина символа 0xf2 в кодировке CP1251
\end{verbatim}

\begin{verbatim}
Warning in grid.Call.graphics(C_text, as.graphicsAnnot(x$label), x$x, x$y, :
неизвестна ширина символа 0xfb в кодировке CP1251
\end{verbatim}

\begin{verbatim}
Warning in grid.Call.graphics(C_text, as.graphicsAnnot(x$label), x$x, x$y, :
неизвестна ширина символа 0xfd в кодировке CP1251
\end{verbatim}

\begin{verbatim}
Warning in grid.Call.graphics(C_text, as.graphicsAnnot(x$label), x$x, x$y, :
неизвестна ширина символа 0xea в кодировке CP1251
\end{verbatim}

\begin{verbatim}
Warning in grid.Call.graphics(C_text, as.graphicsAnnot(x$label), x$x, x$y, :
неизвестна ширина символа 0xe7 в кодировке CP1251
\end{verbatim}

\begin{verbatim}
Warning in grid.Call.graphics(C_text, as.graphicsAnnot(x$label), x$x, x$y, :
неизвестна ширина символа 0xcc в кодировке CP1251
\end{verbatim}

\begin{verbatim}
Warning in grid.Call.graphics(C_text, as.graphicsAnnot(x$label), x$x, x$y, :
неизвестна ширина символа 0xee в кодировке CP1251
\end{verbatim}

\begin{verbatim}
Warning in grid.Call.graphics(C_text, as.graphicsAnnot(x$label), x$x, x$y, :
неизвестна ширина символа 0xe4 в кодировке CP1251
\end{verbatim}

\begin{verbatim}
Warning in grid.Call.graphics(C_text, as.graphicsAnnot(x$label), x$x, x$y, :
неизвестна ширина символа 0xe5 в кодировке CP1251
\end{verbatim}

\begin{verbatim}
Warning in grid.Call.graphics(C_text, as.graphicsAnnot(x$label), x$x, x$y, :
неизвестна ширина символа 0xeb в кодировке CP1251
\end{verbatim}

\begin{verbatim}
Warning in grid.Call.graphics(C_text, as.graphicsAnnot(x$label), x$x, x$y, :
неизвестна ширина символа 0xe8 в кодировке CP1251
\end{verbatim}

\begin{verbatim}
Warning in grid.Call.graphics(C_text, as.graphicsAnnot(x$label), x$x, x$y, :
неизвестна ширина символа 0xf0 в кодировке CP1251
\end{verbatim}

\begin{verbatim}
Warning in grid.Call.graphics(C_text, as.graphicsAnnot(x$label), x$x, x$y, :
неизвестна ширина символа 0xf3 в кодировке CP1251
\end{verbatim}

\begin{verbatim}
Warning in grid.Call.graphics(C_text, as.graphicsAnnot(x$label), x$x, x$y, :
неизвестна ширина символа 0xe5 в кодировке CP1251
\end{verbatim}

\begin{verbatim}
Warning in grid.Call.graphics(C_text, as.graphicsAnnot(x$label), x$x, x$y, :
неизвестна ширина символа 0xec в кодировке CP1251
\end{verbatim}

\begin{verbatim}
Warning in grid.Call.graphics(C_text, as.graphicsAnnot(x$label), x$x, x$y, :
неизвестна ширина символа 0xe0 в кодировке CP1251
\end{verbatim}

\begin{verbatim}
Warning in grid.Call.graphics(C_text, as.graphicsAnnot(x$label), x$x, x$y, :
неизвестна ширина символа 0xff в кодировке CP1251
\end{verbatim}

\begin{verbatim}
Warning in grid.Call.graphics(C_text, as.graphicsAnnot(x$label), x$x, x$y, :
неизвестна ширина символа 0xe8 в кодировке CP1251
\end{verbatim}

\begin{verbatim}
Warning in grid.Call.graphics(C_text, as.graphicsAnnot(x$label), x$x, x$y, :
неизвестна ширина символа 0xf4 в кодировке CP1251
\end{verbatim}

\begin{verbatim}
Warning in grid.Call.graphics(C_text, as.graphicsAnnot(x$label), x$x, x$y, :
неизвестна ширина символа 0xe0 в кодировке CP1251
\end{verbatim}

\begin{verbatim}
Warning in grid.Call.graphics(C_text, as.graphicsAnnot(x$label), x$x, x$y, :
неизвестна ширина символа 0xea в кодировке CP1251
\end{verbatim}

\begin{verbatim}
Warning in grid.Call.graphics(C_text, as.graphicsAnnot(x$label), x$x, x$y, :
неизвестна ширина символа 0xf2 в кодировке CP1251
\end{verbatim}

\begin{verbatim}
Warning in grid.Call.graphics(C_text, as.graphicsAnnot(x$label), x$x, x$y, :
неизвестна ширина символа 0xe8 в кодировке CP1251
\end{verbatim}

\begin{verbatim}
Warning in grid.Call.graphics(C_text, as.graphicsAnnot(x$label), x$x, x$y, :
неизвестна ширина символа 0xf7 в кодировке CP1251
\end{verbatim}

\begin{verbatim}
Warning in grid.Call.graphics(C_text, as.graphicsAnnot(x$label), x$x, x$y, :
неизвестна ширина символа 0xe5 в кодировке CP1251
\end{verbatim}

\begin{verbatim}
Warning in grid.Call.graphics(C_text, as.graphicsAnnot(x$label), x$x, x$y, :
неизвестна ширина символа 0xf1 в кодировке CP1251
\end{verbatim}

\begin{verbatim}
Warning in grid.Call.graphics(C_text, as.graphicsAnnot(x$label), x$x, x$y, :
неизвестна ширина символа 0xea в кодировке CP1251
\end{verbatim}

\begin{verbatim}
Warning in grid.Call.graphics(C_text, as.graphicsAnnot(x$label), x$x, x$y, :
неизвестна ширина символа 0xe0 в кодировке CP1251
\end{verbatim}

\begin{verbatim}
Warning in grid.Call.graphics(C_text, as.graphicsAnnot(x$label), x$x, x$y, :
неизвестна ширина символа 0xff в кодировке CP1251
\end{verbatim}

\begin{verbatim}
Warning in grid.Call.graphics(C_text, as.graphicsAnnot(x$label), x$x, x$y, :
неизвестна ширина символа 0xf2 в кодировке CP1251
\end{verbatim}

\begin{verbatim}
Warning in grid.Call.graphics(C_text, as.graphicsAnnot(x$label), x$x, x$y, :
неизвестна ширина символа 0xee в кодировке CP1251
\end{verbatim}

\begin{verbatim}
Warning in grid.Call.graphics(C_text, as.graphicsAnnot(x$label), x$x, x$y, :
неизвестна ширина символа 0xf7 в кодировке CP1251
\end{verbatim}

\begin{verbatim}
Warning in grid.Call.graphics(C_text, as.graphicsAnnot(x$label), x$x, x$y, :
неизвестна ширина символа 0xea в кодировке CP1251
\end{verbatim}

\begin{verbatim}
Warning in grid.Call.graphics(C_text, as.graphicsAnnot(x$label), x$x, x$y, :
неизвестна ширина символа 0xe8 в кодировке CP1251
\end{verbatim}

\begin{verbatim}
Warning in grid.Call.graphics(C_text, as.graphicsAnnot(x$label), x$x, x$y, :
неизвестна ширина символа 0xe4 в кодировке CP1251
\end{verbatim}

\begin{verbatim}
Warning in grid.Call.graphics(C_text, as.graphicsAnnot(x$label), x$x, x$y, :
неизвестна ширина символа 0xe8 в кодировке CP1251
\end{verbatim}

\begin{verbatim}
Warning in grid.Call.graphics(C_text, as.graphicsAnnot(x$label), x$x, x$y, :
неизвестна ширина символа 0xed в кодировке CP1251
\end{verbatim}

\begin{verbatim}
Warning in grid.Call.graphics(C_text, as.graphicsAnnot(x$label), x$x, x$y, :
неизвестна ширина символа 0xe0 в кодировке CP1251
\end{verbatim}

\begin{verbatim}
Warning in grid.Call.graphics(C_text, as.graphicsAnnot(x$label), x$x, x$y, :
неизвестна ширина символа 0xec в кодировке CP1251
\end{verbatim}

\begin{verbatim}
Warning in grid.Call.graphics(C_text, as.graphicsAnnot(x$label), x$x, x$y, :
неизвестна ширина символа 0xe8 в кодировке CP1251
\end{verbatim}

\begin{verbatim}
Warning in grid.Call.graphics(C_text, as.graphicsAnnot(x$label), x$x, x$y, :
неизвестна ширина символа 0xea в кодировке CP1251
\end{verbatim}

\begin{verbatim}
Warning in grid.Call.graphics(C_text, as.graphicsAnnot(x$label), x$x, x$y, :
неизвестна ширина символа 0xe0 в кодировке CP1251
\end{verbatim}

\begin{verbatim}
Warning in grid.Call.graphics(C_text, as.graphicsAnnot(x$label), x$x, x$y, :
неизвестна ширина символа 0xf0 в кодировке CP1251
\end{verbatim}

\begin{verbatim}
Warning in grid.Call.graphics(C_text, as.graphicsAnnot(x$label), x$x, x$y, :
неизвестна ширина символа 0xe5 в кодировке CP1251
\end{verbatim}

\begin{verbatim}
Warning in grid.Call.graphics(C_text, as.graphicsAnnot(x$label), x$x, x$y, :
неизвестна ширина символа 0xea в кодировке CP1251
\end{verbatim}

\begin{verbatim}
Warning in grid.Call.graphics(C_text, as.graphicsAnnot(x$label), x$x, x$y, :
неизвестна ширина символа 0xf0 в кодировке CP1251
\end{verbatim}

\begin{verbatim}
Warning in grid.Call.graphics(C_text, as.graphicsAnnot(x$label), x$x, x$y, :
неизвестна ширина символа 0xf3 в кодировке CP1251
\end{verbatim}

\begin{verbatim}
Warning in grid.Call.graphics(C_text, as.graphicsAnnot(x$label), x$x, x$y, :
неизвестна ширина символа 0xf2 в кодировке CP1251
\end{verbatim}

\begin{verbatim}
Warning in grid.Call.graphics(C_text, as.graphicsAnnot(x$label), x$x, x$y, :
неизвестна ширина символа 0xee в кодировке CP1251
\end{verbatim}

\begin{verbatim}
Warning in grid.Call.graphics(C_text, as.graphicsAnnot(x$label), x$x, x$y, :
неизвестна ширина символа 0xe2 в кодировке CP1251
\end{verbatim}

\pandocbounded{\includegraphics[keepaspectratio]{chapter8_files/figure-pdf/unnamed-chunk-1-4.pdf}}

\begin{Shaded}
\begin{Highlighting}[]
\NormalTok{p\_P }\OtherTok{\textless{}{-}} \FunctionTok{ggplot}\NormalTok{(df\_p, }\FunctionTok{aes}\NormalTok{(}\AttributeTok{x =}\NormalTok{ YEAR)) }\SpecialCharTok{+}
  \FunctionTok{geom\_ribbon}\NormalTok{(}\FunctionTok{aes}\NormalTok{(}\AttributeTok{ymin =}\NormalTok{ lo, }\AttributeTok{ymax =}\NormalTok{ hi), }\AttributeTok{fill =} \StringTok{"\#ff7f0e"}\NormalTok{, }\AttributeTok{alpha =} \FloatTok{0.2}\NormalTok{) }\SpecialCharTok{+}
  \FunctionTok{geom\_line}\NormalTok{(}\FunctionTok{aes}\NormalTok{(}\AttributeTok{y =}\NormalTok{ median), }\AttributeTok{color =} \StringTok{"\#ff7f0e"}\NormalTok{, }\AttributeTok{linewidth =} \DecValTok{1}\NormalTok{) }\SpecialCharTok{+}
  \FunctionTok{geom\_point}\NormalTok{(}\FunctionTok{aes}\NormalTok{(}\AttributeTok{y =}\NormalTok{ obs), }\AttributeTok{shape =} \DecValTok{21}\NormalTok{, }\AttributeTok{size =} \DecValTok{2}\NormalTok{, }\AttributeTok{color =} \StringTok{"black"}\NormalTok{, }\AttributeTok{fill =} \StringTok{"white"}\NormalTok{, }\AttributeTok{na.rm =} \ConstantTok{TRUE}\NormalTok{) }\SpecialCharTok{+}
  \FunctionTok{scale\_x\_continuous}\NormalTok{(}\AttributeTok{breaks =} \FunctionTok{seq}\NormalTok{(}\DecValTok{2000}\NormalTok{, }\DecValTok{2015}\NormalTok{, }\AttributeTok{by =} \DecValTok{2}\NormalTok{), }\AttributeTok{limits =} \FunctionTok{c}\NormalTok{(}\DecValTok{2000}\NormalTok{, }\DecValTok{2015}\NormalTok{)) }\SpecialCharTok{+}
  \FunctionTok{labs}\NormalTok{(}\AttributeTok{title =} \StringTok{"Моделируемая и фактическая (точки) динамика пострекрутов"}\NormalTok{, }\AttributeTok{x =} \StringTok{"Годы"}\NormalTok{, }\AttributeTok{y =} \StringTok{"Пострекруты (экз.)"}\NormalTok{) }\SpecialCharTok{+}
  \FunctionTok{theme\_minimal}\NormalTok{(}\AttributeTok{base\_size =} \DecValTok{12}\NormalTok{)}

\FunctionTok{print}\NormalTok{(p\_P)}
\end{Highlighting}
\end{Shaded}

\begin{verbatim}
Warning in grid.Call(C_textBounds, as.graphicsAnnot(x$label), x$x, x$y, :
неизвестна ширина символа 0xcf в кодировке CP1251
\end{verbatim}

\begin{verbatim}
Warning in grid.Call(C_textBounds, as.graphicsAnnot(x$label), x$x, x$y, :
неизвестна ширина символа 0xee в кодировке CP1251
\end{verbatim}

\begin{verbatim}
Warning in grid.Call(C_textBounds, as.graphicsAnnot(x$label), x$x, x$y, :
неизвестна ширина символа 0xf1 в кодировке CP1251
\end{verbatim}

\begin{verbatim}
Warning in grid.Call(C_textBounds, as.graphicsAnnot(x$label), x$x, x$y, :
неизвестна ширина символа 0xf2 в кодировке CP1251
\end{verbatim}

\begin{verbatim}
Warning in grid.Call(C_textBounds, as.graphicsAnnot(x$label), x$x, x$y, :
неизвестна ширина символа 0xf0 в кодировке CP1251
\end{verbatim}

\begin{verbatim}
Warning in grid.Call(C_textBounds, as.graphicsAnnot(x$label), x$x, x$y, :
неизвестна ширина символа 0xe5 в кодировке CP1251
\end{verbatim}

\begin{verbatim}
Warning in grid.Call(C_textBounds, as.graphicsAnnot(x$label), x$x, x$y, :
неизвестна ширина символа 0xea в кодировке CP1251
\end{verbatim}

\begin{verbatim}
Warning in grid.Call(C_textBounds, as.graphicsAnnot(x$label), x$x, x$y, :
неизвестна ширина символа 0xf0 в кодировке CP1251
\end{verbatim}

\begin{verbatim}
Warning in grid.Call(C_textBounds, as.graphicsAnnot(x$label), x$x, x$y, :
неизвестна ширина символа 0xf3 в кодировке CP1251
\end{verbatim}

\begin{verbatim}
Warning in grid.Call(C_textBounds, as.graphicsAnnot(x$label), x$x, x$y, :
неизвестна ширина символа 0xf2 в кодировке CP1251
\end{verbatim}

\begin{verbatim}
Warning in grid.Call(C_textBounds, as.graphicsAnnot(x$label), x$x, x$y, :
неизвестна ширина символа 0xfb в кодировке CP1251
\end{verbatim}

\begin{verbatim}
Warning in grid.Call(C_textBounds, as.graphicsAnnot(x$label), x$x, x$y, :
неизвестна ширина символа 0xfd в кодировке CP1251
\end{verbatim}

\begin{verbatim}
Warning in grid.Call(C_textBounds, as.graphicsAnnot(x$label), x$x, x$y, :
неизвестна ширина символа 0xea в кодировке CP1251
\end{verbatim}

\begin{verbatim}
Warning in grid.Call(C_textBounds, as.graphicsAnnot(x$label), x$x, x$y, :
неизвестна ширина символа 0xe7 в кодировке CP1251
\end{verbatim}

\begin{verbatim}
Warning in grid.Call(C_textBounds, as.graphicsAnnot(x$label), x$x, x$y, :
неизвестна ширина символа 0xcc в кодировке CP1251
\end{verbatim}

\begin{verbatim}
Warning in grid.Call(C_textBounds, as.graphicsAnnot(x$label), x$x, x$y, :
неизвестна ширина символа 0xee в кодировке CP1251
\end{verbatim}

\begin{verbatim}
Warning in grid.Call(C_textBounds, as.graphicsAnnot(x$label), x$x, x$y, :
неизвестна ширина символа 0xe4 в кодировке CP1251
\end{verbatim}

\begin{verbatim}
Warning in grid.Call(C_textBounds, as.graphicsAnnot(x$label), x$x, x$y, :
неизвестна ширина символа 0xe5 в кодировке CP1251
\end{verbatim}

\begin{verbatim}
Warning in grid.Call(C_textBounds, as.graphicsAnnot(x$label), x$x, x$y, :
неизвестна ширина символа 0xeb в кодировке CP1251
\end{verbatim}

\begin{verbatim}
Warning in grid.Call(C_textBounds, as.graphicsAnnot(x$label), x$x, x$y, :
неизвестна ширина символа 0xe8 в кодировке CP1251
\end{verbatim}

\begin{verbatim}
Warning in grid.Call(C_textBounds, as.graphicsAnnot(x$label), x$x, x$y, :
неизвестна ширина символа 0xf0 в кодировке CP1251
\end{verbatim}

\begin{verbatim}
Warning in grid.Call(C_textBounds, as.graphicsAnnot(x$label), x$x, x$y, :
неизвестна ширина символа 0xf3 в кодировке CP1251
\end{verbatim}

\begin{verbatim}
Warning in grid.Call(C_textBounds, as.graphicsAnnot(x$label), x$x, x$y, :
неизвестна ширина символа 0xe5 в кодировке CP1251
\end{verbatim}

\begin{verbatim}
Warning in grid.Call(C_textBounds, as.graphicsAnnot(x$label), x$x, x$y, :
неизвестна ширина символа 0xec в кодировке CP1251
\end{verbatim}

\begin{verbatim}
Warning in grid.Call(C_textBounds, as.graphicsAnnot(x$label), x$x, x$y, :
неизвестна ширина символа 0xe0 в кодировке CP1251
\end{verbatim}

\begin{verbatim}
Warning in grid.Call(C_textBounds, as.graphicsAnnot(x$label), x$x, x$y, :
неизвестна ширина символа 0xff в кодировке CP1251
\end{verbatim}

\begin{verbatim}
Warning in grid.Call(C_textBounds, as.graphicsAnnot(x$label), x$x, x$y, :
неизвестна ширина символа 0xe8 в кодировке CP1251
\end{verbatim}

\begin{verbatim}
Warning in grid.Call(C_textBounds, as.graphicsAnnot(x$label), x$x, x$y, :
неизвестна ширина символа 0xf4 в кодировке CP1251
\end{verbatim}

\begin{verbatim}
Warning in grid.Call(C_textBounds, as.graphicsAnnot(x$label), x$x, x$y, :
неизвестна ширина символа 0xe0 в кодировке CP1251
\end{verbatim}

\begin{verbatim}
Warning in grid.Call(C_textBounds, as.graphicsAnnot(x$label), x$x, x$y, :
неизвестна ширина символа 0xea в кодировке CP1251
\end{verbatim}

\begin{verbatim}
Warning in grid.Call(C_textBounds, as.graphicsAnnot(x$label), x$x, x$y, :
неизвестна ширина символа 0xf2 в кодировке CP1251
\end{verbatim}

\begin{verbatim}
Warning in grid.Call(C_textBounds, as.graphicsAnnot(x$label), x$x, x$y, :
неизвестна ширина символа 0xe8 в кодировке CP1251
\end{verbatim}

\begin{verbatim}
Warning in grid.Call(C_textBounds, as.graphicsAnnot(x$label), x$x, x$y, :
неизвестна ширина символа 0xf7 в кодировке CP1251
\end{verbatim}

\begin{verbatim}
Warning in grid.Call(C_textBounds, as.graphicsAnnot(x$label), x$x, x$y, :
неизвестна ширина символа 0xe5 в кодировке CP1251
\end{verbatim}

\begin{verbatim}
Warning in grid.Call(C_textBounds, as.graphicsAnnot(x$label), x$x, x$y, :
неизвестна ширина символа 0xf1 в кодировке CP1251
\end{verbatim}

\begin{verbatim}
Warning in grid.Call(C_textBounds, as.graphicsAnnot(x$label), x$x, x$y, :
неизвестна ширина символа 0xea в кодировке CP1251
\end{verbatim}

\begin{verbatim}
Warning in grid.Call(C_textBounds, as.graphicsAnnot(x$label), x$x, x$y, :
неизвестна ширина символа 0xe0 в кодировке CP1251
\end{verbatim}

\begin{verbatim}
Warning in grid.Call(C_textBounds, as.graphicsAnnot(x$label), x$x, x$y, :
неизвестна ширина символа 0xff в кодировке CP1251
\end{verbatim}

\begin{verbatim}
Warning in grid.Call(C_textBounds, as.graphicsAnnot(x$label), x$x, x$y, :
неизвестна ширина символа 0xf2 в кодировке CP1251
\end{verbatim}

\begin{verbatim}
Warning in grid.Call(C_textBounds, as.graphicsAnnot(x$label), x$x, x$y, :
неизвестна ширина символа 0xee в кодировке CP1251
\end{verbatim}

\begin{verbatim}
Warning in grid.Call(C_textBounds, as.graphicsAnnot(x$label), x$x, x$y, :
неизвестна ширина символа 0xf7 в кодировке CP1251
\end{verbatim}

\begin{verbatim}
Warning in grid.Call(C_textBounds, as.graphicsAnnot(x$label), x$x, x$y, :
неизвестна ширина символа 0xea в кодировке CP1251
\end{verbatim}

\begin{verbatim}
Warning in grid.Call(C_textBounds, as.graphicsAnnot(x$label), x$x, x$y, :
неизвестна ширина символа 0xe8 в кодировке CP1251
\end{verbatim}

\begin{verbatim}
Warning in grid.Call(C_textBounds, as.graphicsAnnot(x$label), x$x, x$y, :
неизвестна ширина символа 0xe4 в кодировке CP1251
\end{verbatim}

\begin{verbatim}
Warning in grid.Call(C_textBounds, as.graphicsAnnot(x$label), x$x, x$y, :
неизвестна ширина символа 0xe8 в кодировке CP1251
\end{verbatim}

\begin{verbatim}
Warning in grid.Call(C_textBounds, as.graphicsAnnot(x$label), x$x, x$y, :
неизвестна ширина символа 0xed в кодировке CP1251
\end{verbatim}

\begin{verbatim}
Warning in grid.Call(C_textBounds, as.graphicsAnnot(x$label), x$x, x$y, :
неизвестна ширина символа 0xe0 в кодировке CP1251
\end{verbatim}

\begin{verbatim}
Warning in grid.Call(C_textBounds, as.graphicsAnnot(x$label), x$x, x$y, :
неизвестна ширина символа 0xec в кодировке CP1251
\end{verbatim}

\begin{verbatim}
Warning in grid.Call(C_textBounds, as.graphicsAnnot(x$label), x$x, x$y, :
неизвестна ширина символа 0xe8 в кодировке CP1251
\end{verbatim}

\begin{verbatim}
Warning in grid.Call(C_textBounds, as.graphicsAnnot(x$label), x$x, x$y, :
неизвестна ширина символа 0xea в кодировке CP1251
\end{verbatim}

\begin{verbatim}
Warning in grid.Call(C_textBounds, as.graphicsAnnot(x$label), x$x, x$y, :
неизвестна ширина символа 0xe0 в кодировке CP1251
\end{verbatim}

\begin{verbatim}
Warning in grid.Call(C_textBounds, as.graphicsAnnot(x$label), x$x, x$y, :
неизвестна ширина символа 0xef в кодировке CP1251
\end{verbatim}

\begin{verbatim}
Warning in grid.Call(C_textBounds, as.graphicsAnnot(x$label), x$x, x$y, :
неизвестна ширина символа 0xee в кодировке CP1251
\end{verbatim}

\begin{verbatim}
Warning in grid.Call(C_textBounds, as.graphicsAnnot(x$label), x$x, x$y, :
неизвестна ширина символа 0xf1 в кодировке CP1251
\end{verbatim}

\begin{verbatim}
Warning in grid.Call(C_textBounds, as.graphicsAnnot(x$label), x$x, x$y, :
неизвестна ширина символа 0xf2 в кодировке CP1251
\end{verbatim}

\begin{verbatim}
Warning in grid.Call(C_textBounds, as.graphicsAnnot(x$label), x$x, x$y, :
неизвестна ширина символа 0xf0 в кодировке CP1251
\end{verbatim}

\begin{verbatim}
Warning in grid.Call(C_textBounds, as.graphicsAnnot(x$label), x$x, x$y, :
неизвестна ширина символа 0xe5 в кодировке CP1251
\end{verbatim}

\begin{verbatim}
Warning in grid.Call(C_textBounds, as.graphicsAnnot(x$label), x$x, x$y, :
неизвестна ширина символа 0xea в кодировке CP1251
\end{verbatim}

\begin{verbatim}
Warning in grid.Call(C_textBounds, as.graphicsAnnot(x$label), x$x, x$y, :
неизвестна ширина символа 0xf0 в кодировке CP1251
\end{verbatim}

\begin{verbatim}
Warning in grid.Call(C_textBounds, as.graphicsAnnot(x$label), x$x, x$y, :
неизвестна ширина символа 0xf3 в кодировке CP1251
\end{verbatim}

\begin{verbatim}
Warning in grid.Call(C_textBounds, as.graphicsAnnot(x$label), x$x, x$y, :
неизвестна ширина символа 0xf2 в кодировке CP1251
\end{verbatim}

\begin{verbatim}
Warning in grid.Call(C_textBounds, as.graphicsAnnot(x$label), x$x, x$y, :
неизвестна ширина символа 0xee в кодировке CP1251
\end{verbatim}

\begin{verbatim}
Warning in grid.Call(C_textBounds, as.graphicsAnnot(x$label), x$x, x$y, :
неизвестна ширина символа 0xe2 в кодировке CP1251
\end{verbatim}

\begin{verbatim}
Warning in grid.Call(C_textBounds, as.graphicsAnnot(x$label), x$x, x$y, :
неизвестна ширина символа 0xc3 в кодировке CP1251
\end{verbatim}

\begin{verbatim}
Warning in grid.Call(C_textBounds, as.graphicsAnnot(x$label), x$x, x$y, :
неизвестна ширина символа 0xee в кодировке CP1251
\end{verbatim}

\begin{verbatim}
Warning in grid.Call(C_textBounds, as.graphicsAnnot(x$label), x$x, x$y, :
неизвестна ширина символа 0xe4 в кодировке CP1251
\end{verbatim}

\begin{verbatim}
Warning in grid.Call(C_textBounds, as.graphicsAnnot(x$label), x$x, x$y, :
неизвестна ширина символа 0xfb в кодировке CP1251
\end{verbatim}

\begin{verbatim}
Warning in grid.Call.graphics(C_text, as.graphicsAnnot(x$label), x$x, x$y, :
неизвестна ширина символа 0xc3 в кодировке CP1251
\end{verbatim}

\begin{verbatim}
Warning in grid.Call.graphics(C_text, as.graphicsAnnot(x$label), x$x, x$y, :
неизвестна ширина символа 0xee в кодировке CP1251
\end{verbatim}

\begin{verbatim}
Warning in grid.Call.graphics(C_text, as.graphicsAnnot(x$label), x$x, x$y, :
неизвестна ширина символа 0xe4 в кодировке CP1251
\end{verbatim}

\begin{verbatim}
Warning in grid.Call.graphics(C_text, as.graphicsAnnot(x$label), x$x, x$y, :
неизвестна ширина символа 0xfb в кодировке CP1251
\end{verbatim}

\begin{verbatim}
Warning in grid.Call.graphics(C_text, as.graphicsAnnot(x$label), x$x, x$y, :
неизвестна ширина символа 0xcf в кодировке CP1251
\end{verbatim}

\begin{verbatim}
Warning in grid.Call.graphics(C_text, as.graphicsAnnot(x$label), x$x, x$y, :
неизвестна ширина символа 0xee в кодировке CP1251
\end{verbatim}

\begin{verbatim}
Warning in grid.Call.graphics(C_text, as.graphicsAnnot(x$label), x$x, x$y, :
неизвестна ширина символа 0xf1 в кодировке CP1251
\end{verbatim}

\begin{verbatim}
Warning in grid.Call.graphics(C_text, as.graphicsAnnot(x$label), x$x, x$y, :
неизвестна ширина символа 0xf2 в кодировке CP1251
\end{verbatim}

\begin{verbatim}
Warning in grid.Call.graphics(C_text, as.graphicsAnnot(x$label), x$x, x$y, :
неизвестна ширина символа 0xf0 в кодировке CP1251
\end{verbatim}

\begin{verbatim}
Warning in grid.Call.graphics(C_text, as.graphicsAnnot(x$label), x$x, x$y, :
неизвестна ширина символа 0xe5 в кодировке CP1251
\end{verbatim}

\begin{verbatim}
Warning in grid.Call.graphics(C_text, as.graphicsAnnot(x$label), x$x, x$y, :
неизвестна ширина символа 0xea в кодировке CP1251
\end{verbatim}

\begin{verbatim}
Warning in grid.Call.graphics(C_text, as.graphicsAnnot(x$label), x$x, x$y, :
неизвестна ширина символа 0xf0 в кодировке CP1251
\end{verbatim}

\begin{verbatim}
Warning in grid.Call.graphics(C_text, as.graphicsAnnot(x$label), x$x, x$y, :
неизвестна ширина символа 0xf3 в кодировке CP1251
\end{verbatim}

\begin{verbatim}
Warning in grid.Call.graphics(C_text, as.graphicsAnnot(x$label), x$x, x$y, :
неизвестна ширина символа 0xf2 в кодировке CP1251
\end{verbatim}

\begin{verbatim}
Warning in grid.Call.graphics(C_text, as.graphicsAnnot(x$label), x$x, x$y, :
неизвестна ширина символа 0xfb в кодировке CP1251
\end{verbatim}

\begin{verbatim}
Warning in grid.Call.graphics(C_text, as.graphicsAnnot(x$label), x$x, x$y, :
неизвестна ширина символа 0xfd в кодировке CP1251
\end{verbatim}

\begin{verbatim}
Warning in grid.Call.graphics(C_text, as.graphicsAnnot(x$label), x$x, x$y, :
неизвестна ширина символа 0xea в кодировке CP1251
\end{verbatim}

\begin{verbatim}
Warning in grid.Call.graphics(C_text, as.graphicsAnnot(x$label), x$x, x$y, :
неизвестна ширина символа 0xe7 в кодировке CP1251
\end{verbatim}

\begin{verbatim}
Warning in grid.Call.graphics(C_text, as.graphicsAnnot(x$label), x$x, x$y, :
неизвестна ширина символа 0xcc в кодировке CP1251
\end{verbatim}

\begin{verbatim}
Warning in grid.Call.graphics(C_text, as.graphicsAnnot(x$label), x$x, x$y, :
неизвестна ширина символа 0xee в кодировке CP1251
\end{verbatim}

\begin{verbatim}
Warning in grid.Call.graphics(C_text, as.graphicsAnnot(x$label), x$x, x$y, :
неизвестна ширина символа 0xe4 в кодировке CP1251
\end{verbatim}

\begin{verbatim}
Warning in grid.Call.graphics(C_text, as.graphicsAnnot(x$label), x$x, x$y, :
неизвестна ширина символа 0xe5 в кодировке CP1251
\end{verbatim}

\begin{verbatim}
Warning in grid.Call.graphics(C_text, as.graphicsAnnot(x$label), x$x, x$y, :
неизвестна ширина символа 0xeb в кодировке CP1251
\end{verbatim}

\begin{verbatim}
Warning in grid.Call.graphics(C_text, as.graphicsAnnot(x$label), x$x, x$y, :
неизвестна ширина символа 0xe8 в кодировке CP1251
\end{verbatim}

\begin{verbatim}
Warning in grid.Call.graphics(C_text, as.graphicsAnnot(x$label), x$x, x$y, :
неизвестна ширина символа 0xf0 в кодировке CP1251
\end{verbatim}

\begin{verbatim}
Warning in grid.Call.graphics(C_text, as.graphicsAnnot(x$label), x$x, x$y, :
неизвестна ширина символа 0xf3 в кодировке CP1251
\end{verbatim}

\begin{verbatim}
Warning in grid.Call.graphics(C_text, as.graphicsAnnot(x$label), x$x, x$y, :
неизвестна ширина символа 0xe5 в кодировке CP1251
\end{verbatim}

\begin{verbatim}
Warning in grid.Call.graphics(C_text, as.graphicsAnnot(x$label), x$x, x$y, :
неизвестна ширина символа 0xec в кодировке CP1251
\end{verbatim}

\begin{verbatim}
Warning in grid.Call.graphics(C_text, as.graphicsAnnot(x$label), x$x, x$y, :
неизвестна ширина символа 0xe0 в кодировке CP1251
\end{verbatim}

\begin{verbatim}
Warning in grid.Call.graphics(C_text, as.graphicsAnnot(x$label), x$x, x$y, :
неизвестна ширина символа 0xff в кодировке CP1251
\end{verbatim}

\begin{verbatim}
Warning in grid.Call.graphics(C_text, as.graphicsAnnot(x$label), x$x, x$y, :
неизвестна ширина символа 0xe8 в кодировке CP1251
\end{verbatim}

\begin{verbatim}
Warning in grid.Call.graphics(C_text, as.graphicsAnnot(x$label), x$x, x$y, :
неизвестна ширина символа 0xf4 в кодировке CP1251
\end{verbatim}

\begin{verbatim}
Warning in grid.Call.graphics(C_text, as.graphicsAnnot(x$label), x$x, x$y, :
неизвестна ширина символа 0xe0 в кодировке CP1251
\end{verbatim}

\begin{verbatim}
Warning in grid.Call.graphics(C_text, as.graphicsAnnot(x$label), x$x, x$y, :
неизвестна ширина символа 0xea в кодировке CP1251
\end{verbatim}

\begin{verbatim}
Warning in grid.Call.graphics(C_text, as.graphicsAnnot(x$label), x$x, x$y, :
неизвестна ширина символа 0xf2 в кодировке CP1251
\end{verbatim}

\begin{verbatim}
Warning in grid.Call.graphics(C_text, as.graphicsAnnot(x$label), x$x, x$y, :
неизвестна ширина символа 0xe8 в кодировке CP1251
\end{verbatim}

\begin{verbatim}
Warning in grid.Call.graphics(C_text, as.graphicsAnnot(x$label), x$x, x$y, :
неизвестна ширина символа 0xf7 в кодировке CP1251
\end{verbatim}

\begin{verbatim}
Warning in grid.Call.graphics(C_text, as.graphicsAnnot(x$label), x$x, x$y, :
неизвестна ширина символа 0xe5 в кодировке CP1251
\end{verbatim}

\begin{verbatim}
Warning in grid.Call.graphics(C_text, as.graphicsAnnot(x$label), x$x, x$y, :
неизвестна ширина символа 0xf1 в кодировке CP1251
\end{verbatim}

\begin{verbatim}
Warning in grid.Call.graphics(C_text, as.graphicsAnnot(x$label), x$x, x$y, :
неизвестна ширина символа 0xea в кодировке CP1251
\end{verbatim}

\begin{verbatim}
Warning in grid.Call.graphics(C_text, as.graphicsAnnot(x$label), x$x, x$y, :
неизвестна ширина символа 0xe0 в кодировке CP1251
\end{verbatim}

\begin{verbatim}
Warning in grid.Call.graphics(C_text, as.graphicsAnnot(x$label), x$x, x$y, :
неизвестна ширина символа 0xff в кодировке CP1251
\end{verbatim}

\begin{verbatim}
Warning in grid.Call.graphics(C_text, as.graphicsAnnot(x$label), x$x, x$y, :
неизвестна ширина символа 0xf2 в кодировке CP1251
\end{verbatim}

\begin{verbatim}
Warning in grid.Call.graphics(C_text, as.graphicsAnnot(x$label), x$x, x$y, :
неизвестна ширина символа 0xee в кодировке CP1251
\end{verbatim}

\begin{verbatim}
Warning in grid.Call.graphics(C_text, as.graphicsAnnot(x$label), x$x, x$y, :
неизвестна ширина символа 0xf7 в кодировке CP1251
\end{verbatim}

\begin{verbatim}
Warning in grid.Call.graphics(C_text, as.graphicsAnnot(x$label), x$x, x$y, :
неизвестна ширина символа 0xea в кодировке CP1251
\end{verbatim}

\begin{verbatim}
Warning in grid.Call.graphics(C_text, as.graphicsAnnot(x$label), x$x, x$y, :
неизвестна ширина символа 0xe8 в кодировке CP1251
\end{verbatim}

\begin{verbatim}
Warning in grid.Call.graphics(C_text, as.graphicsAnnot(x$label), x$x, x$y, :
неизвестна ширина символа 0xe4 в кодировке CP1251
\end{verbatim}

\begin{verbatim}
Warning in grid.Call.graphics(C_text, as.graphicsAnnot(x$label), x$x, x$y, :
неизвестна ширина символа 0xe8 в кодировке CP1251
\end{verbatim}

\begin{verbatim}
Warning in grid.Call.graphics(C_text, as.graphicsAnnot(x$label), x$x, x$y, :
неизвестна ширина символа 0xed в кодировке CP1251
\end{verbatim}

\begin{verbatim}
Warning in grid.Call.graphics(C_text, as.graphicsAnnot(x$label), x$x, x$y, :
неизвестна ширина символа 0xe0 в кодировке CP1251
\end{verbatim}

\begin{verbatim}
Warning in grid.Call.graphics(C_text, as.graphicsAnnot(x$label), x$x, x$y, :
неизвестна ширина символа 0xec в кодировке CP1251
\end{verbatim}

\begin{verbatim}
Warning in grid.Call.graphics(C_text, as.graphicsAnnot(x$label), x$x, x$y, :
неизвестна ширина символа 0xe8 в кодировке CP1251
\end{verbatim}

\begin{verbatim}
Warning in grid.Call.graphics(C_text, as.graphicsAnnot(x$label), x$x, x$y, :
неизвестна ширина символа 0xea в кодировке CP1251
\end{verbatim}

\begin{verbatim}
Warning in grid.Call.graphics(C_text, as.graphicsAnnot(x$label), x$x, x$y, :
неизвестна ширина символа 0xe0 в кодировке CP1251
\end{verbatim}

\begin{verbatim}
Warning in grid.Call.graphics(C_text, as.graphicsAnnot(x$label), x$x, x$y, :
неизвестна ширина символа 0xef в кодировке CP1251
\end{verbatim}

\begin{verbatim}
Warning in grid.Call.graphics(C_text, as.graphicsAnnot(x$label), x$x, x$y, :
неизвестна ширина символа 0xee в кодировке CP1251
\end{verbatim}

\begin{verbatim}
Warning in grid.Call.graphics(C_text, as.graphicsAnnot(x$label), x$x, x$y, :
неизвестна ширина символа 0xf1 в кодировке CP1251
\end{verbatim}

\begin{verbatim}
Warning in grid.Call.graphics(C_text, as.graphicsAnnot(x$label), x$x, x$y, :
неизвестна ширина символа 0xf2 в кодировке CP1251
\end{verbatim}

\begin{verbatim}
Warning in grid.Call.graphics(C_text, as.graphicsAnnot(x$label), x$x, x$y, :
неизвестна ширина символа 0xf0 в кодировке CP1251
\end{verbatim}

\begin{verbatim}
Warning in grid.Call.graphics(C_text, as.graphicsAnnot(x$label), x$x, x$y, :
неизвестна ширина символа 0xe5 в кодировке CP1251
\end{verbatim}

\begin{verbatim}
Warning in grid.Call.graphics(C_text, as.graphicsAnnot(x$label), x$x, x$y, :
неизвестна ширина символа 0xea в кодировке CP1251
\end{verbatim}

\begin{verbatim}
Warning in grid.Call.graphics(C_text, as.graphicsAnnot(x$label), x$x, x$y, :
неизвестна ширина символа 0xf0 в кодировке CP1251
\end{verbatim}

\begin{verbatim}
Warning in grid.Call.graphics(C_text, as.graphicsAnnot(x$label), x$x, x$y, :
неизвестна ширина символа 0xf3 в кодировке CP1251
\end{verbatim}

\begin{verbatim}
Warning in grid.Call.graphics(C_text, as.graphicsAnnot(x$label), x$x, x$y, :
неизвестна ширина символа 0xf2 в кодировке CP1251
\end{verbatim}

\begin{verbatim}
Warning in grid.Call.graphics(C_text, as.graphicsAnnot(x$label), x$x, x$y, :
неизвестна ширина символа 0xee в кодировке CP1251
\end{verbatim}

\begin{verbatim}
Warning in grid.Call.graphics(C_text, as.graphicsAnnot(x$label), x$x, x$y, :
неизвестна ширина символа 0xe2 в кодировке CP1251
\end{verbatim}

\pandocbounded{\includegraphics[keepaspectratio]{chapter8_files/figure-pdf/unnamed-chunk-1-5.pdf}}

\bookmarksetup{startatroot}

\chapter{Стандартизация CPUE: GLM, GAM,
GAMM}\label{ux441ux442ux430ux43dux434ux430ux440ux442ux438ux437ux430ux446ux438ux44f-cpue-glm-gam-gamm}

\section{Введение}\label{ux432ux432ux435ux434ux435ux43dux438ux435-9}

Начнём с привычной ловушки. Когда на столе лежит «сырое» CPUE и график
по годам, очень хочется увидеть в нём прямое отражение численности
запаса и, тем самым, штурвал управления промыслом. Это эффект удобной
истории: мозг мгновенно дорисовывает причинность там, где в данных много
посторонней вариации --- сезон, район, глубина, тип орудий, «почерк»
судна, длительность постановок. Если эту вариацию не отделить, CPUE
превращается в градусник, который меняет показания температуры тела
вместе с погодой в комнате. Наша задача в этом занятии --- аккуратно
«изолировать» измерение: стандартизировать CPUE так, чтобы оно
преимущественно отражало динамику запаса, а не всё остальное.

Что именно мы стандартизируем. CPUE --- это маркер доступной части
популяции при данных условиях промысла. Мы хотим извлечь из него индекс,
сопоставимый между годами, сведя к «норме» всё, что не про численность
или биомассу запаса: различия по месяцам, районам, глубинам, орудиям,
судам. Ключ --- заранее определить целевой «эталон»: к чему приводим? К
маргинальным средним по дизайну (сбалансированная виртуальная съёмка)
или к фиксированному референс‑уровню (базовый месяц/район/судно)? Первый
вариант лучше отражает «среднедоступную» систему, второй --- удобен для
прозрачной интерпретации. Оба корректны, если оговорены и
последовательно применяются.

Почему GLM/GAM/GAMM. GLM с гамма‑распределением и лог‑ссылкой ---
рабочая лошадка для положительных, правоскошенных величин; лог‑связь
естественно переводит эффекты в относительные (мультипликативные)
изменения. GAM добавляет гибкость: там, где линейные контрасты по
глубине или температуре ломают картину, гладкие функции честно
показывают оптимумы и изгибы. GAMM вводит случайные эффекты (например,
по судну), отделяя структурные тенденции от «личной биографии» флотилии
--- то, что в фиксированных эффектах превращается в лишний шум и
завышенную уверенность. Выбор не догматичен: начинаем с простого GLM,
расширяем до GAM, добавляем случайные эффекты только тогда, когда это
подтверждается диагностикой и улучшает калибровку индекса.

Про подводные камни, которые мы будем ловить. Во‑первых, нули и «почти
нули»: для гамма‑семейства понадобится аккуратная положительная
поправка; если нулей много, разумна двухступенчатая дельта‑постановка
(биномиальная часть + положительная часть). Во‑вторых, дрейф уловистости
(q‑drift): изменение орудий и практик может имитировать «падение
запаса». Мы не устраняем его магией, но делаем явным --- через
факторы/ковариаты и, при необходимости, случайные эффекты. В‑третьих,
утечка информации из будущего: сравнивая индексы, мы придерживаемся
блокировок по годам и не «подмешиваем» композиционные сдвиги выборки
(например, если флот переместился в другой район). В‑четвёртых,
concurvity/мультиколлинеарность: GAM может «переобъяснить» одно и то же
несколькими гладкими --- проверяем concurvity и упрощаем формулу. И,
конечно, диагностика остатков (DHARMa), проверка дисперсии и влияния
наблюдений --- обязательна перед любыми управленческими выводами.

Как читать результат. Стандартизированный индекс --- это не абсолютная
биомасса и не прогноз улова; это аккуратно отчищенный маркер
относительной доступности/численности, сопоставимый между годами при
прочих равных. Он публикуется с доверительными интервалами и явной
оговоркой о референсе (маргинальные средние или фиксированный профиль).
Нормирование «к среднему» или «к первому году» --- это про удобство
сравнения, а не про «истинную шкалу». Чувствительность к выбору формулы
(GLM vs GAM vs GAMM), к набору факторов и к способу усреднения --- часть
честного отчёта: если индексы согласованы, доверие растёт; если нет ---
разбираем, где скрыт драйвер расхождений.

Что делаем по шагам. Загружаем и чистим данные (включая корректную
работу с нулями и факторами), строим GLM (Gamma‑log) с ключевыми
факторами (год, сезон, район, судно), проверяем остатки и считаем
маргинальные индексы с интервалами. Затем ставим GAM там, где предметная
логика допускает нелинейности, и повторяем диагностику (gam.check,
concurvity). После --- GAMM со случайным эффектом по судну, чтобы
отделить «командный почерк» от тренда, и оцениваем индекс через
предсказания на сбалансированной сетке и бутстреп‑интервалы. Финально
сводим индексы на одном графике вместе с фактическими медианами CPUE и
сравниваем модели по AIC/BIC и адекватности остатков. Если «сложнее» не
лучше --- остаёмся на более простой, но калиброванной модели.

И напоследок --- про уверенность. Хорошая стандартизация CPUE --- это не
способ «победить» неопределённость, а способ честно на неё смотреть. Мы
покажем не только линию, но и ленту интервалов; не только медиану, но и
альтернативы формул; не только тренд, но и диагностику. Это ровно тот
прогресс, который работает в долгую: меньше иллюзий контроля, больше
прозрачности, лучше решения.

Полный скрипт можно скачать по
\href{https://mombus.github.io/cRab/data/GLM.R}{ссылке}.

\textbf{Для работы скрипта:}

\begin{enumerate}
\def\labelenumi{\arabic{enumi}.}
\item
  Скачайте файл данных
  (\href{https://mombus.github.io/cRab/data/KARTOGRAPHIC.xlsx}{KARTOGRAPHIC.xlsx})
\item
  Установите рабочую директорию в setwd()
\item
  Установите необходимые пакеты :
  \textbf{\texttt{install.packages(c("readxl",\ "tidyverse,\ "mgcv",\ "gamm4",\ "DHARMa"\ ))}}
  \texttt{и\ др.}
\end{enumerate}

\section{Пошаговое описание
скрипта}\label{ux43fux43eux448ux430ux433ux43eux432ux43eux435-ux43eux43fux438ux441ux430ux43dux438ux435-ux441ux43aux440ux438ux43fux442ux430}

Скрипт начинается с загрузки необходимых пакетов для обработки данных,
построения моделей и визуализации результатов. Среда настраивается путем
установки рабочей директории и фиксации случайного зерна для обеспечения
воспроизводимости всех последующих вычислений.

На следующем этапе происходит загрузка исходных данных из файла Excel.
Данные представляют собой промысловую статистику, содержащую информацию
о году, месяце, судне, районе, величине улова на усилие (CPUE) и др.
Выполняется их предварительная обработка: фильтрация по осенним месяцам,
преобразование типов переменных в факторы и числовой формат, а также
удаление пропущенных значений. Поскольку для моделирования с
гамма-распределением требуются строго положительные значения, для
нулевых и отрицательных величин CPUE рассчитывается и добавляется малая
поправка. Для первичного ознакомления с данными строится диаграмма
размаха, показывающая распределение CPUE по годам.

Далее определяются вспомогательные функции. Одна функция предназначена
для нормировки рассчитанных индексов либо на среднее значение, либо на
значение первого года. Другая функция использует метод маргинальных
средних для расчета стандартизированных индексов и их доверительных
интервалов на основе подобранной модели. Третья функция реализует расчет
индексов и оценку неопределенности через бутстреп для моделей со сложной
структурой.

Основная часть скрипта посвящена построению и анализу трех типов
моделей. Первой подбирается обобщенная линейная модель (GLM) с
гамма-распределением ошибок и логарифмической связью. В качестве
предикторов используются факторы: год, месяц, судно и район. Для
визуальной и численной диагностики адекватности модели выводятся ее
сводка, таблица коэффициентов, стандартные диагностические графики и
графики остатков, проверенные с помощью пакета DHARMa.

Следующей строится обобщенная аддитивная модель (GAM). На этом этапе
используется та же формула и семейство распределений, что и в GLM, но
метод подбора гиперпараметров отличается. Проводится аналогичная
диагностика модели с помощью функций \textbf{\texttt{summary}} и
\textbf{\texttt{gam.check}}.

Затем подбирается обобщенная аддитивная смешанная модель (GAMM), которая
дополнительно включает случайный эффект от судна. Это позволяет учесть
вариацию, вызванную индивидуальными особенностями каждого судна, которые
не описываются другими факторами. Диагностика этой модели более сложна и
включает анализ остатков, проверку случайных эффектов и тест на
гетероскедастичность.

После построения всех моделей для каждой из них рассчитываются
стандартизированные индексы CPUE и их доверительные интервалы. Для GLM и
GAM это делается с помощью функции, основанной на маргинальных средних,
а для GAMM применяется метод бутстрепа.

Финальный этап включает объединение результатов всех трех моделей в
единую таблицу и их визуальное сравнение на графике. На этот же график
добавляются фактические медианные значения CPUE по годам из исходных
данных для сопоставления со стандартизированными кривыми. В заключение
модели сравниваются по информационным критериям (AIC, BIC) и другим
метрикам, чтобы дать рекомендации по выбору наиболее адекватной из них.

Ниже приводится скрипт, а ниже скрипта - описание результатов
моделирования.

\begin{Shaded}
\begin{Highlighting}[]
\CommentTok{\# ========================================================================================================================}
\CommentTok{\# ПРАКТИЧЕСКОЕ ЗАНЯТИЕ: СТАНДАРТИЗАЦИЯ CPUE С ИСПОЛЬЗОВАНИЕМ GLM, GAM И GAMM}
\CommentTok{\# Курс: "Оценка водных биоресурсов в среде R (для начинающих)"}
\CommentTok{\# Автор: Баканев С.В. }
\CommentTok{\# Дата: 20.08.2025}
\CommentTok{\# }
\CommentTok{\# Структура:}
\CommentTok{\# 1) Загрузка пакетов и настройка среды}
\CommentTok{\# 2) Загрузка и предварительная обработка данных}
\CommentTok{\# 3) Вспомогательные функции для расчета индексов}
\CommentTok{\# 4) Моделирование GLM (Gamma с лог{-}ссылкой)}
\CommentTok{\# 5) Моделирование GAM (обобщенная аддитивная модель)}
\CommentTok{\# 6) Моделирование GAMM (смешанная модель со случайными эффектами)}
\CommentTok{\# 7) Сравнение моделей и финальная визуализация результатов}
\CommentTok{\# ========================================================================================================================}


\CommentTok{\# ==============================================================================}
\CommentTok{\# БЛОК 1: ЗАГРУЗКА ПАКЕТОВ И НАСТРОЙКА СРЕДЫ}
\CommentTok{\# ==============================================================================}

\CommentTok{\# Отключаем вспомогательные сообщения при загрузке пакетов}
\FunctionTok{suppressPackageStartupMessages}\NormalTok{(\{}
  \FunctionTok{library}\NormalTok{(tidyverse)   }\CommentTok{\# Основные пакеты для обработки данных и визуализации}
  \FunctionTok{library}\NormalTok{(readxl)      }\CommentTok{\# Чтение данных из Excel{-}файлов}
  \FunctionTok{library}\NormalTok{(mgcv)        }\CommentTok{\# Обобщенные аддитивные модели (GAM)}
  \FunctionTok{library}\NormalTok{(gamm4)       }\CommentTok{\# GAM со смешанными эффектами}
  \FunctionTok{library}\NormalTok{(emmeans)     }\CommentTok{\# Расчет маргинальных средних и контрастов}
  \FunctionTok{library}\NormalTok{(broom)       }\CommentTok{\# Преобразование результатов моделей в таблицы}
  \FunctionTok{library}\NormalTok{(broom.mixed) }\CommentTok{\# Поддержка смешанных моделей для broom}
  \FunctionTok{library}\NormalTok{(DHARMa)      }\CommentTok{\# Диагностика остатков обобщенных моделей}
  \FunctionTok{library}\NormalTok{(knitr)       }\CommentTok{\# Форматирование таблиц для отчетов}
\NormalTok{\})}

\CommentTok{\# Установка рабочей директории}
\FunctionTok{setwd}\NormalTok{(}\StringTok{"C:/GLM/"}\NormalTok{)}

\CommentTok{\# Фиксируем случайное зерно для воспроизводимости результатов}
\FunctionTok{set.seed}\NormalTok{(}\DecValTok{42}\NormalTok{)}

\CommentTok{\# ==============================================================================}
\CommentTok{\# БЛОК 2: ЗАГРУЗКА И ПРЕДОБРАБОТКА ДАННЫХ}
\CommentTok{\# ==============================================================================}

\CommentTok{\# Определяем путь к файлу с данными}
\NormalTok{DATA\_PATH }\OtherTok{\textless{}{-}} \StringTok{"C:/GLM/data/KARTOGRAPHIC.xlsx"}

\CommentTok{\# Чтение данных из листа "FISHERY" и фильтрация осенних месяцев}
\NormalTok{DATA }\OtherTok{\textless{}{-}} \FunctionTok{read\_excel}\NormalTok{(DATA\_PATH, }\AttributeTok{sheet =} \StringTok{"FISHERY"}\NormalTok{) }\SpecialCharTok{\%\textgreater{}\%}
  \FunctionTok{as\_tibble}\NormalTok{() }\SpecialCharTok{\%\textgreater{}\%}  \CommentTok{\# Преобразуем в современный формат таблицы}
  \FunctionTok{filter}\NormalTok{(MONTH }\SpecialCharTok{\textgreater{}} \DecValTok{8} \SpecialCharTok{\&}\NormalTok{ MONTH }\SpecialCharTok{\textless{}} \DecValTok{12}\NormalTok{)  }\CommentTok{\# Сентябрь{-}ноябрь (осенний сезон)}

\CommentTok{\# Преобразование типов переменных и обработка пропусков}
\NormalTok{DATA }\OtherTok{\textless{}{-}}\NormalTok{ DATA }\SpecialCharTok{\%\textgreater{}\%}
  \FunctionTok{mutate}\NormalTok{(}
    \AttributeTok{YEAR =} \FunctionTok{as.factor}\NormalTok{(YEAR),           }\CommentTok{\# Год как категориальная переменная}
    \AttributeTok{MONTH =} \FunctionTok{as.factor}\NormalTok{(MONTH),         }\CommentTok{\# Месяц как фактор}
    \AttributeTok{CALL =} \FunctionTok{as.factor}\NormalTok{(CALL),           }\CommentTok{\# Идентификатор судна}
    \AttributeTok{REGION =} \FunctionTok{as.factor}\NormalTok{(REGION),       }\CommentTok{\# Рыбохозяйственный район}
    \AttributeTok{VESSELNUMBER =} \FunctionTok{as.factor}\NormalTok{(VESSELNUMBER),  }\CommentTok{\# Номер судна}
    \AttributeTok{CPUE =} \FunctionTok{as.numeric}\NormalTok{(CPUE)           }\CommentTok{\# Целевой показатель {-} улов на усилие}
\NormalTok{  ) }\SpecialCharTok{\%\textgreater{}\%}
  \FunctionTok{filter}\NormalTok{(}\SpecialCharTok{!}\FunctionTok{is.na}\NormalTok{(CPUE))  }\CommentTok{\# Удаление строк с пропусками в CPUE}

\CommentTok{\# Обработка нулевых значений CPUE для Gamma{-}моделей}
\ControlFlowTok{if}\NormalTok{ (}\FunctionTok{any}\NormalTok{(DATA}\SpecialCharTok{$}\NormalTok{CPUE }\SpecialCharTok{\textless{}=} \DecValTok{0}\NormalTok{, }\AttributeTok{na.rm =} \ConstantTok{TRUE}\NormalTok{)) \{}
\NormalTok{  min\_pos }\OtherTok{\textless{}{-}} \FunctionTok{min}\NormalTok{(DATA}\SpecialCharTok{$}\NormalTok{CPUE[DATA}\SpecialCharTok{$}\NormalTok{CPUE }\SpecialCharTok{\textgreater{}} \DecValTok{0}\NormalTok{], }\AttributeTok{na.rm =} \ConstantTok{TRUE}\NormalTok{)  }\CommentTok{\# Минимальный положительный улов}
\NormalTok{  offset }\OtherTok{\textless{}{-}}\NormalTok{ min\_pos }\SpecialCharTok{/} \DecValTok{2}  \CommentTok{\# Величина поправки}
\NormalTok{  DATA }\OtherTok{\textless{}{-}}\NormalTok{ DATA }\SpecialCharTok{\%\textgreater{}\%} 
    \FunctionTok{mutate}\NormalTok{(}\AttributeTok{CPUE\_POS =} \FunctionTok{if\_else}\NormalTok{(CPUE }\SpecialCharTok{\textless{}=} \DecValTok{0}\NormalTok{, CPUE }\SpecialCharTok{+}\NormalTok{ offset, CPUE))  }\CommentTok{\# Добавляем поправку}
\NormalTok{\} }\ControlFlowTok{else}\NormalTok{ \{}
\NormalTok{  DATA }\OtherTok{\textless{}{-}}\NormalTok{ DATA }\SpecialCharTok{\%\textgreater{}\%} 
    \FunctionTok{mutate}\NormalTok{(}\AttributeTok{CPUE\_POS =}\NormalTok{ CPUE)  }\CommentTok{\# Исходные данные если нулей нет}
\NormalTok{\}}

\CommentTok{\# Рассчитываем медианные значения CPUE по годам из исходных данных}
\NormalTok{actual\_medians }\OtherTok{\textless{}{-}}\NormalTok{ DATA }\SpecialCharTok{\%\textgreater{}\%}
  \FunctionTok{group\_by}\NormalTok{(YEAR) }\SpecialCharTok{\%\textgreater{}\%}
  \FunctionTok{summarise}\NormalTok{(}\AttributeTok{median\_cpue =} \FunctionTok{median}\NormalTok{(CPUE, }\AttributeTok{na.rm =} \ConstantTok{TRUE}\NormalTok{))}
\CommentTok{\# Рассчитываем медианные значения CPUE по годам из исходных данных для последующих графиков}
\NormalTok{actual\_medians}
\end{Highlighting}
\end{Shaded}

\begin{verbatim}
# A tibble: 6 x 2
  YEAR  median_cpue
  <fct>       <dbl>
1 2019        200. 
2 2020        116. 
3 2021        132. 
4 2022         84  
5 2023         79.4
6 2024         58.3
\end{verbatim}

\begin{Shaded}
\begin{Highlighting}[]
\CommentTok{\# Экспресс{-}визуализация распределения CPUE по годам}
\NormalTok{DATA }\SpecialCharTok{\%\textgreater{}\%}
  \FunctionTok{ggplot}\NormalTok{(}\FunctionTok{aes}\NormalTok{(}\AttributeTok{x =}\NormalTok{ YEAR, }\AttributeTok{y =}\NormalTok{ CPUE)) }\SpecialCharTok{+}
  \FunctionTok{geom\_boxplot}\NormalTok{(}\AttributeTok{outlier.alpha =} \FloatTok{0.2}\NormalTok{) }\SpecialCharTok{+}
  \FunctionTok{labs}\NormalTok{(}\AttributeTok{title =} \StringTok{"Распределение CPUE по годам"}\NormalTok{, }
       \AttributeTok{x =} \StringTok{"Год"}\NormalTok{, }
       \AttributeTok{y =} \StringTok{"CPUE (улов на усилие)"}\NormalTok{)}
\end{Highlighting}
\end{Shaded}

\begin{verbatim}
Warning in grid.Call(C_textBounds, as.graphicsAnnot(x$label), x$x, x$y, :
неизвестна ширина символа 0xf3 в кодировке CP1251
\end{verbatim}

\begin{verbatim}
Warning in grid.Call(C_textBounds, as.graphicsAnnot(x$label), x$x, x$y, :
неизвестна ширина символа 0xeb в кодировке CP1251
\end{verbatim}

\begin{verbatim}
Warning in grid.Call(C_textBounds, as.graphicsAnnot(x$label), x$x, x$y, :
неизвестна ширина символа 0xee в кодировке CP1251
\end{verbatim}

\begin{verbatim}
Warning in grid.Call(C_textBounds, as.graphicsAnnot(x$label), x$x, x$y, :
неизвестна ширина символа 0xe2 в кодировке CP1251
\end{verbatim}

\begin{verbatim}
Warning in grid.Call(C_textBounds, as.graphicsAnnot(x$label), x$x, x$y, :
неизвестна ширина символа 0xed в кодировке CP1251
\end{verbatim}

\begin{verbatim}
Warning in grid.Call(C_textBounds, as.graphicsAnnot(x$label), x$x, x$y, :
неизвестна ширина символа 0xe0 в кодировке CP1251
\end{verbatim}

\begin{verbatim}
Warning in grid.Call(C_textBounds, as.graphicsAnnot(x$label), x$x, x$y, :
неизвестна ширина символа 0xf3 в кодировке CP1251
\end{verbatim}

\begin{verbatim}
Warning in grid.Call(C_textBounds, as.graphicsAnnot(x$label), x$x, x$y, :
неизвестна ширина символа 0xf1 в кодировке CP1251
\end{verbatim}

\begin{verbatim}
Warning in grid.Call(C_textBounds, as.graphicsAnnot(x$label), x$x, x$y, :
неизвестна ширина символа 0xe8 в кодировке CP1251
\end{verbatim}

\begin{verbatim}
Warning in grid.Call(C_textBounds, as.graphicsAnnot(x$label), x$x, x$y, :
неизвестна ширина символа 0xeb в кодировке CP1251
\end{verbatim}

\begin{verbatim}
Warning in grid.Call(C_textBounds, as.graphicsAnnot(x$label), x$x, x$y, :
неизвестна ширина символа 0xe8 в кодировке CP1251
\end{verbatim}

\begin{verbatim}
Warning in grid.Call(C_textBounds, as.graphicsAnnot(x$label), x$x, x$y, :
неизвестна ширина символа 0xe5 в кодировке CP1251
\end{verbatim}

\begin{verbatim}
Warning in grid.Call(C_textBounds, as.graphicsAnnot(x$label), x$x, x$y, :
неизвестна ширина символа 0xd0 в кодировке CP1251
\end{verbatim}

\begin{verbatim}
Warning in grid.Call(C_textBounds, as.graphicsAnnot(x$label), x$x, x$y, :
неизвестна ширина символа 0xe0 в кодировке CP1251
\end{verbatim}

\begin{verbatim}
Warning in grid.Call(C_textBounds, as.graphicsAnnot(x$label), x$x, x$y, :
неизвестна ширина символа 0xf1 в кодировке CP1251
\end{verbatim}

\begin{verbatim}
Warning in grid.Call(C_textBounds, as.graphicsAnnot(x$label), x$x, x$y, :
неизвестна ширина символа 0xef в кодировке CP1251
\end{verbatim}

\begin{verbatim}
Warning in grid.Call(C_textBounds, as.graphicsAnnot(x$label), x$x, x$y, :
неизвестна ширина символа 0xf0 в кодировке CP1251
\end{verbatim}

\begin{verbatim}
Warning in grid.Call(C_textBounds, as.graphicsAnnot(x$label), x$x, x$y, :
неизвестна ширина символа 0xe5 в кодировке CP1251
\end{verbatim}

\begin{verbatim}
Warning in grid.Call(C_textBounds, as.graphicsAnnot(x$label), x$x, x$y, :
неизвестна ширина символа 0xe4 в кодировке CP1251
\end{verbatim}

\begin{verbatim}
Warning in grid.Call(C_textBounds, as.graphicsAnnot(x$label), x$x, x$y, :
неизвестна ширина символа 0xe5 в кодировке CP1251
\end{verbatim}

\begin{verbatim}
Warning in grid.Call(C_textBounds, as.graphicsAnnot(x$label), x$x, x$y, :
неизвестна ширина символа 0xeb в кодировке CP1251
\end{verbatim}

\begin{verbatim}
Warning in grid.Call(C_textBounds, as.graphicsAnnot(x$label), x$x, x$y, :
неизвестна ширина символа 0xe5 в кодировке CP1251
\end{verbatim}

\begin{verbatim}
Warning in grid.Call(C_textBounds, as.graphicsAnnot(x$label), x$x, x$y, :
неизвестна ширина символа 0xed в кодировке CP1251
\end{verbatim}

\begin{verbatim}
Warning in grid.Call(C_textBounds, as.graphicsAnnot(x$label), x$x, x$y, :
неизвестна ширина символа 0xe8 в кодировке CP1251
\end{verbatim}

\begin{verbatim}
Warning in grid.Call(C_textBounds, as.graphicsAnnot(x$label), x$x, x$y, :
неизвестна ширина символа 0xe5 в кодировке CP1251
\end{verbatim}

\begin{verbatim}
Warning in grid.Call(C_textBounds, as.graphicsAnnot(x$label), x$x, x$y, :
неизвестна ширина символа 0xef в кодировке CP1251
\end{verbatim}

\begin{verbatim}
Warning in grid.Call(C_textBounds, as.graphicsAnnot(x$label), x$x, x$y, :
неизвестна ширина символа 0xee в кодировке CP1251
\end{verbatim}

\begin{verbatim}
Warning in grid.Call(C_textBounds, as.graphicsAnnot(x$label), x$x, x$y, :
неизвестна ширина символа 0xe3 в кодировке CP1251
\end{verbatim}

\begin{verbatim}
Warning in grid.Call(C_textBounds, as.graphicsAnnot(x$label), x$x, x$y, :
неизвестна ширина символа 0xee в кодировке CP1251
\end{verbatim}

\begin{verbatim}
Warning in grid.Call(C_textBounds, as.graphicsAnnot(x$label), x$x, x$y, :
неизвестна ширина символа 0xe4 в кодировке CP1251
\end{verbatim}

\begin{verbatim}
Warning in grid.Call(C_textBounds, as.graphicsAnnot(x$label), x$x, x$y, :
неизвестна ширина символа 0xe0 в кодировке CP1251
\end{verbatim}

\begin{verbatim}
Warning in grid.Call(C_textBounds, as.graphicsAnnot(x$label), x$x, x$y, :
неизвестна ширина символа 0xec в кодировке CP1251
\end{verbatim}

\begin{verbatim}
Warning in grid.Call(C_textBounds, as.graphicsAnnot(x$label), x$x, x$y, :
неизвестна ширина символа 0xc3 в кодировке CP1251
\end{verbatim}

\begin{verbatim}
Warning in grid.Call(C_textBounds, as.graphicsAnnot(x$label), x$x, x$y, :
неизвестна ширина символа 0xee в кодировке CP1251
\end{verbatim}

\begin{verbatim}
Warning in grid.Call(C_textBounds, as.graphicsAnnot(x$label), x$x, x$y, :
неизвестна ширина символа 0xe4 в кодировке CP1251
\end{verbatim}

\begin{verbatim}
Warning in grid.Call.graphics(C_text, as.graphicsAnnot(x$label), x$x, x$y, :
неизвестна ширина символа 0xc3 в кодировке CP1251
\end{verbatim}

\begin{verbatim}
Warning in grid.Call.graphics(C_text, as.graphicsAnnot(x$label), x$x, x$y, :
неизвестна ширина символа 0xee в кодировке CP1251
\end{verbatim}

\begin{verbatim}
Warning in grid.Call.graphics(C_text, as.graphicsAnnot(x$label), x$x, x$y, :
неизвестна ширина символа 0xe4 в кодировке CP1251
\end{verbatim}

\begin{verbatim}
Warning in grid.Call.graphics(C_text, as.graphicsAnnot(x$label), x$x, x$y, :
неизвестна ширина символа 0xf3 в кодировке CP1251
\end{verbatim}

\begin{verbatim}
Warning in grid.Call.graphics(C_text, as.graphicsAnnot(x$label), x$x, x$y, :
неизвестна ширина символа 0xeb в кодировке CP1251
\end{verbatim}

\begin{verbatim}
Warning in grid.Call.graphics(C_text, as.graphicsAnnot(x$label), x$x, x$y, :
неизвестна ширина символа 0xee в кодировке CP1251
\end{verbatim}

\begin{verbatim}
Warning in grid.Call.graphics(C_text, as.graphicsAnnot(x$label), x$x, x$y, :
неизвестна ширина символа 0xe2 в кодировке CP1251
\end{verbatim}

\begin{verbatim}
Warning in grid.Call.graphics(C_text, as.graphicsAnnot(x$label), x$x, x$y, :
неизвестна ширина символа 0xed в кодировке CP1251
\end{verbatim}

\begin{verbatim}
Warning in grid.Call.graphics(C_text, as.graphicsAnnot(x$label), x$x, x$y, :
неизвестна ширина символа 0xe0 в кодировке CP1251
\end{verbatim}

\begin{verbatim}
Warning in grid.Call.graphics(C_text, as.graphicsAnnot(x$label), x$x, x$y, :
неизвестна ширина символа 0xf3 в кодировке CP1251
\end{verbatim}

\begin{verbatim}
Warning in grid.Call.graphics(C_text, as.graphicsAnnot(x$label), x$x, x$y, :
неизвестна ширина символа 0xf1 в кодировке CP1251
\end{verbatim}

\begin{verbatim}
Warning in grid.Call.graphics(C_text, as.graphicsAnnot(x$label), x$x, x$y, :
неизвестна ширина символа 0xe8 в кодировке CP1251
\end{verbatim}

\begin{verbatim}
Warning in grid.Call.graphics(C_text, as.graphicsAnnot(x$label), x$x, x$y, :
неизвестна ширина символа 0xeb в кодировке CP1251
\end{verbatim}

\begin{verbatim}
Warning in grid.Call.graphics(C_text, as.graphicsAnnot(x$label), x$x, x$y, :
неизвестна ширина символа 0xe8 в кодировке CP1251
\end{verbatim}

\begin{verbatim}
Warning in grid.Call.graphics(C_text, as.graphicsAnnot(x$label), x$x, x$y, :
неизвестна ширина символа 0xe5 в кодировке CP1251
\end{verbatim}

\begin{verbatim}
Warning in grid.Call.graphics(C_text, as.graphicsAnnot(x$label), x$x, x$y, :
неизвестна ширина символа 0xd0 в кодировке CP1251
\end{verbatim}

\begin{verbatim}
Warning in grid.Call.graphics(C_text, as.graphicsAnnot(x$label), x$x, x$y, :
неизвестна ширина символа 0xe0 в кодировке CP1251
\end{verbatim}

\begin{verbatim}
Warning in grid.Call.graphics(C_text, as.graphicsAnnot(x$label), x$x, x$y, :
неизвестна ширина символа 0xf1 в кодировке CP1251
\end{verbatim}

\begin{verbatim}
Warning in grid.Call.graphics(C_text, as.graphicsAnnot(x$label), x$x, x$y, :
неизвестна ширина символа 0xef в кодировке CP1251
\end{verbatim}

\begin{verbatim}
Warning in grid.Call.graphics(C_text, as.graphicsAnnot(x$label), x$x, x$y, :
неизвестна ширина символа 0xf0 в кодировке CP1251
\end{verbatim}

\begin{verbatim}
Warning in grid.Call.graphics(C_text, as.graphicsAnnot(x$label), x$x, x$y, :
неизвестна ширина символа 0xe5 в кодировке CP1251
\end{verbatim}

\begin{verbatim}
Warning in grid.Call.graphics(C_text, as.graphicsAnnot(x$label), x$x, x$y, :
неизвестна ширина символа 0xe4 в кодировке CP1251
\end{verbatim}

\begin{verbatim}
Warning in grid.Call.graphics(C_text, as.graphicsAnnot(x$label), x$x, x$y, :
неизвестна ширина символа 0xe5 в кодировке CP1251
\end{verbatim}

\begin{verbatim}
Warning in grid.Call.graphics(C_text, as.graphicsAnnot(x$label), x$x, x$y, :
неизвестна ширина символа 0xeb в кодировке CP1251
\end{verbatim}

\begin{verbatim}
Warning in grid.Call.graphics(C_text, as.graphicsAnnot(x$label), x$x, x$y, :
неизвестна ширина символа 0xe5 в кодировке CP1251
\end{verbatim}

\begin{verbatim}
Warning in grid.Call.graphics(C_text, as.graphicsAnnot(x$label), x$x, x$y, :
неизвестна ширина символа 0xed в кодировке CP1251
\end{verbatim}

\begin{verbatim}
Warning in grid.Call.graphics(C_text, as.graphicsAnnot(x$label), x$x, x$y, :
неизвестна ширина символа 0xe8 в кодировке CP1251
\end{verbatim}

\begin{verbatim}
Warning in grid.Call.graphics(C_text, as.graphicsAnnot(x$label), x$x, x$y, :
неизвестна ширина символа 0xe5 в кодировке CP1251
\end{verbatim}

\begin{verbatim}
Warning in grid.Call.graphics(C_text, as.graphicsAnnot(x$label), x$x, x$y, :
неизвестна ширина символа 0xef в кодировке CP1251
\end{verbatim}

\begin{verbatim}
Warning in grid.Call.graphics(C_text, as.graphicsAnnot(x$label), x$x, x$y, :
неизвестна ширина символа 0xee в кодировке CP1251
\end{verbatim}

\begin{verbatim}
Warning in grid.Call.graphics(C_text, as.graphicsAnnot(x$label), x$x, x$y, :
неизвестна ширина символа 0xe3 в кодировке CP1251
\end{verbatim}

\begin{verbatim}
Warning in grid.Call.graphics(C_text, as.graphicsAnnot(x$label), x$x, x$y, :
неизвестна ширина символа 0xee в кодировке CP1251
\end{verbatim}

\begin{verbatim}
Warning in grid.Call.graphics(C_text, as.graphicsAnnot(x$label), x$x, x$y, :
неизвестна ширина символа 0xe4 в кодировке CP1251
\end{verbatim}

\begin{verbatim}
Warning in grid.Call.graphics(C_text, as.graphicsAnnot(x$label), x$x, x$y, :
неизвестна ширина символа 0xe0 в кодировке CP1251
\end{verbatim}

\begin{verbatim}
Warning in grid.Call.graphics(C_text, as.graphicsAnnot(x$label), x$x, x$y, :
неизвестна ширина символа 0xec в кодировке CP1251
\end{verbatim}

\pandocbounded{\includegraphics[keepaspectratio]{chapter9_files/figure-pdf/unnamed-chunk-1-1.pdf}}

\begin{Shaded}
\begin{Highlighting}[]
\CommentTok{\# ==============================================================================}
\CommentTok{\# БЛОК 3: ВСПОМОГАТЕЛЬНЫЕ ФУНКЦИИ ДЛЯ РАСЧЕТА ИНДЕКСОВ}
\CommentTok{\# ==============================================================================}

\CommentTok{\# Функция нормировки индексов}
\NormalTok{scale\_to\_index }\OtherTok{\textless{}{-}} \ControlFlowTok{function}\NormalTok{(values, }\AttributeTok{method =} \FunctionTok{c}\NormalTok{(}\StringTok{"mean"}\NormalTok{, }\StringTok{"first"}\NormalTok{)) \{}
\NormalTok{  method }\OtherTok{\textless{}{-}} \FunctionTok{match.arg}\NormalTok{(method)}
  \ControlFlowTok{if}\NormalTok{ (method }\SpecialCharTok{==} \StringTok{"mean"}\NormalTok{) \{}
    \CommentTok{\# Нормировка на среднее значение}
    \FunctionTok{return}\NormalTok{(}\FunctionTok{as.numeric}\NormalTok{(values) }\SpecialCharTok{/} \FunctionTok{mean}\NormalTok{(}\FunctionTok{as.numeric}\NormalTok{(values), }\AttributeTok{na.rm =} \ConstantTok{TRUE}\NormalTok{))}
\NormalTok{  \}}
  \ControlFlowTok{if}\NormalTok{ (method }\SpecialCharTok{==} \StringTok{"first"}\NormalTok{) \{}
    \CommentTok{\# Нормировка на значение первого года}
    \FunctionTok{return}\NormalTok{(}\FunctionTok{as.numeric}\NormalTok{(values) }\SpecialCharTok{/} \FunctionTok{as.numeric}\NormalTok{(values[}\DecValTok{1}\NormalTok{]))}
\NormalTok{  \}}
\NormalTok{\}}

\CommentTok{\# Функция расчета индексов для GLM/GAM через маргинальные средние}
\NormalTok{emmeans\_standardized\_index }\OtherTok{\textless{}{-}} \ControlFlowTok{function}\NormalTok{(model, }\AttributeTok{variable =} \StringTok{"YEAR"}\NormalTok{) \{}
\NormalTok{  out }\OtherTok{\textless{}{-}} \FunctionTok{suppressWarnings}\NormalTok{(}
    \FunctionTok{emmeans}\NormalTok{(model, }
            \AttributeTok{specs =} \FunctionTok{as.formula}\NormalTok{(}\FunctionTok{paste0}\NormalTok{(}\StringTok{"\textasciitilde{} "}\NormalTok{, variable)), }
            \AttributeTok{type =} \StringTok{"response"}\NormalTok{)}
\NormalTok{  )}
\NormalTok{  df }\OtherTok{\textless{}{-}} \FunctionTok{as\_tibble}\NormalTok{(out) }\SpecialCharTok{\%\textgreater{}\%} 
    \FunctionTok{select}\NormalTok{(}\SpecialCharTok{!!}\FunctionTok{sym}\NormalTok{(variable), }\AttributeTok{response =}\NormalTok{ response, lower.CL, upper.CL)}
  \FunctionTok{colnames}\NormalTok{(df) }\OtherTok{\textless{}{-}} \FunctionTok{c}\NormalTok{(}\StringTok{"YEAR"}\NormalTok{, }\StringTok{"value"}\NormalTok{, }\StringTok{"lcl"}\NormalTok{, }\StringTok{"ucl"}\NormalTok{)}
\NormalTok{  df}
\NormalTok{\}}

\CommentTok{\# Функция расчета индексов для GAMM через бутстреп}
\NormalTok{compute\_standardized\_index }\OtherTok{\textless{}{-}} \ControlFlowTok{function}\NormalTok{(model, base\_data, year\_levels, predict\_fun,}
                                      \AttributeTok{response\_transform =}\NormalTok{ identity, }
                                      \AttributeTok{n\_boot =} \DecValTok{200}\NormalTok{L, }
                                      \AttributeTok{seed =} \DecValTok{7}\NormalTok{L) \{}
  \FunctionTok{set.seed}\NormalTok{(seed)}
\NormalTok{  acc }\OtherTok{\textless{}{-}} \FunctionTok{vector}\NormalTok{(}\StringTok{"list"}\NormalTok{, }\FunctionTok{length}\NormalTok{(year\_levels))}
  \ControlFlowTok{for}\NormalTok{ (i }\ControlFlowTok{in} \FunctionTok{seq\_along}\NormalTok{(year\_levels)) \{}
\NormalTok{    newdata }\OtherTok{\textless{}{-}}\NormalTok{ base\_data}
\NormalTok{    newdata}\SpecialCharTok{$}\NormalTok{YEAR }\OtherTok{\textless{}{-}} \FunctionTok{factor}\NormalTok{(year\_levels[i], }\AttributeTok{levels =} \FunctionTok{levels}\NormalTok{(base\_data}\SpecialCharTok{$}\NormalTok{YEAR))}
\NormalTok{    preds }\OtherTok{\textless{}{-}} \FunctionTok{suppressWarnings}\NormalTok{(}\FunctionTok{predict\_fun}\NormalTok{(model, newdata))}
\NormalTok{    mu }\OtherTok{\textless{}{-}} \FunctionTok{mean}\NormalTok{(}\FunctionTok{response\_transform}\NormalTok{(preds), }\AttributeTok{na.rm =} \ConstantTok{TRUE}\NormalTok{)}
    \CommentTok{\# Бутстреп для оценки неопределенности}
\NormalTok{    boot\_vals }\OtherTok{\textless{}{-}} \FunctionTok{replicate}\NormalTok{(n\_boot, \{}
\NormalTok{      idx }\OtherTok{\textless{}{-}} \FunctionTok{sample.int}\NormalTok{(}\FunctionTok{nrow}\NormalTok{(base\_data), }\FunctionTok{nrow}\NormalTok{(base\_data), }\AttributeTok{replace =} \ConstantTok{TRUE}\NormalTok{)}
\NormalTok{      bd }\OtherTok{\textless{}{-}}\NormalTok{ newdata[idx, , drop }\OtherTok{=} \ConstantTok{FALSE}\NormalTok{]}
\NormalTok{      p }\OtherTok{\textless{}{-}} \FunctionTok{suppressWarnings}\NormalTok{(}\FunctionTok{predict\_fun}\NormalTok{(model, bd))}
      \FunctionTok{mean}\NormalTok{(}\FunctionTok{response\_transform}\NormalTok{(p), }\AttributeTok{na.rm =} \ConstantTok{TRUE}\NormalTok{)}
\NormalTok{    \})}
\NormalTok{    ci }\OtherTok{\textless{}{-}} \FunctionTok{quantile}\NormalTok{(boot\_vals, }\FunctionTok{c}\NormalTok{(}\FloatTok{0.025}\NormalTok{, }\FloatTok{0.975}\NormalTok{), }\AttributeTok{na.rm =} \ConstantTok{TRUE}\NormalTok{)}
\NormalTok{    acc[[i]] }\OtherTok{\textless{}{-}} \FunctionTok{tibble}\NormalTok{(}\AttributeTok{YEAR =}\NormalTok{ year\_levels[i], }\AttributeTok{value =}\NormalTok{ mu, }\AttributeTok{lcl =}\NormalTok{ ci[[}\DecValTok{1}\NormalTok{]], }\AttributeTok{ucl =}\NormalTok{ ci[[}\DecValTok{2}\NormalTok{]])}
\NormalTok{  \}}
  \FunctionTok{bind\_rows}\NormalTok{(acc)}
\NormalTok{\}}

\CommentTok{\# ==============================================================================}
\CommentTok{\# БЛОК 4: МОДЕЛИРОВАНИЕ GLM (GAMMA С ЛОГ{-}ССЫЛКОЙ)}
\CommentTok{\# ==============================================================================}

\CommentTok{\# Подбор модели с фиксированными эффектами}
\NormalTok{glm\_gamma\_fit }\OtherTok{\textless{}{-}} \FunctionTok{glm}\NormalTok{(}
\NormalTok{  CPUE\_POS }\SpecialCharTok{\textasciitilde{}}\NormalTok{ YEAR }\SpecialCharTok{+}\NormalTok{ MONTH }\SpecialCharTok{+}\NormalTok{ CALL }\SpecialCharTok{+}\NormalTok{ REGION,  }\CommentTok{\# Формула с факторными предикторами}
  \AttributeTok{family =} \FunctionTok{Gamma}\NormalTok{(}\AttributeTok{link =} \StringTok{"log"}\NormalTok{),            }\CommentTok{\# Гамма{-}распределение с логарифмической связью}
  \AttributeTok{data =}\NormalTok{ DATA}
\NormalTok{)}

\CommentTok{\# Диагностика модели}
\FunctionTok{summary}\NormalTok{(glm\_gamma\_fit)  }\CommentTok{\# Стандартная сводка модели}
\end{Highlighting}
\end{Shaded}

\begin{verbatim}

Call:
glm(formula = CPUE_POS ~ YEAR + MONTH + CALL + REGION, family = Gamma(link = "log"), 
    data = DATA)

Coefficients:
                                                  Estimate Std. Error t value
(Intercept)                                        5.57642    0.07942  70.216
YEAR2020                                          -0.22713    0.04586  -4.953
YEAR2021                                          -0.22438    0.04557  -4.924
YEAR2022                                          -0.64329    0.04345 -14.806
YEAR2023                                          -0.77311    0.04647 -16.637
YEAR2024                                          -1.12817    0.05157 -21.879
MONTH10                                           -0.13725    0.02443  -5.617
MONTH11                                           -0.13714    0.03540  -3.874
CALLUAAK                                          -0.29534    0.06092  -4.848
CALLUAKC                                          -0.53490    0.06381  -8.382
CALLUBEV                                          -3.67233    0.12187 -30.133
CALLUBQQ                                          -0.35433    0.07057  -5.021
CALLUBSR                                          -0.33516    0.06266  -5.349
CALLUBUR                                          -0.58246    0.06175  -9.433
CALLUBYT                                          -0.21520    0.06088  -3.535
CALLUCFF                                          -0.06756    0.09825  -0.688
CALLUCXF                                          -2.34302    0.10638 -22.025
CALLUDII                                          -0.46287    0.05710  -8.107
CALLUDUT                                          -1.02162    0.07550 -13.532
CALLUDWM                                          -2.42502    0.09480 -25.582
CALLUEBK                                           0.25852    0.13371   1.934
CALLUEMO                                          -0.17024    0.08103  -2.101
CALLUENZ                                           0.04928    0.06442   0.765
CALLUFIK                                           0.29700    0.13043   2.277
CALLUGXE                                          -0.56384    0.06265  -9.000
CALLUIVO                                          -0.21572    0.06500  -3.319
REGIONCEB.-ЦEHTPAЛЬHЫЙ P-H                         0.16521    0.46672   0.354
REGIONCEB.CKЛOH MУPMAHCKOГO MEЛKOBOДЬЯ            -0.07615    0.15513  -0.491
REGIONCEBEPO-KAHИHCKAЯ БAHKA                       0.16332    0.08236   1.983
REGIONKAHИHCKAЯ БAHKA                              0.06291    0.07875   0.799
REGIONKAHИHCKO- KOЛГУEBCKOE MEЛKOBOДЬE(CEB.CKЛOH)  0.21310    0.08519   2.501
REGIONKAHИHCKO-KOЛГУEBCKOE MEЛKOBOДЬE              0.18223    0.07757   2.349
REGIONMУPMAHCKOE MEЛKOBOДЬE                        0.07737    0.07753   0.998
REGIONЗAП.-ПPИБPEЖHЫЙ P-H                          0.72882    0.46327   1.573
REGIONЗAП.-ЦEHTPAЛЬHЫЙ P-H                         0.13328    0.08416   1.584
REGIONЗAП.CKЛOH ГУCИHOЙ БAHKИ                      0.64238    0.11646   5.516
REGIONЮЖ.CKЛOH ГУCИHOЙ БAHKИ                       0.69320    0.13068   5.305
                                                  Pr(>|t|)    
(Intercept)                                        < 2e-16 ***
YEAR2020                                          7.62e-07 ***
YEAR2021                                          8.85e-07 ***
YEAR2022                                           < 2e-16 ***
YEAR2023                                           < 2e-16 ***
YEAR2024                                           < 2e-16 ***
MONTH10                                           2.08e-08 ***
MONTH11                                           0.000109 ***
CALLUAAK                                          1.30e-06 ***
CALLUAKC                                           < 2e-16 ***
CALLUBEV                                           < 2e-16 ***
CALLUBQQ                                          5.36e-07 ***
CALLUBSR                                          9.37e-08 ***
CALLUBUR                                           < 2e-16 ***
CALLUBYT                                          0.000413 ***
CALLUCFF                                          0.491691    
CALLUCXF                                           < 2e-16 ***
CALLUDII                                          6.93e-16 ***
CALLUDUT                                           < 2e-16 ***
CALLUDWM                                           < 2e-16 ***
CALLUEBK                                          0.053247 .  
CALLUEMO                                          0.035718 *  
CALLUENZ                                          0.444299    
CALLUFIK                                          0.022839 *  
CALLUGXE                                           < 2e-16 ***
CALLUIVO                                          0.000913 ***
REGIONCEB.-ЦEHTPAЛЬHЫЙ P-H                        0.723377    
REGIONCEB.CKЛOH MУPMAHCKOГO MEЛKOBOДЬЯ            0.623533    
REGIONCEBEPO-KAHИHCKAЯ БAHKA                      0.047425 *  
REGIONKAHИHCKAЯ БAHKA                             0.424416    
REGIONKAHИHCKO- KOЛГУEBCKOE MEЛKOBOДЬE(CEB.CKЛOH) 0.012413 *  
REGIONKAHИHCKO-KOЛГУEBCKOE MEЛKOBOДЬE             0.018863 *  
REGIONMУPMAHCKOE MEЛKOBOДЬE                       0.318379    
REGIONЗAП.-ПPИБPEЖHЫЙ P-H                         0.115753    
REGIONЗAП.-ЦEHTPAЛЬHЫЙ P-H                        0.113361    
REGIONЗAП.CKЛOH ГУCИHOЙ БAHKИ                     3.69e-08 ***
REGIONЮЖ.CKЛOH ГУCИHOЙ БAHKИ                      1.19e-07 ***
---
Signif. codes:  0 '***' 0.001 '**' 0.01 '*' 0.05 '.' 0.1 ' ' 1

(Dispersion parameter for Gamma family taken to be 0.4172272)

    Null deviance: 2980.2  on 3890  degrees of freedom
Residual deviance: 1785.6  on 3854  degrees of freedom
AIC: 42851

Number of Fisher Scoring iterations: 11
\end{verbatim}

\begin{Shaded}
\begin{Highlighting}[]
\CommentTok{\# Таблица коэффициентов в форматированном виде}
\NormalTok{broom}\SpecialCharTok{::}\FunctionTok{tidy}\NormalTok{(glm\_gamma\_fit) }\SpecialCharTok{\%\textgreater{}\%}
  \FunctionTok{mutate}\NormalTok{(}\FunctionTok{across}\NormalTok{(estimate}\SpecialCharTok{:}\NormalTok{statistic, }\SpecialCharTok{\textasciitilde{}}\FunctionTok{round}\NormalTok{(.x, }\DecValTok{4}\NormalTok{))) }\SpecialCharTok{\%\textgreater{}\%}
  \FunctionTok{kable}\NormalTok{(}\AttributeTok{caption =} \StringTok{"Коэффициенты GLM модели"}\NormalTok{, }\AttributeTok{align =} \StringTok{"lrrrr"}\NormalTok{)}
\end{Highlighting}
\end{Shaded}

\begin{longtable}[]{@{}
  >{\raggedright\arraybackslash}p{(\linewidth - 8\tabcolsep) * \real{0.5618}}
  >{\raggedleft\arraybackslash}p{(\linewidth - 8\tabcolsep) * \real{0.1011}}
  >{\raggedleft\arraybackslash}p{(\linewidth - 8\tabcolsep) * \real{0.1124}}
  >{\raggedleft\arraybackslash}p{(\linewidth - 8\tabcolsep) * \real{0.1124}}
  >{\raggedleft\arraybackslash}p{(\linewidth - 8\tabcolsep) * \real{0.1124}}@{}}
\caption{Коэффициенты GLM модели}\tabularnewline
\toprule\noalign{}
\begin{minipage}[b]{\linewidth}\raggedright
term
\end{minipage} & \begin{minipage}[b]{\linewidth}\raggedleft
estimate
\end{minipage} & \begin{minipage}[b]{\linewidth}\raggedleft
std.error
\end{minipage} & \begin{minipage}[b]{\linewidth}\raggedleft
statistic
\end{minipage} & \begin{minipage}[b]{\linewidth}\raggedleft
p.value
\end{minipage} \\
\midrule\noalign{}
\endfirsthead
\toprule\noalign{}
\begin{minipage}[b]{\linewidth}\raggedright
term
\end{minipage} & \begin{minipage}[b]{\linewidth}\raggedleft
estimate
\end{minipage} & \begin{minipage}[b]{\linewidth}\raggedleft
std.error
\end{minipage} & \begin{minipage}[b]{\linewidth}\raggedleft
statistic
\end{minipage} & \begin{minipage}[b]{\linewidth}\raggedleft
p.value
\end{minipage} \\
\midrule\noalign{}
\endhead
\bottomrule\noalign{}
\endlastfoot
(Intercept) & 5.5764 & 0.0794 & 70.2164 & 0.0000000 \\
YEAR2020 & -0.2271 & 0.0459 & -4.9530 & 0.0000008 \\
YEAR2021 & -0.2244 & 0.0456 & -4.9236 & 0.0000009 \\
YEAR2022 & -0.6433 & 0.0434 & -14.8059 & 0.0000000 \\
YEAR2023 & -0.7731 & 0.0465 & -16.6372 & 0.0000000 \\
YEAR2024 & -1.1282 & 0.0516 & -21.8785 & 0.0000000 \\
MONTH10 & -0.1372 & 0.0244 & -5.6170 & 0.0000000 \\
MONTH11 & -0.1371 & 0.0354 & -3.8736 & 0.0001090 \\
CALLUAAK & -0.2953 & 0.0609 & -4.8479 & 0.0000013 \\
CALLUAKC & -0.5349 & 0.0638 & -8.3823 & 0.0000000 \\
CALLUBEV & -3.6723 & 0.1219 & -30.1334 & 0.0000000 \\
CALLUBQQ & -0.3543 & 0.0706 & -5.0213 & 0.0000005 \\
CALLUBSR & -0.3352 & 0.0627 & -5.3488 & 0.0000001 \\
CALLUBUR & -0.5825 & 0.0618 & -9.4326 & 0.0000000 \\
CALLUBYT & -0.2152 & 0.0609 & -3.5347 & 0.0004130 \\
CALLUCFF & -0.0676 & 0.0982 & -0.6877 & 0.4916914 \\
CALLUCXF & -2.3430 & 0.1064 & -22.0254 & 0.0000000 \\
CALLUDII & -0.4629 & 0.0571 & -8.1065 & 0.0000000 \\
CALLUDUT & -1.0216 & 0.0755 & -13.5319 & 0.0000000 \\
CALLUDWM & -2.4250 & 0.0948 & -25.5817 & 0.0000000 \\
CALLUEBK & 0.2585 & 0.1337 & 1.9335 & 0.0532474 \\
CALLUEMO & -0.1702 & 0.0810 & -2.1009 & 0.0357185 \\
CALLUENZ & 0.0493 & 0.0644 & 0.7650 & 0.4442989 \\
CALLUFIK & 0.2970 & 0.1304 & 2.2770 & 0.0228387 \\
CALLUGXE & -0.5638 & 0.0627 & -8.9996 & 0.0000000 \\
CALLUIVO & -0.2157 & 0.0650 & -3.3187 & 0.0009126 \\
REGIONCEB.-ЦEHTPAЛЬHЫЙ P-H & 0.1652 & 0.4667 & 0.3540 & 0.7233768 \\
REGIONCEB.CKЛOH MУPMAHCKOГO MEЛKOBOДЬЯ & -0.0762 & 0.1551 & -0.4909 &
0.6235329 \\
REGIONCEBEPO-KAHИHCKAЯ БAHKA & 0.1633 & 0.0824 & 1.9831 & 0.0474248 \\
REGIONKAHИHCKAЯ БAHKA & 0.0629 & 0.0788 & 0.7989 & 0.4244161 \\
REGIONKAHИHCKO- KOЛГУEBCKOE MEЛKOBOДЬE(CEB.CKЛOH) & 0.2131 & 0.0852 &
2.5013 & 0.0124133 \\
REGIONKAHИHCKO-KOЛГУEBCKOE MEЛKOBOДЬE & 0.1822 & 0.0776 & 2.3492 &
0.0188631 \\
REGIONMУPMAHCKOE MEЛKOBOДЬE & 0.0774 & 0.0775 & 0.9979 & 0.3183790 \\
REGIONЗAП.-ПPИБPEЖHЫЙ P-H & 0.7288 & 0.4633 & 1.5732 & 0.1157534 \\
REGIONЗAП.-ЦEHTPAЛЬHЫЙ P-H & 0.1333 & 0.0842 & 1.5836 & 0.1133609 \\
REGIONЗAП.CKЛOH ГУCИHOЙ БAHKИ & 0.6424 & 0.1165 & 5.5160 & 0.0000000 \\
REGIONЮЖ.CKЛOH ГУCИHOЙ БAHKИ & 0.6932 & 0.1307 & 5.3046 & 0.0000001 \\
\end{longtable}

\begin{Shaded}
\begin{Highlighting}[]
\CommentTok{\# Графики диагностики остатков}
\FunctionTok{par}\NormalTok{(}\AttributeTok{mfrow =} \FunctionTok{c}\NormalTok{(}\DecValTok{2}\NormalTok{, }\DecValTok{2}\NormalTok{))}
\FunctionTok{plot}\NormalTok{(glm\_gamma\_fit)  }\CommentTok{\# Стандартные диагностические графики GLM}
\end{Highlighting}
\end{Shaded}

\pandocbounded{\includegraphics[keepaspectratio]{chapter9_files/figure-pdf/unnamed-chunk-1-2.pdf}}

\begin{Shaded}
\begin{Highlighting}[]
\FunctionTok{par}\NormalTok{(}\AttributeTok{mfrow =} \FunctionTok{c}\NormalTok{(}\DecValTok{1}\NormalTok{, }\DecValTok{1}\NormalTok{))}

\CommentTok{\# Диагностика остатков GLM с использованием DHARMa}
\NormalTok{sim\_glm }\OtherTok{\textless{}{-}} \FunctionTok{simulateResiduals}\NormalTok{(glm\_gamma\_fit, }\AttributeTok{n =} \DecValTok{1000}\NormalTok{, }\AttributeTok{refit =} \ConstantTok{FALSE}\NormalTok{)}
\FunctionTok{plot}\NormalTok{(sim\_glm, }\AttributeTok{main =} \StringTok{"GLM"}\NormalTok{)}
\end{Highlighting}
\end{Shaded}

\pandocbounded{\includegraphics[keepaspectratio]{chapter9_files/figure-pdf/unnamed-chunk-1-3.pdf}}

\begin{Shaded}
\begin{Highlighting}[]
\CommentTok{\# Расчет и визуализация индексов}
\NormalTok{idx\_glm }\OtherTok{\textless{}{-}} \FunctionTok{emmeans\_standardized\_index}\NormalTok{(glm\_gamma\_fit) }\SpecialCharTok{\%\textgreater{}\%}
  \FunctionTok{mutate}\NormalTok{(}\AttributeTok{model =} \StringTok{"GLM\_Gamma"}\NormalTok{,}
         \AttributeTok{index\_mean =} \FunctionTok{scale\_to\_index}\NormalTok{(value, }\StringTok{"mean"}\NormalTok{),}
         \AttributeTok{index\_first =} \FunctionTok{scale\_to\_index}\NormalTok{(value, }\StringTok{"first"}\NormalTok{))}

\CommentTok{\# Добавление доверительных интервалов}
\NormalTok{idx\_glm }\OtherTok{\textless{}{-}}\NormalTok{ idx\_glm }\SpecialCharTok{\%\textgreater{}\%}
  \FunctionTok{mutate}\NormalTok{(}
    \AttributeTok{lcl\_index\_mean =} \FunctionTok{scale\_to\_index}\NormalTok{(lcl, }\StringTok{"mean"}\NormalTok{),}
    \AttributeTok{ucl\_index\_mean =} \FunctionTok{scale\_to\_index}\NormalTok{(ucl, }\StringTok{"mean"}\NormalTok{)}
\NormalTok{  )}

\CommentTok{\# Визуализация индексов GLM}

\NormalTok{idx\_glm }\SpecialCharTok{\%\textgreater{}\%}
  \FunctionTok{ggplot}\NormalTok{(}\FunctionTok{aes}\NormalTok{(}\AttributeTok{x =}\NormalTok{ YEAR, }\AttributeTok{y =}\NormalTok{ value, }\AttributeTok{group =} \DecValTok{1}\NormalTok{)) }\SpecialCharTok{+}
  \FunctionTok{geom\_line}\NormalTok{() }\SpecialCharTok{+}
  \FunctionTok{geom\_point}\NormalTok{() }\SpecialCharTok{+}
  \FunctionTok{geom\_ribbon}\NormalTok{(}\FunctionTok{aes}\NormalTok{(}\AttributeTok{ymin =}\NormalTok{ lcl, }\AttributeTok{ymax =}\NormalTok{ ucl), }\AttributeTok{alpha =} \FloatTok{0.3}\NormalTok{) }\SpecialCharTok{+}
  \FunctionTok{geom\_point}\NormalTok{(}\AttributeTok{data =}\NormalTok{ actual\_medians, }
           \FunctionTok{aes}\NormalTok{(}\AttributeTok{x =}\NormalTok{ YEAR, }\AttributeTok{y =}\NormalTok{ median\_cpue), }
           \AttributeTok{shape =} \DecValTok{4}\NormalTok{,  }\CommentTok{\# 4 соответствует крестику (x)}
           \AttributeTok{size =} \DecValTok{3}\NormalTok{, }
           \AttributeTok{color =} \StringTok{"black"}\NormalTok{, }
           \AttributeTok{inherit.aes =} \ConstantTok{FALSE}\NormalTok{)}\SpecialCharTok{+}
\FunctionTok{labs}\NormalTok{(}\AttributeTok{title =} \StringTok{"Индексы CPUE по GLM модели (крестики {-} факт)"}\NormalTok{, }
       \AttributeTok{x =} \StringTok{"Год"}\NormalTok{, }
       \AttributeTok{y =} \StringTok{"Стандартизированный индекс"}\NormalTok{)}
\end{Highlighting}
\end{Shaded}

\begin{verbatim}
Warning in grid.Call(C_textBounds, as.graphicsAnnot(x$label), x$x, x$y, :
неизвестна ширина символа 0xd1 в кодировке CP1251
\end{verbatim}

\begin{verbatim}
Warning in grid.Call(C_textBounds, as.graphicsAnnot(x$label), x$x, x$y, :
неизвестна ширина символа 0xf2 в кодировке CP1251
\end{verbatim}

\begin{verbatim}
Warning in grid.Call(C_textBounds, as.graphicsAnnot(x$label), x$x, x$y, :
неизвестна ширина символа 0xe0 в кодировке CP1251
\end{verbatim}

\begin{verbatim}
Warning in grid.Call(C_textBounds, as.graphicsAnnot(x$label), x$x, x$y, :
неизвестна ширина символа 0xed в кодировке CP1251
\end{verbatim}

\begin{verbatim}
Warning in grid.Call(C_textBounds, as.graphicsAnnot(x$label), x$x, x$y, :
неизвестна ширина символа 0xe4 в кодировке CP1251
\end{verbatim}

\begin{verbatim}
Warning in grid.Call(C_textBounds, as.graphicsAnnot(x$label), x$x, x$y, :
неизвестна ширина символа 0xe0 в кодировке CP1251
\end{verbatim}

\begin{verbatim}
Warning in grid.Call(C_textBounds, as.graphicsAnnot(x$label), x$x, x$y, :
неизвестна ширина символа 0xf0 в кодировке CP1251
\end{verbatim}

\begin{verbatim}
Warning in grid.Call(C_textBounds, as.graphicsAnnot(x$label), x$x, x$y, :
неизвестна ширина символа 0xf2 в кодировке CP1251
\end{verbatim}

\begin{verbatim}
Warning in grid.Call(C_textBounds, as.graphicsAnnot(x$label), x$x, x$y, :
неизвестна ширина символа 0xe8 в кодировке CP1251
\end{verbatim}

\begin{verbatim}
Warning in grid.Call(C_textBounds, as.graphicsAnnot(x$label), x$x, x$y, :
неизвестна ширина символа 0xe7 в кодировке CP1251
\end{verbatim}

\begin{verbatim}
Warning in grid.Call(C_textBounds, as.graphicsAnnot(x$label), x$x, x$y, :
неизвестна ширина символа 0xe8 в кодировке CP1251
\end{verbatim}

\begin{verbatim}
Warning in grid.Call(C_textBounds, as.graphicsAnnot(x$label), x$x, x$y, :
неизвестна ширина символа 0xf0 в кодировке CP1251
\end{verbatim}

\begin{verbatim}
Warning in grid.Call(C_textBounds, as.graphicsAnnot(x$label), x$x, x$y, :
неизвестна ширина символа 0xee в кодировке CP1251
\end{verbatim}

\begin{verbatim}
Warning in grid.Call(C_textBounds, as.graphicsAnnot(x$label), x$x, x$y, :
неизвестна ширина символа 0xe2 в кодировке CP1251
\end{verbatim}

\begin{verbatim}
Warning in grid.Call(C_textBounds, as.graphicsAnnot(x$label), x$x, x$y, :
неизвестна ширина символа 0xe0 в кодировке CP1251
\end{verbatim}

\begin{verbatim}
Warning in grid.Call(C_textBounds, as.graphicsAnnot(x$label), x$x, x$y, :
неизвестна ширина символа 0xed в кодировке CP1251
Warning in grid.Call(C_textBounds, as.graphicsAnnot(x$label), x$x, x$y, :
неизвестна ширина символа 0xed в кодировке CP1251
\end{verbatim}

\begin{verbatim}
Warning in grid.Call(C_textBounds, as.graphicsAnnot(x$label), x$x, x$y, :
неизвестна ширина символа 0xfb в кодировке CP1251
\end{verbatim}

\begin{verbatim}
Warning in grid.Call(C_textBounds, as.graphicsAnnot(x$label), x$x, x$y, :
неизвестна ширина символа 0xe9 в кодировке CP1251
\end{verbatim}

\begin{verbatim}
Warning in grid.Call(C_textBounds, as.graphicsAnnot(x$label), x$x, x$y, :
неизвестна ширина символа 0xe8 в кодировке CP1251
\end{verbatim}

\begin{verbatim}
Warning in grid.Call(C_textBounds, as.graphicsAnnot(x$label), x$x, x$y, :
неизвестна ширина символа 0xed в кодировке CP1251
\end{verbatim}

\begin{verbatim}
Warning in grid.Call(C_textBounds, as.graphicsAnnot(x$label), x$x, x$y, :
неизвестна ширина символа 0xe4 в кодировке CP1251
\end{verbatim}

\begin{verbatim}
Warning in grid.Call(C_textBounds, as.graphicsAnnot(x$label), x$x, x$y, :
неизвестна ширина символа 0xe5 в кодировке CP1251
\end{verbatim}

\begin{verbatim}
Warning in grid.Call(C_textBounds, as.graphicsAnnot(x$label), x$x, x$y, :
неизвестна ширина символа 0xea в кодировке CP1251
\end{verbatim}

\begin{verbatim}
Warning in grid.Call(C_textBounds, as.graphicsAnnot(x$label), x$x, x$y, :
неизвестна ширина символа 0xf1 в кодировке CP1251
\end{verbatim}

\begin{verbatim}
Warning in grid.Call(C_textBounds, as.graphicsAnnot(x$label), x$x, x$y, :
неизвестна ширина символа 0xc8 в кодировке CP1251
\end{verbatim}

\begin{verbatim}
Warning in grid.Call(C_textBounds, as.graphicsAnnot(x$label), x$x, x$y, :
неизвестна ширина символа 0xed в кодировке CP1251
\end{verbatim}

\begin{verbatim}
Warning in grid.Call(C_textBounds, as.graphicsAnnot(x$label), x$x, x$y, :
неизвестна ширина символа 0xe4 в кодировке CP1251
\end{verbatim}

\begin{verbatim}
Warning in grid.Call(C_textBounds, as.graphicsAnnot(x$label), x$x, x$y, :
неизвестна ширина символа 0xe5 в кодировке CP1251
\end{verbatim}

\begin{verbatim}
Warning in grid.Call(C_textBounds, as.graphicsAnnot(x$label), x$x, x$y, :
неизвестна ширина символа 0xea в кодировке CP1251
\end{verbatim}

\begin{verbatim}
Warning in grid.Call(C_textBounds, as.graphicsAnnot(x$label), x$x, x$y, :
неизвестна ширина символа 0xf1 в кодировке CP1251
\end{verbatim}

\begin{verbatim}
Warning in grid.Call(C_textBounds, as.graphicsAnnot(x$label), x$x, x$y, :
неизвестна ширина символа 0xfb в кодировке CP1251
\end{verbatim}

\begin{verbatim}
Warning in grid.Call(C_textBounds, as.graphicsAnnot(x$label), x$x, x$y, :
неизвестна ширина символа 0xef в кодировке CP1251
\end{verbatim}

\begin{verbatim}
Warning in grid.Call(C_textBounds, as.graphicsAnnot(x$label), x$x, x$y, :
неизвестна ширина символа 0xee в кодировке CP1251
\end{verbatim}

\begin{verbatim}
Warning in grid.Call(C_textBounds, as.graphicsAnnot(x$label), x$x, x$y, :
неизвестна ширина символа 0xec в кодировке CP1251
\end{verbatim}

\begin{verbatim}
Warning in grid.Call(C_textBounds, as.graphicsAnnot(x$label), x$x, x$y, :
неизвестна ширина символа 0xee в кодировке CP1251
\end{verbatim}

\begin{verbatim}
Warning in grid.Call(C_textBounds, as.graphicsAnnot(x$label), x$x, x$y, :
неизвестна ширина символа 0xe4 в кодировке CP1251
\end{verbatim}

\begin{verbatim}
Warning in grid.Call(C_textBounds, as.graphicsAnnot(x$label), x$x, x$y, :
неизвестна ширина символа 0xe5 в кодировке CP1251
\end{verbatim}

\begin{verbatim}
Warning in grid.Call(C_textBounds, as.graphicsAnnot(x$label), x$x, x$y, :
неизвестна ширина символа 0xeb в кодировке CP1251
\end{verbatim}

\begin{verbatim}
Warning in grid.Call(C_textBounds, as.graphicsAnnot(x$label), x$x, x$y, :
неизвестна ширина символа 0xe8 в кодировке CP1251
\end{verbatim}

\begin{verbatim}
Warning in grid.Call(C_textBounds, as.graphicsAnnot(x$label), x$x, x$y, :
неизвестна ширина символа 0xea в кодировке CP1251
\end{verbatim}

\begin{verbatim}
Warning in grid.Call(C_textBounds, as.graphicsAnnot(x$label), x$x, x$y, :
неизвестна ширина символа 0xf0 в кодировке CP1251
\end{verbatim}

\begin{verbatim}
Warning in grid.Call(C_textBounds, as.graphicsAnnot(x$label), x$x, x$y, :
неизвестна ширина символа 0xe5 в кодировке CP1251
\end{verbatim}

\begin{verbatim}
Warning in grid.Call(C_textBounds, as.graphicsAnnot(x$label), x$x, x$y, :
неизвестна ширина символа 0xf1 в кодировке CP1251
\end{verbatim}

\begin{verbatim}
Warning in grid.Call(C_textBounds, as.graphicsAnnot(x$label), x$x, x$y, :
неизвестна ширина символа 0xf2 в кодировке CP1251
\end{verbatim}

\begin{verbatim}
Warning in grid.Call(C_textBounds, as.graphicsAnnot(x$label), x$x, x$y, :
неизвестна ширина символа 0xe8 в кодировке CP1251
\end{verbatim}

\begin{verbatim}
Warning in grid.Call(C_textBounds, as.graphicsAnnot(x$label), x$x, x$y, :
неизвестна ширина символа 0xea в кодировке CP1251
\end{verbatim}

\begin{verbatim}
Warning in grid.Call(C_textBounds, as.graphicsAnnot(x$label), x$x, x$y, :
неизвестна ширина символа 0xe8 в кодировке CP1251
\end{verbatim}

\begin{verbatim}
Warning in grid.Call(C_textBounds, as.graphicsAnnot(x$label), x$x, x$y, :
неизвестна ширина символа 0xf4 в кодировке CP1251
\end{verbatim}

\begin{verbatim}
Warning in grid.Call(C_textBounds, as.graphicsAnnot(x$label), x$x, x$y, :
неизвестна ширина символа 0xe0 в кодировке CP1251
\end{verbatim}

\begin{verbatim}
Warning in grid.Call(C_textBounds, as.graphicsAnnot(x$label), x$x, x$y, :
неизвестна ширина символа 0xea в кодировке CP1251
\end{verbatim}

\begin{verbatim}
Warning in grid.Call(C_textBounds, as.graphicsAnnot(x$label), x$x, x$y, :
неизвестна ширина символа 0xf2 в кодировке CP1251
\end{verbatim}

\begin{verbatim}
Warning in grid.Call(C_textBounds, as.graphicsAnnot(x$label), x$x, x$y, :
неизвестна ширина символа 0xc3 в кодировке CP1251
\end{verbatim}

\begin{verbatim}
Warning in grid.Call(C_textBounds, as.graphicsAnnot(x$label), x$x, x$y, :
неизвестна ширина символа 0xee в кодировке CP1251
\end{verbatim}

\begin{verbatim}
Warning in grid.Call(C_textBounds, as.graphicsAnnot(x$label), x$x, x$y, :
неизвестна ширина символа 0xe4 в кодировке CP1251
\end{verbatim}

\begin{verbatim}
Warning in grid.Call.graphics(C_text, as.graphicsAnnot(x$label), x$x, x$y, :
неизвестна ширина символа 0xc3 в кодировке CP1251
\end{verbatim}

\begin{verbatim}
Warning in grid.Call.graphics(C_text, as.graphicsAnnot(x$label), x$x, x$y, :
неизвестна ширина символа 0xee в кодировке CP1251
\end{verbatim}

\begin{verbatim}
Warning in grid.Call.graphics(C_text, as.graphicsAnnot(x$label), x$x, x$y, :
неизвестна ширина символа 0xe4 в кодировке CP1251
\end{verbatim}

\begin{verbatim}
Warning in grid.Call.graphics(C_text, as.graphicsAnnot(x$label), x$x, x$y, :
неизвестна ширина символа 0xd1 в кодировке CP1251
\end{verbatim}

\begin{verbatim}
Warning in grid.Call.graphics(C_text, as.graphicsAnnot(x$label), x$x, x$y, :
неизвестна ширина символа 0xf2 в кодировке CP1251
\end{verbatim}

\begin{verbatim}
Warning in grid.Call.graphics(C_text, as.graphicsAnnot(x$label), x$x, x$y, :
неизвестна ширина символа 0xe0 в кодировке CP1251
\end{verbatim}

\begin{verbatim}
Warning in grid.Call.graphics(C_text, as.graphicsAnnot(x$label), x$x, x$y, :
неизвестна ширина символа 0xed в кодировке CP1251
\end{verbatim}

\begin{verbatim}
Warning in grid.Call.graphics(C_text, as.graphicsAnnot(x$label), x$x, x$y, :
неизвестна ширина символа 0xe4 в кодировке CP1251
\end{verbatim}

\begin{verbatim}
Warning in grid.Call.graphics(C_text, as.graphicsAnnot(x$label), x$x, x$y, :
неизвестна ширина символа 0xe0 в кодировке CP1251
\end{verbatim}

\begin{verbatim}
Warning in grid.Call.graphics(C_text, as.graphicsAnnot(x$label), x$x, x$y, :
неизвестна ширина символа 0xf0 в кодировке CP1251
\end{verbatim}

\begin{verbatim}
Warning in grid.Call.graphics(C_text, as.graphicsAnnot(x$label), x$x, x$y, :
неизвестна ширина символа 0xf2 в кодировке CP1251
\end{verbatim}

\begin{verbatim}
Warning in grid.Call.graphics(C_text, as.graphicsAnnot(x$label), x$x, x$y, :
неизвестна ширина символа 0xe8 в кодировке CP1251
\end{verbatim}

\begin{verbatim}
Warning in grid.Call.graphics(C_text, as.graphicsAnnot(x$label), x$x, x$y, :
неизвестна ширина символа 0xe7 в кодировке CP1251
\end{verbatim}

\begin{verbatim}
Warning in grid.Call.graphics(C_text, as.graphicsAnnot(x$label), x$x, x$y, :
неизвестна ширина символа 0xe8 в кодировке CP1251
\end{verbatim}

\begin{verbatim}
Warning in grid.Call.graphics(C_text, as.graphicsAnnot(x$label), x$x, x$y, :
неизвестна ширина символа 0xf0 в кодировке CP1251
\end{verbatim}

\begin{verbatim}
Warning in grid.Call.graphics(C_text, as.graphicsAnnot(x$label), x$x, x$y, :
неизвестна ширина символа 0xee в кодировке CP1251
\end{verbatim}

\begin{verbatim}
Warning in grid.Call.graphics(C_text, as.graphicsAnnot(x$label), x$x, x$y, :
неизвестна ширина символа 0xe2 в кодировке CP1251
\end{verbatim}

\begin{verbatim}
Warning in grid.Call.graphics(C_text, as.graphicsAnnot(x$label), x$x, x$y, :
неизвестна ширина символа 0xe0 в кодировке CP1251
\end{verbatim}

\begin{verbatim}
Warning in grid.Call.graphics(C_text, as.graphicsAnnot(x$label), x$x, x$y, :
неизвестна ширина символа 0xed в кодировке CP1251
Warning in grid.Call.graphics(C_text, as.graphicsAnnot(x$label), x$x, x$y, :
неизвестна ширина символа 0xed в кодировке CP1251
\end{verbatim}

\begin{verbatim}
Warning in grid.Call.graphics(C_text, as.graphicsAnnot(x$label), x$x, x$y, :
неизвестна ширина символа 0xfb в кодировке CP1251
\end{verbatim}

\begin{verbatim}
Warning in grid.Call.graphics(C_text, as.graphicsAnnot(x$label), x$x, x$y, :
неизвестна ширина символа 0xe9 в кодировке CP1251
\end{verbatim}

\begin{verbatim}
Warning in grid.Call.graphics(C_text, as.graphicsAnnot(x$label), x$x, x$y, :
неизвестна ширина символа 0xe8 в кодировке CP1251
\end{verbatim}

\begin{verbatim}
Warning in grid.Call.graphics(C_text, as.graphicsAnnot(x$label), x$x, x$y, :
неизвестна ширина символа 0xed в кодировке CP1251
\end{verbatim}

\begin{verbatim}
Warning in grid.Call.graphics(C_text, as.graphicsAnnot(x$label), x$x, x$y, :
неизвестна ширина символа 0xe4 в кодировке CP1251
\end{verbatim}

\begin{verbatim}
Warning in grid.Call.graphics(C_text, as.graphicsAnnot(x$label), x$x, x$y, :
неизвестна ширина символа 0xe5 в кодировке CP1251
\end{verbatim}

\begin{verbatim}
Warning in grid.Call.graphics(C_text, as.graphicsAnnot(x$label), x$x, x$y, :
неизвестна ширина символа 0xea в кодировке CP1251
\end{verbatim}

\begin{verbatim}
Warning in grid.Call.graphics(C_text, as.graphicsAnnot(x$label), x$x, x$y, :
неизвестна ширина символа 0xf1 в кодировке CP1251
\end{verbatim}

\begin{verbatim}
Warning in grid.Call.graphics(C_text, as.graphicsAnnot(x$label), x$x, x$y, :
неизвестна ширина символа 0xc8 в кодировке CP1251
\end{verbatim}

\begin{verbatim}
Warning in grid.Call.graphics(C_text, as.graphicsAnnot(x$label), x$x, x$y, :
неизвестна ширина символа 0xed в кодировке CP1251
\end{verbatim}

\begin{verbatim}
Warning in grid.Call.graphics(C_text, as.graphicsAnnot(x$label), x$x, x$y, :
неизвестна ширина символа 0xe4 в кодировке CP1251
\end{verbatim}

\begin{verbatim}
Warning in grid.Call.graphics(C_text, as.graphicsAnnot(x$label), x$x, x$y, :
неизвестна ширина символа 0xe5 в кодировке CP1251
\end{verbatim}

\begin{verbatim}
Warning in grid.Call.graphics(C_text, as.graphicsAnnot(x$label), x$x, x$y, :
неизвестна ширина символа 0xea в кодировке CP1251
\end{verbatim}

\begin{verbatim}
Warning in grid.Call.graphics(C_text, as.graphicsAnnot(x$label), x$x, x$y, :
неизвестна ширина символа 0xf1 в кодировке CP1251
\end{verbatim}

\begin{verbatim}
Warning in grid.Call.graphics(C_text, as.graphicsAnnot(x$label), x$x, x$y, :
неизвестна ширина символа 0xfb в кодировке CP1251
\end{verbatim}

\begin{verbatim}
Warning in grid.Call.graphics(C_text, as.graphicsAnnot(x$label), x$x, x$y, :
неизвестна ширина символа 0xef в кодировке CP1251
\end{verbatim}

\begin{verbatim}
Warning in grid.Call.graphics(C_text, as.graphicsAnnot(x$label), x$x, x$y, :
неизвестна ширина символа 0xee в кодировке CP1251
\end{verbatim}

\begin{verbatim}
Warning in grid.Call.graphics(C_text, as.graphicsAnnot(x$label), x$x, x$y, :
неизвестна ширина символа 0xec в кодировке CP1251
\end{verbatim}

\begin{verbatim}
Warning in grid.Call.graphics(C_text, as.graphicsAnnot(x$label), x$x, x$y, :
неизвестна ширина символа 0xee в кодировке CP1251
\end{verbatim}

\begin{verbatim}
Warning in grid.Call.graphics(C_text, as.graphicsAnnot(x$label), x$x, x$y, :
неизвестна ширина символа 0xe4 в кодировке CP1251
\end{verbatim}

\begin{verbatim}
Warning in grid.Call.graphics(C_text, as.graphicsAnnot(x$label), x$x, x$y, :
неизвестна ширина символа 0xe5 в кодировке CP1251
\end{verbatim}

\begin{verbatim}
Warning in grid.Call.graphics(C_text, as.graphicsAnnot(x$label), x$x, x$y, :
неизвестна ширина символа 0xeb в кодировке CP1251
\end{verbatim}

\begin{verbatim}
Warning in grid.Call.graphics(C_text, as.graphicsAnnot(x$label), x$x, x$y, :
неизвестна ширина символа 0xe8 в кодировке CP1251
\end{verbatim}

\begin{verbatim}
Warning in grid.Call.graphics(C_text, as.graphicsAnnot(x$label), x$x, x$y, :
неизвестна ширина символа 0xea в кодировке CP1251
\end{verbatim}

\begin{verbatim}
Warning in grid.Call.graphics(C_text, as.graphicsAnnot(x$label), x$x, x$y, :
неизвестна ширина символа 0xf0 в кодировке CP1251
\end{verbatim}

\begin{verbatim}
Warning in grid.Call.graphics(C_text, as.graphicsAnnot(x$label), x$x, x$y, :
неизвестна ширина символа 0xe5 в кодировке CP1251
\end{verbatim}

\begin{verbatim}
Warning in grid.Call.graphics(C_text, as.graphicsAnnot(x$label), x$x, x$y, :
неизвестна ширина символа 0xf1 в кодировке CP1251
\end{verbatim}

\begin{verbatim}
Warning in grid.Call.graphics(C_text, as.graphicsAnnot(x$label), x$x, x$y, :
неизвестна ширина символа 0xf2 в кодировке CP1251
\end{verbatim}

\begin{verbatim}
Warning in grid.Call.graphics(C_text, as.graphicsAnnot(x$label), x$x, x$y, :
неизвестна ширина символа 0xe8 в кодировке CP1251
\end{verbatim}

\begin{verbatim}
Warning in grid.Call.graphics(C_text, as.graphicsAnnot(x$label), x$x, x$y, :
неизвестна ширина символа 0xea в кодировке CP1251
\end{verbatim}

\begin{verbatim}
Warning in grid.Call.graphics(C_text, as.graphicsAnnot(x$label), x$x, x$y, :
неизвестна ширина символа 0xe8 в кодировке CP1251
\end{verbatim}

\begin{verbatim}
Warning in grid.Call.graphics(C_text, as.graphicsAnnot(x$label), x$x, x$y, :
неизвестна ширина символа 0xf4 в кодировке CP1251
\end{verbatim}

\begin{verbatim}
Warning in grid.Call.graphics(C_text, as.graphicsAnnot(x$label), x$x, x$y, :
неизвестна ширина символа 0xe0 в кодировке CP1251
\end{verbatim}

\begin{verbatim}
Warning in grid.Call.graphics(C_text, as.graphicsAnnot(x$label), x$x, x$y, :
неизвестна ширина символа 0xea в кодировке CP1251
\end{verbatim}

\begin{verbatim}
Warning in grid.Call.graphics(C_text, as.graphicsAnnot(x$label), x$x, x$y, :
неизвестна ширина символа 0xf2 в кодировке CP1251
\end{verbatim}

\pandocbounded{\includegraphics[keepaspectratio]{chapter9_files/figure-pdf/unnamed-chunk-1-4.pdf}}

\begin{Shaded}
\begin{Highlighting}[]
\CommentTok{\# ==============================================================================}
\CommentTok{\# БЛОК 5: МОДЕЛИРОВАНИЕ GAM}
\CommentTok{\# ==============================================================================}

\CommentTok{\# Подбор обобщенной аддитивной модели}
\NormalTok{gam\_fit }\OtherTok{\textless{}{-}} \FunctionTok{gam}\NormalTok{(}
\NormalTok{  CPUE\_POS }\SpecialCharTok{\textasciitilde{}}\NormalTok{ YEAR }\SpecialCharTok{+}\NormalTok{ MONTH }\SpecialCharTok{+}\NormalTok{ CALL }\SpecialCharTok{+}\NormalTok{ REGION,  }\CommentTok{\# Линейная формула без сглаживания}
  \AttributeTok{family =} \FunctionTok{Gamma}\NormalTok{(}\AttributeTok{link =} \StringTok{"log"}\NormalTok{),            }\CommentTok{\# Аналогичное GLM распределение}
  \AttributeTok{method =} \StringTok{"REML"}\NormalTok{,                         }\CommentTok{\# Метод оптимизации гиперпараметров}
  \AttributeTok{data =}\NormalTok{ DATA}
\NormalTok{)}

\FunctionTok{summary}\NormalTok{(gam\_fit)  }\CommentTok{\# Сводка модели}
\end{Highlighting}
\end{Shaded}

\begin{verbatim}

Family: Gamma 
Link function: log 

Formula:
CPUE_POS ~ YEAR + MONTH + CALL + REGION

Parametric coefficients:
                                                  Estimate Std. Error t value
(Intercept)                                        5.57646    0.07942  70.217
YEAR2020                                          -0.22712    0.04586  -4.953
YEAR2021                                          -0.22438    0.04557  -4.924
YEAR2022                                          -0.64329    0.04345 -14.806
YEAR2023                                          -0.77309    0.04647 -16.637
YEAR2024                                          -1.12812    0.05156 -21.878
MONTH10                                           -0.13725    0.02443  -5.617
MONTH11                                           -0.13714    0.03540  -3.874
CALLUAAK                                          -0.29528    0.06092  -4.847
CALLUAKC                                          -0.53484    0.06381  -8.381
CALLUBEV                                          -3.67226    0.12187 -30.133
CALLUBQQ                                          -0.35427    0.07057  -5.020
CALLUBSR                                          -0.33512    0.06266  -5.348
CALLUBUR                                          -0.58242    0.06175  -9.432
CALLUBYT                                          -0.21516    0.06088  -3.534
CALLUCFF                                          -0.06750    0.09825  -0.687
CALLUCXF                                          -2.34296    0.10638 -22.025
CALLUDII                                          -0.46282    0.05710  -8.106
CALLUDUT                                          -1.02155    0.07550 -13.531
CALLUDWM                                          -2.42497    0.09479 -25.581
CALLUEBK                                           0.25858    0.13371   1.934
CALLUEMO                                          -0.17018    0.08103  -2.100
CALLUENZ                                           0.04934    0.06442   0.766
CALLUFIK                                           0.29706    0.13043   2.278
CALLUGXE                                          -0.56375    0.06265  -8.998
CALLUIVO                                          -0.21565    0.06500  -3.318
REGIONCEB.-ЦEHTPAЛЬHЫЙ P-H                         0.16508    0.46672   0.354
REGIONCEB.CKЛOH MУPMAHCKOГO MEЛKOBOДЬЯ            -0.07627    0.15513  -0.492
REGIONCEBEPO-KAHИHCKAЯ БAHKA                       0.16322    0.08236   1.982
REGIONKAHИHCKAЯ БAHKA                              0.06281    0.07875   0.798
REGIONKAHИHCKO- KOЛГУEBCKOE MEЛKOBOДЬE(CEB.CKЛOH)  0.21299    0.08519   2.500
REGIONKAHИHCKO-KOЛГУEBCKOE MEЛKOBOДЬE              0.18213    0.07757   2.348
REGIONMУPMAHCKOE MEЛKOBOДЬE                        0.07728    0.07753   0.997
REGIONЗAП.-ПPИБPEЖHЫЙ P-H                          0.72877    0.46327   1.573
REGIONЗAП.-ЦEHTPAЛЬHЫЙ P-H                         0.13318    0.08416   1.583
REGIONЗAП.CKЛOH ГУCИHOЙ БAHKИ                      0.64226    0.11646   5.515
REGIONЮЖ.CKЛOH ГУCИHOЙ БAHKИ                       0.69310    0.13068   5.304
                                                  Pr(>|t|)    
(Intercept)                                        < 2e-16 ***
YEAR2020                                          7.62e-07 ***
YEAR2021                                          8.85e-07 ***
YEAR2022                                           < 2e-16 ***
YEAR2023                                           < 2e-16 ***
YEAR2024                                           < 2e-16 ***
MONTH10                                           2.08e-08 ***
MONTH11                                           0.000109 ***
CALLUAAK                                          1.30e-06 ***
CALLUAKC                                           < 2e-16 ***
CALLUBEV                                           < 2e-16 ***
CALLUBQQ                                          5.39e-07 ***
CALLUBSR                                          9.39e-08 ***
CALLUBUR                                           < 2e-16 ***
CALLUBYT                                          0.000414 ***
CALLUCFF                                          0.492100    
CALLUCXF                                           < 2e-16 ***
CALLUDII                                          6.98e-16 ***
CALLUDUT                                           < 2e-16 ***
CALLUDWM                                           < 2e-16 ***
CALLUEBK                                          0.053187 .  
CALLUEMO                                          0.035785 *  
CALLUENZ                                          0.443735    
CALLUFIK                                          0.022810 *  
CALLUGXE                                           < 2e-16 ***
CALLUIVO                                          0.000916 ***
REGIONCEB.-ЦEHTPAЛЬHЫЙ P-H                        0.723578    
REGIONCEB.CKЛOH MУPMAHCKOГO MEЛKOBOДЬЯ            0.623006    
REGIONCEBEPO-KAHИHCKAЯ БAHKA                      0.047563 *  
REGIONKAHИHCKAЯ БAHKA                             0.425185    
REGIONKAHИHCKO- KOЛГУEBCKOE MEЛKOBOДЬE(CEB.CKЛOH) 0.012456 *  
REGIONKAHИHCKO-KOЛГУEBCKOE MEЛKOBOДЬE             0.018931 *  
REGIONMУPMAHCKOE MEЛKOBOДЬE                       0.318981    
REGIONЗAП.-ПPИБPEЖHЫЙ P-H                         0.115776    
REGIONЗAП.-ЦEHTPAЛЬHЫЙ P-H                        0.113615    
REGIONЗAП.CKЛOH ГУCИHOЙ БAHKИ                     3.72e-08 ***
REGIONЮЖ.CKЛOH ГУCИHOЙ БAHKИ                      1.20e-07 ***
---
Signif. codes:  0 '***' 0.001 '**' 0.01 '*' 0.05 '.' 0.1 ' ' 1


R-sq.(adj) =  0.397   Deviance explained = 40.1%
-REML =  21455  Scale est. = 0.41722   n = 3891
\end{verbatim}

\begin{Shaded}
\begin{Highlighting}[]
\CommentTok{\# Проверка адекватности модели (графики остатков)}
\NormalTok{mgcv}\SpecialCharTok{::}\FunctionTok{gam.check}\NormalTok{(gam\_fit)}
\end{Highlighting}
\end{Shaded}

\pandocbounded{\includegraphics[keepaspectratio]{chapter9_files/figure-pdf/unnamed-chunk-1-5.pdf}}

\begin{verbatim}

Method: REML   Optimizer: outer newton
full convergence after 5 iterations.
Gradient range [-0.0003574844,-0.0003574844]
(score 21454.84 & scale 0.4172217).
Hessian positive definite, eigenvalue range [2198.061,2198.061].
Model rank =  37 / 37 
\end{verbatim}

\begin{Shaded}
\begin{Highlighting}[]
\CommentTok{\# Диагностика остатков GAM с использованием DHARMa}
\NormalTok{sim\_gam }\OtherTok{\textless{}{-}} \FunctionTok{simulateResiduals}\NormalTok{(gam\_fit, }\AttributeTok{n =} \DecValTok{1000}\NormalTok{, }\AttributeTok{refit =} \ConstantTok{FALSE}\NormalTok{)}
\end{Highlighting}
\end{Shaded}

\begin{verbatim}
Registered S3 method overwritten by 'mgcViz':
  method from   
  +.gg   ggplot2
\end{verbatim}

\begin{Shaded}
\begin{Highlighting}[]
\FunctionTok{plot}\NormalTok{(sim\_gam, }\AttributeTok{main =} \StringTok{"GAM"}\NormalTok{)}
\end{Highlighting}
\end{Shaded}

\pandocbounded{\includegraphics[keepaspectratio]{chapter9_files/figure-pdf/unnamed-chunk-1-6.pdf}}

\begin{Shaded}
\begin{Highlighting}[]
\CommentTok{\# Расчет индексов}
\NormalTok{idx\_gam }\OtherTok{\textless{}{-}} \FunctionTok{emmeans\_standardized\_index}\NormalTok{(gam\_fit) }\SpecialCharTok{\%\textgreater{}\%}
  \FunctionTok{mutate}\NormalTok{(}\AttributeTok{model =} \StringTok{"GAM"}\NormalTok{,}
         \AttributeTok{index\_mean =} \FunctionTok{scale\_to\_index}\NormalTok{(value, }\StringTok{"mean"}\NormalTok{),}
         \AttributeTok{index\_first =} \FunctionTok{scale\_to\_index}\NormalTok{(value, }\StringTok{"first"}\NormalTok{))}

\CommentTok{\# Доверительные интервалы}
\NormalTok{idx\_gam }\OtherTok{\textless{}{-}}\NormalTok{ idx\_gam }\SpecialCharTok{\%\textgreater{}\%}
  \FunctionTok{mutate}\NormalTok{(}
    \AttributeTok{lcl\_index\_mean =} \FunctionTok{scale\_to\_index}\NormalTok{(lcl, }\StringTok{"mean"}\NormalTok{),}
    \AttributeTok{ucl\_index\_mean =} \FunctionTok{scale\_to\_index}\NormalTok{(ucl, }\StringTok{"mean"}\NormalTok{)}
\NormalTok{  )}


\CommentTok{\# Визуализация}
\NormalTok{idx\_gam }\SpecialCharTok{\%\textgreater{}\%}
  \FunctionTok{ggplot}\NormalTok{(}\FunctionTok{aes}\NormalTok{(}\AttributeTok{x =}\NormalTok{ YEAR, }\AttributeTok{y =}\NormalTok{ value, }\AttributeTok{group =} \DecValTok{1}\NormalTok{)) }\SpecialCharTok{+}
  \FunctionTok{geom\_line}\NormalTok{() }\SpecialCharTok{+}
  \FunctionTok{geom\_point}\NormalTok{() }\SpecialCharTok{+}
  \FunctionTok{geom\_ribbon}\NormalTok{(}\FunctionTok{aes}\NormalTok{(}\AttributeTok{ymin =}\NormalTok{ lcl, }\AttributeTok{ymax =}\NormalTok{ ucl), }\AttributeTok{alpha =} \FloatTok{0.3}\NormalTok{) }\SpecialCharTok{+}
  \FunctionTok{geom\_point}\NormalTok{(}\AttributeTok{data =}\NormalTok{ actual\_medians, }
           \FunctionTok{aes}\NormalTok{(}\AttributeTok{x =}\NormalTok{ YEAR, }\AttributeTok{y =}\NormalTok{ median\_cpue), }
           \AttributeTok{shape =} \DecValTok{4}\NormalTok{,  }\CommentTok{\# 4 соответствует крестику (x)}
           \AttributeTok{size =} \DecValTok{3}\NormalTok{, }
           \AttributeTok{color =} \StringTok{"black"}\NormalTok{, }
           \AttributeTok{inherit.aes =} \ConstantTok{FALSE}\NormalTok{)}\SpecialCharTok{+}
  \FunctionTok{labs}\NormalTok{(}\AttributeTok{title =} \StringTok{"Индексы CPUE по GAM модели (крестики {-} факт"}\NormalTok{, }
       \AttributeTok{x =} \StringTok{"Год"}\NormalTok{, }
       \AttributeTok{y =} \StringTok{"Стандартизированный индекс"}\NormalTok{)}
\end{Highlighting}
\end{Shaded}

\begin{verbatim}
Warning: <ggplot> %+% x was deprecated in ggplot2 4.0.0.
i Please use <ggplot> + x instead.
\end{verbatim}

\begin{verbatim}
Warning in grid.Call(C_textBounds, as.graphicsAnnot(x$label), x$x, x$y, :
неизвестна ширина символа 0xd1 в кодировке CP1251
\end{verbatim}

\begin{verbatim}
Warning in grid.Call(C_textBounds, as.graphicsAnnot(x$label), x$x, x$y, :
неизвестна ширина символа 0xf2 в кодировке CP1251
\end{verbatim}

\begin{verbatim}
Warning in grid.Call(C_textBounds, as.graphicsAnnot(x$label), x$x, x$y, :
неизвестна ширина символа 0xe0 в кодировке CP1251
\end{verbatim}

\begin{verbatim}
Warning in grid.Call(C_textBounds, as.graphicsAnnot(x$label), x$x, x$y, :
неизвестна ширина символа 0xed в кодировке CP1251
\end{verbatim}

\begin{verbatim}
Warning in grid.Call(C_textBounds, as.graphicsAnnot(x$label), x$x, x$y, :
неизвестна ширина символа 0xe4 в кодировке CP1251
\end{verbatim}

\begin{verbatim}
Warning in grid.Call(C_textBounds, as.graphicsAnnot(x$label), x$x, x$y, :
неизвестна ширина символа 0xe0 в кодировке CP1251
\end{verbatim}

\begin{verbatim}
Warning in grid.Call(C_textBounds, as.graphicsAnnot(x$label), x$x, x$y, :
неизвестна ширина символа 0xf0 в кодировке CP1251
\end{verbatim}

\begin{verbatim}
Warning in grid.Call(C_textBounds, as.graphicsAnnot(x$label), x$x, x$y, :
неизвестна ширина символа 0xf2 в кодировке CP1251
\end{verbatim}

\begin{verbatim}
Warning in grid.Call(C_textBounds, as.graphicsAnnot(x$label), x$x, x$y, :
неизвестна ширина символа 0xe8 в кодировке CP1251
\end{verbatim}

\begin{verbatim}
Warning in grid.Call(C_textBounds, as.graphicsAnnot(x$label), x$x, x$y, :
неизвестна ширина символа 0xe7 в кодировке CP1251
\end{verbatim}

\begin{verbatim}
Warning in grid.Call(C_textBounds, as.graphicsAnnot(x$label), x$x, x$y, :
неизвестна ширина символа 0xe8 в кодировке CP1251
\end{verbatim}

\begin{verbatim}
Warning in grid.Call(C_textBounds, as.graphicsAnnot(x$label), x$x, x$y, :
неизвестна ширина символа 0xf0 в кодировке CP1251
\end{verbatim}

\begin{verbatim}
Warning in grid.Call(C_textBounds, as.graphicsAnnot(x$label), x$x, x$y, :
неизвестна ширина символа 0xee в кодировке CP1251
\end{verbatim}

\begin{verbatim}
Warning in grid.Call(C_textBounds, as.graphicsAnnot(x$label), x$x, x$y, :
неизвестна ширина символа 0xe2 в кодировке CP1251
\end{verbatim}

\begin{verbatim}
Warning in grid.Call(C_textBounds, as.graphicsAnnot(x$label), x$x, x$y, :
неизвестна ширина символа 0xe0 в кодировке CP1251
\end{verbatim}

\begin{verbatim}
Warning in grid.Call(C_textBounds, as.graphicsAnnot(x$label), x$x, x$y, :
неизвестна ширина символа 0xed в кодировке CP1251
Warning in grid.Call(C_textBounds, as.graphicsAnnot(x$label), x$x, x$y, :
неизвестна ширина символа 0xed в кодировке CP1251
\end{verbatim}

\begin{verbatim}
Warning in grid.Call(C_textBounds, as.graphicsAnnot(x$label), x$x, x$y, :
неизвестна ширина символа 0xfb в кодировке CP1251
\end{verbatim}

\begin{verbatim}
Warning in grid.Call(C_textBounds, as.graphicsAnnot(x$label), x$x, x$y, :
неизвестна ширина символа 0xe9 в кодировке CP1251
\end{verbatim}

\begin{verbatim}
Warning in grid.Call(C_textBounds, as.graphicsAnnot(x$label), x$x, x$y, :
неизвестна ширина символа 0xe8 в кодировке CP1251
\end{verbatim}

\begin{verbatim}
Warning in grid.Call(C_textBounds, as.graphicsAnnot(x$label), x$x, x$y, :
неизвестна ширина символа 0xed в кодировке CP1251
\end{verbatim}

\begin{verbatim}
Warning in grid.Call(C_textBounds, as.graphicsAnnot(x$label), x$x, x$y, :
неизвестна ширина символа 0xe4 в кодировке CP1251
\end{verbatim}

\begin{verbatim}
Warning in grid.Call(C_textBounds, as.graphicsAnnot(x$label), x$x, x$y, :
неизвестна ширина символа 0xe5 в кодировке CP1251
\end{verbatim}

\begin{verbatim}
Warning in grid.Call(C_textBounds, as.graphicsAnnot(x$label), x$x, x$y, :
неизвестна ширина символа 0xea в кодировке CP1251
\end{verbatim}

\begin{verbatim}
Warning in grid.Call(C_textBounds, as.graphicsAnnot(x$label), x$x, x$y, :
неизвестна ширина символа 0xf1 в кодировке CP1251
\end{verbatim}

\begin{verbatim}
Warning in grid.Call(C_textBounds, as.graphicsAnnot(x$label), x$x, x$y, :
неизвестна ширина символа 0xc8 в кодировке CP1251
\end{verbatim}

\begin{verbatim}
Warning in grid.Call(C_textBounds, as.graphicsAnnot(x$label), x$x, x$y, :
неизвестна ширина символа 0xed в кодировке CP1251
\end{verbatim}

\begin{verbatim}
Warning in grid.Call(C_textBounds, as.graphicsAnnot(x$label), x$x, x$y, :
неизвестна ширина символа 0xe4 в кодировке CP1251
\end{verbatim}

\begin{verbatim}
Warning in grid.Call(C_textBounds, as.graphicsAnnot(x$label), x$x, x$y, :
неизвестна ширина символа 0xe5 в кодировке CP1251
\end{verbatim}

\begin{verbatim}
Warning in grid.Call(C_textBounds, as.graphicsAnnot(x$label), x$x, x$y, :
неизвестна ширина символа 0xea в кодировке CP1251
\end{verbatim}

\begin{verbatim}
Warning in grid.Call(C_textBounds, as.graphicsAnnot(x$label), x$x, x$y, :
неизвестна ширина символа 0xf1 в кодировке CP1251
\end{verbatim}

\begin{verbatim}
Warning in grid.Call(C_textBounds, as.graphicsAnnot(x$label), x$x, x$y, :
неизвестна ширина символа 0xfb в кодировке CP1251
\end{verbatim}

\begin{verbatim}
Warning in grid.Call(C_textBounds, as.graphicsAnnot(x$label), x$x, x$y, :
неизвестна ширина символа 0xef в кодировке CP1251
\end{verbatim}

\begin{verbatim}
Warning in grid.Call(C_textBounds, as.graphicsAnnot(x$label), x$x, x$y, :
неизвестна ширина символа 0xee в кодировке CP1251
\end{verbatim}

\begin{verbatim}
Warning in grid.Call(C_textBounds, as.graphicsAnnot(x$label), x$x, x$y, :
неизвестна ширина символа 0xec в кодировке CP1251
\end{verbatim}

\begin{verbatim}
Warning in grid.Call(C_textBounds, as.graphicsAnnot(x$label), x$x, x$y, :
неизвестна ширина символа 0xee в кодировке CP1251
\end{verbatim}

\begin{verbatim}
Warning in grid.Call(C_textBounds, as.graphicsAnnot(x$label), x$x, x$y, :
неизвестна ширина символа 0xe4 в кодировке CP1251
\end{verbatim}

\begin{verbatim}
Warning in grid.Call(C_textBounds, as.graphicsAnnot(x$label), x$x, x$y, :
неизвестна ширина символа 0xe5 в кодировке CP1251
\end{verbatim}

\begin{verbatim}
Warning in grid.Call(C_textBounds, as.graphicsAnnot(x$label), x$x, x$y, :
неизвестна ширина символа 0xeb в кодировке CP1251
\end{verbatim}

\begin{verbatim}
Warning in grid.Call(C_textBounds, as.graphicsAnnot(x$label), x$x, x$y, :
неизвестна ширина символа 0xe8 в кодировке CP1251
\end{verbatim}

\begin{verbatim}
Warning in grid.Call(C_textBounds, as.graphicsAnnot(x$label), x$x, x$y, :
неизвестна ширина символа 0xea в кодировке CP1251
\end{verbatim}

\begin{verbatim}
Warning in grid.Call(C_textBounds, as.graphicsAnnot(x$label), x$x, x$y, :
неизвестна ширина символа 0xf0 в кодировке CP1251
\end{verbatim}

\begin{verbatim}
Warning in grid.Call(C_textBounds, as.graphicsAnnot(x$label), x$x, x$y, :
неизвестна ширина символа 0xe5 в кодировке CP1251
\end{verbatim}

\begin{verbatim}
Warning in grid.Call(C_textBounds, as.graphicsAnnot(x$label), x$x, x$y, :
неизвестна ширина символа 0xf1 в кодировке CP1251
\end{verbatim}

\begin{verbatim}
Warning in grid.Call(C_textBounds, as.graphicsAnnot(x$label), x$x, x$y, :
неизвестна ширина символа 0xf2 в кодировке CP1251
\end{verbatim}

\begin{verbatim}
Warning in grid.Call(C_textBounds, as.graphicsAnnot(x$label), x$x, x$y, :
неизвестна ширина символа 0xe8 в кодировке CP1251
\end{verbatim}

\begin{verbatim}
Warning in grid.Call(C_textBounds, as.graphicsAnnot(x$label), x$x, x$y, :
неизвестна ширина символа 0xea в кодировке CP1251
\end{verbatim}

\begin{verbatim}
Warning in grid.Call(C_textBounds, as.graphicsAnnot(x$label), x$x, x$y, :
неизвестна ширина символа 0xe8 в кодировке CP1251
\end{verbatim}

\begin{verbatim}
Warning in grid.Call(C_textBounds, as.graphicsAnnot(x$label), x$x, x$y, :
неизвестна ширина символа 0xf4 в кодировке CP1251
\end{verbatim}

\begin{verbatim}
Warning in grid.Call(C_textBounds, as.graphicsAnnot(x$label), x$x, x$y, :
неизвестна ширина символа 0xe0 в кодировке CP1251
\end{verbatim}

\begin{verbatim}
Warning in grid.Call(C_textBounds, as.graphicsAnnot(x$label), x$x, x$y, :
неизвестна ширина символа 0xea в кодировке CP1251
\end{verbatim}

\begin{verbatim}
Warning in grid.Call(C_textBounds, as.graphicsAnnot(x$label), x$x, x$y, :
неизвестна ширина символа 0xf2 в кодировке CP1251
\end{verbatim}

\begin{verbatim}
Warning in grid.Call(C_textBounds, as.graphicsAnnot(x$label), x$x, x$y, :
неизвестна ширина символа 0xc3 в кодировке CP1251
\end{verbatim}

\begin{verbatim}
Warning in grid.Call(C_textBounds, as.graphicsAnnot(x$label), x$x, x$y, :
неизвестна ширина символа 0xee в кодировке CP1251
\end{verbatim}

\begin{verbatim}
Warning in grid.Call(C_textBounds, as.graphicsAnnot(x$label), x$x, x$y, :
неизвестна ширина символа 0xe4 в кодировке CP1251
\end{verbatim}

\begin{verbatim}
Warning in grid.Call.graphics(C_text, as.graphicsAnnot(x$label), x$x, x$y, :
неизвестна ширина символа 0xc3 в кодировке CP1251
\end{verbatim}

\begin{verbatim}
Warning in grid.Call.graphics(C_text, as.graphicsAnnot(x$label), x$x, x$y, :
неизвестна ширина символа 0xee в кодировке CP1251
\end{verbatim}

\begin{verbatim}
Warning in grid.Call.graphics(C_text, as.graphicsAnnot(x$label), x$x, x$y, :
неизвестна ширина символа 0xe4 в кодировке CP1251
\end{verbatim}

\begin{verbatim}
Warning in grid.Call.graphics(C_text, as.graphicsAnnot(x$label), x$x, x$y, :
неизвестна ширина символа 0xd1 в кодировке CP1251
\end{verbatim}

\begin{verbatim}
Warning in grid.Call.graphics(C_text, as.graphicsAnnot(x$label), x$x, x$y, :
неизвестна ширина символа 0xf2 в кодировке CP1251
\end{verbatim}

\begin{verbatim}
Warning in grid.Call.graphics(C_text, as.graphicsAnnot(x$label), x$x, x$y, :
неизвестна ширина символа 0xe0 в кодировке CP1251
\end{verbatim}

\begin{verbatim}
Warning in grid.Call.graphics(C_text, as.graphicsAnnot(x$label), x$x, x$y, :
неизвестна ширина символа 0xed в кодировке CP1251
\end{verbatim}

\begin{verbatim}
Warning in grid.Call.graphics(C_text, as.graphicsAnnot(x$label), x$x, x$y, :
неизвестна ширина символа 0xe4 в кодировке CP1251
\end{verbatim}

\begin{verbatim}
Warning in grid.Call.graphics(C_text, as.graphicsAnnot(x$label), x$x, x$y, :
неизвестна ширина символа 0xe0 в кодировке CP1251
\end{verbatim}

\begin{verbatim}
Warning in grid.Call.graphics(C_text, as.graphicsAnnot(x$label), x$x, x$y, :
неизвестна ширина символа 0xf0 в кодировке CP1251
\end{verbatim}

\begin{verbatim}
Warning in grid.Call.graphics(C_text, as.graphicsAnnot(x$label), x$x, x$y, :
неизвестна ширина символа 0xf2 в кодировке CP1251
\end{verbatim}

\begin{verbatim}
Warning in grid.Call.graphics(C_text, as.graphicsAnnot(x$label), x$x, x$y, :
неизвестна ширина символа 0xe8 в кодировке CP1251
\end{verbatim}

\begin{verbatim}
Warning in grid.Call.graphics(C_text, as.graphicsAnnot(x$label), x$x, x$y, :
неизвестна ширина символа 0xe7 в кодировке CP1251
\end{verbatim}

\begin{verbatim}
Warning in grid.Call.graphics(C_text, as.graphicsAnnot(x$label), x$x, x$y, :
неизвестна ширина символа 0xe8 в кодировке CP1251
\end{verbatim}

\begin{verbatim}
Warning in grid.Call.graphics(C_text, as.graphicsAnnot(x$label), x$x, x$y, :
неизвестна ширина символа 0xf0 в кодировке CP1251
\end{verbatim}

\begin{verbatim}
Warning in grid.Call.graphics(C_text, as.graphicsAnnot(x$label), x$x, x$y, :
неизвестна ширина символа 0xee в кодировке CP1251
\end{verbatim}

\begin{verbatim}
Warning in grid.Call.graphics(C_text, as.graphicsAnnot(x$label), x$x, x$y, :
неизвестна ширина символа 0xe2 в кодировке CP1251
\end{verbatim}

\begin{verbatim}
Warning in grid.Call.graphics(C_text, as.graphicsAnnot(x$label), x$x, x$y, :
неизвестна ширина символа 0xe0 в кодировке CP1251
\end{verbatim}

\begin{verbatim}
Warning in grid.Call.graphics(C_text, as.graphicsAnnot(x$label), x$x, x$y, :
неизвестна ширина символа 0xed в кодировке CP1251
Warning in grid.Call.graphics(C_text, as.graphicsAnnot(x$label), x$x, x$y, :
неизвестна ширина символа 0xed в кодировке CP1251
\end{verbatim}

\begin{verbatim}
Warning in grid.Call.graphics(C_text, as.graphicsAnnot(x$label), x$x, x$y, :
неизвестна ширина символа 0xfb в кодировке CP1251
\end{verbatim}

\begin{verbatim}
Warning in grid.Call.graphics(C_text, as.graphicsAnnot(x$label), x$x, x$y, :
неизвестна ширина символа 0xe9 в кодировке CP1251
\end{verbatim}

\begin{verbatim}
Warning in grid.Call.graphics(C_text, as.graphicsAnnot(x$label), x$x, x$y, :
неизвестна ширина символа 0xe8 в кодировке CP1251
\end{verbatim}

\begin{verbatim}
Warning in grid.Call.graphics(C_text, as.graphicsAnnot(x$label), x$x, x$y, :
неизвестна ширина символа 0xed в кодировке CP1251
\end{verbatim}

\begin{verbatim}
Warning in grid.Call.graphics(C_text, as.graphicsAnnot(x$label), x$x, x$y, :
неизвестна ширина символа 0xe4 в кодировке CP1251
\end{verbatim}

\begin{verbatim}
Warning in grid.Call.graphics(C_text, as.graphicsAnnot(x$label), x$x, x$y, :
неизвестна ширина символа 0xe5 в кодировке CP1251
\end{verbatim}

\begin{verbatim}
Warning in grid.Call.graphics(C_text, as.graphicsAnnot(x$label), x$x, x$y, :
неизвестна ширина символа 0xea в кодировке CP1251
\end{verbatim}

\begin{verbatim}
Warning in grid.Call.graphics(C_text, as.graphicsAnnot(x$label), x$x, x$y, :
неизвестна ширина символа 0xf1 в кодировке CP1251
\end{verbatim}

\begin{verbatim}
Warning in grid.Call.graphics(C_text, as.graphicsAnnot(x$label), x$x, x$y, :
неизвестна ширина символа 0xc8 в кодировке CP1251
\end{verbatim}

\begin{verbatim}
Warning in grid.Call.graphics(C_text, as.graphicsAnnot(x$label), x$x, x$y, :
неизвестна ширина символа 0xed в кодировке CP1251
\end{verbatim}

\begin{verbatim}
Warning in grid.Call.graphics(C_text, as.graphicsAnnot(x$label), x$x, x$y, :
неизвестна ширина символа 0xe4 в кодировке CP1251
\end{verbatim}

\begin{verbatim}
Warning in grid.Call.graphics(C_text, as.graphicsAnnot(x$label), x$x, x$y, :
неизвестна ширина символа 0xe5 в кодировке CP1251
\end{verbatim}

\begin{verbatim}
Warning in grid.Call.graphics(C_text, as.graphicsAnnot(x$label), x$x, x$y, :
неизвестна ширина символа 0xea в кодировке CP1251
\end{verbatim}

\begin{verbatim}
Warning in grid.Call.graphics(C_text, as.graphicsAnnot(x$label), x$x, x$y, :
неизвестна ширина символа 0xf1 в кодировке CP1251
\end{verbatim}

\begin{verbatim}
Warning in grid.Call.graphics(C_text, as.graphicsAnnot(x$label), x$x, x$y, :
неизвестна ширина символа 0xfb в кодировке CP1251
\end{verbatim}

\begin{verbatim}
Warning in grid.Call.graphics(C_text, as.graphicsAnnot(x$label), x$x, x$y, :
неизвестна ширина символа 0xef в кодировке CP1251
\end{verbatim}

\begin{verbatim}
Warning in grid.Call.graphics(C_text, as.graphicsAnnot(x$label), x$x, x$y, :
неизвестна ширина символа 0xee в кодировке CP1251
\end{verbatim}

\begin{verbatim}
Warning in grid.Call.graphics(C_text, as.graphicsAnnot(x$label), x$x, x$y, :
неизвестна ширина символа 0xec в кодировке CP1251
\end{verbatim}

\begin{verbatim}
Warning in grid.Call.graphics(C_text, as.graphicsAnnot(x$label), x$x, x$y, :
неизвестна ширина символа 0xee в кодировке CP1251
\end{verbatim}

\begin{verbatim}
Warning in grid.Call.graphics(C_text, as.graphicsAnnot(x$label), x$x, x$y, :
неизвестна ширина символа 0xe4 в кодировке CP1251
\end{verbatim}

\begin{verbatim}
Warning in grid.Call.graphics(C_text, as.graphicsAnnot(x$label), x$x, x$y, :
неизвестна ширина символа 0xe5 в кодировке CP1251
\end{verbatim}

\begin{verbatim}
Warning in grid.Call.graphics(C_text, as.graphicsAnnot(x$label), x$x, x$y, :
неизвестна ширина символа 0xeb в кодировке CP1251
\end{verbatim}

\begin{verbatim}
Warning in grid.Call.graphics(C_text, as.graphicsAnnot(x$label), x$x, x$y, :
неизвестна ширина символа 0xe8 в кодировке CP1251
\end{verbatim}

\begin{verbatim}
Warning in grid.Call.graphics(C_text, as.graphicsAnnot(x$label), x$x, x$y, :
неизвестна ширина символа 0xea в кодировке CP1251
\end{verbatim}

\begin{verbatim}
Warning in grid.Call.graphics(C_text, as.graphicsAnnot(x$label), x$x, x$y, :
неизвестна ширина символа 0xf0 в кодировке CP1251
\end{verbatim}

\begin{verbatim}
Warning in grid.Call.graphics(C_text, as.graphicsAnnot(x$label), x$x, x$y, :
неизвестна ширина символа 0xe5 в кодировке CP1251
\end{verbatim}

\begin{verbatim}
Warning in grid.Call.graphics(C_text, as.graphicsAnnot(x$label), x$x, x$y, :
неизвестна ширина символа 0xf1 в кодировке CP1251
\end{verbatim}

\begin{verbatim}
Warning in grid.Call.graphics(C_text, as.graphicsAnnot(x$label), x$x, x$y, :
неизвестна ширина символа 0xf2 в кодировке CP1251
\end{verbatim}

\begin{verbatim}
Warning in grid.Call.graphics(C_text, as.graphicsAnnot(x$label), x$x, x$y, :
неизвестна ширина символа 0xe8 в кодировке CP1251
\end{verbatim}

\begin{verbatim}
Warning in grid.Call.graphics(C_text, as.graphicsAnnot(x$label), x$x, x$y, :
неизвестна ширина символа 0xea в кодировке CP1251
\end{verbatim}

\begin{verbatim}
Warning in grid.Call.graphics(C_text, as.graphicsAnnot(x$label), x$x, x$y, :
неизвестна ширина символа 0xe8 в кодировке CP1251
\end{verbatim}

\begin{verbatim}
Warning in grid.Call.graphics(C_text, as.graphicsAnnot(x$label), x$x, x$y, :
неизвестна ширина символа 0xf4 в кодировке CP1251
\end{verbatim}

\begin{verbatim}
Warning in grid.Call.graphics(C_text, as.graphicsAnnot(x$label), x$x, x$y, :
неизвестна ширина символа 0xe0 в кодировке CP1251
\end{verbatim}

\begin{verbatim}
Warning in grid.Call.graphics(C_text, as.graphicsAnnot(x$label), x$x, x$y, :
неизвестна ширина символа 0xea в кодировке CP1251
\end{verbatim}

\begin{verbatim}
Warning in grid.Call.graphics(C_text, as.graphicsAnnot(x$label), x$x, x$y, :
неизвестна ширина символа 0xf2 в кодировке CP1251
\end{verbatim}

\pandocbounded{\includegraphics[keepaspectratio]{chapter9_files/figure-pdf/unnamed-chunk-1-7.pdf}}

\begin{Shaded}
\begin{Highlighting}[]
\CommentTok{\# ==============================================================================}
\CommentTok{\# БЛОК 6: МОДЕЛИРОВАНИЕ GAMM (СМЕШАННАЯ МОДЕЛЬ)}
\CommentTok{\# ==============================================================================}

\CommentTok{\# Подбор модели со смешанными эффектами}
\NormalTok{gamm\_fit }\OtherTok{\textless{}{-}}\NormalTok{ gamm4}\SpecialCharTok{::}\FunctionTok{gamm4}\NormalTok{(}
  \AttributeTok{formula =}\NormalTok{ CPUE\_POS }\SpecialCharTok{\textasciitilde{}}\NormalTok{ YEAR }\SpecialCharTok{+}\NormalTok{ MONTH }\SpecialCharTok{+}\NormalTok{ REGION,  }\CommentTok{\# Фиксированные эффекты}
  \AttributeTok{random =} \SpecialCharTok{\textasciitilde{}}\NormalTok{ (}\DecValTok{1} \SpecialCharTok{|}\NormalTok{ CALL),                       }\CommentTok{\# Случайный эффект для судна}
  \AttributeTok{family =} \FunctionTok{Gamma}\NormalTok{(}\AttributeTok{link =} \StringTok{"log"}\NormalTok{),               }\CommentTok{\# Распределение}
  \AttributeTok{data =}\NormalTok{ DATA}
\NormalTok{)}

\CommentTok{\# 1. График остатков от предсказанных значений}
\FunctionTok{plot}\NormalTok{(}\FunctionTok{fitted}\NormalTok{(gamm\_fit}\SpecialCharTok{$}\NormalTok{gam), }\FunctionTok{residuals}\NormalTok{(gamm\_fit}\SpecialCharTok{$}\NormalTok{gam, }\AttributeTok{type =} \StringTok{"deviance"}\NormalTok{),}
     \AttributeTok{xlab =} \StringTok{"Предсказанные значения"}\NormalTok{, }\AttributeTok{ylab =} \StringTok{"Девиансные остатки"}\NormalTok{,}
     \AttributeTok{main =} \StringTok{"Остатки GAMM vs. Предсказания"}\NormalTok{)}
\end{Highlighting}
\end{Shaded}

\begin{verbatim}
Warning in title(...): неизвестна ширина символа 0xce в кодировке CP1251
\end{verbatim}

\begin{verbatim}
Warning in title(...): неизвестна ширина символа 0xf1 в кодировке CP1251
\end{verbatim}

\begin{verbatim}
Warning in title(...): неизвестна ширина символа 0xf2 в кодировке CP1251
\end{verbatim}

\begin{verbatim}
Warning in title(...): неизвестна ширина символа 0xe0 в кодировке CP1251
\end{verbatim}

\begin{verbatim}
Warning in title(...): неизвестна ширина символа 0xf2 в кодировке CP1251
\end{verbatim}

\begin{verbatim}
Warning in title(...): неизвестна ширина символа 0xea в кодировке CP1251
\end{verbatim}

\begin{verbatim}
Warning in title(...): неизвестна ширина символа 0xe8 в кодировке CP1251
\end{verbatim}

\begin{verbatim}
Warning in title(...): неизвестна ширина символа 0xcf в кодировке CP1251
\end{verbatim}

\begin{verbatim}
Warning in title(...): неизвестна ширина символа 0xf0 в кодировке CP1251
\end{verbatim}

\begin{verbatim}
Warning in title(...): неизвестна ширина символа 0xe5 в кодировке CP1251
\end{verbatim}

\begin{verbatim}
Warning in title(...): неизвестна ширина символа 0xe4 в кодировке CP1251
\end{verbatim}

\begin{verbatim}
Warning in title(...): неизвестна ширина символа 0xf1 в кодировке CP1251
\end{verbatim}

\begin{verbatim}
Warning in title(...): неизвестна ширина символа 0xea в кодировке CP1251
\end{verbatim}

\begin{verbatim}
Warning in title(...): неизвестна ширина символа 0xe0 в кодировке CP1251
\end{verbatim}

\begin{verbatim}
Warning in title(...): неизвестна ширина символа 0xe7 в кодировке CP1251
\end{verbatim}

\begin{verbatim}
Warning in title(...): неизвестна ширина символа 0xe0 в кодировке CP1251
\end{verbatim}

\begin{verbatim}
Warning in title(...): неизвестна ширина символа 0xed в кодировке CP1251
\end{verbatim}

\begin{verbatim}
Warning in title(...): неизвестна ширина символа 0xe8 в кодировке CP1251
\end{verbatim}

\begin{verbatim}
Warning in title(...): неизвестна ширина символа 0xff в кодировке CP1251
\end{verbatim}

\begin{verbatim}
Warning in title(...): неизвестна ширина символа 0xcf в кодировке CP1251
\end{verbatim}

\begin{verbatim}
Warning in title(...): неизвестна ширина символа 0xf0 в кодировке CP1251
\end{verbatim}

\begin{verbatim}
Warning in title(...): неизвестна ширина символа 0xe5 в кодировке CP1251
\end{verbatim}

\begin{verbatim}
Warning in title(...): неизвестна ширина символа 0xe4 в кодировке CP1251
\end{verbatim}

\begin{verbatim}
Warning in title(...): неизвестна ширина символа 0xf1 в кодировке CP1251
\end{verbatim}

\begin{verbatim}
Warning in title(...): неизвестна ширина символа 0xea в кодировке CP1251
\end{verbatim}

\begin{verbatim}
Warning in title(...): неизвестна ширина символа 0xe0 в кодировке CP1251
\end{verbatim}

\begin{verbatim}
Warning in title(...): неизвестна ширина символа 0xe7 в кодировке CP1251
\end{verbatim}

\begin{verbatim}
Warning in title(...): неизвестна ширина символа 0xe0 в кодировке CP1251
\end{verbatim}

\begin{verbatim}
Warning in title(...): неизвестна ширина символа 0xed в кодировке CP1251
Warning in title(...): неизвестна ширина символа 0xed в кодировке CP1251
\end{verbatim}

\begin{verbatim}
Warning in title(...): неизвестна ширина символа 0xfb в кодировке CP1251
\end{verbatim}

\begin{verbatim}
Warning in title(...): неизвестна ширина символа 0xe5 в кодировке CP1251
\end{verbatim}

\begin{verbatim}
Warning in title(...): неизвестна ширина символа 0xe7 в кодировке CP1251
\end{verbatim}

\begin{verbatim}
Warning in title(...): неизвестна ширина символа 0xed в кодировке CP1251
\end{verbatim}

\begin{verbatim}
Warning in title(...): неизвестна ширина символа 0xe0 в кодировке CP1251
\end{verbatim}

\begin{verbatim}
Warning in title(...): неизвестна ширина символа 0xf7 в кодировке CP1251
\end{verbatim}

\begin{verbatim}
Warning in title(...): неизвестна ширина символа 0xe5 в кодировке CP1251
\end{verbatim}

\begin{verbatim}
Warning in title(...): неизвестна ширина символа 0xed в кодировке CP1251
\end{verbatim}

\begin{verbatim}
Warning in title(...): неизвестна ширина символа 0xe8 в кодировке CP1251
\end{verbatim}

\begin{verbatim}
Warning in title(...): неизвестна ширина символа 0xff в кодировке CP1251
\end{verbatim}

\begin{verbatim}
Warning in title(...): неизвестна ширина символа 0xc4 в кодировке CP1251
\end{verbatim}

\begin{verbatim}
Warning in title(...): неизвестна ширина символа 0xe5 в кодировке CP1251
\end{verbatim}

\begin{verbatim}
Warning in title(...): неизвестна ширина символа 0xe2 в кодировке CP1251
\end{verbatim}

\begin{verbatim}
Warning in title(...): неизвестна ширина символа 0xe8 в кодировке CP1251
\end{verbatim}

\begin{verbatim}
Warning in title(...): неизвестна ширина символа 0xe0 в кодировке CP1251
\end{verbatim}

\begin{verbatim}
Warning in title(...): неизвестна ширина символа 0xed в кодировке CP1251
\end{verbatim}

\begin{verbatim}
Warning in title(...): неизвестна ширина символа 0xf1 в кодировке CP1251
\end{verbatim}

\begin{verbatim}
Warning in title(...): неизвестна ширина символа 0xed в кодировке CP1251
\end{verbatim}

\begin{verbatim}
Warning in title(...): неизвестна ширина символа 0xfb в кодировке CP1251
\end{verbatim}

\begin{verbatim}
Warning in title(...): неизвестна ширина символа 0xe5 в кодировке CP1251
\end{verbatim}

\begin{verbatim}
Warning in title(...): неизвестна ширина символа 0xee в кодировке CP1251
\end{verbatim}

\begin{verbatim}
Warning in title(...): неизвестна ширина символа 0xf1 в кодировке CP1251
\end{verbatim}

\begin{verbatim}
Warning in title(...): неизвестна ширина символа 0xf2 в кодировке CP1251
\end{verbatim}

\begin{verbatim}
Warning in title(...): неизвестна ширина символа 0xe0 в кодировке CP1251
\end{verbatim}

\begin{verbatim}
Warning in title(...): неизвестна ширина символа 0xf2 в кодировке CP1251
\end{verbatim}

\begin{verbatim}
Warning in title(...): неизвестна ширина символа 0xea в кодировке CP1251
\end{verbatim}

\begin{verbatim}
Warning in title(...): неизвестна ширина символа 0xe8 в кодировке CP1251
\end{verbatim}

\begin{Shaded}
\begin{Highlighting}[]
\FunctionTok{abline}\NormalTok{(}\AttributeTok{h =} \DecValTok{0}\NormalTok{, }\AttributeTok{col =} \StringTok{"red"}\NormalTok{, }\AttributeTok{lty =} \DecValTok{2}\NormalTok{)}
\end{Highlighting}
\end{Shaded}

\pandocbounded{\includegraphics[keepaspectratio]{chapter9_files/figure-pdf/unnamed-chunk-1-8.pdf}}

\begin{Shaded}
\begin{Highlighting}[]
\CommentTok{\# 2. QQ{-}plot для остатков}
\FunctionTok{qqnorm}\NormalTok{(}\FunctionTok{residuals}\NormalTok{(gamm\_fit}\SpecialCharTok{$}\NormalTok{gam, }\AttributeTok{type =} \StringTok{"deviance"}\NormalTok{),}
       \AttributeTok{main =} \StringTok{"QQ{-}plot для остатков GAMM"}\NormalTok{)}
\end{Highlighting}
\end{Shaded}

\begin{verbatim}
Warning in title(...): неизвестна ширина символа 0xe4 в кодировке CP1251
\end{verbatim}

\begin{verbatim}
Warning in title(...): неизвестна ширина символа 0xeb в кодировке CP1251
\end{verbatim}

\begin{verbatim}
Warning in title(...): неизвестна ширина символа 0xff в кодировке CP1251
\end{verbatim}

\begin{verbatim}
Warning in title(...): неизвестна ширина символа 0xee в кодировке CP1251
\end{verbatim}

\begin{verbatim}
Warning in title(...): неизвестна ширина символа 0xf1 в кодировке CP1251
\end{verbatim}

\begin{verbatim}
Warning in title(...): неизвестна ширина символа 0xf2 в кодировке CP1251
\end{verbatim}

\begin{verbatim}
Warning in title(...): неизвестна ширина символа 0xe0 в кодировке CP1251
\end{verbatim}

\begin{verbatim}
Warning in title(...): неизвестна ширина символа 0xf2 в кодировке CP1251
\end{verbatim}

\begin{verbatim}
Warning in title(...): неизвестна ширина символа 0xea в кодировке CP1251
\end{verbatim}

\begin{verbatim}
Warning in title(...): неизвестна ширина символа 0xee в кодировке CP1251
\end{verbatim}

\begin{verbatim}
Warning in title(...): неизвестна ширина символа 0xe2 в кодировке CP1251
\end{verbatim}

\begin{Shaded}
\begin{Highlighting}[]
\FunctionTok{qqline}\NormalTok{(}\FunctionTok{residuals}\NormalTok{(gamm\_fit}\SpecialCharTok{$}\NormalTok{gam, }\AttributeTok{type =} \StringTok{"deviance"}\NormalTok{), }\AttributeTok{col =} \StringTok{"red"}\NormalTok{)}
\end{Highlighting}
\end{Shaded}

\pandocbounded{\includegraphics[keepaspectratio]{chapter9_files/figure-pdf/unnamed-chunk-1-9.pdf}}

\begin{Shaded}
\begin{Highlighting}[]
\CommentTok{\# 3. Диагностика случайных эффектов}
\FunctionTok{cat}\NormalTok{(}\StringTok{"}\SpecialCharTok{\textbackslash{}n}\StringTok{Случайные эффекты (CALL):}\SpecialCharTok{\textbackslash{}n}\StringTok{"}\NormalTok{)}
\end{Highlighting}
\end{Shaded}

\begin{verbatim}

Случайные эффекты (CALL):
\end{verbatim}

\begin{Shaded}
\begin{Highlighting}[]
\FunctionTok{print}\NormalTok{(}\FunctionTok{summary}\NormalTok{(}\FunctionTok{ranef}\NormalTok{(gamm\_fit}\SpecialCharTok{$}\NormalTok{mer)}\SpecialCharTok{$}\NormalTok{CALL))}
\end{Highlighting}
\end{Shaded}

\begin{verbatim}
  (Intercept)      
 Min.   :-2.92229  
 1st Qu.: 0.09295  
 Median : 0.32996  
 Mean   : 0.00306  
 3rd Qu.: 0.54065  
 Max.   : 0.93271  
\end{verbatim}

\begin{Shaded}
\begin{Highlighting}[]
\CommentTok{\# График случайных эффектов}
\NormalTok{random\_effects }\OtherTok{\textless{}{-}} \FunctionTok{ranef}\NormalTok{(gamm\_fit}\SpecialCharTok{$}\NormalTok{mer)}\SpecialCharTok{$}\NormalTok{CALL}
\FunctionTok{plot}\NormalTok{(}\FunctionTok{density}\NormalTok{(random\_effects[,}\DecValTok{1}\NormalTok{]), }\AttributeTok{main =} \StringTok{"Распределение случайных эффектов"}\NormalTok{,}
     \AttributeTok{xlab =} \StringTok{"Случайный эффект"}\NormalTok{, }\AttributeTok{ylab =} \StringTok{"Плотность"}\NormalTok{)}
\end{Highlighting}
\end{Shaded}

\begin{verbatim}
Warning in title(...): неизвестна ширина символа 0xd0 в кодировке CP1251
\end{verbatim}

\begin{verbatim}
Warning in title(...): неизвестна ширина символа 0xe0 в кодировке CP1251
\end{verbatim}

\begin{verbatim}
Warning in title(...): неизвестна ширина символа 0xf1 в кодировке CP1251
\end{verbatim}

\begin{verbatim}
Warning in title(...): неизвестна ширина символа 0xef в кодировке CP1251
\end{verbatim}

\begin{verbatim}
Warning in title(...): неизвестна ширина символа 0xf0 в кодировке CP1251
\end{verbatim}

\begin{verbatim}
Warning in title(...): неизвестна ширина символа 0xe5 в кодировке CP1251
\end{verbatim}

\begin{verbatim}
Warning in title(...): неизвестна ширина символа 0xe4 в кодировке CP1251
\end{verbatim}

\begin{verbatim}
Warning in title(...): неизвестна ширина символа 0xe5 в кодировке CP1251
\end{verbatim}

\begin{verbatim}
Warning in title(...): неизвестна ширина символа 0xeb в кодировке CP1251
\end{verbatim}

\begin{verbatim}
Warning in title(...): неизвестна ширина символа 0xe5 в кодировке CP1251
\end{verbatim}

\begin{verbatim}
Warning in title(...): неизвестна ширина символа 0xed в кодировке CP1251
\end{verbatim}

\begin{verbatim}
Warning in title(...): неизвестна ширина символа 0xe8 в кодировке CP1251
\end{verbatim}

\begin{verbatim}
Warning in title(...): неизвестна ширина символа 0xe5 в кодировке CP1251
\end{verbatim}

\begin{verbatim}
Warning in title(...): неизвестна ширина символа 0xf1 в кодировке CP1251
\end{verbatim}

\begin{verbatim}
Warning in title(...): неизвестна ширина символа 0xeb в кодировке CP1251
\end{verbatim}

\begin{verbatim}
Warning in title(...): неизвестна ширина символа 0xf3 в кодировке CP1251
\end{verbatim}

\begin{verbatim}
Warning in title(...): неизвестна ширина символа 0xf7 в кодировке CP1251
\end{verbatim}

\begin{verbatim}
Warning in title(...): неизвестна ширина символа 0xe0 в кодировке CP1251
\end{verbatim}

\begin{verbatim}
Warning in title(...): неизвестна ширина символа 0xe9 в кодировке CP1251
\end{verbatim}

\begin{verbatim}
Warning in title(...): неизвестна ширина символа 0xed в кодировке CP1251
\end{verbatim}

\begin{verbatim}
Warning in title(...): неизвестна ширина символа 0xfb в кодировке CP1251
\end{verbatim}

\begin{verbatim}
Warning in title(...): неизвестна ширина символа 0xf5 в кодировке CP1251
\end{verbatim}

\begin{verbatim}
Warning in title(...): неизвестна ширина символа 0xfd в кодировке CP1251
\end{verbatim}

\begin{verbatim}
Warning in title(...): неизвестна ширина символа 0xf4 в кодировке CP1251
Warning in title(...): неизвестна ширина символа 0xf4 в кодировке CP1251
\end{verbatim}

\begin{verbatim}
Warning in title(...): неизвестна ширина символа 0xe5 в кодировке CP1251
\end{verbatim}

\begin{verbatim}
Warning in title(...): неизвестна ширина символа 0xea в кодировке CP1251
\end{verbatim}

\begin{verbatim}
Warning in title(...): неизвестна ширина символа 0xf2 в кодировке CP1251
\end{verbatim}

\begin{verbatim}
Warning in title(...): неизвестна ширина символа 0xee в кодировке CP1251
\end{verbatim}

\begin{verbatim}
Warning in title(...): неизвестна ширина символа 0xe2 в кодировке CP1251
\end{verbatim}

\begin{verbatim}
Warning in title(...): неизвестна ширина символа 0xd1 в кодировке CP1251
\end{verbatim}

\begin{verbatim}
Warning in title(...): неизвестна ширина символа 0xeb в кодировке CP1251
\end{verbatim}

\begin{verbatim}
Warning in title(...): неизвестна ширина символа 0xf3 в кодировке CP1251
\end{verbatim}

\begin{verbatim}
Warning in title(...): неизвестна ширина символа 0xf7 в кодировке CP1251
\end{verbatim}

\begin{verbatim}
Warning in title(...): неизвестна ширина символа 0xe0 в кодировке CP1251
\end{verbatim}

\begin{verbatim}
Warning in title(...): неизвестна ширина символа 0xe9 в кодировке CP1251
\end{verbatim}

\begin{verbatim}
Warning in title(...): неизвестна ширина символа 0xed в кодировке CP1251
\end{verbatim}

\begin{verbatim}
Warning in title(...): неизвестна ширина символа 0xfb в кодировке CP1251
\end{verbatim}

\begin{verbatim}
Warning in title(...): неизвестна ширина символа 0xe9 в кодировке CP1251
\end{verbatim}

\begin{verbatim}
Warning in title(...): неизвестна ширина символа 0xfd в кодировке CP1251
\end{verbatim}

\begin{verbatim}
Warning in title(...): неизвестна ширина символа 0xf4 в кодировке CP1251
Warning in title(...): неизвестна ширина символа 0xf4 в кодировке CP1251
\end{verbatim}

\begin{verbatim}
Warning in title(...): неизвестна ширина символа 0xe5 в кодировке CP1251
\end{verbatim}

\begin{verbatim}
Warning in title(...): неизвестна ширина символа 0xea в кодировке CP1251
\end{verbatim}

\begin{verbatim}
Warning in title(...): неизвестна ширина символа 0xf2 в кодировке CP1251
\end{verbatim}

\begin{verbatim}
Warning in title(...): неизвестна ширина символа 0xcf в кодировке CP1251
\end{verbatim}

\begin{verbatim}
Warning in title(...): неизвестна ширина символа 0xeb в кодировке CP1251
\end{verbatim}

\begin{verbatim}
Warning in title(...): неизвестна ширина символа 0xee в кодировке CP1251
\end{verbatim}

\begin{verbatim}
Warning in title(...): неизвестна ширина символа 0xf2 в кодировке CP1251
\end{verbatim}

\begin{verbatim}
Warning in title(...): неизвестна ширина символа 0xed в кодировке CP1251
\end{verbatim}

\begin{verbatim}
Warning in title(...): неизвестна ширина символа 0xee в кодировке CP1251
\end{verbatim}

\begin{verbatim}
Warning in title(...): неизвестна ширина символа 0xf1 в кодировке CP1251
\end{verbatim}

\begin{verbatim}
Warning in title(...): неизвестна ширина символа 0xf2 в кодировке CP1251
\end{verbatim}

\begin{verbatim}
Warning in title(...): неизвестна ширина символа 0xfc в кодировке CP1251
\end{verbatim}

\pandocbounded{\includegraphics[keepaspectratio]{chapter9_files/figure-pdf/unnamed-chunk-1-10.pdf}}

\begin{Shaded}
\begin{Highlighting}[]
\CommentTok{\# 5. Проверка гетероскедастичности}
\FunctionTok{library}\NormalTok{(lmtest)}
\end{Highlighting}
\end{Shaded}

\begin{verbatim}
Загрузка требуемого пакета: zoo
\end{verbatim}

\begin{verbatim}

Присоединяю пакет: 'zoo'
\end{verbatim}

\begin{verbatim}
Следующие объекты скрыты от 'package:base':

    as.Date, as.Date.numeric
\end{verbatim}

\begin{Shaded}
\begin{Highlighting}[]
\FunctionTok{bptest}\NormalTok{(gamm\_fit}\SpecialCharTok{$}\NormalTok{gam}\SpecialCharTok{$}\NormalTok{y }\SpecialCharTok{\textasciitilde{}} \FunctionTok{fitted}\NormalTok{(gamm\_fit}\SpecialCharTok{$}\NormalTok{gam)) }\SpecialCharTok{\%\textgreater{}\%} 
  \FunctionTok{print}\NormalTok{()}
\end{Highlighting}
\end{Shaded}

\begin{verbatim}

    studentized Breusch-Pagan test

data:  gamm_fit$gam$y ~ fitted(gamm_fit$gam)
BP = 222.14, df = 1, p-value < 2.2e-16
\end{verbatim}

\begin{Shaded}
\begin{Highlighting}[]
\CommentTok{\# 6. Сводка по модели}
\FunctionTok{cat}\NormalTok{(}\StringTok{"}\SpecialCharTok{\textbackslash{}n}\StringTok{Сводка GAMM модели:}\SpecialCharTok{\textbackslash{}n}\StringTok{"}\NormalTok{)}
\end{Highlighting}
\end{Shaded}

\begin{verbatim}

Сводка GAMM модели:
\end{verbatim}

\begin{Shaded}
\begin{Highlighting}[]
\FunctionTok{print}\NormalTok{(}\FunctionTok{summary}\NormalTok{(gamm\_fit}\SpecialCharTok{$}\NormalTok{gam))}
\end{Highlighting}
\end{Shaded}

\begin{verbatim}

Family: Gamma 
Link function: log 

Formula:
CPUE_POS ~ YEAR + MONTH + REGION

Parametric coefficients:
                                                  Estimate Std. Error t value
(Intercept)                                        4.91649    0.17171  28.632
YEAR2020                                          -0.23162    0.04580  -5.058
YEAR2021                                          -0.22798    0.04552  -5.008
YEAR2022                                          -0.64763    0.04338 -14.928
YEAR2023                                          -0.77570    0.04641 -16.716
YEAR2024                                          -1.13070    0.05152 -21.947
MONTH10                                           -0.13620    0.02445  -5.571
MONTH11                                           -0.13630    0.03542  -3.848
REGIONCEB.-ЦEHTPAЛЬHЫЙ P-H                         0.16149    0.46710   0.346
REGIONCEB.CKЛOH MУPMAHCKOГO MEЛKOBOДЬЯ            -0.08013    0.15524  -0.516
REGIONCEBEPO-KAHИHCKAЯ БAHKA                       0.15971    0.08238   1.939
REGIONKAHИHCKAЯ БAHKA                              0.05886    0.07877   0.747
REGIONKAHИHCKO- KOЛГУEBCKOE MEЛKOBOДЬE(CEB.CKЛOH)  0.20859    0.08522   2.448
REGIONKAHИHCKO-KOЛГУEBCKOE MEЛKOBOДЬE              0.17834    0.07759   2.299
REGIONMУPMAHCKOE MEЛKOBOДЬE                        0.07353    0.07755   0.948
REGIONЗAП.-ПPИБPEЖHЫЙ P-H                          0.72926    0.46366   1.573
REGIONЗAП.-ЦEHTPAЛЬHЫЙ P-H                         0.12967    0.08419   1.540
REGIONЗAП.CKЛOH ГУCИHOЙ БAHKИ                      0.63836    0.11653   5.478
REGIONЮЖ.CKЛOH ГУCИHOЙ БAHKИ                       0.68988    0.13078   5.275
                                                  Pr(>|t|)    
(Intercept)                                        < 2e-16 ***
YEAR2020                                          4.44e-07 ***
YEAR2021                                          5.74e-07 ***
YEAR2022                                           < 2e-16 ***
YEAR2023                                           < 2e-16 ***
YEAR2024                                           < 2e-16 ***
MONTH10                                           2.70e-08 ***
MONTH11                                           0.000121 ***
REGIONCEB.-ЦEHTPAЛЬHЫЙ P-H                        0.729566    
REGIONCEB.CKЛOH MУPMAHCKOГO MEЛKOBOДЬЯ            0.605763    
REGIONCEBEPO-KAHИHCKAЯ БAHKA                      0.052628 .  
REGIONKAHИHCKAЯ БAHKA                             0.455013    
REGIONKAHИHCKO- KOЛГУEBCKOE MEЛKOBOДЬE(CEB.CKЛOH) 0.014424 *  
REGIONKAHИHCKO-KOЛГУEBCKOE MEЛKOBOДЬE             0.021580 *  
REGIONMУPMAHCKOE MEЛKOBOДЬE                       0.343099    
REGIONЗAП.-ПPИБPEЖHЫЙ P-H                         0.115842    
REGIONЗAП.-ЦEHTPAЛЬHЫЙ P-H                        0.123604    
REGIONЗAП.CKЛOH ГУCИHOЙ БAHKИ                     4.57e-08 ***
REGIONЮЖ.CKЛOH ГУCИHOЙ БAHKИ                      1.40e-07 ***
---
Signif. codes:  0 '***' 0.001 '**' 0.01 '*' 0.05 '.' 0.1 ' ' 1


R-sq.(adj) =  0.172   
glmer.ML =   1786  Scale est. = 0.41793   n = 3891
\end{verbatim}

\begin{Shaded}
\begin{Highlighting}[]
\FunctionTok{print}\NormalTok{(}\FunctionTok{summary}\NormalTok{(gamm\_fit}\SpecialCharTok{$}\NormalTok{mer))}
\end{Highlighting}
\end{Shaded}

\begin{verbatim}
Generalized linear mixed model fit by maximum likelihood (Laplace
  Approximation) [glmerMod]
 Family: Gamma  ( log )

      AIC       BIC    logLik -2*log(L)  df.resid 
  42933.5   43065.1  -21445.7   42891.5      3870 

Scaled residuals: 
    Min      1Q  Median      3Q     Max 
-1.5364 -0.6769 -0.1721  0.4716 11.0577 

Random effects:
 Groups   Name        Variance Std.Dev.
 CALL     (Intercept) 0.4320   0.6573  
 Residual             0.4179   0.6465  
Number of obs: 3891, groups:  CALL, 19

Fixed effects:
                                                   Estimate Std. Error t value
X(Intercept)                                        4.91649    0.23488  20.932
XYEAR2020                                          -0.23162    0.02474  -9.363
XYEAR2021                                          -0.22798    0.02513  -9.073
XYEAR2022                                          -0.64763    0.02299 -28.167
XYEAR2023                                          -0.77570    0.02376 -32.652
XYEAR2024                                          -1.13070    0.02603 -43.443
XMONTH10                                           -0.13620    0.01914  -7.116
XMONTH11                                           -0.13630    0.02359  -5.777
XREGIONCEB.-ЦEHTPAЛЬHЫЙ P-H                         0.16149    0.46456   0.348
XREGIONCEB.CKЛOH MУPMAHCKOГO MEЛKOBOДЬЯ            -0.08013    0.03506  -2.285
XREGIONCEBEPO-KAHИHCKAЯ БAHKA                       0.15971    0.02670   5.982
XREGIONKAHИHCKAЯ БAHKA                              0.05886    0.02653   2.218
XREGIONKAHИHCKO- KOЛГУEBCKOE MEЛKOBOДЬE(CEB.CKЛOH)  0.20859    0.02808   7.428
XREGIONKAHИHCKO-KOЛГУEBCKOE MEЛKOBOДЬE              0.17834    0.02200   8.108
XREGIONMУPMAHCKOE MEЛKOBOДЬE                        0.07353    0.02318   3.172
XREGIONЗAП.-ПPИБPEЖHЫЙ P-H                          0.72926    0.46533   1.567
XREGIONЗAП.-ЦEHTPAЛЬHЫЙ P-H                         0.12967    0.02736   4.739
XREGIONЗAП.CKЛOH ГУCИHOЙ БAHKИ                      0.63836    0.03291  19.400
XREGIONЮЖ.CKЛOH ГУCИHOЙ БAHKИ                       0.68988    0.03447  20.015
                                                   Pr(>|z|)    
X(Intercept)                                        < 2e-16 ***
XYEAR2020                                           < 2e-16 ***
XYEAR2021                                           < 2e-16 ***
XYEAR2022                                           < 2e-16 ***
XYEAR2023                                           < 2e-16 ***
XYEAR2024                                           < 2e-16 ***
XMONTH10                                           1.11e-12 ***
XMONTH11                                           7.60e-09 ***
XREGIONCEB.-ЦEHTPAЛЬHЫЙ P-H                         0.72813    
XREGIONCEB.CKЛOH MУPMAHCKOГO MEЛKOBOДЬЯ             0.02229 *  
XREGIONCEBEPO-KAHИHCKAЯ БAHKA                      2.20e-09 ***
XREGIONKAHИHCKAЯ БAHKA                              0.02654 *  
XREGIONKAHИHCKO- KOЛГУEBCKOE MEЛKOBOДЬE(CEB.CKЛOH) 1.10e-13 ***
XREGIONKAHИHCKO-KOЛГУEBCKOE MEЛKOBOДЬE             5.16e-16 ***
XREGIONMУPMAHCKOE MEЛKOBOДЬE                        0.00151 ** 
XREGIONЗAП.-ПPИБPEЖHЫЙ P-H                          0.11707    
XREGIONЗAП.-ЦEHTPAЛЬHЫЙ P-H                        2.15e-06 ***
XREGIONЗAП.CKЛOH ГУCИHOЙ БAHKИ                      < 2e-16 ***
XREGIONЮЖ.CKЛOH ГУCИHOЙ БAHKИ                       < 2e-16 ***
---
Signif. codes:  0 '***' 0.001 '**' 0.01 '*' 0.05 '.' 0.1 ' ' 1
\end{verbatim}

\begin{verbatim}

Correlation matrix not shown by default, as p = 19 > 12.
Use print(summary(gamm_fit$mer), correlation=TRUE)  or
    vcov(summary(gamm_fit$mer))        if you need it
\end{verbatim}

\begin{Shaded}
\begin{Highlighting}[]
\CommentTok{\# Создание сетки для предсказания}
\NormalTok{newdata\_grid }\OtherTok{\textless{}{-}} \FunctionTok{expand.grid}\NormalTok{(}
  \AttributeTok{YEAR =} \FunctionTok{levels}\NormalTok{(DATA}\SpecialCharTok{$}\NormalTok{YEAR),}
  \AttributeTok{MONTH =} \FunctionTok{levels}\NormalTok{(DATA}\SpecialCharTok{$}\NormalTok{MONTH),}
  \AttributeTok{REGION =} \FunctionTok{levels}\NormalTok{(DATA}\SpecialCharTok{$}\NormalTok{REGION),}
  \AttributeTok{CALL =} \FunctionTok{levels}\NormalTok{(DATA}\SpecialCharTok{$}\NormalTok{CALL)[}\DecValTok{1}\NormalTok{]  }\CommentTok{\# Фиксированное значение для случайного эффекта}
\NormalTok{)}

\CommentTok{\# Предсказание на сетке}
\NormalTok{newdata\_grid}\SpecialCharTok{$}\NormalTok{pred }\OtherTok{\textless{}{-}} \FunctionTok{predict}\NormalTok{(gamm\_fit}\SpecialCharTok{$}\NormalTok{gam, }
                            \AttributeTok{newdata =}\NormalTok{ newdata\_grid, }
                            \AttributeTok{type =} \StringTok{"response"}\NormalTok{)}

\CommentTok{\# Усреднение предсказаний по годам}
\NormalTok{idx\_gamm }\OtherTok{\textless{}{-}}\NormalTok{ newdata\_grid }\SpecialCharTok{\%\textgreater{}\%}
  \FunctionTok{group\_by}\NormalTok{(YEAR) }\SpecialCharTok{\%\textgreater{}\%}
  \FunctionTok{summarise}\NormalTok{(}\AttributeTok{value =} \FunctionTok{mean}\NormalTok{(pred, }\AttributeTok{na.rm =} \ConstantTok{TRUE}\NormalTok{)) }\SpecialCharTok{\%\textgreater{}\%}
  \FunctionTok{mutate}\NormalTok{(}
    \AttributeTok{model =} \StringTok{"GAMM (mgcv)"}\NormalTok{,}
    \AttributeTok{index\_mean =} \FunctionTok{scale\_to\_index}\NormalTok{(value, }\StringTok{"mean"}\NormalTok{),}
    \AttributeTok{index\_first =} \FunctionTok{scale\_to\_index}\NormalTok{(value, }\StringTok{"first"}\NormalTok{)}
\NormalTok{  )}

\CommentTok{\# Функция расчета доверительных интервалов через бутстреп}
\NormalTok{compute\_gamm\_ci }\OtherTok{\textless{}{-}} \ControlFlowTok{function}\NormalTok{(model, newdata, }\AttributeTok{n\_boot =} \DecValTok{100}\NormalTok{) \{}
\NormalTok{  boot\_means }\OtherTok{\textless{}{-}} \FunctionTok{replicate}\NormalTok{(n\_boot, \{}
\NormalTok{    boot\_data }\OtherTok{\textless{}{-}}\NormalTok{ newdata[}\FunctionTok{sample}\NormalTok{(}\FunctionTok{nrow}\NormalTok{(newdata), }\AttributeTok{replace =} \ConstantTok{TRUE}\NormalTok{), ]}
\NormalTok{    preds }\OtherTok{\textless{}{-}} \FunctionTok{predict}\NormalTok{(model, }\AttributeTok{newdata =}\NormalTok{ boot\_data, }\AttributeTok{type =} \StringTok{"response"}\NormalTok{)}
\NormalTok{    boot\_data }\SpecialCharTok{\%\textgreater{}\%}
      \FunctionTok{mutate}\NormalTok{(}\AttributeTok{pred =}\NormalTok{ preds) }\SpecialCharTok{\%\textgreater{}\%}
      \FunctionTok{group\_by}\NormalTok{(YEAR) }\SpecialCharTok{\%\textgreater{}\%}
      \FunctionTok{summarise}\NormalTok{(}\AttributeTok{mean\_pred =} \FunctionTok{mean}\NormalTok{(pred, }\AttributeTok{na.rm =} \ConstantTok{TRUE}\NormalTok{)) }\SpecialCharTok{\%\textgreater{}\%}
      \FunctionTok{pull}\NormalTok{(mean\_pred)}
\NormalTok{  \})}
  
\NormalTok{  ci }\OtherTok{\textless{}{-}} \FunctionTok{apply}\NormalTok{(boot\_means, }\DecValTok{1}\NormalTok{, }\ControlFlowTok{function}\NormalTok{(x) }\FunctionTok{quantile}\NormalTok{(x, }\FunctionTok{c}\NormalTok{(}\FloatTok{0.025}\NormalTok{, }\FloatTok{0.975}\NormalTok{), }\AttributeTok{na.rm =} \ConstantTok{TRUE}\NormalTok{))}
  \FunctionTok{return}\NormalTok{(}\FunctionTok{list}\NormalTok{(}\AttributeTok{mean =} \FunctionTok{rowMeans}\NormalTok{(boot\_means), }\AttributeTok{lcl =}\NormalTok{ ci[}\DecValTok{1}\NormalTok{, ], }\AttributeTok{ucl =}\NormalTok{ ci[}\DecValTok{2}\NormalTok{, ]))}
\NormalTok{\}}

\CommentTok{\# Расчет интервалов}
\NormalTok{gamm\_ci }\OtherTok{\textless{}{-}} \FunctionTok{compute\_gamm\_ci}\NormalTok{(gamm\_fit}\SpecialCharTok{$}\NormalTok{gam, newdata\_grid)}

\CommentTok{\# Добавление интервалов к индексам}
\NormalTok{idx\_gamm }\OtherTok{\textless{}{-}}\NormalTok{ idx\_gamm }\SpecialCharTok{\%\textgreater{}\%}
  \FunctionTok{mutate}\NormalTok{(}
    \AttributeTok{lcl =}\NormalTok{ gamm\_ci}\SpecialCharTok{$}\NormalTok{lcl,}
    \AttributeTok{ucl =}\NormalTok{ gamm\_ci}\SpecialCharTok{$}\NormalTok{ucl,}
    \AttributeTok{lcl\_index\_mean =} \FunctionTok{scale\_to\_index}\NormalTok{(lcl, }\StringTok{"mean"}\NormalTok{),}
    \AttributeTok{ucl\_index\_mean =} \FunctionTok{scale\_to\_index}\NormalTok{(ucl, }\StringTok{"mean"}\NormalTok{)}
\NormalTok{  )}

\CommentTok{\# Визуализация}
\NormalTok{idx\_gamm }\SpecialCharTok{\%\textgreater{}\%}
  \FunctionTok{ggplot}\NormalTok{(}\FunctionTok{aes}\NormalTok{(}\AttributeTok{x =}\NormalTok{ YEAR, }\AttributeTok{y =}\NormalTok{ value, }\AttributeTok{group =} \DecValTok{1}\NormalTok{)) }\SpecialCharTok{+}
  \FunctionTok{geom\_line}\NormalTok{() }\SpecialCharTok{+}
  \FunctionTok{geom\_point}\NormalTok{() }\SpecialCharTok{+}
  \FunctionTok{geom\_ribbon}\NormalTok{(}\FunctionTok{aes}\NormalTok{(}\AttributeTok{ymin =}\NormalTok{ lcl, }\AttributeTok{ymax =}\NormalTok{ ucl), }\AttributeTok{alpha =} \FloatTok{0.3}\NormalTok{) }\SpecialCharTok{+}
  \FunctionTok{geom\_point}\NormalTok{(}\AttributeTok{data =}\NormalTok{ actual\_medians, }
           \FunctionTok{aes}\NormalTok{(}\AttributeTok{x =}\NormalTok{ YEAR, }\AttributeTok{y =}\NormalTok{ median\_cpue), }
           \AttributeTok{shape =} \DecValTok{4}\NormalTok{,  }\CommentTok{\# 4 соответствует крестику (x)}
           \AttributeTok{size =} \DecValTok{3}\NormalTok{, }
           \AttributeTok{color =} \StringTok{"black"}\NormalTok{, }
           \AttributeTok{inherit.aes =} \ConstantTok{FALSE}\NormalTok{)}\SpecialCharTok{+}
\FunctionTok{labs}\NormalTok{(}\AttributeTok{title =} \StringTok{"Индексы CPUE по GAMM модели (крестики {-} факт)"}\NormalTok{, }
       \AttributeTok{x =} \StringTok{"Год"}\NormalTok{, }
       \AttributeTok{y =} \StringTok{"Стандартизированный индекс"}\NormalTok{)}
\end{Highlighting}
\end{Shaded}

\begin{verbatim}
Warning in grid.Call(C_textBounds, as.graphicsAnnot(x$label), x$x, x$y, :
неизвестна ширина символа 0xd1 в кодировке CP1251
\end{verbatim}

\begin{verbatim}
Warning in grid.Call(C_textBounds, as.graphicsAnnot(x$label), x$x, x$y, :
неизвестна ширина символа 0xf2 в кодировке CP1251
\end{verbatim}

\begin{verbatim}
Warning in grid.Call(C_textBounds, as.graphicsAnnot(x$label), x$x, x$y, :
неизвестна ширина символа 0xe0 в кодировке CP1251
\end{verbatim}

\begin{verbatim}
Warning in grid.Call(C_textBounds, as.graphicsAnnot(x$label), x$x, x$y, :
неизвестна ширина символа 0xed в кодировке CP1251
\end{verbatim}

\begin{verbatim}
Warning in grid.Call(C_textBounds, as.graphicsAnnot(x$label), x$x, x$y, :
неизвестна ширина символа 0xe4 в кодировке CP1251
\end{verbatim}

\begin{verbatim}
Warning in grid.Call(C_textBounds, as.graphicsAnnot(x$label), x$x, x$y, :
неизвестна ширина символа 0xe0 в кодировке CP1251
\end{verbatim}

\begin{verbatim}
Warning in grid.Call(C_textBounds, as.graphicsAnnot(x$label), x$x, x$y, :
неизвестна ширина символа 0xf0 в кодировке CP1251
\end{verbatim}

\begin{verbatim}
Warning in grid.Call(C_textBounds, as.graphicsAnnot(x$label), x$x, x$y, :
неизвестна ширина символа 0xf2 в кодировке CP1251
\end{verbatim}

\begin{verbatim}
Warning in grid.Call(C_textBounds, as.graphicsAnnot(x$label), x$x, x$y, :
неизвестна ширина символа 0xe8 в кодировке CP1251
\end{verbatim}

\begin{verbatim}
Warning in grid.Call(C_textBounds, as.graphicsAnnot(x$label), x$x, x$y, :
неизвестна ширина символа 0xe7 в кодировке CP1251
\end{verbatim}

\begin{verbatim}
Warning in grid.Call(C_textBounds, as.graphicsAnnot(x$label), x$x, x$y, :
неизвестна ширина символа 0xe8 в кодировке CP1251
\end{verbatim}

\begin{verbatim}
Warning in grid.Call(C_textBounds, as.graphicsAnnot(x$label), x$x, x$y, :
неизвестна ширина символа 0xf0 в кодировке CP1251
\end{verbatim}

\begin{verbatim}
Warning in grid.Call(C_textBounds, as.graphicsAnnot(x$label), x$x, x$y, :
неизвестна ширина символа 0xee в кодировке CP1251
\end{verbatim}

\begin{verbatim}
Warning in grid.Call(C_textBounds, as.graphicsAnnot(x$label), x$x, x$y, :
неизвестна ширина символа 0xe2 в кодировке CP1251
\end{verbatim}

\begin{verbatim}
Warning in grid.Call(C_textBounds, as.graphicsAnnot(x$label), x$x, x$y, :
неизвестна ширина символа 0xe0 в кодировке CP1251
\end{verbatim}

\begin{verbatim}
Warning in grid.Call(C_textBounds, as.graphicsAnnot(x$label), x$x, x$y, :
неизвестна ширина символа 0xed в кодировке CP1251
Warning in grid.Call(C_textBounds, as.graphicsAnnot(x$label), x$x, x$y, :
неизвестна ширина символа 0xed в кодировке CP1251
\end{verbatim}

\begin{verbatim}
Warning in grid.Call(C_textBounds, as.graphicsAnnot(x$label), x$x, x$y, :
неизвестна ширина символа 0xfb в кодировке CP1251
\end{verbatim}

\begin{verbatim}
Warning in grid.Call(C_textBounds, as.graphicsAnnot(x$label), x$x, x$y, :
неизвестна ширина символа 0xe9 в кодировке CP1251
\end{verbatim}

\begin{verbatim}
Warning in grid.Call(C_textBounds, as.graphicsAnnot(x$label), x$x, x$y, :
неизвестна ширина символа 0xe8 в кодировке CP1251
\end{verbatim}

\begin{verbatim}
Warning in grid.Call(C_textBounds, as.graphicsAnnot(x$label), x$x, x$y, :
неизвестна ширина символа 0xed в кодировке CP1251
\end{verbatim}

\begin{verbatim}
Warning in grid.Call(C_textBounds, as.graphicsAnnot(x$label), x$x, x$y, :
неизвестна ширина символа 0xe4 в кодировке CP1251
\end{verbatim}

\begin{verbatim}
Warning in grid.Call(C_textBounds, as.graphicsAnnot(x$label), x$x, x$y, :
неизвестна ширина символа 0xe5 в кодировке CP1251
\end{verbatim}

\begin{verbatim}
Warning in grid.Call(C_textBounds, as.graphicsAnnot(x$label), x$x, x$y, :
неизвестна ширина символа 0xea в кодировке CP1251
\end{verbatim}

\begin{verbatim}
Warning in grid.Call(C_textBounds, as.graphicsAnnot(x$label), x$x, x$y, :
неизвестна ширина символа 0xf1 в кодировке CP1251
\end{verbatim}

\begin{verbatim}
Warning in grid.Call(C_textBounds, as.graphicsAnnot(x$label), x$x, x$y, :
неизвестна ширина символа 0xc8 в кодировке CP1251
\end{verbatim}

\begin{verbatim}
Warning in grid.Call(C_textBounds, as.graphicsAnnot(x$label), x$x, x$y, :
неизвестна ширина символа 0xed в кодировке CP1251
\end{verbatim}

\begin{verbatim}
Warning in grid.Call(C_textBounds, as.graphicsAnnot(x$label), x$x, x$y, :
неизвестна ширина символа 0xe4 в кодировке CP1251
\end{verbatim}

\begin{verbatim}
Warning in grid.Call(C_textBounds, as.graphicsAnnot(x$label), x$x, x$y, :
неизвестна ширина символа 0xe5 в кодировке CP1251
\end{verbatim}

\begin{verbatim}
Warning in grid.Call(C_textBounds, as.graphicsAnnot(x$label), x$x, x$y, :
неизвестна ширина символа 0xea в кодировке CP1251
\end{verbatim}

\begin{verbatim}
Warning in grid.Call(C_textBounds, as.graphicsAnnot(x$label), x$x, x$y, :
неизвестна ширина символа 0xf1 в кодировке CP1251
\end{verbatim}

\begin{verbatim}
Warning in grid.Call(C_textBounds, as.graphicsAnnot(x$label), x$x, x$y, :
неизвестна ширина символа 0xfb в кодировке CP1251
\end{verbatim}

\begin{verbatim}
Warning in grid.Call(C_textBounds, as.graphicsAnnot(x$label), x$x, x$y, :
неизвестна ширина символа 0xef в кодировке CP1251
\end{verbatim}

\begin{verbatim}
Warning in grid.Call(C_textBounds, as.graphicsAnnot(x$label), x$x, x$y, :
неизвестна ширина символа 0xee в кодировке CP1251
\end{verbatim}

\begin{verbatim}
Warning in grid.Call(C_textBounds, as.graphicsAnnot(x$label), x$x, x$y, :
неизвестна ширина символа 0xec в кодировке CP1251
\end{verbatim}

\begin{verbatim}
Warning in grid.Call(C_textBounds, as.graphicsAnnot(x$label), x$x, x$y, :
неизвестна ширина символа 0xee в кодировке CP1251
\end{verbatim}

\begin{verbatim}
Warning in grid.Call(C_textBounds, as.graphicsAnnot(x$label), x$x, x$y, :
неизвестна ширина символа 0xe4 в кодировке CP1251
\end{verbatim}

\begin{verbatim}
Warning in grid.Call(C_textBounds, as.graphicsAnnot(x$label), x$x, x$y, :
неизвестна ширина символа 0xe5 в кодировке CP1251
\end{verbatim}

\begin{verbatim}
Warning in grid.Call(C_textBounds, as.graphicsAnnot(x$label), x$x, x$y, :
неизвестна ширина символа 0xeb в кодировке CP1251
\end{verbatim}

\begin{verbatim}
Warning in grid.Call(C_textBounds, as.graphicsAnnot(x$label), x$x, x$y, :
неизвестна ширина символа 0xe8 в кодировке CP1251
\end{verbatim}

\begin{verbatim}
Warning in grid.Call(C_textBounds, as.graphicsAnnot(x$label), x$x, x$y, :
неизвестна ширина символа 0xea в кодировке CP1251
\end{verbatim}

\begin{verbatim}
Warning in grid.Call(C_textBounds, as.graphicsAnnot(x$label), x$x, x$y, :
неизвестна ширина символа 0xf0 в кодировке CP1251
\end{verbatim}

\begin{verbatim}
Warning in grid.Call(C_textBounds, as.graphicsAnnot(x$label), x$x, x$y, :
неизвестна ширина символа 0xe5 в кодировке CP1251
\end{verbatim}

\begin{verbatim}
Warning in grid.Call(C_textBounds, as.graphicsAnnot(x$label), x$x, x$y, :
неизвестна ширина символа 0xf1 в кодировке CP1251
\end{verbatim}

\begin{verbatim}
Warning in grid.Call(C_textBounds, as.graphicsAnnot(x$label), x$x, x$y, :
неизвестна ширина символа 0xf2 в кодировке CP1251
\end{verbatim}

\begin{verbatim}
Warning in grid.Call(C_textBounds, as.graphicsAnnot(x$label), x$x, x$y, :
неизвестна ширина символа 0xe8 в кодировке CP1251
\end{verbatim}

\begin{verbatim}
Warning in grid.Call(C_textBounds, as.graphicsAnnot(x$label), x$x, x$y, :
неизвестна ширина символа 0xea в кодировке CP1251
\end{verbatim}

\begin{verbatim}
Warning in grid.Call(C_textBounds, as.graphicsAnnot(x$label), x$x, x$y, :
неизвестна ширина символа 0xe8 в кодировке CP1251
\end{verbatim}

\begin{verbatim}
Warning in grid.Call(C_textBounds, as.graphicsAnnot(x$label), x$x, x$y, :
неизвестна ширина символа 0xf4 в кодировке CP1251
\end{verbatim}

\begin{verbatim}
Warning in grid.Call(C_textBounds, as.graphicsAnnot(x$label), x$x, x$y, :
неизвестна ширина символа 0xe0 в кодировке CP1251
\end{verbatim}

\begin{verbatim}
Warning in grid.Call(C_textBounds, as.graphicsAnnot(x$label), x$x, x$y, :
неизвестна ширина символа 0xea в кодировке CP1251
\end{verbatim}

\begin{verbatim}
Warning in grid.Call(C_textBounds, as.graphicsAnnot(x$label), x$x, x$y, :
неизвестна ширина символа 0xf2 в кодировке CP1251
\end{verbatim}

\begin{verbatim}
Warning in grid.Call(C_textBounds, as.graphicsAnnot(x$label), x$x, x$y, :
неизвестна ширина символа 0xc3 в кодировке CP1251
\end{verbatim}

\begin{verbatim}
Warning in grid.Call(C_textBounds, as.graphicsAnnot(x$label), x$x, x$y, :
неизвестна ширина символа 0xee в кодировке CP1251
\end{verbatim}

\begin{verbatim}
Warning in grid.Call(C_textBounds, as.graphicsAnnot(x$label), x$x, x$y, :
неизвестна ширина символа 0xe4 в кодировке CP1251
\end{verbatim}

\begin{verbatim}
Warning in grid.Call.graphics(C_text, as.graphicsAnnot(x$label), x$x, x$y, :
неизвестна ширина символа 0xc3 в кодировке CP1251
\end{verbatim}

\begin{verbatim}
Warning in grid.Call.graphics(C_text, as.graphicsAnnot(x$label), x$x, x$y, :
неизвестна ширина символа 0xee в кодировке CP1251
\end{verbatim}

\begin{verbatim}
Warning in grid.Call.graphics(C_text, as.graphicsAnnot(x$label), x$x, x$y, :
неизвестна ширина символа 0xe4 в кодировке CP1251
\end{verbatim}

\begin{verbatim}
Warning in grid.Call.graphics(C_text, as.graphicsAnnot(x$label), x$x, x$y, :
неизвестна ширина символа 0xd1 в кодировке CP1251
\end{verbatim}

\begin{verbatim}
Warning in grid.Call.graphics(C_text, as.graphicsAnnot(x$label), x$x, x$y, :
неизвестна ширина символа 0xf2 в кодировке CP1251
\end{verbatim}

\begin{verbatim}
Warning in grid.Call.graphics(C_text, as.graphicsAnnot(x$label), x$x, x$y, :
неизвестна ширина символа 0xe0 в кодировке CP1251
\end{verbatim}

\begin{verbatim}
Warning in grid.Call.graphics(C_text, as.graphicsAnnot(x$label), x$x, x$y, :
неизвестна ширина символа 0xed в кодировке CP1251
\end{verbatim}

\begin{verbatim}
Warning in grid.Call.graphics(C_text, as.graphicsAnnot(x$label), x$x, x$y, :
неизвестна ширина символа 0xe4 в кодировке CP1251
\end{verbatim}

\begin{verbatim}
Warning in grid.Call.graphics(C_text, as.graphicsAnnot(x$label), x$x, x$y, :
неизвестна ширина символа 0xe0 в кодировке CP1251
\end{verbatim}

\begin{verbatim}
Warning in grid.Call.graphics(C_text, as.graphicsAnnot(x$label), x$x, x$y, :
неизвестна ширина символа 0xf0 в кодировке CP1251
\end{verbatim}

\begin{verbatim}
Warning in grid.Call.graphics(C_text, as.graphicsAnnot(x$label), x$x, x$y, :
неизвестна ширина символа 0xf2 в кодировке CP1251
\end{verbatim}

\begin{verbatim}
Warning in grid.Call.graphics(C_text, as.graphicsAnnot(x$label), x$x, x$y, :
неизвестна ширина символа 0xe8 в кодировке CP1251
\end{verbatim}

\begin{verbatim}
Warning in grid.Call.graphics(C_text, as.graphicsAnnot(x$label), x$x, x$y, :
неизвестна ширина символа 0xe7 в кодировке CP1251
\end{verbatim}

\begin{verbatim}
Warning in grid.Call.graphics(C_text, as.graphicsAnnot(x$label), x$x, x$y, :
неизвестна ширина символа 0xe8 в кодировке CP1251
\end{verbatim}

\begin{verbatim}
Warning in grid.Call.graphics(C_text, as.graphicsAnnot(x$label), x$x, x$y, :
неизвестна ширина символа 0xf0 в кодировке CP1251
\end{verbatim}

\begin{verbatim}
Warning in grid.Call.graphics(C_text, as.graphicsAnnot(x$label), x$x, x$y, :
неизвестна ширина символа 0xee в кодировке CP1251
\end{verbatim}

\begin{verbatim}
Warning in grid.Call.graphics(C_text, as.graphicsAnnot(x$label), x$x, x$y, :
неизвестна ширина символа 0xe2 в кодировке CP1251
\end{verbatim}

\begin{verbatim}
Warning in grid.Call.graphics(C_text, as.graphicsAnnot(x$label), x$x, x$y, :
неизвестна ширина символа 0xe0 в кодировке CP1251
\end{verbatim}

\begin{verbatim}
Warning in grid.Call.graphics(C_text, as.graphicsAnnot(x$label), x$x, x$y, :
неизвестна ширина символа 0xed в кодировке CP1251
Warning in grid.Call.graphics(C_text, as.graphicsAnnot(x$label), x$x, x$y, :
неизвестна ширина символа 0xed в кодировке CP1251
\end{verbatim}

\begin{verbatim}
Warning in grid.Call.graphics(C_text, as.graphicsAnnot(x$label), x$x, x$y, :
неизвестна ширина символа 0xfb в кодировке CP1251
\end{verbatim}

\begin{verbatim}
Warning in grid.Call.graphics(C_text, as.graphicsAnnot(x$label), x$x, x$y, :
неизвестна ширина символа 0xe9 в кодировке CP1251
\end{verbatim}

\begin{verbatim}
Warning in grid.Call.graphics(C_text, as.graphicsAnnot(x$label), x$x, x$y, :
неизвестна ширина символа 0xe8 в кодировке CP1251
\end{verbatim}

\begin{verbatim}
Warning in grid.Call.graphics(C_text, as.graphicsAnnot(x$label), x$x, x$y, :
неизвестна ширина символа 0xed в кодировке CP1251
\end{verbatim}

\begin{verbatim}
Warning in grid.Call.graphics(C_text, as.graphicsAnnot(x$label), x$x, x$y, :
неизвестна ширина символа 0xe4 в кодировке CP1251
\end{verbatim}

\begin{verbatim}
Warning in grid.Call.graphics(C_text, as.graphicsAnnot(x$label), x$x, x$y, :
неизвестна ширина символа 0xe5 в кодировке CP1251
\end{verbatim}

\begin{verbatim}
Warning in grid.Call.graphics(C_text, as.graphicsAnnot(x$label), x$x, x$y, :
неизвестна ширина символа 0xea в кодировке CP1251
\end{verbatim}

\begin{verbatim}
Warning in grid.Call.graphics(C_text, as.graphicsAnnot(x$label), x$x, x$y, :
неизвестна ширина символа 0xf1 в кодировке CP1251
\end{verbatim}

\begin{verbatim}
Warning in grid.Call.graphics(C_text, as.graphicsAnnot(x$label), x$x, x$y, :
неизвестна ширина символа 0xc8 в кодировке CP1251
\end{verbatim}

\begin{verbatim}
Warning in grid.Call.graphics(C_text, as.graphicsAnnot(x$label), x$x, x$y, :
неизвестна ширина символа 0xed в кодировке CP1251
\end{verbatim}

\begin{verbatim}
Warning in grid.Call.graphics(C_text, as.graphicsAnnot(x$label), x$x, x$y, :
неизвестна ширина символа 0xe4 в кодировке CP1251
\end{verbatim}

\begin{verbatim}
Warning in grid.Call.graphics(C_text, as.graphicsAnnot(x$label), x$x, x$y, :
неизвестна ширина символа 0xe5 в кодировке CP1251
\end{verbatim}

\begin{verbatim}
Warning in grid.Call.graphics(C_text, as.graphicsAnnot(x$label), x$x, x$y, :
неизвестна ширина символа 0xea в кодировке CP1251
\end{verbatim}

\begin{verbatim}
Warning in grid.Call.graphics(C_text, as.graphicsAnnot(x$label), x$x, x$y, :
неизвестна ширина символа 0xf1 в кодировке CP1251
\end{verbatim}

\begin{verbatim}
Warning in grid.Call.graphics(C_text, as.graphicsAnnot(x$label), x$x, x$y, :
неизвестна ширина символа 0xfb в кодировке CP1251
\end{verbatim}

\begin{verbatim}
Warning in grid.Call.graphics(C_text, as.graphicsAnnot(x$label), x$x, x$y, :
неизвестна ширина символа 0xef в кодировке CP1251
\end{verbatim}

\begin{verbatim}
Warning in grid.Call.graphics(C_text, as.graphicsAnnot(x$label), x$x, x$y, :
неизвестна ширина символа 0xee в кодировке CP1251
\end{verbatim}

\begin{verbatim}
Warning in grid.Call.graphics(C_text, as.graphicsAnnot(x$label), x$x, x$y, :
неизвестна ширина символа 0xec в кодировке CP1251
\end{verbatim}

\begin{verbatim}
Warning in grid.Call.graphics(C_text, as.graphicsAnnot(x$label), x$x, x$y, :
неизвестна ширина символа 0xee в кодировке CP1251
\end{verbatim}

\begin{verbatim}
Warning in grid.Call.graphics(C_text, as.graphicsAnnot(x$label), x$x, x$y, :
неизвестна ширина символа 0xe4 в кодировке CP1251
\end{verbatim}

\begin{verbatim}
Warning in grid.Call.graphics(C_text, as.graphicsAnnot(x$label), x$x, x$y, :
неизвестна ширина символа 0xe5 в кодировке CP1251
\end{verbatim}

\begin{verbatim}
Warning in grid.Call.graphics(C_text, as.graphicsAnnot(x$label), x$x, x$y, :
неизвестна ширина символа 0xeb в кодировке CP1251
\end{verbatim}

\begin{verbatim}
Warning in grid.Call.graphics(C_text, as.graphicsAnnot(x$label), x$x, x$y, :
неизвестна ширина символа 0xe8 в кодировке CP1251
\end{verbatim}

\begin{verbatim}
Warning in grid.Call.graphics(C_text, as.graphicsAnnot(x$label), x$x, x$y, :
неизвестна ширина символа 0xea в кодировке CP1251
\end{verbatim}

\begin{verbatim}
Warning in grid.Call.graphics(C_text, as.graphicsAnnot(x$label), x$x, x$y, :
неизвестна ширина символа 0xf0 в кодировке CP1251
\end{verbatim}

\begin{verbatim}
Warning in grid.Call.graphics(C_text, as.graphicsAnnot(x$label), x$x, x$y, :
неизвестна ширина символа 0xe5 в кодировке CP1251
\end{verbatim}

\begin{verbatim}
Warning in grid.Call.graphics(C_text, as.graphicsAnnot(x$label), x$x, x$y, :
неизвестна ширина символа 0xf1 в кодировке CP1251
\end{verbatim}

\begin{verbatim}
Warning in grid.Call.graphics(C_text, as.graphicsAnnot(x$label), x$x, x$y, :
неизвестна ширина символа 0xf2 в кодировке CP1251
\end{verbatim}

\begin{verbatim}
Warning in grid.Call.graphics(C_text, as.graphicsAnnot(x$label), x$x, x$y, :
неизвестна ширина символа 0xe8 в кодировке CP1251
\end{verbatim}

\begin{verbatim}
Warning in grid.Call.graphics(C_text, as.graphicsAnnot(x$label), x$x, x$y, :
неизвестна ширина символа 0xea в кодировке CP1251
\end{verbatim}

\begin{verbatim}
Warning in grid.Call.graphics(C_text, as.graphicsAnnot(x$label), x$x, x$y, :
неизвестна ширина символа 0xe8 в кодировке CP1251
\end{verbatim}

\begin{verbatim}
Warning in grid.Call.graphics(C_text, as.graphicsAnnot(x$label), x$x, x$y, :
неизвестна ширина символа 0xf4 в кодировке CP1251
\end{verbatim}

\begin{verbatim}
Warning in grid.Call.graphics(C_text, as.graphicsAnnot(x$label), x$x, x$y, :
неизвестна ширина символа 0xe0 в кодировке CP1251
\end{verbatim}

\begin{verbatim}
Warning in grid.Call.graphics(C_text, as.graphicsAnnot(x$label), x$x, x$y, :
неизвестна ширина символа 0xea в кодировке CP1251
\end{verbatim}

\begin{verbatim}
Warning in grid.Call.graphics(C_text, as.graphicsAnnot(x$label), x$x, x$y, :
неизвестна ширина символа 0xf2 в кодировке CP1251
\end{verbatim}

\pandocbounded{\includegraphics[keepaspectratio]{chapter9_files/figure-pdf/unnamed-chunk-1-11.pdf}}

\begin{Shaded}
\begin{Highlighting}[]
\CommentTok{\# ==============================================================================}
\CommentTok{\# БЛОК 7: СРАВНЕНИЕ МОДЕЛЕЙ И ФИНАЛЬНАЯ ВИЗУАЛИЗАЦИЯ}
\CommentTok{\# ==============================================================================}

\CommentTok{\# Объединение результатов всех моделей}
\NormalTok{indices\_all }\OtherTok{\textless{}{-}} \FunctionTok{bind\_rows}\NormalTok{(}
\NormalTok{  idx\_glm }\SpecialCharTok{\%\textgreater{}\%} \FunctionTok{select}\NormalTok{(YEAR, value, lcl, ucl, model, index\_mean, lcl\_index\_mean, ucl\_index\_mean),}
\NormalTok{  idx\_gam }\SpecialCharTok{\%\textgreater{}\%} \FunctionTok{select}\NormalTok{(YEAR, value, lcl, ucl, model, index\_mean, lcl\_index\_mean, ucl\_index\_mean),}
\NormalTok{  idx\_gamm }\SpecialCharTok{\%\textgreater{}\%} \FunctionTok{select}\NormalTok{(YEAR, value, lcl, ucl, model, index\_mean, lcl\_index\_mean, ucl\_index\_mean)}
\NormalTok{) }\SpecialCharTok{\%\textgreater{}\%} \FunctionTok{mutate}\NormalTok{(}\AttributeTok{YEAR =} \FunctionTok{factor}\NormalTok{(YEAR, }\AttributeTok{levels =} \FunctionTok{levels}\NormalTok{(DATA}\SpecialCharTok{$}\NormalTok{YEAR)))}

\CommentTok{\# Сводная таблица результатов}
\NormalTok{indices\_all }\SpecialCharTok{\%\textgreater{}\%} 
  \FunctionTok{kable}\NormalTok{(}\AttributeTok{caption =} \StringTok{"Сравнение индексов CPUE по разным моделям"}\NormalTok{)}
\end{Highlighting}
\end{Shaded}

\begin{longtable}[]{@{}
  >{\raggedright\arraybackslash}p{(\linewidth - 14\tabcolsep) * \real{0.0568}}
  >{\raggedleft\arraybackslash}p{(\linewidth - 14\tabcolsep) * \real{0.1136}}
  >{\raggedleft\arraybackslash}p{(\linewidth - 14\tabcolsep) * \real{0.1136}}
  >{\raggedleft\arraybackslash}p{(\linewidth - 14\tabcolsep) * \real{0.1136}}
  >{\raggedright\arraybackslash}p{(\linewidth - 14\tabcolsep) * \real{0.1364}}
  >{\raggedleft\arraybackslash}p{(\linewidth - 14\tabcolsep) * \real{0.1250}}
  >{\raggedleft\arraybackslash}p{(\linewidth - 14\tabcolsep) * \real{0.1705}}
  >{\raggedleft\arraybackslash}p{(\linewidth - 14\tabcolsep) * \real{0.1705}}@{}}
\caption{Сравнение индексов CPUE по разным моделям}\tabularnewline
\toprule\noalign{}
\begin{minipage}[b]{\linewidth}\raggedright
YEAR
\end{minipage} & \begin{minipage}[b]{\linewidth}\raggedleft
value
\end{minipage} & \begin{minipage}[b]{\linewidth}\raggedleft
lcl
\end{minipage} & \begin{minipage}[b]{\linewidth}\raggedleft
ucl
\end{minipage} & \begin{minipage}[b]{\linewidth}\raggedright
model
\end{minipage} & \begin{minipage}[b]{\linewidth}\raggedleft
index\_mean
\end{minipage} & \begin{minipage}[b]{\linewidth}\raggedleft
lcl\_index\_mean
\end{minipage} & \begin{minipage}[b]{\linewidth}\raggedleft
ucl\_index\_mean
\end{minipage} \\
\midrule\noalign{}
\endfirsthead
\toprule\noalign{}
\begin{minipage}[b]{\linewidth}\raggedright
YEAR
\end{minipage} & \begin{minipage}[b]{\linewidth}\raggedleft
value
\end{minipage} & \begin{minipage}[b]{\linewidth}\raggedleft
lcl
\end{minipage} & \begin{minipage}[b]{\linewidth}\raggedleft
ucl
\end{minipage} & \begin{minipage}[b]{\linewidth}\raggedright
model
\end{minipage} & \begin{minipage}[b]{\linewidth}\raggedleft
index\_mean
\end{minipage} & \begin{minipage}[b]{\linewidth}\raggedleft
lcl\_index\_mean
\end{minipage} & \begin{minipage}[b]{\linewidth}\raggedleft
ucl\_index\_mean
\end{minipage} \\
\midrule\noalign{}
\endhead
\bottomrule\noalign{}
\endlastfoot
2019 & 158.81216 & 138.93290 & 181.53585 & GLM\_Gamma & 1.5358638 &
1.5299542 & 1.5417869 \\
2020 & 126.54457 & 111.15088 & 144.07018 & GLM\_Gamma & 1.2238057 &
1.2240136 & 1.2235904 \\
2021 & 126.89292 & 111.57409 & 144.31498 & GLM\_Gamma & 1.2271746 &
1.2286741 & 1.2256695 \\
2022 & 83.46580 & 73.52525 & 94.75032 & GLM\_Gamma & 0.8071933 &
0.8096733 & 0.8047160 \\
2023 & 73.30390 & 64.40714 & 83.42960 & GLM\_Gamma & 0.7089181 &
0.7092631 & 0.7085690 \\
2024 & 51.39565 & 45.26094 & 58.36186 & GLM\_Gamma & 0.4970445 &
0.4984216 & 0.4956683 \\
2019 & 158.81218 & 138.93304 & 181.53572 & GAM & 1.5358554 & 1.5299458 &
1.5417784 \\
2020 & 126.54503 & 111.15138 & 144.07058 & GAM & 1.2238033 & 1.2240112 &
1.2235880 \\
2021 & 126.89280 & 111.57408 & 144.31473 & GAM & 1.2271665 & 1.2286660 &
1.2256615 \\
2022 & 83.46561 & 73.52514 & 94.75002 & GAM & 0.8071869 & 0.8096669 &
0.8047096 \\
2023 & 73.30495 & 64.40811 & 83.43072 & GAM & 0.7089242 & 0.7092692 &
0.7085751 \\
2024 & 51.39792 & 45.26297 & 58.36439 & GAM & 0.4970637 & 0.4984408 &
0.4956874 \\
2019 & 165.88930 & 153.27889 & 186.43794 & GAMM (mgcv) & 1.5400976 &
1.5751123 & 1.5695052 \\
2020 & 131.59142 & 116.54705 & 144.02703 & GAMM (mgcv) & 1.2216800 &
1.1976514 & 1.2124741 \\
2021 & 132.07110 & 118.64017 & 145.21926 & GAMM (mgcv) & 1.2261333 &
1.2191606 & 1.2225107 \\
2022 & 86.80692 & 79.26688 & 96.71028 & GAMM (mgcv) & 0.8059057 &
0.8145559 & 0.8141438 \\
2023 & 76.37202 & 67.94329 & 82.24327 & GAMM (mgcv) & 0.7090292 &
0.6981933 & 0.6923550 \\
2024 & 53.55022 & 48.20170 & 58.08852 & GAMM (mgcv) & 0.4971542 &
0.4953264 & 0.4890111 \\
\end{longtable}

\begin{Shaded}
\begin{Highlighting}[]
\NormalTok{indices\_all }\SpecialCharTok{\%\textgreater{}\%}
  \FunctionTok{ggplot}\NormalTok{(}\FunctionTok{aes}\NormalTok{(}\AttributeTok{x =}\NormalTok{ YEAR, }\AttributeTok{y =}\NormalTok{ value, }\AttributeTok{color =}\NormalTok{ model, }\AttributeTok{group =}\NormalTok{ model, }\AttributeTok{fill =}\NormalTok{ model)) }\SpecialCharTok{+}
  \FunctionTok{geom\_line}\NormalTok{() }\SpecialCharTok{+}
  \FunctionTok{geom\_point}\NormalTok{() }\SpecialCharTok{+}
  \FunctionTok{geom\_ribbon}\NormalTok{(}\FunctionTok{aes}\NormalTok{(}\AttributeTok{ymin =}\NormalTok{ lcl, }\AttributeTok{ymax =}\NormalTok{ ucl), }\AttributeTok{alpha =} \FloatTok{0.1}\NormalTok{, }\AttributeTok{linetype =} \StringTok{"dashed"}\NormalTok{) }\SpecialCharTok{+}
\FunctionTok{geom\_point}\NormalTok{(}\AttributeTok{data =}\NormalTok{ actual\_medians, }
           \FunctionTok{aes}\NormalTok{(}\AttributeTok{x =}\NormalTok{ YEAR, }\AttributeTok{y =}\NormalTok{ median\_cpue), }
           \AttributeTok{shape =} \DecValTok{4}\NormalTok{,  }\CommentTok{\# 4 соответствует крестику (x)}
           \AttributeTok{size =} \DecValTok{3}\NormalTok{, }
           \AttributeTok{color =} \StringTok{"black"}\NormalTok{, }
           \AttributeTok{inherit.aes =} \ConstantTok{FALSE}\NormalTok{)}\SpecialCharTok{+}
  \FunctionTok{labs}\NormalTok{(}\AttributeTok{title =} \StringTok{"Сравнение стандартизированных индексов CPUE (крестики {-} факт)"}\NormalTok{, }
       \AttributeTok{x =} \StringTok{"Год"}\NormalTok{, }
       \AttributeTok{y =} \StringTok{"Индекс CPUE (кг/ловушку)"}\NormalTok{, }
       \AttributeTok{color =} \StringTok{"Модель"}\NormalTok{, }
       \AttributeTok{fill =} \StringTok{"Модель"}\NormalTok{) }\SpecialCharTok{+}
  \FunctionTok{theme}\NormalTok{(}\AttributeTok{legend.position =} \StringTok{"bottom"}\NormalTok{)}
\end{Highlighting}
\end{Shaded}

\begin{verbatim}
Warning in grid.Call(C_textBounds, as.graphicsAnnot(x$label), x$x, x$y, :
неизвестна ширина символа 0xcc в кодировке CP1251
\end{verbatim}

\begin{verbatim}
Warning in grid.Call(C_textBounds, as.graphicsAnnot(x$label), x$x, x$y, :
неизвестна ширина символа 0xee в кодировке CP1251
\end{verbatim}

\begin{verbatim}
Warning in grid.Call(C_textBounds, as.graphicsAnnot(x$label), x$x, x$y, :
неизвестна ширина символа 0xe4 в кодировке CP1251
\end{verbatim}

\begin{verbatim}
Warning in grid.Call(C_textBounds, as.graphicsAnnot(x$label), x$x, x$y, :
неизвестна ширина символа 0xe5 в кодировке CP1251
\end{verbatim}

\begin{verbatim}
Warning in grid.Call(C_textBounds, as.graphicsAnnot(x$label), x$x, x$y, :
неизвестна ширина символа 0xeb в кодировке CP1251
\end{verbatim}

\begin{verbatim}
Warning in grid.Call(C_textBounds, as.graphicsAnnot(x$label), x$x, x$y, :
неизвестна ширина символа 0xfc в кодировке CP1251
\end{verbatim}

\begin{verbatim}
Warning in grid.Call(C_textBounds, as.graphicsAnnot(x$label), x$x, x$y, :
неизвестна ширина символа 0xcc в кодировке CP1251
\end{verbatim}

\begin{verbatim}
Warning in grid.Call(C_textBounds, as.graphicsAnnot(x$label), x$x, x$y, :
неизвестна ширина символа 0xee в кодировке CP1251
\end{verbatim}

\begin{verbatim}
Warning in grid.Call(C_textBounds, as.graphicsAnnot(x$label), x$x, x$y, :
неизвестна ширина символа 0xe4 в кодировке CP1251
\end{verbatim}

\begin{verbatim}
Warning in grid.Call(C_textBounds, as.graphicsAnnot(x$label), x$x, x$y, :
неизвестна ширина символа 0xe5 в кодировке CP1251
\end{verbatim}

\begin{verbatim}
Warning in grid.Call(C_textBounds, as.graphicsAnnot(x$label), x$x, x$y, :
неизвестна ширина символа 0xeb в кодировке CP1251
\end{verbatim}

\begin{verbatim}
Warning in grid.Call(C_textBounds, as.graphicsAnnot(x$label), x$x, x$y, :
неизвестна ширина символа 0xfc в кодировке CP1251
\end{verbatim}

\begin{verbatim}
Warning in grid.Call(C_textBounds, as.graphicsAnnot(x$label), x$x, x$y, :
неизвестна ширина символа 0xc8 в кодировке CP1251
\end{verbatim}

\begin{verbatim}
Warning in grid.Call(C_textBounds, as.graphicsAnnot(x$label), x$x, x$y, :
неизвестна ширина символа 0xed в кодировке CP1251
\end{verbatim}

\begin{verbatim}
Warning in grid.Call(C_textBounds, as.graphicsAnnot(x$label), x$x, x$y, :
неизвестна ширина символа 0xe4 в кодировке CP1251
\end{verbatim}

\begin{verbatim}
Warning in grid.Call(C_textBounds, as.graphicsAnnot(x$label), x$x, x$y, :
неизвестна ширина символа 0xe5 в кодировке CP1251
\end{verbatim}

\begin{verbatim}
Warning in grid.Call(C_textBounds, as.graphicsAnnot(x$label), x$x, x$y, :
неизвестна ширина символа 0xea в кодировке CP1251
\end{verbatim}

\begin{verbatim}
Warning in grid.Call(C_textBounds, as.graphicsAnnot(x$label), x$x, x$y, :
неизвестна ширина символа 0xf1 в кодировке CP1251
\end{verbatim}

\begin{verbatim}
Warning in grid.Call(C_textBounds, as.graphicsAnnot(x$label), x$x, x$y, :
неизвестна ширина символа 0xea в кодировке CP1251
\end{verbatim}

\begin{verbatim}
Warning in grid.Call(C_textBounds, as.graphicsAnnot(x$label), x$x, x$y, :
неизвестна ширина символа 0xe3 в кодировке CP1251
\end{verbatim}

\begin{verbatim}
Warning in grid.Call(C_textBounds, as.graphicsAnnot(x$label), x$x, x$y, :
неизвестна ширина символа 0xeb в кодировке CP1251
\end{verbatim}

\begin{verbatim}
Warning in grid.Call(C_textBounds, as.graphicsAnnot(x$label), x$x, x$y, :
неизвестна ширина символа 0xee в кодировке CP1251
\end{verbatim}

\begin{verbatim}
Warning in grid.Call(C_textBounds, as.graphicsAnnot(x$label), x$x, x$y, :
неизвестна ширина символа 0xe2 в кодировке CP1251
\end{verbatim}

\begin{verbatim}
Warning in grid.Call(C_textBounds, as.graphicsAnnot(x$label), x$x, x$y, :
неизвестна ширина символа 0xf3 в кодировке CP1251
\end{verbatim}

\begin{verbatim}
Warning in grid.Call(C_textBounds, as.graphicsAnnot(x$label), x$x, x$y, :
неизвестна ширина символа 0xf8 в кодировке CP1251
\end{verbatim}

\begin{verbatim}
Warning in grid.Call(C_textBounds, as.graphicsAnnot(x$label), x$x, x$y, :
неизвестна ширина символа 0xea в кодировке CP1251
\end{verbatim}

\begin{verbatim}
Warning in grid.Call(C_textBounds, as.graphicsAnnot(x$label), x$x, x$y, :
неизвестна ширина символа 0xf3 в кодировке CP1251
\end{verbatim}

\begin{verbatim}
Warning in grid.Call(C_textBounds, as.graphicsAnnot(x$label), x$x, x$y, :
неизвестна ширина символа 0xd1 в кодировке CP1251
\end{verbatim}

\begin{verbatim}
Warning in grid.Call(C_textBounds, as.graphicsAnnot(x$label), x$x, x$y, :
неизвестна ширина символа 0xf0 в кодировке CP1251
\end{verbatim}

\begin{verbatim}
Warning in grid.Call(C_textBounds, as.graphicsAnnot(x$label), x$x, x$y, :
неизвестна ширина символа 0xe0 в кодировке CP1251
\end{verbatim}

\begin{verbatim}
Warning in grid.Call(C_textBounds, as.graphicsAnnot(x$label), x$x, x$y, :
неизвестна ширина символа 0xe2 в кодировке CP1251
\end{verbatim}

\begin{verbatim}
Warning in grid.Call(C_textBounds, as.graphicsAnnot(x$label), x$x, x$y, :
неизвестна ширина символа 0xed в кодировке CP1251
\end{verbatim}

\begin{verbatim}
Warning in grid.Call(C_textBounds, as.graphicsAnnot(x$label), x$x, x$y, :
неизвестна ширина символа 0xe5 в кодировке CP1251
\end{verbatim}

\begin{verbatim}
Warning in grid.Call(C_textBounds, as.graphicsAnnot(x$label), x$x, x$y, :
неизвестна ширина символа 0xed в кодировке CP1251
\end{verbatim}

\begin{verbatim}
Warning in grid.Call(C_textBounds, as.graphicsAnnot(x$label), x$x, x$y, :
неизвестна ширина символа 0xe8 в кодировке CP1251
\end{verbatim}

\begin{verbatim}
Warning in grid.Call(C_textBounds, as.graphicsAnnot(x$label), x$x, x$y, :
неизвестна ширина символа 0xe5 в кодировке CP1251
\end{verbatim}

\begin{verbatim}
Warning in grid.Call(C_textBounds, as.graphicsAnnot(x$label), x$x, x$y, :
неизвестна ширина символа 0xf1 в кодировке CP1251
\end{verbatim}

\begin{verbatim}
Warning in grid.Call(C_textBounds, as.graphicsAnnot(x$label), x$x, x$y, :
неизвестна ширина символа 0xf2 в кодировке CP1251
\end{verbatim}

\begin{verbatim}
Warning in grid.Call(C_textBounds, as.graphicsAnnot(x$label), x$x, x$y, :
неизвестна ширина символа 0xe0 в кодировке CP1251
\end{verbatim}

\begin{verbatim}
Warning in grid.Call(C_textBounds, as.graphicsAnnot(x$label), x$x, x$y, :
неизвестна ширина символа 0xed в кодировке CP1251
\end{verbatim}

\begin{verbatim}
Warning in grid.Call(C_textBounds, as.graphicsAnnot(x$label), x$x, x$y, :
неизвестна ширина символа 0xe4 в кодировке CP1251
\end{verbatim}

\begin{verbatim}
Warning in grid.Call(C_textBounds, as.graphicsAnnot(x$label), x$x, x$y, :
неизвестна ширина символа 0xe0 в кодировке CP1251
\end{verbatim}

\begin{verbatim}
Warning in grid.Call(C_textBounds, as.graphicsAnnot(x$label), x$x, x$y, :
неизвестна ширина символа 0xf0 в кодировке CP1251
\end{verbatim}

\begin{verbatim}
Warning in grid.Call(C_textBounds, as.graphicsAnnot(x$label), x$x, x$y, :
неизвестна ширина символа 0xf2 в кодировке CP1251
\end{verbatim}

\begin{verbatim}
Warning in grid.Call(C_textBounds, as.graphicsAnnot(x$label), x$x, x$y, :
неизвестна ширина символа 0xe8 в кодировке CP1251
\end{verbatim}

\begin{verbatim}
Warning in grid.Call(C_textBounds, as.graphicsAnnot(x$label), x$x, x$y, :
неизвестна ширина символа 0xe7 в кодировке CP1251
\end{verbatim}

\begin{verbatim}
Warning in grid.Call(C_textBounds, as.graphicsAnnot(x$label), x$x, x$y, :
неизвестна ширина символа 0xe8 в кодировке CP1251
\end{verbatim}

\begin{verbatim}
Warning in grid.Call(C_textBounds, as.graphicsAnnot(x$label), x$x, x$y, :
неизвестна ширина символа 0xf0 в кодировке CP1251
\end{verbatim}

\begin{verbatim}
Warning in grid.Call(C_textBounds, as.graphicsAnnot(x$label), x$x, x$y, :
неизвестна ширина символа 0xee в кодировке CP1251
\end{verbatim}

\begin{verbatim}
Warning in grid.Call(C_textBounds, as.graphicsAnnot(x$label), x$x, x$y, :
неизвестна ширина символа 0xe2 в кодировке CP1251
\end{verbatim}

\begin{verbatim}
Warning in grid.Call(C_textBounds, as.graphicsAnnot(x$label), x$x, x$y, :
неизвестна ширина символа 0xe0 в кодировке CP1251
\end{verbatim}

\begin{verbatim}
Warning in grid.Call(C_textBounds, as.graphicsAnnot(x$label), x$x, x$y, :
неизвестна ширина символа 0xed в кодировке CP1251
Warning in grid.Call(C_textBounds, as.graphicsAnnot(x$label), x$x, x$y, :
неизвестна ширина символа 0xed в кодировке CP1251
\end{verbatim}

\begin{verbatim}
Warning in grid.Call(C_textBounds, as.graphicsAnnot(x$label), x$x, x$y, :
неизвестна ширина символа 0xfb в кодировке CP1251
\end{verbatim}

\begin{verbatim}
Warning in grid.Call(C_textBounds, as.graphicsAnnot(x$label), x$x, x$y, :
неизвестна ширина символа 0xf5 в кодировке CP1251
\end{verbatim}

\begin{verbatim}
Warning in grid.Call(C_textBounds, as.graphicsAnnot(x$label), x$x, x$y, :
неизвестна ширина символа 0xe8 в кодировке CP1251
\end{verbatim}

\begin{verbatim}
Warning in grid.Call(C_textBounds, as.graphicsAnnot(x$label), x$x, x$y, :
неизвестна ширина символа 0xed в кодировке CP1251
\end{verbatim}

\begin{verbatim}
Warning in grid.Call(C_textBounds, as.graphicsAnnot(x$label), x$x, x$y, :
неизвестна ширина символа 0xe4 в кодировке CP1251
\end{verbatim}

\begin{verbatim}
Warning in grid.Call(C_textBounds, as.graphicsAnnot(x$label), x$x, x$y, :
неизвестна ширина символа 0xe5 в кодировке CP1251
\end{verbatim}

\begin{verbatim}
Warning in grid.Call(C_textBounds, as.graphicsAnnot(x$label), x$x, x$y, :
неизвестна ширина символа 0xea в кодировке CP1251
\end{verbatim}

\begin{verbatim}
Warning in grid.Call(C_textBounds, as.graphicsAnnot(x$label), x$x, x$y, :
неизвестна ширина символа 0xf1 в кодировке CP1251
\end{verbatim}

\begin{verbatim}
Warning in grid.Call(C_textBounds, as.graphicsAnnot(x$label), x$x, x$y, :
неизвестна ширина символа 0xee в кодировке CP1251
\end{verbatim}

\begin{verbatim}
Warning in grid.Call(C_textBounds, as.graphicsAnnot(x$label), x$x, x$y, :
неизвестна ширина символа 0xe2 в кодировке CP1251
\end{verbatim}

\begin{verbatim}
Warning in grid.Call(C_textBounds, as.graphicsAnnot(x$label), x$x, x$y, :
неизвестна ширина символа 0xea в кодировке CP1251
\end{verbatim}

\begin{verbatim}
Warning in grid.Call(C_textBounds, as.graphicsAnnot(x$label), x$x, x$y, :
неизвестна ширина символа 0xf0 в кодировке CP1251
\end{verbatim}

\begin{verbatim}
Warning in grid.Call(C_textBounds, as.graphicsAnnot(x$label), x$x, x$y, :
неизвестна ширина символа 0xe5 в кодировке CP1251
\end{verbatim}

\begin{verbatim}
Warning in grid.Call(C_textBounds, as.graphicsAnnot(x$label), x$x, x$y, :
неизвестна ширина символа 0xf1 в кодировке CP1251
\end{verbatim}

\begin{verbatim}
Warning in grid.Call(C_textBounds, as.graphicsAnnot(x$label), x$x, x$y, :
неизвестна ширина символа 0xf2 в кодировке CP1251
\end{verbatim}

\begin{verbatim}
Warning in grid.Call(C_textBounds, as.graphicsAnnot(x$label), x$x, x$y, :
неизвестна ширина символа 0xe8 в кодировке CP1251
\end{verbatim}

\begin{verbatim}
Warning in grid.Call(C_textBounds, as.graphicsAnnot(x$label), x$x, x$y, :
неизвестна ширина символа 0xea в кодировке CP1251
\end{verbatim}

\begin{verbatim}
Warning in grid.Call(C_textBounds, as.graphicsAnnot(x$label), x$x, x$y, :
неизвестна ширина символа 0xe8 в кодировке CP1251
\end{verbatim}

\begin{verbatim}
Warning in grid.Call(C_textBounds, as.graphicsAnnot(x$label), x$x, x$y, :
неизвестна ширина символа 0xf4 в кодировке CP1251
\end{verbatim}

\begin{verbatim}
Warning in grid.Call(C_textBounds, as.graphicsAnnot(x$label), x$x, x$y, :
неизвестна ширина символа 0xe0 в кодировке CP1251
\end{verbatim}

\begin{verbatim}
Warning in grid.Call(C_textBounds, as.graphicsAnnot(x$label), x$x, x$y, :
неизвестна ширина символа 0xea в кодировке CP1251
\end{verbatim}

\begin{verbatim}
Warning in grid.Call(C_textBounds, as.graphicsAnnot(x$label), x$x, x$y, :
неизвестна ширина символа 0xf2 в кодировке CP1251
\end{verbatim}

\begin{verbatim}
Warning in grid.Call(C_textBounds, as.graphicsAnnot(x$label), x$x, x$y, :
неизвестна ширина символа 0xc3 в кодировке CP1251
\end{verbatim}

\begin{verbatim}
Warning in grid.Call(C_textBounds, as.graphicsAnnot(x$label), x$x, x$y, :
неизвестна ширина символа 0xee в кодировке CP1251
\end{verbatim}

\begin{verbatim}
Warning in grid.Call(C_textBounds, as.graphicsAnnot(x$label), x$x, x$y, :
неизвестна ширина символа 0xe4 в кодировке CP1251
\end{verbatim}

\begin{verbatim}
Warning in grid.Call.graphics(C_text, as.graphicsAnnot(x$label), x$x, x$y, :
неизвестна ширина символа 0xc3 в кодировке CP1251
\end{verbatim}

\begin{verbatim}
Warning in grid.Call.graphics(C_text, as.graphicsAnnot(x$label), x$x, x$y, :
неизвестна ширина символа 0xee в кодировке CP1251
\end{verbatim}

\begin{verbatim}
Warning in grid.Call.graphics(C_text, as.graphicsAnnot(x$label), x$x, x$y, :
неизвестна ширина символа 0xe4 в кодировке CP1251
\end{verbatim}

\begin{verbatim}
Warning in grid.Call.graphics(C_text, as.graphicsAnnot(x$label), x$x, x$y, :
неизвестна ширина символа 0xc8 в кодировке CP1251
\end{verbatim}

\begin{verbatim}
Warning in grid.Call.graphics(C_text, as.graphicsAnnot(x$label), x$x, x$y, :
неизвестна ширина символа 0xed в кодировке CP1251
\end{verbatim}

\begin{verbatim}
Warning in grid.Call.graphics(C_text, as.graphicsAnnot(x$label), x$x, x$y, :
неизвестна ширина символа 0xe4 в кодировке CP1251
\end{verbatim}

\begin{verbatim}
Warning in grid.Call.graphics(C_text, as.graphicsAnnot(x$label), x$x, x$y, :
неизвестна ширина символа 0xe5 в кодировке CP1251
\end{verbatim}

\begin{verbatim}
Warning in grid.Call.graphics(C_text, as.graphicsAnnot(x$label), x$x, x$y, :
неизвестна ширина символа 0xea в кодировке CP1251
\end{verbatim}

\begin{verbatim}
Warning in grid.Call.graphics(C_text, as.graphicsAnnot(x$label), x$x, x$y, :
неизвестна ширина символа 0xf1 в кодировке CP1251
\end{verbatim}

\begin{verbatim}
Warning in grid.Call.graphics(C_text, as.graphicsAnnot(x$label), x$x, x$y, :
неизвестна ширина символа 0xea в кодировке CP1251
\end{verbatim}

\begin{verbatim}
Warning in grid.Call.graphics(C_text, as.graphicsAnnot(x$label), x$x, x$y, :
неизвестна ширина символа 0xe3 в кодировке CP1251
\end{verbatim}

\begin{verbatim}
Warning in grid.Call.graphics(C_text, as.graphicsAnnot(x$label), x$x, x$y, :
неизвестна ширина символа 0xeb в кодировке CP1251
\end{verbatim}

\begin{verbatim}
Warning in grid.Call.graphics(C_text, as.graphicsAnnot(x$label), x$x, x$y, :
неизвестна ширина символа 0xee в кодировке CP1251
\end{verbatim}

\begin{verbatim}
Warning in grid.Call.graphics(C_text, as.graphicsAnnot(x$label), x$x, x$y, :
неизвестна ширина символа 0xe2 в кодировке CP1251
\end{verbatim}

\begin{verbatim}
Warning in grid.Call.graphics(C_text, as.graphicsAnnot(x$label), x$x, x$y, :
неизвестна ширина символа 0xf3 в кодировке CP1251
\end{verbatim}

\begin{verbatim}
Warning in grid.Call.graphics(C_text, as.graphicsAnnot(x$label), x$x, x$y, :
неизвестна ширина символа 0xf8 в кодировке CP1251
\end{verbatim}

\begin{verbatim}
Warning in grid.Call.graphics(C_text, as.graphicsAnnot(x$label), x$x, x$y, :
неизвестна ширина символа 0xea в кодировке CP1251
\end{verbatim}

\begin{verbatim}
Warning in grid.Call.graphics(C_text, as.graphicsAnnot(x$label), x$x, x$y, :
неизвестна ширина символа 0xf3 в кодировке CP1251
\end{verbatim}

\begin{verbatim}
Warning in grid.Call.graphics(C_text, as.graphicsAnnot(x$label), x$x, x$y, :
неизвестна ширина символа 0xcc в кодировке CP1251
\end{verbatim}

\begin{verbatim}
Warning in grid.Call.graphics(C_text, as.graphicsAnnot(x$label), x$x, x$y, :
неизвестна ширина символа 0xee в кодировке CP1251
\end{verbatim}

\begin{verbatim}
Warning in grid.Call.graphics(C_text, as.graphicsAnnot(x$label), x$x, x$y, :
неизвестна ширина символа 0xe4 в кодировке CP1251
\end{verbatim}

\begin{verbatim}
Warning in grid.Call.graphics(C_text, as.graphicsAnnot(x$label), x$x, x$y, :
неизвестна ширина символа 0xe5 в кодировке CP1251
\end{verbatim}

\begin{verbatim}
Warning in grid.Call.graphics(C_text, as.graphicsAnnot(x$label), x$x, x$y, :
неизвестна ширина символа 0xeb в кодировке CP1251
\end{verbatim}

\begin{verbatim}
Warning in grid.Call.graphics(C_text, as.graphicsAnnot(x$label), x$x, x$y, :
неизвестна ширина символа 0xfc в кодировке CP1251
\end{verbatim}

\begin{verbatim}
Warning in grid.Call.graphics(C_text, as.graphicsAnnot(x$label), x$x, x$y, :
неизвестна ширина символа 0xd1 в кодировке CP1251
\end{verbatim}

\begin{verbatim}
Warning in grid.Call.graphics(C_text, as.graphicsAnnot(x$label), x$x, x$y, :
неизвестна ширина символа 0xf0 в кодировке CP1251
\end{verbatim}

\begin{verbatim}
Warning in grid.Call.graphics(C_text, as.graphicsAnnot(x$label), x$x, x$y, :
неизвестна ширина символа 0xe0 в кодировке CP1251
\end{verbatim}

\begin{verbatim}
Warning in grid.Call.graphics(C_text, as.graphicsAnnot(x$label), x$x, x$y, :
неизвестна ширина символа 0xe2 в кодировке CP1251
\end{verbatim}

\begin{verbatim}
Warning in grid.Call.graphics(C_text, as.graphicsAnnot(x$label), x$x, x$y, :
неизвестна ширина символа 0xed в кодировке CP1251
\end{verbatim}

\begin{verbatim}
Warning in grid.Call.graphics(C_text, as.graphicsAnnot(x$label), x$x, x$y, :
неизвестна ширина символа 0xe5 в кодировке CP1251
\end{verbatim}

\begin{verbatim}
Warning in grid.Call.graphics(C_text, as.graphicsAnnot(x$label), x$x, x$y, :
неизвестна ширина символа 0xed в кодировке CP1251
\end{verbatim}

\begin{verbatim}
Warning in grid.Call.graphics(C_text, as.graphicsAnnot(x$label), x$x, x$y, :
неизвестна ширина символа 0xe8 в кодировке CP1251
\end{verbatim}

\begin{verbatim}
Warning in grid.Call.graphics(C_text, as.graphicsAnnot(x$label), x$x, x$y, :
неизвестна ширина символа 0xe5 в кодировке CP1251
\end{verbatim}

\begin{verbatim}
Warning in grid.Call.graphics(C_text, as.graphicsAnnot(x$label), x$x, x$y, :
неизвестна ширина символа 0xf1 в кодировке CP1251
\end{verbatim}

\begin{verbatim}
Warning in grid.Call.graphics(C_text, as.graphicsAnnot(x$label), x$x, x$y, :
неизвестна ширина символа 0xf2 в кодировке CP1251
\end{verbatim}

\begin{verbatim}
Warning in grid.Call.graphics(C_text, as.graphicsAnnot(x$label), x$x, x$y, :
неизвестна ширина символа 0xe0 в кодировке CP1251
\end{verbatim}

\begin{verbatim}
Warning in grid.Call.graphics(C_text, as.graphicsAnnot(x$label), x$x, x$y, :
неизвестна ширина символа 0xed в кодировке CP1251
\end{verbatim}

\begin{verbatim}
Warning in grid.Call.graphics(C_text, as.graphicsAnnot(x$label), x$x, x$y, :
неизвестна ширина символа 0xe4 в кодировке CP1251
\end{verbatim}

\begin{verbatim}
Warning in grid.Call.graphics(C_text, as.graphicsAnnot(x$label), x$x, x$y, :
неизвестна ширина символа 0xe0 в кодировке CP1251
\end{verbatim}

\begin{verbatim}
Warning in grid.Call.graphics(C_text, as.graphicsAnnot(x$label), x$x, x$y, :
неизвестна ширина символа 0xf0 в кодировке CP1251
\end{verbatim}

\begin{verbatim}
Warning in grid.Call.graphics(C_text, as.graphicsAnnot(x$label), x$x, x$y, :
неизвестна ширина символа 0xf2 в кодировке CP1251
\end{verbatim}

\begin{verbatim}
Warning in grid.Call.graphics(C_text, as.graphicsAnnot(x$label), x$x, x$y, :
неизвестна ширина символа 0xe8 в кодировке CP1251
\end{verbatim}

\begin{verbatim}
Warning in grid.Call.graphics(C_text, as.graphicsAnnot(x$label), x$x, x$y, :
неизвестна ширина символа 0xe7 в кодировке CP1251
\end{verbatim}

\begin{verbatim}
Warning in grid.Call.graphics(C_text, as.graphicsAnnot(x$label), x$x, x$y, :
неизвестна ширина символа 0xe8 в кодировке CP1251
\end{verbatim}

\begin{verbatim}
Warning in grid.Call.graphics(C_text, as.graphicsAnnot(x$label), x$x, x$y, :
неизвестна ширина символа 0xf0 в кодировке CP1251
\end{verbatim}

\begin{verbatim}
Warning in grid.Call.graphics(C_text, as.graphicsAnnot(x$label), x$x, x$y, :
неизвестна ширина символа 0xee в кодировке CP1251
\end{verbatim}

\begin{verbatim}
Warning in grid.Call.graphics(C_text, as.graphicsAnnot(x$label), x$x, x$y, :
неизвестна ширина символа 0xe2 в кодировке CP1251
\end{verbatim}

\begin{verbatim}
Warning in grid.Call.graphics(C_text, as.graphicsAnnot(x$label), x$x, x$y, :
неизвестна ширина символа 0xe0 в кодировке CP1251
\end{verbatim}

\begin{verbatim}
Warning in grid.Call.graphics(C_text, as.graphicsAnnot(x$label), x$x, x$y, :
неизвестна ширина символа 0xed в кодировке CP1251
Warning in grid.Call.graphics(C_text, as.graphicsAnnot(x$label), x$x, x$y, :
неизвестна ширина символа 0xed в кодировке CP1251
\end{verbatim}

\begin{verbatim}
Warning in grid.Call.graphics(C_text, as.graphicsAnnot(x$label), x$x, x$y, :
неизвестна ширина символа 0xfb в кодировке CP1251
\end{verbatim}

\begin{verbatim}
Warning in grid.Call.graphics(C_text, as.graphicsAnnot(x$label), x$x, x$y, :
неизвестна ширина символа 0xf5 в кодировке CP1251
\end{verbatim}

\begin{verbatim}
Warning in grid.Call.graphics(C_text, as.graphicsAnnot(x$label), x$x, x$y, :
неизвестна ширина символа 0xe8 в кодировке CP1251
\end{verbatim}

\begin{verbatim}
Warning in grid.Call.graphics(C_text, as.graphicsAnnot(x$label), x$x, x$y, :
неизвестна ширина символа 0xed в кодировке CP1251
\end{verbatim}

\begin{verbatim}
Warning in grid.Call.graphics(C_text, as.graphicsAnnot(x$label), x$x, x$y, :
неизвестна ширина символа 0xe4 в кодировке CP1251
\end{verbatim}

\begin{verbatim}
Warning in grid.Call.graphics(C_text, as.graphicsAnnot(x$label), x$x, x$y, :
неизвестна ширина символа 0xe5 в кодировке CP1251
\end{verbatim}

\begin{verbatim}
Warning in grid.Call.graphics(C_text, as.graphicsAnnot(x$label), x$x, x$y, :
неизвестна ширина символа 0xea в кодировке CP1251
\end{verbatim}

\begin{verbatim}
Warning in grid.Call.graphics(C_text, as.graphicsAnnot(x$label), x$x, x$y, :
неизвестна ширина символа 0xf1 в кодировке CP1251
\end{verbatim}

\begin{verbatim}
Warning in grid.Call.graphics(C_text, as.graphicsAnnot(x$label), x$x, x$y, :
неизвестна ширина символа 0xee в кодировке CP1251
\end{verbatim}

\begin{verbatim}
Warning in grid.Call.graphics(C_text, as.graphicsAnnot(x$label), x$x, x$y, :
неизвестна ширина символа 0xe2 в кодировке CP1251
\end{verbatim}

\begin{verbatim}
Warning in grid.Call.graphics(C_text, as.graphicsAnnot(x$label), x$x, x$y, :
неизвестна ширина символа 0xea в кодировке CP1251
\end{verbatim}

\begin{verbatim}
Warning in grid.Call.graphics(C_text, as.graphicsAnnot(x$label), x$x, x$y, :
неизвестна ширина символа 0xf0 в кодировке CP1251
\end{verbatim}

\begin{verbatim}
Warning in grid.Call.graphics(C_text, as.graphicsAnnot(x$label), x$x, x$y, :
неизвестна ширина символа 0xe5 в кодировке CP1251
\end{verbatim}

\begin{verbatim}
Warning in grid.Call.graphics(C_text, as.graphicsAnnot(x$label), x$x, x$y, :
неизвестна ширина символа 0xf1 в кодировке CP1251
\end{verbatim}

\begin{verbatim}
Warning in grid.Call.graphics(C_text, as.graphicsAnnot(x$label), x$x, x$y, :
неизвестна ширина символа 0xf2 в кодировке CP1251
\end{verbatim}

\begin{verbatim}
Warning in grid.Call.graphics(C_text, as.graphicsAnnot(x$label), x$x, x$y, :
неизвестна ширина символа 0xe8 в кодировке CP1251
\end{verbatim}

\begin{verbatim}
Warning in grid.Call.graphics(C_text, as.graphicsAnnot(x$label), x$x, x$y, :
неизвестна ширина символа 0xea в кодировке CP1251
\end{verbatim}

\begin{verbatim}
Warning in grid.Call.graphics(C_text, as.graphicsAnnot(x$label), x$x, x$y, :
неизвестна ширина символа 0xe8 в кодировке CP1251
\end{verbatim}

\begin{verbatim}
Warning in grid.Call.graphics(C_text, as.graphicsAnnot(x$label), x$x, x$y, :
неизвестна ширина символа 0xf4 в кодировке CP1251
\end{verbatim}

\begin{verbatim}
Warning in grid.Call.graphics(C_text, as.graphicsAnnot(x$label), x$x, x$y, :
неизвестна ширина символа 0xe0 в кодировке CP1251
\end{verbatim}

\begin{verbatim}
Warning in grid.Call.graphics(C_text, as.graphicsAnnot(x$label), x$x, x$y, :
неизвестна ширина символа 0xea в кодировке CP1251
\end{verbatim}

\begin{verbatim}
Warning in grid.Call.graphics(C_text, as.graphicsAnnot(x$label), x$x, x$y, :
неизвестна ширина символа 0xf2 в кодировке CP1251
\end{verbatim}

\pandocbounded{\includegraphics[keepaspectratio]{chapter9_files/figure-pdf/unnamed-chunk-1-12.pdf}}

\begin{Shaded}
\begin{Highlighting}[]
\CommentTok{\# ==============================================================================}
\CommentTok{\# БЛОК 9: СРАВНЕНИЕ МОДЕЛЕЙ ПО ИНФОРМАЦИОННЫМ КРИТЕРИЯМ}
\CommentTok{\# ==============================================================================}

\FunctionTok{cat}\NormalTok{(}\StringTok{"}\SpecialCharTok{\textbackslash{}n}\StringTok{=== СРАВНИТЕЛЬНЫЙ АНАЛИЗ МОДЕЛЕЙ ПО ИНФОРМАЦИОННЫМ КРИТЕРИЯМ ===}\SpecialCharTok{\textbackslash{}n}\StringTok{"}\NormalTok{)}
\end{Highlighting}
\end{Shaded}

\begin{verbatim}

=== СРАВНИТЕЛЬНЫЙ АНАЛИЗ МОДЕЛЕЙ ПО ИНФОРМАЦИОННЫМ КРИТЕРИЯМ ===
\end{verbatim}

\begin{Shaded}
\begin{Highlighting}[]
\CommentTok{\# Упрощенная функция для извлечения ключевых критериев из моделей}
\NormalTok{extract\_model\_metrics }\OtherTok{\textless{}{-}} \ControlFlowTok{function}\NormalTok{(model, model\_name, }\AttributeTok{model\_type =} \StringTok{"glm"}\NormalTok{) \{}
  \ControlFlowTok{if}\NormalTok{ (model\_type }\SpecialCharTok{==} \StringTok{"glm"}\NormalTok{) \{}
\NormalTok{    aic\_val }\OtherTok{\textless{}{-}} \FunctionTok{AIC}\NormalTok{(model)}
\NormalTok{    bic\_val }\OtherTok{\textless{}{-}} \FunctionTok{BIC}\NormalTok{(model)}
\NormalTok{    loglik\_val }\OtherTok{\textless{}{-}} \FunctionTok{as.numeric}\NormalTok{(}\FunctionTok{logLik}\NormalTok{(model))}
\NormalTok{    df\_val }\OtherTok{\textless{}{-}}\NormalTok{ model}\SpecialCharTok{$}\NormalTok{rank}
\NormalTok{    null\_dev }\OtherTok{\textless{}{-}}\NormalTok{ model}\SpecialCharTok{$}\NormalTok{null.deviance}
\NormalTok{    dev }\OtherTok{\textless{}{-}}\NormalTok{ model}\SpecialCharTok{$}\NormalTok{deviance}
\NormalTok{  \} }\ControlFlowTok{else} \ControlFlowTok{if}\NormalTok{ (model\_type }\SpecialCharTok{==} \StringTok{"gam"}\NormalTok{) \{}
\NormalTok{    aic\_val }\OtherTok{\textless{}{-}} \FunctionTok{AIC}\NormalTok{(model)}
\NormalTok{    bic\_val }\OtherTok{\textless{}{-}} \FunctionTok{BIC}\NormalTok{(model)}
\NormalTok{    loglik\_val }\OtherTok{\textless{}{-}} \FunctionTok{as.numeric}\NormalTok{(}\FunctionTok{logLik}\NormalTok{(model))}
\NormalTok{    df\_val }\OtherTok{\textless{}{-}} \FunctionTok{sum}\NormalTok{(model}\SpecialCharTok{$}\NormalTok{edf)}
\NormalTok{    null\_dev }\OtherTok{\textless{}{-}}\NormalTok{ model}\SpecialCharTok{$}\NormalTok{null.deviance}
\NormalTok{    dev }\OtherTok{\textless{}{-}}\NormalTok{ model}\SpecialCharTok{$}\NormalTok{deviance}
\NormalTok{  \} }\ControlFlowTok{else} \ControlFlowTok{if}\NormalTok{ (model\_type }\SpecialCharTok{==} \StringTok{"gamm"}\NormalTok{) \{}
\NormalTok{    aic\_val }\OtherTok{\textless{}{-}} \FunctionTok{AIC}\NormalTok{(model}\SpecialCharTok{$}\NormalTok{mer)}
\NormalTok{    bic\_val }\OtherTok{\textless{}{-}} \FunctionTok{BIC}\NormalTok{(model}\SpecialCharTok{$}\NormalTok{mer)}
\NormalTok{    loglik\_val }\OtherTok{\textless{}{-}} \FunctionTok{as.numeric}\NormalTok{(}\FunctionTok{logLik}\NormalTok{(model}\SpecialCharTok{$}\NormalTok{mer))}
\NormalTok{    df\_val }\OtherTok{\textless{}{-}} \FunctionTok{length}\NormalTok{(}\FunctionTok{fixef}\NormalTok{(model}\SpecialCharTok{$}\NormalTok{mer)) }\SpecialCharTok{+} \DecValTok{1}  \CommentTok{\# +1 для случайного эффекта}
\NormalTok{    null\_dev }\OtherTok{\textless{}{-}}\NormalTok{ model}\SpecialCharTok{$}\NormalTok{gam}\SpecialCharTok{$}\NormalTok{null.deviance}
\NormalTok{    dev }\OtherTok{\textless{}{-}}\NormalTok{ model}\SpecialCharTok{$}\NormalTok{gam}\SpecialCharTok{$}\NormalTok{deviance}
\NormalTok{  \}}
  
  \CommentTok{\# Вычисляем долю объясненной девиации}
\NormalTok{  deviance\_explained }\OtherTok{\textless{}{-}} \FunctionTok{ifelse}\NormalTok{(}\SpecialCharTok{!}\FunctionTok{is.null}\NormalTok{(null\_dev) }\SpecialCharTok{\&\&} \SpecialCharTok{!}\FunctionTok{is.null}\NormalTok{(dev) }\SpecialCharTok{\&\&}\NormalTok{ null\_dev }\SpecialCharTok{\textgreater{}} \DecValTok{0}\NormalTok{,}
\NormalTok{                              (null\_dev }\SpecialCharTok{{-}}\NormalTok{ dev) }\SpecialCharTok{/}\NormalTok{ null\_dev, }\ConstantTok{NA}\NormalTok{)}
  
  \FunctionTok{data.frame}\NormalTok{(}
    \AttributeTok{Model =}\NormalTok{ model\_name,}
    \AttributeTok{AIC =} \FunctionTok{round}\NormalTok{(aic\_val, }\DecValTok{2}\NormalTok{),}
    \AttributeTok{BIC =} \FunctionTok{round}\NormalTok{(bic\_val, }\DecValTok{2}\NormalTok{),}
    \AttributeTok{LogLik =} \FunctionTok{round}\NormalTok{(loglik\_val, }\DecValTok{2}\NormalTok{),}
    \AttributeTok{DF =} \FunctionTok{round}\NormalTok{(df\_val, }\DecValTok{2}\NormalTok{),}
    \AttributeTok{Deviance\_Explained =} \FunctionTok{round}\NormalTok{(deviance\_explained, }\DecValTok{4}\NormalTok{)}
\NormalTok{  )}
\NormalTok{\}}

\CommentTok{\# Извлекаем метрики для всех моделей}
\NormalTok{model\_metrics }\OtherTok{\textless{}{-}} \FunctionTok{bind\_rows}\NormalTok{(}
  \FunctionTok{extract\_model\_metrics}\NormalTok{(glm\_gamma\_fit, }\StringTok{"GLM (Gamma)"}\NormalTok{, }\StringTok{"glm"}\NormalTok{),}
  \FunctionTok{extract\_model\_metrics}\NormalTok{(gam\_fit, }\StringTok{"GAM"}\NormalTok{, }\StringTok{"gam"}\NormalTok{),}
  \FunctionTok{extract\_model\_metrics}\NormalTok{(gamm\_fit, }\StringTok{"GAMM"}\NormalTok{, }\StringTok{"gamm"}\NormalTok{)}
\NormalTok{)}

\CommentTok{\# Добавляем разницу в AIC относительно наилучшей модели}
\NormalTok{min\_aic }\OtherTok{\textless{}{-}} \FunctionTok{min}\NormalTok{(model\_metrics}\SpecialCharTok{$}\NormalTok{AIC)}
\NormalTok{model\_metrics }\OtherTok{\textless{}{-}}\NormalTok{ model\_metrics }\SpecialCharTok{\%\textgreater{}\%}
  \FunctionTok{mutate}\NormalTok{(}\AttributeTok{Delta\_AIC =}\NormalTok{ AIC }\SpecialCharTok{{-}}\NormalTok{ min\_aic,}
         \AttributeTok{AIC\_Weight =} \FunctionTok{exp}\NormalTok{(}\SpecialCharTok{{-}}\FloatTok{0.5} \SpecialCharTok{*}\NormalTok{ Delta\_AIC) }\SpecialCharTok{/} \FunctionTok{sum}\NormalTok{(}\FunctionTok{exp}\NormalTok{(}\SpecialCharTok{{-}}\FloatTok{0.5} \SpecialCharTok{*}\NormalTok{ Delta\_AIC)))}

\CommentTok{\# Форматируем таблицу для вывода}
\NormalTok{comparison\_table }\OtherTok{\textless{}{-}}\NormalTok{ model\_metrics }\SpecialCharTok{\%\textgreater{}\%}
  \FunctionTok{mutate}\NormalTok{(}\FunctionTok{across}\NormalTok{(}\FunctionTok{where}\NormalTok{(is.numeric), }\SpecialCharTok{\textasciitilde{}}\FunctionTok{round}\NormalTok{(., }\DecValTok{3}\NormalTok{))) }\SpecialCharTok{\%\textgreater{}\%}
  \FunctionTok{arrange}\NormalTok{(AIC)  }\CommentTok{\# Сортируем по AIC (лучшая модель первая)}

\CommentTok{\# Выводим таблицу сравнения}
\FunctionTok{cat}\NormalTok{(}\StringTok{"}\SpecialCharTok{\textbackslash{}n}\StringTok{ТАБЛИЦА СРАВНЕНИЯ МОДЕЛЕЙ:}\SpecialCharTok{\textbackslash{}n}\StringTok{"}\NormalTok{)}
\end{Highlighting}
\end{Shaded}

\begin{verbatim}

ТАБЛИЦА СРАВНЕНИЯ МОДЕЛЕЙ:
\end{verbatim}

\begin{Shaded}
\begin{Highlighting}[]
\FunctionTok{print}\NormalTok{(comparison\_table)}
\end{Highlighting}
\end{Shaded}

\begin{verbatim}
        Model      AIC      BIC    LogLik DF Deviance_Explained Delta_AIC
1         GAM 42840.91 43079.03 -21382.45 37              0.401      0.00
2 GLM (Gamma) 42850.76 43088.88 -21387.38 37              0.401      9.85
3        GAMM 42933.49 43065.09 -21445.75 20                 NA     92.58
  AIC_Weight
1      0.993
2      0.007
3      0.000
\end{verbatim}

\begin{Shaded}
\begin{Highlighting}[]
\CommentTok{\# Выводим итоговые рекомендации}
\FunctionTok{cat}\NormalTok{(}\StringTok{"}\SpecialCharTok{\textbackslash{}n}\StringTok{=== ИТОГОВЫЕ РЕКОМЕНДАЦИИ ПО ВЫБОРУ МОДЕЛИ ===}\SpecialCharTok{\textbackslash{}n}\StringTok{"}\NormalTok{)}
\end{Highlighting}
\end{Shaded}

\begin{verbatim}

=== ИТОГОВЫЕ РЕКОМЕНДАЦИИ ПО ВЫБОРУ МОДЕЛИ ===
\end{verbatim}

\begin{Shaded}
\begin{Highlighting}[]
\NormalTok{best\_model }\OtherTok{\textless{}{-}}\NormalTok{ comparison\_table}\SpecialCharTok{$}\NormalTok{Model[}\DecValTok{1}\NormalTok{]}
\FunctionTok{cat}\NormalTok{(}\StringTok{"Наилучшая модель по критерию AIC:"}\NormalTok{, best\_model, }\StringTok{"}\SpecialCharTok{\textbackslash{}n}\StringTok{"}\NormalTok{)}
\end{Highlighting}
\end{Shaded}

\begin{verbatim}
Наилучшая модель по критерию AIC: GAM 
\end{verbatim}

\begin{Shaded}
\begin{Highlighting}[]
\FunctionTok{cat}\NormalTok{(}\StringTok{"Вес AIC для наилучшей модели:"}\NormalTok{, }\FunctionTok{round}\NormalTok{(comparison\_table}\SpecialCharTok{$}\NormalTok{AIC\_Weight[}\DecValTok{1}\NormalTok{], }\DecValTok{3}\NormalTok{), }\StringTok{"}\SpecialCharTok{\textbackslash{}n}\StringTok{"}\NormalTok{)}
\end{Highlighting}
\end{Shaded}

\begin{verbatim}
Вес AIC для наилучшей модели: 0.993 
\end{verbatim}

\begin{Shaded}
\begin{Highlighting}[]
\ControlFlowTok{if}\NormalTok{ (}\FunctionTok{nrow}\NormalTok{(comparison\_table) }\SpecialCharTok{\textgreater{}} \DecValTok{1} \SpecialCharTok{\&\&}\NormalTok{ comparison\_table}\SpecialCharTok{$}\NormalTok{Delta\_AIC[}\DecValTok{2}\NormalTok{] }\SpecialCharTok{\textgreater{}} \DecValTok{2}\NormalTok{) \{}
  \FunctionTok{cat}\NormalTok{(}\StringTok{"Наилучшая модель существенно лучше остальных (?AIC \textgreater{} 2).}\SpecialCharTok{\textbackslash{}n}\StringTok{"}\NormalTok{)}
\NormalTok{\} }\ControlFlowTok{else} \ControlFlowTok{if}\NormalTok{ (}\FunctionTok{nrow}\NormalTok{(comparison\_table) }\SpecialCharTok{\textgreater{}} \DecValTok{1}\NormalTok{) \{}
  \FunctionTok{cat}\NormalTok{(}\StringTok{"Несколько моделей имеют сходное качество (?AIC \textless{} 2).}\SpecialCharTok{\textbackslash{}n}\StringTok{"}\NormalTok{)}
\NormalTok{\}}
\end{Highlighting}
\end{Shaded}

\begin{verbatim}
Наилучшая модель существенно лучше остальных (?AIC > 2).
\end{verbatim}

\begin{Shaded}
\begin{Highlighting}[]
\FunctionTok{cat}\NormalTok{(}\StringTok{"Доля объясненной девиации наилучшей модели:"}\NormalTok{, }
    \FunctionTok{round}\NormalTok{(comparison\_table}\SpecialCharTok{$}\NormalTok{Deviance\_Explained[}\DecValTok{1}\NormalTok{], }\DecValTok{3}\NormalTok{), }\StringTok{"}\SpecialCharTok{\textbackslash{}n}\StringTok{"}\NormalTok{)}
\end{Highlighting}
\end{Shaded}

\begin{verbatim}
Доля объясненной девиации наилучшей модели: 0.401 
\end{verbatim}

\section{Анализ GLM модели с гамма-распределением и
лог-ссылкой}\label{ux430ux43dux430ux43bux438ux437-glm-ux43cux43eux434ux435ux43bux438-ux441-ux433ux430ux43cux43cux430-ux440ux430ux441ux43fux440ux435ux434ux435ux43bux435ux43dux438ux435ux43c-ux438-ux43bux43eux433-ux441ux441ux44bux43bux43aux43eux439}

Подобранная обобщенная линейная модель (GLM) использует
гамма-распределение для ошибок и логарифмическую функцию связи. Данное
распределение выбрано, поскольку CPUE представляет собой непрерывную
положительную величину, а гамма-распределение хорошо описывает такие
данные. Логарифмическая связь обеспечивает мультипликативность эффектов
факторов, что интерпретируется как относительное изменение CPUE при
изменении фактора. Модель включает четыре факторные переменные: год
(YEAR), месяц (MONTH), позывной судна (CALL) и район промысла (REGION).
Все переменные представлены как факторы, что означает, что для каждого
уровня фактора оценивается свой коэффициент, интерпретируемый как
отклонение от базового уровня. Базовыми уровнями являются: 2019 год для
YEAR, сентябрь (MONTH9) для MONTH, первое судно в алфавитном порядке для
CALL и первый район для REGION. Из сводки модели видно, что многие
коэффициенты статистически значимы. Все годовые коэффициенты
отрицательны и значимы, что указывает на снижение CPUE относительно
базового 2019 года. Наибольшее снижение наблюдается в 2024 году
(коэффициент -1.128). Месячные коэффициенты также отрицательны и
значимы, что говорит о снижении CPUE в октябре и ноябре по сравнению с
сентябрем. Большинство коэффициентов для судов значимы и отрицательны,
что указывает на то, что уловы на усилие у этих судов в среднем ниже,
чем у базового судна. Однако некоторые суда имеют положительные
коэффициенты, что означает более высокую производительность. Для районов
значимыми оказались лишь некоторые коэффициенты, в основном
положительные, что говорит о более высоких уловах в этих районах по
сравнению с базовым. Дисперсионный параметр для гамма-семейства равен
0.417, что указывает на умеренную дисперсию. Null deviance составляет
2980.2 при 3890 степенях свободы, а остаточная deviance --- 1785.6 при
3854 степенях свободы. Снижение девиации указывает на то, что модель
объясняет существенную часть вариации данных. AIC модели равен 42851.
Диагностические графики стандартных остатков GLM включают график
остатков против предсказанных значений, Q-Q plot, график
масштаба-местоположения и график остатков против влияния. Эти графики
позволяют оценить гомоскедастичность, нормальность остатков и наличие
выбросов. Дополнительная диагностика с помощью пакета DHARMa показывает,
что распределение остатков соответствует ожидаемому, что подтверждает
адекватность выбранного семейства распределений. Расчет
стандартизированных индексов с помощью функции emmeans показывает, что
индекс CPUE постепенно снижается с 2019 по 2024 год, что согласуется с
отрицательными годовыми коэффициентами. В 2019 году индекс составляет
159, а к 2024 падает до 51.4. Доверительные интервалы не перекрываются
между крайними годами, что указывает на статистически значимое снижение.
Сравнение с медианными значениями CPUE по годам из исходных данных
показывает, что модель несколько сглаживает исходные данные, но общая
тенденция снижения сохраняется. Например, в 2019 году медианное значение
CPUE было 200, а стандартизированный индекс --- 159, что может быть
связано с учетом влияния других факторов. Преимущества GLM подхода
включают простоту интерпретации коэффициентов, вычислительную
эффективность и широкую распространенность. Недостатки заключаются в
том, что GLM предполагает линейность влияния факторов на логарифм
отклика, что может не всегда выполняться. Кроме того, модель с
фиксированными эффектами может не учитывать некоторые источники
вариации, такие как пространственно-временная автокорреляция или
случайные эффекты судов. В целом, модель адекватно описывает данные и
может быть использована для стандартизации CPUE, но для более сложных
данных могут потребоваться более гибкие модели, такие как GAM или GAMM.

\section{Анализ GAM модели с гамма-распределением и логарифмической
связью}\label{ux430ux43dux430ux43bux438ux437-gam-ux43cux43eux434ux435ux43bux438-ux441-ux433ux430ux43cux43cux430-ux440ux430ux441ux43fux440ux435ux434ux435ux43bux435ux43dux438ux435ux43c-ux438-ux43bux43eux433ux430ux440ux438ux444ux43cux438ux447ux435ux441ux43aux43eux439-ux441ux432ux44fux437ux44cux44e}

Анализ подобранной обобщенной аддитивной модели (GAM) с
гамма-распределением и логарифмической связью показывает результаты,
практически идентичные полученным ранее для GLM модели, что ожидаемо,
поскольку в данной реализации GAM использовалась полностью
параметрическая формула без сглаживающих функций. Модель была построена
с теми же предикторами - годом, месяцем, идентификатором судна и районом
промысла.

Сводка модели демонстрирует параметрические коэффициенты, которые
практически не отличаются от оценок GLM модели. Все годовые коэффициенты
остаются отрицательными и статистически значимыми, подтверждая
устойчивую тенденцию снижения стандартизированного индекса CPUE с 2019
по 2024 год. Месячные коэффициенты также значимы и отрицательны,
указывая на сезонное снижение уловов в октябре и ноябре по сравнению с
сентябрем. Оценки для различных судов и районов промысла практически
идентичны полученным в GLM, с сохранением статистической значимости для
тех же уровней факторов.

Модель объясняет 40.1\% девиации данных, что полностью соответствует
показателю GLM. Значение REML составляет 21455, а оценка дисперсии равна
0.41722, что также практически совпадает с соответствующими показателями
GLM модели.

Проверка адекватности модели с помощью функции gam.check показывает
успешную сходимость алгоритма оптимизации после 5 итераций. Градиент
близок к нулю, а гессиан положительно определен, что свидетельствует о
достижении устойчивого решения. Поскольку в модели отсутствуют
сглаживающие компоненты, диагностика не выявляет проблем, связанных с
выбором базовой размерности или неадекватностью сглаживания.

Диагностика остатков с использованием пакета DHARMa показывает
равномерное распределение без систематических паттернов, что указывает
на соответствие остатков теоретическому гамма-распределению. Графики
остатков демонстрируют отсутствие гетероскедастичности и значимых
выбросов, что подтверждает адекватность модели.

Расчет стандартизированных индексов методом маргинальных средних дает
значения, практически идентичные полученным из GLM модели. Индекс
снижается с 159 в 2019 году до 51.4 в 2024 году, с доверительными
интервалами, не перекрывающимися между крайними годами. Нормированные
индексы относительно среднего и первого года также полностью совпадают с
GLM результатами.

Основное преимущество использования GAM в данном случае заключается в
методологическом подходе - использовании метода REML для оптимизации,
который может обеспечивать более стабильные оценки параметров по
сравнению с методом максимального правдоподобия, используемым в GLM.
Хотя в данной конкретной реализации с полностью параметрической формулой
это преимущество не реализуется в полной мере, GAM предоставляет основу
для легкого включения нелинейных эффектов через сглаживающие функции,
если такая необходимость возникнет в дальнейшем.

К недостаткам данного подхода можно отнести избыточную сложность GAM для
полностью параметрической модели, поскольку вычислительные затраты выше,
чем для GLM, без существенного улучшения качества подгонки. Фактически,
в данном случае GAM работает как GLM, но с более сложным алгоритмом
оптимизации. Кроме того, диагностика GAM требует дополнительных
проверок, связанных со сходимостью алгоритма и адекватностью
сглаживания, которые не актуальны для параметрических моделей.

В целом, данная реализация GAM не демонстрирует преимуществ перед GLM
моделью, но предоставляет основу для будущего расширения модели за счет
включения нелинейных эффектов, если анализ данных покажет такую
необходимость. Результаты стандартизации CPUE полностью согласуются с
полученными ранее средствами GLM.

\section{Анализ GAMM
модели}\label{ux430ux43dux430ux43bux438ux437-gamm-ux43cux43eux434ux435ux43bux438}

\textbf{Особенности смешанных моделей и случайные эффекты}

Смешанные модели, включая GAMM, расширяют возможности стандартных
моделей за счет введения случайных эффектов. В то время как
фиксированные эффекты оценивают среднее влияние факторов на всю
популяцию, случайные эффекты позволяют учесть вариацию, связанную с
отдельными группами наблюдений. В ихтиологических исследованиях
случайные эффекты часто применяются для учета индивидуальных
особенностей судов, различий между районами промысла, или временной
автокорреляции.

Случайные эффекты особенно полезны, когда:

\begin{itemize}
\item
  Данные имеют иерархическую структуру (например, уловы по нескольким
  судам)
\item
  Наблюдения внутри групп коррелированы
\item
  Количество уровней фактора велико, и мы хотим обобщить выводы на всю
  популяцию групп
\item
  Нас интересует вариация между группами, а не конкретные сравнения
  между отдельными уровнями
\end{itemize}

В данном случае случайный эффект для судна (CALL) позволяет учесть, что
разные суда могут иметь систематические различия в эффективности
промысла, не объясняемые другими переменными модели.

\textbf{Анализ результатов GAMM модели}

Анализ обобщенной аддитивной смешанной модели показывает несколько
важных особенностей. Модель включает фиксированные эффекты года, месяца
и района, а также случайный эффект для судна, что позволяет учесть
индивидуальные различия между судами в уровне уловов.

График остатков от предсказанных значений показывает распределение
девиансных остатков вокруг нулевой линии. Наблюдается некоторая
гетероскедастичность - разброс остатков увеличивается с ростом
предсказанных значений, что характерно для данных по уловам. QQ-plot
демонстрирует отклонение распределения остатков от нормального в крайних
значениях, что ожидаемо для данных с гамма-распределением.

Анализ случайных эффектов для судов показывает существенную вариацию
между разными судами. Значения случайных эффектов варьируют от -2.92 до
0.93, что указывает на значительные различия в эффективности промысла
между судами после учета влияния года, месяца и района. Распределение
случайных эффектов близко к нормальному с центром около нуля.

Тест Бреуша-Пагана подтверждает наличие гетероскедастичности в модели,
что является общей проблемой для моделей с данными по уловам.

Сводка параметрических коэффициентов показывает, что все годовые эффекты
статистически значимы и отрицательны, подтверждая общую тенденцию
снижения уловов с течением времени. Месячные эффекты также значимы и
отрицательны, указывая на сезонное снижение уловов в октябре и ноябре по
сравнению с сентябрем. Среди районов промысла несколько показали
статистически значимые отличия от базового уровня.

Модель объясняет 17.2\% дисперсии данных, что меньше, чем в предыдущих
моделях, что может быть связано с учетом части вариации через случайные
эффекты. Информационные критерии AIC (42933.5) и BIC (43065.1) выше, чем
у GLM и GAM моделей, что указывает на худшее соответствие данных этой
модели с учетом ее сложности.

\textbf{\emph{График случайных эффектов}} с тремя модами на значениях
0.5, -1.5 и -3 демонстрирует выраженную стратификацию судов по их
промысловой эффективности. Такое распределение указывает на наличие трех
различных групп в промысловом флоте, каждая со своими характеристиками.
Группа с модой на 0.5 представляет суда с повышенной эффективностью, чьи
уловы примерно на 65\% (exp(0.5) ≈ 0.65) превышают средний уровень. Эти
суда, вероятно, оснащены современным оборудованием, укомплектованы
опытными экипажами и работают на наиболее продуктивных участках.

Вторая группа с модой на -1.5 соответствует судам со значительно
сниженной эффективностью, показывающим уловы примерно на 78\% ниже
среднего показателя. Такие результаты могут быть связаны с устаревшим
техническим оснащением, менее оптимальными методами лова или работой в
менее продуктивных районах. Третья группа с модой на -3 представляет
суда с крайне низкой эффективностью, демонстрирующие уловы на 95\% ниже
среднего уровня. Столь значительное отставание может объясняться
серьезными техническими проблемами, отсутствием современного
оборудования, неопытностью экипажей или систематическими
организационными трудностями.А возможно работой не на мороженном крабе,
а живом - требующим другой технологической работы.

Наличие трех четких мод в распределении случайных эффектов
свидетельствует о существенной неоднородности промыслового флота. Это
указывает на то, что предположение о нормальном распределении случайных
эффектов не выполняется, а данные имеют выраженную групповую структуру.
Модель успешно выявляет эту скрытую стратификацию, что подтверждает
важность учета случайных эффектов при анализе промысловых данных.
Полученные результаты подчеркивают необходимость дифференцированного
подхода к анализу эффективности судов и разработки управленческих
решений с учетом выявленной группировки. Различные моды могут отражать
не только технические различия между судами, но и различные стратегии
промысла, доступ к ресурсам или уровень организации работы.

\textbf{Преимущества и недостатки подхода GAMM}

Основное преимущество GAMM подхода заключается в возможности учета
групповой структуры данных через случайные эффекты. Это позволяет более
адекватно оценить неопределенность предсказаний и избежать завышения
значимости эффектов из-за псевдорепликации. Модель обеспечивает более
реалистичную оценку вариации в данных, учитывая как фиксированные
эффекты, так и случайную вариацию между группами.

К недостаткам можно отнести повышенную вычислительную сложность и
потенциальные проблемы со сходимостью алгоритмов оптимизации.
Интерпретация результатов становится сложнее, особенно при наличии
взаимодействий между фиксированными и случайными эффектами. В данном
случае модель показала худшие показатели качества подгонки по сравнению
с более простыми GLM и GAM моделями, что может свидетельствовать о
избыточной сложности модели для данного набора данных.

В целом, GAMM представляет собой мощный инструмент для анализа данных с
иерархической структурой, но его применение должно быть обосновано
теоретически и подтверждено улучшением качества модели по сравнению с
более простыми альтернативами.

\section{Сравнительный анализ моделей по информационным
критериям}\label{ux441ux440ux430ux432ux43dux438ux442ux435ux43bux44cux43dux44bux439-ux430ux43dux430ux43bux438ux437-ux43cux43eux434ux435ux43bux435ux439-ux43fux43e-ux438ux43dux444ux43eux440ux43cux430ux446ux438ux43eux43dux43dux44bux43c-ux43aux440ux438ux442ux435ux440ux438ux44fux43c}

В данном разделе проводится систематическое сравнение трех
альтернативных моделей - GLM, GAM и GAMM - с использованием
информационных критериев и других метрик качества. Для унификации
процесса сравнения создана специализированная функция
extract\_model\_metrics, которая адаптирована для извлечения
сопоставимых показателей из моделей разной структуры.

Для GLM и GAM моделей используются стандартные методы расчета критериев,
включая AIC, BIC, логарифмическое правдоподобие и долю объясненной
девиации. Для GAMM модели, имеющей более сложную смешанную структуру,
метрики извлекаются из компонентов mer и gam объекта, с дополнительным
учетом случайных эффектов при расчете сложности модели.

Результаты сравнения представлены в виде структурированной таблицы, где
модели упорядочены по возрастанию AIC - информационного критерия Акаике,
который балансирует качество подгонки и сложность модели. Дополнительно
вычисляются дельта-AIC (разница относительно наилучшей модели) и веса
AIC, которые интерпретируются как вероятности того, что данная модель
является наилучшей среди рассматриваемых.

Анализ результатов показывает четкое разделение моделей по качеству.
Модель GAM демонстрирует наилучшие показатели с AIC = 42840.91 и весом
AIC 0.993, что означает 99.3\% вероятность того, что эта модель является
наилучшей среди сравниваемых. Модель GLM показывает очень близкие
результаты по объясненной дисперсии (0.401), но несколько худшие
значения AIC (42850.76) и минимальный вес (0.007). Модель GAMM
значительно уступает по всем критериям с AIC = 42933.49 и нулевым весом
в рамках данного сравнения.

Разница в AIC между GAM и GLM составляет 9.85 единиц, что превышает
пороговое значение 2, принятое для утверждения о существенном
преимуществе одной модели над другой. Еще более значительная разница в
92.58 единиц между GAM и GAMM подтверждает статистически значимое
превосходство GAM модели.

На основе проведенного анализа формулируются итоговые рекомендации по
выбору модели. Модель GAM идентифицируется как наилучшая с очень высокой
степенью уверенности (вес AIC 0.993). Объясняющая способность модели
составляет 40.1\%, что указывает на хорошее соответствие модели данным.

Данный сравнительный подход обеспечивает объективную основу для выбора
окончательной модели, позволяя учесть как качество подгонки, так и
сложность модели, избегая таким образом как избыточного усложнения, так
и излишнего упрощения.

\bookmarksetup{startatroot}

\chapter{Съёмка: оптимизация
маршрута}\label{ux441ux44aux451ux43cux43aux430-ux43eux43fux442ux438ux43cux438ux437ux430ux446ux438ux44f-ux43cux430ux440ux448ux440ux443ux442ux430}

\section{Введение}\label{ux432ux432ux435ux434ux435ux43dux438ux435-10}

Если смотреть на планирование съёмки из высоты «Космоса» Карла Сагана и
«Краткой истории времени» Стивена Хокинга, всё сводится к простой
геометрии и неумолимой физике: расстояния складываются, время утекает, а
топология района диктует, сколько топлива и суток мы действительно можем
себе позволить. Но как только мы возвращаемся к карте, «быстрый» ум
спешит дорисовать удобную историю: «расширим полигон --- маршрут
вырастет вдвое». На практике рост длины маршрута ведёт себя нелинейно, а
форма «важнее площади» чаще, чем нам кажется. Это занятие --- про то,
как дисциплинировать интуицию, превратить картинку в числа и принять
решения, которые выдерживают встречу с реальностью.

Мы используем минималистичный, но честный инструментариум: переводим
исходный полигон в метрическую проекцию UTM, генерируем равномерную
сетку станций и решаем задачу коммивояжёра для маршрута судна. Дальше
--- серия контролируемых экспериментов: асимметрично «растягиваем»
полигон к северу (имитируя расширение арктического фронта работ) в 1.5,
2 и 3 раза, каждый раз оставляя одно и то же число станций. Сравниваем
площади и оптимальные маршруты. Наша цель проста: понять, насколько
дороже становится логистика от изменения площади и формы, когда часы и
топливо конечны.

Мы заранее признаём «чёрных лебедей»: погодные окна, запреты и ледовую
обстановку скрипт не видит. Но и это хорошая новость ---
систематический, воспроизводимый подход всё равно делает нас лучше:
регулярная оценка маршрутов при разных сценариях снижает долю
импровизации и помогает не переплачивать за «красивые» идеи. Здесь
важнее не одна точная цифра, а устойчивая картина: где логистика растёт
медленнее, чем площадь, а где внезапно «взлетает» из‑за длинной узкой
шейки полигона или неудобно расположенных станций.

Эта логистическая «эволюция маршрутов» чем‑то напоминает «Эгоистичный
ген» Ричарда Докинза: выживают те траектории, которые экономят ресурс
--- и не важно, как они появились, важно, что они меньше «едят» времени
и топлива. А из «SPQR» Мэри Бирд можно позаимствовать трезвость римских
инженеров: дороги строили не по вдохновению, а по задачам снабжения; в
море у нас те же дороги, только из линий на карте. Управление --- это
всегда история, которую мы рассказываем про будущее рейса; хорошая
история --- та, которая прошла проверку расчётом. И, да, «воля и
самоконтроль» здесь не про героизм, а про дисциплину: держать число
станций, окна и границы сценариев в рамках, чтобы сравнение было
честным.

Чуть техники, но по делу. Мы:

\begin{itemize}
\item
  загружаем полигон 2020 года и считаем его площадь в километрах,
\item
  строим три северные экспансии (×1.5, ×2, ×3), не симметрично раздувая
  фигуру, а добавляя «арктический хвост»,
\item
  генерируем внутри каждого полигона по 137 регулярных станций ---
  одинаковая «нагрузка» для справедливости сравнения,
\item
  решаем TSP с геодезическими расстояниями (Haversine) и фиксируем
  начало/конец маршрута в самых западных точках --- реалистичная
  постановка для судна, идущего «с запада»,
\item
  визуализируем и сводим метрики: площадь и длина маршрута по каждому
  сценарию.
\end{itemize}

Что мы хотим увидеть. Во‑первых, нелинейность: умеренное расширение
площади иногда почти «бесплатно», а узкая северная протяжка может
внезапно удорожить рейс. Во‑вторых, роль формы: два полигона одинаковой
площади могут иметь разные «ценники» из‑за геометрии. В‑третьих, смысл
«предельной» станции: где последняя добавленная точка дороже всех
предыдущих вместе. Это те наблюдения, которые превращают космическую
картинку Нила Деграсса Тайсона о «геометрии вселенной» в бытовую
навигацию: за изгибами карты стоят очень приземлённые деньги и часы.

Есть и границы. Наш TSP‑эвристик быстрый, но не гарантирует глобальный
оптимум; станции расставлены равномерно и не знают про глубины;
старт/финиш жёстко заданы; мы не моделируем штормовые задержки и ледовые
коридоры. Нам нужны «хорошие объяснения», а не идеально подогнанные:
скрипт даёт прозрачную, проверяемую основу для решений и для постановки
более сложных задач --- со слоями глубин, реальными портами, окнами и
альтернативными алгоритмами маршрутизации. А следующий шаг: добавить
предиктивные слои (погода/лед) и дать планировщику инструмент «думать
наперёд».

И последнее --- о масштабе. История учит смотреть на мир одновременно
широко и конкретно. Наши полигоны --- маленькие кусочки океана, но в них
решается вечная задача: как превратить энергию и время в полезные данные
с минимальными потерями. Этот скрипт --- простой, «большой» по духу
метод: он берёт карту, делает из неё информацию и возвращает
управленческое решение. Именно такие мосты между картинкой и числом
уменьшают долю случайности и, шаг за шагом, ведут к более разумной
экспедиции.

И так, этот R-скрипт выполняет геопространственный анализ и
моделирование станций исследований в ходе научно-исследовательских
съемок.

\textbf{Основная цель}

Скрипт моделирует \textbf{оптимальные маршруты для станций} в разных
вариантах исследовательского полигона, чтобы определить, как изменение
площади и формы полигона влияет на длину необходимого маршрута. Полный
скрипт можно скачать по
\href{https://mombus.github.io/cRab/data/SWEPT.R}{ссылке}.

\textbf{Для работы скрипта:}

\begin{enumerate}
\def\labelenumi{\arabic{enumi}.}
\tightlist
\item
  Скачайте файл данных
  (\href{https://mombus.github.io/cRab/data/polygons.RData}{polygons.RData})
\item
  Установите рабочую директорию в setwd()
\item
  Установите необходимые пакеты.`
\end{enumerate}

\textbf{Пошаговая работа скрипта}

\textbf{1. Подготовка данных} - Загружается полигон за 2020 год из файла
\texttt{polygons.RData} - Полигон преобразуется в систему координат UTM
зоны 40N (EPSG:32640) для точных метрических вычислений - Рассчитывается
площадь полигона: \textbf{63,101.31 км²}

\textbf{2. Моделирование расширенных полигонов} Создан алгоритм, который
\textbf{асимметрично расширяет полигон на север} (имитируя расширение
зоны исследования в арктическом направлении): -
\texttt{expand\_polygon\_north()} определяет северную часть (верхние
30\% точек) - Расширяет только эту часть на заданный коэффициент -
Создаются 3 варианта: увеличенные на 1.5x, 2x и 3x

\textbf{3. Генерация траловых станций} - Для каждого полигона
генерируются \textbf{137 равномерно распределенных точек} (метод
``regular'') - Это имитирует расположение траловых станций в
исследовательском полигоне - Количество станций одинаково для всех
вариантов полигона

\textbf{4. Оптимизация маршрутов} Ключевая часть скрипта --- решение
\textbf{задачи коммивояжера (TSP)} для оптимизации маршрута судна: -
Начало и конец маршрута фиксируются как \textbf{две самые западные
точки} (логично для судна, приходящего с запада) - Используется алгоритм
ближайшего включения (\texttt{nearest\_insertion}) - Матрица расстояний
рассчитывается с помощью \texttt{distHaversine} (точное геодезическое
расстояние)

\textbf{5. Анализ результатов} Скрипт рассчитывает и сравнивает: -
Площадь каждого полигона- Длину оптимального маршрута для 137 станций

\begin{itemize}
\tightlist
\item
  Визуализирует все варианты на карте
\end{itemize}

\textbf{Интерпретация результатов}

Из сводной таблицы видно, как увеличение площади полигона влияет на
длину маршрута:

\begin{longtable}[]{@{}
  >{\raggedright\arraybackslash}p{(\linewidth - 6\tabcolsep) * \real{0.2500}}
  >{\raggedright\arraybackslash}p{(\linewidth - 6\tabcolsep) * \real{0.2500}}
  >{\raggedright\arraybackslash}p{(\linewidth - 6\tabcolsep) * \real{0.2500}}
  >{\raggedright\arraybackslash}p{(\linewidth - 6\tabcolsep) * \real{0.2500}}@{}}
\toprule\noalign{}
\begin{minipage}[b]{\linewidth}\raggedright
Вариант
\end{minipage} & \begin{minipage}[b]{\linewidth}\raggedright
Количество тралений
\end{minipage} & \begin{minipage}[b]{\linewidth}\raggedright
Длина маршрута (км)
\end{minipage} & \begin{minipage}[b]{\linewidth}\raggedright
Площадь полигона (км²)
\end{minipage} \\
\midrule\noalign{}
\endhead
\bottomrule\noalign{}
\endlastfoot
Original & 137 & 3,745 & 63,101.3 \\
Expanded\_x1.5 & 137 & 4,076 & 99,791.0 \\
Expanded\_x2 & 137 & 4,701 & 135,352.0 \\
Expanded\_x3 & 137 & 6,038 & 207,066.0 \\
\end{longtable}

\textbf{Важные наблюдения:} 1. При увеличении площади на 57\% (до 99,791
км²) длина маршрута возрастает только на 9\% (до 4,076 км) 2. При
троекратном увеличении площади (до 207,066 км²) длина маршрута
возрастает на 61\% (до 6,038 км) 3. Это показывает \textbf{нелинейную
зависимость} между площадью полигона и длиной маршрута

\textbf{Практическое применение} Этот анализ полезен для: - Планирования
гидробиологических исследований в условиях ограниченного бюджета -
Оценки дополнительных затрат при расширении зоны исследования -
Оптимизации маршрутов научно-исследовательских судов - Моделирования
последствий изменения границ охраняемых территорий

Скрипт демонстрирует, как геопространственный анализ и алгоритмы
оптимизации могут помочь в принятии решений при планировании полевых
исследований водных биоресурсов, особенно в условиях арктических морей,
где логистика особенно сложна и дорогостояща.

\begin{Shaded}
\begin{Highlighting}[]
\CommentTok{\# ========================================================================================================================}
\CommentTok{\# ПРАКТИЧЕСКОЕ ЗАНЯТИЕ: ПЛАНИРОВАНИЕ МАРШРУТА СЪЕМКИ ПРИ ОГРАНИЧЕННОМ ВРЕМЕННОМ РЕСУРСЕ}
\CommentTok{\# Курс: "Современные методы анализа данных в оценке водных биоресурсов"}
\CommentTok{\# Автор: Баканев С.В.}
\CommentTok{\# Дата: 24.04.2025}
\CommentTok{\# }
\CommentTok{\# Цель: Освоить методы геопространственного анализа и оптимизации для планирования научно{-}исследовательских съемок}
\CommentTok{\# }
\CommentTok{\# Структура:}
\CommentTok{\# 1. Загрузка пакетов и настройка среды}
\CommentTok{\# 2. Загрузка и подготовка исходных данных}
\CommentTok{\# 3. Создание расширенных полигонов}
\CommentTok{\# 4. Генерация схемы станций}
\CommentTok{\# 5. Построение оптимальных маршрутов}
\CommentTok{\# 6. Визуализация результатов}
\CommentTok{\# 7. Сравнительный анализ и экспорт результатов}
\CommentTok{\# }
\CommentTok{\# Описание: Скрипт демонстрирует подход к планированию морских исследований с использованием}
\CommentTok{\#           методов пространственного анализа и решения задачи коммивояжера (TSP).}
\CommentTok{\#           Моделируются различные сценарии расширения района работ с оценкой}
\CommentTok{\#           их влияния на протяженность маршрута.}
\CommentTok{\# ========================================================================================================================}

\CommentTok{\# ЗАГРУЗКА НЕОБХОДИМЫХ ПАКЕТОВ {-}{-}{-}{-}{-}{-}{-}{-}{-}{-}{-}{-}{-}{-}{-}{-}{-}{-}{-}{-}{-}{-}{-}{-}{-}{-}{-}{-}{-}{-}{-}{-}{-}{-}{-}{-}{-}{-}{-}{-}{-}{-}{-}{-}{-}{-}{-}}
\CommentTok{\# Все пакеты должны быть предварительно установлены (install.packages(...))}

\FunctionTok{library}\NormalTok{(sf)            }\CommentTok{\# Базовые операции с пространственными данными (вектор)}
\end{Highlighting}
\end{Shaded}

\begin{verbatim}
Linking to GEOS 3.13.1, GDAL 3.11.0, PROJ 9.6.0; sf_use_s2() is TRUE
\end{verbatim}

\begin{Shaded}
\begin{Highlighting}[]
\FunctionTok{library}\NormalTok{(tidyverse)     }\CommentTok{\# Метасборка пакетов для обработки и визуализации данных}
\end{Highlighting}
\end{Shaded}

\begin{verbatim}
-- Attaching core tidyverse packages ------------------------ tidyverse 2.0.0 --
v dplyr     1.1.4     v readr     2.1.5
v forcats   1.0.0     v stringr   1.5.2
v ggplot2   4.0.0     v tibble    3.2.1
v lubridate 1.9.4     v tidyr     1.3.1
v purrr     1.0.4     
\end{verbatim}

\begin{verbatim}
-- Conflicts ------------------------------------------ tidyverse_conflicts() --
x dplyr::filter() masks stats::filter()
x dplyr::lag()    masks stats::lag()
i Use the conflicted package (<http://conflicted.r-lib.org/>) to force all conflicts to become errors
\end{verbatim}

\begin{Shaded}
\begin{Highlighting}[]
\FunctionTok{library}\NormalTok{(rnaturalearth) }\CommentTok{\# Загрузка готовых полигональных карт мира (для фона)}
\FunctionTok{library}\NormalTok{(ggplot2)       }\CommentTok{\# Создание продвинутых графиков (входит в tidyverse, но подключаем для ясности)}
\FunctionTok{library}\NormalTok{(geosphere)     }\CommentTok{\# Геодезические расчеты на сфере (расчет расстояний между точками)}
\FunctionTok{library}\NormalTok{(TSP)           }\CommentTok{\# Решение "Задачи коммивояжера" (Traveling Salesman Problem)}


\CommentTok{\# 1. НАСТРОЙКА СРЕДЫ {-}{-}{-}{-}{-}{-}{-}{-}{-}{-}{-}{-}{-}{-}{-}{-}{-}{-}{-}{-}{-}{-}{-}{-}{-}{-}{-}{-}{-}{-}{-}{-}{-}{-}{-}{-}{-}{-}{-}{-}{-}{-}{-}{-}{-}{-}{-}{-}{-}{-}{-}{-}{-}{-}{-}{-}{-}{-}}
\CommentTok{\# Установка рабочей директории (замените на путь к своей папке)}
\FunctionTok{setwd}\NormalTok{(}\StringTok{"C:/SURVEY/"}\NormalTok{)}

\CommentTok{\# 2. ЗАГРУЗКА ИСХОДНЫХ ДАННЫХ {-}{-}{-}{-}{-}{-}{-}{-}{-}{-}{-}{-}{-}{-}{-}{-}{-}{-}{-}{-}{-}{-}{-}{-}{-}{-}{-}{-}{-}{-}{-}{-}{-}{-}{-}{-}{-}{-}{-}{-}{-}{-}{-}{-}{-}{-}{-}{-}{-}}
\CommentTok{\# Загрузка предварительно созданного файла с полигонами}
\FunctionTok{load}\NormalTok{(}\StringTok{"polygons.RData"}\NormalTok{)}

\CommentTok{\# Фильтрация полигона за 2020 год и извлечение его геометрии}
\NormalTok{polygon\_2020 }\OtherTok{\textless{}{-}}\NormalTok{ polygons }\SpecialCharTok{\%\textgreater{}\%}
  \FunctionTok{filter}\NormalTok{(YEAR }\SpecialCharTok{==} \DecValTok{2020}\NormalTok{) }\SpecialCharTok{\%\textgreater{}\%}
  \FunctionTok{st\_geometry}\NormalTok{()}

\CommentTok{\# 3. ПРЕОБРАЗОВАНИЕ СИСТЕМЫ КООРДИНАТ {-}{-}{-}{-}{-}{-}{-}{-}{-}{-}{-}{-}{-}{-}{-}{-}{-}{-}{-}{-}{-}{-}{-}{-}{-}{-}{-}{-}{-}{-}{-}{-}{-}{-}{-}{-}{-}{-}{-}{-}{-}}
\CommentTok{\# Преобразование из географических координат (WGS84) в проекцию UTM (зона 40N).}
\CommentTok{\# Это необходимо для корректного вычисления площадей и расстояний в метрах.}
\NormalTok{polygon\_2020\_utm }\OtherTok{\textless{}{-}} \FunctionTok{st\_transform}\NormalTok{(polygon\_2020, }\DecValTok{32640}\NormalTok{)}

\CommentTok{\# 4. РАСЧЕТ ПЛОЩАДИ ИСХОДНОГО ПОЛИГОНА {-}{-}{-}{-}{-}{-}{-}{-}{-}{-}{-}{-}{-}{-}{-}{-}{-}{-}{-}{-}{-}{-}{-}{-}{-}{-}{-}{-}{-}{-}{-}{-}{-}{-}{-}{-}{-}{-}{-}{-}}
\CommentTok{\# Вычисление площади в кв. метрах (st\_area) и конвертация в кв. километры (/ 1e6)}
\NormalTok{area\_km2\_original }\OtherTok{\textless{}{-}}\NormalTok{ (}\FunctionTok{st\_area}\NormalTok{(polygon\_2020\_utm) }\SpecialCharTok{/} \FloatTok{1e6}\NormalTok{) }\SpecialCharTok{\%\textgreater{}\%} \FunctionTok{as.numeric}\NormalTok{()}
\CommentTok{\# Вывод результата в консоль}
\FunctionTok{cat}\NormalTok{(}\StringTok{"Площадь полигона 2020 года: "}\NormalTok{, }\FunctionTok{round}\NormalTok{(area\_km2\_original, }\DecValTok{2}\NormalTok{), }\StringTok{" км²}\SpecialCharTok{\textbackslash{}n}\StringTok{"}\NormalTok{)}
\end{Highlighting}
\end{Shaded}

\begin{verbatim}
Площадь полигона 2020 года:  63101.31  км<U+00B2>
\end{verbatim}

\begin{Shaded}
\begin{Highlighting}[]
\CommentTok{\# 5. ФУНКЦИЯ ДЛЯ ЭКСПАНСИИ ПОЛИГОНА НА СЕВЕР {-}{-}{-}{-}{-}{-}{-}{-}{-}{-}{-}{-}{-}{-}{-}{-}{-}{-}{-}{-}{-}{-}{-}{-}{-}{-}{-}{-}{-}{-}{-}{-}{-}{-}{-}}
\CommentTok{\# Создание функции, которая "растягивает" северную часть полигона на заданный множитель.}
\CommentTok{\# Аргументы:}
\CommentTok{\#   poly {-} исходный полигон (в UTM)}
\CommentTok{\#   factor {-} коэффициент расширения (например, 1.5 {-} увеличить на 50\%)}
\NormalTok{expand\_polygon\_north }\OtherTok{\textless{}{-}} \ControlFlowTok{function}\NormalTok{(poly, factor) \{}
  \CommentTok{\# Извлечение координат вершин полигона и преобразование в DataFrame}
\NormalTok{  coords }\OtherTok{\textless{}{-}} \FunctionTok{st\_coordinates}\NormalTok{(poly)[, }\DecValTok{1}\SpecialCharTok{:}\DecValTok{2}\NormalTok{] }\SpecialCharTok{\%\textgreater{}\%}
    \FunctionTok{as.data.frame}\NormalTok{() }\SpecialCharTok{\%\textgreater{}\%}
    \FunctionTok{rename}\NormalTok{(}\AttributeTok{x =}\NormalTok{ X, }\AttributeTok{y =}\NormalTok{ Y)}

  \CommentTok{\# ЛОГИКА РАСШИРЕНИЯ:}
  \CommentTok{\# 1. Определяем "северную" часть полигона (верхние 30\% точек по оси Y)}
\NormalTok{  north\_threshold }\OtherTok{\textless{}{-}} \FunctionTok{quantile}\NormalTok{(coords}\SpecialCharTok{$}\NormalTok{y, }\FloatTok{0.7}\NormalTok{)}
  \CommentTok{\# 2. Находим общий разброс полигона по оси Y (высоту)}
\NormalTok{  y\_range }\OtherTok{\textless{}{-}} \FunctionTok{diff}\NormalTok{(}\FunctionTok{range}\NormalTok{(coords}\SpecialCharTok{$}\NormalTok{y))}
  \CommentTok{\# 3. Сдвигаем все северные точки на север на величину (factor {-} 1) * высоту\_полигона}
\NormalTok{  coords[coords}\SpecialCharTok{$}\NormalTok{y }\SpecialCharTok{\textgreater{}=}\NormalTok{ north\_threshold, }\StringTok{"y"}\NormalTok{] }\OtherTok{\textless{}{-}}
\NormalTok{    coords[coords}\SpecialCharTok{$}\NormalTok{y }\SpecialCharTok{\textgreater{}=}\NormalTok{ north\_threshold, }\StringTok{"y"}\NormalTok{] }\SpecialCharTok{+}\NormalTok{ y\_range }\SpecialCharTok{*}\NormalTok{ (factor }\SpecialCharTok{{-}} \DecValTok{1}\NormalTok{)}

  \CommentTok{\# Создание нового полигона из модифицированных точек:}
  \CommentTok{\# 1. Преобразование точек в spatial object}
  \CommentTok{\# 2. Объединение точек в один объект}
  \CommentTok{\# 3. Построение выпуклой оболочки для получения гладкого полигона}
\NormalTok{  expanded\_poly }\OtherTok{\textless{}{-}} \FunctionTok{st\_as\_sf}\NormalTok{(coords, }\AttributeTok{coords =} \FunctionTok{c}\NormalTok{(}\StringTok{"x"}\NormalTok{, }\StringTok{"y"}\NormalTok{), }\AttributeTok{crs =} \FunctionTok{st\_crs}\NormalTok{(poly)) }\SpecialCharTok{\%\textgreater{}\%}
    \FunctionTok{st\_combine}\NormalTok{() }\SpecialCharTok{\%\textgreater{}\%}
    \FunctionTok{st\_convex\_hull}\NormalTok{()}

  \FunctionTok{return}\NormalTok{(expanded\_poly)}
\NormalTok{\}}

\CommentTok{\# 6. СОЗДАНИЕ НАБОРА ЭКСПАНДИРОВАННЫХ ПОЛИГОНОВ {-}{-}{-}{-}{-}{-}{-}{-}{-}{-}{-}{-}{-}{-}{-}{-}{-}{-}{-}{-}{-}{-}{-}{-}{-}{-}{-}{-}{-}{-}{-}}
\CommentTok{\# Применение функции к коэффициентам 1.5, 2 и 3}
\NormalTok{factors }\OtherTok{\textless{}{-}} \FunctionTok{c}\NormalTok{(}\FloatTok{1.5}\NormalTok{, }\DecValTok{2}\NormalTok{, }\DecValTok{3}\NormalTok{)}
\NormalTok{expanded\_polygons }\OtherTok{\textless{}{-}} \FunctionTok{map}\NormalTok{(factors, }\SpecialCharTok{\textasciitilde{}} \FunctionTok{expand\_polygon\_north}\NormalTok{(polygon\_2020\_utm, .x))}

\CommentTok{\# 7. РАСЧЕТ ПЛОЩАДЕЙ НОВЫХ ПОЛИГОНОВ {-}{-}{-}{-}{-}{-}{-}{-}{-}{-}{-}{-}{-}{-}{-}{-}{-}{-}{-}{-}{-}{-}{-}{-}{-}{-}{-}{-}{-}{-}{-}{-}{-}{-}{-}{-}{-}{-}{-}{-}{-}{-}}
\NormalTok{areas\_expanded }\OtherTok{\textless{}{-}} \FunctionTok{map\_dbl}\NormalTok{(expanded\_polygons, }\SpecialCharTok{\textasciitilde{}}\NormalTok{ \{}
\NormalTok{  (}\FunctionTok{st\_area}\NormalTok{(.x) }\SpecialCharTok{/} \FloatTok{1e6}\NormalTok{) }\SpecialCharTok{\%\textgreater{}\%} \FunctionTok{as.numeric}\NormalTok{() }\CommentTok{\# Площадь в кв. км.}
\NormalTok{\})}

\CommentTok{\# 8. ФОРМИРОВАНИЕ ЕДИНОЙ ТАБЛИЦЫ С ВСЕМИ ПОЛИГОНАМИ {-}{-}{-}{-}{-}{-}{-}{-}{-}{-}{-}{-}{-}{-}{-}{-}{-}{-}{-}{-}{-}{-}{-}{-}{-}{-}{-}}
\CommentTok{\# Создание именованного списка всех полигонов}
\NormalTok{all\_polygons }\OtherTok{\textless{}{-}} \FunctionTok{list}\NormalTok{(}
  \AttributeTok{Original =}\NormalTok{ polygon\_2020\_utm,}
  \AttributeTok{Expanded\_x1\_5 =}\NormalTok{ expanded\_polygons[[}\DecValTok{1}\NormalTok{]],}
  \AttributeTok{Expanded\_x2 =}\NormalTok{ expanded\_polygons[[}\DecValTok{2}\NormalTok{]],}
  \AttributeTok{Expanded\_x3 =}\NormalTok{ expanded\_polygons[[}\DecValTok{3}\NormalTok{]]}
\NormalTok{)}

\CommentTok{\# Вектор меток и площадей}
\NormalTok{labels }\OtherTok{\textless{}{-}} \FunctionTok{c}\NormalTok{(}\StringTok{"Original"}\NormalTok{, }\StringTok{"Expanded\_x1\_5"}\NormalTok{, }\StringTok{"Expanded\_x2"}\NormalTok{, }\StringTok{"Expanded\_x3"}\NormalTok{)}
\NormalTok{areas }\OtherTok{\textless{}{-}} \FunctionTok{c}\NormalTok{(area\_km2\_original, areas\_expanded)}

\CommentTok{\# Комбинирование геометрий и создание итогового SF{-}объекта}
\NormalTok{geometry }\OtherTok{\textless{}{-}} \FunctionTok{do.call}\NormalTok{(c, all\_polygons) }\CommentTok{\# Объединение геометрий в один вектор}

\NormalTok{polygon\_df }\OtherTok{\textless{}{-}} \FunctionTok{tibble}\NormalTok{(}
  \AttributeTok{label =} \FunctionTok{factor}\NormalTok{(labels, }\AttributeTok{levels =}\NormalTok{ labels), }\CommentTok{\# Метка как фактор для сохранения порядка}
  \AttributeTok{area\_km2 =}\NormalTok{ areas,                        }\CommentTok{\# Площадь}
  \AttributeTok{geometry =} \FunctionTok{st\_sfc}\NormalTok{(geometry)              }\CommentTok{\# Геометрия}
\NormalTok{) }\SpecialCharTok{\%\textgreater{}\%}
  \FunctionTok{st\_as\_sf}\NormalTok{() }\CommentTok{\# Преобразование в пространственный объект}

\CommentTok{\# 9. ГЕНЕРАЦИЯ СЕТКИ ТОЧЕК (СТАНЦИЙ) ВНУТРИ КАЖДОГО ПОЛИГОНА {-}{-}{-}{-}{-}{-}{-}{-}{-}{-}{-}{-}{-}{-}{-}{-}{-}}
\CommentTok{\# Генерация 137 точек по регулярной сетке внутри каждого полигона.}
\CommentTok{\# Используется rowwise для применения функции к каждой строке{-}полигону.}
\NormalTok{sample\_points }\OtherTok{\textless{}{-}}\NormalTok{ polygon\_df }\SpecialCharTok{\%\textgreater{}\%}
  \FunctionTok{rowwise}\NormalTok{() }\SpecialCharTok{\%\textgreater{}\%}
  \FunctionTok{mutate}\NormalTok{(}\AttributeTok{points =} \FunctionTok{list}\NormalTok{(}\FunctionTok{st\_sample}\NormalTok{(geometry, }\AttributeTok{size =} \DecValTok{137}\NormalTok{, }\AttributeTok{type =} \StringTok{"regular"}\NormalTok{))) }\SpecialCharTok{\%\textgreater{}\%}
  \FunctionTok{ungroup}\NormalTok{()}

\CommentTok{\# 10. СОЗДАНИЕ ЕДИНОЙ ТАБЛИЦЫ ВСЕХ ТОЧЕК {-}{-}{-}{-}{-}{-}{-}{-}{-}{-}{-}{-}{-}{-}{-}{-}{-}{-}{-}{-}{-}{-}{-}{-}{-}{-}{-}{-}{-}{-}{-}{-}{-}{-}{-}{-}{-}{-}}
\CommentTok{\# Преобразование вложенного списка точек в плоскую таблицу}
\NormalTok{points\_list }\OtherTok{\textless{}{-}} \FunctionTok{map2}\NormalTok{(sample\_points}\SpecialCharTok{$}\NormalTok{geometry, sample\_points}\SpecialCharTok{$}\NormalTok{label, }\SpecialCharTok{\textasciitilde{}}\NormalTok{ \{}
  \FunctionTok{st\_sample}\NormalTok{(.x, }\AttributeTok{size =} \DecValTok{137}\NormalTok{, }\AttributeTok{type =} \StringTok{"regular"}\NormalTok{) }\SpecialCharTok{\%\textgreater{}\%} \CommentTok{\# Извлечение точек}
    \FunctionTok{st\_as\_sf}\NormalTok{() }\SpecialCharTok{\%\textgreater{}\%}                                \CommentTok{\# Конвертация в SF}
    \FunctionTok{mutate}\NormalTok{(}\AttributeTok{label =}\NormalTok{ .y)                            }\CommentTok{\# Добавление метки полигона}
\NormalTok{\})}

\CommentTok{\# Объединение всех точек в один DataFrame}
\NormalTok{points\_df }\OtherTok{\textless{}{-}} \FunctionTok{do.call}\NormalTok{(rbind, points\_list) }\SpecialCharTok{\%\textgreater{}\%}
  \FunctionTok{rename}\NormalTok{(}\AttributeTok{geometry =}\NormalTok{ x) }\SpecialCharTok{\%\textgreater{}\%}
  \FunctionTok{st\_set\_geometry}\NormalTok{(}\StringTok{"geometry"}\NormalTok{) }\SpecialCharTok{\%\textgreater{}\%}
  \FunctionTok{st\_set\_crs}\NormalTok{(}\FunctionTok{st\_crs}\NormalTok{(polygon\_2020\_utm)) }\CommentTok{\# Важно: явно задаем систему координат}

\CommentTok{\# 11. ПРЕОБРАЗОВАНИЕ В WGS84 ДЛЯ ВИЗУАЛИЗАЦИИ И РАСЧЕТОВ {-}{-}{-}{-}{-}{-}{-}{-}{-}{-}{-}{-}{-}{-}{-}{-}{-}{-}{-}{-}{-}{-}}
\CommentTok{\# Большинство картографических пакетов и функций расчета расстояний работают с WGS84}
\NormalTok{polygon\_wgs84 }\OtherTok{\textless{}{-}} \FunctionTok{st\_transform}\NormalTok{(polygon\_df, }\DecValTok{4326}\NormalTok{) }\CommentTok{\# WGS84 (широта/долгота)}
\NormalTok{points\_wgs84 }\OtherTok{\textless{}{-}} \FunctionTok{st\_transform}\NormalTok{(points\_df, }\DecValTok{4326}\NormalTok{)}

\CommentTok{\# 12. ЗАГРУЗКА ФОНОВОЙ КАРТЫ (ГРАНИЦЫ РОССИИ) {-}{-}{-}{-}{-}{-}{-}{-}{-}{-}{-}{-}{-}{-}{-}{-}{-}{-}{-}{-}{-}{-}{-}{-}{-}{-}{-}{-}{-}{-}{-}{-}{-}}
\NormalTok{russia }\OtherTok{\textless{}{-}} \FunctionTok{ne\_countries}\NormalTok{(}\AttributeTok{scale =} \StringTok{"medium"}\NormalTok{, }\AttributeTok{country =} \StringTok{"Russia"}\NormalTok{, }\AttributeTok{returnclass =} \StringTok{"sf"}\NormalTok{) }\SpecialCharTok{\%\textgreater{}\%}
  \FunctionTok{st\_transform}\NormalTok{(}\DecValTok{4326}\NormalTok{) }\CommentTok{\# Преобразование в WGS84}

\CommentTok{\# 13. ВИЗУАЛИЗАЦИЯ: ПОЛИГОНЫ И ТОЧКИ {-}{-}{-}{-}{-}{-}{-}{-}{-}{-}{-}{-}{-}{-}{-}{-}{-}{-}{-}{-}{-}{-}{-}{-}{-}{-}{-}{-}{-}{-}{-}{-}{-}{-}{-}{-}{-}{-}{-}{-}{-}{-}}
\CommentTok{\# Построение карты с фацетами (subplots) для каждого варианта полигона}
\FunctionTok{ggplot}\NormalTok{() }\SpecialCharTok{+}
  \FunctionTok{geom\_sf}\NormalTok{(}\AttributeTok{data =}\NormalTok{ russia, }\AttributeTok{fill =} \StringTok{"lightgray"}\NormalTok{, }\AttributeTok{color =} \StringTok{"black"}\NormalTok{, }\AttributeTok{linewidth =} \FloatTok{0.3}\NormalTok{) }\SpecialCharTok{+} \CommentTok{\# Фон}
  \FunctionTok{geom\_sf}\NormalTok{(}\AttributeTok{data =}\NormalTok{ polygon\_wgs84, }\FunctionTok{aes}\NormalTok{(}\AttributeTok{fill =}\NormalTok{ label), }\AttributeTok{alpha =} \FloatTok{0.6}\NormalTok{, }\AttributeTok{color =} \StringTok{"darkred"}\NormalTok{, }\AttributeTok{linewidth =} \FloatTok{0.5}\NormalTok{) }\SpecialCharTok{+} \CommentTok{\# Полигоны}
  \FunctionTok{geom\_sf}\NormalTok{(}\AttributeTok{data =}\NormalTok{ points\_wgs84, }\FunctionTok{aes}\NormalTok{(}\AttributeTok{color =}\NormalTok{ label), }\AttributeTok{shape =} \DecValTok{16}\NormalTok{, }\AttributeTok{size =} \DecValTok{1}\NormalTok{) }\SpecialCharTok{+} \CommentTok{\# Точки}
  \FunctionTok{facet\_wrap}\NormalTok{(}\SpecialCharTok{\textasciitilde{}}\NormalTok{ label, }\AttributeTok{scales =} \StringTok{"fixed"}\NormalTok{) }\SpecialCharTok{+} \CommentTok{\# Фацеты по варианту полигона}
  \FunctionTok{coord\_sf}\NormalTok{(}\AttributeTok{xlim =} \FunctionTok{c}\NormalTok{(}\DecValTok{35}\NormalTok{, }\DecValTok{50}\NormalTok{), }\AttributeTok{ylim =} \FunctionTok{c}\NormalTok{(}\DecValTok{68}\NormalTok{, }\DecValTok{75}\NormalTok{)) }\SpecialCharTok{+} \CommentTok{\# Обрезка карты до нужного региона}
  \FunctionTok{labs}\NormalTok{(}
    \AttributeTok{title =} \StringTok{"Полигоны с разной площадью и траловыми станциями"}\NormalTok{,}
    \AttributeTok{subtitle =} \StringTok{"Регулярное распределение 137 точек внутри каждого полигона"}\NormalTok{,}
    \AttributeTok{fill =} \StringTok{"Вариант"}\NormalTok{, }\AttributeTok{color =} \StringTok{"Вариант"}
\NormalTok{  ) }\SpecialCharTok{+}
  \FunctionTok{theme\_minimal}\NormalTok{() }\SpecialCharTok{+}
  \FunctionTok{theme}\NormalTok{(}\AttributeTok{strip.background =} \FunctionTok{element\_rect}\NormalTok{(}\AttributeTok{fill =} \StringTok{"lightblue"}\NormalTok{))}
\end{Highlighting}
\end{Shaded}

\begin{verbatim}
Warning in grid.Call(C_textBounds, as.graphicsAnnot(x$label), x$x, x$y, :
неизвестна ширина символа 0xc2 в кодировке CP1251
\end{verbatim}

\begin{verbatim}
Warning in grid.Call(C_textBounds, as.graphicsAnnot(x$label), x$x, x$y, :
неизвестна ширина символа 0xe0 в кодировке CP1251
\end{verbatim}

\begin{verbatim}
Warning in grid.Call(C_textBounds, as.graphicsAnnot(x$label), x$x, x$y, :
неизвестна ширина символа 0xf0 в кодировке CP1251
\end{verbatim}

\begin{verbatim}
Warning in grid.Call(C_textBounds, as.graphicsAnnot(x$label), x$x, x$y, :
неизвестна ширина символа 0xe8 в кодировке CP1251
\end{verbatim}

\begin{verbatim}
Warning in grid.Call(C_textBounds, as.graphicsAnnot(x$label), x$x, x$y, :
неизвестна ширина символа 0xe0 в кодировке CP1251
\end{verbatim}

\begin{verbatim}
Warning in grid.Call(C_textBounds, as.graphicsAnnot(x$label), x$x, x$y, :
неизвестна ширина символа 0xed в кодировке CP1251
\end{verbatim}

\begin{verbatim}
Warning in grid.Call(C_textBounds, as.graphicsAnnot(x$label), x$x, x$y, :
неизвестна ширина символа 0xf2 в кодировке CP1251
\end{verbatim}

\begin{verbatim}
Warning in grid.Call(C_textBounds, as.graphicsAnnot(x$label), x$x, x$y, :
неизвестна ширина символа 0xc2 в кодировке CP1251
\end{verbatim}

\begin{verbatim}
Warning in grid.Call(C_textBounds, as.graphicsAnnot(x$label), x$x, x$y, :
неизвестна ширина символа 0xe0 в кодировке CP1251
\end{verbatim}

\begin{verbatim}
Warning in grid.Call(C_textBounds, as.graphicsAnnot(x$label), x$x, x$y, :
неизвестна ширина символа 0xf0 в кодировке CP1251
\end{verbatim}

\begin{verbatim}
Warning in grid.Call(C_textBounds, as.graphicsAnnot(x$label), x$x, x$y, :
неизвестна ширина символа 0xe8 в кодировке CP1251
\end{verbatim}

\begin{verbatim}
Warning in grid.Call(C_textBounds, as.graphicsAnnot(x$label), x$x, x$y, :
неизвестна ширина символа 0xe0 в кодировке CP1251
\end{verbatim}

\begin{verbatim}
Warning in grid.Call(C_textBounds, as.graphicsAnnot(x$label), x$x, x$y, :
неизвестна ширина символа 0xed в кодировке CP1251
\end{verbatim}

\begin{verbatim}
Warning in grid.Call(C_textBounds, as.graphicsAnnot(x$label), x$x, x$y, :
неизвестна ширина символа 0xf2 в кодировке CP1251
\end{verbatim}

\begin{verbatim}
Warning in grid.Call(C_textBounds, as.graphicsAnnot(x$label), x$x, x$y, :
неизвестна ширина символа 0xcf в кодировке CP1251
\end{verbatim}

\begin{verbatim}
Warning in grid.Call(C_textBounds, as.graphicsAnnot(x$label), x$x, x$y, :
неизвестна ширина символа 0xee в кодировке CP1251
\end{verbatim}

\begin{verbatim}
Warning in grid.Call(C_textBounds, as.graphicsAnnot(x$label), x$x, x$y, :
неизвестна ширина символа 0xeb в кодировке CP1251
\end{verbatim}

\begin{verbatim}
Warning in grid.Call(C_textBounds, as.graphicsAnnot(x$label), x$x, x$y, :
неизвестна ширина символа 0xe8 в кодировке CP1251
\end{verbatim}

\begin{verbatim}
Warning in grid.Call(C_textBounds, as.graphicsAnnot(x$label), x$x, x$y, :
неизвестна ширина символа 0xe3 в кодировке CP1251
\end{verbatim}

\begin{verbatim}
Warning in grid.Call(C_textBounds, as.graphicsAnnot(x$label), x$x, x$y, :
неизвестна ширина символа 0xee в кодировке CP1251
\end{verbatim}

\begin{verbatim}
Warning in grid.Call(C_textBounds, as.graphicsAnnot(x$label), x$x, x$y, :
неизвестна ширина символа 0xed в кодировке CP1251
\end{verbatim}

\begin{verbatim}
Warning in grid.Call(C_textBounds, as.graphicsAnnot(x$label), x$x, x$y, :
неизвестна ширина символа 0xfb в кодировке CP1251
\end{verbatim}

\begin{verbatim}
Warning in grid.Call(C_textBounds, as.graphicsAnnot(x$label), x$x, x$y, :
неизвестна ширина символа 0xf1 в кодировке CP1251
\end{verbatim}

\begin{verbatim}
Warning in grid.Call(C_textBounds, as.graphicsAnnot(x$label), x$x, x$y, :
неизвестна ширина символа 0xf0 в кодировке CP1251
\end{verbatim}

\begin{verbatim}
Warning in grid.Call(C_textBounds, as.graphicsAnnot(x$label), x$x, x$y, :
неизвестна ширина символа 0xe0 в кодировке CP1251
\end{verbatim}

\begin{verbatim}
Warning in grid.Call(C_textBounds, as.graphicsAnnot(x$label), x$x, x$y, :
неизвестна ширина символа 0xe7 в кодировке CP1251
\end{verbatim}

\begin{verbatim}
Warning in grid.Call(C_textBounds, as.graphicsAnnot(x$label), x$x, x$y, :
неизвестна ширина символа 0xed в кодировке CP1251
\end{verbatim}

\begin{verbatim}
Warning in grid.Call(C_textBounds, as.graphicsAnnot(x$label), x$x, x$y, :
неизвестна ширина символа 0xee в кодировке CP1251
\end{verbatim}

\begin{verbatim}
Warning in grid.Call(C_textBounds, as.graphicsAnnot(x$label), x$x, x$y, :
неизвестна ширина символа 0xe9 в кодировке CP1251
\end{verbatim}

\begin{verbatim}
Warning in grid.Call(C_textBounds, as.graphicsAnnot(x$label), x$x, x$y, :
неизвестна ширина символа 0xef в кодировке CP1251
\end{verbatim}

\begin{verbatim}
Warning in grid.Call(C_textBounds, as.graphicsAnnot(x$label), x$x, x$y, :
неизвестна ширина символа 0xeb в кодировке CP1251
\end{verbatim}

\begin{verbatim}
Warning in grid.Call(C_textBounds, as.graphicsAnnot(x$label), x$x, x$y, :
неизвестна ширина символа 0xee в кодировке CP1251
\end{verbatim}

\begin{verbatim}
Warning in grid.Call(C_textBounds, as.graphicsAnnot(x$label), x$x, x$y, :
неизвестна ширина символа 0xf9 в кодировке CP1251
\end{verbatim}

\begin{verbatim}
Warning in grid.Call(C_textBounds, as.graphicsAnnot(x$label), x$x, x$y, :
неизвестна ширина символа 0xe0 в кодировке CP1251
\end{verbatim}

\begin{verbatim}
Warning in grid.Call(C_textBounds, as.graphicsAnnot(x$label), x$x, x$y, :
неизвестна ширина символа 0xe4 в кодировке CP1251
\end{verbatim}

\begin{verbatim}
Warning in grid.Call(C_textBounds, as.graphicsAnnot(x$label), x$x, x$y, :
неизвестна ширина символа 0xfc в кодировке CP1251
\end{verbatim}

\begin{verbatim}
Warning in grid.Call(C_textBounds, as.graphicsAnnot(x$label), x$x, x$y, :
неизвестна ширина символа 0xfe в кодировке CP1251
\end{verbatim}

\begin{verbatim}
Warning in grid.Call(C_textBounds, as.graphicsAnnot(x$label), x$x, x$y, :
неизвестна ширина символа 0xe8 в кодировке CP1251
\end{verbatim}

\begin{verbatim}
Warning in grid.Call(C_textBounds, as.graphicsAnnot(x$label), x$x, x$y, :
неизвестна ширина символа 0xf2 в кодировке CP1251
\end{verbatim}

\begin{verbatim}
Warning in grid.Call(C_textBounds, as.graphicsAnnot(x$label), x$x, x$y, :
неизвестна ширина символа 0xf0 в кодировке CP1251
\end{verbatim}

\begin{verbatim}
Warning in grid.Call(C_textBounds, as.graphicsAnnot(x$label), x$x, x$y, :
неизвестна ширина символа 0xe0 в кодировке CP1251
\end{verbatim}

\begin{verbatim}
Warning in grid.Call(C_textBounds, as.graphicsAnnot(x$label), x$x, x$y, :
неизвестна ширина символа 0xeb в кодировке CP1251
\end{verbatim}

\begin{verbatim}
Warning in grid.Call(C_textBounds, as.graphicsAnnot(x$label), x$x, x$y, :
неизвестна ширина символа 0xee в кодировке CP1251
\end{verbatim}

\begin{verbatim}
Warning in grid.Call(C_textBounds, as.graphicsAnnot(x$label), x$x, x$y, :
неизвестна ширина символа 0xe2 в кодировке CP1251
\end{verbatim}

\begin{verbatim}
Warning in grid.Call(C_textBounds, as.graphicsAnnot(x$label), x$x, x$y, :
неизвестна ширина символа 0xfb в кодировке CP1251
\end{verbatim}

\begin{verbatim}
Warning in grid.Call(C_textBounds, as.graphicsAnnot(x$label), x$x, x$y, :
неизвестна ширина символа 0xec в кодировке CP1251
\end{verbatim}

\begin{verbatim}
Warning in grid.Call(C_textBounds, as.graphicsAnnot(x$label), x$x, x$y, :
неизвестна ширина символа 0xe8 в кодировке CP1251
\end{verbatim}

\begin{verbatim}
Warning in grid.Call(C_textBounds, as.graphicsAnnot(x$label), x$x, x$y, :
неизвестна ширина символа 0xf1 в кодировке CP1251
\end{verbatim}

\begin{verbatim}
Warning in grid.Call(C_textBounds, as.graphicsAnnot(x$label), x$x, x$y, :
неизвестна ширина символа 0xf2 в кодировке CP1251
\end{verbatim}

\begin{verbatim}
Warning in grid.Call(C_textBounds, as.graphicsAnnot(x$label), x$x, x$y, :
неизвестна ширина символа 0xe0 в кодировке CP1251
\end{verbatim}

\begin{verbatim}
Warning in grid.Call(C_textBounds, as.graphicsAnnot(x$label), x$x, x$y, :
неизвестна ширина символа 0xed в кодировке CP1251
\end{verbatim}

\begin{verbatim}
Warning in grid.Call(C_textBounds, as.graphicsAnnot(x$label), x$x, x$y, :
неизвестна ширина символа 0xf6 в кодировке CP1251
\end{verbatim}

\begin{verbatim}
Warning in grid.Call(C_textBounds, as.graphicsAnnot(x$label), x$x, x$y, :
неизвестна ширина символа 0xe8 в кодировке CP1251
\end{verbatim}

\begin{verbatim}
Warning in grid.Call(C_textBounds, as.graphicsAnnot(x$label), x$x, x$y, :
неизвестна ширина символа 0xff в кодировке CP1251
\end{verbatim}

\begin{verbatim}
Warning in grid.Call(C_textBounds, as.graphicsAnnot(x$label), x$x, x$y, :
неизвестна ширина символа 0xec в кодировке CP1251
\end{verbatim}

\begin{verbatim}
Warning in grid.Call(C_textBounds, as.graphicsAnnot(x$label), x$x, x$y, :
неизвестна ширина символа 0xe8 в кодировке CP1251
\end{verbatim}

\begin{verbatim}
Warning in grid.Call(C_textBounds, as.graphicsAnnot(x$label), x$x, x$y, :
неизвестна ширина символа 0xd0 в кодировке CP1251
\end{verbatim}

\begin{verbatim}
Warning in grid.Call(C_textBounds, as.graphicsAnnot(x$label), x$x, x$y, :
неизвестна ширина символа 0xe5 в кодировке CP1251
\end{verbatim}

\begin{verbatim}
Warning in grid.Call(C_textBounds, as.graphicsAnnot(x$label), x$x, x$y, :
неизвестна ширина символа 0xe3 в кодировке CP1251
\end{verbatim}

\begin{verbatim}
Warning in grid.Call(C_textBounds, as.graphicsAnnot(x$label), x$x, x$y, :
неизвестна ширина символа 0xf3 в кодировке CP1251
\end{verbatim}

\begin{verbatim}
Warning in grid.Call(C_textBounds, as.graphicsAnnot(x$label), x$x, x$y, :
неизвестна ширина символа 0xeb в кодировке CP1251
\end{verbatim}

\begin{verbatim}
Warning in grid.Call(C_textBounds, as.graphicsAnnot(x$label), x$x, x$y, :
неизвестна ширина символа 0xff в кодировке CP1251
\end{verbatim}

\begin{verbatim}
Warning in grid.Call(C_textBounds, as.graphicsAnnot(x$label), x$x, x$y, :
неизвестна ширина символа 0xf0 в кодировке CP1251
\end{verbatim}

\begin{verbatim}
Warning in grid.Call(C_textBounds, as.graphicsAnnot(x$label), x$x, x$y, :
неизвестна ширина символа 0xed в кодировке CP1251
\end{verbatim}

\begin{verbatim}
Warning in grid.Call(C_textBounds, as.graphicsAnnot(x$label), x$x, x$y, :
неизвестна ширина символа 0xee в кодировке CP1251
\end{verbatim}

\begin{verbatim}
Warning in grid.Call(C_textBounds, as.graphicsAnnot(x$label), x$x, x$y, :
неизвестна ширина символа 0xe5 в кодировке CP1251
\end{verbatim}

\begin{verbatim}
Warning in grid.Call(C_textBounds, as.graphicsAnnot(x$label), x$x, x$y, :
неизвестна ширина символа 0xf0 в кодировке CP1251
\end{verbatim}

\begin{verbatim}
Warning in grid.Call(C_textBounds, as.graphicsAnnot(x$label), x$x, x$y, :
неизвестна ширина символа 0xe0 в кодировке CP1251
\end{verbatim}

\begin{verbatim}
Warning in grid.Call(C_textBounds, as.graphicsAnnot(x$label), x$x, x$y, :
неизвестна ширина символа 0xf1 в кодировке CP1251
\end{verbatim}

\begin{verbatim}
Warning in grid.Call(C_textBounds, as.graphicsAnnot(x$label), x$x, x$y, :
неизвестна ширина символа 0xef в кодировке CP1251
\end{verbatim}

\begin{verbatim}
Warning in grid.Call(C_textBounds, as.graphicsAnnot(x$label), x$x, x$y, :
неизвестна ширина символа 0xf0 в кодировке CP1251
\end{verbatim}

\begin{verbatim}
Warning in grid.Call(C_textBounds, as.graphicsAnnot(x$label), x$x, x$y, :
неизвестна ширина символа 0xe5 в кодировке CP1251
\end{verbatim}

\begin{verbatim}
Warning in grid.Call(C_textBounds, as.graphicsAnnot(x$label), x$x, x$y, :
неизвестна ширина символа 0xe4 в кодировке CP1251
\end{verbatim}

\begin{verbatim}
Warning in grid.Call(C_textBounds, as.graphicsAnnot(x$label), x$x, x$y, :
неизвестна ширина символа 0xe5 в кодировке CP1251
\end{verbatim}

\begin{verbatim}
Warning in grid.Call(C_textBounds, as.graphicsAnnot(x$label), x$x, x$y, :
неизвестна ширина символа 0xeb в кодировке CP1251
\end{verbatim}

\begin{verbatim}
Warning in grid.Call(C_textBounds, as.graphicsAnnot(x$label), x$x, x$y, :
неизвестна ширина символа 0xe5 в кодировке CP1251
\end{verbatim}

\begin{verbatim}
Warning in grid.Call(C_textBounds, as.graphicsAnnot(x$label), x$x, x$y, :
неизвестна ширина символа 0xed в кодировке CP1251
\end{verbatim}

\begin{verbatim}
Warning in grid.Call(C_textBounds, as.graphicsAnnot(x$label), x$x, x$y, :
неизвестна ширина символа 0xe8 в кодировке CP1251
\end{verbatim}

\begin{verbatim}
Warning in grid.Call(C_textBounds, as.graphicsAnnot(x$label), x$x, x$y, :
неизвестна ширина символа 0xe5 в кодировке CP1251
\end{verbatim}

\begin{verbatim}
Warning in grid.Call(C_textBounds, as.graphicsAnnot(x$label), x$x, x$y, :
неизвестна ширина символа 0xf2 в кодировке CP1251
\end{verbatim}

\begin{verbatim}
Warning in grid.Call(C_textBounds, as.graphicsAnnot(x$label), x$x, x$y, :
неизвестна ширина символа 0xee в кодировке CP1251
\end{verbatim}

\begin{verbatim}
Warning in grid.Call(C_textBounds, as.graphicsAnnot(x$label), x$x, x$y, :
неизвестна ширина символа 0xf7 в кодировке CP1251
\end{verbatim}

\begin{verbatim}
Warning in grid.Call(C_textBounds, as.graphicsAnnot(x$label), x$x, x$y, :
неизвестна ширина символа 0xe5 в кодировке CP1251
\end{verbatim}

\begin{verbatim}
Warning in grid.Call(C_textBounds, as.graphicsAnnot(x$label), x$x, x$y, :
неизвестна ширина символа 0xea в кодировке CP1251
\end{verbatim}

\begin{verbatim}
Warning in grid.Call(C_textBounds, as.graphicsAnnot(x$label), x$x, x$y, :
неизвестна ширина символа 0xe2 в кодировке CP1251
\end{verbatim}

\begin{verbatim}
Warning in grid.Call(C_textBounds, as.graphicsAnnot(x$label), x$x, x$y, :
неизвестна ширина символа 0xed в кодировке CP1251
\end{verbatim}

\begin{verbatim}
Warning in grid.Call(C_textBounds, as.graphicsAnnot(x$label), x$x, x$y, :
неизвестна ширина символа 0xf3 в кодировке CP1251
\end{verbatim}

\begin{verbatim}
Warning in grid.Call(C_textBounds, as.graphicsAnnot(x$label), x$x, x$y, :
неизвестна ширина символа 0xf2 в кодировке CP1251
\end{verbatim}

\begin{verbatim}
Warning in grid.Call(C_textBounds, as.graphicsAnnot(x$label), x$x, x$y, :
неизвестна ширина символа 0xf0 в кодировке CP1251
\end{verbatim}

\begin{verbatim}
Warning in grid.Call(C_textBounds, as.graphicsAnnot(x$label), x$x, x$y, :
неизвестна ширина символа 0xe8 в кодировке CP1251
\end{verbatim}

\begin{verbatim}
Warning in grid.Call(C_textBounds, as.graphicsAnnot(x$label), x$x, x$y, :
неизвестна ширина символа 0xea в кодировке CP1251
\end{verbatim}

\begin{verbatim}
Warning in grid.Call(C_textBounds, as.graphicsAnnot(x$label), x$x, x$y, :
неизвестна ширина символа 0xe0 в кодировке CP1251
\end{verbatim}

\begin{verbatim}
Warning in grid.Call(C_textBounds, as.graphicsAnnot(x$label), x$x, x$y, :
неизвестна ширина символа 0xe6 в кодировке CP1251
\end{verbatim}

\begin{verbatim}
Warning in grid.Call(C_textBounds, as.graphicsAnnot(x$label), x$x, x$y, :
неизвестна ширина символа 0xe4 в кодировке CP1251
\end{verbatim}

\begin{verbatim}
Warning in grid.Call(C_textBounds, as.graphicsAnnot(x$label), x$x, x$y, :
неизвестна ширина символа 0xee в кодировке CP1251
\end{verbatim}

\begin{verbatim}
Warning in grid.Call(C_textBounds, as.graphicsAnnot(x$label), x$x, x$y, :
неизвестна ширина символа 0xe3 в кодировке CP1251
\end{verbatim}

\begin{verbatim}
Warning in grid.Call(C_textBounds, as.graphicsAnnot(x$label), x$x, x$y, :
неизвестна ширина символа 0xee в кодировке CP1251
\end{verbatim}

\begin{verbatim}
Warning in grid.Call(C_textBounds, as.graphicsAnnot(x$label), x$x, x$y, :
неизвестна ширина символа 0xef в кодировке CP1251
\end{verbatim}

\begin{verbatim}
Warning in grid.Call(C_textBounds, as.graphicsAnnot(x$label), x$x, x$y, :
неизвестна ширина символа 0xee в кодировке CP1251
\end{verbatim}

\begin{verbatim}
Warning in grid.Call(C_textBounds, as.graphicsAnnot(x$label), x$x, x$y, :
неизвестна ширина символа 0xeb в кодировке CP1251
\end{verbatim}

\begin{verbatim}
Warning in grid.Call(C_textBounds, as.graphicsAnnot(x$label), x$x, x$y, :
неизвестна ширина символа 0xe8 в кодировке CP1251
\end{verbatim}

\begin{verbatim}
Warning in grid.Call(C_textBounds, as.graphicsAnnot(x$label), x$x, x$y, :
неизвестна ширина символа 0xe3 в кодировке CP1251
\end{verbatim}

\begin{verbatim}
Warning in grid.Call(C_textBounds, as.graphicsAnnot(x$label), x$x, x$y, :
неизвестна ширина символа 0xee в кодировке CP1251
\end{verbatim}

\begin{verbatim}
Warning in grid.Call(C_textBounds, as.graphicsAnnot(x$label), x$x, x$y, :
неизвестна ширина символа 0xed в кодировке CP1251
\end{verbatim}

\begin{verbatim}
Warning in grid.Call(C_textBounds, as.graphicsAnnot(x$label), x$x, x$y, :
неизвестна ширина символа 0xe0 в кодировке CP1251
\end{verbatim}

\begin{verbatim}
Warning in grid.Call.graphics(C_text, as.graphicsAnnot(x$label), x$x, x$y, :
неизвестна ширина символа 0xc2 в кодировке CP1251
\end{verbatim}

\begin{verbatim}
Warning in grid.Call.graphics(C_text, as.graphicsAnnot(x$label), x$x, x$y, :
неизвестна ширина символа 0xe0 в кодировке CP1251
\end{verbatim}

\begin{verbatim}
Warning in grid.Call.graphics(C_text, as.graphicsAnnot(x$label), x$x, x$y, :
неизвестна ширина символа 0xf0 в кодировке CP1251
\end{verbatim}

\begin{verbatim}
Warning in grid.Call.graphics(C_text, as.graphicsAnnot(x$label), x$x, x$y, :
неизвестна ширина символа 0xe8 в кодировке CP1251
\end{verbatim}

\begin{verbatim}
Warning in grid.Call.graphics(C_text, as.graphicsAnnot(x$label), x$x, x$y, :
неизвестна ширина символа 0xe0 в кодировке CP1251
\end{verbatim}

\begin{verbatim}
Warning in grid.Call.graphics(C_text, as.graphicsAnnot(x$label), x$x, x$y, :
неизвестна ширина символа 0xed в кодировке CP1251
\end{verbatim}

\begin{verbatim}
Warning in grid.Call.graphics(C_text, as.graphicsAnnot(x$label), x$x, x$y, :
неизвестна ширина символа 0xf2 в кодировке CP1251
\end{verbatim}

\begin{verbatim}
Warning in grid.Call.graphics(C_text, as.graphicsAnnot(x$label), x$x, x$y, :
неизвестна ширина символа 0xd0 в кодировке CP1251
\end{verbatim}

\begin{verbatim}
Warning in grid.Call.graphics(C_text, as.graphicsAnnot(x$label), x$x, x$y, :
неизвестна ширина символа 0xe5 в кодировке CP1251
\end{verbatim}

\begin{verbatim}
Warning in grid.Call.graphics(C_text, as.graphicsAnnot(x$label), x$x, x$y, :
неизвестна ширина символа 0xe3 в кодировке CP1251
\end{verbatim}

\begin{verbatim}
Warning in grid.Call.graphics(C_text, as.graphicsAnnot(x$label), x$x, x$y, :
неизвестна ширина символа 0xf3 в кодировке CP1251
\end{verbatim}

\begin{verbatim}
Warning in grid.Call.graphics(C_text, as.graphicsAnnot(x$label), x$x, x$y, :
неизвестна ширина символа 0xeb в кодировке CP1251
\end{verbatim}

\begin{verbatim}
Warning in grid.Call.graphics(C_text, as.graphicsAnnot(x$label), x$x, x$y, :
неизвестна ширина символа 0xff в кодировке CP1251
\end{verbatim}

\begin{verbatim}
Warning in grid.Call.graphics(C_text, as.graphicsAnnot(x$label), x$x, x$y, :
неизвестна ширина символа 0xf0 в кодировке CP1251
\end{verbatim}

\begin{verbatim}
Warning in grid.Call.graphics(C_text, as.graphicsAnnot(x$label), x$x, x$y, :
неизвестна ширина символа 0xed в кодировке CP1251
\end{verbatim}

\begin{verbatim}
Warning in grid.Call.graphics(C_text, as.graphicsAnnot(x$label), x$x, x$y, :
неизвестна ширина символа 0xee в кодировке CP1251
\end{verbatim}

\begin{verbatim}
Warning in grid.Call.graphics(C_text, as.graphicsAnnot(x$label), x$x, x$y, :
неизвестна ширина символа 0xe5 в кодировке CP1251
\end{verbatim}

\begin{verbatim}
Warning in grid.Call.graphics(C_text, as.graphicsAnnot(x$label), x$x, x$y, :
неизвестна ширина символа 0xf0 в кодировке CP1251
\end{verbatim}

\begin{verbatim}
Warning in grid.Call.graphics(C_text, as.graphicsAnnot(x$label), x$x, x$y, :
неизвестна ширина символа 0xe0 в кодировке CP1251
\end{verbatim}

\begin{verbatim}
Warning in grid.Call.graphics(C_text, as.graphicsAnnot(x$label), x$x, x$y, :
неизвестна ширина символа 0xf1 в кодировке CP1251
\end{verbatim}

\begin{verbatim}
Warning in grid.Call.graphics(C_text, as.graphicsAnnot(x$label), x$x, x$y, :
неизвестна ширина символа 0xef в кодировке CP1251
\end{verbatim}

\begin{verbatim}
Warning in grid.Call.graphics(C_text, as.graphicsAnnot(x$label), x$x, x$y, :
неизвестна ширина символа 0xf0 в кодировке CP1251
\end{verbatim}

\begin{verbatim}
Warning in grid.Call.graphics(C_text, as.graphicsAnnot(x$label), x$x, x$y, :
неизвестна ширина символа 0xe5 в кодировке CP1251
\end{verbatim}

\begin{verbatim}
Warning in grid.Call.graphics(C_text, as.graphicsAnnot(x$label), x$x, x$y, :
неизвестна ширина символа 0xe4 в кодировке CP1251
\end{verbatim}

\begin{verbatim}
Warning in grid.Call.graphics(C_text, as.graphicsAnnot(x$label), x$x, x$y, :
неизвестна ширина символа 0xe5 в кодировке CP1251
\end{verbatim}

\begin{verbatim}
Warning in grid.Call.graphics(C_text, as.graphicsAnnot(x$label), x$x, x$y, :
неизвестна ширина символа 0xeb в кодировке CP1251
\end{verbatim}

\begin{verbatim}
Warning in grid.Call.graphics(C_text, as.graphicsAnnot(x$label), x$x, x$y, :
неизвестна ширина символа 0xe5 в кодировке CP1251
\end{verbatim}

\begin{verbatim}
Warning in grid.Call.graphics(C_text, as.graphicsAnnot(x$label), x$x, x$y, :
неизвестна ширина символа 0xed в кодировке CP1251
\end{verbatim}

\begin{verbatim}
Warning in grid.Call.graphics(C_text, as.graphicsAnnot(x$label), x$x, x$y, :
неизвестна ширина символа 0xe8 в кодировке CP1251
\end{verbatim}

\begin{verbatim}
Warning in grid.Call.graphics(C_text, as.graphicsAnnot(x$label), x$x, x$y, :
неизвестна ширина символа 0xe5 в кодировке CP1251
\end{verbatim}

\begin{verbatim}
Warning in grid.Call.graphics(C_text, as.graphicsAnnot(x$label), x$x, x$y, :
неизвестна ширина символа 0xf2 в кодировке CP1251
\end{verbatim}

\begin{verbatim}
Warning in grid.Call.graphics(C_text, as.graphicsAnnot(x$label), x$x, x$y, :
неизвестна ширина символа 0xee в кодировке CP1251
\end{verbatim}

\begin{verbatim}
Warning in grid.Call.graphics(C_text, as.graphicsAnnot(x$label), x$x, x$y, :
неизвестна ширина символа 0xf7 в кодировке CP1251
\end{verbatim}

\begin{verbatim}
Warning in grid.Call.graphics(C_text, as.graphicsAnnot(x$label), x$x, x$y, :
неизвестна ширина символа 0xe5 в кодировке CP1251
\end{verbatim}

\begin{verbatim}
Warning in grid.Call.graphics(C_text, as.graphicsAnnot(x$label), x$x, x$y, :
неизвестна ширина символа 0xea в кодировке CP1251
\end{verbatim}

\begin{verbatim}
Warning in grid.Call.graphics(C_text, as.graphicsAnnot(x$label), x$x, x$y, :
неизвестна ширина символа 0xe2 в кодировке CP1251
\end{verbatim}

\begin{verbatim}
Warning in grid.Call.graphics(C_text, as.graphicsAnnot(x$label), x$x, x$y, :
неизвестна ширина символа 0xed в кодировке CP1251
\end{verbatim}

\begin{verbatim}
Warning in grid.Call.graphics(C_text, as.graphicsAnnot(x$label), x$x, x$y, :
неизвестна ширина символа 0xf3 в кодировке CP1251
\end{verbatim}

\begin{verbatim}
Warning in grid.Call.graphics(C_text, as.graphicsAnnot(x$label), x$x, x$y, :
неизвестна ширина символа 0xf2 в кодировке CP1251
\end{verbatim}

\begin{verbatim}
Warning in grid.Call.graphics(C_text, as.graphicsAnnot(x$label), x$x, x$y, :
неизвестна ширина символа 0xf0 в кодировке CP1251
\end{verbatim}

\begin{verbatim}
Warning in grid.Call.graphics(C_text, as.graphicsAnnot(x$label), x$x, x$y, :
неизвестна ширина символа 0xe8 в кодировке CP1251
\end{verbatim}

\begin{verbatim}
Warning in grid.Call.graphics(C_text, as.graphicsAnnot(x$label), x$x, x$y, :
неизвестна ширина символа 0xea в кодировке CP1251
\end{verbatim}

\begin{verbatim}
Warning in grid.Call.graphics(C_text, as.graphicsAnnot(x$label), x$x, x$y, :
неизвестна ширина символа 0xe0 в кодировке CP1251
\end{verbatim}

\begin{verbatim}
Warning in grid.Call.graphics(C_text, as.graphicsAnnot(x$label), x$x, x$y, :
неизвестна ширина символа 0xe6 в кодировке CP1251
\end{verbatim}

\begin{verbatim}
Warning in grid.Call.graphics(C_text, as.graphicsAnnot(x$label), x$x, x$y, :
неизвестна ширина символа 0xe4 в кодировке CP1251
\end{verbatim}

\begin{verbatim}
Warning in grid.Call.graphics(C_text, as.graphicsAnnot(x$label), x$x, x$y, :
неизвестна ширина символа 0xee в кодировке CP1251
\end{verbatim}

\begin{verbatim}
Warning in grid.Call.graphics(C_text, as.graphicsAnnot(x$label), x$x, x$y, :
неизвестна ширина символа 0xe3 в кодировке CP1251
\end{verbatim}

\begin{verbatim}
Warning in grid.Call.graphics(C_text, as.graphicsAnnot(x$label), x$x, x$y, :
неизвестна ширина символа 0xee в кодировке CP1251
\end{verbatim}

\begin{verbatim}
Warning in grid.Call.graphics(C_text, as.graphicsAnnot(x$label), x$x, x$y, :
неизвестна ширина символа 0xef в кодировке CP1251
\end{verbatim}

\begin{verbatim}
Warning in grid.Call.graphics(C_text, as.graphicsAnnot(x$label), x$x, x$y, :
неизвестна ширина символа 0xee в кодировке CP1251
\end{verbatim}

\begin{verbatim}
Warning in grid.Call.graphics(C_text, as.graphicsAnnot(x$label), x$x, x$y, :
неизвестна ширина символа 0xeb в кодировке CP1251
\end{verbatim}

\begin{verbatim}
Warning in grid.Call.graphics(C_text, as.graphicsAnnot(x$label), x$x, x$y, :
неизвестна ширина символа 0xe8 в кодировке CP1251
\end{verbatim}

\begin{verbatim}
Warning in grid.Call.graphics(C_text, as.graphicsAnnot(x$label), x$x, x$y, :
неизвестна ширина символа 0xe3 в кодировке CP1251
\end{verbatim}

\begin{verbatim}
Warning in grid.Call.graphics(C_text, as.graphicsAnnot(x$label), x$x, x$y, :
неизвестна ширина символа 0xee в кодировке CP1251
\end{verbatim}

\begin{verbatim}
Warning in grid.Call.graphics(C_text, as.graphicsAnnot(x$label), x$x, x$y, :
неизвестна ширина символа 0xed в кодировке CP1251
\end{verbatim}

\begin{verbatim}
Warning in grid.Call.graphics(C_text, as.graphicsAnnot(x$label), x$x, x$y, :
неизвестна ширина символа 0xe0 в кодировке CP1251
\end{verbatim}

\begin{verbatim}
Warning in grid.Call.graphics(C_text, as.graphicsAnnot(x$label), x$x, x$y, :
неизвестна ширина символа 0xcf в кодировке CP1251
\end{verbatim}

\begin{verbatim}
Warning in grid.Call.graphics(C_text, as.graphicsAnnot(x$label), x$x, x$y, :
неизвестна ширина символа 0xee в кодировке CP1251
\end{verbatim}

\begin{verbatim}
Warning in grid.Call.graphics(C_text, as.graphicsAnnot(x$label), x$x, x$y, :
неизвестна ширина символа 0xeb в кодировке CP1251
\end{verbatim}

\begin{verbatim}
Warning in grid.Call.graphics(C_text, as.graphicsAnnot(x$label), x$x, x$y, :
неизвестна ширина символа 0xe8 в кодировке CP1251
\end{verbatim}

\begin{verbatim}
Warning in grid.Call.graphics(C_text, as.graphicsAnnot(x$label), x$x, x$y, :
неизвестна ширина символа 0xe3 в кодировке CP1251
\end{verbatim}

\begin{verbatim}
Warning in grid.Call.graphics(C_text, as.graphicsAnnot(x$label), x$x, x$y, :
неизвестна ширина символа 0xee в кодировке CP1251
\end{verbatim}

\begin{verbatim}
Warning in grid.Call.graphics(C_text, as.graphicsAnnot(x$label), x$x, x$y, :
неизвестна ширина символа 0xed в кодировке CP1251
\end{verbatim}

\begin{verbatim}
Warning in grid.Call.graphics(C_text, as.graphicsAnnot(x$label), x$x, x$y, :
неизвестна ширина символа 0xfb в кодировке CP1251
\end{verbatim}

\begin{verbatim}
Warning in grid.Call.graphics(C_text, as.graphicsAnnot(x$label), x$x, x$y, :
неизвестна ширина символа 0xf1 в кодировке CP1251
\end{verbatim}

\begin{verbatim}
Warning in grid.Call.graphics(C_text, as.graphicsAnnot(x$label), x$x, x$y, :
неизвестна ширина символа 0xf0 в кодировке CP1251
\end{verbatim}

\begin{verbatim}
Warning in grid.Call.graphics(C_text, as.graphicsAnnot(x$label), x$x, x$y, :
неизвестна ширина символа 0xe0 в кодировке CP1251
\end{verbatim}

\begin{verbatim}
Warning in grid.Call.graphics(C_text, as.graphicsAnnot(x$label), x$x, x$y, :
неизвестна ширина символа 0xe7 в кодировке CP1251
\end{verbatim}

\begin{verbatim}
Warning in grid.Call.graphics(C_text, as.graphicsAnnot(x$label), x$x, x$y, :
неизвестна ширина символа 0xed в кодировке CP1251
\end{verbatim}

\begin{verbatim}
Warning in grid.Call.graphics(C_text, as.graphicsAnnot(x$label), x$x, x$y, :
неизвестна ширина символа 0xee в кодировке CP1251
\end{verbatim}

\begin{verbatim}
Warning in grid.Call.graphics(C_text, as.graphicsAnnot(x$label), x$x, x$y, :
неизвестна ширина символа 0xe9 в кодировке CP1251
\end{verbatim}

\begin{verbatim}
Warning in grid.Call.graphics(C_text, as.graphicsAnnot(x$label), x$x, x$y, :
неизвестна ширина символа 0xef в кодировке CP1251
\end{verbatim}

\begin{verbatim}
Warning in grid.Call.graphics(C_text, as.graphicsAnnot(x$label), x$x, x$y, :
неизвестна ширина символа 0xeb в кодировке CP1251
\end{verbatim}

\begin{verbatim}
Warning in grid.Call.graphics(C_text, as.graphicsAnnot(x$label), x$x, x$y, :
неизвестна ширина символа 0xee в кодировке CP1251
\end{verbatim}

\begin{verbatim}
Warning in grid.Call.graphics(C_text, as.graphicsAnnot(x$label), x$x, x$y, :
неизвестна ширина символа 0xf9 в кодировке CP1251
\end{verbatim}

\begin{verbatim}
Warning in grid.Call.graphics(C_text, as.graphicsAnnot(x$label), x$x, x$y, :
неизвестна ширина символа 0xe0 в кодировке CP1251
\end{verbatim}

\begin{verbatim}
Warning in grid.Call.graphics(C_text, as.graphicsAnnot(x$label), x$x, x$y, :
неизвестна ширина символа 0xe4 в кодировке CP1251
\end{verbatim}

\begin{verbatim}
Warning in grid.Call.graphics(C_text, as.graphicsAnnot(x$label), x$x, x$y, :
неизвестна ширина символа 0xfc в кодировке CP1251
\end{verbatim}

\begin{verbatim}
Warning in grid.Call.graphics(C_text, as.graphicsAnnot(x$label), x$x, x$y, :
неизвестна ширина символа 0xfe в кодировке CP1251
\end{verbatim}

\begin{verbatim}
Warning in grid.Call.graphics(C_text, as.graphicsAnnot(x$label), x$x, x$y, :
неизвестна ширина символа 0xe8 в кодировке CP1251
\end{verbatim}

\begin{verbatim}
Warning in grid.Call.graphics(C_text, as.graphicsAnnot(x$label), x$x, x$y, :
неизвестна ширина символа 0xf2 в кодировке CP1251
\end{verbatim}

\begin{verbatim}
Warning in grid.Call.graphics(C_text, as.graphicsAnnot(x$label), x$x, x$y, :
неизвестна ширина символа 0xf0 в кодировке CP1251
\end{verbatim}

\begin{verbatim}
Warning in grid.Call.graphics(C_text, as.graphicsAnnot(x$label), x$x, x$y, :
неизвестна ширина символа 0xe0 в кодировке CP1251
\end{verbatim}

\begin{verbatim}
Warning in grid.Call.graphics(C_text, as.graphicsAnnot(x$label), x$x, x$y, :
неизвестна ширина символа 0xeb в кодировке CP1251
\end{verbatim}

\begin{verbatim}
Warning in grid.Call.graphics(C_text, as.graphicsAnnot(x$label), x$x, x$y, :
неизвестна ширина символа 0xee в кодировке CP1251
\end{verbatim}

\begin{verbatim}
Warning in grid.Call.graphics(C_text, as.graphicsAnnot(x$label), x$x, x$y, :
неизвестна ширина символа 0xe2 в кодировке CP1251
\end{verbatim}

\begin{verbatim}
Warning in grid.Call.graphics(C_text, as.graphicsAnnot(x$label), x$x, x$y, :
неизвестна ширина символа 0xfb в кодировке CP1251
\end{verbatim}

\begin{verbatim}
Warning in grid.Call.graphics(C_text, as.graphicsAnnot(x$label), x$x, x$y, :
неизвестна ширина символа 0xec в кодировке CP1251
\end{verbatim}

\begin{verbatim}
Warning in grid.Call.graphics(C_text, as.graphicsAnnot(x$label), x$x, x$y, :
неизвестна ширина символа 0xe8 в кодировке CP1251
\end{verbatim}

\begin{verbatim}
Warning in grid.Call.graphics(C_text, as.graphicsAnnot(x$label), x$x, x$y, :
неизвестна ширина символа 0xf1 в кодировке CP1251
\end{verbatim}

\begin{verbatim}
Warning in grid.Call.graphics(C_text, as.graphicsAnnot(x$label), x$x, x$y, :
неизвестна ширина символа 0xf2 в кодировке CP1251
\end{verbatim}

\begin{verbatim}
Warning in grid.Call.graphics(C_text, as.graphicsAnnot(x$label), x$x, x$y, :
неизвестна ширина символа 0xe0 в кодировке CP1251
\end{verbatim}

\begin{verbatim}
Warning in grid.Call.graphics(C_text, as.graphicsAnnot(x$label), x$x, x$y, :
неизвестна ширина символа 0xed в кодировке CP1251
\end{verbatim}

\begin{verbatim}
Warning in grid.Call.graphics(C_text, as.graphicsAnnot(x$label), x$x, x$y, :
неизвестна ширина символа 0xf6 в кодировке CP1251
\end{verbatim}

\begin{verbatim}
Warning in grid.Call.graphics(C_text, as.graphicsAnnot(x$label), x$x, x$y, :
неизвестна ширина символа 0xe8 в кодировке CP1251
\end{verbatim}

\begin{verbatim}
Warning in grid.Call.graphics(C_text, as.graphicsAnnot(x$label), x$x, x$y, :
неизвестна ширина символа 0xff в кодировке CP1251
\end{verbatim}

\begin{verbatim}
Warning in grid.Call.graphics(C_text, as.graphicsAnnot(x$label), x$x, x$y, :
неизвестна ширина символа 0xec в кодировке CP1251
\end{verbatim}

\begin{verbatim}
Warning in grid.Call.graphics(C_text, as.graphicsAnnot(x$label), x$x, x$y, :
неизвестна ширина символа 0xe8 в кодировке CP1251
\end{verbatim}

\pandocbounded{\includegraphics[keepaspectratio]{chapter10_files/figure-pdf/unnamed-chunk-1-1.pdf}}

\begin{Shaded}
\begin{Highlighting}[]
\CommentTok{\# 14. ФУНКЦИЯ ДЛЯ ПОСТРОЕНИЯ ОПТИМАЛЬНОГО МАРШРУТА (TSP) {-}{-}{-}{-}{-}{-}{-}{-}{-}{-}{-}{-}{-}{-}{-}{-}{-}{-}{-}{-}{-}{-}}
\CommentTok{\# Создает маршрут, проходящий через все точки, с фиксацией начала и конца.}
\NormalTok{create\_route }\OtherTok{\textless{}{-}} \ControlFlowTok{function}\NormalTok{(points) \{}
  \CommentTok{\# Извлечение координат (долгота, широта) из SF{-}объекта}
\NormalTok{  coords }\OtherTok{\textless{}{-}} \FunctionTok{st\_coordinates}\NormalTok{(points)}

  \CommentTok{\# СТРАТЕГИЯ: Начало и конец маршрута {-} две самые западные точки.}
  \CommentTok{\# Это имитирует выход судна из порта и возврат в него.}
\NormalTok{  west\_points }\OtherTok{\textless{}{-}} \FunctionTok{order}\NormalTok{(coords[,}\DecValTok{1}\NormalTok{])[}\DecValTok{1}\SpecialCharTok{:}\DecValTok{2}\NormalTok{] }\CommentTok{\# Индексы двух точек с min долготой}

  \CommentTok{\# Расчет матрицы расстояний между всеми точками (в метрах)}
\NormalTok{  dist\_matrix }\OtherTok{\textless{}{-}} \FunctionTok{distm}\NormalTok{(coords, }\AttributeTok{fun =}\NormalTok{ distHaversine)}

  \CommentTok{\# СОЗДАНИЕ И НАСТРОЙКА ЗАДАЧИ KОММИВОЯЖЕРА (TSP):}
\NormalTok{  tsp }\OtherTok{\textless{}{-}} \FunctionTok{TSP}\NormalTok{(dist\_matrix }\SpecialCharTok{/} \DecValTok{1000}\NormalTok{) }\CommentTok{\# Создание объекта TSP (расстояния в км)}
\NormalTok{  atsp }\OtherTok{\textless{}{-}} \FunctionTok{as.ATSP}\NormalTok{(tsp)           }\CommentTok{\# Преобразование в Asymmetric TSP}

  \CommentTok{\# ФИКСАЦИЯ НАЧАЛА И КОНЦА:}
  \CommentTok{\# Обнуляем расстояния ДО стартовой точки {-}\textgreater{} она станет первой.}
\NormalTok{  atsp[, west\_points[}\DecValTok{1}\NormalTok{]] }\OtherTok{\textless{}{-}} \DecValTok{0}
  \CommentTok{\# Обнуляем расстояния ОТ конечной точки {-}\textgreater{} она станет последней.}
\NormalTok{  atsp[west\_points[}\DecValTok{2}\NormalTok{], ] }\OtherTok{\textless{}{-}} \DecValTok{0}

  \CommentTok{\# РЕШЕНИЕ TSP: метод "Ближайшая вставка" (быстрый, но не всегда оптимальный)}
\NormalTok{  tour }\OtherTok{\textless{}{-}} \FunctionTok{solve\_TSP}\NormalTok{(atsp, }\AttributeTok{method =} \StringTok{"nearest\_insertion"}\NormalTok{)}

  \CommentTok{\# ФОРМИРОВАНИЕ ПОСЛЕДОВАТЕЛЬНОСТИ ТОЧЕК МАРШРУТА:}
  \CommentTok{\# Начало {-}\textgreater{} Маршрут {-}\textgreater{} Конец. unique() убирает возможные дубликаты.}
\NormalTok{  ordered\_indices }\OtherTok{\textless{}{-}} \FunctionTok{c}\NormalTok{(west\_points[}\DecValTok{1}\NormalTok{], }\FunctionTok{as.integer}\NormalTok{(tour), west\_points[}\DecValTok{2}\NormalTok{])}
\NormalTok{  ordered\_indices }\OtherTok{\textless{}{-}} \FunctionTok{unique}\NormalTok{(ordered\_indices)}

  \FunctionTok{return}\NormalTok{(ordered\_indices)}
\NormalTok{\}}

\CommentTok{\# 15. ПОСТРОЕНИЕ МАРШРУТОВ ДЛЯ КАЖДОГО ПОЛИГОНА {-}{-}{-}{-}{-}{-}{-}{-}{-}{-}{-}{-}{-}{-}{-}{-}{-}{-}{-}{-}{-}{-}{-}{-}{-}{-}{-}{-}{-}{-}{-}}
\CommentTok{\# Применение функции create\_route к каждой группе точек}
\NormalTok{routes }\OtherTok{\textless{}{-}}\NormalTok{ points\_wgs84 }\SpecialCharTok{\%\textgreater{}\%}
  \FunctionTok{group\_by}\NormalTok{(label) }\SpecialCharTok{\%\textgreater{}\%}        \CommentTok{\# Группировка по варианту полигона}
  \FunctionTok{group\_modify}\NormalTok{(}\SpecialCharTok{\textasciitilde{}}\NormalTok{ \{}
\NormalTok{    ids }\OtherTok{\textless{}{-}} \FunctionTok{create\_route}\NormalTok{(.x)  }\CommentTok{\# Получение упорядоченного списка индексов точек}
    \CommentTok{\# Соединение точек в линию (маршрут)}
\NormalTok{    route\_line }\OtherTok{\textless{}{-}} \FunctionTok{st\_combine}\NormalTok{(.x}\SpecialCharTok{$}\NormalTok{geometry[ids]) }\SpecialCharTok{\%\textgreater{}\%}
      \FunctionTok{st\_cast}\NormalTok{(}\StringTok{"LINESTRING"}\NormalTok{)  }\CommentTok{\# Явное указание типа геометрии}

    \FunctionTok{tibble}\NormalTok{(}\AttributeTok{geometry =} \FunctionTok{st\_sfc}\NormalTok{(route\_line, }\AttributeTok{crs =} \DecValTok{4326}\NormalTok{)) }\CommentTok{\# Возврат маршрута}
\NormalTok{  \}) }\SpecialCharTok{\%\textgreater{}\%}
  \FunctionTok{st\_as\_sf}\NormalTok{() }\CommentTok{\# Преобразование результата в SF{-}объект}

\CommentTok{\# 16. ВИЗУАЛИЗАЦИЯ: ПОЛИГОНЫ, ТОЧКИ И МАРШРУТЫ {-}{-}{-}{-}{-}{-}{-}{-}{-}{-}{-}{-}{-}{-}{-}{-}{-}{-}{-}{-}{-}{-}{-}{-}{-}{-}{-}{-}{-}{-}{-}{-}}
\FunctionTok{ggplot}\NormalTok{() }\SpecialCharTok{+}
  \FunctionTok{geom\_sf}\NormalTok{(}\AttributeTok{data =}\NormalTok{ russia, }\AttributeTok{fill =} \StringTok{"lightgray"}\NormalTok{, }\AttributeTok{color =} \StringTok{"black"}\NormalTok{, }\AttributeTok{linewidth =} \FloatTok{0.3}\NormalTok{) }\SpecialCharTok{+}
  \FunctionTok{geom\_sf}\NormalTok{(}\AttributeTok{data =}\NormalTok{ polygon\_wgs84, }\FunctionTok{aes}\NormalTok{(}\AttributeTok{fill =}\NormalTok{ label), }\AttributeTok{alpha =} \FloatTok{0.6}\NormalTok{, }\AttributeTok{linewidth =} \FloatTok{0.5}\NormalTok{) }\SpecialCharTok{+}
  \FunctionTok{geom\_sf}\NormalTok{(}\AttributeTok{data =}\NormalTok{ points\_wgs84, }\FunctionTok{aes}\NormalTok{(}\AttributeTok{color =}\NormalTok{ label), }\AttributeTok{shape =} \DecValTok{16}\NormalTok{, }\AttributeTok{size =} \DecValTok{1}\NormalTok{) }\SpecialCharTok{+}
  \FunctionTok{geom\_sf}\NormalTok{(}\AttributeTok{data =}\NormalTok{ routes, }\AttributeTok{color =} \StringTok{"darkblue"}\NormalTok{, }\AttributeTok{linewidth =} \FloatTok{0.8}\NormalTok{) }\SpecialCharTok{+} \CommentTok{\# Маршруты}
  \FunctionTok{facet\_wrap}\NormalTok{(}\SpecialCharTok{\textasciitilde{}}\NormalTok{ label, }\AttributeTok{scales =} \StringTok{"fixed"}\NormalTok{) }\SpecialCharTok{+}
  \FunctionTok{coord\_sf}\NormalTok{(}\AttributeTok{xlim =} \FunctionTok{c}\NormalTok{(}\DecValTok{35}\NormalTok{, }\DecValTok{50}\NormalTok{), }\AttributeTok{ylim =} \FunctionTok{c}\NormalTok{(}\DecValTok{68}\NormalTok{, }\FloatTok{74.5}\NormalTok{)) }\SpecialCharTok{+}
  \FunctionTok{labs}\NormalTok{(}
    \AttributeTok{title =} \StringTok{"Полигоны с оптимальными маршрутами"}\NormalTok{,}
    \AttributeTok{subtitle =} \StringTok{"Начало и конец маршрута {-} две самые западные точки"}\NormalTok{,}
    \AttributeTok{fill =} \StringTok{"Вариант"}\NormalTok{, }\AttributeTok{color =} \StringTok{"Вариант"}
\NormalTok{  ) }\SpecialCharTok{+}
  \FunctionTok{theme\_minimal}\NormalTok{() }\SpecialCharTok{+}
  \FunctionTok{theme}\NormalTok{(}\AttributeTok{strip.background =} \FunctionTok{element\_rect}\NormalTok{(}\AttributeTok{fill =} \StringTok{"lightblue"}\NormalTok{))}
\end{Highlighting}
\end{Shaded}

\begin{verbatim}
Warning in grid.Call(C_textBounds, as.graphicsAnnot(x$label), x$x, x$y, :
неизвестна ширина символа 0xc2 в кодировке CP1251
\end{verbatim}

\begin{verbatim}
Warning in grid.Call(C_textBounds, as.graphicsAnnot(x$label), x$x, x$y, :
неизвестна ширина символа 0xe0 в кодировке CP1251
\end{verbatim}

\begin{verbatim}
Warning in grid.Call(C_textBounds, as.graphicsAnnot(x$label), x$x, x$y, :
неизвестна ширина символа 0xf0 в кодировке CP1251
\end{verbatim}

\begin{verbatim}
Warning in grid.Call(C_textBounds, as.graphicsAnnot(x$label), x$x, x$y, :
неизвестна ширина символа 0xe8 в кодировке CP1251
\end{verbatim}

\begin{verbatim}
Warning in grid.Call(C_textBounds, as.graphicsAnnot(x$label), x$x, x$y, :
неизвестна ширина символа 0xe0 в кодировке CP1251
\end{verbatim}

\begin{verbatim}
Warning in grid.Call(C_textBounds, as.graphicsAnnot(x$label), x$x, x$y, :
неизвестна ширина символа 0xed в кодировке CP1251
\end{verbatim}

\begin{verbatim}
Warning in grid.Call(C_textBounds, as.graphicsAnnot(x$label), x$x, x$y, :
неизвестна ширина символа 0xf2 в кодировке CP1251
\end{verbatim}

\begin{verbatim}
Warning in grid.Call(C_textBounds, as.graphicsAnnot(x$label), x$x, x$y, :
неизвестна ширина символа 0xc2 в кодировке CP1251
\end{verbatim}

\begin{verbatim}
Warning in grid.Call(C_textBounds, as.graphicsAnnot(x$label), x$x, x$y, :
неизвестна ширина символа 0xe0 в кодировке CP1251
\end{verbatim}

\begin{verbatim}
Warning in grid.Call(C_textBounds, as.graphicsAnnot(x$label), x$x, x$y, :
неизвестна ширина символа 0xf0 в кодировке CP1251
\end{verbatim}

\begin{verbatim}
Warning in grid.Call(C_textBounds, as.graphicsAnnot(x$label), x$x, x$y, :
неизвестна ширина символа 0xe8 в кодировке CP1251
\end{verbatim}

\begin{verbatim}
Warning in grid.Call(C_textBounds, as.graphicsAnnot(x$label), x$x, x$y, :
неизвестна ширина символа 0xe0 в кодировке CP1251
\end{verbatim}

\begin{verbatim}
Warning in grid.Call(C_textBounds, as.graphicsAnnot(x$label), x$x, x$y, :
неизвестна ширина символа 0xed в кодировке CP1251
\end{verbatim}

\begin{verbatim}
Warning in grid.Call(C_textBounds, as.graphicsAnnot(x$label), x$x, x$y, :
неизвестна ширина символа 0xf2 в кодировке CP1251
\end{verbatim}

\begin{verbatim}
Warning in grid.Call(C_textBounds, as.graphicsAnnot(x$label), x$x, x$y, :
неизвестна ширина символа 0xcf в кодировке CP1251
\end{verbatim}

\begin{verbatim}
Warning in grid.Call(C_textBounds, as.graphicsAnnot(x$label), x$x, x$y, :
неизвестна ширина символа 0xee в кодировке CP1251
\end{verbatim}

\begin{verbatim}
Warning in grid.Call(C_textBounds, as.graphicsAnnot(x$label), x$x, x$y, :
неизвестна ширина символа 0xeb в кодировке CP1251
\end{verbatim}

\begin{verbatim}
Warning in grid.Call(C_textBounds, as.graphicsAnnot(x$label), x$x, x$y, :
неизвестна ширина символа 0xe8 в кодировке CP1251
\end{verbatim}

\begin{verbatim}
Warning in grid.Call(C_textBounds, as.graphicsAnnot(x$label), x$x, x$y, :
неизвестна ширина символа 0xe3 в кодировке CP1251
\end{verbatim}

\begin{verbatim}
Warning in grid.Call(C_textBounds, as.graphicsAnnot(x$label), x$x, x$y, :
неизвестна ширина символа 0xee в кодировке CP1251
\end{verbatim}

\begin{verbatim}
Warning in grid.Call(C_textBounds, as.graphicsAnnot(x$label), x$x, x$y, :
неизвестна ширина символа 0xed в кодировке CP1251
\end{verbatim}

\begin{verbatim}
Warning in grid.Call(C_textBounds, as.graphicsAnnot(x$label), x$x, x$y, :
неизвестна ширина символа 0xfb в кодировке CP1251
\end{verbatim}

\begin{verbatim}
Warning in grid.Call(C_textBounds, as.graphicsAnnot(x$label), x$x, x$y, :
неизвестна ширина символа 0xf1 в кодировке CP1251
\end{verbatim}

\begin{verbatim}
Warning in grid.Call(C_textBounds, as.graphicsAnnot(x$label), x$x, x$y, :
неизвестна ширина символа 0xee в кодировке CP1251
\end{verbatim}

\begin{verbatim}
Warning in grid.Call(C_textBounds, as.graphicsAnnot(x$label), x$x, x$y, :
неизвестна ширина символа 0xef в кодировке CP1251
\end{verbatim}

\begin{verbatim}
Warning in grid.Call(C_textBounds, as.graphicsAnnot(x$label), x$x, x$y, :
неизвестна ширина символа 0xf2 в кодировке CP1251
\end{verbatim}

\begin{verbatim}
Warning in grid.Call(C_textBounds, as.graphicsAnnot(x$label), x$x, x$y, :
неизвестна ширина символа 0xe8 в кодировке CP1251
\end{verbatim}

\begin{verbatim}
Warning in grid.Call(C_textBounds, as.graphicsAnnot(x$label), x$x, x$y, :
неизвестна ширина символа 0xec в кодировке CP1251
\end{verbatim}

\begin{verbatim}
Warning in grid.Call(C_textBounds, as.graphicsAnnot(x$label), x$x, x$y, :
неизвестна ширина символа 0xe0 в кодировке CP1251
\end{verbatim}

\begin{verbatim}
Warning in grid.Call(C_textBounds, as.graphicsAnnot(x$label), x$x, x$y, :
неизвестна ширина символа 0xeb в кодировке CP1251
\end{verbatim}

\begin{verbatim}
Warning in grid.Call(C_textBounds, as.graphicsAnnot(x$label), x$x, x$y, :
неизвестна ширина символа 0xfc в кодировке CP1251
\end{verbatim}

\begin{verbatim}
Warning in grid.Call(C_textBounds, as.graphicsAnnot(x$label), x$x, x$y, :
неизвестна ширина символа 0xed в кодировке CP1251
\end{verbatim}

\begin{verbatim}
Warning in grid.Call(C_textBounds, as.graphicsAnnot(x$label), x$x, x$y, :
неизвестна ширина символа 0xfb в кодировке CP1251
\end{verbatim}

\begin{verbatim}
Warning in grid.Call(C_textBounds, as.graphicsAnnot(x$label), x$x, x$y, :
неизвестна ширина символа 0xec в кодировке CP1251
\end{verbatim}

\begin{verbatim}
Warning in grid.Call(C_textBounds, as.graphicsAnnot(x$label), x$x, x$y, :
неизвестна ширина символа 0xe8 в кодировке CP1251
\end{verbatim}

\begin{verbatim}
Warning in grid.Call(C_textBounds, as.graphicsAnnot(x$label), x$x, x$y, :
неизвестна ширина символа 0xec в кодировке CP1251
\end{verbatim}

\begin{verbatim}
Warning in grid.Call(C_textBounds, as.graphicsAnnot(x$label), x$x, x$y, :
неизвестна ширина символа 0xe0 в кодировке CP1251
\end{verbatim}

\begin{verbatim}
Warning in grid.Call(C_textBounds, as.graphicsAnnot(x$label), x$x, x$y, :
неизвестна ширина символа 0xf0 в кодировке CP1251
\end{verbatim}

\begin{verbatim}
Warning in grid.Call(C_textBounds, as.graphicsAnnot(x$label), x$x, x$y, :
неизвестна ширина символа 0xf8 в кодировке CP1251
\end{verbatim}

\begin{verbatim}
Warning in grid.Call(C_textBounds, as.graphicsAnnot(x$label), x$x, x$y, :
неизвестна ширина символа 0xf0 в кодировке CP1251
\end{verbatim}

\begin{verbatim}
Warning in grid.Call(C_textBounds, as.graphicsAnnot(x$label), x$x, x$y, :
неизвестна ширина символа 0xf3 в кодировке CP1251
\end{verbatim}

\begin{verbatim}
Warning in grid.Call(C_textBounds, as.graphicsAnnot(x$label), x$x, x$y, :
неизвестна ширина символа 0xf2 в кодировке CP1251
\end{verbatim}

\begin{verbatim}
Warning in grid.Call(C_textBounds, as.graphicsAnnot(x$label), x$x, x$y, :
неизвестна ширина символа 0xe0 в кодировке CP1251
\end{verbatim}

\begin{verbatim}
Warning in grid.Call(C_textBounds, as.graphicsAnnot(x$label), x$x, x$y, :
неизвестна ширина символа 0xec в кодировке CP1251
\end{verbatim}

\begin{verbatim}
Warning in grid.Call(C_textBounds, as.graphicsAnnot(x$label), x$x, x$y, :
неизвестна ширина символа 0xe8 в кодировке CP1251
\end{verbatim}

\begin{verbatim}
Warning in grid.Call(C_textBounds, as.graphicsAnnot(x$label), x$x, x$y, :
неизвестна ширина символа 0xcd в кодировке CP1251
\end{verbatim}

\begin{verbatim}
Warning in grid.Call(C_textBounds, as.graphicsAnnot(x$label), x$x, x$y, :
неизвестна ширина символа 0xe0 в кодировке CP1251
\end{verbatim}

\begin{verbatim}
Warning in grid.Call(C_textBounds, as.graphicsAnnot(x$label), x$x, x$y, :
неизвестна ширина символа 0xf7 в кодировке CP1251
\end{verbatim}

\begin{verbatim}
Warning in grid.Call(C_textBounds, as.graphicsAnnot(x$label), x$x, x$y, :
неизвестна ширина символа 0xe0 в кодировке CP1251
\end{verbatim}

\begin{verbatim}
Warning in grid.Call(C_textBounds, as.graphicsAnnot(x$label), x$x, x$y, :
неизвестна ширина символа 0xeb в кодировке CP1251
\end{verbatim}

\begin{verbatim}
Warning in grid.Call(C_textBounds, as.graphicsAnnot(x$label), x$x, x$y, :
неизвестна ширина символа 0xee в кодировке CP1251
\end{verbatim}

\begin{verbatim}
Warning in grid.Call(C_textBounds, as.graphicsAnnot(x$label), x$x, x$y, :
неизвестна ширина символа 0xe8 в кодировке CP1251
\end{verbatim}

\begin{verbatim}
Warning in grid.Call(C_textBounds, as.graphicsAnnot(x$label), x$x, x$y, :
неизвестна ширина символа 0xea в кодировке CP1251
\end{verbatim}

\begin{verbatim}
Warning in grid.Call(C_textBounds, as.graphicsAnnot(x$label), x$x, x$y, :
неизвестна ширина символа 0xee в кодировке CP1251
\end{verbatim}

\begin{verbatim}
Warning in grid.Call(C_textBounds, as.graphicsAnnot(x$label), x$x, x$y, :
неизвестна ширина символа 0xed в кодировке CP1251
\end{verbatim}

\begin{verbatim}
Warning in grid.Call(C_textBounds, as.graphicsAnnot(x$label), x$x, x$y, :
неизвестна ширина символа 0xe5 в кодировке CP1251
\end{verbatim}

\begin{verbatim}
Warning in grid.Call(C_textBounds, as.graphicsAnnot(x$label), x$x, x$y, :
неизвестна ширина символа 0xf6 в кодировке CP1251
\end{verbatim}

\begin{verbatim}
Warning in grid.Call(C_textBounds, as.graphicsAnnot(x$label), x$x, x$y, :
неизвестна ширина символа 0xec в кодировке CP1251
\end{verbatim}

\begin{verbatim}
Warning in grid.Call(C_textBounds, as.graphicsAnnot(x$label), x$x, x$y, :
неизвестна ширина символа 0xe0 в кодировке CP1251
\end{verbatim}

\begin{verbatim}
Warning in grid.Call(C_textBounds, as.graphicsAnnot(x$label), x$x, x$y, :
неизвестна ширина символа 0xf0 в кодировке CP1251
\end{verbatim}

\begin{verbatim}
Warning in grid.Call(C_textBounds, as.graphicsAnnot(x$label), x$x, x$y, :
неизвестна ширина символа 0xf8 в кодировке CP1251
\end{verbatim}

\begin{verbatim}
Warning in grid.Call(C_textBounds, as.graphicsAnnot(x$label), x$x, x$y, :
неизвестна ширина символа 0xf0 в кодировке CP1251
\end{verbatim}

\begin{verbatim}
Warning in grid.Call(C_textBounds, as.graphicsAnnot(x$label), x$x, x$y, :
неизвестна ширина символа 0xf3 в кодировке CP1251
\end{verbatim}

\begin{verbatim}
Warning in grid.Call(C_textBounds, as.graphicsAnnot(x$label), x$x, x$y, :
неизвестна ширина символа 0xf2 в кодировке CP1251
\end{verbatim}

\begin{verbatim}
Warning in grid.Call(C_textBounds, as.graphicsAnnot(x$label), x$x, x$y, :
неизвестна ширина символа 0xe0 в кодировке CP1251
\end{verbatim}

\begin{verbatim}
Warning in grid.Call(C_textBounds, as.graphicsAnnot(x$label), x$x, x$y, :
неизвестна ширина символа 0xe4 в кодировке CP1251
\end{verbatim}

\begin{verbatim}
Warning in grid.Call(C_textBounds, as.graphicsAnnot(x$label), x$x, x$y, :
неизвестна ширина символа 0xe2 в кодировке CP1251
\end{verbatim}

\begin{verbatim}
Warning in grid.Call(C_textBounds, as.graphicsAnnot(x$label), x$x, x$y, :
неизвестна ширина символа 0xe5 в кодировке CP1251
\end{verbatim}

\begin{verbatim}
Warning in grid.Call(C_textBounds, as.graphicsAnnot(x$label), x$x, x$y, :
неизвестна ширина символа 0xf1 в кодировке CP1251
\end{verbatim}

\begin{verbatim}
Warning in grid.Call(C_textBounds, as.graphicsAnnot(x$label), x$x, x$y, :
неизвестна ширина символа 0xe0 в кодировке CP1251
\end{verbatim}

\begin{verbatim}
Warning in grid.Call(C_textBounds, as.graphicsAnnot(x$label), x$x, x$y, :
неизвестна ширина символа 0xec в кодировке CP1251
\end{verbatim}

\begin{verbatim}
Warning in grid.Call(C_textBounds, as.graphicsAnnot(x$label), x$x, x$y, :
неизвестна ширина символа 0xfb в кодировке CP1251
\end{verbatim}

\begin{verbatim}
Warning in grid.Call(C_textBounds, as.graphicsAnnot(x$label), x$x, x$y, :
неизвестна ширина символа 0xe5 в кодировке CP1251
\end{verbatim}

\begin{verbatim}
Warning in grid.Call(C_textBounds, as.graphicsAnnot(x$label), x$x, x$y, :
неизвестна ширина символа 0xe7 в кодировке CP1251
\end{verbatim}

\begin{verbatim}
Warning in grid.Call(C_textBounds, as.graphicsAnnot(x$label), x$x, x$y, :
неизвестна ширина символа 0xe0 в кодировке CP1251
\end{verbatim}

\begin{verbatim}
Warning in grid.Call(C_textBounds, as.graphicsAnnot(x$label), x$x, x$y, :
неизвестна ширина символа 0xef в кодировке CP1251
\end{verbatim}

\begin{verbatim}
Warning in grid.Call(C_textBounds, as.graphicsAnnot(x$label), x$x, x$y, :
неизвестна ширина символа 0xe0 в кодировке CP1251
\end{verbatim}

\begin{verbatim}
Warning in grid.Call(C_textBounds, as.graphicsAnnot(x$label), x$x, x$y, :
неизвестна ширина символа 0xe4 в кодировке CP1251
\end{verbatim}

\begin{verbatim}
Warning in grid.Call(C_textBounds, as.graphicsAnnot(x$label), x$x, x$y, :
неизвестна ширина символа 0xed в кодировке CP1251
\end{verbatim}

\begin{verbatim}
Warning in grid.Call(C_textBounds, as.graphicsAnnot(x$label), x$x, x$y, :
неизвестна ширина символа 0xfb в кодировке CP1251
\end{verbatim}

\begin{verbatim}
Warning in grid.Call(C_textBounds, as.graphicsAnnot(x$label), x$x, x$y, :
неизвестна ширина символа 0xe5 в кодировке CP1251
\end{verbatim}

\begin{verbatim}
Warning in grid.Call(C_textBounds, as.graphicsAnnot(x$label), x$x, x$y, :
неизвестна ширина символа 0xf2 в кодировке CP1251
\end{verbatim}

\begin{verbatim}
Warning in grid.Call(C_textBounds, as.graphicsAnnot(x$label), x$x, x$y, :
неизвестна ширина символа 0xee в кодировке CP1251
\end{verbatim}

\begin{verbatim}
Warning in grid.Call(C_textBounds, as.graphicsAnnot(x$label), x$x, x$y, :
неизвестна ширина символа 0xf7 в кодировке CP1251
\end{verbatim}

\begin{verbatim}
Warning in grid.Call(C_textBounds, as.graphicsAnnot(x$label), x$x, x$y, :
неизвестна ширина символа 0xea в кодировке CP1251
\end{verbatim}

\begin{verbatim}
Warning in grid.Call(C_textBounds, as.graphicsAnnot(x$label), x$x, x$y, :
неизвестна ширина символа 0xe8 в кодировке CP1251
\end{verbatim}

\begin{verbatim}
Warning in grid.Call.graphics(C_text, as.graphicsAnnot(x$label), x$x, x$y, :
неизвестна ширина символа 0xc2 в кодировке CP1251
\end{verbatim}

\begin{verbatim}
Warning in grid.Call.graphics(C_text, as.graphicsAnnot(x$label), x$x, x$y, :
неизвестна ширина символа 0xe0 в кодировке CP1251
\end{verbatim}

\begin{verbatim}
Warning in grid.Call.graphics(C_text, as.graphicsAnnot(x$label), x$x, x$y, :
неизвестна ширина символа 0xf0 в кодировке CP1251
\end{verbatim}

\begin{verbatim}
Warning in grid.Call.graphics(C_text, as.graphicsAnnot(x$label), x$x, x$y, :
неизвестна ширина символа 0xe8 в кодировке CP1251
\end{verbatim}

\begin{verbatim}
Warning in grid.Call.graphics(C_text, as.graphicsAnnot(x$label), x$x, x$y, :
неизвестна ширина символа 0xe0 в кодировке CP1251
\end{verbatim}

\begin{verbatim}
Warning in grid.Call.graphics(C_text, as.graphicsAnnot(x$label), x$x, x$y, :
неизвестна ширина символа 0xed в кодировке CP1251
\end{verbatim}

\begin{verbatim}
Warning in grid.Call.graphics(C_text, as.graphicsAnnot(x$label), x$x, x$y, :
неизвестна ширина символа 0xf2 в кодировке CP1251
\end{verbatim}

\begin{verbatim}
Warning in grid.Call.graphics(C_text, as.graphicsAnnot(x$label), x$x, x$y, :
неизвестна ширина символа 0xcd в кодировке CP1251
\end{verbatim}

\begin{verbatim}
Warning in grid.Call.graphics(C_text, as.graphicsAnnot(x$label), x$x, x$y, :
неизвестна ширина символа 0xe0 в кодировке CP1251
\end{verbatim}

\begin{verbatim}
Warning in grid.Call.graphics(C_text, as.graphicsAnnot(x$label), x$x, x$y, :
неизвестна ширина символа 0xf7 в кодировке CP1251
\end{verbatim}

\begin{verbatim}
Warning in grid.Call.graphics(C_text, as.graphicsAnnot(x$label), x$x, x$y, :
неизвестна ширина символа 0xe0 в кодировке CP1251
\end{verbatim}

\begin{verbatim}
Warning in grid.Call.graphics(C_text, as.graphicsAnnot(x$label), x$x, x$y, :
неизвестна ширина символа 0xeb в кодировке CP1251
\end{verbatim}

\begin{verbatim}
Warning in grid.Call.graphics(C_text, as.graphicsAnnot(x$label), x$x, x$y, :
неизвестна ширина символа 0xee в кодировке CP1251
\end{verbatim}

\begin{verbatim}
Warning in grid.Call.graphics(C_text, as.graphicsAnnot(x$label), x$x, x$y, :
неизвестна ширина символа 0xe8 в кодировке CP1251
\end{verbatim}

\begin{verbatim}
Warning in grid.Call.graphics(C_text, as.graphicsAnnot(x$label), x$x, x$y, :
неизвестна ширина символа 0xea в кодировке CP1251
\end{verbatim}

\begin{verbatim}
Warning in grid.Call.graphics(C_text, as.graphicsAnnot(x$label), x$x, x$y, :
неизвестна ширина символа 0xee в кодировке CP1251
\end{verbatim}

\begin{verbatim}
Warning in grid.Call.graphics(C_text, as.graphicsAnnot(x$label), x$x, x$y, :
неизвестна ширина символа 0xed в кодировке CP1251
\end{verbatim}

\begin{verbatim}
Warning in grid.Call.graphics(C_text, as.graphicsAnnot(x$label), x$x, x$y, :
неизвестна ширина символа 0xe5 в кодировке CP1251
\end{verbatim}

\begin{verbatim}
Warning in grid.Call.graphics(C_text, as.graphicsAnnot(x$label), x$x, x$y, :
неизвестна ширина символа 0xf6 в кодировке CP1251
\end{verbatim}

\begin{verbatim}
Warning in grid.Call.graphics(C_text, as.graphicsAnnot(x$label), x$x, x$y, :
неизвестна ширина символа 0xec в кодировке CP1251
\end{verbatim}

\begin{verbatim}
Warning in grid.Call.graphics(C_text, as.graphicsAnnot(x$label), x$x, x$y, :
неизвестна ширина символа 0xe0 в кодировке CP1251
\end{verbatim}

\begin{verbatim}
Warning in grid.Call.graphics(C_text, as.graphicsAnnot(x$label), x$x, x$y, :
неизвестна ширина символа 0xf0 в кодировке CP1251
\end{verbatim}

\begin{verbatim}
Warning in grid.Call.graphics(C_text, as.graphicsAnnot(x$label), x$x, x$y, :
неизвестна ширина символа 0xf8 в кодировке CP1251
\end{verbatim}

\begin{verbatim}
Warning in grid.Call.graphics(C_text, as.graphicsAnnot(x$label), x$x, x$y, :
неизвестна ширина символа 0xf0 в кодировке CP1251
\end{verbatim}

\begin{verbatim}
Warning in grid.Call.graphics(C_text, as.graphicsAnnot(x$label), x$x, x$y, :
неизвестна ширина символа 0xf3 в кодировке CP1251
\end{verbatim}

\begin{verbatim}
Warning in grid.Call.graphics(C_text, as.graphicsAnnot(x$label), x$x, x$y, :
неизвестна ширина символа 0xf2 в кодировке CP1251
\end{verbatim}

\begin{verbatim}
Warning in grid.Call.graphics(C_text, as.graphicsAnnot(x$label), x$x, x$y, :
неизвестна ширина символа 0xe0 в кодировке CP1251
\end{verbatim}

\begin{verbatim}
Warning in grid.Call.graphics(C_text, as.graphicsAnnot(x$label), x$x, x$y, :
неизвестна ширина символа 0xe4 в кодировке CP1251
\end{verbatim}

\begin{verbatim}
Warning in grid.Call.graphics(C_text, as.graphicsAnnot(x$label), x$x, x$y, :
неизвестна ширина символа 0xe2 в кодировке CP1251
\end{verbatim}

\begin{verbatim}
Warning in grid.Call.graphics(C_text, as.graphicsAnnot(x$label), x$x, x$y, :
неизвестна ширина символа 0xe5 в кодировке CP1251
\end{verbatim}

\begin{verbatim}
Warning in grid.Call.graphics(C_text, as.graphicsAnnot(x$label), x$x, x$y, :
неизвестна ширина символа 0xf1 в кодировке CP1251
\end{verbatim}

\begin{verbatim}
Warning in grid.Call.graphics(C_text, as.graphicsAnnot(x$label), x$x, x$y, :
неизвестна ширина символа 0xe0 в кодировке CP1251
\end{verbatim}

\begin{verbatim}
Warning in grid.Call.graphics(C_text, as.graphicsAnnot(x$label), x$x, x$y, :
неизвестна ширина символа 0xec в кодировке CP1251
\end{verbatim}

\begin{verbatim}
Warning in grid.Call.graphics(C_text, as.graphicsAnnot(x$label), x$x, x$y, :
неизвестна ширина символа 0xfb в кодировке CP1251
\end{verbatim}

\begin{verbatim}
Warning in grid.Call.graphics(C_text, as.graphicsAnnot(x$label), x$x, x$y, :
неизвестна ширина символа 0xe5 в кодировке CP1251
\end{verbatim}

\begin{verbatim}
Warning in grid.Call.graphics(C_text, as.graphicsAnnot(x$label), x$x, x$y, :
неизвестна ширина символа 0xe7 в кодировке CP1251
\end{verbatim}

\begin{verbatim}
Warning in grid.Call.graphics(C_text, as.graphicsAnnot(x$label), x$x, x$y, :
неизвестна ширина символа 0xe0 в кодировке CP1251
\end{verbatim}

\begin{verbatim}
Warning in grid.Call.graphics(C_text, as.graphicsAnnot(x$label), x$x, x$y, :
неизвестна ширина символа 0xef в кодировке CP1251
\end{verbatim}

\begin{verbatim}
Warning in grid.Call.graphics(C_text, as.graphicsAnnot(x$label), x$x, x$y, :
неизвестна ширина символа 0xe0 в кодировке CP1251
\end{verbatim}

\begin{verbatim}
Warning in grid.Call.graphics(C_text, as.graphicsAnnot(x$label), x$x, x$y, :
неизвестна ширина символа 0xe4 в кодировке CP1251
\end{verbatim}

\begin{verbatim}
Warning in grid.Call.graphics(C_text, as.graphicsAnnot(x$label), x$x, x$y, :
неизвестна ширина символа 0xed в кодировке CP1251
\end{verbatim}

\begin{verbatim}
Warning in grid.Call.graphics(C_text, as.graphicsAnnot(x$label), x$x, x$y, :
неизвестна ширина символа 0xfb в кодировке CP1251
\end{verbatim}

\begin{verbatim}
Warning in grid.Call.graphics(C_text, as.graphicsAnnot(x$label), x$x, x$y, :
неизвестна ширина символа 0xe5 в кодировке CP1251
\end{verbatim}

\begin{verbatim}
Warning in grid.Call.graphics(C_text, as.graphicsAnnot(x$label), x$x, x$y, :
неизвестна ширина символа 0xf2 в кодировке CP1251
\end{verbatim}

\begin{verbatim}
Warning in grid.Call.graphics(C_text, as.graphicsAnnot(x$label), x$x, x$y, :
неизвестна ширина символа 0xee в кодировке CP1251
\end{verbatim}

\begin{verbatim}
Warning in grid.Call.graphics(C_text, as.graphicsAnnot(x$label), x$x, x$y, :
неизвестна ширина символа 0xf7 в кодировке CP1251
\end{verbatim}

\begin{verbatim}
Warning in grid.Call.graphics(C_text, as.graphicsAnnot(x$label), x$x, x$y, :
неизвестна ширина символа 0xea в кодировке CP1251
\end{verbatim}

\begin{verbatim}
Warning in grid.Call.graphics(C_text, as.graphicsAnnot(x$label), x$x, x$y, :
неизвестна ширина символа 0xe8 в кодировке CP1251
\end{verbatim}

\begin{verbatim}
Warning in grid.Call.graphics(C_text, as.graphicsAnnot(x$label), x$x, x$y, :
неизвестна ширина символа 0xcf в кодировке CP1251
\end{verbatim}

\begin{verbatim}
Warning in grid.Call.graphics(C_text, as.graphicsAnnot(x$label), x$x, x$y, :
неизвестна ширина символа 0xee в кодировке CP1251
\end{verbatim}

\begin{verbatim}
Warning in grid.Call.graphics(C_text, as.graphicsAnnot(x$label), x$x, x$y, :
неизвестна ширина символа 0xeb в кодировке CP1251
\end{verbatim}

\begin{verbatim}
Warning in grid.Call.graphics(C_text, as.graphicsAnnot(x$label), x$x, x$y, :
неизвестна ширина символа 0xe8 в кодировке CP1251
\end{verbatim}

\begin{verbatim}
Warning in grid.Call.graphics(C_text, as.graphicsAnnot(x$label), x$x, x$y, :
неизвестна ширина символа 0xe3 в кодировке CP1251
\end{verbatim}

\begin{verbatim}
Warning in grid.Call.graphics(C_text, as.graphicsAnnot(x$label), x$x, x$y, :
неизвестна ширина символа 0xee в кодировке CP1251
\end{verbatim}

\begin{verbatim}
Warning in grid.Call.graphics(C_text, as.graphicsAnnot(x$label), x$x, x$y, :
неизвестна ширина символа 0xed в кодировке CP1251
\end{verbatim}

\begin{verbatim}
Warning in grid.Call.graphics(C_text, as.graphicsAnnot(x$label), x$x, x$y, :
неизвестна ширина символа 0xfb в кодировке CP1251
\end{verbatim}

\begin{verbatim}
Warning in grid.Call.graphics(C_text, as.graphicsAnnot(x$label), x$x, x$y, :
неизвестна ширина символа 0xf1 в кодировке CP1251
\end{verbatim}

\begin{verbatim}
Warning in grid.Call.graphics(C_text, as.graphicsAnnot(x$label), x$x, x$y, :
неизвестна ширина символа 0xee в кодировке CP1251
\end{verbatim}

\begin{verbatim}
Warning in grid.Call.graphics(C_text, as.graphicsAnnot(x$label), x$x, x$y, :
неизвестна ширина символа 0xef в кодировке CP1251
\end{verbatim}

\begin{verbatim}
Warning in grid.Call.graphics(C_text, as.graphicsAnnot(x$label), x$x, x$y, :
неизвестна ширина символа 0xf2 в кодировке CP1251
\end{verbatim}

\begin{verbatim}
Warning in grid.Call.graphics(C_text, as.graphicsAnnot(x$label), x$x, x$y, :
неизвестна ширина символа 0xe8 в кодировке CP1251
\end{verbatim}

\begin{verbatim}
Warning in grid.Call.graphics(C_text, as.graphicsAnnot(x$label), x$x, x$y, :
неизвестна ширина символа 0xec в кодировке CP1251
\end{verbatim}

\begin{verbatim}
Warning in grid.Call.graphics(C_text, as.graphicsAnnot(x$label), x$x, x$y, :
неизвестна ширина символа 0xe0 в кодировке CP1251
\end{verbatim}

\begin{verbatim}
Warning in grid.Call.graphics(C_text, as.graphicsAnnot(x$label), x$x, x$y, :
неизвестна ширина символа 0xeb в кодировке CP1251
\end{verbatim}

\begin{verbatim}
Warning in grid.Call.graphics(C_text, as.graphicsAnnot(x$label), x$x, x$y, :
неизвестна ширина символа 0xfc в кодировке CP1251
\end{verbatim}

\begin{verbatim}
Warning in grid.Call.graphics(C_text, as.graphicsAnnot(x$label), x$x, x$y, :
неизвестна ширина символа 0xed в кодировке CP1251
\end{verbatim}

\begin{verbatim}
Warning in grid.Call.graphics(C_text, as.graphicsAnnot(x$label), x$x, x$y, :
неизвестна ширина символа 0xfb в кодировке CP1251
\end{verbatim}

\begin{verbatim}
Warning in grid.Call.graphics(C_text, as.graphicsAnnot(x$label), x$x, x$y, :
неизвестна ширина символа 0xec в кодировке CP1251
\end{verbatim}

\begin{verbatim}
Warning in grid.Call.graphics(C_text, as.graphicsAnnot(x$label), x$x, x$y, :
неизвестна ширина символа 0xe8 в кодировке CP1251
\end{verbatim}

\begin{verbatim}
Warning in grid.Call.graphics(C_text, as.graphicsAnnot(x$label), x$x, x$y, :
неизвестна ширина символа 0xec в кодировке CP1251
\end{verbatim}

\begin{verbatim}
Warning in grid.Call.graphics(C_text, as.graphicsAnnot(x$label), x$x, x$y, :
неизвестна ширина символа 0xe0 в кодировке CP1251
\end{verbatim}

\begin{verbatim}
Warning in grid.Call.graphics(C_text, as.graphicsAnnot(x$label), x$x, x$y, :
неизвестна ширина символа 0xf0 в кодировке CP1251
\end{verbatim}

\begin{verbatim}
Warning in grid.Call.graphics(C_text, as.graphicsAnnot(x$label), x$x, x$y, :
неизвестна ширина символа 0xf8 в кодировке CP1251
\end{verbatim}

\begin{verbatim}
Warning in grid.Call.graphics(C_text, as.graphicsAnnot(x$label), x$x, x$y, :
неизвестна ширина символа 0xf0 в кодировке CP1251
\end{verbatim}

\begin{verbatim}
Warning in grid.Call.graphics(C_text, as.graphicsAnnot(x$label), x$x, x$y, :
неизвестна ширина символа 0xf3 в кодировке CP1251
\end{verbatim}

\begin{verbatim}
Warning in grid.Call.graphics(C_text, as.graphicsAnnot(x$label), x$x, x$y, :
неизвестна ширина символа 0xf2 в кодировке CP1251
\end{verbatim}

\begin{verbatim}
Warning in grid.Call.graphics(C_text, as.graphicsAnnot(x$label), x$x, x$y, :
неизвестна ширина символа 0xe0 в кодировке CP1251
\end{verbatim}

\begin{verbatim}
Warning in grid.Call.graphics(C_text, as.graphicsAnnot(x$label), x$x, x$y, :
неизвестна ширина символа 0xec в кодировке CP1251
\end{verbatim}

\begin{verbatim}
Warning in grid.Call.graphics(C_text, as.graphicsAnnot(x$label), x$x, x$y, :
неизвестна ширина символа 0xe8 в кодировке CP1251
\end{verbatim}

\pandocbounded{\includegraphics[keepaspectratio]{chapter10_files/figure-pdf/unnamed-chunk-1-2.pdf}}

\begin{Shaded}
\begin{Highlighting}[]
\CommentTok{\# 17. СОХРАНЕНИЕ КАРТЫ В ФАЙЛ {-}{-}{-}{-}{-}{-}{-}{-}{-}{-}{-}{-}{-}{-}{-}{-}{-}{-}{-}{-}{-}{-}{-}{-}{-}{-}{-}{-}{-}{-}{-}{-}{-}{-}{-}{-}{-}{-}{-}{-}{-}{-}{-}{-}{-}{-}{-}{-}{-}}
\FunctionTok{ggsave}\NormalTok{(}\StringTok{"polygon\_with\_optimal\_routes.png"}\NormalTok{, }\AttributeTok{width =} \DecValTok{12}\NormalTok{, }\AttributeTok{height =} \DecValTok{10}\NormalTok{, }\AttributeTok{dpi =} \DecValTok{300}\NormalTok{)}

\CommentTok{\# 18. СОХРАНЕНИЕ ДАННЫХ МАРШРУТОВ (Ошибка: файл уже существует) {-}{-}{-}{-}{-}{-}{-}{-}{-}{-}{-}{-}{-}{-}{-}}
\CommentTok{\# st\_write(routes, "optimal\_routes.gpkg") \# Раскомментируйте и используйте append=FALSE для перезаписи}
\CommentTok{\# st\_write(routes, "optimal\_routes.gpkg", append=FALSE)}

\CommentTok{\# 19. ФИНАЛЬНЫЙ СРАВНИТЕЛЬНЫЙ АНАЛИЗ {-}{-}{-}{-}{-}{-}{-}{-}{-}{-}{-}{-}{-}{-}{-}{-}{-}{-}{-}{-}{-}{-}{-}{-}{-}{-}{-}{-}{-}{-}{-}{-}{-}{-}{-}{-}{-}{-}{-}{-}{-}{-}}
\CommentTok{\# Создание сводной таблицы с ключевыми метриками для каждого сценария}
\NormalTok{survey\_summary }\OtherTok{\textless{}{-}}\NormalTok{ routes }\SpecialCharTok{\%\textgreater{}\%}
  \CommentTok{\# Добавление данных о площади из таблицы полигонов}
  \FunctionTok{left\_join}\NormalTok{(}\FunctionTok{st\_drop\_geometry}\NormalTok{(polygon\_wgs84), }\AttributeTok{by =} \StringTok{"label"}\NormalTok{) }\SpecialCharTok{\%\textgreater{}\%}
  \FunctionTok{mutate}\NormalTok{(}
    \StringTok{\textasciigrave{}}\AttributeTok{Количество тралений}\StringTok{\textasciigrave{}} \OtherTok{=} \DecValTok{137}\NormalTok{, }\CommentTok{\# Константа по условию задачи}
    \CommentTok{\# Расчет длины маршрута: st\_length(geometry) возвращает длину в метрах, делим на 1000 для перевода в км.}
    \StringTok{\textasciigrave{}}\AttributeTok{Длина маршрута (км)}\StringTok{\textasciigrave{}} \OtherTok{=} \FunctionTok{round}\NormalTok{(}\FunctionTok{as.numeric}\NormalTok{(}\FunctionTok{st\_length}\NormalTok{(geometry)) }\SpecialCharTok{/} \DecValTok{1000}\NormalTok{, }\DecValTok{1}\NormalTok{),}
    \StringTok{\textasciigrave{}}\AttributeTok{Площадь полигона (км2)}\StringTok{\textasciigrave{}} \OtherTok{=} \FunctionTok{round}\NormalTok{(area\_km2, }\DecValTok{1}\NormalTok{)}
\NormalTok{  ) }\SpecialCharTok{\%\textgreater{}\%}
  \FunctionTok{select}\NormalTok{( }\CommentTok{\# Выбор и переименование колонок для итоговой таблицы}
\NormalTok{    Вариант }\OtherTok{=}\NormalTok{ label,}
    \StringTok{\textasciigrave{}}\AttributeTok{Количество тралений}\StringTok{\textasciigrave{}}\NormalTok{,}
    \StringTok{\textasciigrave{}}\AttributeTok{Длина маршрута (км)}\StringTok{\textasciigrave{}}\NormalTok{,}
    \StringTok{\textasciigrave{}}\AttributeTok{Площадь полигона (км2)}\StringTok{\textasciigrave{}}
\NormalTok{  )}

\CommentTok{\# 20. ВЫВОД ИТОГОВОЙ ТАБЛИЦЫ В КОНСОЛЬ {-}{-}{-}{-}{-}{-}{-}{-}{-}{-}{-}{-}{-}{-}{-}{-}{-}{-}{-}{-}{-}{-}{-}{-}{-}{-}{-}{-}{-}{-}{-}{-}{-}{-}{-}{-}{-}{-}{-}{-}}
\FunctionTok{cat}\NormalTok{(}\StringTok{"}\SpecialCharTok{\textbackslash{}n}\StringTok{Сводная статистика по вариантам:}\SpecialCharTok{\textbackslash{}n}\StringTok{"}\NormalTok{)}
\end{Highlighting}
\end{Shaded}

\begin{verbatim}

Сводная статистика по вариантам:
\end{verbatim}

\begin{Shaded}
\begin{Highlighting}[]
\FunctionTok{print}\NormalTok{(survey\_summary)}
\end{Highlighting}
\end{Shaded}

\begin{verbatim}
Simple feature collection with 4 features and 4 fields
Geometry type: LINESTRING
Dimension:     XY
Bounding box:  xmin: 36.71121 ymin: 68.49533 xmax: 47.7634 ymax: 74.30188
Geodetic CRS:  WGS 84
# A tibble: 4 x 5
# Groups:   Вариант [4]
  Вариант     `Количество тралений` `Длина маршрута (км)` Площадь полигона (км~1
  <fct>                       <dbl>                 <dbl>                  <dbl>
1 Original                      137                 3434.                 63101.
2 Expanded_x~                   137                 4144                  99791.
3 Expanded_x2                   137                 4810.                135352.
4 Expanded_x3                   137                 5948.                207066.
# i abbreviated name: 1: `Площадь полигона (км2)`
# i 1 more variable: geometry <LINESTRING [arc_degree]>
\end{verbatim}

\bookmarksetup{startatroot}

\chapter{Картография: повышение
разрешения}\label{ux43aux430ux440ux442ux43eux433ux440ux430ux444ux438ux44f-ux43fux43eux432ux44bux448ux435ux43dux438ux435-ux440ux430ux437ux440ux435ux448ux435ux43dux438ux44f}

Если смотреть на карты океана широко, быстро понимаешь: «зерно»
наблюдений определяет, что мы вообще способны увидеть. Масштаб меняет
саму формулировку вопросов, --- и редкая сетка
\href{https://climatedataguide.ucar.edu/climate-data/en4-subsurface-temperature-and-salinity-global-oceans}{EN4}
в Баренцевом море порой даёт не «карту», а грубую мозаику. Ресэмплинг
--- это наш способ подружить масштаб и видимость: мы увеличиваем
пространственное разрешение растра и интерполируем значения, чтобы
получить непрерывную поверхность, пригодную для изолиний и анализа.

Важно не перепутать ясность с иллюзией знания. Ум любит гладкие картинки
и готов принять их за истину; по мир полон неожиданностей, и
интерполяция не создаёт новых данных, а лишь аккуратно «зашивает»
пустоты. Всегда держим критерий хорошего объяснения: метод должен быть
воспроизводимым, проверяемым и не обещать лишнего. В этом скрипте выбран
билинейный ресэмплинг как честный компромисс между сглаживанием и
сохранением крупных структур: визуализация становится читабельной, а
исходная информативность не «приписывается задним числом».

Подход эволюционен и прагматичен --- выживает не «самая умная»
интерполяция, а та, что лучше служит цели карты и не вносит артефактов.
Мы задаём область (Баренцево море), собираем растровую поверхность из
точек в WGS84, формируем целевую сетку с более тонким шагом,
интерполируем, категоризируем значения по осмысленным бинам и строим
итоговую карту: растровая подложка, изолинии, координатная сетка,
береговая линия. Геометрия и проекции названы своими именами, а не
«растворены» в красивой картинке.

Методологический оптимизм уместен, но дисциплинирован: систематическая,
воспроизводимая обработка действительно делает нас лучше, если помнить
границы. Для строгих расчётов держим нативное разрешение; для
картографирования --- используем сглаженную сетку. Если исходная сетка
слишком редка, тестируем альтернативы (бикубика, IDW, сплайны, кригинг)
и выбираем то, что минимизирует артефакты. Это ровно тот случай, где
«будущее разума» --- не метафора: добавление предиктивных слоёв (погода,
лед) и смена интерполяторов --- путь к более умным картам.

Хорошая история опирается на честные визуализации. Этот скрипт даёт
именно такую «витрину»: карты до/после ресэмплинга и TIFF высокого
разрешения для публикаций. Он не заменяет данные --- он помогает их
видеть. А видеть лучше --- значит принимать решения, которые выдерживают
столкновение с реальностью.

Полный скрипт можно скачать по
\href{https://mombus.github.io/cRab/data/RESAMPLE.R}{ссылке}.

\textbf{Для работы скрипта:}

\begin{enumerate}
\def\labelenumi{\arabic{enumi}.}
\item
  Скачайте файл данных
  (\href{https://mombus.github.io/cRab/data/diffTemp.xlsx}{diffTemp.xlsx})
\item
  Установите рабочую директорию в setwd()
\item
  Установите необходимые пакеты.`
\end{enumerate}

Скрипт начинается с очистки рабочей среды и загрузки необходимых
пакетов. Задаются параметры области исследования: минимальная и
максимальная долгота и широта, ограничивающие регион Баренцева моря.
Загружаются данные из Excel-файла, содержащего разности температур, и
преобразуются в data.frame с колонками долготы, широты и значения
температуры. Создается исходный растровый объект из этих данных с
указанием системы координат. Определяется целевой растр с более высоким
разрешением (0.01 градуса), после чего выполняется ресэмплинг исходного
растра с использованием билинейной интерполяции для сглаживания
значений. Данные ресэмплинга преобразуются в data.frame с фильтрацией по
заданной области и категоризацией температурных значений для построения
цветовых градаций. Загружаются географические данные мировых береговых
линий и создается координатная сетка. Определяется цветовая палитра для
отображения температурных аномалий. Строится финальный график с
использованием сглаженных данных: растровая поверхность, контуры
температурных изолиний, координатная сетка и подложка мировых границ.
График сохраняется в формате TIFF с высоким разрешением.

\begin{Shaded}
\begin{Highlighting}[]
\CommentTok{\# ОЧИСТКА РАБОЧЕЙ СРЕДЫ И ЗАГРУЗКА БИБЛИОТЕК {-}{-}{-}{-}{-}{-}{-}{-}{-}{-}{-}{-}{-}{-}{-}{-}{-}{-}{-}{-}{-}{-}{-}{-}{-}{-}{-}{-}{-}{-}{-}{-}}
\FunctionTok{rm}\NormalTok{(}\AttributeTok{list =} \FunctionTok{ls}\NormalTok{())  }\CommentTok{\# Очистка среды от предыдущих объектов}

\CommentTok{\# Загрузка необходимых библиотек:}
\FunctionTok{library}\NormalTok{(dplyr)      }\CommentTok{\# Для манипуляций с данными}
\end{Highlighting}
\end{Shaded}

\begin{verbatim}

Присоединяю пакет: 'dplyr'
\end{verbatim}

\begin{verbatim}
Следующие объекты скрыты от 'package:stats':

    filter, lag
\end{verbatim}

\begin{verbatim}
Следующие объекты скрыты от 'package:base':

    intersect, setdiff, setequal, union
\end{verbatim}

\begin{Shaded}
\begin{Highlighting}[]
\FunctionTok{library}\NormalTok{(sf)         }\CommentTok{\# Для работы с пространственными данными}
\end{Highlighting}
\end{Shaded}

\begin{verbatim}
Linking to GEOS 3.13.1, GDAL 3.11.0, PROJ 9.6.0; sf_use_s2() is TRUE
\end{verbatim}

\begin{Shaded}
\begin{Highlighting}[]
\FunctionTok{library}\NormalTok{(ggplot2)    }\CommentTok{\# Для построения графиков}
\FunctionTok{library}\NormalTok{(rnaturalearth)  }\CommentTok{\# Для получения географических данных}
\FunctionTok{library}\NormalTok{(terra)      }\CommentTok{\# Для работы с растровыми данными}
\end{Highlighting}
\end{Shaded}

\begin{verbatim}
terra 1.8.60
\end{verbatim}

\begin{Shaded}
\begin{Highlighting}[]
\FunctionTok{library}\NormalTok{(metR)       }\CommentTok{\# Дополнительные функции для визуализации}

\CommentTok{\# УСТАНОВКА РАБОЧЕЙ ДИРЕКТОРИИ {-}{-}{-}{-}{-}{-}{-}{-}{-}{-}{-}{-}{-}{-}{-}{-}{-}{-}{-}{-}{-}{-}{-}{-}{-}{-}{-}{-}{-}{-}{-}{-}{-}{-}{-}{-}{-}{-}{-}{-}{-}{-}{-}{-}{-}{-}{-}}
\FunctionTok{setwd}\NormalTok{(}\StringTok{"C:/SUPERPIC/"}\NormalTok{)  }\CommentTok{\# Замените на актуальный путь к вашим данным}

\CommentTok{\# ПАРАМЕТРЫ ОБЛАСТИ ИССЛЕДОВАНИЯ {-}{-}{-}{-}{-}{-}{-}{-}{-}{-}{-}{-}{-}{-}{-}{-}{-}{-}{-}{-}{-}{-}{-}{-}{-}{-}{-}{-}{-}{-}{-}{-}{-}{-}{-}{-}{-}{-}{-}{-}{-}{-}{-}{-}{-}}
\NormalTok{xmin }\OtherTok{\textless{}{-}} \DecValTok{10}      \CommentTok{\# Минимальная долгота (границы Баренцева моря)}
\NormalTok{xmax }\OtherTok{\textless{}{-}} \DecValTok{65}      \CommentTok{\# Максимальная долгота}
\NormalTok{ymin }\OtherTok{\textless{}{-}} \DecValTok{68}      \CommentTok{\# Минимальная широта}
\NormalTok{ymax }\OtherTok{\textless{}{-}} \DecValTok{82}      \CommentTok{\# Максимальная широта}

\CommentTok{\# ПАРАМЕТРЫ РЕСЭМПЛИНГА {-}{-}{-}{-}{-}{-}{-}{-}{-}{-}{-}{-}{-}{-}{-}{-}{-}{-}{-}{-}{-}{-}{-}{-}{-}{-}{-}{-}{-}{-}{-}{-}{-}{-}{-}{-}{-}{-}{-}{-}{-}{-}{-}{-}{-}{-}{-}{-}{-}{-}{-}{-}{-}{-}}
\NormalTok{new\_res }\OtherTok{\textless{}{-}} \FloatTok{0.1}  \CommentTok{\# Новое разрешение в градусах после передискретизации}

\CommentTok{\# 1. ЗАГРУЗКА ИСХОДНЫХ ДАННЫХ {-}{-}{-}{-}{-}{-}{-}{-}{-}{-}{-}{-}{-}{-}{-}{-}{-}{-}{-}{-}{-}{-}{-}{-}{-}{-}{-}{-}{-}{-}{-}{-}{-}{-}{-}{-}{-}{-}{-}{-}{-}{-}{-}{-}{-}{-}{-}{-}{-}}
\CommentTok{\# Чтение данных из Excel{-}файла}
\NormalTok{NEMO }\OtherTok{\textless{}{-}}\NormalTok{ readxl}\SpecialCharTok{::}\FunctionTok{read\_excel}\NormalTok{(}\StringTok{"diffTemp.xlsx"}\NormalTok{, }\AttributeTok{sheet =} \StringTok{"diffTemp"}\NormalTok{)}

\CommentTok{\# Создание датафрейма с координатами и температурными данными}
\NormalTok{df }\OtherTok{\textless{}{-}} \FunctionTok{data.frame}\NormalTok{(}
  \AttributeTok{longitude =}\NormalTok{ NEMO}\SpecialCharTok{$}\NormalTok{Lon,}
  \AttributeTok{latitude =}\NormalTok{ NEMO}\SpecialCharTok{$}\NormalTok{Lat,}
  \AttributeTok{TEMP =}\NormalTok{ NEMO}\SpecialCharTok{$}\NormalTok{dif}
\NormalTok{)}

\CommentTok{\# 2. ПОДГОТОВКА ИСХОДНЫХ ДАННЫХ ДЛЯ ВИЗУАЛИЗАЦИИ {-}{-}{-}{-}{-}{-}{-}{-}{-}{-}{-}{-}{-}{-}{-}{-}{-}{-}{-}{-}{-}{-}{-}{-}{-}{-}{-}{-}{-}{-}}
\CommentTok{\# Фильтрация данных по области исследования и категоризация температур}
\NormalTok{DF }\OtherTok{\textless{}{-}} \FunctionTok{as.data.frame}\NormalTok{(df, }\AttributeTok{xy =} \ConstantTok{TRUE}\NormalTok{, }\AttributeTok{na.rm =} \ConstantTok{TRUE}\NormalTok{) }\SpecialCharTok{\%\textgreater{}\%} 
  \FunctionTok{filter}\NormalTok{(}
\NormalTok{    longitude }\SpecialCharTok{\textgreater{}=}\NormalTok{ xmin, longitude }\SpecialCharTok{\textless{}=}\NormalTok{ xmax,}
\NormalTok{    latitude }\SpecialCharTok{\textgreater{}=}\NormalTok{ ymin, latitude }\SpecialCharTok{\textless{}=}\NormalTok{ ymax}
\NormalTok{  ) }\SpecialCharTok{\%\textgreater{}\%}
  \FunctionTok{mutate}\NormalTok{(}
    \AttributeTok{TEMP\_cat =} \FunctionTok{cut}\NormalTok{(}
\NormalTok{      TEMP,}
      \AttributeTok{breaks =} \FunctionTok{c}\NormalTok{(}\SpecialCharTok{{-}}\ConstantTok{Inf}\NormalTok{, }\DecValTok{0}\NormalTok{, }\FloatTok{0.25}\NormalTok{, }\FloatTok{0.5}\NormalTok{, }\FloatTok{0.75}\NormalTok{, }\DecValTok{1}\NormalTok{, }\ConstantTok{Inf}\NormalTok{),}
      \AttributeTok{labels =} \FunctionTok{c}\NormalTok{(}\StringTok{"\textless{}0"}\NormalTok{, }\StringTok{"0\textasciitilde{}0.25"}\NormalTok{, }\StringTok{"0.25\textasciitilde{}0.5"}\NormalTok{, }\StringTok{"0.5\textasciitilde{}0.75"}\NormalTok{, }\StringTok{"0.75\textasciitilde{}1"}\NormalTok{, }\StringTok{"\textgreater{}1"}\NormalTok{),}
      \AttributeTok{include.lowest =} \ConstantTok{TRUE}
\NormalTok{    )}
\NormalTok{  )}

\CommentTok{\# 3. ЗАГРУЗКА ГЕОГРАФИЧЕСКИХ ДАННЫХ {-}{-}{-}{-}{-}{-}{-}{-}{-}{-}{-}{-}{-}{-}{-}{-}{-}{-}{-}{-}{-}{-}{-}{-}{-}{-}{-}{-}{-}{-}{-}{-}{-}{-}{-}{-}{-}{-}{-}{-}{-}{-}{-}}
\CommentTok{\# Получение данных о береговых линиях мира}
\NormalTok{world }\OtherTok{\textless{}{-}} \FunctionTok{ne\_countries}\NormalTok{(}\AttributeTok{scale =} \DecValTok{50}\NormalTok{, }\AttributeTok{returnclass =} \StringTok{"sf"}\NormalTok{) }\SpecialCharTok{\%\textgreater{}\%} 
  \FunctionTok{st\_transform}\NormalTok{(}\DecValTok{4326}\NormalTok{) }\SpecialCharTok{\%\textgreater{}\%}  \CommentTok{\# Преобразование в WGS84}
  \FunctionTok{st\_wrap\_dateline}\NormalTok{()      }\CommentTok{\# Коррекция линии перемены дат}

\CommentTok{\# Создание координатной сетки}
\NormalTok{graticule }\OtherTok{\textless{}{-}} \FunctionTok{st\_graticule}\NormalTok{(}
  \AttributeTok{lat =} \FunctionTok{seq}\NormalTok{(ymin, ymax, }\DecValTok{2}\NormalTok{),  }\CommentTok{\# Шаг по широте: 2 градуса}
  \AttributeTok{lon =} \FunctionTok{seq}\NormalTok{(xmin, xmax, }\DecValTok{5}\NormalTok{),  }\CommentTok{\# Шаг по долготе: 5 градусов}
  \AttributeTok{datum =} \FunctionTok{st\_crs}\NormalTok{(}\DecValTok{4326}\NormalTok{)       }\CommentTok{\# Система координат}
\NormalTok{)}

\CommentTok{\# 4. НАСТРОЙКА ЦВЕТОВОЙ СХЕМЫ {-}{-}{-}{-}{-}{-}{-}{-}{-}{-}{-}{-}{-}{-}{-}{-}{-}{-}{-}{-}{-}{-}{-}{-}{-}{-}{-}{-}{-}{-}{-}{-}{-}{-}{-}{-}{-}{-}{-}{-}{-}{-}{-}{-}{-}{-}{-}{-}{-}}
\CommentTok{\# Цвета для отрицательных температур (холодные тона)}
\NormalTok{cool\_colors }\OtherTok{\textless{}{-}} \FunctionTok{c}\NormalTok{(}\StringTok{"\#2171b5"}\NormalTok{, }\StringTok{"\#6baed6"}\NormalTok{, }\StringTok{"\#9ecae1"}\NormalTok{)}
\CommentTok{\# Цвета для положительных температур (теплые тона)}
\NormalTok{warm\_colors }\OtherTok{\textless{}{-}} \FunctionTok{c}\NormalTok{(}\StringTok{"\#fee391"}\NormalTok{, }\StringTok{"\#fe9929"}\NormalTok{, }\StringTok{"\#d95f0e"}\NormalTok{)}
\CommentTok{\# Объединенная палитра}
\NormalTok{palette }\OtherTok{\textless{}{-}} \FunctionTok{c}\NormalTok{(cool\_colors, warm\_colors)}

\CommentTok{\# 5. ПОСТРОЕНИЕ ГРАФИКА С ИСХОДНЫМ РАЗРЕШЕНИЕМ {-}{-}{-}{-}{-}{-}{-}{-}{-}{-}{-}{-}{-}{-}{-}{-}{-}{-}{-}{-}{-}{-}{-}{-}{-}{-}{-}{-}{-}{-}{-}{-}}
\CommentTok{\# Создание карты с исходными данными (низкое разрешение)}
\NormalTok{plot\_lowres }\OtherTok{\textless{}{-}} \FunctionTok{ggplot}\NormalTok{() }\SpecialCharTok{+}
  \CommentTok{\# Отображение данных в виде растровых плиток}
  \FunctionTok{geom\_tile}\NormalTok{(}\AttributeTok{data =}\NormalTok{ DF, }\FunctionTok{aes}\NormalTok{(}\AttributeTok{x =}\NormalTok{ longitude, }\AttributeTok{y =}\NormalTok{ latitude, }\AttributeTok{fill =}\NormalTok{ TEMP\_cat), }\AttributeTok{alpha =} \FloatTok{0.7}\NormalTok{) }\SpecialCharTok{+}
  \CommentTok{\# Добавление изолиний температур}
  \FunctionTok{geom\_contour}\NormalTok{(}
    \AttributeTok{data =}\NormalTok{ DF,}
    \FunctionTok{aes}\NormalTok{(}\AttributeTok{x =}\NormalTok{ longitude, }\AttributeTok{y =}\NormalTok{ latitude, }\AttributeTok{z =}\NormalTok{ TEMP),}
    \AttributeTok{breaks =} \FunctionTok{c}\NormalTok{(}\DecValTok{0}\NormalTok{, }\FloatTok{0.25}\NormalTok{, }\FloatTok{0.5}\NormalTok{, }\FloatTok{0.75}\NormalTok{, }\DecValTok{1}\NormalTok{),}
    \AttributeTok{color =} \StringTok{"black"}\NormalTok{,}
    \AttributeTok{linewidth =} \FloatTok{0.2}
\NormalTok{  ) }\SpecialCharTok{+}
  \CommentTok{\# Добавление координатной сетки}
  \FunctionTok{geom\_sf}\NormalTok{(}\AttributeTok{data =}\NormalTok{ graticule, }\AttributeTok{color =} \StringTok{"gray70"}\NormalTok{, }\AttributeTok{linewidth =} \FloatTok{0.3}\NormalTok{) }\SpecialCharTok{+}
  \CommentTok{\# Добавление береговых линий}
  \FunctionTok{geom\_sf}\NormalTok{(}
    \AttributeTok{data =}\NormalTok{ world, }
    \AttributeTok{color =} \StringTok{"gray30"}\NormalTok{,}
    \AttributeTok{fill =} \StringTok{"\#E8E5D6"}\NormalTok{,}
    \AttributeTok{lwd =} \FloatTok{0.3}
\NormalTok{  ) }\SpecialCharTok{+}
  \CommentTok{\# Настройка области отображения}
  \FunctionTok{coord\_sf}\NormalTok{(}
    \AttributeTok{xlim =} \FunctionTok{c}\NormalTok{(xmin, xmax),}
    \AttributeTok{ylim =} \FunctionTok{c}\NormalTok{(ymin, ymax),}
    \AttributeTok{expand =} \ConstantTok{FALSE}\NormalTok{,}
    \AttributeTok{crs =} \FunctionTok{st\_crs}\NormalTok{(}\DecValTok{4326}\NormalTok{)}
\NormalTok{  ) }\SpecialCharTok{+}
  \CommentTok{\# Настройка цветовой шкалы}
  \FunctionTok{scale\_fill\_manual}\NormalTok{(}
    \AttributeTok{name =} \StringTok{"T (°C)"}\NormalTok{,  }\CommentTok{\# Знак дельта вместо "?"}
    \AttributeTok{values =}\NormalTok{ palette,}
    \AttributeTok{drop =} \ConstantTok{FALSE}\NormalTok{,}
    \AttributeTok{na.value =} \StringTok{"grey90"}
\NormalTok{  ) }\SpecialCharTok{+}
  \CommentTok{\# Настройка внешнего вида графика}
  \FunctionTok{theme}\NormalTok{(}
    \AttributeTok{panel.background =} \FunctionTok{element\_rect}\NormalTok{(}\AttributeTok{fill =} \StringTok{"white"}\NormalTok{),}
    \AttributeTok{panel.border =} \FunctionTok{element\_rect}\NormalTok{(}\AttributeTok{color =} \StringTok{"black"}\NormalTok{, }\AttributeTok{fill =} \ConstantTok{NA}\NormalTok{, }\AttributeTok{linewidth =} \FloatTok{1.5}\NormalTok{),}
    \AttributeTok{legend.position =} \StringTok{"bottom"}\NormalTok{,}
    \AttributeTok{axis.title =} \FunctionTok{element\_blank}\NormalTok{(),}
    \AttributeTok{text =} \FunctionTok{element\_text}\NormalTok{(}\AttributeTok{size =} \DecValTok{12}\NormalTok{),}
    \AttributeTok{legend.text =} \FunctionTok{element\_text}\NormalTok{(}\AttributeTok{size =} \DecValTok{10}\NormalTok{)}
\NormalTok{  )}

\CommentTok{\# Отображение графика}
\NormalTok{plot\_lowres}
\end{Highlighting}
\end{Shaded}

\pandocbounded{\includegraphics[keepaspectratio]{chapter11_files/figure-pdf/unnamed-chunk-1-1.pdf}}

\begin{Shaded}
\begin{Highlighting}[]
\CommentTok{\# 6. ПРОЦЕДУРА РЕСЭМПЛИНГА {-}{-}{-}{-}{-}{-}{-}{-}{-}{-}{-}{-}{-}{-}{-}{-}{-}{-}{-}{-}{-}{-}{-}{-}{-}{-}{-}{-}{-}{-}{-}{-}{-}{-}{-}{-}{-}{-}{-}{-}{-}{-}{-}{-}{-}{-}{-}{-}{-}{-}{-}{-}}
\CommentTok{\# Создание исходного растра из данных}
\NormalTok{r }\OtherTok{\textless{}{-}} \FunctionTok{rast}\NormalTok{(df, }\AttributeTok{type =} \StringTok{"xyz"}\NormalTok{, }\AttributeTok{crs =} \StringTok{"EPSG:4326"}\NormalTok{)}

\CommentTok{\# Создание целевого растра с новым разрешением}
\NormalTok{target }\OtherTok{\textless{}{-}} \FunctionTok{rast}\NormalTok{(}
  \AttributeTok{extent =} \FunctionTok{ext}\NormalTok{(}\FunctionTok{c}\NormalTok{(xmin, xmax, ymin, ymax)),}
  \AttributeTok{resolution =}\NormalTok{ new\_res,}
  \AttributeTok{crs =} \StringTok{"EPSG:4326"}
\NormalTok{)}

\CommentTok{\# Выполнение ресэмплинга с билинейной интерполяцией}
\NormalTok{r\_resampled }\OtherTok{\textless{}{-}} \FunctionTok{resample}\NormalTok{(r, target, }\AttributeTok{method =} \StringTok{"bilinear"}\NormalTok{)}

\CommentTok{\# 7. ПОДГОТОВКА РЕСЭМПЛИРОВАННЫХ ДАННЫХ {-}{-}{-}{-}{-}{-}{-}{-}{-}{-}{-}{-}{-}{-}{-}{-}{-}{-}{-}{-}{-}{-}{-}{-}{-}{-}{-}{-}{-}{-}{-}{-}{-}{-}{-}{-}{-}{-}{-}}
\CommentTok{\# Преобразование растра в датафрейм для ggplot2}
\NormalTok{TEMPERATURE }\OtherTok{\textless{}{-}} \FunctionTok{as.data.frame}\NormalTok{(r\_resampled, }\AttributeTok{xy =} \ConstantTok{TRUE}\NormalTok{, }\AttributeTok{na.rm =} \ConstantTok{TRUE}\NormalTok{) }\SpecialCharTok{\%\textgreater{}\%} 
  \FunctionTok{filter}\NormalTok{(}
\NormalTok{    x }\SpecialCharTok{\textgreater{}=}\NormalTok{ xmin, x }\SpecialCharTok{\textless{}=}\NormalTok{ xmax,}
\NormalTok{    y }\SpecialCharTok{\textgreater{}=}\NormalTok{ ymin, y }\SpecialCharTok{\textless{}=}\NormalTok{ ymax}
\NormalTok{  ) }\SpecialCharTok{\%\textgreater{}\%}
  \FunctionTok{mutate}\NormalTok{(}
    \AttributeTok{TEMP\_cat =} \FunctionTok{cut}\NormalTok{(}
\NormalTok{      TEMP,}
      \AttributeTok{breaks =} \FunctionTok{c}\NormalTok{(}\SpecialCharTok{{-}}\ConstantTok{Inf}\NormalTok{, }\DecValTok{0}\NormalTok{, }\FloatTok{0.25}\NormalTok{, }\FloatTok{0.5}\NormalTok{, }\FloatTok{0.75}\NormalTok{, }\DecValTok{1}\NormalTok{, }\ConstantTok{Inf}\NormalTok{),}
      \AttributeTok{labels =} \FunctionTok{c}\NormalTok{(}\StringTok{"\textless{}0"}\NormalTok{, }\StringTok{"0\textasciitilde{}0.25"}\NormalTok{, }\StringTok{"0.25\textasciitilde{}0.5"}\NormalTok{, }\StringTok{"0.5\textasciitilde{}0.75"}\NormalTok{, }\StringTok{"0.75\textasciitilde{}1"}\NormalTok{, }\StringTok{"\textgreater{}1"}\NormalTok{),}
      \AttributeTok{include.lowest =} \ConstantTok{TRUE}
\NormalTok{    )}
\NormalTok{  )}

\CommentTok{\# 8. ПОСТРОЕНИЕ ГРАФИКА С ВЫСОКИМ РАЗРЕШЕНИЕМ {-}{-}{-}{-}{-}{-}{-}{-}{-}{-}{-}{-}{-}{-}{-}{-}{-}{-}{-}{-}{-}{-}{-}{-}{-}{-}{-}{-}{-}{-}{-}{-}{-}}
\CommentTok{\# Создание карты с ресэмплированными данными}
\NormalTok{final\_plot }\OtherTok{\textless{}{-}} \FunctionTok{ggplot}\NormalTok{() }\SpecialCharTok{+}
  \FunctionTok{geom\_tile}\NormalTok{(}\AttributeTok{data =}\NormalTok{ TEMPERATURE, }\FunctionTok{aes}\NormalTok{(}\AttributeTok{x =}\NormalTok{ x, }\AttributeTok{y =}\NormalTok{ y, }\AttributeTok{fill =}\NormalTok{ TEMP\_cat), }\AttributeTok{alpha =} \FloatTok{0.7}\NormalTok{) }\SpecialCharTok{+}
  \FunctionTok{geom\_contour}\NormalTok{(}
    \AttributeTok{data =}\NormalTok{ TEMPERATURE,}
    \FunctionTok{aes}\NormalTok{(}\AttributeTok{x =}\NormalTok{ x, }\AttributeTok{y =}\NormalTok{ y, }\AttributeTok{z =}\NormalTok{ TEMP),}
    \AttributeTok{breaks =} \FunctionTok{c}\NormalTok{(}\DecValTok{0}\NormalTok{, }\FloatTok{0.25}\NormalTok{, }\FloatTok{0.5}\NormalTok{, }\FloatTok{0.75}\NormalTok{, }\DecValTok{1}\NormalTok{),}
    \AttributeTok{color =} \StringTok{"black"}\NormalTok{,}
    \AttributeTok{linewidth =} \FloatTok{0.2}
\NormalTok{  ) }\SpecialCharTok{+}
  \FunctionTok{geom\_sf}\NormalTok{(}\AttributeTok{data =}\NormalTok{ graticule, }\AttributeTok{color =} \StringTok{"gray70"}\NormalTok{, }\AttributeTok{linewidth =} \FloatTok{0.3}\NormalTok{) }\SpecialCharTok{+}
  \FunctionTok{geom\_sf}\NormalTok{(}
    \AttributeTok{data =}\NormalTok{ world, }
    \AttributeTok{color =} \StringTok{"gray30"}\NormalTok{,}
    \AttributeTok{fill =} \StringTok{"\#E8E5D6"}\NormalTok{,}
    \AttributeTok{lwd =} \FloatTok{0.3}
\NormalTok{  ) }\SpecialCharTok{+}
  \FunctionTok{coord\_sf}\NormalTok{(}
    \AttributeTok{xlim =} \FunctionTok{c}\NormalTok{(xmin, xmax),}
    \AttributeTok{ylim =} \FunctionTok{c}\NormalTok{(ymin, ymax),}
    \AttributeTok{expand =} \ConstantTok{FALSE}\NormalTok{,}
    \AttributeTok{crs =} \FunctionTok{st\_crs}\NormalTok{(}\DecValTok{4326}\NormalTok{)}
\NormalTok{  ) }\SpecialCharTok{+}
  \FunctionTok{scale\_fill\_manual}\NormalTok{(}
    \AttributeTok{name =} \StringTok{"T (°C)"}\NormalTok{,  }\CommentTok{\# Знак дельта вместо "?"}
    \AttributeTok{values =}\NormalTok{ palette,}
    \AttributeTok{drop =} \ConstantTok{FALSE}\NormalTok{,}
    \AttributeTok{na.value =} \StringTok{"grey90"}
\NormalTok{  ) }\SpecialCharTok{+}
  \FunctionTok{theme}\NormalTok{(}
    \AttributeTok{panel.background =} \FunctionTok{element\_rect}\NormalTok{(}\AttributeTok{fill =} \StringTok{"white"}\NormalTok{),}
    \AttributeTok{panel.border =} \FunctionTok{element\_rect}\NormalTok{(}\AttributeTok{color =} \StringTok{"black"}\NormalTok{, }\AttributeTok{fill =} \ConstantTok{NA}\NormalTok{, }\AttributeTok{linewidth =} \FloatTok{1.5}\NormalTok{),}
    \AttributeTok{legend.position =} \StringTok{"bottom"}\NormalTok{,}
    \AttributeTok{axis.title =} \FunctionTok{element\_blank}\NormalTok{(),}
    \AttributeTok{text =} \FunctionTok{element\_text}\NormalTok{(}\AttributeTok{size =} \DecValTok{12}\NormalTok{),}
    \AttributeTok{legend.text =} \FunctionTok{element\_text}\NormalTok{(}\AttributeTok{size =} \DecValTok{10}\NormalTok{)}
\NormalTok{  )}

\CommentTok{\# Отображение финального графика}
\NormalTok{final\_plot}
\end{Highlighting}
\end{Shaded}

\pandocbounded{\includegraphics[keepaspectratio]{chapter11_files/figure-pdf/unnamed-chunk-1-2.pdf}}

\begin{Shaded}
\begin{Highlighting}[]
\CommentTok{\# 9. СОХРАНЕНИЕ РЕЗУЛЬТАТОВ {-}{-}{-}{-}{-}{-}{-}{-}{-}{-}{-}{-}{-}{-}{-}{-}{-}{-}{-}{-}{-}{-}{-}{-}{-}{-}{-}{-}{-}{-}{-}{-}{-}{-}{-}{-}{-}{-}{-}{-}{-}{-}{-}{-}{-}{-}{-}{-}{-}{-}{-}}
\CommentTok{\# Сохранение карты с высоким разрешением в файл}
\FunctionTok{ggsave}\NormalTok{(}
  \AttributeTok{filename =} \StringTok{"Temperature\_Map.tiff"}\NormalTok{,}
  \AttributeTok{plot =}\NormalTok{ final\_plot,}
  \AttributeTok{device =} \StringTok{"tiff"}\NormalTok{,}
  \AttributeTok{width =} \DecValTok{17}\NormalTok{,}
  \AttributeTok{height =} \DecValTok{15}\NormalTok{,}
  \AttributeTok{units =} \StringTok{"cm"}\NormalTok{,}
  \AttributeTok{dpi =} \DecValTok{600}\NormalTok{,}
  \AttributeTok{compression =} \StringTok{"lzw"}\NormalTok{,}
  \AttributeTok{bg =} \StringTok{"white"}
\NormalTok{)}

\CommentTok{\# Дополнительно: сохранение карты с исходным разрешением}
\FunctionTok{ggsave}\NormalTok{(}
  \AttributeTok{filename =} \StringTok{"Temperature\_Map\_LowRes.tiff"}\NormalTok{,}
  \AttributeTok{plot =}\NormalTok{ plot\_lowres,}
  \AttributeTok{device =} \StringTok{"tiff"}\NormalTok{,}
  \AttributeTok{width =} \DecValTok{17}\NormalTok{,}
  \AttributeTok{height =} \DecValTok{15}\NormalTok{,}
  \AttributeTok{units =} \StringTok{"cm"}\NormalTok{,}
  \AttributeTok{dpi =} \DecValTok{600}\NormalTok{,}
  \AttributeTok{compression =} \StringTok{"lzw"}\NormalTok{,}
  \AttributeTok{bg =} \StringTok{"white"}
\NormalTok{)}
\end{Highlighting}
\end{Shaded}

\bookmarksetup{startatroot}

\chapter{SDM: моделирование пространственного распределения
видов}\label{sdm-ux43cux43eux434ux435ux43bux438ux440ux43eux432ux430ux43dux438ux435-ux43fux440ux43eux441ux442ux440ux430ux43dux441ux442ux432ux435ux43dux43dux43eux433ux43e-ux440ux430ux441ux43fux440ux435ux434ux435ux43bux435ux43dux438ux44f-ux432ux438ux434ux43eux432}

\section{Введение}\label{ux432ux432ux435ux434ux435ux43dux438ux435-11}

Представьте, что вы пытаетесь услышать шёпот в шумной комнате. Шум ---
это всё остальное: температура, освещение, посторонние разговоры. Шёпот
--- сигнал, который вам нужен. Именно так обстоит дело с моделированием
пространственного распределения видов (SDM). Когда мы видим точки на
карте, где вид был обнаружен, наш мозг мгновенно дорисовывает
причинно-следственные связи: «Он здесь, потому что тут холодно» или «Его
привлекает эта глубина». Но реальность сложнее. Точки наблюдений --- это
не чистый сигнал о предпочтениях вида, а сложная смесь его истинной
экологической ниши, доступности мест обитания, усилий исследователей и
случайных факторов. Наша задача --- отделить шёпот от шума.

Что такое SDM? По своей сути, это попытка найти вероятность присутствия
вида в зависимости от ковариат --- био-физических переменных среды. Мы
хотим построить функцию, которая из сырых, зашумленных данных наблюдений
извлекает устойчивые паттерны: как вид реагирует на температуру,
солёность, глубину, расстояние до берега. Но здесь нас подстерегает та
же ловушка, что и с CPUE: если не аккуратно отделить сигнал от шума, мы
получим термометр, который показывает температуру в комнате, а не у
больного.

Почему это так важно? Потому что SDM --- это мост между теорией и
практикой. Он позволяет предсказать, где вид может обитать в условиях
меняющегося климата, где стоит искать новые популяции, как защитить
уязвимые местообитания. Но этот мост должен быть построен на прочном
фундаменте, а не на песке иллюзий и переобученных моделей.

В этом занятии мы пройдем весь путь: от сырых данных до ансамблевых
прогнозов. Мы будем использовать три скрипта:

Подготовка данных: мы создадим регулярную сетку, агрегируем данные,
исключим сушу, рассчитаем расстояние до берега и извлечем био-физические
переменные из NetCDF файлов. Это основа, своего рода «чистка» данных от
артефактов и приведение их к единому формату.

Выбор предикторов: здесь мы применим машинное обучение чтобы выявить
сложные, нелинейные зависимости между средой и присутствием вида. Мы
будем бороться с переобучением, мультиколлинеарностью и шумом, используя
методы отбора признаков (Boruta, LASSO) и кросс-валидацию.

Моделирование, ансамблирование и прогноз: поскольку ни одна модель не
идеальна, мы объединим несколько моделей в ансамбль, чтобы снизить
неопределенность и получить более надежные прогнозы. Мы спроецируем эти
модели на будущие сценарии, оценим риски экстраполяции с помощью
MESS-анализа и визуализируем результаты.

Как и в примере с CPUE, мы не стремимся победить неопределенность, а
хотим честно на нее посмотреть. Мы покажем не только карты вероятностей,
но и доверительные интервалы, не только предсказания, но и диагностику
моделей. Это тот подход, который работает в долгую: меньше сказок,
больше науки.

И помните: хорошая SDM --- это не просто «черный ящик», который выдает
прогнозы. Это тщательно настроенный инструмент, который помогает
услышать шёпот вида в шуме данных. Давайте начнем.

\section{Данные и
скрипты}\label{ux434ux430ux43dux43dux44bux435-ux438-ux441ux43aux440ux438ux43fux442ux44b}

Для минимальной работы 3-го скрипта необходимо подгрузить два csv-файла.
Первый
\href{https://mombus.github.io/cRab/data/final_sdm_table_with_na.csv}{файл}
для моделирования текущих данных и
\href{https://mombus.github.io/cRab/data/future_sdm_table_with_na.csv}{файл}
- данные по будущему распределению предикторов, собранных с
\href{https://www.bio-oracle.org/}{Bio-ORACLE}. Скрипты целиком:
\href{https://mombus.github.io/cRab/data/SDM_prepare_input_data.R}{первый},
\href{https://mombus.github.io/cRab/data/SDM_pred.R}{второй} и
\href{https://mombus.github.io/cRab/data/SDM_diag.R}{третий}.

\begin{Shaded}
\begin{Highlighting}[]
\CommentTok{\# ========================================================================================================================}
\CommentTok{\# ПОДГОТОВКА ДАННЫХ ДЛЯ МОДЕЛИРОВАНИЯ ПРОСТРАНСТВЕННОГО РАСПРЕДЕЛЕНИЯ ВИДОВ (SDM)}
\CommentTok{\# }
\CommentTok{\# Скрипт выполняет:}
\CommentTok{\# 1. Загрузку и фильтрацию данных наблюдений}
\CommentTok{\# 2. Создание регулярной сетки для анализа}
\CommentTok{\# 3. Агрегацию данных по ячейкам сетки}
\CommentTok{\# 4. Исключение наземных территорий}
\CommentTok{\# 5. Расчет расстояния до берега}
\CommentTok{\# 6. Извлечение био{-}физических переменных из NetCDF файлов}
\CommentTok{\# 7. Сохранение итогового набора данных}
\CommentTok{\# }
\CommentTok{\# Курс: "Оценка водных биоресурсов в среде R (для начинающих)"}
\CommentTok{\# Автор: Баканев С. В. }
\CommentTok{\# Дата: 27.08.2025}
\CommentTok{\# ========================================================================================================================}
\CommentTok{\# Очистка рабочей среды}
\FunctionTok{rm}\NormalTok{(}\AttributeTok{list =} \FunctionTok{ls}\NormalTok{())}

\CommentTok{\# Установка рабочей директории}
\FunctionTok{setwd}\NormalTok{(}\StringTok{"C:/SDM"}\NormalTok{)}

\CommentTok{\# Загрузка необходимых библиотек}
\FunctionTok{suppressPackageStartupMessages}\NormalTok{(\{}
\FunctionTok{library}\NormalTok{(tidyverse)    }\CommentTok{\# Обработка данных и визуализация}
\FunctionTok{library}\NormalTok{(readxl)       }\CommentTok{\# Чтение Excel{-}файлов}
\FunctionTok{library}\NormalTok{(rnaturalearth) }\CommentTok{\# Векторные карты мира}
\FunctionTok{library}\NormalTok{(sf)           }\CommentTok{\# Пространственный анализ}
\FunctionTok{library}\NormalTok{(ggOceanMaps)  }\CommentTok{\# Расчет дистанции до берега}
\FunctionTok{library}\NormalTok{(terra)        }\CommentTok{\# Работа с растровыми данными}
\NormalTok{\})}

\CommentTok{\# {-}{-}{-}{-}{-}{-}{-}{-}{-}{-}{-}{-}{-}{-}{-}{-}{-}{-}{-}{-}{-}{-}{-}{-}{-}{-}{-}}
\CommentTok{\# 2. ЗАГРУЗКА И ФИЛЬТРАЦИЯ ДАННЫХ}
\CommentTok{\# {-}{-}{-}{-}{-}{-}{-}{-}{-}{-}{-}{-}{-}{-}{-}{-}{-}{-}{-}{-}{-}{-}{-}{-}{-}{-}{-}}
\NormalTok{DATA }\OtherTok{\textless{}{-}} \FunctionTok{read\_excel}\NormalTok{(}\StringTok{"PECTEN.xlsx"}\NormalTok{, }\AttributeTok{sheet =} \StringTok{"PECTEN"}\NormalTok{)}
\FunctionTok{str}\NormalTok{(DATA)}
\end{Highlighting}
\end{Shaded}

\begin{Shaded}
\begin{Highlighting}[]
\NormalTok{tibble [}\DecValTok{6}\NormalTok{,}\DecValTok{573}\NormalTok{ x }\DecValTok{4}\NormalTok{] (S3}\SpecialCharTok{:}\NormalTok{ tbl\_df}\SpecialCharTok{/}\NormalTok{tbl}\SpecialCharTok{/}\NormalTok{data.frame)}
 \SpecialCharTok{$}\NormalTok{ TYPE}\SpecialCharTok{:}\NormalTok{ chr [}\DecValTok{1}\SpecialCharTok{:}\DecValTok{6573}\NormalTok{] }\StringTok{"A"} \StringTok{"A"} \StringTok{"A"} \StringTok{"A"}\NormalTok{ ...}
 \SpecialCharTok{$}\NormalTok{ Y   }\SpecialCharTok{:}\NormalTok{ num [}\DecValTok{1}\SpecialCharTok{:}\DecValTok{6573}\NormalTok{] }\FloatTok{28.2} \FloatTok{32.8} \FloatTok{32.8} \FloatTok{32.8} \FloatTok{32.8}\NormalTok{ ...}
 \SpecialCharTok{$}\NormalTok{ X   }\SpecialCharTok{:}\NormalTok{ num [}\DecValTok{1}\SpecialCharTok{:}\DecValTok{6573}\NormalTok{] }\SpecialCharTok{{-}}\FloatTok{15.75} \FloatTok{13.25} \SpecialCharTok{{-}}\FloatTok{9.25} \SpecialCharTok{{-}}\FloatTok{16.75} \SpecialCharTok{{-}}\FloatTok{17.25}\NormalTok{ ...}
 \SpecialCharTok{$}\NormalTok{ OCC }\SpecialCharTok{:}\NormalTok{ num [}\DecValTok{1}\SpecialCharTok{:}\DecValTok{6573}\NormalTok{] }\DecValTok{1} \DecValTok{1} \DecValTok{1} \DecValTok{1} \DecValTok{1} \DecValTok{1} \DecValTok{1} \DecValTok{1} \DecValTok{1} \DecValTok{1}\NormalTok{ ...}
\end{Highlighting}
\end{Shaded}

\begin{Shaded}
\begin{Highlighting}[]
\CommentTok{\# Получение границ Европы}
\NormalTok{europe }\OtherTok{\textless{}{-}} \FunctionTok{ne\_countries}\NormalTok{(}\AttributeTok{scale =} \DecValTok{10}\NormalTok{, }\AttributeTok{continent =} \StringTok{"Europe"}\NormalTok{)  }\CommentTok{\# Загрузка векторных границ Европы (масштаб 1:10м)}


\CommentTok{\# Установка границ отображаемой области (долгота/широта)}
\NormalTok{xmin}\OtherTok{=}\DecValTok{10}  \CommentTok{\# Западная граница}
\NormalTok{xmax}\OtherTok{=}\DecValTok{45}  \CommentTok{\# Восточная граница}
\NormalTok{ymin}\OtherTok{=}\DecValTok{66} \CommentTok{\# Южная граница}
\NormalTok{ymax}\OtherTok{=}\DecValTok{72} \CommentTok{\# Северная граница}

\CommentTok{\# Фильтрация данных по заданным координатам}
\NormalTok{PECTEN }\OtherTok{\textless{}{-}}\NormalTok{ DATA }\SpecialCharTok{\%\textgreater{}\%}
  \FunctionTok{filter}\NormalTok{(X }\SpecialCharTok{\textgreater{}=}\NormalTok{ xmin }\SpecialCharTok{\&}\NormalTok{ X }\SpecialCharTok{\textless{}=}\NormalTok{ xmax }\SpecialCharTok{\&}\NormalTok{ Y }\SpecialCharTok{\textgreater{}=}\NormalTok{ ymin }\SpecialCharTok{\&}\NormalTok{ Y }\SpecialCharTok{\textless{}=}\NormalTok{ ymax) }\SpecialCharTok{\%\textgreater{}\%}
  \FunctionTok{select}\NormalTok{(X, Y, OCC)  }\CommentTok{\# Выбор только нужных колонок}

\CommentTok{\# Построение карты}
\FunctionTok{ggplot}\NormalTok{() }\SpecialCharTok{+}
  \CommentTok{\# Базовая карта Европы}
  \FunctionTok{geom\_sf}\NormalTok{(}\AttributeTok{data =}\NormalTok{ europe, }\AttributeTok{fill =} \StringTok{"\#E8E5D6"}\NormalTok{) }\SpecialCharTok{+} 
  \CommentTok{\# Ограничение области отображения}
  \FunctionTok{coord\_sf}\NormalTok{(}\AttributeTok{xlim =} \FunctionTok{c}\NormalTok{(xmin, xmax), }\AttributeTok{ylim =} \FunctionTok{c}\NormalTok{(ymin, ymax)) }\SpecialCharTok{+}
  \CommentTok{\# Точки наблюдений с размером и цветом по переменной OCC}
  \FunctionTok{geom\_point}\NormalTok{(}\FunctionTok{aes}\NormalTok{(}\AttributeTok{x =}\NormalTok{ X, }\AttributeTok{y =}\NormalTok{ Y, }\AttributeTok{size =}\NormalTok{ OCC, }\AttributeTok{color =}\NormalTok{ OCC),}
             \AttributeTok{data =}\NormalTok{ PECTEN, }\AttributeTok{alpha =} \FloatTok{0.6}\NormalTok{) }\SpecialCharTok{+}
  \CommentTok{\# Цветовая шкала (viridis, вариант H)}
  \FunctionTok{scale\_color\_viridis\_c}\NormalTok{(}\AttributeTok{option =} \StringTok{"H"}\NormalTok{) }\SpecialCharTok{+}
  \CommentTok{\# Подписи осей}
  \FunctionTok{labs}\NormalTok{(}\AttributeTok{x =} \StringTok{"Долгота"}\NormalTok{, }\AttributeTok{y =} \StringTok{"Широта"}\NormalTok{, }
       \AttributeTok{size =} \StringTok{"Наличие вида"}\NormalTok{, }\AttributeTok{color =} \StringTok{"Наличие вида"}\NormalTok{,}
       \AttributeTok{title =} \StringTok{"Распределение вида"}\NormalTok{)}
\end{Highlighting}
\end{Shaded}

\begin{figure}[H]

{\centering \includegraphics[width=0.6\linewidth,height=\textheight,keepaspectratio]{images/SDM1.PNG}

}

\caption{Рис. 1.: Данные по встречаемости вида}

\end{figure}%

\begin{Shaded}
\begin{Highlighting}[]
\CommentTok{\# Просмотр отфильтрованной таблицы}
\FunctionTok{str}\NormalTok{(PECTEN)}
\CommentTok{\# {-}{-}{-}{-}{-}{-}{-}{-}{-}{-}{-}{-}{-}{-}{-}{-}{-}{-}{-}{-}{-}{-}{-}{-}{-}{-}{-}}
\CommentTok{\# 3. СОЗДАНИЕ СЕТКИ И АГРЕГАЦИЯ ДАННЫХ}
\CommentTok{\# {-}{-}{-}{-}{-}{-}{-}{-}{-}{-}{-}{-}{-}{-}{-}{-}{-}{-}{-}{-}{-}{-}{-}{-}{-}{-}{-}}

\CommentTok{\# Если объект europe не в формате sf, приведем:}
\NormalTok{europe }\OtherTok{\textless{}{-}} \FunctionTok{suppressWarnings}\NormalTok{(sf}\SpecialCharTok{::}\FunctionTok{st\_as\_sf}\NormalTok{(europe))}

\CommentTok{\# Границы бинов}
\NormalTok{x\_breaks }\OtherTok{\textless{}{-}} \FunctionTok{seq}\NormalTok{(xmin, xmax, }\AttributeTok{by =} \FloatTok{0.05}\NormalTok{)}
\NormalTok{y\_breaks }\OtherTok{\textless{}{-}} \FunctionTok{seq}\NormalTok{(ymin, ymax, }\AttributeTok{by =} \FloatTok{0.05}\NormalTok{)}

\CommentTok{\# 2.1. Присвоим наблюдениям индексы ячеек и посчитаем среднее OCC по ячейке}
\NormalTok{PECTEN\_binned }\OtherTok{\textless{}{-}}\NormalTok{ PECTEN }\SpecialCharTok{\%\textgreater{}\%}
  \FunctionTok{mutate}\NormalTok{(}
    \AttributeTok{x\_id =} \FunctionTok{cut}\NormalTok{(X, }\AttributeTok{breaks =}\NormalTok{ x\_breaks, }\AttributeTok{include.lowest =} \ConstantTok{TRUE}\NormalTok{, }\AttributeTok{right =} \ConstantTok{FALSE}\NormalTok{, }\AttributeTok{labels =} \ConstantTok{FALSE}\NormalTok{),}
    \AttributeTok{y\_id =} \FunctionTok{cut}\NormalTok{(Y, }\AttributeTok{breaks =}\NormalTok{ y\_breaks, }\AttributeTok{include.lowest =} \ConstantTok{TRUE}\NormalTok{, }\AttributeTok{right =} \ConstantTok{FALSE}\NormalTok{, }\AttributeTok{labels =} \ConstantTok{FALSE}\NormalTok{)}
\NormalTok{  ) }\SpecialCharTok{\%\textgreater{}\%}
  \CommentTok{\# точки, попавшие ровно в xmax/ymax, уйдут в NA — это нормально, т.к. верхняя граница полуоткрытая}
  \FunctionTok{filter}\NormalTok{(}\SpecialCharTok{!}\FunctionTok{is.na}\NormalTok{(x\_id), }\SpecialCharTok{!}\FunctionTok{is.na}\NormalTok{(y\_id)) }\SpecialCharTok{\%\textgreater{}\%}
  \FunctionTok{group\_by}\NormalTok{(x\_id, y\_id) }\SpecialCharTok{\%\textgreater{}\%}
  \FunctionTok{summarise}\NormalTok{(}\AttributeTok{OCC =} \FunctionTok{mean}\NormalTok{(OCC, }\AttributeTok{na.rm =} \ConstantTok{TRUE}\NormalTok{), }\AttributeTok{.groups =} \StringTok{"drop"}\NormalTok{) }\SpecialCharTok{\%\textgreater{}\%}
  \FunctionTok{mutate}\NormalTok{(}\AttributeTok{OCC =} \FunctionTok{na\_if}\NormalTok{(OCC, }\ConstantTok{NaN}\NormalTok{))  }\CommentTok{\# если все NA в ячейке \textgreater{} NA, а не NaN}

\CommentTok{\# 2.2. Полная таблица ячеек (все комбинации), координаты центров}
\NormalTok{grid\_cells }\OtherTok{\textless{}{-}}\NormalTok{ tidyr}\SpecialCharTok{::}\FunctionTok{expand\_grid}\NormalTok{(}
  \AttributeTok{x\_id =} \FunctionTok{seq\_along}\NormalTok{(}\FunctionTok{head}\NormalTok{(x\_breaks, }\SpecialCharTok{{-}}\DecValTok{1}\NormalTok{)),}
  \AttributeTok{y\_id =} \FunctionTok{seq\_along}\NormalTok{(}\FunctionTok{head}\NormalTok{(y\_breaks, }\SpecialCharTok{{-}}\DecValTok{1}\NormalTok{))}
\NormalTok{) }\SpecialCharTok{\%\textgreater{}\%}
  \FunctionTok{mutate}\NormalTok{(}
    \AttributeTok{X\_center =}\NormalTok{ (x\_breaks[x\_id] }\SpecialCharTok{+}\NormalTok{ x\_breaks[x\_id }\SpecialCharTok{+} \DecValTok{1}\NormalTok{]) }\SpecialCharTok{/} \DecValTok{2}\NormalTok{,}
    \AttributeTok{Y\_center =}\NormalTok{ (y\_breaks[y\_id] }\SpecialCharTok{+}\NormalTok{ y\_breaks[y\_id }\SpecialCharTok{+} \DecValTok{1}\NormalTok{]) }\SpecialCharTok{/} \DecValTok{2}
\NormalTok{  )}

\CommentTok{\# 2.3. Приклеим средние OCC к полной решетке (пустые \textgreater{} NA)}
\NormalTok{grid\_occ }\OtherTok{\textless{}{-}}\NormalTok{ grid\_cells }\SpecialCharTok{\%\textgreater{}\%}
  \FunctionTok{left\_join}\NormalTok{(PECTEN\_binned, }\AttributeTok{by =} \FunctionTok{c}\NormalTok{(}\StringTok{"x\_id"}\NormalTok{, }\StringTok{"y\_id"}\NormalTok{))}

\CommentTok{\# 2.4. Оставим только ячейки океана (центры, не попадающие на сушу Европы)}
\NormalTok{centers\_sf }\OtherTok{\textless{}{-}} \FunctionTok{st\_as\_sf}\NormalTok{(grid\_occ, }\AttributeTok{coords =} \FunctionTok{c}\NormalTok{(}\StringTok{"X\_center"}\NormalTok{, }\StringTok{"Y\_center"}\NormalTok{), }\AttributeTok{crs =} \DecValTok{4326}\NormalTok{, }\AttributeTok{remove =} \ConstantTok{FALSE}\NormalTok{)}
\NormalTok{on\_land\_mat }\OtherTok{\textless{}{-}} \FunctionTok{st\_intersects}\NormalTok{(centers\_sf, europe, }\AttributeTok{sparse =} \ConstantTok{FALSE}\NormalTok{)}
\NormalTok{on\_land }\OtherTok{\textless{}{-}} \FunctionTok{apply}\NormalTok{(on\_land\_mat, }\DecValTok{1}\NormalTok{, any)}

\NormalTok{grid\_ocean }\OtherTok{\textless{}{-}}\NormalTok{ grid\_occ }\SpecialCharTok{\%\textgreater{}\%}
  \FunctionTok{filter}\NormalTok{(}\SpecialCharTok{!}\NormalTok{on\_land) }\SpecialCharTok{\%\textgreater{}\%}
  \FunctionTok{select}\NormalTok{(X\_center, Y\_center, OCC) }\SpecialCharTok{\%\textgreater{}\%}
  \FunctionTok{arrange}\NormalTok{(Y\_center, X\_center)}

\CommentTok{\# Переименование столбцов}
\NormalTok{grid\_ocean }\OtherTok{\textless{}{-}}\NormalTok{ grid\_ocean }\SpecialCharTok{\%\textgreater{}\%}
  \FunctionTok{rename}\NormalTok{(}\AttributeTok{X =}\NormalTok{ X\_center,}
         \AttributeTok{Y =}\NormalTok{ Y\_center,}
         \AttributeTok{OCC =}\NormalTok{ OCC)}

\CommentTok{\# Вывод результата: таблица центров ячеек и средняя OCC (NA, если нет данных)}

\FunctionTok{str}\NormalTok{(grid\_ocean)}

\CommentTok{\# Построение карты+ проверка сетки}
\FunctionTok{ggplot}\NormalTok{() }\SpecialCharTok{+}
  \CommentTok{\# Базовая карта Европы}
  \FunctionTok{geom\_sf}\NormalTok{(}\AttributeTok{data =}\NormalTok{ europe, }\AttributeTok{fill =} \StringTok{"\#E8E5D6"}\NormalTok{) }\SpecialCharTok{+} 
  \CommentTok{\# Ограничение области отображения}
  \FunctionTok{coord\_sf}\NormalTok{(}\AttributeTok{xlim =} \FunctionTok{c}\NormalTok{(xmin, xmax), }\AttributeTok{ylim =} \FunctionTok{c}\NormalTok{(ymin, ymax)) }\SpecialCharTok{+}
  \CommentTok{\# Точки наблюдений с размером и цветом по переменной OCC}
  \FunctionTok{geom\_point}\NormalTok{(}\FunctionTok{aes}\NormalTok{(}\AttributeTok{x =}\NormalTok{ X, }\AttributeTok{y =}\NormalTok{ Y, }\AttributeTok{size =}\NormalTok{ X, }\AttributeTok{color =}\NormalTok{ Y),}
             \AttributeTok{data =}\NormalTok{ grid\_ocean, }\AttributeTok{alpha =} \FloatTok{0.6}\NormalTok{) }\SpecialCharTok{+}
  \CommentTok{\# Цветовая шкала (viridis, вариант H)}
  \FunctionTok{scale\_color\_viridis\_c}\NormalTok{(}\AttributeTok{option =} \StringTok{"H"}\NormalTok{) }\SpecialCharTok{+}
  \CommentTok{\# Подписи осей}
  \FunctionTok{labs}\NormalTok{(}\AttributeTok{x =} \StringTok{"Долгота"}\NormalTok{, }\AttributeTok{y =} \StringTok{"Широта"}\NormalTok{, }
       \AttributeTok{size =} \StringTok{"Наличие вида"}\NormalTok{, }\AttributeTok{color =} \StringTok{"Наличие вида"}\NormalTok{,}
       \AttributeTok{title =} \StringTok{"Распределение сетки"}\NormalTok{)}
\end{Highlighting}
\end{Shaded}

\begin{figure}[H]

{\centering \includegraphics[width=0.6\linewidth,height=\textheight,keepaspectratio]{images/SDM2.PNG}

}

\caption{Рис. 2.: Визуализация сетки для моделирования}

\end{figure}%

\begin{Shaded}
\begin{Highlighting}[]
\CommentTok{\# Присоединение к базовой таблице (X, Y, OCC) переменной DIST (рсстояние до берега)}
\NormalTok{dt }\OtherTok{\textless{}{-}} \FunctionTok{data.frame}\NormalTok{(}\AttributeTok{lon =}\NormalTok{ grid\_ocean}\SpecialCharTok{$}\NormalTok{X, }\AttributeTok{lat =}\NormalTok{ grid\_ocean}\SpecialCharTok{$}\NormalTok{Y)}
\NormalTok{dt }\OtherTok{\textless{}{-}} \FunctionTok{dist2land}\NormalTok{(dt, }\AttributeTok{verbose =} \ConstantTok{FALSE}\NormalTok{)}
\FunctionTok{qmap}\NormalTok{(dt, }\AttributeTok{color =}\NormalTok{ ldist) }\SpecialCharTok{+} \FunctionTok{scale\_color\_viridis\_c}\NormalTok{()}
\end{Highlighting}
\end{Shaded}

\begin{figure}[H]

{\centering \includegraphics[width=0.6\linewidth,height=\textheight,keepaspectratio]{images/SDM3.PNG}

}

\caption{Рис. 3.: Визуализация переменной (дистанция до берега)}

\end{figure}%

\begin{Shaded}
\begin{Highlighting}[]
\NormalTok{DATA }\OtherTok{\textless{}{-}} \FunctionTok{tibble}\NormalTok{(}
  \AttributeTok{X =}\NormalTok{ grid\_ocean}\SpecialCharTok{$}\NormalTok{X,}
  \AttributeTok{Y =}\NormalTok{ grid\_ocean}\SpecialCharTok{$}\NormalTok{Y,}
  \AttributeTok{OCC =}\NormalTok{ grid\_ocean}\SpecialCharTok{$}\NormalTok{OCC,}
  \AttributeTok{DIST =}\NormalTok{ dt}\SpecialCharTok{$}\NormalTok{ldist}
\NormalTok{)}
\CommentTok{\# Выводим первые несколько строк для проверки}
\FunctionTok{str}\NormalTok{(DATA)}
\CommentTok{\# Присоединение к базовой таблице (X, Y, OCC, DIST) переменных с сайта BioORACLE}
\CommentTok{\#}
\NormalTok{TEMP }\OtherTok{\textless{}{-}} \FunctionTok{rast}\NormalTok{(}\StringTok{"C:/SDM/BIO/Temperature [mean].nc"}\NormalTok{)}
\CommentTok{\# визуализация переменной}
\FunctionTok{plot}\NormalTok{(TEMP , }\AttributeTok{xlim =} \FunctionTok{c}\NormalTok{(xmin, xmax), }\AttributeTok{ylim =} \FunctionTok{c}\NormalTok{(ymin, }\AttributeTok{ymax=}\DecValTok{72}\NormalTok{)) }
\CommentTok{\# Автоматическое присоединение переменных из всех nc файлов в папке BIO}
\end{Highlighting}
\end{Shaded}

\begin{figure}[H]

{\centering \includegraphics[width=0.6\linewidth,height=\textheight,keepaspectratio]{images/SDM4.PNG}

}

\caption{Рис. 4.: Визуализация переменной (средней придонной
температуры)}

\end{figure}%

\begin{Shaded}
\begin{Highlighting}[]
\CommentTok{\# Получаем список всех nc файлов в папке BIO}
\NormalTok{nc\_files }\OtherTok{\textless{}{-}} \FunctionTok{list.files}\NormalTok{(}\StringTok{"C:/SDM/BIO"}\NormalTok{, }\AttributeTok{pattern =} \StringTok{"}\SpecialCharTok{\textbackslash{}\textbackslash{}}\StringTok{.nc$"}\NormalTok{, }\AttributeTok{full.names =} \ConstantTok{TRUE}\NormalTok{)}

\CommentTok{\# Проверяем, что файлы найдены}
\ControlFlowTok{if}\NormalTok{ (}\FunctionTok{length}\NormalTok{(nc\_files) }\SpecialCharTok{==} \DecValTok{0}\NormalTok{) \{}
  \FunctionTok{stop}\NormalTok{(}\StringTok{"Не найдено nc файлов в папке C:/SDM/BIO"}\NormalTok{)}
\NormalTok{\}}

\CommentTok{\# Создаем координатный датафрейм один раз}
\NormalTok{coord }\OtherTok{\textless{}{-}} \FunctionTok{data.frame}\NormalTok{(}\AttributeTok{x =}\NormalTok{ DATA}\SpecialCharTok{$}\NormalTok{X, }\AttributeTok{y =}\NormalTok{ DATA}\SpecialCharTok{$}\NormalTok{Y)}

\CommentTok{\# Проходим по всем nc файлам}
\ControlFlowTok{for}\NormalTok{ (file\_path }\ControlFlowTok{in}\NormalTok{ nc\_files) \{}
  \CommentTok{\# Извлекаем имя переменной из названия файла (убираем расширение .nc)}
\NormalTok{  var\_name }\OtherTok{\textless{}{-}}\NormalTok{ tools}\SpecialCharTok{::}\FunctionTok{file\_path\_sans\_ext}\NormalTok{(}\FunctionTok{basename}\NormalTok{(file\_path))}
  
  \CommentTok{\# Загружаем raster файл}
\NormalTok{  temp\_rast }\OtherTok{\textless{}{-}} \FunctionTok{rast}\NormalTok{(file\_path)}
  
  \CommentTok{\# Извлекаем значения для координат DATA}
\NormalTok{  extracted\_values }\OtherTok{\textless{}{-}} \FunctionTok{extract}\NormalTok{(temp\_rast, coord, }\AttributeTok{method =} \StringTok{"bilinear"}\NormalTok{)}
  
  \CommentTok{\# Убираем столбец ID и оставляем только значения}
\NormalTok{  values }\OtherTok{\textless{}{-}}\NormalTok{ extracted\_values[, }\SpecialCharTok{{-}}\DecValTok{1}\NormalTok{, drop }\OtherTok{=} \ConstantTok{FALSE}\NormalTok{]}
  
  \CommentTok{\# Переименовываем столбец в имя переменной}
  \FunctionTok{colnames}\NormalTok{(values) }\OtherTok{\textless{}{-}}\NormalTok{ var\_name}
  
  \CommentTok{\# Присоединяем к DATA}
\NormalTok{  DATA }\OtherTok{\textless{}{-}} \FunctionTok{cbind}\NormalTok{(DATA, values)}
  
  \CommentTok{\# Сообщение о прогрессе}
  \FunctionTok{message}\NormalTok{(}\StringTok{"Добавлена переменная: "}\NormalTok{, var\_name, }\StringTok{" из файла: "}\NormalTok{, }\FunctionTok{basename}\NormalTok{(file\_path))}
\NormalTok{\}}

\CommentTok{\# Проверяем результат}
\FunctionTok{str}\NormalTok{(DATA)}
\FunctionTok{write.csv}\NormalTok{(DATA, }\StringTok{"SDM\_all\_pred\_full\_set.csv"}\NormalTok{, }\AttributeTok{row.names =} \ConstantTok{FALSE}\NormalTok{)}
\end{Highlighting}
\end{Shaded}

\section{Выбор
предикторов}\label{ux432ux44bux431ux43eux440-ux43fux440ux435ux434ux438ux43aux442ux43eux440ux43eux432-1}

Ниже приводится скрипт, а после него - его анализ.

\begin{Shaded}
\begin{Highlighting}[]
\CommentTok{\# ========================================================================================================================}
\CommentTok{\# ПРАКТИЧЕСКОЕ ЗАНЯТИЕ: Модели пространственного распределения видов (SDM)}
\CommentTok{\# Курс: "Оценка водных биоресурсов в среде R (для начинающих)"}
\CommentTok{\# Автор: Баканев С. В. Дата: 27.08.2025}
\CommentTok{\# Структура:}
\CommentTok{\# 1) Подготовка и визуализация данных}
\CommentTok{\# 2) Отбор переменных и анализ важности признаков}
\CommentTok{\# 3) Построение моделей и анализ результатов}
\CommentTok{\# ========================================================================================================================}

\CommentTok{\# Установка рабочей директории}
\FunctionTok{setwd}\NormalTok{(}\StringTok{"C:/SDM"}\NormalTok{)}
\FunctionTok{rm}\NormalTok{(}\AttributeTok{list =} \FunctionTok{ls}\NormalTok{())}

\CommentTok{\# Подключение необходимых библиотек}
\FunctionTok{suppressPackageStartupMessages}\NormalTok{(\{}
\FunctionTok{library}\NormalTok{(tidyverse)    }\CommentTok{\# Обработка данных и визуализация}
\FunctionTok{library}\NormalTok{(janitor)      }\CommentTok{\# Очистка имен переменных}
\FunctionTok{library}\NormalTok{(recipes)      }\CommentTok{\# Предобработка данных}
\FunctionTok{library}\NormalTok{(caret)        }\CommentTok{\# Машинное обучение}
\FunctionTok{library}\NormalTok{(car)          }\CommentTok{\# VIF анализ}
\FunctionTok{library}\NormalTok{(Boruta)       }\CommentTok{\# Отбор признаков}
\FunctionTok{library}\NormalTok{(glmnet)       }\CommentTok{\# LASSO регрессия}
\FunctionTok{library}\NormalTok{(randomForest) }\CommentTok{\# Случайный лес}
\FunctionTok{library}\NormalTok{(mgcv)         }\CommentTok{\# GAM модели}
\FunctionTok{library}\NormalTok{(terra)        }\CommentTok{\# Пространственный анализ}
\FunctionTok{library}\NormalTok{(scales)       }\CommentTok{\# Форматирование графиков}
\NormalTok{\})}
\CommentTok{\# ========================================================================================================================}
\CommentTok{\# 1. ЗАГРУЗКА И ПРЕДВАРИТЕЛЬНАЯ ОБРАБОТКА ДАННЫХ}
\CommentTok{\# ========================================================================================================================}

\CommentTok{\# Загрузка данных}
\NormalTok{DATA }\OtherTok{\textless{}{-}} \FunctionTok{read.csv}\NormalTok{(}\StringTok{"SDM\_all\_pred\_full\_set.csv"}\NormalTok{)}

\CommentTok{\# Предварительная обработка: безопасные имена и удаление пропусков в целевой переменной}
\NormalTok{df0 }\OtherTok{\textless{}{-}}\NormalTok{ DATA }\SpecialCharTok{\%\textgreater{}\%}
\NormalTok{  janitor}\SpecialCharTok{::}\FunctionTok{clean\_names}\NormalTok{() }\SpecialCharTok{\%\textgreater{}\%}
  \FunctionTok{filter}\NormalTok{(}\SpecialCharTok{!}\FunctionTok{is.na}\NormalTok{(occ)) }\SpecialCharTok{\%\textgreater{}\%}
  \FunctionTok{select}\NormalTok{(}\SpecialCharTok{{-}}\NormalTok{x, }\SpecialCharTok{{-}}\NormalTok{y)  }\CommentTok{\# Удаление координат и ненужных переменных}

\CommentTok{\# Удаление переменных с near{-}zero variance}
\NormalTok{preds0 }\OtherTok{\textless{}{-}}\NormalTok{ df0 }\SpecialCharTok{\%\textgreater{}\%} \FunctionTok{select}\NormalTok{(}\SpecialCharTok{{-}}\NormalTok{occ)}
\NormalTok{nzv\_idx }\OtherTok{\textless{}{-}}\NormalTok{ caret}\SpecialCharTok{::}\FunctionTok{nearZeroVar}\NormalTok{(preds0)}
\ControlFlowTok{if}\NormalTok{ (}\FunctionTok{length}\NormalTok{(nzv\_idx) }\SpecialCharTok{\textgreater{}} \DecValTok{0}\NormalTok{) preds0 }\OtherTok{\textless{}{-}}\NormalTok{ preds0[, }\SpecialCharTok{{-}}\NormalTok{nzv\_idx, drop }\OtherTok{=} \ConstantTok{FALSE}\NormalTok{]}
\NormalTok{df1 }\OtherTok{\textless{}{-}} \FunctionTok{bind\_cols}\NormalTok{(}\AttributeTok{occ =}\NormalTok{ df0}\SpecialCharTok{$}\NormalTok{occ, preds0)}

\CommentTok{\# Импутация пропусков и стандартизация данных}
\NormalTok{rec }\OtherTok{\textless{}{-}} \FunctionTok{recipe}\NormalTok{(occ }\SpecialCharTok{\textasciitilde{}}\NormalTok{ ., }\AttributeTok{data =}\NormalTok{ df1) }\SpecialCharTok{\%\textgreater{}\%}
  \FunctionTok{step\_impute\_knn}\NormalTok{(}\FunctionTok{all\_predictors}\NormalTok{()) }\SpecialCharTok{\%\textgreater{}\%}
  \FunctionTok{step\_normalize}\NormalTok{(}\FunctionTok{all\_predictors}\NormalTok{())}

\NormalTok{prep\_rec }\OtherTok{\textless{}{-}} \FunctionTok{prep}\NormalTok{(rec)}
\NormalTok{dat }\OtherTok{\textless{}{-}} \FunctionTok{bake}\NormalTok{(prep\_rec, }\AttributeTok{new\_data =} \ConstantTok{NULL}\NormalTok{)}

\CommentTok{\# ========================================================================================================================}
\CommentTok{\# 2. ОТБОР ПЕРЕМЕННЫХ И АНАЛИЗ МУЛЬТИКОЛЛИНЕАРНОСТИ}
\CommentTok{\# ========================================================================================================================}

\CommentTok{\# Удаление высококоррелированных переменных (коэффициент \textgreater{} 0.8)}
\NormalTok{corr }\OtherTok{\textless{}{-}} \FunctionTok{cor}\NormalTok{(dat }\SpecialCharTok{\%\textgreater{}\%} \FunctionTok{select}\NormalTok{(}\SpecialCharTok{{-}}\NormalTok{occ), }\AttributeTok{use =} \StringTok{"pairwise.complete.obs"}\NormalTok{)}
\NormalTok{highCorr }\OtherTok{\textless{}{-}}\NormalTok{ caret}\SpecialCharTok{::}\FunctionTok{findCorrelation}\NormalTok{(corr, }\AttributeTok{cutoff =} \FloatTok{0.8}\NormalTok{, }\AttributeTok{names =} \ConstantTok{TRUE}\NormalTok{, }\AttributeTok{exact =} \ConstantTok{TRUE}\NormalTok{)}
\NormalTok{dat\_cf }\OtherTok{\textless{}{-}}\NormalTok{ dat }\SpecialCharTok{\%\textgreater{}\%} \FunctionTok{select}\NormalTok{(occ, }\FunctionTok{any\_of}\NormalTok{(}\FunctionTok{setdiff}\NormalTok{(}\FunctionTok{names}\NormalTok{(dat)[}\FunctionTok{names}\NormalTok{(dat) }\SpecialCharTok{!=} \StringTok{"occ"}\NormalTok{], highCorr)))}

\CommentTok{\# Функция для фильтрации по VIF}
\NormalTok{vif\_filter }\OtherTok{\textless{}{-}} \ControlFlowTok{function}\NormalTok{(df, }\AttributeTok{thresh =} \DecValTok{5}\NormalTok{) \{}
\NormalTok{  vars }\OtherTok{\textless{}{-}} \FunctionTok{setdiff}\NormalTok{(}\FunctionTok{names}\NormalTok{(df), }\StringTok{"occ"}\NormalTok{)}
  \ControlFlowTok{repeat}\NormalTok{ \{}
\NormalTok{    fit }\OtherTok{\textless{}{-}} \FunctionTok{lm}\NormalTok{(occ }\SpecialCharTok{\textasciitilde{}}\NormalTok{ ., }\AttributeTok{data =}\NormalTok{ df[, }\FunctionTok{c}\NormalTok{(}\StringTok{"occ"}\NormalTok{, vars)])}
\NormalTok{    v }\OtherTok{\textless{}{-}}\NormalTok{ car}\SpecialCharTok{::}\FunctionTok{vif}\NormalTok{(fit)}
    \ControlFlowTok{if}\NormalTok{ (}\FunctionTok{max}\NormalTok{(v) }\SpecialCharTok{\textless{}}\NormalTok{ thresh) }\ControlFlowTok{break}
\NormalTok{    drop\_var }\OtherTok{\textless{}{-}} \FunctionTok{names}\NormalTok{(}\FunctionTok{which.max}\NormalTok{(v))}
\NormalTok{    vars }\OtherTok{\textless{}{-}} \FunctionTok{setdiff}\NormalTok{(vars, drop\_var)}
    \ControlFlowTok{if}\NormalTok{ (}\FunctionTok{length}\NormalTok{(vars) }\SpecialCharTok{==} \DecValTok{0}\NormalTok{) }\ControlFlowTok{break}
\NormalTok{  \}}
\NormalTok{  df[, }\FunctionTok{c}\NormalTok{(}\StringTok{"occ"}\NormalTok{, vars)]}
\NormalTok{\}}

\NormalTok{dat\_vif }\OtherTok{\textless{}{-}} \FunctionTok{vif\_filter}\NormalTok{(dat\_cf, }\AttributeTok{thresh =} \DecValTok{5}\NormalTok{)}

\CommentTok{\# Отбор признаков с помощью Boruta}
\FunctionTok{set.seed}\NormalTok{(}\DecValTok{42}\NormalTok{)}
\NormalTok{bor }\OtherTok{\textless{}{-}} \FunctionTok{Boruta}\NormalTok{(occ }\SpecialCharTok{\textasciitilde{}}\NormalTok{ ., }\AttributeTok{data =}\NormalTok{ dat\_vif, }\AttributeTok{maxRuns =} \DecValTok{200}\NormalTok{, }\AttributeTok{doTrace =} \DecValTok{0}\NormalTok{)}
\NormalTok{bor\_selected }\OtherTok{\textless{}{-}} \FunctionTok{getSelectedAttributes}\NormalTok{(bor, }\AttributeTok{withTentative =} \ConstantTok{FALSE}\NormalTok{)}

\CommentTok{\# Визуализация результатов Boruta}
\FunctionTok{plot}\NormalTok{(bor, }\AttributeTok{las =} \DecValTok{2}\NormalTok{, }\AttributeTok{cex.axis =} \FloatTok{0.7}\NormalTok{)}
\FunctionTok{print}\NormalTok{(bor)}
\end{Highlighting}
\end{Shaded}

\begin{figure}[H]

{\centering \includegraphics[width=0.6\linewidth,height=\textheight,keepaspectratio]{images/SDM5.PNG}

}

\caption{Рис. 5.: Визуализация результатов Boruta}

\end{figure}%

\begin{Shaded}
\begin{Highlighting}[]
\CommentTok{\# Отбор признаков с помощью LASSO}
\NormalTok{x }\OtherTok{\textless{}{-}} \FunctionTok{as.matrix}\NormalTok{(dat\_vif }\SpecialCharTok{\%\textgreater{}\%} \FunctionTok{select}\NormalTok{(}\SpecialCharTok{{-}}\NormalTok{occ))}
\NormalTok{y }\OtherTok{\textless{}{-}}\NormalTok{ dat\_vif}\SpecialCharTok{$}\NormalTok{occ}
\NormalTok{cv }\OtherTok{\textless{}{-}} \FunctionTok{cv.glmnet}\NormalTok{(x, y, }\AttributeTok{alpha =} \DecValTok{1}\NormalTok{, }\AttributeTok{family =} \StringTok{"binomial"}\NormalTok{, }\AttributeTok{standardize =} \ConstantTok{FALSE}\NormalTok{)}
\NormalTok{coef\_1se }\OtherTok{\textless{}{-}} \FunctionTok{as.matrix}\NormalTok{(}\FunctionTok{coef}\NormalTok{(cv, }\AttributeTok{s =} \StringTok{"lambda.1se"}\NormalTok{))}
\NormalTok{lasso\_selected }\OtherTok{\textless{}{-}} \FunctionTok{rownames}\NormalTok{(coef\_1se)[coef\_1se[, }\DecValTok{1}\NormalTok{] }\SpecialCharTok{!=} \DecValTok{0}\NormalTok{]}
\NormalTok{lasso\_selected }\OtherTok{\textless{}{-}} \FunctionTok{setdiff}\NormalTok{(lasso\_selected, }\StringTok{"(Intercept)"}\NormalTok{)}

\CommentTok{\# Ранжирование переменных по важности в LASSO}
\NormalTok{coef\_all }\OtherTok{\textless{}{-}} \FunctionTok{as.matrix}\NormalTok{(}\FunctionTok{coef}\NormalTok{(cv, }\AttributeTok{s =} \StringTok{"lambda.1se"}\NormalTok{))}
\NormalTok{coef\_tbl }\OtherTok{\textless{}{-}} \FunctionTok{tibble}\NormalTok{(}\AttributeTok{var =} \FunctionTok{rownames}\NormalTok{(coef\_all), }\AttributeTok{coef =} \FunctionTok{as.numeric}\NormalTok{(coef\_all[, }\DecValTok{1}\NormalTok{])) }\SpecialCharTok{\%\textgreater{}\%}
  \FunctionTok{filter}\NormalTok{(var }\SpecialCharTok{!=} \StringTok{"(Intercept)"}\NormalTok{) }\SpecialCharTok{\%\textgreater{}\%}
  \FunctionTok{arrange}\NormalTok{(}\FunctionTok{desc}\NormalTok{(}\FunctionTok{abs}\NormalTok{(coef)))}

\CommentTok{\# Формирование финального набора переменных (5{-}8 наиболее важных)}
\NormalTok{consensus }\OtherTok{\textless{}{-}} \FunctionTok{intersect}\NormalTok{(bor\_selected, lasso\_selected)}
\NormalTok{fill\_from\_lasso }\OtherTok{\textless{}{-}} \FunctionTok{setdiff}\NormalTok{(coef\_tbl}\SpecialCharTok{$}\NormalTok{var, consensus)}
\NormalTok{fill\_from\_boruta }\OtherTok{\textless{}{-}} \FunctionTok{setdiff}\NormalTok{(bor\_selected, }\FunctionTok{c}\NormalTok{(consensus, fill\_from\_lasso))}

\NormalTok{final\_vars }\OtherTok{\textless{}{-}} \FunctionTok{unique}\NormalTok{(}\FunctionTok{c}\NormalTok{(consensus, fill\_from\_lasso, fill\_from\_boruta))}
\NormalTok{final\_vars }\OtherTok{\textless{}{-}} \FunctionTok{head}\NormalTok{(final\_vars, }\DecValTok{8}\NormalTok{)  }\CommentTok{\# Ограничение до 8 переменных}

\ControlFlowTok{if}\NormalTok{ (}\FunctionTok{length}\NormalTok{(final\_vars) }\SpecialCharTok{\textless{}} \DecValTok{5}\NormalTok{) \{}
\NormalTok{  final\_vars }\OtherTok{\textless{}{-}} \FunctionTok{unique}\NormalTok{(}\FunctionTok{c}\NormalTok{(final\_vars, }\FunctionTok{head}\NormalTok{(coef\_tbl}\SpecialCharTok{$}\NormalTok{var, }\DecValTok{5}\NormalTok{)))[}\DecValTok{1}\SpecialCharTok{:}\DecValTok{5}\NormalTok{]}
\NormalTok{\}}

\NormalTok{final\_df }\OtherTok{\textless{}{-}}\NormalTok{ dat\_vif }\SpecialCharTok{\%\textgreater{}\%} \FunctionTok{select}\NormalTok{(occ, }\FunctionTok{all\_of}\NormalTok{(final\_vars))}

\CommentTok{\# ========================================================================================================================}
\CommentTok{\# 3. АНАЛИЗ ВАЖНОСТИ ПЕРЕМЕННЫХ}
\CommentTok{\# ========================================================================================================================}

\CommentTok{\# Важность переменных по LASSO}
\NormalTok{lasso\_imp }\OtherTok{\textless{}{-}}\NormalTok{ coef\_tbl }\SpecialCharTok{\%\textgreater{}\%}
  \FunctionTok{mutate}\NormalTok{(}\AttributeTok{abs\_coef =} \FunctionTok{abs}\NormalTok{(coef)) }\SpecialCharTok{\%\textgreater{}\%}
  \FunctionTok{filter}\NormalTok{(var }\SpecialCharTok{!=} \StringTok{"(Intercept)"}\NormalTok{, abs\_coef }\SpecialCharTok{\textgreater{}} \DecValTok{0}\NormalTok{) }\SpecialCharTok{\%\textgreater{}\%}
  \FunctionTok{arrange}\NormalTok{(}\FunctionTok{desc}\NormalTok{(abs\_coef)) }\SpecialCharTok{\%\textgreater{}\%}
  \FunctionTok{slice\_head}\NormalTok{(}\AttributeTok{n =} \DecValTok{20}\NormalTok{)}

\FunctionTok{ggplot}\NormalTok{(lasso\_imp, }\FunctionTok{aes}\NormalTok{(}\AttributeTok{x =} \FunctionTok{reorder}\NormalTok{(var, abs\_coef), }\AttributeTok{y =}\NormalTok{ abs\_coef)) }\SpecialCharTok{+}
  \FunctionTok{geom\_col}\NormalTok{(}\AttributeTok{fill =} \StringTok{"\#3B82F6"}\NormalTok{) }\SpecialCharTok{+}
  \FunctionTok{coord\_flip}\NormalTok{() }\SpecialCharTok{+}
  \FunctionTok{labs}\NormalTok{(}\AttributeTok{x =} \StringTok{"Переменная"}\NormalTok{, }\AttributeTok{y =} \StringTok{"|коэффициент| (lambda.1se)"}\NormalTok{,}
       \AttributeTok{title =} \StringTok{"LASSO важность (топ{-}20 по |коэффициенту|)"}\NormalTok{) }\SpecialCharTok{+}
  \FunctionTok{theme\_minimal}\NormalTok{(}\AttributeTok{base\_size =} \DecValTok{12}\NormalTok{)}
\end{Highlighting}
\end{Shaded}

\begin{figure}[H]

{\centering \includegraphics[width=0.6\linewidth,height=\textheight,keepaspectratio]{images/SDM6.PNG}

}

\caption{Рис. 6.: Визуализация результатов LASSO}

\end{figure}%

\begin{Shaded}
\begin{Highlighting}[]
\CommentTok{\# Важность переменных по Random Forest}
\NormalTok{rf\_data }\OtherTok{\textless{}{-}}\NormalTok{ dat\_vif }\SpecialCharTok{\%\textgreater{}\%}\NormalTok{ dplyr}\SpecialCharTok{::}\FunctionTok{select}\NormalTok{(occ, dplyr}\SpecialCharTok{::}\FunctionTok{all\_of}\NormalTok{(final\_vars))}
\NormalTok{is\_classif }\OtherTok{\textless{}{-}} \FunctionTok{is.factor}\NormalTok{(rf\_data}\SpecialCharTok{$}\NormalTok{occ) }\SpecialCharTok{||} \FunctionTok{all}\NormalTok{(rf\_data}\SpecialCharTok{$}\NormalTok{occ }\SpecialCharTok{\%in\%} \FunctionTok{c}\NormalTok{(}\DecValTok{0}\NormalTok{, }\DecValTok{1}\NormalTok{), }\AttributeTok{na.rm =} \ConstantTok{TRUE}\NormalTok{)}
\ControlFlowTok{if}\NormalTok{ (is\_classif }\SpecialCharTok{\&\&} \SpecialCharTok{!}\FunctionTok{is.factor}\NormalTok{(rf\_data}\SpecialCharTok{$}\NormalTok{occ)) \{}
\NormalTok{  rf\_data}\SpecialCharTok{$}\NormalTok{occ }\OtherTok{\textless{}{-}} \FunctionTok{factor}\NormalTok{(rf\_data}\SpecialCharTok{$}\NormalTok{occ)}
\NormalTok{\}}

\NormalTok{rf }\OtherTok{\textless{}{-}} \FunctionTok{randomForest}\NormalTok{(occ }\SpecialCharTok{\textasciitilde{}}\NormalTok{ ., }\AttributeTok{data =}\NormalTok{ rf\_data, }\AttributeTok{importance =} \ConstantTok{TRUE}\NormalTok{, }\AttributeTok{na.action =}\NormalTok{ na.omit)}
\NormalTok{imp\_mat }\OtherTok{\textless{}{-}}\NormalTok{ randomForest}\SpecialCharTok{::}\FunctionTok{importance}\NormalTok{(rf)}
\NormalTok{imp\_df }\OtherTok{\textless{}{-}} \FunctionTok{as.data.frame}\NormalTok{(imp\_mat) }\SpecialCharTok{\%\textgreater{}\%}
\NormalTok{  tibble}\SpecialCharTok{::}\FunctionTok{rownames\_to\_column}\NormalTok{(}\StringTok{"var"}\NormalTok{) }\SpecialCharTok{\%\textgreater{}\%}
  \FunctionTok{pivot\_longer}\NormalTok{(}\AttributeTok{cols =} \SpecialCharTok{{-}}\NormalTok{var, }\AttributeTok{names\_to =} \StringTok{"metric"}\NormalTok{, }\AttributeTok{values\_to =} \StringTok{"importance"}\NormalTok{)}

\FunctionTok{ggplot}\NormalTok{(imp\_df, }\FunctionTok{aes}\NormalTok{(}\AttributeTok{x =} \FunctionTok{reorder}\NormalTok{(var, importance), }\AttributeTok{y =}\NormalTok{ importance, }\AttributeTok{fill =}\NormalTok{ metric)) }\SpecialCharTok{+}
  \FunctionTok{geom\_col}\NormalTok{(}\AttributeTok{show.legend =} \ConstantTok{FALSE}\NormalTok{) }\SpecialCharTok{+}
  \FunctionTok{coord\_flip}\NormalTok{() }\SpecialCharTok{+}
  \FunctionTok{facet\_wrap}\NormalTok{(}\SpecialCharTok{\textasciitilde{}}\NormalTok{ metric, }\AttributeTok{scales =} \StringTok{"free\_y"}\NormalTok{) }\SpecialCharTok{+}
  \FunctionTok{scale\_fill\_manual}\NormalTok{(}\AttributeTok{values =} \FunctionTok{c}\NormalTok{(}\StringTok{"\#10B981"}\NormalTok{, }\StringTok{"\#F59E0B"}\NormalTok{, }\StringTok{"\#6366F1"}\NormalTok{, }\StringTok{"\#EF4444"}\NormalTok{)) }\SpecialCharTok{+}
  \FunctionTok{labs}\NormalTok{(}\AttributeTok{x =} \StringTok{"Переменная"}\NormalTok{, }\AttributeTok{y =} \StringTok{"Важность"}\NormalTok{, }\AttributeTok{title =} \StringTok{"Важность признаков по Random Forest"}\NormalTok{) }\SpecialCharTok{+}
  \FunctionTok{theme\_minimal}\NormalTok{(}\AttributeTok{base\_size =} \DecValTok{12}\NormalTok{)}
\end{Highlighting}
\end{Shaded}

\begin{figure}[H]

{\centering \includegraphics[width=0.6\linewidth,height=\textheight,keepaspectratio]{images/SDM7.PNG}

}

\caption{Рис. 7.: Важность признаков по Random Forest}

\end{figure}%

\begin{Shaded}
\begin{Highlighting}[]
\CommentTok{\# ========================================================================================================================}
\CommentTok{\# 4. ПОСТРОЕНИЕ И АНАЛИЗ GAM МОДЕЛЕЙ}
\CommentTok{\# ========================================================================================================================}

\NormalTok{df }\OtherTok{\textless{}{-}}\NormalTok{ final\_df}
\ControlFlowTok{if}\NormalTok{ (}\FunctionTok{is.factor}\NormalTok{(df}\SpecialCharTok{$}\NormalTok{occ)) df}\SpecialCharTok{$}\NormalTok{occ }\OtherTok{\textless{}{-}} \FunctionTok{as.numeric}\NormalTok{(df}\SpecialCharTok{$}\NormalTok{occ) }\SpecialCharTok{{-}} \DecValTok{1}\NormalTok{L}
\ControlFlowTok{if}\NormalTok{ (}\SpecialCharTok{!}\FunctionTok{all}\NormalTok{(df}\SpecialCharTok{$}\NormalTok{occ }\SpecialCharTok{\%in\%} \FunctionTok{c}\NormalTok{(}\DecValTok{0}\NormalTok{, }\DecValTok{1}\NormalTok{))) df}\SpecialCharTok{$}\NormalTok{occ }\OtherTok{\textless{}{-}} \FunctionTok{ifelse}\NormalTok{(df}\SpecialCharTok{$}\NormalTok{occ }\SpecialCharTok{\textgreater{}} \DecValTok{0}\NormalTok{, }\DecValTok{1}\NormalTok{L, }\DecValTok{0}\NormalTok{L)}

\NormalTok{pred\_vars }\OtherTok{\textless{}{-}} \FunctionTok{setdiff}\NormalTok{(}\FunctionTok{names}\NormalTok{(df), }\StringTok{"occ"}\NormalTok{)}
\NormalTok{k\_basis }\OtherTok{\textless{}{-}} \DecValTok{8}

\CommentTok{\# Функция для построения унимодальных GAM моделей}
\NormalTok{fit\_gam\_uni }\OtherTok{\textless{}{-}} \ControlFlowTok{function}\NormalTok{(var\_name) \{}
\NormalTok{  form }\OtherTok{\textless{}{-}} \FunctionTok{as.formula}\NormalTok{(}\FunctionTok{paste0}\NormalTok{(}\StringTok{"occ \textasciitilde{} s("}\NormalTok{, var\_name, }\StringTok{", k="}\NormalTok{, k\_basis, }\StringTok{")"}\NormalTok{))}
\NormalTok{  fams }\OtherTok{\textless{}{-}} \FunctionTok{list}\NormalTok{(}
    \FunctionTok{binomial}\NormalTok{(}\AttributeTok{link =} \StringTok{"identity"}\NormalTok{),}
    \FunctionTok{binomial}\NormalTok{(}\AttributeTok{link =} \StringTok{"probit"}\NormalTok{),}
    \FunctionTok{binomial}\NormalTok{(}\AttributeTok{link =} \StringTok{"logit"}\NormalTok{)}
\NormalTok{  )}
\NormalTok{  model }\OtherTok{\textless{}{-}} \ConstantTok{NULL}
\NormalTok{  fam\_used }\OtherTok{\textless{}{-}} \ConstantTok{NULL}
  \ControlFlowTok{for}\NormalTok{ (f }\ControlFlowTok{in}\NormalTok{ fams) \{}
\NormalTok{    model\_try }\OtherTok{\textless{}{-}} \FunctionTok{try}\NormalTok{(}
      \FunctionTok{gam}\NormalTok{(form, }\AttributeTok{data =}\NormalTok{ df, }\AttributeTok{family =}\NormalTok{ f, }\AttributeTok{method =} \StringTok{"REML"}\NormalTok{, }\AttributeTok{select =} \ConstantTok{TRUE}\NormalTok{),}
      \AttributeTok{silent =} \ConstantTok{TRUE}
\NormalTok{    )}
    \ControlFlowTok{if}\NormalTok{ (}\SpecialCharTok{!}\FunctionTok{inherits}\NormalTok{(model\_try, }\StringTok{"try{-}error"}\NormalTok{)) \{}
\NormalTok{      model }\OtherTok{\textless{}{-}}\NormalTok{ model\_try}
\NormalTok{      fam\_used }\OtherTok{\textless{}{-}}\NormalTok{ model}\SpecialCharTok{$}\NormalTok{family}\SpecialCharTok{$}\NormalTok{link}
      \ControlFlowTok{break}
\NormalTok{    \}}
\NormalTok{  \}}
  \ControlFlowTok{if}\NormalTok{ (}\FunctionTok{is.null}\NormalTok{(model)) }\FunctionTok{stop}\NormalTok{(}\FunctionTok{paste}\NormalTok{(}\StringTok{"Не удалось обучить GAM для"}\NormalTok{, var\_name))}

\NormalTok{  x }\OtherTok{\textless{}{-}}\NormalTok{ df[[var\_name]]}
\NormalTok{  x\_seq }\OtherTok{\textless{}{-}} \FunctionTok{seq}\NormalTok{(}\FunctionTok{quantile}\NormalTok{(x, }\FloatTok{0.02}\NormalTok{, }\AttributeTok{na.rm =} \ConstantTok{TRUE}\NormalTok{),}
               \FunctionTok{quantile}\NormalTok{(x, }\FloatTok{0.98}\NormalTok{, }\AttributeTok{na.rm =} \ConstantTok{TRUE}\NormalTok{),}
               \AttributeTok{length.out =} \DecValTok{200}\NormalTok{)}

\NormalTok{  newd }\OtherTok{\textless{}{-}} \FunctionTok{tibble}\NormalTok{(}\SpecialCharTok{!!}\AttributeTok{var\_name :=}\NormalTok{ x\_seq)}
\NormalTok{  pr }\OtherTok{\textless{}{-}} \FunctionTok{predict}\NormalTok{(model, }\AttributeTok{newdata =}\NormalTok{ newd, }\AttributeTok{type =} \StringTok{"link"}\NormalTok{, }\AttributeTok{se.fit =} \ConstantTok{TRUE}\NormalTok{)}

\NormalTok{  inv }\OtherTok{\textless{}{-}}\NormalTok{ model}\SpecialCharTok{$}\NormalTok{family}\SpecialCharTok{$}\NormalTok{linkinv}
  \FunctionTok{tibble}\NormalTok{(}
    \AttributeTok{variable =}\NormalTok{ var\_name,}
    \AttributeTok{x =}\NormalTok{ x\_seq,}
    \AttributeTok{prob  =} \FunctionTok{pmax}\NormalTok{(}\FunctionTok{pmin}\NormalTok{(}\FunctionTok{inv}\NormalTok{(pr}\SpecialCharTok{$}\NormalTok{fit), }\DecValTok{1}\NormalTok{), }\DecValTok{0}\NormalTok{),}
    \AttributeTok{lower =} \FunctionTok{pmax}\NormalTok{(}\FunctionTok{pmin}\NormalTok{(}\FunctionTok{inv}\NormalTok{(pr}\SpecialCharTok{$}\NormalTok{fit }\SpecialCharTok{{-}} \FloatTok{1.96} \SpecialCharTok{*}\NormalTok{ pr}\SpecialCharTok{$}\NormalTok{se.fit), }\DecValTok{1}\NormalTok{), }\DecValTok{0}\NormalTok{),}
    \AttributeTok{upper =} \FunctionTok{pmax}\NormalTok{(}\FunctionTok{pmin}\NormalTok{(}\FunctionTok{inv}\NormalTok{(pr}\SpecialCharTok{$}\NormalTok{fit }\SpecialCharTok{+} \FloatTok{1.96} \SpecialCharTok{*}\NormalTok{ pr}\SpecialCharTok{$}\NormalTok{se.fit), }\DecValTok{1}\NormalTok{), }\DecValTok{0}\NormalTok{),}
    \AttributeTok{link  =}\NormalTok{ fam\_used}
\NormalTok{  ) }\SpecialCharTok{\%\textgreater{}\%}
    \FunctionTok{mutate}\NormalTok{(}\AttributeTok{model\_link =} \FunctionTok{paste0}\NormalTok{(}\StringTok{"binomial("}\NormalTok{, link, }\StringTok{")"}\NormalTok{))}
\NormalTok{\}}

\CommentTok{\# Обучение GAM моделей и сбор кривых влияния}
\NormalTok{pd\_all }\OtherTok{\textless{}{-}} \FunctionTok{suppressWarnings}\NormalTok{(}\FunctionTok{map\_dfr}\NormalTok{(pred\_vars, fit\_gam\_uni))}

\CommentTok{\# Подготовка данных для визуализации}
\NormalTok{rug\_data }\OtherTok{\textless{}{-}}\NormalTok{ df }\SpecialCharTok{\%\textgreater{}\%}
  \FunctionTok{filter}\NormalTok{(occ }\SpecialCharTok{==} \DecValTok{1}\NormalTok{) }\SpecialCharTok{\%\textgreater{}\%}
  \FunctionTok{pivot\_longer}\NormalTok{(}\AttributeTok{cols =} \FunctionTok{all\_of}\NormalTok{(pred\_vars), }\AttributeTok{names\_to =} \StringTok{"variable"}\NormalTok{, }\AttributeTok{values\_to =} \StringTok{"x"}\NormalTok{)}

\CommentTok{\# Первый график: базовые кривые GAM с риджинами}
\FunctionTok{ggplot}\NormalTok{(pd\_all, }\FunctionTok{aes}\NormalTok{(}\AttributeTok{x =}\NormalTok{ x, }\AttributeTok{y =}\NormalTok{ prob)) }\SpecialCharTok{+}
  \FunctionTok{geom\_ribbon}\NormalTok{(}\FunctionTok{aes}\NormalTok{(}\AttributeTok{ymin =}\NormalTok{ lower, }\AttributeTok{ymax =}\NormalTok{ upper), }\AttributeTok{fill =} \StringTok{"\#93C5FD"}\NormalTok{, }\AttributeTok{alpha =} \FloatTok{0.35}\NormalTok{) }\SpecialCharTok{+}
  \FunctionTok{geom\_line}\NormalTok{(}\AttributeTok{color =} \StringTok{"\#1D4ED8"}\NormalTok{, }\AttributeTok{linewidth =} \DecValTok{1}\NormalTok{) }\SpecialCharTok{+}
  \FunctionTok{geom\_rug}\NormalTok{(}\AttributeTok{data =}\NormalTok{ rug\_data, }\FunctionTok{aes}\NormalTok{(}\AttributeTok{x =}\NormalTok{ x), }\AttributeTok{sides =} \StringTok{"b"}\NormalTok{, }\AttributeTok{alpha =} \FloatTok{0.25}\NormalTok{, }\AttributeTok{inherit.aes =} \ConstantTok{FALSE}\NormalTok{) }\SpecialCharTok{+}
  \FunctionTok{facet\_wrap}\NormalTok{(}\SpecialCharTok{\textasciitilde{}}\NormalTok{ variable, }\AttributeTok{scales =} \StringTok{"free\_x"}\NormalTok{) }\SpecialCharTok{+}
  \FunctionTok{scale\_y\_continuous}\NormalTok{(}\AttributeTok{limits =} \FunctionTok{c}\NormalTok{(}\DecValTok{0}\NormalTok{, }\DecValTok{1}\NormalTok{)) }\SpecialCharTok{+}
  \FunctionTok{labs}\NormalTok{(}\AttributeTok{x =} \StringTok{"Значение предиктора (стандартизовано)"}\NormalTok{,}
       \AttributeTok{y =} \StringTok{"Вероятность присутствия"}\NormalTok{,}
       \AttributeTok{title =} \StringTok{"Унимодельные GAM (binomial): влияние каждого предиктора"}\NormalTok{,}
       \AttributeTok{subtitle =} \StringTok{"Ссылка: identity (fallback \textgreater{} probit \textgreater{} logit); ленты — 95\% ДИ"}\NormalTok{) }\SpecialCharTok{+}
  \FunctionTok{theme\_minimal}\NormalTok{(}\AttributeTok{base\_size =} \DecValTok{12}\NormalTok{)}
\end{Highlighting}
\end{Shaded}

\begin{figure}[H]

{\centering \includegraphics[width=0.8\linewidth,height=\textheight,keepaspectratio]{images/SDM8.PNG}

}

\caption{Рис. 8.: Унимодельные GAM: влияние каждого предиктора}

\end{figure}%

\begin{Shaded}
\begin{Highlighting}[]
\CommentTok{\# Второй график: с джиттером и биновой эмпирической вероятностью}
\NormalTok{obs\_raw }\OtherTok{\textless{}{-}}\NormalTok{ df }\SpecialCharTok{\%\textgreater{}\%}
  \FunctionTok{pivot\_longer}\NormalTok{(}\AttributeTok{cols =} \FunctionTok{all\_of}\NormalTok{(pred\_vars), }\AttributeTok{names\_to =} \StringTok{"variable"}\NormalTok{, }\AttributeTok{values\_to =} \StringTok{"x"}\NormalTok{) }\SpecialCharTok{\%\textgreater{}\%}
  \FunctionTok{select}\NormalTok{(variable, x, occ)}

\NormalTok{obs\_bin }\OtherTok{\textless{}{-}}\NormalTok{ obs\_raw }\SpecialCharTok{\%\textgreater{}\%}
  \FunctionTok{group\_by}\NormalTok{(variable) }\SpecialCharTok{\%\textgreater{}\%}
  \FunctionTok{mutate}\NormalTok{(}\AttributeTok{bin =} \FunctionTok{cut\_number}\NormalTok{(x, }\DecValTok{20}\NormalTok{)) }\SpecialCharTok{\%\textgreater{}\%}
  \FunctionTok{group\_by}\NormalTok{(variable, bin) }\SpecialCharTok{\%\textgreater{}\%}
  \FunctionTok{summarise}\NormalTok{(}
    \AttributeTok{x\_mid =} \FunctionTok{mean}\NormalTok{(x, }\AttributeTok{na.rm =} \ConstantTok{TRUE}\NormalTok{),}
    \AttributeTok{occ\_mean =} \FunctionTok{mean}\NormalTok{(occ),}
    \AttributeTok{n =} \FunctionTok{n}\NormalTok{(), }\AttributeTok{.groups =} \StringTok{"drop"}
\NormalTok{  )}

\FunctionTok{ggplot}\NormalTok{(pd\_all, }\FunctionTok{aes}\NormalTok{(}\AttributeTok{x =}\NormalTok{ x, }\AttributeTok{y =}\NormalTok{ prob)) }\SpecialCharTok{+}
  \FunctionTok{geom\_ribbon}\NormalTok{(}\FunctionTok{aes}\NormalTok{(}\AttributeTok{ymin =}\NormalTok{ lower, }\AttributeTok{ymax =}\NormalTok{ upper), }\AttributeTok{fill =} \StringTok{"\#93C5FD"}\NormalTok{, }\AttributeTok{alpha =} \FloatTok{0.35}\NormalTok{) }\SpecialCharTok{+}
  \FunctionTok{geom\_line}\NormalTok{(}\AttributeTok{color =} \StringTok{"\#1D4ED8"}\NormalTok{, }\AttributeTok{linewidth =} \DecValTok{1}\NormalTok{) }\SpecialCharTok{+}
  \FunctionTok{geom\_jitter}\NormalTok{(}
    \AttributeTok{data =}\NormalTok{ obs\_raw, }\AttributeTok{inherit.aes =} \ConstantTok{FALSE}\NormalTok{,}
    \FunctionTok{aes}\NormalTok{(}\AttributeTok{x =}\NormalTok{ x, }\AttributeTok{y =}\NormalTok{ occ),}
    \AttributeTok{width =} \DecValTok{0}\NormalTok{, }\AttributeTok{height =} \FloatTok{0.04}\NormalTok{, }\AttributeTok{alpha =} \FloatTok{0.25}\NormalTok{, }\AttributeTok{size =} \FloatTok{0.9}\NormalTok{, }\AttributeTok{color =} \StringTok{"black"}
\NormalTok{  ) }\SpecialCharTok{+}
  \FunctionTok{geom\_point}\NormalTok{(}
    \AttributeTok{data =}\NormalTok{ obs\_bin, }\AttributeTok{inherit.aes =} \ConstantTok{FALSE}\NormalTok{,}
    \FunctionTok{aes}\NormalTok{(}\AttributeTok{x =}\NormalTok{ x\_mid, }\AttributeTok{y =}\NormalTok{ occ\_mean),}
    \AttributeTok{color =} \StringTok{"\#111827"}\NormalTok{, }\AttributeTok{size =} \FloatTok{1.6}
\NormalTok{  ) }\SpecialCharTok{+}
  \FunctionTok{facet\_wrap}\NormalTok{(}\SpecialCharTok{\textasciitilde{}}\NormalTok{ variable, }\AttributeTok{scales =} \StringTok{"free\_x"}\NormalTok{) }\SpecialCharTok{+}
  \FunctionTok{scale\_y\_continuous}\NormalTok{(}\AttributeTok{limits =} \FunctionTok{c}\NormalTok{(}\DecValTok{0}\NormalTok{, }\DecValTok{1}\NormalTok{), }\AttributeTok{labels =} \FunctionTok{percent\_format}\NormalTok{(}\AttributeTok{accuracy =} \DecValTok{1}\NormalTok{)) }\SpecialCharTok{+}
  \FunctionTok{labs}\NormalTok{(}
    \AttributeTok{x =} \StringTok{"Значение предиктора (стандартизовано)"}\NormalTok{,}
    \AttributeTok{y =} \StringTok{"Вероятность присутствия"}\NormalTok{,}
    \AttributeTok{title =} \StringTok{"Унимодельные GAM (binomial): влияние предикторов"}\NormalTok{,}
    \AttributeTok{subtitle =} \StringTok{"Серые точки — фактические (джиттер); Черные точки — биновая средняя встречаемость"}
\NormalTok{  ) }\SpecialCharTok{+}
  \FunctionTok{theme\_minimal}\NormalTok{(}\AttributeTok{base\_size =} \DecValTok{12}\NormalTok{)}
\end{Highlighting}
\end{Shaded}

\begin{figure}[H]

{\centering \includegraphics[width=0.8\linewidth,height=\textheight,keepaspectratio]{images/SDM9.PNG}

}

\caption{Рис. 9.: Унимодельные GAM: влияние каждого предиктора}

\end{figure}%

\begin{Shaded}
\begin{Highlighting}[]
\CommentTok{\# ========================================================================================================================}
\CommentTok{\# 5. ПОСТРОЕНИЕ ТАБЛИЦЫ НА ОСНОВЕ ГРАФИКОВ}
\CommentTok{\# ========================================================================================================================}

\CommentTok{\# Биннинг (20 квантильных бинов) и эмпирическая вероятность}
\NormalTok{obs\_raw }\OtherTok{\textless{}{-}}\NormalTok{ df }\SpecialCharTok{\%\textgreater{}\%}
  \FunctionTok{pivot\_longer}\NormalTok{(}\AttributeTok{cols =} \FunctionTok{all\_of}\NormalTok{(pred\_vars), }\AttributeTok{names\_to =} \StringTok{"variable"}\NormalTok{, }\AttributeTok{values\_to =} \StringTok{"x"}\NormalTok{) }\SpecialCharTok{\%\textgreater{}\%}
  \FunctionTok{select}\NormalTok{(variable, x, occ)}

\NormalTok{obs\_bin }\OtherTok{\textless{}{-}}\NormalTok{ obs\_raw }\SpecialCharTok{\%\textgreater{}\%}
  \FunctionTok{group\_by}\NormalTok{(variable) }\SpecialCharTok{\%\textgreater{}\%}
  \FunctionTok{mutate}\NormalTok{(}\AttributeTok{bin =} \FunctionTok{cut\_number}\NormalTok{(x, }\DecValTok{20}\NormalTok{)) }\SpecialCharTok{\%\textgreater{}\%}
  \FunctionTok{group\_by}\NormalTok{(variable, bin, }\AttributeTok{.add =} \ConstantTok{FALSE}\NormalTok{) }\SpecialCharTok{\%\textgreater{}\%}
  \FunctionTok{summarise}\NormalTok{(}
    \AttributeTok{bin\_id =} \FunctionTok{cur\_group\_id}\NormalTok{(),}
    \AttributeTok{x\_min =} \FunctionTok{min}\NormalTok{(x, }\AttributeTok{na.rm =} \ConstantTok{TRUE}\NormalTok{),}
    \AttributeTok{x\_max =} \FunctionTok{max}\NormalTok{(x, }\AttributeTok{na.rm =} \ConstantTok{TRUE}\NormalTok{),}
    \AttributeTok{x\_mid =} \FunctionTok{median}\NormalTok{(x, }\AttributeTok{na.rm =} \ConstantTok{TRUE}\NormalTok{),}
    \AttributeTok{n =} \FunctionTok{n}\NormalTok{(),}
    \AttributeTok{occ\_mean =} \FunctionTok{mean}\NormalTok{(occ, }\AttributeTok{na.rm =} \ConstantTok{TRUE}\NormalTok{),}
    \AttributeTok{.groups =} \StringTok{"drop"}
\NormalTok{  ) }\SpecialCharTok{\%\textgreater{}\%}
  \FunctionTok{arrange}\NormalTok{(variable, bin\_id)}

\CommentTok{\# Обучение GAM моделей для таблицы}
\NormalTok{fit\_gam\_uni\_model }\OtherTok{\textless{}{-}} \ControlFlowTok{function}\NormalTok{(var\_name) \{}
\NormalTok{  form }\OtherTok{\textless{}{-}} \FunctionTok{as.formula}\NormalTok{(}\FunctionTok{paste0}\NormalTok{(}\StringTok{"occ \textasciitilde{} s("}\NormalTok{, var\_name, }\StringTok{", k="}\NormalTok{, k\_basis, }\StringTok{")"}\NormalTok{))}
\NormalTok{  fams }\OtherTok{\textless{}{-}} \FunctionTok{list}\NormalTok{(}
    \FunctionTok{binomial}\NormalTok{(}\AttributeTok{link =} \StringTok{"identity"}\NormalTok{),}
    \FunctionTok{binomial}\NormalTok{(}\AttributeTok{link =} \StringTok{"probit"}\NormalTok{),}
    \FunctionTok{binomial}\NormalTok{(}\AttributeTok{link =} \StringTok{"logit"}\NormalTok{)}
\NormalTok{  )}
  \ControlFlowTok{for}\NormalTok{ (f }\ControlFlowTok{in}\NormalTok{ fams) \{}
\NormalTok{    m\_try }\OtherTok{\textless{}{-}} \FunctionTok{try}\NormalTok{(}\FunctionTok{gam}\NormalTok{(form, }\AttributeTok{data =}\NormalTok{ df, }\AttributeTok{family =}\NormalTok{ f, }\AttributeTok{method =} \StringTok{"REML"}\NormalTok{, }\AttributeTok{select =} \ConstantTok{TRUE}\NormalTok{), }\AttributeTok{silent =} \ConstantTok{TRUE}\NormalTok{)}
    \ControlFlowTok{if}\NormalTok{ (}\SpecialCharTok{!}\FunctionTok{inherits}\NormalTok{(m\_try, }\StringTok{"try{-}error"}\NormalTok{)) }\FunctionTok{return}\NormalTok{(m\_try)}
\NormalTok{  \}}
  \FunctionTok{stop}\NormalTok{(}\FunctionTok{paste}\NormalTok{(}\StringTok{"Не удалось обучить GAM для"}\NormalTok{, var\_name))}
\NormalTok{\}}

\NormalTok{models }\OtherTok{\textless{}{-}} \FunctionTok{suppressWarnings}\NormalTok{(}\FunctionTok{set\_names}\NormalTok{(}\FunctionTok{map}\NormalTok{(pred\_vars, fit\_gam\_uni\_model), pred\_vars))}

\CommentTok{\# Предсказания GAM в центрах бинов}
\NormalTok{table\_20bins }\OtherTok{\textless{}{-}}\NormalTok{ obs\_bin }\SpecialCharTok{\%\textgreater{}\%}
  \FunctionTok{group\_by}\NormalTok{(variable) }\SpecialCharTok{\%\textgreater{}\%}
  \FunctionTok{group\_modify}\NormalTok{(}\ControlFlowTok{function}\NormalTok{(.x, .y) \{}
\NormalTok{    var }\OtherTok{\textless{}{-}}\NormalTok{ .y}\SpecialCharTok{$}\NormalTok{variable[[}\DecValTok{1}\NormalTok{]]}
\NormalTok{    mdl }\OtherTok{\textless{}{-}}\NormalTok{ models[[var]]}
\NormalTok{    newd }\OtherTok{\textless{}{-}} \FunctionTok{tibble}\NormalTok{(}\SpecialCharTok{!!}\NormalTok{rlang}\SpecialCharTok{::}\FunctionTok{sym}\NormalTok{(var) }\SpecialCharTok{:}\ErrorTok{=}\NormalTok{ .x}\SpecialCharTok{$}\NormalTok{x\_mid)}
\NormalTok{    pr }\OtherTok{\textless{}{-}} \FunctionTok{predict}\NormalTok{(mdl, }\AttributeTok{newdata =}\NormalTok{ newd, }\AttributeTok{type =} \StringTok{"link"}\NormalTok{, }\AttributeTok{se.fit =} \ConstantTok{TRUE}\NormalTok{)}
\NormalTok{    inv }\OtherTok{\textless{}{-}}\NormalTok{ mdl}\SpecialCharTok{$}\NormalTok{family}\SpecialCharTok{$}\NormalTok{linkinv}
    \FunctionTok{mutate}\NormalTok{(}
\NormalTok{      .x,}
      \AttributeTok{gam\_prob  =} \FunctionTok{pmin}\NormalTok{(}\FunctionTok{pmax}\NormalTok{(}\FunctionTok{inv}\NormalTok{(pr}\SpecialCharTok{$}\NormalTok{fit), }\DecValTok{0}\NormalTok{), }\DecValTok{1}\NormalTok{),}
      \AttributeTok{gam\_lower =} \FunctionTok{pmin}\NormalTok{(}\FunctionTok{pmax}\NormalTok{(}\FunctionTok{inv}\NormalTok{(pr}\SpecialCharTok{$}\NormalTok{fit }\SpecialCharTok{{-}} \FloatTok{1.96} \SpecialCharTok{*}\NormalTok{ pr}\SpecialCharTok{$}\NormalTok{se.fit), }\DecValTok{0}\NormalTok{), }\DecValTok{1}\NormalTok{),}
      \AttributeTok{gam\_upper =} \FunctionTok{pmin}\NormalTok{(}\FunctionTok{pmax}\NormalTok{(}\FunctionTok{inv}\NormalTok{(pr}\SpecialCharTok{$}\NormalTok{fit }\SpecialCharTok{+} \FloatTok{1.96} \SpecialCharTok{*}\NormalTok{ pr}\SpecialCharTok{$}\NormalTok{se.fit), }\DecValTok{0}\NormalTok{), }\DecValTok{1}\NormalTok{),}
      \AttributeTok{link      =}\NormalTok{ mdl}\SpecialCharTok{$}\NormalTok{family}\SpecialCharTok{$}\NormalTok{link}
\NormalTok{    )}
\NormalTok{  \}) }\SpecialCharTok{\%\textgreater{}\%}
  \FunctionTok{ungroup}\NormalTok{() }\SpecialCharTok{\%\textgreater{}\%}
  \FunctionTok{select}\NormalTok{(variable, bin\_id, x\_min, x\_max, x\_mid, n, occ\_mean, gam\_prob, gam\_lower, gam\_upper, link)}

\CommentTok{\# Вывод таблицы}
\FunctionTok{print}\NormalTok{(table\_20bins, }\AttributeTok{n =} \DecValTok{5}\NormalTok{)}

\CommentTok{\# ========================================================================================================================}
\CommentTok{\# 6. ФОРМИРОВАНИЕ ФИНАЛЬНОЙ ТАБЛИЦЫ ДАННЫХ}
\CommentTok{\# ========================================================================================================================}

\CommentTok{\# Создание финальной таблицы с исходными данными}
\NormalTok{final\_table }\OtherTok{\textless{}{-}}\NormalTok{ DATA }\SpecialCharTok{\%\textgreater{}\%}
\NormalTok{  janitor}\SpecialCharTok{::}\FunctionTok{clean\_names}\NormalTok{() }\SpecialCharTok{\%\textgreater{}\%}
  \FunctionTok{select}\NormalTok{(x, y, occ, }\FunctionTok{all\_of}\NormalTok{(final\_vars))}

\CommentTok{\# Сохранение результатов}
\FunctionTok{write.csv}\NormalTok{(final\_table, }\StringTok{"final\_sdm\_table\_with\_na.csv"}\NormalTok{, }\AttributeTok{row.names =} \ConstantTok{FALSE}\NormalTok{)}

\CommentTok{\# Вывод структуры финальной таблицы}
\FunctionTok{str}\NormalTok{(final\_table)}
\end{Highlighting}
\end{Shaded}

\begin{Shaded}
\begin{Highlighting}[]
\StringTok{\textquotesingle{}data.frame\textquotesingle{}}\SpecialCharTok{:}   \DecValTok{44929}\NormalTok{ obs. of  }\DecValTok{11}\NormalTok{ variables}\SpecialCharTok{:}
 \ErrorTok{$}\NormalTok{ x                        }\SpecialCharTok{:}\NormalTok{ num  }\DecValTok{10} \FloatTok{10.1} \FloatTok{10.1} \FloatTok{10.2} \FloatTok{10.2}\NormalTok{ ...}
 \SpecialCharTok{$}\NormalTok{ y                        }\SpecialCharTok{:}\NormalTok{ num  }\DecValTok{66} \DecValTok{66} \DecValTok{66} \DecValTok{66} \DecValTok{66}\NormalTok{ ...}
 \SpecialCharTok{$}\NormalTok{ occ                      }\SpecialCharTok{:}\NormalTok{ int  }\ConstantTok{NA} \ConstantTok{NA} \ConstantTok{NA} \ConstantTok{NA} \ConstantTok{NA} \ConstantTok{NA} \ConstantTok{NA} \ConstantTok{NA} \ConstantTok{NA} \ConstantTok{NA}\NormalTok{ ...}
 \SpecialCharTok{$}\NormalTok{ dist                     }\SpecialCharTok{:}\NormalTok{ num  }\FloatTok{89.6} \FloatTok{87.5} \FloatTok{85.5} \FloatTok{83.5} \FloatTok{81.4}\NormalTok{ ...}
 \SpecialCharTok{$}\NormalTok{ chlorophyll\_range        }\SpecialCharTok{:}\NormalTok{ num  }\FloatTok{2.14} \FloatTok{2.16} \FloatTok{2.16} \FloatTok{2.13} \FloatTok{2.1}\NormalTok{ ...}
 \SpecialCharTok{$}\NormalTok{ current\_velocity\_range   }\SpecialCharTok{:}\NormalTok{ num  }\FloatTok{0.0301} \FloatTok{0.0204} \FloatTok{0.0184} \FloatTok{0.0123} \FloatTok{0.0133}\NormalTok{ ...}
 \SpecialCharTok{$}\NormalTok{ diffuse\_attenuation\_range}\SpecialCharTok{:}\NormalTok{ num  }\FloatTok{0.1} \FloatTok{0.101} \FloatTok{0.101} \FloatTok{0.102} \FloatTok{0.107}\NormalTok{ ...}
 \SpecialCharTok{$}\NormalTok{ phosphate\_range          }\SpecialCharTok{:}\NormalTok{ num  }\FloatTok{0.189} \FloatTok{0.186} \FloatTok{0.182} \FloatTok{0.178} \FloatTok{0.176}\NormalTok{ ...}
 \SpecialCharTok{$}\NormalTok{ silicate\_range           }\SpecialCharTok{:}\NormalTok{ num  }\FloatTok{4.25} \FloatTok{4.03} \FloatTok{3.77} \FloatTok{3.63} \FloatTok{3.58}\NormalTok{ ...}
 \SpecialCharTok{$}\NormalTok{ slope                    }\SpecialCharTok{:}\NormalTok{ num  }\FloatTok{0.0765} \FloatTok{0.1032} \FloatTok{0.1524} \FloatTok{0.2003} \FloatTok{0.2112}\NormalTok{ ...}
 \SpecialCharTok{$}\NormalTok{ temperature\_mean         }\SpecialCharTok{:}\NormalTok{ num  }\FloatTok{7.37} \FloatTok{7.39} \FloatTok{7.41} \FloatTok{7.43} \FloatTok{7.43}\NormalTok{ ...}
\SpecialCharTok{\textgreater{}} 
\end{Highlighting}
\end{Shaded}

Начнем с главного парадокса моделирования. Наш мозг жаждет простых
причинно-следственных цепочек: «глубже --- значит холоднее --- значит
вид там». Но реальная система --- это сеть нелинейных, запутанных
взаимодействий, где тот же фактор на разных уровнях может давать
противоположные эффекты. Мы имеем дело с высокомерным предсказанием:
пытаемся экстраполировать сложную экологическую реальность из
ограниченной, зашумленной выборки. Задача второго скрипта --- не
«доказать» связь, а аккуратно, с помощью статистического инструментария,
выявить устойчивые сигналы и честно оценить их силу и неопределенность.

Что было сделано: от сырых данных к структурированному пространству
признаков Предобработка и «чистка» данных. Исходный датасет содержал 44
929 наблюдений (точек в сетке) и 46 переменных. Первым делом мы:

Удалили пропуски в целевой переменной (occ --- наличие вида), оставив
550 точек (35 присутствий, 515 отсутствий).

Привели имена переменных к удобному формату, удалили координаты и
заведомо малополезные переменные.

Исключили предикторы с near-zero variance, которые не несут
информационной нагрузки.

Провели импутацию пропущенных значений в предикторах методом k-NN и
стандартизацию данных (центрирование и scaling). Это важно для сравнения
вклада переменных, измеренных в разных единицах.

Битва с мультиколлинеарностью --- устранение «зеркальных» переменных.
Многие экологические предикторы тесно коррелируют друг с другом
(например, разные производные от температуры). Если оставить их все в
модели, они будут «делить» объясняющую силу, делая оценки ненадежными.
Мы применили два последовательных фильтра:

Корреляционный анализ: Удалили переменные с коэффициентом корреляции
\textgreater{} 0.8.

Анализ VIF (Variance Inflation Factor): Итеративно исключали переменные
с VIF \textgreater{} 5, пока мультиколлинеарность не была устранена. Это
оставило нас с набором из 20 переменных.

Отбор признаков: Boruta vs.~LASSO --- «согласие двух свидетелей». Чтобы
выбрать наиболее прогностические переменные, мы использовали два
принципиально разных метода, подходя к задаче с разных сторон:

Boruta (Random Forest based - обёртка для случайного леса): Алгоритм,
который создает «теневые» переменные и сравнивает важность реальных
переменных с этим случайным шумом. Он подтвердил важность 16 атрибутов.

LASSO-регрессия: Метод, который «штрафует» модель за сложность,
автоматически обнуляя веса наименее важных переменных. Он отобрал свой
набор значимых предикторов.

Консенсус: Переменные, признанные важными обоими методами, были включены
в финальную модель. Это наш «золотой набор» из 9 предикторов: dist
(расстояние до берега), chlorophyll\_range, current\_velocity\_range,
diffuse\_attenuation\_mean, diffuse\_attenuation\_range,
phosphate\_range, silicate\_range, slope (уклон дна),temperature\_mean.

Анализ важности переменных: Чей голос громче? По версии LASSO:
Наибольший абсолютный вес (и, следовательно, влияние на вероятность
присутствия) у переменной temperature\_mean. Это краеугольный камень
модели.

По версии Random Forest: Картина важности более сбалансирована. Метрики
MeanDecreaseAccuracy и MeanDecreaseGini также подтверждают ведущую роль
temperature\_mean, но выделяют и вклад других переменных, таких как dist
(дистанция до берега) и diffuse\_attenuation\_mean.

Это типичная ситуация: линейный метод (LASSO) выделяет одного «лидера»,
в то время как ансамблевый метод (RF) показывает, что прогноз --- это
результат коллективного решения комитета переменных.

Построение GAM: Где линейность ломается Generalized Additive Models
(GAM) были выбраны для того, чтобы уловить нелинейные, плавные
зависимости, которые линейные модели опишут грубо и с ошибкой.

Для каждой из 9 отобранных переменных был построен унимодальный GAM
(только с одним предиктором) с биномиальным распределением.

Диагностика: Кривые зависимости вероятности присутствия от значения
предиктора были визуализированы вместе с:

Джиттером наблюдений: Фактическими точками данных (0 или 1), чтобы
видеть raw data.

Биновой эмпирической вероятностью: Усредненными значениями по 20-ти
квантильным бинам, чтобы сгладить шум и увидеть реальный тренд.

Сигнал есть: Модель уверенно выявляет устойчивые паттерны
взаимоотношения вида со средой. Пространственное распределение не
случайно.

Ведущие драйверы: Распределение вида в наибольшей степени контролируется
батиметрией и дистанцией от берега (dist, slope), что указывает на его
бентальную природу и приуроченность к шельфовым местообитаниям.

Роль гидрохимии: Второстепенную, но значимую роль играют динамические
факторы, связанные с продуктивностью и динамикой вод (chlorophyll,
diffuse\_attenuation, phosphate).

Подводные камни, на которые мы смотрим прямо Дисбаланс классов: Всего 35
присутствий на 515 отсутствий. Это могло сместить модель в сторону
предсказания отсутствия. Мы компенсировали это использованием
биномиальной семьи и внимательной диагностикой.

Экстраполяция: Модель обучена на ограниченном диапазоне условий.
Предсказания за пределами этого диапазона (например, на очень больших
глубинах) ненадежны. Это будет критически важно учитывать в третьем
скрипте при прогнозе на будущее.

Скрытая неопределенность: Диагностические графики (например, ширина
доверительных интервалов на кривых GAM) показывают, что уверенность
модели сильно варьирует в разных участках градиента переменных.

Заключение: Второй скрипт выполнил роль старателя, который не просто
намыл песок, а нашел в нем крупицы золота --- устойчивые статистические
сигналы. Мы не утверждаем, что нашли истину в последней инстанции, но мы
построили калиброванную, диагностированную и интерпретируемую модель,
которая отражает наши лучшие на данный момент знания о взаимоотношениях
вида со средой. Это надежный фундамент для следующего шага ---
ансамблевого прогнозирования.

\section{SDM и
прогноз}\label{sdm-ux438-ux43fux440ux43eux433ux43dux43eux437}

Ниже приводится скрипт, а после него - его анализ.

\begin{Shaded}
\begin{Highlighting}[]
\CommentTok{\# ========================================================================================================================}
\CommentTok{\# ПРАКТИЧЕСКОЕ ЗАНЯТИЕ:  Модели пространственного распределения видов (SDM) с biomod2}
\CommentTok{\# }
\CommentTok{\# Автор: Баканев С. В.  | Обновлено: Sys.Date()}
\CommentTok{\# ========================================================================================================================}

\CommentTok{\# Библиотеки {-}{-}{-}{-}{-}{-}{-}{-}{-}{-}{-}{-}{-}{-}{-}{-}{-}{-}{-}{-}{-}{-}{-}{-}{-}{-}{-}{-}{-}{-}{-}{-}{-}{-}{-}{-}{-}{-}{-}{-}{-}{-}{-}{-}{-}{-}{-}{-}{-}{-}{-}{-}{-}{-}{-}{-}{-}{-}{-}{-}{-}{-}{-}{-}{-}{-}{-}{-}{-}{-}{-}{-}{-}{-}{-}{-}{-}{-}{-}{-}{-}{-}{-}{-}{-}{-}{-}{-}{-}{-}{-}{-}{-}{-}{-}{-}{-}{-}{-}{-}{-}{-}{-}{-}{-}{-}{-}{-} \#}
\FunctionTok{suppressPackageStartupMessages}\NormalTok{(\{}
  \FunctionTok{library}\NormalTok{(biomod2)}
  \FunctionTok{library}\NormalTok{(sf)}
  \FunctionTok{library}\NormalTok{(marmap)}
  \FunctionTok{library}\NormalTok{(dplyr)}
  \FunctionTok{library}\NormalTok{(tidyr)}
  \FunctionTok{library}\NormalTok{(purrr)}
  \FunctionTok{library}\NormalTok{(ggplot2)}
  \FunctionTok{library}\NormalTok{(readr)}
  \FunctionTok{library}\NormalTok{(pROC)}
  \FunctionTok{library}\NormalTok{(precrec)}
  \FunctionTok{library}\NormalTok{(ecospat)}
  \FunctionTok{library}\NormalTok{(dismo)}
  \FunctionTok{library}\NormalTok{(rnaturalearth)}
  \FunctionTok{library}\NormalTok{(ggspatial)}
  \FunctionTok{library}\NormalTok{(raster)}
\NormalTok{\})}

\CommentTok{\# Опции и воспроизводимость {-}{-}{-}{-}{-}{-}{-}{-}{-}{-}{-}{-}{-}{-}{-}{-}{-}{-}{-}{-}{-}{-}{-}{-}{-}{-}{-}{-}{-}{-}{-}{-}{-}{-}{-}{-}{-}{-}{-}{-}{-}{-}{-}{-}{-}{-}{-}{-}{-}{-}{-}{-}{-}{-}{-}{-}{-}{-}{-}{-}{-}{-}{-}{-}{-}{-}{-}{-}{-}{-}{-}{-}{-}{-}{-}{-}{-}{-}{-}{-}{-}{-}{-}{-}{-}{-}{-}{-}{-}{-}{-}{-} \#}
\FunctionTok{options}\NormalTok{(}\AttributeTok{stringsAsFactors =} \ConstantTok{FALSE}\NormalTok{)}
\FunctionTok{set.seed}\NormalTok{(}\DecValTok{42}\NormalTok{)}

 \FunctionTok{setwd}\NormalTok{(}\StringTok{"C:/SDM"}\NormalTok{)  }\CommentTok{\# при необходимости}

\CommentTok{\# Хелперы устойчивые к ошибкам {-}{-}{-}{-}{-}{-}{-}{-}{-}{-}{-}{-}{-}{-}{-}{-}{-}{-}{-}{-}{-}{-}{-}{-}{-}{-}{-}{-}{-}{-}{-}{-}{-}{-}{-}{-}{-}{-}{-}{-}{-}{-}{-}{-}{-}{-}{-}{-}{-}{-}{-}{-}{-}{-}{-}{-}{-}{-}{-}{-}{-}{-}{-}{-}{-}{-}{-}{-}{-}{-}{-}{-}{-}{-}{-}{-}{-}{-}{-}{-}{-}{-}{-}{-}{-}{-}{-}{-}{-}{-} \#}
\NormalTok{ensure\_dir }\OtherTok{\textless{}{-}} \ControlFlowTok{function}\NormalTok{(path) \{}
  \ControlFlowTok{if}\NormalTok{ (}\SpecialCharTok{!}\FunctionTok{dir.exists}\NormalTok{(path)) }\FunctionTok{dir.create}\NormalTok{(path, }\AttributeTok{recursive =} \ConstantTok{TRUE}\NormalTok{, }\AttributeTok{showWarnings =} \ConstantTok{FALSE}\NormalTok{)}
\NormalTok{\}}

\NormalTok{scale\_predictions\_01 }\OtherTok{\textless{}{-}} \ControlFlowTok{function}\NormalTok{(predictions\_numeric) \{}
\NormalTok{  mx }\OtherTok{\textless{}{-}} \FunctionTok{suppressWarnings}\NormalTok{(}\FunctionTok{max}\NormalTok{(predictions\_numeric, }\AttributeTok{na.rm =} \ConstantTok{TRUE}\NormalTok{))}
  \ControlFlowTok{if}\NormalTok{ (}\FunctionTok{is.finite}\NormalTok{(mx) }\SpecialCharTok{\&\&}\NormalTok{ mx }\SpecialCharTok{\textgreater{}} \FloatTok{1.5}\NormalTok{) }\FunctionTok{return}\NormalTok{(}\FunctionTok{pmin}\NormalTok{(}\FunctionTok{pmax}\NormalTok{(predictions\_numeric }\SpecialCharTok{/} \DecValTok{1000}\NormalTok{, }\DecValTok{0}\NormalTok{), }\DecValTok{1}\NormalTok{))}
  \FunctionTok{pmin}\NormalTok{(}\FunctionTok{pmax}\NormalTok{(predictions\_numeric, }\DecValTok{0}\NormalTok{), }\DecValTok{1}\NormalTok{)}
\NormalTok{\}}

\NormalTok{calibration\_table }\OtherTok{\textless{}{-}} \ControlFlowTok{function}\NormalTok{(labels\_binary, probs\_01, }\AttributeTok{num\_bins =} \DecValTok{10}\NormalTok{) \{}
  \FunctionTok{stopifnot}\NormalTok{(}\FunctionTok{length}\NormalTok{(labels\_binary) }\SpecialCharTok{==} \FunctionTok{length}\NormalTok{(probs\_01))}
\NormalTok{  idx }\OtherTok{\textless{}{-}} \FunctionTok{is.finite}\NormalTok{(probs\_01) }\SpecialCharTok{\&} \FunctionTok{is.finite}\NormalTok{(labels\_binary)}
\NormalTok{  labels\_binary }\OtherTok{\textless{}{-}} \FunctionTok{as.integer}\NormalTok{(labels\_binary[idx])}
\NormalTok{  probs\_01 }\OtherTok{\textless{}{-}} \FunctionTok{as.numeric}\NormalTok{(probs\_01[idx])}
\NormalTok{  keep }\OtherTok{\textless{}{-}}\NormalTok{ labels\_binary }\SpecialCharTok{\%in\%} \FunctionTok{c}\NormalTok{(}\DecValTok{0}\NormalTok{, }\DecValTok{1}\NormalTok{)}
\NormalTok{  labels\_binary }\OtherTok{\textless{}{-}}\NormalTok{ labels\_binary[keep]}
\NormalTok{  probs\_01 }\OtherTok{\textless{}{-}}\NormalTok{ probs\_01[keep]}
  \ControlFlowTok{if}\NormalTok{ (}\SpecialCharTok{!}\FunctionTok{length}\NormalTok{(probs\_01)) }\FunctionTok{return}\NormalTok{(}\FunctionTok{list}\NormalTok{(}\AttributeTok{table =} \FunctionTok{tibble}\NormalTok{(), }\AttributeTok{brier =} \ConstantTok{NA\_real\_}\NormalTok{, }\AttributeTok{ece =} \ConstantTok{NA\_real\_}\NormalTok{))}
\NormalTok{  breaks }\OtherTok{\textless{}{-}} \FunctionTok{seq}\NormalTok{(}\DecValTok{0}\NormalTok{, }\DecValTok{1}\NormalTok{, }\AttributeTok{length.out =}\NormalTok{ num\_bins }\SpecialCharTok{+} \DecValTok{1}\NormalTok{)}
\NormalTok{  bin\_id }\OtherTok{\textless{}{-}} \FunctionTok{cut}\NormalTok{(probs\_01, }\AttributeTok{breaks =}\NormalTok{ breaks, }\AttributeTok{include.lowest =} \ConstantTok{TRUE}\NormalTok{, }\AttributeTok{labels =} \ConstantTok{FALSE}\NormalTok{)}
\NormalTok{  mids }\OtherTok{\textless{}{-}}\NormalTok{ (breaks[}\SpecialCharTok{{-}}\FunctionTok{length}\NormalTok{(breaks)] }\SpecialCharTok{+}\NormalTok{ breaks[}\SpecialCharTok{{-}}\DecValTok{1}\NormalTok{]) }\SpecialCharTok{/} \DecValTok{2}
\NormalTok{  tb }\OtherTok{\textless{}{-}} \FunctionTok{tibble}\NormalTok{(}\AttributeTok{bin\_id =}\NormalTok{ bin\_id, }\AttributeTok{prob =}\NormalTok{ probs\_01, }\AttributeTok{label =}\NormalTok{ labels\_binary) }\SpecialCharTok{\%\textgreater{}\%}
    \FunctionTok{group\_by}\NormalTok{(bin\_id) }\SpecialCharTok{\%\textgreater{}\%}
    \FunctionTok{summarise}\NormalTok{(}
      \AttributeTok{bin\_mid =}\NormalTok{ mids[}\FunctionTok{unique}\NormalTok{(bin\_id)],}
      \AttributeTok{prob\_mean =} \FunctionTok{mean}\NormalTok{(prob, }\AttributeTok{na.rm =} \ConstantTok{TRUE}\NormalTok{),}
      \AttributeTok{obs\_rate =} \FunctionTok{mean}\NormalTok{(label, }\AttributeTok{na.rm =} \ConstantTok{TRUE}\NormalTok{),}
      \AttributeTok{n =}\NormalTok{ dplyr}\SpecialCharTok{::}\FunctionTok{n}\NormalTok{(),}
      \AttributeTok{.groups =} \StringTok{"drop"}
\NormalTok{    ) }\SpecialCharTok{\%\textgreater{}\%} \FunctionTok{arrange}\NormalTok{(bin\_id)}
\NormalTok{  brier }\OtherTok{\textless{}{-}} \FunctionTok{mean}\NormalTok{((probs\_01 }\SpecialCharTok{{-}}\NormalTok{ labels\_binary)}\SpecialCharTok{\^{}}\DecValTok{2}\NormalTok{, }\AttributeTok{na.rm =} \ConstantTok{TRUE}\NormalTok{)}
\NormalTok{  ece }\OtherTok{\textless{}{-}} \FunctionTok{sum}\NormalTok{((tb}\SpecialCharTok{$}\NormalTok{n }\SpecialCharTok{/} \FunctionTok{sum}\NormalTok{(tb}\SpecialCharTok{$}\NormalTok{n)) }\SpecialCharTok{*} \FunctionTok{abs}\NormalTok{(tb}\SpecialCharTok{$}\NormalTok{obs\_rate }\SpecialCharTok{{-}}\NormalTok{ tb}\SpecialCharTok{$}\NormalTok{prob\_mean))}
  \FunctionTok{list}\NormalTok{(}\AttributeTok{table =}\NormalTok{ tb, }\AttributeTok{brier =}\NormalTok{ brier, }\AttributeTok{ece =}\NormalTok{ ece)}
\NormalTok{\}}

\NormalTok{plot\_calibration }\OtherTok{\textless{}{-}} \ControlFlowTok{function}\NormalTok{(tbl, }\AttributeTok{title =} \StringTok{"Калибровка (reliability)"}\NormalTok{) \{}
  \FunctionTok{ggplot}\NormalTok{(tbl, }\FunctionTok{aes}\NormalTok{(}\AttributeTok{x =}\NormalTok{ prob\_mean, }\AttributeTok{y =}\NormalTok{ obs\_rate, }\AttributeTok{size =}\NormalTok{ n)) }\SpecialCharTok{+}
    \FunctionTok{geom\_abline}\NormalTok{(}\AttributeTok{slope =} \DecValTok{1}\NormalTok{, }\AttributeTok{intercept =} \DecValTok{0}\NormalTok{, }\AttributeTok{linetype =} \StringTok{"dashed"}\NormalTok{, }\AttributeTok{color =} \StringTok{"gray50"}\NormalTok{) }\SpecialCharTok{+}
    \FunctionTok{geom\_point}\NormalTok{(}\AttributeTok{color =} \StringTok{"\#2C7FB8"}\NormalTok{, }\AttributeTok{alpha =} \FloatTok{0.85}\NormalTok{) }\SpecialCharTok{+}
    \FunctionTok{scale\_size\_continuous}\NormalTok{(}\AttributeTok{name =} \StringTok{"N"}\NormalTok{) }\SpecialCharTok{+}
    \FunctionTok{coord\_fixed}\NormalTok{(}\AttributeTok{xlim =} \FunctionTok{c}\NormalTok{(}\DecValTok{0}\NormalTok{, }\DecValTok{1}\NormalTok{), }\AttributeTok{ylim =} \FunctionTok{c}\NormalTok{(}\DecValTok{0}\NormalTok{, }\DecValTok{1}\NormalTok{)) }\SpecialCharTok{+}
    \FunctionTok{labs}\NormalTok{(}\AttributeTok{x =} \StringTok{"Средняя предсказанная вероятность"}\NormalTok{, }\AttributeTok{y =} \StringTok{"Наблюдаемая доля присутствий"}\NormalTok{, }\AttributeTok{title =}\NormalTok{ title) }\SpecialCharTok{+}
    \FunctionTok{theme\_minimal}\NormalTok{(}\AttributeTok{base\_size =} \DecValTok{12}\NormalTok{)}
\NormalTok{\}}

\NormalTok{roc\_pr\_metrics }\OtherTok{\textless{}{-}} \ControlFlowTok{function}\NormalTok{(labels\_binary, probs\_01) \{}
\NormalTok{  idx }\OtherTok{\textless{}{-}} \FunctionTok{is.finite}\NormalTok{(probs\_01) }\SpecialCharTok{\&} \FunctionTok{is.finite}\NormalTok{(labels\_binary)}
\NormalTok{  labels\_binary }\OtherTok{\textless{}{-}} \FunctionTok{as.integer}\NormalTok{(labels\_binary[idx])}
\NormalTok{  probs\_01 }\OtherTok{\textless{}{-}} \FunctionTok{as.numeric}\NormalTok{(probs\_01[idx])}
\NormalTok{  keep }\OtherTok{\textless{}{-}}\NormalTok{ labels\_binary }\SpecialCharTok{\%in\%} \FunctionTok{c}\NormalTok{(}\DecValTok{0}\NormalTok{, }\DecValTok{1}\NormalTok{)}
\NormalTok{  labels\_binary }\OtherTok{\textless{}{-}}\NormalTok{ labels\_binary[keep]}
\NormalTok{  probs\_01 }\OtherTok{\textless{}{-}}\NormalTok{ probs\_01[keep]}
  \ControlFlowTok{if}\NormalTok{ (}\FunctionTok{length}\NormalTok{(}\FunctionTok{unique}\NormalTok{(labels\_binary)) }\SpecialCharTok{\textless{}} \DecValTok{2}\NormalTok{) \{}
    \FunctionTok{return}\NormalTok{(}\FunctionTok{list}\NormalTok{(}\AttributeTok{roc =} \ConstantTok{NULL}\NormalTok{, }\AttributeTok{auc\_roc =} \ConstantTok{NA\_real\_}\NormalTok{, }\AttributeTok{pr =} \ConstantTok{NULL}\NormalTok{, }\AttributeTok{auc\_pr =} \ConstantTok{NA\_real\_}\NormalTok{))}
\NormalTok{  \}}
\NormalTok{  roc\_obj }\OtherTok{\textless{}{-}} \FunctionTok{tryCatch}\NormalTok{(pROC}\SpecialCharTok{::}\FunctionTok{roc}\NormalTok{(}\AttributeTok{response =}\NormalTok{ labels\_binary, }\AttributeTok{predictor =}\NormalTok{ probs\_01, }\AttributeTok{quiet =} \ConstantTok{TRUE}\NormalTok{), }\AttributeTok{error =} \ControlFlowTok{function}\NormalTok{(e) }\ConstantTok{NULL}\NormalTok{)}
\NormalTok{  auc\_roc }\OtherTok{\textless{}{-}} \ControlFlowTok{if}\NormalTok{ (}\SpecialCharTok{!}\FunctionTok{is.null}\NormalTok{(roc\_obj)) }\FunctionTok{as.numeric}\NormalTok{(pROC}\SpecialCharTok{::}\FunctionTok{auc}\NormalTok{(roc\_obj)[}\DecValTok{1}\NormalTok{]) }\ControlFlowTok{else} \ConstantTok{NA\_real\_}
\NormalTok{  pr\_obj }\OtherTok{\textless{}{-}} \FunctionTok{tryCatch}\NormalTok{(precrec}\SpecialCharTok{::}\FunctionTok{evalmod}\NormalTok{(}\AttributeTok{scores =}\NormalTok{ probs\_01, }\AttributeTok{labels =}\NormalTok{ labels\_binary), }\AttributeTok{error =} \ControlFlowTok{function}\NormalTok{(e) }\ConstantTok{NULL}\NormalTok{)}
\NormalTok{  auc\_pr }\OtherTok{\textless{}{-}} \ControlFlowTok{if}\NormalTok{ (}\SpecialCharTok{!}\FunctionTok{is.null}\NormalTok{(pr\_obj)) \{}
\NormalTok{    dd }\OtherTok{\textless{}{-}}\NormalTok{ precrec}\SpecialCharTok{::}\FunctionTok{auc}\NormalTok{(pr\_obj)}
    \FunctionTok{as.numeric}\NormalTok{(dd }\SpecialCharTok{\%\textgreater{}\%}\NormalTok{ dplyr}\SpecialCharTok{::}\FunctionTok{filter}\NormalTok{(curvetypes }\SpecialCharTok{==} \StringTok{"PRC"}\NormalTok{) }\SpecialCharTok{\%\textgreater{}\%}\NormalTok{ dplyr}\SpecialCharTok{::}\FunctionTok{pull}\NormalTok{(aucs))}
\NormalTok{  \} }\ControlFlowTok{else} \ConstantTok{NA\_real\_}
  \FunctionTok{list}\NormalTok{(}\AttributeTok{roc =}\NormalTok{ roc\_obj, }\AttributeTok{auc\_roc =}\NormalTok{ auc\_roc, }\AttributeTok{pr =}\NormalTok{ pr\_obj, }\AttributeTok{auc\_pr =}\NormalTok{ auc\_pr)}
\NormalTok{\}}

\NormalTok{plot\_roc\_curve }\OtherTok{\textless{}{-}} \ControlFlowTok{function}\NormalTok{(roc\_obj, }\AttributeTok{title =} \StringTok{"ROC кривая"}\NormalTok{) \{}
  \ControlFlowTok{if}\NormalTok{ (}\FunctionTok{is.null}\NormalTok{(roc\_obj)) }\FunctionTok{return}\NormalTok{(}\FunctionTok{ggplot}\NormalTok{() }\SpecialCharTok{+} \FunctionTok{labs}\NormalTok{(}\AttributeTok{title =} \FunctionTok{paste}\NormalTok{(title, }\StringTok{"(недостаточно классов)"}\NormalTok{)))}
\NormalTok{  pROC}\SpecialCharTok{::}\FunctionTok{ggroc}\NormalTok{(roc\_obj, }\AttributeTok{colour =} \StringTok{"\#1B9E77"}\NormalTok{, }\AttributeTok{size =} \DecValTok{1}\NormalTok{) }\SpecialCharTok{+}
    \FunctionTok{geom\_abline}\NormalTok{(}\AttributeTok{slope =} \DecValTok{1}\NormalTok{, }\AttributeTok{intercept =} \DecValTok{1}\NormalTok{, }\AttributeTok{linetype =} \StringTok{"dashed"}\NormalTok{, }\AttributeTok{color =} \StringTok{"gray50"}\NormalTok{) }\SpecialCharTok{+}
    \FunctionTok{coord\_equal}\NormalTok{(}\AttributeTok{xlim =} \FunctionTok{c}\NormalTok{(}\DecValTok{1}\NormalTok{, }\DecValTok{0}\NormalTok{), }\AttributeTok{ylim =} \FunctionTok{c}\NormalTok{(}\DecValTok{0}\NormalTok{, }\DecValTok{1}\NormalTok{)) }\SpecialCharTok{+}
    \FunctionTok{labs}\NormalTok{(}\AttributeTok{x =} \StringTok{"1 {-} Specificity"}\NormalTok{, }\AttributeTok{y =} \StringTok{"Sensitivity"}\NormalTok{, }\AttributeTok{title =}\NormalTok{ title) }\SpecialCharTok{+}
    \FunctionTok{theme\_minimal}\NormalTok{(}\AttributeTok{base\_size =} \DecValTok{12}\NormalTok{)}
\NormalTok{\}}

\NormalTok{plot\_pr\_curve }\OtherTok{\textless{}{-}} \ControlFlowTok{function}\NormalTok{(pr\_obj, }\AttributeTok{title =} \StringTok{"PR кривая"}\NormalTok{) \{}
  \ControlFlowTok{if}\NormalTok{ (}\FunctionTok{is.null}\NormalTok{(pr\_obj)) }\FunctionTok{return}\NormalTok{(}\FunctionTok{ggplot}\NormalTok{() }\SpecialCharTok{+} \FunctionTok{labs}\NormalTok{(}\AttributeTok{title =} \FunctionTok{paste}\NormalTok{(title, }\StringTok{"(недостаточно классов)"}\NormalTok{)))}
  \FunctionTok{autoplot}\NormalTok{(pr\_obj) }\SpecialCharTok{+} \FunctionTok{labs}\NormalTok{(}\AttributeTok{title =}\NormalTok{ title) }\SpecialCharTok{+} \FunctionTok{theme\_minimal}\NormalTok{(}\AttributeTok{base\_size =} \DecValTok{12}\NormalTok{)}
\NormalTok{\}}

\NormalTok{optimal\_threshold\_tss }\OtherTok{\textless{}{-}} \ControlFlowTok{function}\NormalTok{(labels\_binary, probs\_01, }\AttributeTok{step =} \FloatTok{0.01}\NormalTok{) \{}
\NormalTok{  thresholds }\OtherTok{\textless{}{-}} \FunctionTok{seq}\NormalTok{(}\DecValTok{0}\NormalTok{, }\DecValTok{1}\NormalTok{, }\AttributeTok{by =}\NormalTok{ step)}
\NormalTok{  idx }\OtherTok{\textless{}{-}} \FunctionTok{is.finite}\NormalTok{(probs\_01) }\SpecialCharTok{\&} \FunctionTok{is.finite}\NormalTok{(labels\_binary)}
\NormalTok{  labels\_binary }\OtherTok{\textless{}{-}} \FunctionTok{as.integer}\NormalTok{(labels\_binary[idx])}
\NormalTok{  probs\_01 }\OtherTok{\textless{}{-}} \FunctionTok{as.numeric}\NormalTok{(probs\_01[idx])}
\NormalTok{  keep }\OtherTok{\textless{}{-}}\NormalTok{ labels\_binary }\SpecialCharTok{\%in\%} \FunctionTok{c}\NormalTok{(}\DecValTok{0}\NormalTok{, }\DecValTok{1}\NormalTok{)}
\NormalTok{  labels\_binary }\OtherTok{\textless{}{-}}\NormalTok{ labels\_binary[keep]}
\NormalTok{  probs\_01 }\OtherTok{\textless{}{-}}\NormalTok{ probs\_01[keep]}
  \ControlFlowTok{if}\NormalTok{ (}\SpecialCharTok{!}\FunctionTok{length}\NormalTok{(probs\_01)) \{}
    \FunctionTok{return}\NormalTok{(}\FunctionTok{data.frame}\NormalTok{(}\AttributeTok{threshold =} \ConstantTok{NA\_real\_}\NormalTok{, }\AttributeTok{TSS =} \ConstantTok{NA\_real\_}\NormalTok{, }\AttributeTok{Sensitivity =} \ConstantTok{NA\_real\_}\NormalTok{, }\AttributeTok{Specificity =} \ConstantTok{NA\_real\_}\NormalTok{))}
\NormalTok{  \}}
\NormalTok{  metrics }\OtherTok{\textless{}{-}} \FunctionTok{lapply}\NormalTok{(thresholds, }\ControlFlowTok{function}\NormalTok{(th) \{}
\NormalTok{    pred\_class }\OtherTok{\textless{}{-}} \FunctionTok{as.integer}\NormalTok{(probs\_01 }\SpecialCharTok{\textgreater{}=}\NormalTok{ th)}
\NormalTok{    tp }\OtherTok{\textless{}{-}} \FunctionTok{sum}\NormalTok{(pred\_class }\SpecialCharTok{==} \DecValTok{1} \SpecialCharTok{\&}\NormalTok{ labels\_binary }\SpecialCharTok{==} \DecValTok{1}\NormalTok{)}
\NormalTok{    tn }\OtherTok{\textless{}{-}} \FunctionTok{sum}\NormalTok{(pred\_class }\SpecialCharTok{==} \DecValTok{0} \SpecialCharTok{\&}\NormalTok{ labels\_binary }\SpecialCharTok{==} \DecValTok{0}\NormalTok{)}
\NormalTok{    fp }\OtherTok{\textless{}{-}} \FunctionTok{sum}\NormalTok{(pred\_class }\SpecialCharTok{==} \DecValTok{1} \SpecialCharTok{\&}\NormalTok{ labels\_binary }\SpecialCharTok{==} \DecValTok{0}\NormalTok{)}
\NormalTok{    fn }\OtherTok{\textless{}{-}} \FunctionTok{sum}\NormalTok{(pred\_class }\SpecialCharTok{==} \DecValTok{0} \SpecialCharTok{\&}\NormalTok{ labels\_binary }\SpecialCharTok{==} \DecValTok{1}\NormalTok{)}
\NormalTok{    tpr }\OtherTok{\textless{}{-}} \FunctionTok{ifelse}\NormalTok{((tp }\SpecialCharTok{+}\NormalTok{ fn) }\SpecialCharTok{\textgreater{}} \DecValTok{0}\NormalTok{, tp }\SpecialCharTok{/}\NormalTok{ (tp }\SpecialCharTok{+}\NormalTok{ fn), }\ConstantTok{NA\_real\_}\NormalTok{)}
\NormalTok{    tnr }\OtherTok{\textless{}{-}} \FunctionTok{ifelse}\NormalTok{((tn }\SpecialCharTok{+}\NormalTok{ fp) }\SpecialCharTok{\textgreater{}} \DecValTok{0}\NormalTok{, tn }\SpecialCharTok{/}\NormalTok{ (tn }\SpecialCharTok{+}\NormalTok{ fp), }\ConstantTok{NA\_real\_}\NormalTok{)}
    \FunctionTok{c}\NormalTok{(}\AttributeTok{threshold =}\NormalTok{ th, }\AttributeTok{TSS =}\NormalTok{ (tpr }\SpecialCharTok{+}\NormalTok{ tnr }\SpecialCharTok{{-}} \DecValTok{1}\NormalTok{), }\AttributeTok{Sensitivity =}\NormalTok{ tpr, }\AttributeTok{Specificity =}\NormalTok{ tnr)}
\NormalTok{  \})}
\NormalTok{  m }\OtherTok{\textless{}{-}} \FunctionTok{do.call}\NormalTok{(rbind, metrics)}
\NormalTok{  m }\OtherTok{\textless{}{-}} \FunctionTok{as.data.frame}\NormalTok{(m)}
\NormalTok{  m}\SpecialCharTok{$}\NormalTok{threshold }\OtherTok{\textless{}{-}} \FunctionTok{as.numeric}\NormalTok{(m}\SpecialCharTok{$}\NormalTok{threshold)}
\NormalTok{  m}\SpecialCharTok{$}\NormalTok{TSS }\OtherTok{\textless{}{-}} \FunctionTok{as.numeric}\NormalTok{(m}\SpecialCharTok{$}\NormalTok{TSS)}
\NormalTok{  m}\SpecialCharTok{$}\NormalTok{Sensitivity }\OtherTok{\textless{}{-}} \FunctionTok{as.numeric}\NormalTok{(m}\SpecialCharTok{$}\NormalTok{Sensitivity)}
\NormalTok{  m}\SpecialCharTok{$}\NormalTok{Specificity }\OtherTok{\textless{}{-}} \FunctionTok{as.numeric}\NormalTok{(m}\SpecialCharTok{$}\NormalTok{Specificity)}
\NormalTok{  best\_row }\OtherTok{\textless{}{-}}\NormalTok{ m[}\FunctionTok{which.max}\NormalTok{(m}\SpecialCharTok{$}\NormalTok{TSS), , drop }\OtherTok{=} \ConstantTok{FALSE}\NormalTok{]}
\NormalTok{  best\_row}
\NormalTok{\}}

\NormalTok{boyce\_index }\OtherTok{\textless{}{-}} \ControlFlowTok{function}\NormalTok{(labels\_binary, probs\_01, }\AttributeTok{num\_class =} \DecValTok{0}\NormalTok{, }\AttributeTok{window\_w =} \ConstantTok{NULL}\NormalTok{) \{}
\NormalTok{  idx }\OtherTok{\textless{}{-}} \FunctionTok{is.finite}\NormalTok{(probs\_01) }\SpecialCharTok{\&} \FunctionTok{is.finite}\NormalTok{(labels\_binary)}
\NormalTok{  labels\_binary }\OtherTok{\textless{}{-}} \FunctionTok{as.integer}\NormalTok{(labels\_binary[idx])}
\NormalTok{  probs\_01 }\OtherTok{\textless{}{-}} \FunctionTok{as.numeric}\NormalTok{(probs\_01[idx])}
\NormalTok{  pres }\OtherTok{\textless{}{-}}\NormalTok{ probs\_01[labels\_binary }\SpecialCharTok{==} \DecValTok{1}\NormalTok{]}
\NormalTok{  back }\OtherTok{\textless{}{-}}\NormalTok{ probs\_01}
  \FunctionTok{tryCatch}\NormalTok{(\{}
\NormalTok{    res }\OtherTok{\textless{}{-}}\NormalTok{ ecospat}\SpecialCharTok{::}\FunctionTok{ecospat.boyce}\NormalTok{(}\AttributeTok{fit =}\NormalTok{ back, }\AttributeTok{obs =}\NormalTok{ pres, }\AttributeTok{nclass =}\NormalTok{ num\_class, }\AttributeTok{window.w =}\NormalTok{ window\_w)}
    \FunctionTok{list}\NormalTok{(}\AttributeTok{CBI =} \FunctionTok{as.numeric}\NormalTok{(res}\SpecialCharTok{$}\NormalTok{Spearman.cor), }\AttributeTok{curve =}\NormalTok{ res}\SpecialCharTok{$}\NormalTok{F.ratio)}
\NormalTok{  \}, }\AttributeTok{error =} \ControlFlowTok{function}\NormalTok{(e) }\FunctionTok{list}\NormalTok{(}\AttributeTok{CBI =} \ConstantTok{NA\_real\_}\NormalTok{, }\AttributeTok{curve =} \ConstantTok{NULL}\NormalTok{))}
\NormalTok{\}}

\NormalTok{uncertainty\_per\_point }\OtherTok{\textless{}{-}} \ControlFlowTok{function}\NormalTok{(pred\_long\_df) \{}
  \CommentTok{\# Ожидаем столбцы: run, algo, points, pred}
  \ControlFlowTok{if}\NormalTok{ (}\SpecialCharTok{!}\FunctionTok{all}\NormalTok{(}\FunctionTok{c}\NormalTok{(}\StringTok{"points"}\NormalTok{, }\StringTok{"pred"}\NormalTok{) }\SpecialCharTok{\%in\%} \FunctionTok{names}\NormalTok{(pred\_long\_df))) \{}
    \FunctionTok{stop}\NormalTok{(}\StringTok{"Для неопределенности нужен длинный формат с колонками \textquotesingle{}points\textquotesingle{} и \textquotesingle{}pred\textquotesingle{}."}\NormalTok{)}
\NormalTok{  \}}
\NormalTok{  run\_levels }\OtherTok{\textless{}{-}} \FunctionTok{unique}\NormalTok{(pred\_long\_df}\SpecialCharTok{$}\NormalTok{run)}
\NormalTok{  run\_pick }\OtherTok{\textless{}{-}} \ControlFlowTok{if}\NormalTok{ (}\StringTok{"allRun"} \SpecialCharTok{\%in\%}\NormalTok{ run\_levels) }\StringTok{"allRun"} \ControlFlowTok{else}\NormalTok{ run\_levels[}\DecValTok{1}\NormalTok{]}
\NormalTok{  df }\OtherTok{\textless{}{-}} \FunctionTok{subset}\NormalTok{(pred\_long\_df, run }\SpecialCharTok{==}\NormalTok{ run\_pick)}
  \CommentTok{\# Агрегируем по точкам: SD и среднее по всем алгоритмам}
\NormalTok{  agg\_sd }\OtherTok{\textless{}{-}} \FunctionTok{aggregate}\NormalTok{(pred }\SpecialCharTok{\textasciitilde{}}\NormalTok{ points, }\AttributeTok{data =}\NormalTok{ df, }\AttributeTok{FUN =} \ControlFlowTok{function}\NormalTok{(x) }\FunctionTok{sd}\NormalTok{(}\FunctionTok{as.numeric}\NormalTok{(x), }\AttributeTok{na.rm =} \ConstantTok{TRUE}\NormalTok{))}
  \FunctionTok{names}\NormalTok{(agg\_sd)[}\DecValTok{2}\NormalTok{] }\OtherTok{\textless{}{-}} \StringTok{"UNC\_SD"}
\NormalTok{  agg\_mean }\OtherTok{\textless{}{-}} \FunctionTok{aggregate}\NormalTok{(pred }\SpecialCharTok{\textasciitilde{}}\NormalTok{ points, }\AttributeTok{data =}\NormalTok{ df, }\AttributeTok{FUN =} \ControlFlowTok{function}\NormalTok{(x) }\FunctionTok{mean}\NormalTok{(}\FunctionTok{as.numeric}\NormalTok{(x), }\AttributeTok{na.rm =} \ConstantTok{TRUE}\NormalTok{))}
  \FunctionTok{names}\NormalTok{(agg\_mean)[}\DecValTok{2}\NormalTok{] }\OtherTok{\textless{}{-}} \StringTok{"MEAN\_PRED"}
\NormalTok{  unc }\OtherTok{\textless{}{-}} \FunctionTok{merge}\NormalTok{(agg\_sd, agg\_mean, }\AttributeTok{by =} \StringTok{"points"}\NormalTok{, }\AttributeTok{all =} \ConstantTok{TRUE}\NormalTok{)}
\NormalTok{  unc}\SpecialCharTok{$}\NormalTok{UNC\_CV }\OtherTok{\textless{}{-}}\NormalTok{ unc}\SpecialCharTok{$}\NormalTok{UNC\_SD }\SpecialCharTok{/} \FunctionTok{ifelse}\NormalTok{(unc}\SpecialCharTok{$}\NormalTok{MEAN\_PRED }\SpecialCharTok{==} \DecValTok{0}\NormalTok{, }\ConstantTok{NA}\NormalTok{, unc}\SpecialCharTok{$}\NormalTok{MEAN\_PRED)}
\NormalTok{  unc}
\NormalTok{\}}

\NormalTok{mess\_scores }\OtherTok{\textless{}{-}} \ControlFlowTok{function}\NormalTok{(reference\_env\_df, target\_env\_df) \{}
  \FunctionTok{tryCatch}\NormalTok{(\{}
    \FunctionTok{as.numeric}\NormalTok{(dismo}\SpecialCharTok{::}\FunctionTok{mess}\NormalTok{(}\AttributeTok{x =} \FunctionTok{as.data.frame}\NormalTok{(target\_env\_df), }\AttributeTok{v =} \FunctionTok{as.data.frame}\NormalTok{(reference\_env\_df)))}
\NormalTok{  \}, }\AttributeTok{error =} \ControlFlowTok{function}\NormalTok{(e) }\FunctionTok{rep}\NormalTok{(}\ConstantTok{NA\_real\_}\NormalTok{, }\FunctionTok{nrow}\NormalTok{(target\_env\_df)))}
\NormalTok{\}}

\CommentTok{\# ДАННЫЕ: текущее состояние {-}{-}{-}{-}{-}{-}{-}{-}{-}{-}{-}{-}{-}{-}{-}{-}{-}{-}{-}{-}{-}{-}{-}{-}{-}{-}{-}{-}{-}{-}{-}{-}{-}{-}{-}{-}{-}{-}{-}{-}{-}{-}{-}{-}{-}{-}{-}{-}{-}{-}{-}{-}{-}{-}{-}{-}{-}{-}{-}{-}{-}{-}{-}{-}{-}{-}{-}{-}{-}{-}{-}{-}{-}{-}{-}{-}{-}{-}{-}{-}{-}{-}{-}{-}{-}{-}{-}{-}{-}{-}{-}{-} \#}
\NormalTok{DATA }\OtherTok{\textless{}{-}} \FunctionTok{read.csv}\NormalTok{(}\StringTok{"final\_sdm\_table\_with\_na.csv"}\NormalTok{)}
\FunctionTok{str}\NormalTok{(DATA)}
\end{Highlighting}
\end{Shaded}

\begin{verbatim}
'data.frame':   44929 obs. of  11 variables:
 $ x                        : num  10 10.1 10.1 10.2 10.2 ...
 $ y                        : num  66 66 66 66 66 ...
 $ occ                      : int  NA NA NA NA NA NA NA NA NA NA ...
 $ dist                     : num  89.6 87.5 85.5 83.5 81.4 ...
 $ chlorophyll_range        : num  2.14 2.16 2.16 2.13 2.1 ...
 $ current_velocity_range   : num  0.0301 0.0204 0.0184 0.0123 0.0133 ...
 $ diffuse_attenuation_range: num  0.1 0.101 0.101 0.102 0.107 ...
 $ phosphate_range          : num  0.189 0.186 0.182 0.178 0.176 ...
 $ silicate_range           : num  4.25 4.03 3.77 3.63 3.58 ...
 $ slope                    : num  0.0765 0.1032 0.1524 0.2003 0.2112 ...
 $ temperature_mean         : num  7.37 7.39 7.41 7.43 7.43 ...
\end{verbatim}

\begin{Shaded}
\begin{Highlighting}[]
\NormalTok{DataSpecies }\OtherTok{\textless{}{-}} \FunctionTok{as.data.frame}\NormalTok{(DATA)}
\NormalTok{myRespName }\OtherTok{\textless{}{-}} \StringTok{\textquotesingle{}occ\textquotesingle{}}
\NormalTok{myResp }\OtherTok{\textless{}{-}} \FunctionTok{as.numeric}\NormalTok{(DataSpecies[[myRespName]])}
\NormalTok{myRespXY }\OtherTok{\textless{}{-}}\NormalTok{ DataSpecies[, }\FunctionTok{c}\NormalTok{(}\StringTok{"x"}\NormalTok{, }\StringTok{"y"}\NormalTok{)]}
\NormalTok{myExpl }\OtherTok{\textless{}{-}}\NormalTok{ DataSpecies[, }\DecValTok{4}\SpecialCharTok{:}\DecValTok{11}\NormalTok{]}

\NormalTok{myBiomodData }\OtherTok{\textless{}{-}} \FunctionTok{BIOMOD\_FormatingData}\NormalTok{(}
  \AttributeTok{resp.var =}\NormalTok{ myResp,}
  \AttributeTok{expl.var =}\NormalTok{ myExpl,}
  \AttributeTok{resp.xy =}\NormalTok{ myRespXY,}
  \AttributeTok{resp.name =}\NormalTok{ myRespName}
\NormalTok{)}
\end{Highlighting}
\end{Shaded}

\begin{verbatim}

-=-=-=-=-=-=-=-=-=-=-=-=-=-=-= occ Data Formating -=-=-=-=-=-=-=-=-=-=-=-=-=-=-=

      ! No data has been set aside for modeling evaluation
 ! Some NAs have been automatically removed from your data
-=-=-=-=-=-=-=-=-=-=-=-=-=-=-=-=-=-= Done -=-=-=-=-=-=-=-=-=-=-=-=-=-=-=-=-=-=
\end{verbatim}

\begin{Shaded}
\begin{Highlighting}[]
\FunctionTok{print}\NormalTok{(myBiomodData)}
\end{Highlighting}
\end{Shaded}

\begin{verbatim}

-=-=-=-=-=-=-=-=-=-=-=-=-=-= BIOMOD.formated.data -=-=-=-=-=-=-=-=-=-=-=-=-=-=

dir.name =  .

sp.name =  occ

     35 presences,  515 true absences and  43970 undefined points in dataset


     8 explanatory variables

      dist        chlorophyll_range current_velocity_range
 Min.   :  0.00   Min.   :0.8033    Min.   :3.900e-07     
 1st Qu.: 29.25   1st Qu.:1.7376    1st Qu.:1.394e-02     
 Median : 85.09   Median :1.9896    Median :3.461e-02     
 Mean   :108.12   Mean   :2.0218    Mean   :5.922e-02     
 3rd Qu.:171.22   3rd Qu.:2.3441    3rd Qu.:7.576e-02     
 Max.   :370.42   Max.   :3.6717    Max.   :6.866e-01     
 diffuse_attenuation_range phosphate_range   silicate_range    
 Min.   :0.00219           Min.   :0.02029   Min.   :  0.8998  
 1st Qu.:0.10350           1st Qu.:0.17822   1st Qu.:  2.6714  
 Median :0.13125           Median :0.26244   Median :  4.2001  
 Mean   :0.15390           Mean   :0.26352   Mean   :  5.8540  
 3rd Qu.:0.17859           3rd Qu.:0.34718   3rd Qu.:  5.6003  
 Max.   :1.13555           Max.   :1.06630   Max.   :150.8238  
     slope          temperature_mean 
 Min.   : 0.00000   Min.   :-0.9655  
 1st Qu.: 0.06876   1st Qu.: 1.4951  
 Median : 0.15116   Median : 3.5395  
 Mean   : 0.37223   Mean   : 3.4536  
 3rd Qu.: 0.36902   3rd Qu.: 5.8108  
 Max.   :10.47394   Max.   : 9.0227  

-=-=-=-=-=-=-=-=-=-=-=-=-=-=-=-=-=-=-=-=-=-=-=-=-=-=-=-=-=-=-=-=-=-=-=-=-=-=-=-=
\end{verbatim}

\begin{Shaded}
\begin{Highlighting}[]
\FunctionTok{plot}\NormalTok{(myBiomodData)}
\end{Highlighting}
\end{Shaded}

\begin{verbatim}
Загрузка требуемого пакета: ggtext
\end{verbatim}

\pandocbounded{\includegraphics[keepaspectratio]{chapter12_files/figure-pdf/unnamed-chunk-15-1.pdf}}

\begin{verbatim}
$data.vect
 class       : SpatVector 
 geometry    : points 
 dimensions  : 550, 2  (geometries, attributes)
 extent      : 12.075, 44.925, 66.375, 71.975  (xmin, xmax, ymin, ymax)
 coord. ref. :  
 names       :  resp         dataset
 type        : <num>           <chr>
 values      :    10 Initial dataset
                  10 Initial dataset
                  10 Initial dataset

$data.label
                              9                              10 
                "**Presences**"       "Presences (calibration)" 
                             11                              12 
       "Presences (validation)"        "Presences (evaluation)" 
                             19                              20 
            "**True Absences**"   "True Absences (calibration)" 
                             21                              22 
   "True Absences (validation)"    "True Absences (evaluation)" 
                             29                              30 
          "**Pseudo-Absences**" "Pseudo-Absences (calibration)" 
                             31                               1 
 "Pseudo-Absences (validation)"                    "Background" 

$data.plot
\end{verbatim}

\pandocbounded{\includegraphics[keepaspectratio]{chapter12_files/figure-pdf/unnamed-chunk-15-2.pdf}}

\begin{figure}[H]

{\centering \includegraphics[width=0.8\linewidth,height=\textheight,keepaspectratio]{images/SDM10.PNG}

}

\caption{Рис. 10.: Входные данные по встречаемости}

\end{figure}%

\begin{Shaded}
\begin{Highlighting}[]
\CommentTok{\# ОБУЧЕНИЕ ЕДИНИЧНЫХ МОДЕЛЕЙ {-}{-}{-}{-}{-}{-}{-}{-}{-}{-}{-}{-}{-}{-}{-}{-}{-}{-}{-}{-}{-}{-}{-}{-}{-}{-}{-}{-}{-}{-}{-}{-}{-}{-}{-}{-}{-}{-}{-}{-}{-}{-}{-}{-}{-}{-}{-}{-}{-}{-}{-}{-}{-}{-}{-}{-}{-}{-}{-}{-}{-}{-}{-}{-}{-}{-}{-}{-}{-}{-}{-}{-}{-}{-}{-}{-}{-}{-}{-}{-}{-}{-}{-}{-}{-}{-}{-}{-}{-}{-}{-} \#}
\NormalTok{algos }\OtherTok{\textless{}{-}} \FunctionTok{c}\NormalTok{(}\StringTok{"ANN"}\NormalTok{, }\StringTok{"CTA"}\NormalTok{, }\StringTok{"FDA"}\NormalTok{, }\StringTok{"GAM"}\NormalTok{, }\StringTok{"GBM"}\NormalTok{, }\StringTok{"GLM"}\NormalTok{, }\StringTok{"MAXENT"}\NormalTok{, }\StringTok{"MAXNET"}\NormalTok{, }\StringTok{"RF"}\NormalTok{, }\StringTok{"XGBOOST"}\NormalTok{)}
\CommentTok{\# Уберем MAXENT, если нет maxent.jar рядом с рабочей директорией}
\ControlFlowTok{if}\NormalTok{ (}\SpecialCharTok{!}\FunctionTok{file.exists}\NormalTok{(}\FunctionTok{file.path}\NormalTok{(}\FunctionTok{getwd}\NormalTok{(), }\StringTok{"maxent.jar"}\NormalTok{))) \{}
\NormalTok{  algos }\OtherTok{\textless{}{-}} \FunctionTok{setdiff}\NormalTok{(algos, }\StringTok{"MAXENT"}\NormalTok{)}
\NormalTok{\}}

\NormalTok{myBiomodModelOut }\OtherTok{\textless{}{-}} \FunctionTok{BIOMOD\_Modeling}\NormalTok{(}
  \AttributeTok{bm.format =}\NormalTok{ myBiomodData,}
  \AttributeTok{modeling.id =} \StringTok{\textquotesingle{}AllModels\textquotesingle{}}\NormalTok{,}
  \AttributeTok{models =}\NormalTok{ algos,}
  \AttributeTok{CV.strategy =} \StringTok{\textquotesingle{}random\textquotesingle{}}\NormalTok{,}
  \AttributeTok{CV.nb.rep =} \DecValTok{2}\NormalTok{,}
  \AttributeTok{CV.perc =} \FloatTok{0.8}\NormalTok{,}
  \AttributeTok{OPT.strategy =} \StringTok{\textquotesingle{}bigboss\textquotesingle{}}\NormalTok{,}
  \AttributeTok{metric.eval =} \FunctionTok{c}\NormalTok{(}\StringTok{\textquotesingle{}TSS\textquotesingle{}}\NormalTok{,}\StringTok{\textquotesingle{}ROC\textquotesingle{}}\NormalTok{),}
  \AttributeTok{var.import =} \DecValTok{2}\NormalTok{,}
  \AttributeTok{seed.val =} \DecValTok{42}
\NormalTok{)}
\FunctionTok{print}\NormalTok{(myBiomodModelOut)}

\CommentTok{\# Оценки и важности}
\FunctionTok{str}\NormalTok{(}\FunctionTok{get\_evaluations}\NormalTok{(myBiomodModelOut))}
\FunctionTok{str}\NormalTok{(}\FunctionTok{get\_variables\_importance}\NormalTok{(myBiomodModelOut))}
\SpecialCharTok{\textgreater{}} \FunctionTok{str}\NormalTok{(}\FunctionTok{get\_evaluations}\NormalTok{(myBiomodModelOut))}
\StringTok{\textquotesingle{}data.frame\textquotesingle{}}\SpecialCharTok{:}   \DecValTok{48}\NormalTok{ obs. of  }\DecValTok{11}\NormalTok{ variables}\SpecialCharTok{:}
 \ErrorTok{$}\NormalTok{ full.name  }\SpecialCharTok{:}\NormalTok{ chr  }\StringTok{"occ\_allData\_RUN1\_ANN"} \StringTok{"occ\_allData\_RUN1\_ANN"} \StringTok{"occ\_allData\_RUN1\_CTA"} \StringTok{"occ\_allData\_RUN1\_CTA"}\NormalTok{ ...}
 \SpecialCharTok{$}\NormalTok{ PA         }\SpecialCharTok{:}\NormalTok{ chr  }\StringTok{"allData"} \StringTok{"allData"} \StringTok{"allData"} \StringTok{"allData"}\NormalTok{ ...}
 \SpecialCharTok{$}\NormalTok{ run        }\SpecialCharTok{:}\NormalTok{ chr  }\StringTok{"RUN1"} \StringTok{"RUN1"} \StringTok{"RUN1"} \StringTok{"RUN1"}\NormalTok{ ...}
 \SpecialCharTok{$}\NormalTok{ algo       }\SpecialCharTok{:}\NormalTok{ chr  }\StringTok{"ANN"} \StringTok{"ANN"} \StringTok{"CTA"} \StringTok{"CTA"}\NormalTok{ ...}
 \SpecialCharTok{$}\NormalTok{ metric.eval}\SpecialCharTok{:}\NormalTok{ chr  }\StringTok{"TSS"} \StringTok{"ROC"} \StringTok{"TSS"} \StringTok{"ROC"}\NormalTok{ ...}
 \SpecialCharTok{$}\NormalTok{ cutoff     }\SpecialCharTok{:}\NormalTok{ num  }\DecValTok{966} \DecValTok{972} \DecValTok{421} \DecValTok{426} \DecValTok{756}\NormalTok{ ...}
 \SpecialCharTok{$}\NormalTok{ sensitivity}\SpecialCharTok{:}\NormalTok{ num  }\DecValTok{100} \DecValTok{100} \DecValTok{100} \DecValTok{100} \FloatTok{96.4}\NormalTok{ ...}
 \SpecialCharTok{$}\NormalTok{ specificity}\SpecialCharTok{:}\NormalTok{ num  }\FloatTok{98.2} \FloatTok{98.2} \FloatTok{82.5} \FloatTok{82.5} \FloatTok{94.8}\NormalTok{ ...}
 \SpecialCharTok{$}\NormalTok{ calibration}\SpecialCharTok{:}\NormalTok{ num  }\FloatTok{0.982} \FloatTok{0.987} \FloatTok{0.826} \FloatTok{0.913} \FloatTok{0.912} \FloatTok{0.977} \FloatTok{0.98} \FloatTok{0.992} \FloatTok{0.934} \FloatTok{0.979}\NormalTok{ ...}
 \SpecialCharTok{$}\NormalTok{ validation }\SpecialCharTok{:}\NormalTok{ num  }\FloatTok{0.411} \FloatTok{0.909} \FloatTok{0.677} \FloatTok{0.839} \FloatTok{0.804} \FloatTok{0.972} \FloatTok{0.694} \FloatTok{0.915} \FloatTok{0.789} \FloatTok{0.97}\NormalTok{ ...}
 \SpecialCharTok{$}\NormalTok{ evaluation }\SpecialCharTok{:}\NormalTok{ num  }\ConstantTok{NA} \ConstantTok{NA} \ConstantTok{NA} \ConstantTok{NA} \ConstantTok{NA} \ConstantTok{NA} \ConstantTok{NA} \ConstantTok{NA} \ConstantTok{NA} \ConstantTok{NA}\NormalTok{ ...}
\SpecialCharTok{\textgreater{}} \FunctionTok{str}\NormalTok{(}\FunctionTok{get\_variables\_importance}\NormalTok{(myBiomodModelOut))}
\StringTok{\textquotesingle{}data.frame\textquotesingle{}}\SpecialCharTok{:}   \DecValTok{384}\NormalTok{ obs. of  }\DecValTok{7}\NormalTok{ variables}\SpecialCharTok{:}
 \ErrorTok{$}\NormalTok{ full.name}\SpecialCharTok{:}\NormalTok{ chr  }\StringTok{"occ\_allData\_RUN1\_ANN"} \StringTok{"occ\_allData\_RUN1\_ANN"} \StringTok{"occ\_allData\_RUN1\_ANN"} \StringTok{"occ\_allData\_RUN1\_ANN"}\NormalTok{ ...}
 \SpecialCharTok{$}\NormalTok{ PA       }\SpecialCharTok{:}\NormalTok{ chr  }\StringTok{"allData"} \StringTok{"allData"} \StringTok{"allData"} \StringTok{"allData"}\NormalTok{ ...}
 \SpecialCharTok{$}\NormalTok{ run      }\SpecialCharTok{:}\NormalTok{ chr  }\StringTok{"RUN1"} \StringTok{"RUN1"} \StringTok{"RUN1"} \StringTok{"RUN1"}\NormalTok{ ...}
 \SpecialCharTok{$}\NormalTok{ algo     }\SpecialCharTok{:}\NormalTok{ chr  }\StringTok{"ANN"} \StringTok{"ANN"} \StringTok{"ANN"} \StringTok{"ANN"}\NormalTok{ ...}
 \SpecialCharTok{$}\NormalTok{ expl.var }\SpecialCharTok{:}\NormalTok{ chr  }\StringTok{"dist"} \StringTok{"chlorophyll\_range"} \StringTok{"current\_velocity\_range"} \StringTok{"diffuse\_attenuation\_range"} 
 \SpecialCharTok{$}\NormalTok{ rand     }\SpecialCharTok{:}\NormalTok{ int  }\DecValTok{1} \DecValTok{1} \DecValTok{1} \DecValTok{1} \DecValTok{1} \DecValTok{1} \DecValTok{1} \DecValTok{1} \DecValTok{2} \DecValTok{2}\NormalTok{ ...}
 \SpecialCharTok{$}\NormalTok{ var.imp  }\SpecialCharTok{:}\NormalTok{ num  }\FloatTok{0.8326} \FloatTok{0.1984} \FloatTok{0.1621} \FloatTok{0.0672} \FloatTok{0.0783}\NormalTok{ ...}
\SpecialCharTok{\textgreater{}} 

\FunctionTok{bm\_PlotEvalMean}\NormalTok{(}\AttributeTok{bm.out =}\NormalTok{ myBiomodModelOut, }\AttributeTok{dataset =} \StringTok{\textquotesingle{}calibration\textquotesingle{}}\NormalTok{)}
\end{Highlighting}
\end{Shaded}

\begin{figure}[H]

{\centering \includegraphics[width=0.8\linewidth,height=\textheight,keepaspectratio]{images/SDM11.PNG}

}

\caption{Рис. 11.: ROC vs TSS (calibration)}

\end{figure}%

\begin{Shaded}
\begin{Highlighting}[]
\FunctionTok{bm\_PlotEvalMean}\NormalTok{(}\AttributeTok{bm.out =}\NormalTok{ myBiomodModelOut, }\AttributeTok{dataset =} \StringTok{\textquotesingle{}validation\textquotesingle{}}\NormalTok{)}
\end{Highlighting}
\end{Shaded}

\begin{figure}[H]

{\centering \includegraphics[width=0.8\linewidth,height=\textheight,keepaspectratio]{images/SDM12.PNG}

}

\caption{Рис. 12.: ROC vs TSS (validation)}

\end{figure}%

\section{\texorpdfstring{\textbf{Основные
понятия}}{Основные понятия}}\label{ux43eux441ux43dux43eux432ux43dux44bux435-ux43fux43eux43dux44fux442ux438ux44f}

\textbf{ROC (Receiver Operating Characteristic)} и \textbf{TSS (True
Skill Statistic)} --- это метрики для оценки качества моделей
распределения видов, которые предсказывают, где может обитать вид (1 -
присутствует, 0 - отсутствует).

\section{\texorpdfstring{\textbf{ROC vs TSS: Калибровка
(Calibration)}}{ROC vs TSS: Калибровка (Calibration)}}\label{roc-vs-tss-ux43aux430ux43bux438ux431ux440ux43eux432ux43aux430-calibration}

\subsection{\texorpdfstring{\textbf{ROC (кривая рабочих характеристик
приемника)}}{ROC (кривая рабочих характеристик приемника)}}\label{roc-ux43aux440ux438ux432ux430ux44f-ux440ux430ux431ux43eux447ux438ux445-ux445ux430ux440ux430ux43aux442ux435ux440ux438ux441ux442ux438ux43a-ux43fux440ux438ux435ux43cux43dux438ux43aux430}

\begin{itemize}
\item
  \textbf{Что это}: График, показывающий соотношение между долей
  правильных обнаружений (True Positive Rate) и долей ложных тревог
  (False Positive Rate) при различных порогах классификации
\item
  \textbf{Как читать}:

  \begin{itemize}
  \item
    Площадь под кривой (AUC) от 0.5 до 1.0
  \item
    0.5 --- модель не лучше случайного угадывания
  \item
    0.7-0.8 --- acceptable (приемлемо)
  \item
    0.8-0.9 --- excellent (отлично)
  \item
    \begin{quote}
    0.9 --- outstanding (превосходно)
    \end{quote}
  \end{itemize}
\end{itemize}

\subsection{\texorpdfstring{\textbf{TSS (истинная статистика
навыка)}}{TSS (истинная статистика навыка)}}\label{tss-ux438ux441ux442ux438ux43dux43dux430ux44f-ux441ux442ux430ux442ux438ux441ux442ux438ux43aux430-ux43dux430ux432ux44bux43aux430}

\begin{itemize}
\item
  \textbf{Что это}: Метрика, которая учитывает как правильные
  обнаружения, так и правильные определения отсутствия
\item
  \textbf{Формула}: TSS = Sensitivity + Specificity - 1
\item
  \textbf{Диапазон}: от -1 до +1

  \begin{itemize}
  \item
    ≤0 --- модель не лучше случайной
  \item
    0.4-0.6 --- хорошая модель
  \item
    0.6-0.8 --- очень хорошая модель
  \item
    \begin{quote}
    0.8 --- отличная модель
    \end{quote}
  \end{itemize}
\end{itemize}

\subsection{\texorpdfstring{\textbf{Ключевые различия при
калибровке:}}{Ключевые различия при калибровке:}}\label{ux43aux43bux44eux447ux435ux432ux44bux435-ux440ux430ux437ux43bux438ux447ux438ux44f-ux43fux440ux438-ux43aux430ux43bux438ux431ux440ux43eux432ux43aux435}

\begin{longtable}[]{@{}
  >{\raggedright\arraybackslash}p{(\linewidth - 4\tabcolsep) * \real{0.4306}}
  >{\raggedright\arraybackslash}p{(\linewidth - 4\tabcolsep) * \real{0.2361}}
  >{\raggedright\arraybackslash}p{(\linewidth - 4\tabcolsep) * \real{0.3333}}@{}}
\toprule\noalign{}
\begin{minipage}[b]{\linewidth}\raggedright
\textbf{Критерий}
\end{minipage} & \begin{minipage}[b]{\linewidth}\raggedright
\textbf{ROC}
\end{minipage} & \begin{minipage}[b]{\linewidth}\raggedright
\textbf{TSS}
\end{minipage} \\
\midrule\noalign{}
\endhead
\bottomrule\noalign{}
\endlastfoot
\textbf{Зависимость от prevalence} & Не зависит & Не зависит \\
\textbf{Учет баланса классов} & Слабее & Сильнее \\
\textbf{Интерпретация} & Общее качество & Сбалансированная точность \\
\textbf{Чувствительность к дисбалансу} & Низкая & Средняя \\
\end{longtable}

\section{\texorpdfstring{\textbf{ROC vs TSS: Валидация
(Validation)}}{ROC vs TSS: Валидация (Validation)}}\label{roc-vs-tss-ux432ux430ux43bux438ux434ux430ux446ux438ux44f-validation}

\subsection{\texorpdfstring{\textbf{При валидации на независимых
данных:}}{При валидации на независимых данных:}}\label{ux43fux440ux438-ux432ux430ux43bux438ux434ux430ux446ux438ux438-ux43dux430-ux43dux435ux437ux430ux432ux438ux441ux438ux43cux44bux445-ux434ux430ux43dux43dux44bux445}

\textbf{ROC} чаще используется когда:

\begin{itemize}
\item
  Набор данных сильно несбалансирован
\item
  Важнее оценить общую производительность модели
\item
  Нет четкого порога классификации
\end{itemize}

\textbf{TSS} предпочтительнее когда:

\begin{itemize}
\item
  Нужно выбрать оптимальный порог классификации
\item
  Важны как правильные обнаружения, так и правильные определения
  отсутствия
\item
  Данные относительно сбалансированы
\end{itemize}

\begin{Shaded}
\begin{Highlighting}[]
\FunctionTok{bm\_PlotEvalBoxplot}\NormalTok{(}\AttributeTok{bm.out =}\NormalTok{ myBiomodModelOut, }\AttributeTok{group.by =} \FunctionTok{c}\NormalTok{(}\StringTok{\textquotesingle{}algo\textquotesingle{}}\NormalTok{, }\StringTok{\textquotesingle{}run\textquotesingle{}}\NormalTok{))}
\end{Highlighting}
\end{Shaded}

\begin{figure}[H]

{\centering \includegraphics[width=0.8\linewidth,height=\textheight,keepaspectratio]{images/SDM13.PNG}

}

\caption{Рис. 13.: Входные данные по встречаемости}

\end{figure}%

\begin{Shaded}
\begin{Highlighting}[]
\CommentTok{\# ПРОГНОЗ: текущее состояние {-}{-}{-}{-}{-}{-}{-}{-}{-}{-}{-}{-}{-}{-}{-}{-}{-}{-}{-}{-}{-}{-}{-}{-}{-}{-}{-}{-}{-}{-}{-}{-}{-}{-}{-}{-}{-}{-}{-}{-}{-}{-}{-}{-}{-}{-}{-}{-}{-}{-}{-}{-}{-}{-}{-}{-}{-}{-}{-}{-}{-}{-}{-}{-}{-}{-}{-}{-}{-}{-}{-}{-}{-}{-}{-}{-}{-}{-}{-}{-}{-}{-}{-}{-}{-}{-}{-}{-}{-}{-}{-}{-} \#}
\NormalTok{myBiomodProj }\OtherTok{\textless{}{-}} \FunctionTok{BIOMOD\_Projection}\NormalTok{(}
  \AttributeTok{bm.mod =}\NormalTok{ myBiomodModelOut,}
  \AttributeTok{proj.name =} \StringTok{\textquotesingle{}Current\textquotesingle{}}\NormalTok{,}
  \AttributeTok{new.env =}\NormalTok{ myExpl,}
  \AttributeTok{models.chosen =} \StringTok{\textquotesingle{}all\textquotesingle{}}
\NormalTok{)}

\CommentTok{\# Предсказания единичных моделей (long)}
\NormalTok{pred\_current\_single }\OtherTok{\textless{}{-}} \FunctionTok{get\_predictions}\NormalTok{(myBiomodProj)}
\FunctionTok{print}\NormalTok{(}\FunctionTok{head}\NormalTok{(pred\_current\_single))}

\CommentTok{\# АНСАМБЛИРОВАНИЕ И ПРОГНОЗ АНСАМБЛЯ {-}{-}{-}{-}{-}{-}{-}{-}{-}{-}{-}{-}{-}{-}{-}{-}{-}{-}{-}{-}{-}{-}{-}{-}{-}{-}{-}{-}{-}{-}{-}{-}{-}{-}{-}{-}{-}{-}{-}{-}{-}{-}{-}{-}{-}{-}{-}{-}{-}{-}{-}{-}{-}{-}{-}{-}{-}{-}{-}{-}{-}{-}{-}{-}{-}{-}{-}{-}{-}{-}{-}{-}{-}{-}{-}{-}{-}{-}{-}{-}{-}{-}{-} \#}
\NormalTok{myBiomodEM }\OtherTok{\textless{}{-}} \FunctionTok{BIOMOD\_EnsembleModeling}\NormalTok{(}
  \AttributeTok{bm.mod =}\NormalTok{ myBiomodModelOut,}
  \AttributeTok{models.chosen =} \StringTok{\textquotesingle{}all\textquotesingle{}}\NormalTok{,}
  \AttributeTok{em.by =} \StringTok{\textquotesingle{}all\textquotesingle{}}\NormalTok{,}
  \AttributeTok{em.algo =} \FunctionTok{c}\NormalTok{(}\StringTok{\textquotesingle{}EMmean\textquotesingle{}}\NormalTok{, }\StringTok{\textquotesingle{}EMca\textquotesingle{}}\NormalTok{),}
  \AttributeTok{metric.select =} \FunctionTok{c}\NormalTok{(}\StringTok{\textquotesingle{}TSS\textquotesingle{}}\NormalTok{),}
  \AttributeTok{metric.select.thresh =} \FunctionTok{c}\NormalTok{(}\FloatTok{0.4}\NormalTok{),}
  \AttributeTok{metric.eval =} \FunctionTok{c}\NormalTok{(}\StringTok{\textquotesingle{}TSS\textquotesingle{}}\NormalTok{, }\StringTok{\textquotesingle{}ROC\textquotesingle{}}\NormalTok{),}
  \AttributeTok{var.import =} \DecValTok{3}\NormalTok{,}
  \AttributeTok{seed.val =} \DecValTok{42}
\NormalTok{)}
\FunctionTok{str}\NormalTok{(}\FunctionTok{get\_evaluations}\NormalTok{(myBiomodEM))}
\FunctionTok{str}\NormalTok{(}\FunctionTok{get\_variables\_importance}\NormalTok{(myBiomodEM))}
\SpecialCharTok{\textgreater{}} \FunctionTok{str}\NormalTok{(}\FunctionTok{get\_evaluations}\NormalTok{(myBiomodEM))}
\StringTok{\textquotesingle{}data.frame\textquotesingle{}}\SpecialCharTok{:}   \DecValTok{4}\NormalTok{ obs. of  }\DecValTok{13}\NormalTok{ variables}\SpecialCharTok{:}
 \ErrorTok{$}\NormalTok{ full.name     }\SpecialCharTok{:}\NormalTok{ chr  }\StringTok{"occ\_EMmeanByTSS\_mergedData\_mergedRun\_mergedAlgo"} \StringTok{"occ\_EMmeanByTSS\_mergedData\_mergedRun\_mergedAlgo"} \StringTok{"occ\_EMcaByTSS\_mergedData\_mergedRun\_mergedAlgo"} \StringTok{"occ\_EMcaByTSS\_mergedData\_mergedRun\_mergedAlgo"}
 \SpecialCharTok{$}\NormalTok{ merged.by.PA  }\SpecialCharTok{:}\NormalTok{ chr  }\StringTok{"mergedData"} \StringTok{"mergedData"} \StringTok{"mergedData"} \StringTok{"mergedData"}
 \SpecialCharTok{$}\NormalTok{ merged.by.run }\SpecialCharTok{:}\NormalTok{ chr  }\StringTok{"mergedRun"} \StringTok{"mergedRun"} \StringTok{"mergedRun"} \StringTok{"mergedRun"}
 \SpecialCharTok{$}\NormalTok{ merged.by.algo}\SpecialCharTok{:}\NormalTok{ chr  }\StringTok{"mergedAlgo"} \StringTok{"mergedAlgo"} \StringTok{"mergedAlgo"} \StringTok{"mergedAlgo"}
 \SpecialCharTok{$}\NormalTok{ filtered.by   }\SpecialCharTok{:}\NormalTok{ chr  }\StringTok{"TSS"} \StringTok{"TSS"} \StringTok{"TSS"} \StringTok{"TSS"}
 \SpecialCharTok{$}\NormalTok{ algo          }\SpecialCharTok{:}\NormalTok{ chr  }\StringTok{"EMmean"} \StringTok{"EMmean"} \StringTok{"EMca"} \StringTok{"EMca"}
 \SpecialCharTok{$}\NormalTok{ metric.eval   }\SpecialCharTok{:}\NormalTok{ chr  }\StringTok{"TSS"} \StringTok{"ROC"} \StringTok{"TSS"} \StringTok{"ROC"}
 \SpecialCharTok{$}\NormalTok{ cutoff        }\SpecialCharTok{:}\NormalTok{ num  }\DecValTok{607} \DecValTok{608} \DecValTok{732} \DecValTok{730}
 \SpecialCharTok{$}\NormalTok{ sensitivity   }\SpecialCharTok{:}\NormalTok{ num  }\FloatTok{97.1} \FloatTok{97.1} \FloatTok{97.1} \FloatTok{97.1}
 \SpecialCharTok{$}\NormalTok{ specificity   }\SpecialCharTok{:}\NormalTok{ num  }\FloatTok{96.6} \FloatTok{96.6} \FloatTok{97.2} \FloatTok{97.2}
 \SpecialCharTok{$}\NormalTok{ calibration   }\SpecialCharTok{:}\NormalTok{ num  }\FloatTok{0.937} \FloatTok{0.99} \FloatTok{0.943} \FloatTok{0.994}
 \SpecialCharTok{$}\NormalTok{ validation    }\SpecialCharTok{:}\NormalTok{ num  }\ConstantTok{NA} \ConstantTok{NA} \ConstantTok{NA} \ConstantTok{NA}
 \SpecialCharTok{$}\NormalTok{ evaluation    }\SpecialCharTok{:}\NormalTok{ num  }\ConstantTok{NA} \ConstantTok{NA} \ConstantTok{NA} \ConstantTok{NA}
\SpecialCharTok{\textgreater{}} \FunctionTok{str}\NormalTok{(}\FunctionTok{get\_variables\_importance}\NormalTok{(myBiomodEM))}
\StringTok{\textquotesingle{}data.frame\textquotesingle{}}\SpecialCharTok{:}   \DecValTok{48}\NormalTok{ obs. of  }\DecValTok{9}\NormalTok{ variables}\SpecialCharTok{:}
 \ErrorTok{$}\NormalTok{ full.name     }\SpecialCharTok{:}\NormalTok{ chr  }\StringTok{"occ\_EMmeanByTSS\_mergedData\_mergedRun\_mergedAlgo"} \StringTok{"occ\_EMmeanByTSS\_mergedData\_mergedRun\_mergedAlgo"} \StringTok{"occ\_EMmeanByTSS\_mergedData\_mergedRun\_mergedAlgo"} \StringTok{"occ\_EMmeanByTSS\_mergedData\_mergedRun\_mergedAlgo"}\NormalTok{ ...}
 \SpecialCharTok{$}\NormalTok{ merged.by.PA  }\SpecialCharTok{:}\NormalTok{ chr  }\StringTok{"mergedData"} \StringTok{"mergedData"} \StringTok{"mergedData"} \StringTok{"mergedData"}\NormalTok{ ...}
 \SpecialCharTok{$}\NormalTok{ merged.by.run }\SpecialCharTok{:}\NormalTok{ chr  }\StringTok{"mergedRun"} \StringTok{"mergedRun"} \StringTok{"mergedRun"} \StringTok{"mergedRun"}\NormalTok{ ...}
 \SpecialCharTok{$}\NormalTok{ merged.by.algo}\SpecialCharTok{:}\NormalTok{ chr  }\StringTok{"mergedAlgo"} \StringTok{"mergedAlgo"} \StringTok{"mergedAlgo"} \StringTok{"mergedAlgo"}\NormalTok{ ...}
 \SpecialCharTok{$}\NormalTok{ filtered.by   }\SpecialCharTok{:}\NormalTok{ chr  }\StringTok{"TSS"} \StringTok{"TSS"} \StringTok{"TSS"} \StringTok{"TSS"}\NormalTok{ ...}
 \SpecialCharTok{$}\NormalTok{ algo          }\SpecialCharTok{:}\NormalTok{ chr  }\StringTok{"EMmean"} \StringTok{"EMmean"} \StringTok{"EMmean"} \StringTok{"EMmean"}\NormalTok{ ...}
 \SpecialCharTok{$}\NormalTok{ expl.var      }\SpecialCharTok{:}\NormalTok{ chr  }\StringTok{"dist"} \StringTok{"chlorophyll\_range"} \StringTok{"current\_velocity\_range"} \StringTok{"diffuse\_attenuation\_range"}\NormalTok{ ...}
 \SpecialCharTok{$}\NormalTok{ rand          }\SpecialCharTok{:}\NormalTok{ int  }\DecValTok{1} \DecValTok{1} \DecValTok{1} \DecValTok{1} \DecValTok{1} \DecValTok{1} \DecValTok{1} \DecValTok{1} \DecValTok{2} \DecValTok{2}\NormalTok{ ...}
 \SpecialCharTok{$}\NormalTok{ var.imp       }\SpecialCharTok{:}\NormalTok{ num  }\FloatTok{0.74975} \FloatTok{0.01052} \FloatTok{0.01369} \FloatTok{0.00352} \FloatTok{0.01036}\NormalTok{ ...}
\SpecialCharTok{\textgreater{}} 
\NormalTok{myBiomodEMProj }\OtherTok{\textless{}{-}} \FunctionTok{BIOMOD\_EnsembleForecasting}\NormalTok{(}
  \AttributeTok{bm.em =}\NormalTok{ myBiomodEM,}
  \AttributeTok{bm.proj =}\NormalTok{ myBiomodProj,}
  \AttributeTok{models.chosen =} \StringTok{\textquotesingle{}all\textquotesingle{}}\NormalTok{,}
  \AttributeTok{metric.binary =} \StringTok{\textquotesingle{}all\textquotesingle{}}\NormalTok{,}
  \AttributeTok{metric.filter =} \StringTok{\textquotesingle{}all\textquotesingle{}}
\NormalTok{)}

\NormalTok{pred\_current\_em }\OtherTok{\textless{}{-}} \FunctionTok{get\_predictions}\NormalTok{(myBiomodEMProj)}
\NormalTok{pred\_current\_emmean }\OtherTok{\textless{}{-}}\NormalTok{ dplyr}\SpecialCharTok{::}\FunctionTok{filter}\NormalTok{(pred\_current\_em, .data}\SpecialCharTok{$}\NormalTok{algo }\SpecialCharTok{==} \StringTok{"EMmean"}\NormalTok{)}
\FunctionTok{print}\NormalTok{(}\FunctionTok{head}\NormalTok{(pred\_current\_emmean))}

\CommentTok{\# КАРТЫ: текущий период {-}{-}{-}{-}{-}{-}{-}{-}{-}{-}{-}{-}{-}{-}{-}{-}{-}{-}{-}{-}{-}{-}{-}{-}{-}{-}{-}{-}{-}{-}{-}{-}{-}{-}{-}{-}{-}{-}{-}{-}{-}{-}{-}{-}{-}{-}{-}{-}{-}{-}{-}{-}{-}{-}{-}{-}{-}{-}{-}{-}{-}{-}{-}{-}{-}{-}{-}{-}{-}{-}{-}{-}{-}{-}{-}{-}{-}{-}{-}{-}{-}{-}{-}{-}{-}{-}{-}{-}{-}{-}{-}{-}{-}{-}{-}{-} \#}
\NormalTok{MAPDATA }\OtherTok{\textless{}{-}} \FunctionTok{tibble}\NormalTok{(}
  \AttributeTok{point\_id =} \FunctionTok{seq\_len}\NormalTok{(}\FunctionTok{nrow}\NormalTok{(DATA)),}
  \AttributeTok{X =}\NormalTok{ DATA}\SpecialCharTok{$}\NormalTok{x,}
  \AttributeTok{Y =}\NormalTok{ DATA}\SpecialCharTok{$}\NormalTok{y,}
  \AttributeTok{PRED =}\NormalTok{ pred\_current\_emmean}\SpecialCharTok{$}\NormalTok{pred}
\NormalTok{)}

\NormalTok{world }\OtherTok{\textless{}{-}}\NormalTok{ rnaturalearth}\SpecialCharTok{::}\FunctionTok{ne\_countries}\NormalTok{(}\AttributeTok{scale =} \DecValTok{50}\NormalTok{, }\AttributeTok{returnclass =} \StringTok{\textquotesingle{}sf\textquotesingle{}}\NormalTok{)}
\NormalTok{xmin }\OtherTok{\textless{}{-}} \DecValTok{10}\NormalTok{; xmax }\OtherTok{\textless{}{-}} \DecValTok{45}\NormalTok{; ymin }\OtherTok{\textless{}{-}} \DecValTok{66}\NormalTok{; ymax }\OtherTok{\textless{}{-}} \DecValTok{72}

\NormalTok{bat }\OtherTok{\textless{}{-}} \FunctionTok{tryCatch}\NormalTok{(}\FunctionTok{getNOAA.bathy}\NormalTok{(xmin, xmax, ymin, ymax, }\AttributeTok{resolution =} \DecValTok{4}\NormalTok{), }\AttributeTok{error =} \ControlFlowTok{function}\NormalTok{(e) }\ConstantTok{NULL}\NormalTok{)}
\NormalTok{bat\_xyz }\OtherTok{\textless{}{-}} \ControlFlowTok{if}\NormalTok{ (}\SpecialCharTok{!}\FunctionTok{is.null}\NormalTok{(bat)) }\FunctionTok{as.xyz}\NormalTok{(bat) }\ControlFlowTok{else} \ConstantTok{NULL}

\NormalTok{p\_points }\OtherTok{\textless{}{-}} \FunctionTok{ggplot}\NormalTok{() }\SpecialCharTok{+}
  \FunctionTok{geom\_sf}\NormalTok{(}\AttributeTok{data =}\NormalTok{ world) }\SpecialCharTok{+}
  \FunctionTok{coord\_sf}\NormalTok{(}\AttributeTok{xlim =} \FunctionTok{c}\NormalTok{(xmin, xmax), }\AttributeTok{ylim =} \FunctionTok{c}\NormalTok{(ymin, ymax)) }\SpecialCharTok{+}
\NormalTok{  \{}\ControlFlowTok{if}\NormalTok{ (}\SpecialCharTok{!}\FunctionTok{is.null}\NormalTok{(bat\_xyz)) }\FunctionTok{geom\_tile}\NormalTok{(}\AttributeTok{data =}\NormalTok{ bat\_xyz, }\FunctionTok{aes}\NormalTok{(}\AttributeTok{x =}\NormalTok{ V1, }\AttributeTok{y =}\NormalTok{ V2, }\AttributeTok{fill =}\NormalTok{ V3), }\AttributeTok{show.legend =} \ConstantTok{FALSE}\NormalTok{)\} }\SpecialCharTok{+}
  \FunctionTok{geom\_point}\NormalTok{(}\AttributeTok{data =}\NormalTok{ MAPDATA, }\FunctionTok{aes}\NormalTok{(}\AttributeTok{x =}\NormalTok{ X, }\AttributeTok{y =}\NormalTok{ Y, }\AttributeTok{size =}\NormalTok{ PRED), }\AttributeTok{color =} \StringTok{"black"}\NormalTok{, }\AttributeTok{fill =} \StringTok{"white"}\NormalTok{, }\AttributeTok{shape =} \DecValTok{21}\NormalTok{, }\AttributeTok{alpha=}\FloatTok{0.8}\NormalTok{) }\SpecialCharTok{+}
\NormalTok{  ggspatial}\SpecialCharTok{::}\FunctionTok{annotation\_scale}\NormalTok{(}\AttributeTok{location =} \StringTok{"tr"}\NormalTok{, }\AttributeTok{width\_hint =} \FloatTok{0.5}\NormalTok{) }\SpecialCharTok{+}
  \FunctionTok{scale\_size}\NormalTok{(}\AttributeTok{name =} \StringTok{"Вероятность"}\NormalTok{, }\AttributeTok{range =} \FunctionTok{c}\NormalTok{(}\DecValTok{1}\NormalTok{, }\DecValTok{5}\NormalTok{)) }\SpecialCharTok{+}
  \FunctionTok{labs}\NormalTok{(}\AttributeTok{title =} \StringTok{"Точки: интенсивность предсказания (Current)"}\NormalTok{)}

\NormalTok{p\_raster }\OtherTok{\textless{}{-}} \FunctionTok{ggplot}\NormalTok{() }\SpecialCharTok{+}
  \FunctionTok{geom\_raster}\NormalTok{(}\AttributeTok{data =}\NormalTok{ MAPDATA, }\FunctionTok{aes}\NormalTok{(}\AttributeTok{x =}\NormalTok{ X, }\AttributeTok{y =}\NormalTok{ Y, }\AttributeTok{fill =}\NormalTok{ PRED), }\AttributeTok{interpolate =} \ConstantTok{FALSE}\NormalTok{) }\SpecialCharTok{+}
  \FunctionTok{scale\_fill\_viridis\_c}\NormalTok{(}\AttributeTok{option =} \StringTok{"D"}\NormalTok{, }\AttributeTok{name =} \StringTok{"PRED"}\NormalTok{) }\SpecialCharTok{+}
  \FunctionTok{geom\_sf}\NormalTok{(}\AttributeTok{data =}\NormalTok{ world, }\AttributeTok{color =} \StringTok{"gray30"}\NormalTok{, }\AttributeTok{fill =} \StringTok{"\#E8E5D6"}\NormalTok{, }\AttributeTok{lwd =} \FloatTok{0.3}\NormalTok{) }\SpecialCharTok{+}
  \FunctionTok{coord\_sf}\NormalTok{(}\AttributeTok{xlim =} \FunctionTok{c}\NormalTok{(xmin}\SpecialCharTok{*}\FloatTok{1.2}\NormalTok{, xmax}\SpecialCharTok{*}\FloatTok{0.96}\NormalTok{), }\AttributeTok{ylim =} \FunctionTok{c}\NormalTok{(ymin}\SpecialCharTok{*}\FloatTok{1.02}\NormalTok{, ymax}\SpecialCharTok{*}\FloatTok{0.99}\NormalTok{)) }\SpecialCharTok{+}
  \FunctionTok{labs}\NormalTok{(}\AttributeTok{title =} \StringTok{"Растер: предсказание EMmean (Current)"}\NormalTok{)}

\FunctionTok{print}\NormalTok{(p\_points)}
\FunctionTok{print}\NormalTok{(p\_raster)}
\end{Highlighting}
\end{Shaded}

\begin{figure}[H]

{\centering \includegraphics[width=0.8\linewidth,height=\textheight,keepaspectratio]{images/SDM14.PNG}

}

\caption{Рис. 14.: Визуалицация SDM - индекс пригодности среды или
вероятность встречаемости вида}

\end{figure}%

\begin{Shaded}
\begin{Highlighting}[]
\CommentTok{\# ДИАГНОСТИКА НАДЕЖНОСТИ: текущее {-}{-}{-}{-}{-}{-}{-}{-}{-}{-}{-}{-}{-}{-}{-}{-}{-}{-}{-}{-}{-}{-}{-}{-}{-}{-}{-}{-}{-}{-}{-}{-}{-}{-}{-}{-}{-}{-}{-}{-}{-}{-}{-}{-}{-}{-}{-}{-}{-}{-}{-}{-}{-}{-}{-}{-}{-}{-}{-}{-}{-}{-}{-}{-}{-}{-}{-}{-}{-}{-}{-}{-}{-}{-}{-}{-}{-}{-}{-}{-}{-}{-}{-}{-}{-}{-} \#}
\NormalTok{labels }\OtherTok{\textless{}{-}} \FunctionTok{as.integer}\NormalTok{(myResp)}
\NormalTok{probs\_current\_01 }\OtherTok{\textless{}{-}} \FunctionTok{scale\_predictions\_01}\NormalTok{(pred\_current\_emmean}\SpecialCharTok{$}\NormalTok{pred)}

\NormalTok{calib }\OtherTok{\textless{}{-}} \FunctionTok{calibration\_table}\NormalTok{(labels, probs\_current\_01, }\AttributeTok{num\_bins =} \DecValTok{10}\NormalTok{)}
\FunctionTok{print}\NormalTok{(calib}\SpecialCharTok{$}\NormalTok{table)}
\FunctionTok{print}\NormalTok{(}\FunctionTok{plot\_calibration}\NormalTok{(calib}\SpecialCharTok{$}\NormalTok{table) }\SpecialCharTok{+} \FunctionTok{labs}\NormalTok{(}\AttributeTok{subtitle =} \FunctionTok{sprintf}\NormalTok{(}\StringTok{"Brier = \%.3f, ECE = \%.3f"}\NormalTok{, calib}\SpecialCharTok{$}\NormalTok{brier, calib}\SpecialCharTok{$}\NormalTok{ece)))}
\end{Highlighting}
\end{Shaded}

\begin{figure}[H]

{\centering \includegraphics[width=0.8\linewidth,height=\textheight,keepaspectratio]{images/SDM15.PNG}

}

\caption{Рис. 15.: График калибровки}

\end{figure}%

График калибровки показывает соответствие между предсказанными
вероятностями модели и фактической частотой наблюдений. По оси X
откладывается средняя предсказанная вероятность в каждом интервале, то
есть то, что модель предполагает, а по оси Y --- фактическая доля
наблюдений, то что происходит в реальности. Идеальная калибровка
represented by диагональная линия, где предсказания полностью совпадают
с реальностью. Если точки графика лежат выше диагонали, это означает что
модель недооценивает вероятность события --- она предсказывает меньшую
вероятность чем фактическая частота. Если точки ниже диагонали ---
модель переоценивает вероятность, giving завышенные прогнозы. В
подзаголовке указаны две ключевые метрики: оценка Брайера и ожидаемая
ошибка калибровки. Оценка Брайера измеряет среднюю квадратичную ошибку
предсказаний и колеблется от 0 до 1, где 0 означает идеальную
калибровку, а значения ниже 0.1 считаются хорошими. Ожидаемая ошибка
калибровки также стремится к нулю при идеальной калибровке и показывает
среднее отклонение от идеального соответствия. Этот анализ позволяет
оценить надежность вероятностных выводов модели и необходимость
дополнительной калибровки для улучшения прогнозов, что особенно важно в
экологических исследованиях где точность прогнозов влияет на принятие
решений по охране видов и управлению биоресурсами.

\begin{Shaded}
\begin{Highlighting}[]
\NormalTok{rp }\OtherTok{\textless{}{-}} \FunctionTok{roc\_pr\_metrics}\NormalTok{(labels, probs\_current\_01)}
\FunctionTok{print}\NormalTok{(}\FunctionTok{plot\_roc\_curve}\NormalTok{(rp}\SpecialCharTok{$}\NormalTok{roc) }\SpecialCharTok{+} \FunctionTok{labs}\NormalTok{(}\AttributeTok{subtitle =} \FunctionTok{sprintf}\NormalTok{(}\StringTok{"AUC = \%.3f"}\NormalTok{, rp}\SpecialCharTok{$}\NormalTok{auc\_roc)))}
\FunctionTok{print}\NormalTok{(}\FunctionTok{plot\_pr\_curve}\NormalTok{(rp}\SpecialCharTok{$}\NormalTok{pr) }\SpecialCharTok{+} \FunctionTok{labs}\NormalTok{(}\AttributeTok{subtitle =} \FunctionTok{sprintf}\NormalTok{(}\StringTok{"AUPRC = \%.3f"}\NormalTok{, rp}\SpecialCharTok{$}\NormalTok{auc\_pr)))}
\end{Highlighting}
\end{Shaded}

\begin{figure}[H]

{\centering \includegraphics[width=0.8\linewidth,height=\textheight,keepaspectratio]{images/SDM16.PNG}

}

\caption{Рис. 16.: График ROC-кривой и Precision-Recall}

\end{figure}%

График ROC-кривой показывает соотношение между долей правильных
обнаружений и долей ложных срабатываний при различных порогах
классификации где идеальная модель стремится к левому верхнему углу что
означает максимальную чувствительность при минимальных ложных тревогах.
Площадь под кривой AUC количественно измеряет качество классификации где
значение 0.5 соответствует случайному угадыванию а значение 1.0
представляет собой идеальное разделение классов при этом в экологических
исследованиях значения выше 0.7 считаются приемлемыми а выше 0.8 ---
хорошими. График Precision-Recall демонстрирует компромисс между
точностью предсказаний и полнотой охвата где высокая точность означает
минимум ложных обнаружений а высокая полнота указывает на способность
модели найти все реальные случаи присутствия вида. Площадь под PR-кривой
AUPRC особенно важна при работе с несбалансированными данными так как
она игнорирует правильно предсказанные отсутствия и фокусируется на
качестве предсказания присутствий где значение 0.5 соответствует
базовому уровню а значения близкие к 1.0 указывают на отличное качество
модели. Вместе эти метрики обеспечивают комплексную оценку
производительности модели учитывая как способность различать классы так
и надежность предсказаний положительных случаев что критически важно для
принятия решений в задачах экологического моделирования и
прогнозирования распределения видов.

\begin{Shaded}
\begin{Highlighting}[]
\NormalTok{opt\_thr }\OtherTok{\textless{}{-}} \FunctionTok{optimal\_threshold\_tss}\NormalTok{(labels, probs\_current\_01, }\AttributeTok{step =} \FloatTok{0.005}\NormalTok{)}
\FunctionTok{print}\NormalTok{(opt\_thr)}

\NormalTok{boy }\OtherTok{\textless{}{-}} \FunctionTok{boyce\_index}\NormalTok{(labels, probs\_current\_01)}
\FunctionTok{print}\NormalTok{(}\FunctionTok{tibble}\NormalTok{(}\AttributeTok{metric =} \StringTok{"Continuous Boyce Index"}\NormalTok{, }\AttributeTok{value =}\NormalTok{ boy}\SpecialCharTok{$}\NormalTok{CBI))}

\CommentTok{\# НЕОПРЕДЕЛЕННОСТЬ МЕЖДУ АЛГОРИТМАМИ (SD, CV) {-}{-}{-}{-}{-}{-}{-}{-}{-}{-}{-}{-}{-}{-}{-}{-}{-}{-}{-}{-}{-}{-}{-}{-}{-}{-}{-}{-}{-}{-}{-}{-}{-}{-}{-}{-}{-}{-}{-}{-}{-}{-}{-}{-}{-}{-}{-}{-}{-}{-}{-}{-}{-}{-}{-}{-}{-}{-}{-}{-}{-}{-}{-}{-}{-}{-}{-}{-}{-}{-}{-}{-}{-}{-} \#}
\NormalTok{unc }\OtherTok{\textless{}{-}} \FunctionTok{uncertainty\_per\_point}\NormalTok{(pred\_current\_single)}
\NormalTok{MAPDATA\_unc }\OtherTok{\textless{}{-}}\NormalTok{ MAPDATA }\SpecialCharTok{\%\textgreater{}\%} \FunctionTok{left\_join}\NormalTok{(unc }\SpecialCharTok{\%\textgreater{}\%} \FunctionTok{transmute}\NormalTok{(}\AttributeTok{point\_id =}\NormalTok{ points, UNC\_SD, UNC\_CV), }\AttributeTok{by =} \StringTok{"point\_id"}\NormalTok{)}

\NormalTok{p\_unc\_sd }\OtherTok{\textless{}{-}} \FunctionTok{ggplot}\NormalTok{(MAPDATA\_unc, }\FunctionTok{aes}\NormalTok{(}\AttributeTok{x =}\NormalTok{ X, }\AttributeTok{y =}\NormalTok{ Y, }\AttributeTok{fill =}\NormalTok{ UNC\_SD)) }\SpecialCharTok{+}
  \FunctionTok{geom\_raster}\NormalTok{() }\SpecialCharTok{+}
  \FunctionTok{scale\_fill\_viridis\_c}\NormalTok{(}\AttributeTok{option =} \StringTok{"C"}\NormalTok{, }\AttributeTok{name =} \StringTok{"SD"}\NormalTok{) }\SpecialCharTok{+}
  \FunctionTok{coord\_equal}\NormalTok{() }\SpecialCharTok{+}
  \FunctionTok{labs}\NormalTok{(}\AttributeTok{title =} \StringTok{"Неопределенность (SD) между алгоритмами — Current"}\NormalTok{)}

\FunctionTok{print}\NormalTok{(p\_unc\_sd)}
\end{Highlighting}
\end{Shaded}

\begin{figure}[H]

{\centering \includegraphics[width=0.8\linewidth,height=\textheight,keepaspectratio]{images/SDM17.PNG}

}

\caption{Рис. 17.: Неопределенность (SD) между алгоритмами --- Current}

\end{figure}%

\begin{Shaded}
\begin{Highlighting}[]
\CommentTok{\# БУДУЩЕЕ: данные, прогноз и диагностика {-}{-}{-}{-}{-}{-}{-}{-}{-}{-}{-}{-}{-}{-}{-}{-}{-}{-}{-}{-}{-}{-}{-}{-}{-}{-}{-}{-}{-}{-}{-}{-}{-}{-}{-}{-}{-}{-}{-}{-}{-}{-}{-}{-}{-}{-}{-}{-}{-}{-}{-}{-}{-}{-}{-}{-}{-}{-}{-}{-}{-}{-}{-}{-}{-}{-}{-}{-}{-}{-}{-}{-}{-}{-}{-}{-}{-}{-}{-}{-} \#}
\NormalTok{DATA\_F }\OtherTok{\textless{}{-}} \FunctionTok{read.csv}\NormalTok{(}\StringTok{"future\_sdm\_table\_with\_na.csv"}\NormalTok{)}
\FunctionTok{str}\NormalTok{(DATA\_F)}
\NormalTok{DataSpeciesF }\OtherTok{\textless{}{-}} \FunctionTok{as.data.frame}\NormalTok{(DATA\_F)}
\NormalTok{myRespF }\OtherTok{\textless{}{-}} \FunctionTok{as.numeric}\NormalTok{(DataSpeciesF[[myRespName]])}
\NormalTok{myRespXYF }\OtherTok{\textless{}{-}}\NormalTok{ DataSpeciesF[, }\FunctionTok{c}\NormalTok{(}\StringTok{"x"}\NormalTok{, }\StringTok{"y"}\NormalTok{)]}
\NormalTok{myExplP1 }\OtherTok{\textless{}{-}}\NormalTok{ DataSpeciesF[, }\DecValTok{4}\SpecialCharTok{:}\DecValTok{11}\NormalTok{]}

\NormalTok{myBiomodProj1 }\OtherTok{\textless{}{-}} \FunctionTok{BIOMOD\_Projection}\NormalTok{(}
  \AttributeTok{bm.mod =}\NormalTok{ myBiomodModelOut,}
  \AttributeTok{proj.name =} \StringTok{\textquotesingle{}Future\textquotesingle{}}\NormalTok{,}
  \AttributeTok{new.env =}\NormalTok{ myExplP1,}
  \AttributeTok{models.chosen =} \StringTok{\textquotesingle{}all\textquotesingle{}}
\NormalTok{)}

\NormalTok{myBiomodEMProj1 }\OtherTok{\textless{}{-}} \FunctionTok{BIOMOD\_EnsembleForecasting}\NormalTok{(}
  \AttributeTok{bm.em =}\NormalTok{ myBiomodEM,}
  \AttributeTok{bm.proj =}\NormalTok{ myBiomodProj1,}
  \AttributeTok{models.chosen =} \StringTok{\textquotesingle{}all\textquotesingle{}}\NormalTok{,}
  \AttributeTok{metric.binary =} \StringTok{\textquotesingle{}all\textquotesingle{}}\NormalTok{,}
  \AttributeTok{metric.filter =} \StringTok{\textquotesingle{}all\textquotesingle{}}
\NormalTok{)}

\NormalTok{pred\_future\_em }\OtherTok{\textless{}{-}} \FunctionTok{get\_predictions}\NormalTok{(myBiomodEMProj1)}
\NormalTok{pred\_future\_emmean }\OtherTok{\textless{}{-}}\NormalTok{ dplyr}\SpecialCharTok{::}\FunctionTok{filter}\NormalTok{(pred\_future\_em, .data}\SpecialCharTok{$}\NormalTok{algo }\SpecialCharTok{==} \StringTok{"EMmean"}\NormalTok{)}

\NormalTok{MAPDATA2 }\OtherTok{\textless{}{-}} \FunctionTok{tibble}\NormalTok{(}
  \AttributeTok{point\_id =} \FunctionTok{seq\_len}\NormalTok{(}\FunctionTok{nrow}\NormalTok{(DATA\_F)),}
  \AttributeTok{X =}\NormalTok{ DATA\_F}\SpecialCharTok{$}\NormalTok{x,}
  \AttributeTok{Y =}\NormalTok{ DATA\_F}\SpecialCharTok{$}\NormalTok{y,}
  \AttributeTok{PRED =}\NormalTok{ pred\_future\_emmean}\SpecialCharTok{$}\NormalTok{pred}
\NormalTok{)}

\CommentTok{\# Карты: будущее и Δ {-}{-}{-}{-}{-}{-}{-}{-}{-}{-}{-}{-}{-}{-}{-}{-}{-}{-}{-}{-}{-}{-}{-}{-}{-}{-}{-}{-}{-}{-}{-}{-}{-}{-}{-}{-}{-}{-}{-}{-}{-}{-}{-}{-}{-}{-}{-}{-}{-}{-}{-}{-}{-}{-}{-}{-}{-}{-}{-}{-}{-}{-}{-}{-}{-}{-}{-}{-}{-}{-}{-}{-}{-}{-}{-}{-}{-}{-}{-}{-}{-}{-}{-}{-}{-}{-}{-}{-}{-}{-}{-}{-}{-}{-}{-}{-}{-}{-}{-}{-} \#}
\NormalTok{p\_points2 }\OtherTok{\textless{}{-}} \FunctionTok{ggplot}\NormalTok{() }\SpecialCharTok{+}
  \FunctionTok{geom\_sf}\NormalTok{(}\AttributeTok{data =}\NormalTok{ world) }\SpecialCharTok{+}
  \FunctionTok{coord\_sf}\NormalTok{(}\AttributeTok{xlim =} \FunctionTok{c}\NormalTok{(xmin, xmax), }\AttributeTok{ylim =} \FunctionTok{c}\NormalTok{(ymin, ymax)) }\SpecialCharTok{+}
\NormalTok{  \{}\ControlFlowTok{if}\NormalTok{ (}\SpecialCharTok{!}\FunctionTok{is.null}\NormalTok{(bat\_xyz)) }\FunctionTok{geom\_tile}\NormalTok{(}\AttributeTok{data =}\NormalTok{ bat\_xyz, }\FunctionTok{aes}\NormalTok{(}\AttributeTok{x =}\NormalTok{ V1, }\AttributeTok{y =}\NormalTok{ V2, }\AttributeTok{fill =}\NormalTok{ V3), }\AttributeTok{show.legend =} \ConstantTok{FALSE}\NormalTok{)\} }\SpecialCharTok{+}
  \FunctionTok{geom\_point}\NormalTok{(}\AttributeTok{data =}\NormalTok{ MAPDATA2, }\FunctionTok{aes}\NormalTok{(}\AttributeTok{x =}\NormalTok{ X, }\AttributeTok{y =}\NormalTok{ Y, }\AttributeTok{size =}\NormalTok{ PRED), }\AttributeTok{color =} \StringTok{"black"}\NormalTok{, }\AttributeTok{fill =} \StringTok{"white"}\NormalTok{, }\AttributeTok{shape =} \DecValTok{21}\NormalTok{, }\AttributeTok{alpha=}\FloatTok{0.8}\NormalTok{) }\SpecialCharTok{+}
\NormalTok{  ggspatial}\SpecialCharTok{::}\FunctionTok{annotation\_scale}\NormalTok{(}\AttributeTok{location =} \StringTok{"tr"}\NormalTok{, }\AttributeTok{width\_hint =} \FloatTok{0.5}\NormalTok{) }\SpecialCharTok{+}
  \FunctionTok{scale\_size}\NormalTok{(}\AttributeTok{name =} \StringTok{"Вероятность"}\NormalTok{, }\AttributeTok{range =} \FunctionTok{c}\NormalTok{(}\DecValTok{1}\NormalTok{, }\DecValTok{5}\NormalTok{)) }\SpecialCharTok{+}
  \FunctionTok{labs}\NormalTok{(}\AttributeTok{title =} \StringTok{"Точки: EMmean (Future)"}\NormalTok{)}

\NormalTok{p\_raster2 }\OtherTok{\textless{}{-}} \FunctionTok{ggplot}\NormalTok{() }\SpecialCharTok{+}
  \FunctionTok{geom\_raster}\NormalTok{(}\AttributeTok{data =}\NormalTok{ MAPDATA2, }\FunctionTok{aes}\NormalTok{(}\AttributeTok{x =}\NormalTok{ X, }\AttributeTok{y =}\NormalTok{ Y, }\AttributeTok{fill =}\NormalTok{ PRED)) }\SpecialCharTok{+}
  \FunctionTok{scale\_fill\_viridis\_c}\NormalTok{(}\AttributeTok{option =} \StringTok{"D"}\NormalTok{, }\AttributeTok{name =} \StringTok{"PRED"}\NormalTok{) }\SpecialCharTok{+}
  \FunctionTok{geom\_sf}\NormalTok{(}\AttributeTok{data =}\NormalTok{ world, }\AttributeTok{color =} \StringTok{"gray30"}\NormalTok{, }\AttributeTok{fill =} \StringTok{"\#E8E5D6"}\NormalTok{, }\AttributeTok{lwd =} \FloatTok{0.3}\NormalTok{) }\SpecialCharTok{+}
  \FunctionTok{coord\_sf}\NormalTok{(}\AttributeTok{xlim =} \FunctionTok{c}\NormalTok{(xmin}\SpecialCharTok{*}\FloatTok{1.2}\NormalTok{, xmax}\SpecialCharTok{*}\FloatTok{0.96}\NormalTok{), }\AttributeTok{ylim =} \FunctionTok{c}\NormalTok{(ymin}\SpecialCharTok{*}\FloatTok{1.01}\NormalTok{, ymax}\SpecialCharTok{*}\FloatTok{0.99}\NormalTok{)) }\SpecialCharTok{+}
  \FunctionTok{labs}\NormalTok{(}\AttributeTok{title =} \StringTok{"Растер: EMmean (Future)"}\NormalTok{)}

\NormalTok{probs\_future\_01 }\OtherTok{\textless{}{-}} \FunctionTok{scale\_predictions\_01}\NormalTok{(pred\_future\_emmean}\SpecialCharTok{$}\NormalTok{pred)}
\NormalTok{probs\_current\_01 }\OtherTok{\textless{}{-}} \FunctionTok{scale\_predictions\_01}\NormalTok{(pred\_current\_emmean}\SpecialCharTok{$}\NormalTok{pred)}
\NormalTok{MAPDATA\_delta }\OtherTok{\textless{}{-}} \FunctionTok{tibble}\NormalTok{(}\AttributeTok{X =}\NormalTok{ MAPDATA}\SpecialCharTok{$}\NormalTok{X, }\AttributeTok{Y =}\NormalTok{ MAPDATA}\SpecialCharTok{$}\NormalTok{Y, }\AttributeTok{delta =}\NormalTok{ probs\_future\_01 }\SpecialCharTok{{-}}\NormalTok{ probs\_current\_01)}

\NormalTok{p\_delta }\OtherTok{\textless{}{-}} \FunctionTok{ggplot}\NormalTok{(MAPDATA\_delta, }\FunctionTok{aes}\NormalTok{(}\AttributeTok{x =}\NormalTok{ X, }\AttributeTok{y =}\NormalTok{ Y, }\AttributeTok{fill =}\NormalTok{ delta)) }\SpecialCharTok{+}
  \FunctionTok{geom\_raster}\NormalTok{() }\SpecialCharTok{+}
  \FunctionTok{scale\_fill\_gradient2}\NormalTok{(}\AttributeTok{low =} \StringTok{"\#D7301F"}\NormalTok{, }\AttributeTok{mid =} \StringTok{"\#FFFFBF"}\NormalTok{, }\AttributeTok{high =} \StringTok{"\#1A9850"}\NormalTok{, }\AttributeTok{midpoint =} \DecValTok{0}\NormalTok{, }\AttributeTok{name =} \StringTok{"Delta Prob"}\NormalTok{) }\SpecialCharTok{+}
  \FunctionTok{coord\_equal}\NormalTok{() }\SpecialCharTok{+}
  \FunctionTok{labs}\NormalTok{(}\AttributeTok{title =} \StringTok{"Delta (Future − Current) EMmean (scaled)"}\NormalTok{)}

\FunctionTok{print}\NormalTok{(p\_points2)}
\FunctionTok{print}\NormalTok{(p\_raster2)}
\end{Highlighting}
\end{Shaded}

\begin{figure}[H]

{\centering \includegraphics[width=0.8\linewidth,height=\textheight,keepaspectratio]{images/SDM18.PNG}

}

\caption{Рис. 18.: Вероятность встречаемости большого гребешка в 2100
г.}

\end{figure}%

\begin{Shaded}
\begin{Highlighting}[]
\FunctionTok{print}\NormalTok{(p\_delta)}
\end{Highlighting}
\end{Shaded}

\begin{center}
\includegraphics[width=0.8\linewidth,height=\textheight,keepaspectratio]{images/SDM19.PNG}
\end{center}
Это карта изменений (дельты) в распределении вероятности присутствия
вида между будущим и текущим сценарием. На карте визуализируется разница
между прогнозируемой вероятностью в будущем и текущей вероятностью где
каждый пиксель показывает изменение вероятности от отрицательных
значений уменьшение вероятности до положительных значений увеличение
вероятности. Красным цветом обозначены области где вероятность
присутствия вида уменьшится в будущем что может указывать на
неблагоприятные условия для вида в этих районах. Зеленым цветом показаны
области где вероятность присутствия вида увеличится что свидетельствует
о потенциальном расширении ареала или улучшении условий. Желтые и
близкие к нейтральным участки отражают незначительные изменения или
стабильность в распределении вероятностей. Эта карта позволяет быстро
оценить общую тенденцию изменений в распределении вида выявить наиболее
уязвимые регионы где вид может исчезнуть и перспективные территории где
вид может появиться или увеличить свою численность что критически важно
для планирования мер охраны и прогнозирования изменений биоразнообразия.

\begin{Shaded}
\begin{Highlighting}[]
\CommentTok{\# Неопределенность (будущее) и MESS {-}{-}{-}{-}{-}{-}{-}{-}{-}{-}{-}{-}{-}{-}{-}{-}{-}{-}{-}{-}{-}{-}{-}{-}{-}{-}{-}{-}{-}{-}{-}{-}{-}{-}{-}{-}{-}{-}{-}{-}{-}{-}{-}{-}{-}{-}{-}{-}{-}{-}{-}{-}{-}{-}{-}{-}{-}{-}{-}{-}{-}{-}{-}{-}{-}{-}{-}{-}{-}{-}{-}{-}{-}{-}{-}{-}{-}{-}{-}{-}{-}{-}{-}{-} \#}
\NormalTok{pred\_future\_single }\OtherTok{\textless{}{-}} \FunctionTok{get\_predictions}\NormalTok{(myBiomodProj1)}
\NormalTok{unc\_f }\OtherTok{\textless{}{-}} \FunctionTok{uncertainty\_per\_point}\NormalTok{(pred\_future\_single)}
\NormalTok{MAPDATA2\_unc }\OtherTok{\textless{}{-}}\NormalTok{ MAPDATA2 }\SpecialCharTok{\%\textgreater{}\%} \FunctionTok{left\_join}\NormalTok{(unc\_f }\SpecialCharTok{\%\textgreater{}\%} \FunctionTok{transmute}\NormalTok{(}\AttributeTok{point\_id =}\NormalTok{ points, }\AttributeTok{UNC\_SD =}\NormalTok{ UNC\_SD, }\AttributeTok{UNC\_CV =}\NormalTok{ UNC\_CV), }\AttributeTok{by =} \StringTok{"point\_id"}\NormalTok{)}

\NormalTok{p\_unc\_sd\_f }\OtherTok{\textless{}{-}} \FunctionTok{ggplot}\NormalTok{(MAPDATA2\_unc, }\FunctionTok{aes}\NormalTok{(}\AttributeTok{x =}\NormalTok{ X, }\AttributeTok{y =}\NormalTok{ Y, }\AttributeTok{fill =}\NormalTok{ UNC\_SD)) }\SpecialCharTok{+}
  \FunctionTok{geom\_raster}\NormalTok{() }\SpecialCharTok{+}
  \FunctionTok{scale\_fill\_viridis\_c}\NormalTok{(}\AttributeTok{option =} \StringTok{"C"}\NormalTok{, }\AttributeTok{name =} \StringTok{"SD"}\NormalTok{) }\SpecialCharTok{+}
  \FunctionTok{coord\_equal}\NormalTok{() }\SpecialCharTok{+}
  \FunctionTok{labs}\NormalTok{(}\AttributeTok{title =} \StringTok{"Неопределенность (SD) между алгоритмами — Future"}\NormalTok{)}

\NormalTok{common }\OtherTok{\textless{}{-}} \FunctionTok{names}\NormalTok{(myExpl)}
\CommentTok{\# убрать предикторы без размаха в референсе}
\NormalTok{ok }\OtherTok{\textless{}{-}} \FunctionTok{sapply}\NormalTok{(myExpl[, common, }\AttributeTok{drop =} \ConstantTok{FALSE}\NormalTok{], }\ControlFlowTok{function}\NormalTok{(z) }\FunctionTok{length}\NormalTok{(}\FunctionTok{unique}\NormalTok{(}\FunctionTok{na.omit}\NormalTok{(z))) }\SpecialCharTok{\textgreater{}=} \DecValTok{2}\NormalTok{)}
\NormalTok{vars }\OtherTok{\textless{}{-}}\NormalTok{ common[ok]}

\CommentTok{\# матрица референса без NA{-}строк}
\NormalTok{ref\_mat }\OtherTok{\textless{}{-}} \FunctionTok{as.matrix}\NormalTok{(}\FunctionTok{na.omit}\NormalTok{(myExpl[, vars, }\AttributeTok{drop =} \ConstantTok{FALSE}\NormalTok{]))}

\CommentTok{\# собрать RasterStack для будущего из точек x,y и значений предикторов}
\NormalTok{layers }\OtherTok{\textless{}{-}} \FunctionTok{lapply}\NormalTok{(vars, }\ControlFlowTok{function}\NormalTok{(nm) \{}
  \FunctionTok{rasterFromXYZ}\NormalTok{(}\FunctionTok{data.frame}\NormalTok{(}\AttributeTok{x =}\NormalTok{ DATA\_F}\SpecialCharTok{$}\NormalTok{x, }\AttributeTok{y =}\NormalTok{ DATA\_F}\SpecialCharTok{$}\NormalTok{y, }\AttributeTok{z =}\NormalTok{ myExplP1[, nm]))}
\NormalTok{\})}
\NormalTok{x\_stack }\OtherTok{\textless{}{-}} \FunctionTok{stack}\NormalTok{(layers); }\FunctionTok{names}\NormalTok{(x\_stack) }\OtherTok{\textless{}{-}}\NormalTok{ vars}

\CommentTok{\# посчитать MESS как растер}
\NormalTok{mess\_r }\OtherTok{\textless{}{-}} \FunctionTok{suppressWarnings}\NormalTok{(dismo}\SpecialCharTok{::}\FunctionTok{mess}\NormalTok{(x\_stack, ref\_mat))}

\CommentTok{\# извлечь значения MESS в порядке строк будущего набора}
\NormalTok{mess\_vals }\OtherTok{\textless{}{-}}\NormalTok{ raster}\SpecialCharTok{::}\FunctionTok{extract}\NormalTok{(mess\_r, DATA\_F[, }\FunctionTok{c}\NormalTok{(}\StringTok{"x"}\NormalTok{,}\StringTok{"y"}\NormalTok{)])}

\NormalTok{MAPDATA2\_MESS }\OtherTok{\textless{}{-}}\NormalTok{ dplyr}\SpecialCharTok{::}\FunctionTok{mutate}\NormalTok{(MAPDATA2, }\AttributeTok{MESS =}\NormalTok{ mess\_vals)}

\NormalTok{p\_mess }\OtherTok{\textless{}{-}} \FunctionTok{ggplot}\NormalTok{(MAPDATA2\_MESS, }\FunctionTok{aes}\NormalTok{(}\AttributeTok{x =}\NormalTok{ X, }\AttributeTok{y =}\NormalTok{ Y, }\AttributeTok{fill =}\NormalTok{ MESS)) }\SpecialCharTok{+}
  \FunctionTok{geom\_raster}\NormalTok{() }\SpecialCharTok{+}
  \FunctionTok{scale\_fill\_gradient2}\NormalTok{(}\AttributeTok{low =} \StringTok{"\#762A83"}\NormalTok{, }\AttributeTok{mid =} \StringTok{"\#F7F7F7"}\NormalTok{, }\AttributeTok{high =} \StringTok{"\#1B7837"}\NormalTok{, }\AttributeTok{midpoint =} \DecValTok{0}\NormalTok{, }\AttributeTok{name =} \StringTok{"MESS"}\NormalTok{) }\SpecialCharTok{+}
  \FunctionTok{coord\_equal}\NormalTok{() }\SpecialCharTok{+}
  \FunctionTok{labs}\NormalTok{(}\AttributeTok{title =} \StringTok{"MESS (экстраполяционный риск) — Future vs Current"}\NormalTok{)}

\FunctionTok{print}\NormalTok{(p\_unc\_sd\_f)}
\end{Highlighting}
\end{Shaded}

\begin{figure}[H]

{\centering \includegraphics[width=0.8\linewidth,height=\textheight,keepaspectratio]{images/SDM20.PNG}

}

\caption{Рис. 20.:Визуализация неопределенности прогнозов}

\end{figure}%

Этот код демонстрирует анализ неопределённости прогнозов модели
распределения видов для будущего сценария и оценку экстраполяционного
риска с помощью MESS-анализа.

Карта p\_unc\_sd\_f визуализирует неопределённость прогнозов между
разными алгоритмами моделирования через стандартное отклонение (SD).
Высокие значения SD (светлые цвета на карте) указывают на участки где
разные алгоритмы дают сильно расходящиеся предсказания что снижает
надёжность прогноза в этих районах. Низкие значения SD (тёмные цвета)
соответствуют областям консенсуса между алгоритмами где прогноз можно
считать более надёжным.

MESS-анализ (Multivariate Environmental Similarity Surface) оценивает
экстраполяционный риск показывая насколько условия в будущем сценарии
отличаются от текущих условий в которых модель обучалась. Положительные
значения MESS (зелёные области) указывают на условия аналогичные
обучающим что делает прогноз более надёжным. Отрицательные значения
(фиолетовые области) сигнализируют об экстраполяции --- условиях
выходящих за пределы обучающего диапазона где прогнозы становятся
ненадёжными поскольку модель экстраполирует а не интерполирует.

Вместе эти два анализа предоставляют важную информацию о качестве и
надёжности прогнозов: карта неопределённости показывает внутреннюю
согласованность модели в то время как MESS-анализ предупреждает о рисках
связанных с экстраполяцией в условиях выходящих за пределы обучающих
данных что особенно важно при моделировании будущих сценариев которые
могут включать ранее не наблюдавшиеся комбинации экологических факторов.

\begin{Shaded}
\begin{Highlighting}[]
\FunctionTok{print}\NormalTok{(p\_mess)}

\CommentTok{\# Конец скрипта {-}{-}{-}{-}{-}{-}{-}{-}{-}{-}{-}{-}{-}{-}{-}{-}{-}{-}{-}{-}{-}{-}{-}{-}{-}{-}{-}{-}{-}{-}{-}{-}{-}{-}{-}{-}{-}{-}{-}{-}{-}{-}{-}{-}{-}{-}{-}{-}{-}{-}{-}{-}{-}{-}{-}{-}{-}{-}{-}{-}{-}{-}{-}{-}{-}{-}{-}{-}{-}{-}{-}{-}{-}{-}{-}{-}{-}{-}{-}{-}{-}{-}{-}{-}{-}{-}{-}{-}{-}{-}{-}{-}{-}{-}{-}{-}{-}{-}{-}{-}{-}{-}{-}{-}{-} \#}
\end{Highlighting}
\end{Shaded}

\begin{figure}[H]

{\centering \includegraphics[width=0.8\linewidth,height=\textheight,keepaspectratio]{images/SDM21.PNG}

}

\caption{Рис. 21.:Визуализация MESS-анализа}

\end{figure}%

Если второй скрипт был работой старателя, отсеивающего песок, то третий
--- работа ювелира, который собирает найденные крупицы в единый сплав и
оценивает его прочность. Здесь мы переходим от отдельных моделей к
ансамблю, от калибровки к прогнозу, от понимания прошлого к
неопределенному будущему. Это кульминация всего проекта, где мы честно
смотрим в лицо самой сложной задаче: предсказанию, основанному на
ограниченных данных и неполном знании.

Краткий обзор выполненного: От комитета моделей к карте рисков Третий
скрипт выполнил несколько ключевых задач:

Ансамблевое моделирование (BIOMOD\_EnsembleModeling): Мы не стали
доверять ни одной единственной модели (GLM, GAM, RF и т.д.), сколь бы
хороша она ни была. Вместо этого мы создали «комитет экспертов» ---
ансамбль, который усреднил прогнозы всех 24 успешно построенных моделей
(по 2-3 реализации на каждый из 9 алгоритмов). В качестве правил
голосования использовались:

EMmean: Среднее взвешенное по вероятностям. Модели с высоким TSS
(\textgreater0.4) имели больший вес.

EMca: Committee Averaging --- бинаризованное голосование на основе
оптимального порога.

Проекция в текущих условиях (BIOMOD\_Projection): Ансамбль спроецировал
свои прогнозы на всю исследуемую акваторию, создав непрерывную карту
вероятности присутствия вида.

Проекция в будущих условиях: Используя данные по будущему климату
(сценарий RCP 8.5, 2100 г.), мы получили прогноз того, как ареал вида
может измениться.

Всесторонняя диагностика и оценка неопределенности: Мы не просто
построили карты, а оценили их надежность с помощью батареи метрик:

Калибровка (Reliability Plot): Насколько предсказанная вероятность
соответствует наблюдаемой частоте встреч.

Discriminative Power (ROC-AUC, PR-AUC): Способность модели отличать
presence от absence.

Boyce Index: Оценивает, насколько предсказания согласуются с данными
присутствия в пространстве окружающей среды.

Неопределенность между алгоритмами: Карта стандартного отклонения
предсказаний всех моделей --- где модели «согласны», а где их мнения
радикально расходятся.

MESS (Multivariate Environmental Similarity Surface) анализ: Показывает,
насколько условия в будущем экстраполируются за пределы тех, в которых
модель была обучена. Это карта риска ошибочного прогноза.

Результаты: История в трех картах и четырех графиках 1. Прогноз для
текущих условий: Ансамблевая модель (EMmean) выдала высокие и
статистически значимые показатели качества (TSS = 0.925, ROC = 0.987).
Карта прогноза четко показывает оптимальный ареал вида --- он приурочен
к специфическим шельфовым водам с определенным диапазоном глубин и
расстояний от берега. Это не случайный набор точек, а четко
структурированный паттерн, что подтверждает адекватность модели.

\begin{enumerate}
\def\labelenumi{\arabic{enumi}.}
\setcounter{enumi}{1}
\item
  Ключевые драйверы распределения (по версии ансамбля): Ансамбль
  подтвердил выводы второго скрипта, но расставил еще более жесткие
  приоритеты. Доминирующими фактором, определяющим до 75\% важности,
  являются расстояние до берега (dist) и средняя температура. Все
  остальные переменные (хлорофилл, скорость течений и пр.) вносят крайне
  незначительный вклад в сравнении с ним. Это говорит о том, что наш вид
  является узким батиметрическим и термическим специалистом.
\item
  Взгляд в будущее (2040-2100 гг.): Прогноз на будущее показывает
  тревожную динамику:
\end{enumerate}

Сдвиг ареала: Наблюдается смещение областей с высокой пригодностью среды
в северо-восточном направлении, что, вероятно, является реакцией на
потепление вод и изменение продуктивности.

Карта изменений (Delta): Наглядно демонстрирует потери (красные тона) в
традиционных местообитаниях и потенциальное расширение (зеленые тона) в
новых районах, хотя последнее нуждается в крайне осторожной
интерпретации.

Диагностика надежности: Сильные стороны и тревожные сигналы Калибровка:
Кривая идеально легла на биссектрису. Это означает, что если модель
предсказывает вероятность 70\%, то в реальности вид встречается в 70\%
случаев в подобных условиях. Наша модель не просто ранжирует
местоположения по пригодности, она дает хорошо откалиброванные, точные
вероятности.

Discriminative Power: Исключительно высокие значения AUC ROC (0.987) и
AUC PR (0.925) подтверждают, что модель блестяще отделяет ``сигнал'' от
``шума''.

Неопределенность: Карта стандартного отклонения предсказаний показывает,
что наибольшая неопределенность присуща как раз зонам экстраполяции ---
на границах ареала и в будущем сценарии. Там, где условия сильно
отличаются от тех, на которых училась модель, ее прогнозы наименее
надежны.

Главный сигнал --- MESS-анализ: Значительная часть прогноза для будущего
периода, особенно в н глубоководных зонах нативного ареала, попадает в
область экстраполяции (отрицательные значения MESS). Это означает, что
сочетания средовых ковариат в этих клетках не имеют аналогов в текущих
данных для обучения. Доверять прогнозу в этих зонах категорически
нельзя. Это не прогноз, а математическая экстраполяция, лишенная
экологического смысла.

Заключение: Прогноз --- сдержанный и с оговорками На основе проведенного
анализа можно сделать следующие выводы:

Модель адекватна и надежна для текущих условий. Мы построили
статистически робастную ансамблевую модель, которая с высокой точностью
описывает современное распределение вида, ключевым лимитирующим фактором
для которого является температура и расстояние до берега (и связанные с
ним параметры).

Возможные рекомендации и дальнейшие шаги:

Для менеджеров рыболовства/ООПТ: Следует сконцентрировать усилия по
мониторингу и сохранению на известных в настоящее время ключевых
местообитаниях, которые модель идентифицирует как оптимальные. К
прогнозам расширения ареала следует относиться как к гипотезе, требующей
строгой полевой проверки.

Для исследователей: Необходимо расширять данные наблюдений, особенно в
пограничных зонах ареала и в районах, которые в будущем могут стать
подходящими. Это снизит область экстраполяции и повысит надежность
моделей. Целесообразно опробовать более комплексные модели, включающие
биотические взаимодействия и дисперсионные возможности вида.

Философский итог: Мы не предсказали будущее. Мы смоделировали его
возможный сценарий при строго определенных условиях и четко очертили
границы, за которыми наши знания заканчиваются и начинается область
догадок. В этом и заключается честный, научно обоснованный подход к
моделированию сложных систем.

\bookmarksetup{startatroot}

\chapter{I. SPiCT:
МОДЕЛЬ}\label{i.-spict-ux43cux43eux434ux435ux43bux44c}

\section{Введение в SPiCT: от данных к управлению
запасами}\label{ux432ux432ux435ux434ux435ux43dux438ux435-ux432-spict-ux43eux442-ux434ux430ux43dux43dux44bux445-ux43a-ux443ux43fux440ux430ux432ux43bux435ux43dux438ux44e-ux437ux430ux43fux430ux441ux430ux43cux438}

\section{Основы моделирования и диагностики в
SPiCT}\label{ux43eux441ux43dux43eux432ux44b-ux43cux43eux434ux435ux43bux438ux440ux43eux432ux430ux43dux438ux44f-ux438-ux434ux438ux430ux433ux43dux43eux441ux442ux438ux43aux438-ux432-spict}

Если смотреть на оценку запасов «с высоты орбиты», всё кажется почти
прямолинейным: есть выловы, есть индексы биомассы, есть модель ---
значит, будет и рекомендация. Но на палубе реальности мозг слишком любит
простые истории: «подставим числа --- получим \emph{MSY}». SPiCT как раз
дисциплинирует эту склонность: заставляет назвать источники
неопределённости своими именами, зафиксировать момент наблюдения
индексов, выбрать априорные распределения там, где данных мало, и пройти
полноценную диагностику. Это не магический чёрный ящик, а аккуратный
перевод временных рядов выловов и индексов в управленческие ориентиры
--- с явной лентой интервалов.

Что такое SPiCT по сути. Это стохастическая продукционная модель в
непрерывном времени, где динамика биомассы отделена от наблюдательной
ошибки, а промысловая смертность и продукция описываются гладко и
физически правдоподобно. Минимальный набор данных --- годовые выловы и
хотя бы один индекс относительной биомассы (CPUE, научная съёмка),
привязанные ко времени года (середина, конец и т. п.). Выход --- оценка
\emph{r} и \emph{K}, ориентиры MSY (\emph{B\textsubscript{msy}},
\emph{F\textsubscript{msy}}, \emph{MSY}), траектории \emph{B(t)} и
\emph{F(t)}, их отношения к \emph{B\textsubscript{msy}} и
\emph{F\textsubscript{msy}}, а также диагностика, без которой лучше не
приближаться к выработке рекомендаций по вылову.

Где нас поджидают ловушки. - Масштаб и уловистость (\emph{q}):
абсолютные величины \emph{K} и \emph{B} зависят от шкалы индекса;
безопаснее мыслить в \emph{B/B\textsubscript{msy}} и
\emph{F/F\textsubscript{msy}}. «Абсолютная биомасса» без внешней
калибровки --- иллюзия точности. - Время наблюдения: индекс «за июль» и
вылов «за год» --- не одно и то же. Ошибка хронометража превращается в
систематическое смещение. - Неидентифицируемость \emph{r} и \emph{K}:
короткий ряд или один слабый индекс «перетягивают» правдоподобие;
априоры помогают, но создают ответственность за допущения. - Дрейф
уловистости и структура флота: технологический прогресс маскируется под
«рост запаса», если его не вынести на свет. - Последние годы:
предварительные данные искажают тренды; в SPiCT для них есть честный
инструмент --- повышение дисперсии (stdevfac). - Диагностика не
формальность: OSA‑остатки, Ljung--Box, ретроанализ (Mohn's rho) и
профили правдоподобия --- фильтр от «правдоподобных, но хрупких»
решений.

Как читать результат. SPiCT даёт распределения, а не единственные
точечные числа. Ключ --- в относительных показателях и интервалах:
\emph{B/B\textsubscript{msy}} и \emph{F/F\textsubscript{msy}} с 95\% ДИ,
\emph{MSY} с неопределённостью, вероятности пребывания в «зелёной зоне».
Совет по вылову --- не прямой продукт «умножить
\emph{F\textsubscript{msy}} на \emph{B}», а часть цепочки: оценка →
правило пегулирования промысла (HCR) → проверка устойчивости (желательно
--- через MSE). Хорошая история --- та, в которой видно, чему мы обязаны
данным, а чему --- допущениям.

Что делаем в этой главе. - Формируем вход: выловы, индексы, их время
наблюдения; настраиваем априорные распределения (\emph{n}, \emph{K},
начальная доля \emph{bkfrac}), дисперсии и численный шаг. - Оцениваем
модель и проверяем: сходимость, OSA‑остатки, автокорреляция, ретро и
Mohn's rho, чувствительность к априорным распределениям. - Извлекаем
ориентиры: \emph{MSY}, \emph{B\textsubscript{msy}},
\emph{F\textsubscript{msy}}; интерпретируем текущее состояние
(\emph{B/B\textsubscript{msy}}, \emph{F/F\textsubscript{msy}}). -
Готовим мост к управлению: показываем, как из оценок рождаются кандидаты
правил (HCR) и почему без явной неопределённости лучше не говорить про
ОДУ

Границы метода. SPiCT не заменяет прямые съёмки и не исправляет дефекты
данных «по дороге». Он делает видимой структуру неопределённости, где
раньше была уверенность «на глаз». Именно за это его стоит любить:
меньше иллюзий контроля --- больше проверяемых решений.

Полный скрипт можно скачать по
\href{https://mombus.github.io/cRab/data/SPICT_MODEL.R}{ссылке}. Ниже
приводится исполнение скрипта.

\begin{Shaded}
\begin{Highlighting}[]
\CommentTok{\# ===============================================================}
\CommentTok{\#     СКРИПТ 1: ОСНОВЫ МОДЕЛИРОВАНИЯ И ДИАГНОСТИКИ В SPiCT}
\CommentTok{\#     Курс: Оценка водных биоресурсов при недостатке данных в R}
\CommentTok{\#     Автор: Баканев С. В.}
\CommentTok{\#     Дата создания: 28.08.2025}
\CommentTok{\# ===============================================================}

\CommentTok{\# ======================= ВВЕДЕНИЕ =============================}
\CommentTok{\# SPiCT (Surplus Production model in Continuous Time) {-} это}
\CommentTok{\# стохастическая продукционная модель для оценки запасов рыбы}
\CommentTok{\# при ограниченных данных. Модель требует только временные ряды}
\CommentTok{\# уловов и индексов биомассы (например, CPUE или научные съемки)}

\CommentTok{\# {-}{-}{-}{-}{-}{-}{-}{-}{-}{-}{-}{-}{-}{-}{-}{-}{-}{-}{-} 1. ПОДГОТОВКА СРЕДЫ {-}{-}{-}{-}{-}{-}{-}{-}{-}{-}{-}{-}{-}{-}{-}{-}{-}{-}{-}{-}}

\DocumentationTok{\#\# 1.1 Очистка рабочей среды (удаляем все объекты)}
\FunctionTok{rm}\NormalTok{(}\AttributeTok{list =} \FunctionTok{ls}\NormalTok{())}

\DocumentationTok{\#\# 1.2 Загрузка необходимых библиотек}
\FunctionTok{library}\NormalTok{(spict)      }\CommentTok{\# Основной пакет для SPiCT моделирования}
\end{Highlighting}
\end{Shaded}

\begin{verbatim}
Загрузка требуемого пакета: TMB
\end{verbatim}

\begin{verbatim}
Welcome to spict_v1.3.8@107a32
\end{verbatim}

\begin{Shaded}
\begin{Highlighting}[]
\FunctionTok{library}\NormalTok{(tidyverse)  }\CommentTok{\# Для обработки данных и визуализации}
\end{Highlighting}
\end{Shaded}

\begin{verbatim}
-- Attaching core tidyverse packages ------------------------ tidyverse 2.0.0 --
v dplyr     1.1.4     v readr     2.1.5
v forcats   1.0.0     v stringr   1.5.2
v ggplot2   4.0.0     v tibble    3.2.1
v lubridate 1.9.4     v tidyr     1.3.1
v purrr     1.0.4     
\end{verbatim}

\begin{verbatim}
-- Conflicts ------------------------------------------ tidyverse_conflicts() --
x dplyr::filter() masks stats::filter()
x dplyr::lag()    masks stats::lag()
i Use the conflicted package (<http://conflicted.r-lib.org/>) to force all conflicts to become errors
\end{verbatim}

\begin{Shaded}
\begin{Highlighting}[]
\FunctionTok{library}\NormalTok{(ggplot2)    }\CommentTok{\# Дополнительные возможности построения графиков}

\DocumentationTok{\#\# 1.3 Установка рабочей директории}
\CommentTok{\# ВАЖНО: Измените путь на вашу рабочую папку}
\FunctionTok{setwd}\NormalTok{(}\StringTok{"C:/SPICT"}\NormalTok{) }

\DocumentationTok{\#\# 1.4 Настройка вывода чисел (опционально)}
\FunctionTok{options}\NormalTok{(}\AttributeTok{scipen =} \DecValTok{999}\NormalTok{)  }\CommentTok{\# Отключение научной нотации}
\FunctionTok{options}\NormalTok{(}\AttributeTok{digits =} \DecValTok{4}\NormalTok{)    }\CommentTok{\# Количество значащих цифр}

\CommentTok{\# {-}{-}{-}{-}{-}{-}{-}{-}{-}{-}{-}{-}{-}{-}{-}{-}{-}{-}{-} 2. ЗАГРУЗКА ДАННЫХ {-}{-}{-}{-}{-}{-}{-}{-}{-}{-}{-}{-}{-}{-}{-}{-}{-}{-}{-}{-}}

\FunctionTok{cat}\NormalTok{(}\StringTok{"}\SpecialCharTok{\textbackslash{}n}\StringTok{========== ЗАГРУЗКА ДАННЫХ ==========}\SpecialCharTok{\textbackslash{}n}\StringTok{"}\NormalTok{)}
\end{Highlighting}
\end{Shaded}

\begin{verbatim}

========== ЗАГРУЗКА ДАННЫХ ==========
\end{verbatim}

\begin{Shaded}
\begin{Highlighting}[]
\DocumentationTok{\#\# 2.1 Временной ряд (годы наблюдений)}
\CommentTok{\# Период наблюдений с 2005 по 2024 год}
\NormalTok{Year }\OtherTok{\textless{}{-}} \DecValTok{2005}\SpecialCharTok{:}\DecValTok{2024}

\DocumentationTok{\#\# 2.2 Данные по вылову (в тысячах тонн)}
\CommentTok{\# Представляют общий коммерческий вылов по годам}
\CommentTok{\# Обратите внимание на тренд: рост до 2014 г., затем снижение}
\NormalTok{Catch }\OtherTok{\textless{}{-}} \FunctionTok{c}\NormalTok{(}\DecValTok{5}\NormalTok{,  }\DecValTok{7}\NormalTok{,  }\DecValTok{6}\NormalTok{, }\DecValTok{10}\NormalTok{, }\DecValTok{14}\NormalTok{, }\DecValTok{25}\NormalTok{, }\DecValTok{28}\NormalTok{, }\DecValTok{30}\NormalTok{, }\DecValTok{32}\NormalTok{, }\DecValTok{35}\NormalTok{,    }\CommentTok{\# 2005{-}2014}
          \DecValTok{25}\NormalTok{, }\DecValTok{20}\NormalTok{, }\DecValTok{15}\NormalTok{, }\DecValTok{12}\NormalTok{, }\DecValTok{10}\NormalTok{, }\DecValTok{12}\NormalTok{, }\DecValTok{10}\NormalTok{, }\DecValTok{13}\NormalTok{, }\DecValTok{11}\NormalTok{, }\DecValTok{12}\NormalTok{)    }\CommentTok{\# 2015{-}2024}

\DocumentationTok{\#\# 2.3 Индекс CPUE (улов на единицу усилия)}
\CommentTok{\# Промысловый индекс, отражающий относительную биомассу}
\CommentTok{\# Собирается в середине года (июль) во время промысла}
\NormalTok{CPUEIndex }\OtherTok{\textless{}{-}} \FunctionTok{c}\NormalTok{(}\FloatTok{27.427120}\NormalTok{, }\FloatTok{26.775958}\NormalTok{, }\FloatTok{16.811997}\NormalTok{, }\FloatTok{22.979653}\NormalTok{, }\FloatTok{29.048568}\NormalTok{, }
               \FloatTok{29.996072}\NormalTok{, }\FloatTok{16.476301}\NormalTok{, }\FloatTok{17.174455}\NormalTok{, }\FloatTok{10.537272}\NormalTok{, }\FloatTok{14.590435}\NormalTok{,}
                \FloatTok{8.286352}\NormalTok{, }\FloatTok{11.394168}\NormalTok{, }\FloatTok{15.537878}\NormalTok{, }\FloatTok{13.791166}\NormalTok{, }\FloatTok{11.527548}\NormalTok{, }
               \FloatTok{15.336093}\NormalTok{, }\FloatTok{12.154069}\NormalTok{, }\FloatTok{15.568450}\NormalTok{, }\FloatTok{16.221933}\NormalTok{, }\FloatTok{13.421132}\NormalTok{)}

\DocumentationTok{\#\# 2.4 Индекс BESS (биомасса по научной съемке)}
\CommentTok{\# Независимая оценка биомассы из научных траловых съемок}
\CommentTok{\# Проводится в 4{-}м квартале года (октябрь)}
\CommentTok{\# NA в первый год означает отсутствие съемки}
\NormalTok{BESSIndex }\OtherTok{\textless{}{-}} \FunctionTok{c}\NormalTok{(       }\ConstantTok{NA}\NormalTok{, }\FloatTok{16.270375}\NormalTok{, }\FloatTok{20.691355}\NormalTok{, }\FloatTok{15.141784}\NormalTok{, }\FloatTok{18.594620}\NormalTok{, }
               \FloatTok{15.975548}\NormalTok{, }\FloatTok{13.792012}\NormalTok{, }\FloatTok{13.328805}\NormalTok{, }\FloatTok{11.659744}\NormalTok{, }\FloatTok{11.753855}\NormalTok{,}
                \FloatTok{9.309859}\NormalTok{,  }\FloatTok{7.104886}\NormalTok{,  }\FloatTok{7.963839}\NormalTok{,  }\FloatTok{9.161322}\NormalTok{, }\FloatTok{10.271221}\NormalTok{, }
                \FloatTok{9.822960}\NormalTok{, }\FloatTok{10.347376}\NormalTok{, }\FloatTok{11.703610}\NormalTok{, }\FloatTok{13.679876}\NormalTok{, }\FloatTok{13.413696}\NormalTok{)}

\DocumentationTok{\#\# 2.5 Визуализация исходных данных}
\FunctionTok{par}\NormalTok{(}\AttributeTok{mfrow =} \FunctionTok{c}\NormalTok{(}\DecValTok{2}\NormalTok{, }\DecValTok{2}\NormalTok{), }\AttributeTok{mar =} \FunctionTok{c}\NormalTok{(}\DecValTok{4}\NormalTok{, }\DecValTok{4}\NormalTok{, }\DecValTok{2}\NormalTok{, }\DecValTok{1}\NormalTok{))}

\CommentTok{\# График вылова}
\FunctionTok{plot}\NormalTok{(Year, Catch, }\AttributeTok{type =} \StringTok{"b"}\NormalTok{, }\AttributeTok{pch =} \DecValTok{19}\NormalTok{, }\AttributeTok{col =} \StringTok{"red"}\NormalTok{,}
     \AttributeTok{main =} \StringTok{"Динамика вылова"}\NormalTok{, }\AttributeTok{xlab =} \StringTok{"Год"}\NormalTok{, }\AttributeTok{ylab =} \StringTok{"Вылов (тыс. т)"}\NormalTok{)}
\end{Highlighting}
\end{Shaded}

\begin{verbatim}
Warning in title(...): неизвестна ширина символа 0xc4 в кодировке CP1251
\end{verbatim}

\begin{verbatim}
Warning in title(...): неизвестна ширина символа 0xe8 в кодировке CP1251
\end{verbatim}

\begin{verbatim}
Warning in title(...): неизвестна ширина символа 0xed в кодировке CP1251
\end{verbatim}

\begin{verbatim}
Warning in title(...): неизвестна ширина символа 0xe0 в кодировке CP1251
\end{verbatim}

\begin{verbatim}
Warning in title(...): неизвестна ширина символа 0xec в кодировке CP1251
\end{verbatim}

\begin{verbatim}
Warning in title(...): неизвестна ширина символа 0xe8 в кодировке CP1251
\end{verbatim}

\begin{verbatim}
Warning in title(...): неизвестна ширина символа 0xea в кодировке CP1251
\end{verbatim}

\begin{verbatim}
Warning in title(...): неизвестна ширина символа 0xe0 в кодировке CP1251
\end{verbatim}

\begin{verbatim}
Warning in title(...): неизвестна ширина символа 0xe2 в кодировке CP1251
\end{verbatim}

\begin{verbatim}
Warning in title(...): неизвестна ширина символа 0xfb в кодировке CP1251
\end{verbatim}

\begin{verbatim}
Warning in title(...): неизвестна ширина символа 0xeb в кодировке CP1251
\end{verbatim}

\begin{verbatim}
Warning in title(...): неизвестна ширина символа 0xee в кодировке CP1251
\end{verbatim}

\begin{verbatim}
Warning in title(...): неизвестна ширина символа 0xe2 в кодировке CP1251
\end{verbatim}

\begin{verbatim}
Warning in title(...): неизвестна ширина символа 0xe0 в кодировке CP1251
\end{verbatim}

\begin{verbatim}
Warning in title(...): неизвестна ширина символа 0xc3 в кодировке CP1251
\end{verbatim}

\begin{verbatim}
Warning in title(...): неизвестна ширина символа 0xee в кодировке CP1251
\end{verbatim}

\begin{verbatim}
Warning in title(...): неизвестна ширина символа 0xe4 в кодировке CP1251
\end{verbatim}

\begin{verbatim}
Warning in title(...): неизвестна ширина символа 0xc2 в кодировке CP1251
\end{verbatim}

\begin{verbatim}
Warning in title(...): неизвестна ширина символа 0xfb в кодировке CP1251
\end{verbatim}

\begin{verbatim}
Warning in title(...): неизвестна ширина символа 0xeb в кодировке CP1251
\end{verbatim}

\begin{verbatim}
Warning in title(...): неизвестна ширина символа 0xee в кодировке CP1251
\end{verbatim}

\begin{verbatim}
Warning in title(...): неизвестна ширина символа 0xe2 в кодировке CP1251
\end{verbatim}

\begin{verbatim}
Warning in title(...): неизвестна ширина символа 0xf2 в кодировке CP1251
\end{verbatim}

\begin{verbatim}
Warning in title(...): неизвестна ширина символа 0xfb в кодировке CP1251
\end{verbatim}

\begin{verbatim}
Warning in title(...): неизвестна ширина символа 0xf1 в кодировке CP1251
\end{verbatim}

\begin{verbatim}
Warning in title(...): неизвестна ширина символа 0xf2 в кодировке CP1251
\end{verbatim}

\begin{Shaded}
\begin{Highlighting}[]
\FunctionTok{grid}\NormalTok{()}

\CommentTok{\# График CPUE}
\FunctionTok{plot}\NormalTok{(Year, CPUEIndex, }\AttributeTok{type =} \StringTok{"b"}\NormalTok{, }\AttributeTok{pch =} \DecValTok{19}\NormalTok{, }\AttributeTok{col =} \StringTok{"blue"}\NormalTok{,}
     \AttributeTok{main =} \StringTok{"Индекс CPUE"}\NormalTok{, }\AttributeTok{xlab =} \StringTok{"Год"}\NormalTok{, }\AttributeTok{ylab =} \StringTok{"CPUE"}\NormalTok{)}
\end{Highlighting}
\end{Shaded}

\begin{verbatim}
Warning in title(...): неизвестна ширина символа 0xc8 в кодировке CP1251
\end{verbatim}

\begin{verbatim}
Warning in title(...): неизвестна ширина символа 0xed в кодировке CP1251
\end{verbatim}

\begin{verbatim}
Warning in title(...): неизвестна ширина символа 0xe4 в кодировке CP1251
\end{verbatim}

\begin{verbatim}
Warning in title(...): неизвестна ширина символа 0xe5 в кодировке CP1251
\end{verbatim}

\begin{verbatim}
Warning in title(...): неизвестна ширина символа 0xea в кодировке CP1251
\end{verbatim}

\begin{verbatim}
Warning in title(...): неизвестна ширина символа 0xf1 в кодировке CP1251
\end{verbatim}

\begin{verbatim}
Warning in title(...): неизвестна ширина символа 0xc3 в кодировке CP1251
\end{verbatim}

\begin{verbatim}
Warning in title(...): неизвестна ширина символа 0xee в кодировке CP1251
\end{verbatim}

\begin{verbatim}
Warning in title(...): неизвестна ширина символа 0xe4 в кодировке CP1251
\end{verbatim}

\begin{Shaded}
\begin{Highlighting}[]
\FunctionTok{grid}\NormalTok{()}

\CommentTok{\# График BESS}
\FunctionTok{plot}\NormalTok{(Year, BESSIndex, }\AttributeTok{type =} \StringTok{"b"}\NormalTok{, }\AttributeTok{pch =} \DecValTok{19}\NormalTok{, }\AttributeTok{col =} \StringTok{"darkgreen"}\NormalTok{,}
     \AttributeTok{main =} \StringTok{"Индекс BESS"}\NormalTok{, }\AttributeTok{xlab =} \StringTok{"Год"}\NormalTok{, }\AttributeTok{ylab =} \StringTok{"BESS"}\NormalTok{)}
\end{Highlighting}
\end{Shaded}

\begin{verbatim}
Warning in title(...): неизвестна ширина символа 0xc8 в кодировке CP1251
\end{verbatim}

\begin{verbatim}
Warning in title(...): неизвестна ширина символа 0xed в кодировке CP1251
\end{verbatim}

\begin{verbatim}
Warning in title(...): неизвестна ширина символа 0xe4 в кодировке CP1251
\end{verbatim}

\begin{verbatim}
Warning in title(...): неизвестна ширина символа 0xe5 в кодировке CP1251
\end{verbatim}

\begin{verbatim}
Warning in title(...): неизвестна ширина символа 0xea в кодировке CP1251
\end{verbatim}

\begin{verbatim}
Warning in title(...): неизвестна ширина символа 0xf1 в кодировке CP1251
\end{verbatim}

\begin{verbatim}
Warning in title(...): неизвестна ширина символа 0xc3 в кодировке CP1251
\end{verbatim}

\begin{verbatim}
Warning in title(...): неизвестна ширина символа 0xee в кодировке CP1251
\end{verbatim}

\begin{verbatim}
Warning in title(...): неизвестна ширина символа 0xe4 в кодировке CP1251
\end{verbatim}

\begin{Shaded}
\begin{Highlighting}[]
\FunctionTok{grid}\NormalTok{()}

\CommentTok{\# График всех индексов (нормализованных)}
\FunctionTok{plot}\NormalTok{(Year, CPUEIndex}\SpecialCharTok{/}\FunctionTok{mean}\NormalTok{(CPUEIndex, }\AttributeTok{na.rm =} \ConstantTok{TRUE}\NormalTok{), }\AttributeTok{type =} \StringTok{"l"}\NormalTok{, }
     \AttributeTok{col =} \StringTok{"blue"}\NormalTok{, }\AttributeTok{lwd =} \DecValTok{2}\NormalTok{, }\AttributeTok{ylim =} \FunctionTok{c}\NormalTok{(}\DecValTok{0}\NormalTok{, }\DecValTok{2}\NormalTok{),}
     \AttributeTok{main =} \StringTok{"Сравнение индексов"}\NormalTok{, }\AttributeTok{xlab =} \StringTok{"Год"}\NormalTok{, }\AttributeTok{ylab =} \StringTok{"Отн. индекс"}\NormalTok{)}
\end{Highlighting}
\end{Shaded}

\begin{verbatim}
Warning in title(...): неизвестна ширина символа 0xd1 в кодировке CP1251
\end{verbatim}

\begin{verbatim}
Warning in title(...): неизвестна ширина символа 0xf0 в кодировке CP1251
\end{verbatim}

\begin{verbatim}
Warning in title(...): неизвестна ширина символа 0xe0 в кодировке CP1251
\end{verbatim}

\begin{verbatim}
Warning in title(...): неизвестна ширина символа 0xe2 в кодировке CP1251
\end{verbatim}

\begin{verbatim}
Warning in title(...): неизвестна ширина символа 0xed в кодировке CP1251
\end{verbatim}

\begin{verbatim}
Warning in title(...): неизвестна ширина символа 0xe5 в кодировке CP1251
\end{verbatim}

\begin{verbatim}
Warning in title(...): неизвестна ширина символа 0xed в кодировке CP1251
\end{verbatim}

\begin{verbatim}
Warning in title(...): неизвестна ширина символа 0xe8 в кодировке CP1251
\end{verbatim}

\begin{verbatim}
Warning in title(...): неизвестна ширина символа 0xe5 в кодировке CP1251
\end{verbatim}

\begin{verbatim}
Warning in title(...): неизвестна ширина символа 0xe8 в кодировке CP1251
\end{verbatim}

\begin{verbatim}
Warning in title(...): неизвестна ширина символа 0xed в кодировке CP1251
\end{verbatim}

\begin{verbatim}
Warning in title(...): неизвестна ширина символа 0xe4 в кодировке CP1251
\end{verbatim}

\begin{verbatim}
Warning in title(...): неизвестна ширина символа 0xe5 в кодировке CP1251
\end{verbatim}

\begin{verbatim}
Warning in title(...): неизвестна ширина символа 0xea в кодировке CP1251
\end{verbatim}

\begin{verbatim}
Warning in title(...): неизвестна ширина символа 0xf1 в кодировке CP1251
\end{verbatim}

\begin{verbatim}
Warning in title(...): неизвестна ширина символа 0xee в кодировке CP1251
\end{verbatim}

\begin{verbatim}
Warning in title(...): неизвестна ширина символа 0xe2 в кодировке CP1251
\end{verbatim}

\begin{verbatim}
Warning in title(...): неизвестна ширина символа 0xc3 в кодировке CP1251
\end{verbatim}

\begin{verbatim}
Warning in title(...): неизвестна ширина символа 0xee в кодировке CP1251
\end{verbatim}

\begin{verbatim}
Warning in title(...): неизвестна ширина символа 0xe4 в кодировке CP1251
\end{verbatim}

\begin{verbatim}
Warning in title(...): неизвестна ширина символа 0xce в кодировке CP1251
\end{verbatim}

\begin{verbatim}
Warning in title(...): неизвестна ширина символа 0xf2 в кодировке CP1251
\end{verbatim}

\begin{verbatim}
Warning in title(...): неизвестна ширина символа 0xed в кодировке CP1251
\end{verbatim}

\begin{verbatim}
Warning in title(...): неизвестна ширина символа 0xe8 в кодировке CP1251
\end{verbatim}

\begin{verbatim}
Warning in title(...): неизвестна ширина символа 0xed в кодировке CP1251
\end{verbatim}

\begin{verbatim}
Warning in title(...): неизвестна ширина символа 0xe4 в кодировке CP1251
\end{verbatim}

\begin{verbatim}
Warning in title(...): неизвестна ширина символа 0xe5 в кодировке CP1251
\end{verbatim}

\begin{verbatim}
Warning in title(...): неизвестна ширина символа 0xea в кодировке CP1251
\end{verbatim}

\begin{verbatim}
Warning in title(...): неизвестна ширина символа 0xf1 в кодировке CP1251
\end{verbatim}

\begin{Shaded}
\begin{Highlighting}[]
\FunctionTok{lines}\NormalTok{(Year, BESSIndex}\SpecialCharTok{/}\FunctionTok{mean}\NormalTok{(BESSIndex, }\AttributeTok{na.rm =} \ConstantTok{TRUE}\NormalTok{), }\AttributeTok{col =} \StringTok{"darkgreen"}\NormalTok{, }\AttributeTok{lwd =} \DecValTok{2}\NormalTok{)}
\FunctionTok{legend}\NormalTok{(}\StringTok{"topright"}\NormalTok{, }\FunctionTok{c}\NormalTok{(}\StringTok{"CPUE"}\NormalTok{, }\StringTok{"BESS"}\NormalTok{), }\AttributeTok{col =} \FunctionTok{c}\NormalTok{(}\StringTok{"blue"}\NormalTok{, }\StringTok{"darkgreen"}\NormalTok{), }\AttributeTok{lty =} \DecValTok{1}\NormalTok{, }\AttributeTok{lwd =} \DecValTok{2}\NormalTok{)}
\FunctionTok{grid}\NormalTok{()}
\end{Highlighting}
\end{Shaded}

\pandocbounded{\includegraphics[keepaspectratio]{chapter13_files/figure-pdf/unnamed-chunk-1-1.pdf}}

\begin{Shaded}
\begin{Highlighting}[]
\FunctionTok{par}\NormalTok{(}\AttributeTok{mfrow =} \FunctionTok{c}\NormalTok{(}\DecValTok{1}\NormalTok{, }\DecValTok{1}\NormalTok{))}

\CommentTok{\# {-}{-}{-}{-}{-}{-}{-}{-}{-}{-}{-}{-}{-}{-}{-}{-}{-}{-}{-} 3. ФОРМАТИРОВАНИЕ ДАННЫХ ДЛЯ SPiCT {-}{-}{-}{-}{-}{-}{-}{-}{-}{-}{-}{-}{-}{-}{-}{-}{-}{-}{-}{-}}

\FunctionTok{cat}\NormalTok{(}\StringTok{"}\SpecialCharTok{\textbackslash{}n}\StringTok{========== ПОДГОТОВКА ДАННЫХ ДЛЯ МОДЕЛИ ==========}\SpecialCharTok{\textbackslash{}n}\StringTok{"}\NormalTok{)}
\end{Highlighting}
\end{Shaded}

\begin{verbatim}

========== ПОДГОТОВКА ДАННЫХ ДЛЯ МОДЕЛИ ==========
\end{verbatim}

\begin{Shaded}
\begin{Highlighting}[]
\DocumentationTok{\#\# 3.1 Создание входного объекта для SPiCT}
\CommentTok{\# SPiCT требует специальный формат данных в виде списка}
\NormalTok{input\_data }\OtherTok{\textless{}{-}} \FunctionTok{list}\NormalTok{(}
  
  \CommentTok{\# Временной ряд вылова (обычно конец года)}
  \AttributeTok{timeC =}\NormalTok{ Year,        }
  \AttributeTok{obsC =}\NormalTok{ Catch,        }
  
  \CommentTok{\# Временные ряды индексов}
  \CommentTok{\# ВАЖНО: время индексов должно отражать когда они собраны}
  \AttributeTok{timeI =} \FunctionTok{list}\NormalTok{(}
\NormalTok{    Year }\SpecialCharTok{+} \FloatTok{0.5}\NormalTok{,      }\CommentTok{\# CPUE собирается в середине года (июль)}
\NormalTok{    Year }\SpecialCharTok{+} \FloatTok{0.75}      \CommentTok{\# BESS проводится в 4{-}м квартале (октябрь)}
\NormalTok{  ),}
  
  \CommentTok{\# Значения индексов (в том же порядке, что и timeI)}
  \AttributeTok{obsI =} \FunctionTok{list}\NormalTok{(}
\NormalTok{    CPUEIndex,       }
\NormalTok{    BESSIndex        }
\NormalTok{  )}
\NormalTok{)}

\DocumentationTok{\#\# 3.2 Проверка и валидация входных данных}
\CommentTok{\# Функция check.inp выполняет базовую проверку данных}
\CommentTok{\# и удаляет нулевые, отрицательные и NA значения}
\NormalTok{inp }\OtherTok{\textless{}{-}} \FunctionTok{check.inp}\NormalTok{(input\_data, }\AttributeTok{verbose =} \ConstantTok{TRUE}\NormalTok{)}
\end{Highlighting}
\end{Shaded}

\begin{verbatim}
Removing zero, negative, and NAs in  I  series  2  
\end{verbatim}

\begin{Shaded}
\begin{Highlighting}[]
\CommentTok{\# Вывод структуры данных}
\FunctionTok{cat}\NormalTok{(}\StringTok{"}\SpecialCharTok{\textbackslash{}n}\StringTok{Структура входных данных:}\SpecialCharTok{\textbackslash{}n}\StringTok{"}\NormalTok{)}
\end{Highlighting}
\end{Shaded}

\begin{verbatim}

Структура входных данных:
\end{verbatim}

\begin{Shaded}
\begin{Highlighting}[]
\FunctionTok{cat}\NormalTok{(}\StringTok{"Количество наблюдений вылова:"}\NormalTok{, }\FunctionTok{length}\NormalTok{(inp}\SpecialCharTok{$}\NormalTok{obsC), }\StringTok{"}\SpecialCharTok{\textbackslash{}n}\StringTok{"}\NormalTok{)}
\end{Highlighting}
\end{Shaded}

\begin{verbatim}
Количество наблюдений вылова: 20 
\end{verbatim}

\begin{Shaded}
\begin{Highlighting}[]
\FunctionTok{cat}\NormalTok{(}\StringTok{"Количество индексов:"}\NormalTok{, }\FunctionTok{length}\NormalTok{(inp}\SpecialCharTok{$}\NormalTok{obsI), }\StringTok{"}\SpecialCharTok{\textbackslash{}n}\StringTok{"}\NormalTok{)}
\end{Highlighting}
\end{Shaded}

\begin{verbatim}
Количество индексов: 2 
\end{verbatim}

\begin{Shaded}
\begin{Highlighting}[]
\FunctionTok{cat}\NormalTok{(}\StringTok{"Годы наблюдений:"}\NormalTok{, }\FunctionTok{range}\NormalTok{(inp}\SpecialCharTok{$}\NormalTok{timeC), }\StringTok{"}\SpecialCharTok{\textbackslash{}n}\StringTok{"}\NormalTok{)}
\end{Highlighting}
\end{Shaded}

\begin{verbatim}
Годы наблюдений: 2005 2024 
\end{verbatim}

\begin{Shaded}
\begin{Highlighting}[]
\CommentTok{\# {-}{-}{-}{-}{-}{-}{-}{-}{-}{-}{-}{-}{-}{-}{-}{-}{-}{-}{-} 4. НАСТРОЙКА МОДЕЛИ {-}{-}{-}{-}{-}{-}{-}{-}{-}{-}{-}{-}{-}{-}{-}{-}{-}{-}{-}{-}}

\FunctionTok{cat}\NormalTok{(}\StringTok{"}\SpecialCharTok{\textbackslash{}n}\StringTok{========== НАСТРОЙКА ПАРАМЕТРОВ МОДЕЛИ ==========}\SpecialCharTok{\textbackslash{}n}\StringTok{"}\NormalTok{)}
\end{Highlighting}
\end{Shaded}

\begin{verbatim}

========== НАСТРОЙКА ПАРАМЕТРОВ МОДЕЛИ ==========
\end{verbatim}

\begin{Shaded}
\begin{Highlighting}[]
\DocumentationTok{\#\# 4.1 Установка априорных распределений (priors)}

\DocumentationTok{\#\#\# 4.1.1 Параметр формы продукционной кривой (n)}
\CommentTok{\# n = 2 соответствует модели Шефера (симметричная кривая)}
\CommentTok{\# Фиксируем n = 2, так как данных недостаточно для его оценки}
\NormalTok{inp}\SpecialCharTok{$}\NormalTok{priors}\SpecialCharTok{$}\NormalTok{logn }\OtherTok{\textless{}{-}} \FunctionTok{c}\NormalTok{(}\FunctionTok{log}\NormalTok{(}\DecValTok{2}\NormalTok{), }\FloatTok{0.1}\NormalTok{, }\DecValTok{1}\NormalTok{)  }\CommentTok{\# (среднее, SD, использовать?)}
\NormalTok{inp}\SpecialCharTok{$}\NormalTok{ini}\SpecialCharTok{$}\NormalTok{logn }\OtherTok{\textless{}{-}} \FunctionTok{log}\NormalTok{(}\DecValTok{2}\NormalTok{)                 }\CommentTok{\# Начальное значение}
\NormalTok{inp}\SpecialCharTok{$}\NormalTok{phases}\SpecialCharTok{$}\NormalTok{logn }\OtherTok{\textless{}{-}} \SpecialCharTok{{-}}\DecValTok{1}                  \CommentTok{\# {-}1 означает не оценивать}

\DocumentationTok{\#\#\# 4.1.2 Априор для несущей способности (K)}
\CommentTok{\# K {-} максимальная биомасса, которую может поддерживать среда}
\CommentTok{\# Используем информативный априор на основе экспертных оценок}
\NormalTok{inp}\SpecialCharTok{$}\NormalTok{priors}\SpecialCharTok{$}\NormalTok{logK }\OtherTok{\textless{}{-}} \FunctionTok{c}\NormalTok{(}\FunctionTok{log}\NormalTok{(}\DecValTok{150}\NormalTok{), }\FloatTok{0.7}\NormalTok{, }\DecValTok{1}\NormalTok{)  }\CommentTok{\# log(150) тыс. тонн, CV ≈ 70\%}

\DocumentationTok{\#\#\# 4.1.3 Априор для начального состояния запаса}
\CommentTok{\# logbkfrac = log(B\_начальное/K)}
\CommentTok{\# 0.75 означает, что в начале временного ряда запас был на 75\% от K}
\NormalTok{inp}\SpecialCharTok{$}\NormalTok{priors}\SpecialCharTok{$}\NormalTok{logbkfrac }\OtherTok{\textless{}{-}} \FunctionTok{c}\NormalTok{(}\FunctionTok{log}\NormalTok{(}\FloatTok{0.75}\NormalTok{), }\FloatTok{0.25}\NormalTok{, }\DecValTok{1}\NormalTok{)}

\DocumentationTok{\#\#\# 4.1.4 Априоры для параметров роста (опционально)}
\CommentTok{\# Если есть информация о темпе роста популяции}
\CommentTok{\# inp$priors$logr \textless{}{-} c(log(0.3), 0.5, 1)}

\DocumentationTok{\#\# 4.2 Настройка неопределенности данных}

\DocumentationTok{\#\#\# 4.2.1 Увеличение неопределенности для последних наблюдений}
\CommentTok{\# Последние данные часто предварительные и менее надежные}
\NormalTok{inp}\SpecialCharTok{$}\NormalTok{stdevfacC[}\FunctionTok{length}\NormalTok{(inp}\SpecialCharTok{$}\NormalTok{stdevfacC)] }\OtherTok{\textless{}{-}} \DecValTok{2}      \CommentTok{\# Удваиваем SD для последнего вылова}
\NormalTok{inp}\SpecialCharTok{$}\NormalTok{stdevfacI[[}\DecValTok{2}\NormalTok{]][}\FunctionTok{length}\NormalTok{(inp}\SpecialCharTok{$}\NormalTok{stdevfacI[[}\DecValTok{2}\NormalTok{]])] }\OtherTok{\textless{}{-}} \DecValTok{2}  \CommentTok{\# Для последнего BESS}

\DocumentationTok{\#\#\# 4.2.2 Установка минимальной неопределенности (опционально)}
\CommentTok{\# inp$stdevfacC \textless{}{-} pmax(inp$stdevfacC, 0.2)  \# Минимум 20\% CV}

\DocumentationTok{\#\# 4.3 Технические настройки модели}

\DocumentationTok{\#\#\# 4.3.1 Временной шаг для численного интегрирования}
\CommentTok{\# Меньшие значения = выше точность, но медленнее расчет}
\NormalTok{inp}\SpecialCharTok{$}\NormalTok{dteuler }\OtherTok{\textless{}{-}} \DecValTok{1}\SpecialCharTok{/}\DecValTok{16}  \CommentTok{\# 16 шагов в году}

\DocumentationTok{\#\#\# 4.3.2 Включение расчета матрицы ковариации}
\CommentTok{\# Необходимо для оценки неопределенности и диагностики}
\NormalTok{inp}\SpecialCharTok{$}\NormalTok{getJointPrecision }\OtherTok{\textless{}{-}} \ConstantTok{TRUE}

\DocumentationTok{\#\#\# 4.3.3 Робастность оценок (опционально)}
\CommentTok{\# inp$robflag \textless{}{-} TRUE  \# Устойчивость к выбросам}

\CommentTok{\# {-}{-}{-}{-}{-}{-}{-}{-}{-}{-}{-}{-}{-}{-}{-}{-}{-}{-}{-} 5. ЗАПУСК МОДЕЛИ {-}{-}{-}{-}{-}{-}{-}{-}{-}{-}{-}{-}{-}{-}{-}{-}{-}{-}{-}{-}}

\FunctionTok{cat}\NormalTok{(}\StringTok{"}\SpecialCharTok{\textbackslash{}n}\StringTok{========== ОЦЕНКА ПАРАМЕТРОВ МОДЕЛИ ==========}\SpecialCharTok{\textbackslash{}n}\StringTok{"}\NormalTok{)}
\end{Highlighting}
\end{Shaded}

\begin{verbatim}

========== ОЦЕНКА ПАРАМЕТРОВ МОДЕЛИ ==========
\end{verbatim}

\begin{Shaded}
\begin{Highlighting}[]
\DocumentationTok{\#\# 5.1 Настройка оптимизатора}
\CommentTok{\# Увеличиваем лимиты итераций для сложных случаев}
\NormalTok{inp}\SpecialCharTok{$}\NormalTok{optimiser.control }\OtherTok{=} \FunctionTok{list}\NormalTok{(}
  \AttributeTok{iter.max =} \FloatTok{1e5}\NormalTok{,    }\CommentTok{\# Максимум итераций}
  \AttributeTok{eval.max =} \FloatTok{1e5}\NormalTok{,    }\CommentTok{\# Максимум вычислений функции}
  \AttributeTok{rel.tol =} \FloatTok{1e{-}10}    \CommentTok{\# Относительная точность}
\NormalTok{)}

\DocumentationTok{\#\# 5.2 Подгонка модели}
\CommentTok{\# Основная функция для оценки параметров}
\FunctionTok{cat}\NormalTok{(}\StringTok{"Запуск оптимизации...}\SpecialCharTok{\textbackslash{}n}\StringTok{"}\NormalTok{)}
\end{Highlighting}
\end{Shaded}

\begin{verbatim}
Запуск оптимизации...
\end{verbatim}

\begin{Shaded}
\begin{Highlighting}[]
\NormalTok{fit }\OtherTok{\textless{}{-}} \FunctionTok{fit.spict}\NormalTok{(inp)}

\DocumentationTok{\#\# 5.3 Проверка сходимости}
\ControlFlowTok{if}\NormalTok{ (fit}\SpecialCharTok{$}\NormalTok{opt}\SpecialCharTok{$}\NormalTok{convergence }\SpecialCharTok{==} \DecValTok{0}\NormalTok{) \{}
  \FunctionTok{cat}\NormalTok{(}\StringTok{"✓ Модель успешно сошлась}\SpecialCharTok{\textbackslash{}n}\StringTok{"}\NormalTok{)}
\NormalTok{\} }\ControlFlowTok{else}\NormalTok{ \{}
  \FunctionTok{cat}\NormalTok{(}\StringTok{"⚠ Проблемы со сходимостью. Код:"}\NormalTok{, fit}\SpecialCharTok{$}\NormalTok{opt}\SpecialCharTok{$}\NormalTok{convergence, }\StringTok{"}\SpecialCharTok{\textbackslash{}n}\StringTok{"}\NormalTok{)}
  \FunctionTok{cat}\NormalTok{(}\StringTok{"Сообщение:"}\NormalTok{, fit}\SpecialCharTok{$}\NormalTok{opt}\SpecialCharTok{$}\NormalTok{message, }\StringTok{"}\SpecialCharTok{\textbackslash{}n}\StringTok{"}\NormalTok{)}
\NormalTok{\}}
\end{Highlighting}
\end{Shaded}

\begin{verbatim}
<U+2713> Модель успешно сошлась
\end{verbatim}

\begin{Shaded}
\begin{Highlighting}[]
\DocumentationTok{\#\# 5.4 Добавление остатков OSA для диагностики}
\CommentTok{\# OSA (One{-}Step{-}Ahead) остатки используются для проверки модели}
\NormalTok{fit }\OtherTok{\textless{}{-}} \FunctionTok{calc.osa.resid}\NormalTok{(fit)}

\DocumentationTok{\#\# 5.5 Вывод основных результатов}
\FunctionTok{print}\NormalTok{(}\FunctionTok{summary}\NormalTok{(fit))}
\end{Highlighting}
\end{Shaded}

\begin{verbatim}
Convergence: 0  MSG: relative convergence (4)
Objective function at optimum: -4.835549
Euler time step (years):  1/16 or 0.0625
Nobs C: 20,  Nobs I1: 20,  Nobs I2: 19

Residual diagnostics (p-values)
    shapiro   bias    acf   LBox shapiro bias acf LBox  
 C   0.1269 0.4387 0.0562 0.1348       -    -   .    -  
 I1  0.9555 0.7813 0.3360 0.6800       -    -   -    -  
 I2  0.9825 0.9472 0.1390 0.3601       -    -   -    -  

Priors
      logn  ~  dnorm[log(2), 0.1^2]
  logalpha  ~  dnorm[log(1), 2^2]
   logbeta  ~  dnorm[log(1), 2^2]
      logK  ~  dnorm[log(150), 0.7^2]
 logbkfrac  ~  dnorm[log(0.75), 0.25^2]

Fixed parameters
   fixed.value  
 n           2  

Model parameter estimates w 95% CI 
         estimate      cilow     ciupp log.est  
 alpha1  11.94775   2.680587  53.25274  2.4805  
 alpha2   5.06977   1.127092  22.80433  1.6233  
 beta     0.18792   0.037152   0.95057 -1.6717  
 r        0.37684   0.289449   0.49063 -0.9759  
 rc       0.37684   0.289449   0.49063 -0.9759  
 rold     0.37684   0.289449   0.49063 -0.9759  
 m       17.85915  16.267157  19.60694  2.8825  
 K      189.56564 153.817727 233.62154  5.2447  
 q1       0.14458   0.111028   0.18827 -1.9339  
 q2       0.11048   0.086140   0.14168 -2.2030  
 sdb      0.01792   0.004082   0.07867 -4.0219  
 sdf      0.32055   0.218175   0.47097 -1.1377  
 sdi1     0.21410   0.156324   0.29322 -1.5413  
 sdi2     0.09085   0.064838   0.12729 -2.3986  
 sdc      0.06024   0.014357   0.25275 -2.8094  
 
Deterministic reference points (Drp)
       estimate   cilow    ciupp log.est  
 Bmsyd  94.7828 76.9089 116.8108   4.552  
 Fmsyd   0.1884  0.1447   0.2453  -1.669  
 MSYd   17.8591 16.2672  19.6069   2.883  
Stochastic reference points (Srp)
       estimate   cilow    ciupp log.est rel.diff.Drp  
 Bmsys  94.7336 76.8594 116.7646   4.551   -0.0005195  
 Fmsys   0.1883  0.1447   0.2452  -1.669   -0.0004215  
 MSYs   17.8423 16.2533  19.5868   2.882   -0.0009416  

States w 95% CI (inp$msytype: s)
                estimate    cilow    ciupp log.est  
 B_2024.94      118.8550 96.94825 145.7118  4.7779  
 F_2024.94        0.1032  0.06467   0.1646 -2.2713  
 B_2024.94/Bmsy   1.2546  1.08443   1.4515  0.2268  
 F_2024.94/Fmsy   0.5478  0.34183   0.8779 -0.6018  

Predictions w 95% CI (inp$msytype: s)
                prediction     cilow    ciupp log.est  
 B_2026.00        123.1089 100.33664 151.0495  4.8131  
 F_2026.00          0.1032   0.04643   0.2293 -2.2713  
 B_2026.00/Bmsy     1.2995   1.11491   1.5147  0.2620  
 F_2026.00/Fmsy     0.5478   0.24587   1.2206 -0.6018  
 Catch_2025.00     12.4910   7.30335  21.3633  2.5250  
 E(B_inf)         137.4812        NA       NA  4.9235  
NULL
\end{verbatim}

\begin{Shaded}
\begin{Highlighting}[]
\CommentTok{\# {-}{-}{-}{-}{-}{-}{-}{-}{-}{-}{-}{-}{-}{-}{-}{-}{-}{-}{-} 6. ДИАГНОСТИКА МОДЕЛИ {-}{-}{-}{-}{-}{-}{-}{-}{-}{-}{-}{-}{-}{-}{-}{-}{-}{-}{-}{-}}

\FunctionTok{cat}\NormalTok{(}\StringTok{"}\SpecialCharTok{\textbackslash{}n}\StringTok{========== ДИАГНОСТИКА МОДЕЛИ ==========}\SpecialCharTok{\textbackslash{}n}\StringTok{"}\NormalTok{)}
\end{Highlighting}
\end{Shaded}

\begin{verbatim}

========== ДИАГНОСТИКА МОДЕЛИ ==========
\end{verbatim}

\begin{Shaded}
\begin{Highlighting}[]
\DocumentationTok{\#\# 6.1 Проверка остатков}

\DocumentationTok{\#\#\# 6.1.1 Тест Шапиро{-}Уилка на нормальность}
\FunctionTok{cat}\NormalTok{(}\StringTok{"}\SpecialCharTok{\textbackslash{}n}\StringTok{{-}{-}{-} Тест на нормальность остатков (Shapiro{-}Wilk) {-}{-}{-}}\SpecialCharTok{\textbackslash{}n}\StringTok{"}\NormalTok{)}
\end{Highlighting}
\end{Shaded}

\begin{verbatim}

--- Тест на нормальность остатков (Shapiro-Wilk) ---
\end{verbatim}

\begin{Shaded}
\begin{Highlighting}[]
\FunctionTok{cat}\NormalTok{(}\FunctionTok{sprintf}\NormalTok{(}\StringTok{"Уловы (C): p{-}value = \%.4f \%s}\SpecialCharTok{\textbackslash{}n}\StringTok{"}\NormalTok{, }
\NormalTok{            fit}\SpecialCharTok{$}\NormalTok{diagn}\SpecialCharTok{$}\NormalTok{shapiroC.p, }
            \FunctionTok{ifelse}\NormalTok{(fit}\SpecialCharTok{$}\NormalTok{diagn}\SpecialCharTok{$}\NormalTok{shapiroC.p }\SpecialCharTok{\textgreater{}} \FloatTok{0.05}\NormalTok{, }\StringTok{"✓"}\NormalTok{, }\StringTok{"⚠"}\NormalTok{)))}
\end{Highlighting}
\end{Shaded}

\begin{verbatim}
Уловы (C): p-value = 0.1269 <U+2713>
\end{verbatim}

\begin{Shaded}
\begin{Highlighting}[]
\FunctionTok{cat}\NormalTok{(}\FunctionTok{sprintf}\NormalTok{(}\StringTok{"Индекс 1 (I1): p{-}value = \%.4f \%s}\SpecialCharTok{\textbackslash{}n}\StringTok{"}\NormalTok{, }
\NormalTok{            fit}\SpecialCharTok{$}\NormalTok{diagn}\SpecialCharTok{$}\NormalTok{shapiroI1.p, }
            \FunctionTok{ifelse}\NormalTok{(fit}\SpecialCharTok{$}\NormalTok{diagn}\SpecialCharTok{$}\NormalTok{shapiroI1.p }\SpecialCharTok{\textgreater{}} \FloatTok{0.05}\NormalTok{, }\StringTok{"✓"}\NormalTok{, }\StringTok{"⚠"}\NormalTok{)))}
\end{Highlighting}
\end{Shaded}

\begin{verbatim}
Индекс 1 (I1): p-value = 0.9555 <U+2713>
\end{verbatim}

\begin{Shaded}
\begin{Highlighting}[]
\FunctionTok{cat}\NormalTok{(}\FunctionTok{sprintf}\NormalTok{(}\StringTok{"Индекс 2 (I2): p{-}value = \%.4f \%s}\SpecialCharTok{\textbackslash{}n}\StringTok{"}\NormalTok{, }
\NormalTok{            fit}\SpecialCharTok{$}\NormalTok{diagn}\SpecialCharTok{$}\NormalTok{shapiroI2.p, }
            \FunctionTok{ifelse}\NormalTok{(fit}\SpecialCharTok{$}\NormalTok{diagn}\SpecialCharTok{$}\NormalTok{shapiroI2.p }\SpecialCharTok{\textgreater{}} \FloatTok{0.05}\NormalTok{, }\StringTok{"✓"}\NormalTok{, }\StringTok{"⚠"}\NormalTok{)))}
\end{Highlighting}
\end{Shaded}

\begin{verbatim}
Индекс 2 (I2): p-value = 0.9825 <U+2713>
\end{verbatim}

\begin{Shaded}
\begin{Highlighting}[]
\DocumentationTok{\#\#\# 6.1.2 Проверка автокорреляции (Ljung{-}Box тест)}
\FunctionTok{cat}\NormalTok{(}\StringTok{"}\SpecialCharTok{\textbackslash{}n}\StringTok{{-}{-}{-} Проверка автокорреляции (Ljung{-}Box тест) {-}{-}{-}}\SpecialCharTok{\textbackslash{}n}\StringTok{"}\NormalTok{)}
\end{Highlighting}
\end{Shaded}

\begin{verbatim}

--- Проверка автокорреляции (Ljung-Box тест) ---
\end{verbatim}

\begin{Shaded}
\begin{Highlighting}[]
\FunctionTok{cat}\NormalTok{(}\FunctionTok{sprintf}\NormalTok{(}\StringTok{"Уловы (C): p{-}value = \%.4f \%s}\SpecialCharTok{\textbackslash{}n}\StringTok{"}\NormalTok{, }
\NormalTok{            fit}\SpecialCharTok{$}\NormalTok{diagn}\SpecialCharTok{$}\NormalTok{LBoxC.p, }
            \FunctionTok{ifelse}\NormalTok{(fit}\SpecialCharTok{$}\NormalTok{diagn}\SpecialCharTok{$}\NormalTok{LBoxC.p }\SpecialCharTok{\textgreater{}} \FloatTok{0.05}\NormalTok{, }\StringTok{"✓"}\NormalTok{, }\StringTok{"⚠"}\NormalTok{)))}
\end{Highlighting}
\end{Shaded}

\begin{verbatim}
Уловы (C): p-value = 0.1348 <U+2713>
\end{verbatim}

\begin{Shaded}
\begin{Highlighting}[]
\FunctionTok{cat}\NormalTok{(}\FunctionTok{sprintf}\NormalTok{(}\StringTok{"Индекс 1 (I1): p{-}value = \%.4f \%s}\SpecialCharTok{\textbackslash{}n}\StringTok{"}\NormalTok{, }
\NormalTok{            fit}\SpecialCharTok{$}\NormalTok{diagn}\SpecialCharTok{$}\NormalTok{LBoxI1.p, }
            \FunctionTok{ifelse}\NormalTok{(fit}\SpecialCharTok{$}\NormalTok{diagn}\SpecialCharTok{$}\NormalTok{LBoxI1.p }\SpecialCharTok{\textgreater{}} \FloatTok{0.05}\NormalTok{, }\StringTok{"✓"}\NormalTok{, }\StringTok{"⚠"}\NormalTok{)))}
\end{Highlighting}
\end{Shaded}

\begin{verbatim}
Индекс 1 (I1): p-value = 0.6800 <U+2713>
\end{verbatim}

\begin{Shaded}
\begin{Highlighting}[]
\FunctionTok{cat}\NormalTok{(}\FunctionTok{sprintf}\NormalTok{(}\StringTok{"Индекс 2 (I2): p{-}value = \%.4f \%s}\SpecialCharTok{\textbackslash{}n}\StringTok{"}\NormalTok{, }
\NormalTok{            fit}\SpecialCharTok{$}\NormalTok{diagn}\SpecialCharTok{$}\NormalTok{LBoxI2.p, }
            \FunctionTok{ifelse}\NormalTok{(fit}\SpecialCharTok{$}\NormalTok{diagn}\SpecialCharTok{$}\NormalTok{LBoxI2.p }\SpecialCharTok{\textgreater{}} \FloatTok{0.05}\NormalTok{, }\StringTok{"✓"}\NormalTok{, }\StringTok{"⚠"}\NormalTok{)))}
\end{Highlighting}
\end{Shaded}

\begin{verbatim}
Индекс 2 (I2): p-value = 0.3601 <U+2713>
\end{verbatim}

\begin{Shaded}
\begin{Highlighting}[]
\DocumentationTok{\#\#\# 6.1.3 Дополнительные диагностические тесты}
\FunctionTok{cat}\NormalTok{(}\StringTok{"}\SpecialCharTok{\textbackslash{}n}\StringTok{{-}{-}{-} Дополнительная диагностика {-}{-}{-}}\SpecialCharTok{\textbackslash{}n}\StringTok{"}\NormalTok{)}
\end{Highlighting}
\end{Shaded}

\begin{verbatim}

--- Дополнительная диагностика ---
\end{verbatim}

\begin{Shaded}
\begin{Highlighting}[]
\FunctionTok{cat}\NormalTok{(}\FunctionTok{sprintf}\NormalTok{(}\StringTok{"Смещение уловов (biasC): p{-}value = \%.4f \%s}\SpecialCharTok{\textbackslash{}n}\StringTok{"}\NormalTok{, }
\NormalTok{            fit}\SpecialCharTok{$}\NormalTok{diagn}\SpecialCharTok{$}\NormalTok{biasC.p, }
            \FunctionTok{ifelse}\NormalTok{(fit}\SpecialCharTok{$}\NormalTok{diagn}\SpecialCharTok{$}\NormalTok{biasC.p }\SpecialCharTok{\textgreater{}} \FloatTok{0.05}\NormalTok{, }\StringTok{"✓"}\NormalTok{, }\StringTok{"⚠"}\NormalTok{)))}
\end{Highlighting}
\end{Shaded}

\begin{verbatim}
Смещение уловов (biasC): p-value = 0.4387 <U+2713>
\end{verbatim}

\begin{Shaded}
\begin{Highlighting}[]
\FunctionTok{cat}\NormalTok{(}\FunctionTok{sprintf}\NormalTok{(}\StringTok{"Автокорреляция уловов (acfC): p{-}value = \%.4f \%s}\SpecialCharTok{\textbackslash{}n}\StringTok{"}\NormalTok{, }
\NormalTok{            fit}\SpecialCharTok{$}\NormalTok{diagn}\SpecialCharTok{$}\NormalTok{acfC.p, }
            \FunctionTok{ifelse}\NormalTok{(fit}\SpecialCharTok{$}\NormalTok{diagn}\SpecialCharTok{$}\NormalTok{acfC.p }\SpecialCharTok{\textgreater{}} \FloatTok{0.05}\NormalTok{, }\StringTok{"✓"}\NormalTok{, }\StringTok{"⚠"}\NormalTok{)))}
\end{Highlighting}
\end{Shaded}

\begin{verbatim}
Автокорреляция уловов (acfC): p-value = 0.0562 <U+2713>
\end{verbatim}

\begin{Shaded}
\begin{Highlighting}[]
\DocumentationTok{\#\# 6.2 Визуальная диагностика}
\CommentTok{\# Создаем диагностические графики}
\end{Highlighting}
\end{Shaded}

\begin{Shaded}
\begin{Highlighting}[]
\FunctionTok{plotspict.diagnostic}\NormalTok{(fit)}
\end{Highlighting}
\end{Shaded}

\pandocbounded{\includegraphics[keepaspectratio]{chapter13_files/figure-pdf/unnamed-chunk-2-1.pdf}}

\begin{Shaded}
\begin{Highlighting}[]
\DocumentationTok{\#\# 6.3 Ретроспективный анализ}
\CommentTok{\# Проверяет устойчивость оценок при удалении последних лет}
\FunctionTok{cat}\NormalTok{(}\StringTok{"}\SpecialCharTok{\textbackslash{}n}\StringTok{{-}{-}{-} Ретроспективный анализ {-}{-}{-}}\SpecialCharTok{\textbackslash{}n}\StringTok{"}\NormalTok{)}
\end{Highlighting}
\end{Shaded}

\begin{verbatim}

--- Ретроспективный анализ ---
\end{verbatim}

\begin{Shaded}
\begin{Highlighting}[]
\NormalTok{ret }\OtherTok{\textless{}{-}} \FunctionTok{retro}\NormalTok{(fit, }\AttributeTok{nretroyear =} \DecValTok{5}\NormalTok{)}
\FunctionTok{plotspict.retro}\NormalTok{(ret)}
\end{Highlighting}
\end{Shaded}

\pandocbounded{\includegraphics[keepaspectratio]{chapter13_files/figure-pdf/unnamed-chunk-3-1.pdf}}

\begin{verbatim}
     FFmsy      BBmsy 
 0.0029188 -0.0002271 
\end{verbatim}

\begin{Shaded}
\begin{Highlighting}[]
\CommentTok{\# Расчет ретро{-}смещения (Mohn\textquotesingle{}s rho)}
\NormalTok{rho }\OtherTok{\textless{}{-}} \FunctionTok{mohns\_rho}\NormalTok{(ret)}
\FunctionTok{cat}\NormalTok{(}\StringTok{"Mohn\textquotesingle{}s rho для биомассы:"}\NormalTok{, }\FunctionTok{round}\NormalTok{(rho[}\StringTok{"BBmsy"}\NormalTok{], }\DecValTok{4}\NormalTok{), }\StringTok{"}\SpecialCharTok{\textbackslash{}n}\StringTok{"}\NormalTok{)}
\end{Highlighting}
\end{Shaded}

\begin{verbatim}
Mohn's rho для биомассы: -0.0002 
\end{verbatim}

\begin{Shaded}
\begin{Highlighting}[]
\FunctionTok{cat}\NormalTok{(}\StringTok{"Mohn\textquotesingle{}s rho для F:"}\NormalTok{, }\FunctionTok{round}\NormalTok{(rho[}\StringTok{"FFmsy"}\NormalTok{], }\DecValTok{4}\NormalTok{), }\StringTok{"}\SpecialCharTok{\textbackslash{}n}\StringTok{"}\NormalTok{)}
\end{Highlighting}
\end{Shaded}

\begin{verbatim}
Mohn's rho для F: 0.0029 
\end{verbatim}

\begin{Shaded}
\begin{Highlighting}[]
\FunctionTok{cat}\NormalTok{(}\StringTok{"Приемлемые значения: |rho| \textless{} 0.2}\SpecialCharTok{\textbackslash{}n}\StringTok{"}\NormalTok{)}
\end{Highlighting}
\end{Shaded}

\begin{verbatim}
Приемлемые значения: |rho| < 0.2
\end{verbatim}

\begin{Shaded}
\begin{Highlighting}[]
\DocumentationTok{\#\# 6.4 Анализ чувствительности к априорным распределениям}
\FunctionTok{cat}\NormalTok{(}\StringTok{"}\SpecialCharTok{\textbackslash{}n}\StringTok{{-}{-}{-} Анализ чувствительности {-}{-}{-}}\SpecialCharTok{\textbackslash{}n}\StringTok{"}\NormalTok{)}
\end{Highlighting}
\end{Shaded}

\begin{verbatim}

--- Анализ чувствительности ---
\end{verbatim}

\begin{Shaded}
\begin{Highlighting}[]
\CommentTok{\# Тест без априоров}
\NormalTok{inp\_no\_prior }\OtherTok{\textless{}{-}}\NormalTok{ inp}
\NormalTok{inp\_no\_prior}\SpecialCharTok{$}\NormalTok{priors }\OtherTok{\textless{}{-}} \FunctionTok{list}\NormalTok{()}
\NormalTok{fit\_no\_prior }\OtherTok{\textless{}{-}} \FunctionTok{fit.spict}\NormalTok{(inp\_no\_prior)}

\CommentTok{\# Сравнение оценок}
\FunctionTok{cat}\NormalTok{(}\StringTok{"Изменение оценок без априоров:}\SpecialCharTok{\textbackslash{}n}\StringTok{"}\NormalTok{)}
\end{Highlighting}
\end{Shaded}

\begin{verbatim}
Изменение оценок без априоров:
\end{verbatim}

\begin{Shaded}
\begin{Highlighting}[]
\FunctionTok{cat}\NormalTok{(}\StringTok{"K:"}\NormalTok{, }\FunctionTok{round}\NormalTok{((}\FunctionTok{exp}\NormalTok{(fit\_no\_prior}\SpecialCharTok{$}\NormalTok{par.fixed[}\StringTok{"logK"}\NormalTok{]) }\SpecialCharTok{{-}} 
               \FunctionTok{exp}\NormalTok{(fit}\SpecialCharTok{$}\NormalTok{par.fixed[}\StringTok{"logK"}\NormalTok{])) }\SpecialCharTok{/} 
               \FunctionTok{exp}\NormalTok{(fit}\SpecialCharTok{$}\NormalTok{par.fixed[}\StringTok{"logK"}\NormalTok{]) }\SpecialCharTok{*} \DecValTok{100}\NormalTok{, }\DecValTok{1}\NormalTok{), }\StringTok{"\%}\SpecialCharTok{\textbackslash{}n}\StringTok{"}\NormalTok{)}
\end{Highlighting}
\end{Shaded}

\begin{verbatim}
K: 0.4 %
\end{verbatim}

\begin{Shaded}
\begin{Highlighting}[]
\DocumentationTok{\#\# 6.5 Сравнение априорных и апостериорных распределений (Проверка профилей правдоподобия) }
\end{Highlighting}
\end{Shaded}

\begin{Shaded}
\begin{Highlighting}[]
\CommentTok{\# Помогает оценить идентифицируемость параметров}
\FunctionTok{par}\NormalTok{(}\AttributeTok{mfrow =} \FunctionTok{c}\NormalTok{(}\DecValTok{2}\NormalTok{, }\DecValTok{2}\NormalTok{))}
\FunctionTok{plotspict.priors}\NormalTok{(fit)}
\end{Highlighting}
\end{Shaded}

\pandocbounded{\includegraphics[keepaspectratio]{chapter13_files/figure-pdf/unnamed-chunk-4-1.pdf}}

\begin{Shaded}
\begin{Highlighting}[]
\FunctionTok{par}\NormalTok{(}\AttributeTok{mfrow =} \FunctionTok{c}\NormalTok{(}\DecValTok{1}\NormalTok{, }\DecValTok{1}\NormalTok{))}
\end{Highlighting}
\end{Shaded}

\begin{Shaded}
\begin{Highlighting}[]
\CommentTok{\# {-}{-}{-}{-}{-}{-}{-}{-}{-}{-}{-}{-}{-}{-}{-}{-}{-}{-}{-} 7. ИНТЕРПРЕТАЦИЯ РЕЗУЛЬТАТОВ {-}{-}{-}{-}{-}{-}{-}{-}{-}{-}{-}{-}{-}{-}{-}{-}{-}{-}{-}{-}}

\FunctionTok{cat}\NormalTok{(}\StringTok{"}\SpecialCharTok{\textbackslash{}n}\StringTok{========== ИНТЕРПРЕТАЦИЯ РЕЗУЛЬТАТОВ ==========}\SpecialCharTok{\textbackslash{}n}\StringTok{"}\NormalTok{)}
\end{Highlighting}
\end{Shaded}

\begin{verbatim}

========== ИНТЕРПРЕТАЦИЯ РЕЗУЛЬТАТОВ ==========
\end{verbatim}

\begin{Shaded}
\begin{Highlighting}[]
\DocumentationTok{\#\# 7.1 Извлечение ключевых параметров}
\NormalTok{get\_estimate }\OtherTok{\textless{}{-}} \ControlFlowTok{function}\NormalTok{(fit, param) \{}
\NormalTok{  val }\OtherTok{\textless{}{-}} \FunctionTok{get.par}\NormalTok{(param, fit, }\AttributeTok{exp =} \ConstantTok{TRUE}\NormalTok{)}
  \FunctionTok{return}\NormalTok{(}\FunctionTok{c}\NormalTok{(}\AttributeTok{estimate =}\NormalTok{ val[}\DecValTok{1}\NormalTok{], }\AttributeTok{lower =}\NormalTok{ val[}\DecValTok{2}\NormalTok{], }\AttributeTok{upper =}\NormalTok{ val[}\DecValTok{3}\NormalTok{]))}
\NormalTok{\}}

\DocumentationTok{\#\# 7.2 Параметры модели}
\FunctionTok{cat}\NormalTok{(}\StringTok{"}\SpecialCharTok{\textbackslash{}n}\StringTok{{-}{-}{-} Оценки параметров модели {-}{-}{-}}\SpecialCharTok{\textbackslash{}n}\StringTok{"}\NormalTok{)}
\end{Highlighting}
\end{Shaded}

\begin{verbatim}

--- Оценки параметров модели ---
\end{verbatim}

\begin{Shaded}
\begin{Highlighting}[]
\NormalTok{r\_est }\OtherTok{\textless{}{-}} \FunctionTok{get\_estimate}\NormalTok{(fit, }\StringTok{"logr"}\NormalTok{)}
\FunctionTok{cat}\NormalTok{(}\FunctionTok{sprintf}\NormalTok{(}\StringTok{"r (темп роста): \%.3f [\%.3f {-} \%.3f]}\SpecialCharTok{\textbackslash{}n}\StringTok{"}\NormalTok{, }
\NormalTok{            r\_est[}\DecValTok{2}\NormalTok{], r\_est[}\DecValTok{1}\NormalTok{], r\_est[}\DecValTok{3}\NormalTok{]))}
\end{Highlighting}
\end{Shaded}

\begin{verbatim}
r (темп роста): 0.377 [0.289 - 0.491]
\end{verbatim}

\begin{Shaded}
\begin{Highlighting}[]
\NormalTok{K\_est }\OtherTok{\textless{}{-}} \FunctionTok{get\_estimate}\NormalTok{(fit, }\StringTok{"logK"}\NormalTok{)}
\FunctionTok{cat}\NormalTok{(}\FunctionTok{sprintf}\NormalTok{(}\StringTok{"K (несущая способность): \%.1f [\%.1f {-} \%.1f] тыс. т}\SpecialCharTok{\textbackslash{}n}\StringTok{"}\NormalTok{, }
\NormalTok{            K\_est[}\DecValTok{2}\NormalTok{], K\_est[}\DecValTok{1}\NormalTok{], K\_est[}\DecValTok{3}\NormalTok{]))}
\end{Highlighting}
\end{Shaded}

\begin{verbatim}
K (несущая способность): 189.6 [153.8 - 233.6] тыс. т
\end{verbatim}

\begin{Shaded}
\begin{Highlighting}[]
\DocumentationTok{\#\# 7.3 Ориентиры управления (Референсные точки)}
\FunctionTok{cat}\NormalTok{(}\StringTok{"}\SpecialCharTok{\textbackslash{}n}\StringTok{{-}{-}{-} Ориентиры управления (MSY) {-}{-}{-}}\SpecialCharTok{\textbackslash{}n}\StringTok{"}\NormalTok{)}
\end{Highlighting}
\end{Shaded}

\begin{verbatim}

--- Ориентиры управления (MSY) ---
\end{verbatim}

\begin{Shaded}
\begin{Highlighting}[]
\NormalTok{MSY }\OtherTok{\textless{}{-}} \FunctionTok{get\_estimate}\NormalTok{(fit, }\StringTok{"logMSY"}\NormalTok{)}
\FunctionTok{cat}\NormalTok{(}\FunctionTok{sprintf}\NormalTok{(}\StringTok{"MSY: \%.1f [\%.1f {-} \%.1f] тыс. т/год}\SpecialCharTok{\textbackslash{}n}\StringTok{"}\NormalTok{, }
\NormalTok{            MSY[}\DecValTok{2}\NormalTok{], MSY[}\DecValTok{1}\NormalTok{], MSY[}\DecValTok{3}\NormalTok{]))}
\end{Highlighting}
\end{Shaded}

\begin{verbatim}
MSY: 17.8 [16.3 - 19.6] тыс. т/год
\end{verbatim}

\begin{Shaded}
\begin{Highlighting}[]
\NormalTok{Bmsy }\OtherTok{\textless{}{-}} \FunctionTok{get\_estimate}\NormalTok{(fit, }\StringTok{"logBmsy"}\NormalTok{)}
\FunctionTok{cat}\NormalTok{(}\FunctionTok{sprintf}\NormalTok{(}\StringTok{"Bmsy: \%.1f [\%.1f {-} \%.1f] тыс. т}\SpecialCharTok{\textbackslash{}n}\StringTok{"}\NormalTok{, }
\NormalTok{            Bmsy[}\DecValTok{2}\NormalTok{], Bmsy[}\DecValTok{1}\NormalTok{], Bmsy[}\DecValTok{3}\NormalTok{]))}
\end{Highlighting}
\end{Shaded}

\begin{verbatim}
Bmsy: 94.7 [76.9 - 116.8] тыс. т
\end{verbatim}

\begin{Shaded}
\begin{Highlighting}[]
\NormalTok{Fmsy }\OtherTok{\textless{}{-}} \FunctionTok{get\_estimate}\NormalTok{(fit, }\StringTok{"logFmsy"}\NormalTok{)}
\FunctionTok{cat}\NormalTok{(}\FunctionTok{sprintf}\NormalTok{(}\StringTok{"Fmsy: \%.3f [\%.3f {-} \%.3f] год⁻¹}\SpecialCharTok{\textbackslash{}n}\StringTok{"}\NormalTok{, }
\NormalTok{            Fmsy[}\DecValTok{2}\NormalTok{], Fmsy[}\DecValTok{1}\NormalTok{], Fmsy[}\DecValTok{3}\NormalTok{]))}
\end{Highlighting}
\end{Shaded}

\begin{verbatim}
Fmsy: 0.188 [0.145 - 0.245] год<U+207B><U+00B9>
\end{verbatim}

\begin{Shaded}
\begin{Highlighting}[]
\DocumentationTok{\#\# 7.4 Текущее состояние запаса}
\FunctionTok{cat}\NormalTok{(}\StringTok{"}\SpecialCharTok{\textbackslash{}n}\StringTok{{-}{-}{-} Текущее состояние запаса (последний год) {-}{-}{-}}\SpecialCharTok{\textbackslash{}n}\StringTok{"}\NormalTok{)}
\end{Highlighting}
\end{Shaded}

\begin{verbatim}

--- Текущее состояние запаса (последний год) ---
\end{verbatim}

\begin{Shaded}
\begin{Highlighting}[]
\NormalTok{current\_year }\OtherTok{\textless{}{-}} \FunctionTok{max}\NormalTok{(inp}\SpecialCharTok{$}\NormalTok{timeC)}

\NormalTok{B\_current }\OtherTok{\textless{}{-}} \FunctionTok{get\_estimate}\NormalTok{(fit, }\StringTok{"logB"}\NormalTok{)}
\FunctionTok{cat}\NormalTok{(}\FunctionTok{sprintf}\NormalTok{(}\StringTok{"Биомасса: \%.1f [\%.1f {-} \%.1f] тыс. т}\SpecialCharTok{\textbackslash{}n}\StringTok{"}\NormalTok{, }
\NormalTok{            B\_current[}\DecValTok{2}\NormalTok{], B\_current[}\DecValTok{1}\NormalTok{], B\_current[}\DecValTok{3}\NormalTok{]))}
\end{Highlighting}
\end{Shaded}

\begin{verbatim}
Биомасса: 103.8 [103.0 - 104.6] тыс. т
\end{verbatim}

\begin{Shaded}
\begin{Highlighting}[]
\NormalTok{F\_current }\OtherTok{\textless{}{-}} \FunctionTok{get\_estimate}\NormalTok{(fit, }\StringTok{"logF"}\NormalTok{)}
\FunctionTok{cat}\NormalTok{(}\FunctionTok{sprintf}\NormalTok{(}\StringTok{"F: \%.3f [\%.3f {-} \%.3f] год⁻¹}\SpecialCharTok{\textbackslash{}n}\StringTok{"}\NormalTok{, }
\NormalTok{            F\_current[}\DecValTok{2}\NormalTok{], F\_current[}\DecValTok{1}\NormalTok{], F\_current[}\DecValTok{3}\NormalTok{]))}
\end{Highlighting}
\end{Shaded}

\begin{verbatim}
F: 0.019 [0.019 - 0.019] год<U+207B><U+00B9>
\end{verbatim}

\begin{Shaded}
\begin{Highlighting}[]
\CommentTok{\# Относительные показатели}
\NormalTok{B\_Bmsy }\OtherTok{\textless{}{-}} \FunctionTok{get\_estimate}\NormalTok{(fit, }\StringTok{"logBBmsy"}\NormalTok{)}
\FunctionTok{cat}\NormalTok{(}\FunctionTok{sprintf}\NormalTok{(}\StringTok{"B/Bmsy: \%.2f [\%.2f {-} \%.2f]}\SpecialCharTok{\textbackslash{}n}\StringTok{"}\NormalTok{, }
\NormalTok{            B\_Bmsy[}\DecValTok{2}\NormalTok{], B\_Bmsy[}\DecValTok{1}\NormalTok{], B\_Bmsy[}\DecValTok{3}\NormalTok{]))}
\end{Highlighting}
\end{Shaded}

\begin{verbatim}
B/Bmsy: 1.17 [1.16 - 1.18]
\end{verbatim}

\begin{Shaded}
\begin{Highlighting}[]
\NormalTok{F\_Fmsy }\OtherTok{\textless{}{-}} \FunctionTok{get\_estimate}\NormalTok{(fit, }\StringTok{"logFFmsy"}\NormalTok{)}
\FunctionTok{cat}\NormalTok{(}\FunctionTok{sprintf}\NormalTok{(}\StringTok{"F/Fmsy: \%.2f [\%.2f {-} \%.2f]}\SpecialCharTok{\textbackslash{}n}\StringTok{"}\NormalTok{, }
\NormalTok{            F\_Fmsy[}\DecValTok{2}\NormalTok{], F\_Fmsy[}\DecValTok{1}\NormalTok{], F\_Fmsy[}\DecValTok{3}\NormalTok{]))}
\end{Highlighting}
\end{Shaded}

\begin{verbatim}
F/Fmsy: 0.10 [0.10 - 0.11]
\end{verbatim}

\begin{Shaded}
\begin{Highlighting}[]
\CommentTok{\# Интерпретация состояния}
\ControlFlowTok{if}\NormalTok{ (B\_Bmsy[}\DecValTok{1}\NormalTok{] }\SpecialCharTok{\textgreater{}} \DecValTok{1} \SpecialCharTok{\&\&}\NormalTok{ F\_Fmsy[}\DecValTok{1}\NormalTok{] }\SpecialCharTok{\textless{}} \DecValTok{1}\NormalTok{) \{}
  \FunctionTok{cat}\NormalTok{(}\StringTok{"}\SpecialCharTok{\textbackslash{}n}\StringTok{✓ Запас в хорошем состоянии (зеленая зона)}\SpecialCharTok{\textbackslash{}n}\StringTok{"}\NormalTok{)}
\NormalTok{\} }\ControlFlowTok{else} \ControlFlowTok{if}\NormalTok{ (B\_Bmsy[}\DecValTok{1}\NormalTok{] }\SpecialCharTok{\textless{}} \FloatTok{0.5}\NormalTok{) \{}
  \FunctionTok{cat}\NormalTok{(}\StringTok{"}\SpecialCharTok{\textbackslash{}n}\StringTok{⚠ Запас истощен (красная зона)}\SpecialCharTok{\textbackslash{}n}\StringTok{"}\NormalTok{)}
\NormalTok{\} }\ControlFlowTok{else} \ControlFlowTok{if}\NormalTok{ (F\_Fmsy[}\DecValTok{1}\NormalTok{] }\SpecialCharTok{\textgreater{}} \DecValTok{1}\NormalTok{) \{}
  \FunctionTok{cat}\NormalTok{(}\StringTok{"}\SpecialCharTok{\textbackslash{}n}\StringTok{⚠ Происходит перелов (желтая зона)}\SpecialCharTok{\textbackslash{}n}\StringTok{"}\NormalTok{)}
\NormalTok{\} }\ControlFlowTok{else}\NormalTok{ \{}
  \FunctionTok{cat}\NormalTok{(}\StringTok{"}\SpecialCharTok{\textbackslash{}n}\StringTok{⚠ Запас в переходном состоянии}\SpecialCharTok{\textbackslash{}n}\StringTok{"}\NormalTok{)}
\NormalTok{\}}
\end{Highlighting}
\end{Shaded}

\begin{verbatim}

<U+2713> Запас в хорошем состоянии (зеленая зона)
\end{verbatim}

\begin{Shaded}
\begin{Highlighting}[]
\CommentTok{\# {-}{-}{-}{-}{-}{-}{-}{-}{-}{-}{-}{-}{-}{-}{-}{-}{-}{-}{-} 8. ГРАФИЧЕСКАЯ ВИЗУАЛИЗАЦИЯ {-}{-}{-}{-}{-}{-}{-}{-}{-}{-}{-}{-}{-}{-}{-}{-}{-}{-}{-}{-}}

\CommentTok{\#cat("\textbackslash{}n========== СОЗДАНИЕ ГРАФИКОВ ==========\textbackslash{}n")}

\DocumentationTok{\#\# 8.1 Стандартные графики SPiCT}
\CommentTok{\#pdf("SPiCT\_results.pdf", width = 12, height = 10)}

\CommentTok{\# График 1: Сводка результатов}
\FunctionTok{plot}\NormalTok{(fit)}
\end{Highlighting}
\end{Shaded}

\pandocbounded{\includegraphics[keepaspectratio]{chapter13_files/figure-pdf/unnamed-chunk-5-1.pdf}}

\begin{Shaded}
\begin{Highlighting}[]
\CommentTok{\# График 2: Временные ряды}
\FunctionTok{plotspict.biomass}\NormalTok{(fit)}
\end{Highlighting}
\end{Shaded}

\pandocbounded{\includegraphics[keepaspectratio]{chapter13_files/figure-pdf/unnamed-chunk-5-2.pdf}}

\begin{Shaded}
\begin{Highlighting}[]
\FunctionTok{plotspict.f}\NormalTok{(fit)}
\end{Highlighting}
\end{Shaded}

\pandocbounded{\includegraphics[keepaspectratio]{chapter13_files/figure-pdf/unnamed-chunk-5-3.pdf}}

\begin{Shaded}
\begin{Highlighting}[]
\FunctionTok{plotspict.catch}\NormalTok{(fit)}
\end{Highlighting}
\end{Shaded}

\pandocbounded{\includegraphics[keepaspectratio]{chapter13_files/figure-pdf/unnamed-chunk-5-4.pdf}}

\begin{Shaded}
\begin{Highlighting}[]
\CommentTok{\# График 3: Фазовая диаграмма Кобе}
\FunctionTok{plotspict.fb}\NormalTok{(fit)}
\end{Highlighting}
\end{Shaded}

\pandocbounded{\includegraphics[keepaspectratio]{chapter13_files/figure-pdf/unnamed-chunk-5-5.pdf}}

\begin{Shaded}
\begin{Highlighting}[]
\CommentTok{\# График 4: Продукционная кривая}
\FunctionTok{plotspict.production}\NormalTok{(fit)}
\end{Highlighting}
\end{Shaded}

\pandocbounded{\includegraphics[keepaspectratio]{chapter13_files/figure-pdf/unnamed-chunk-5-6.pdf}}

\begin{Shaded}
\begin{Highlighting}[]
\CommentTok{\#dev.off()}
\CommentTok{\#cat("Графики сохранены в файл \textquotesingle{}SPiCT\_results.pdf\textquotesingle{}\textbackslash{}n")}

\DocumentationTok{\#\# 8.2 Альтернативный подход {-} детальные графики}

\CommentTok{\# Функция для извлечения временных рядов из SPiCT}
\NormalTok{extract\_time\_series }\OtherTok{\textless{}{-}} \ControlFlowTok{function}\NormalTok{(fit, param\_name) \{}
  \CommentTok{\# Получаем оценки параметра}
\NormalTok{  param\_values }\OtherTok{\textless{}{-}} \FunctionTok{get.par}\NormalTok{(param\_name, fit, }\AttributeTok{exp =} \ConstantTok{TRUE}\NormalTok{)}
  
  \CommentTok{\# Временные точки}
\NormalTok{  time\_points }\OtherTok{\textless{}{-}}\NormalTok{ fit}\SpecialCharTok{$}\NormalTok{inp}\SpecialCharTok{$}\NormalTok{time}
  
  \CommentTok{\# Создаем dataframe}
\NormalTok{  df }\OtherTok{\textless{}{-}} \FunctionTok{data.frame}\NormalTok{(}
    \AttributeTok{time =}\NormalTok{ time\_points,}
    \AttributeTok{estimate =}\NormalTok{ param\_values[, }\StringTok{"est"}\NormalTok{],}
    \AttributeTok{lower =}\NormalTok{ param\_values[, }\StringTok{"ll"}\NormalTok{],}
    \AttributeTok{upper =}\NormalTok{ param\_values[, }\StringTok{"ul"}\NormalTok{]}
\NormalTok{  )}
  
  \FunctionTok{return}\NormalTok{(df)}
\NormalTok{\}}

\CommentTok{\# Извлекаем все нужные временные ряды}
\NormalTok{df\_B }\OtherTok{\textless{}{-}} \FunctionTok{extract\_time\_series}\NormalTok{(fit, }\StringTok{"logB"}\NormalTok{)}
\NormalTok{df\_F }\OtherTok{\textless{}{-}} \FunctionTok{extract\_time\_series}\NormalTok{(fit, }\StringTok{"logF"}\NormalTok{)}
\NormalTok{df\_BBmsy }\OtherTok{\textless{}{-}} \FunctionTok{extract\_time\_series}\NormalTok{(fit, }\StringTok{"logBBmsy"}\NormalTok{)}
\NormalTok{df\_FFmsy }\OtherTok{\textless{}{-}} \FunctionTok{extract\_time\_series}\NormalTok{(fit, }\StringTok{"logFFmsy"}\NormalTok{)}

\CommentTok{\# Создаем панель графиков}
\FunctionTok{library}\NormalTok{(gridExtra)}
\end{Highlighting}
\end{Shaded}

\begin{verbatim}

Присоединяю пакет: 'gridExtra'
\end{verbatim}

\begin{verbatim}
Следующий объект скрыт от 'package:dplyr':

    combine
\end{verbatim}

\begin{Shaded}
\begin{Highlighting}[]
\CommentTok{\# График 1: Биомасса}
\NormalTok{g1 }\OtherTok{\textless{}{-}} \FunctionTok{ggplot}\NormalTok{(df\_B, }\FunctionTok{aes}\NormalTok{(}\AttributeTok{x =}\NormalTok{ time)) }\SpecialCharTok{+}
  \FunctionTok{geom\_ribbon}\NormalTok{(}\FunctionTok{aes}\NormalTok{(}\AttributeTok{ymin =}\NormalTok{ lower, }\AttributeTok{ymax =}\NormalTok{ upper), }\AttributeTok{alpha =} \FloatTok{0.3}\NormalTok{, }\AttributeTok{fill =} \StringTok{"blue"}\NormalTok{) }\SpecialCharTok{+}
  \FunctionTok{geom\_line}\NormalTok{(}\FunctionTok{aes}\NormalTok{(}\AttributeTok{y =}\NormalTok{ estimate), }\AttributeTok{color =} \StringTok{"darkblue"}\NormalTok{, }\AttributeTok{size =} \FloatTok{1.2}\NormalTok{) }\SpecialCharTok{+}
  \FunctionTok{geom\_hline}\NormalTok{(}\AttributeTok{yintercept =}\NormalTok{ Bmsy[}\DecValTok{1}\NormalTok{], }\AttributeTok{linetype =} \StringTok{"dashed"}\NormalTok{, }\AttributeTok{color =} \StringTok{"red"}\NormalTok{) }\SpecialCharTok{+}
  \FunctionTok{labs}\NormalTok{(}\AttributeTok{title =} \StringTok{"A. Биомасса"}\NormalTok{, }\AttributeTok{x =} \StringTok{"Год"}\NormalTok{, }\AttributeTok{y =} \StringTok{"Биомасса (тыс. т)"}\NormalTok{) }\SpecialCharTok{+}
  \FunctionTok{theme\_minimal}\NormalTok{()}
\end{Highlighting}
\end{Shaded}

\begin{verbatim}
Warning: Using `size` aesthetic for lines was deprecated in ggplot2 3.4.0.
i Please use `linewidth` instead.
\end{verbatim}

\begin{Shaded}
\begin{Highlighting}[]
\CommentTok{\# График 2: Промысловая смертность}
\NormalTok{g2 }\OtherTok{\textless{}{-}} \FunctionTok{ggplot}\NormalTok{(df\_F, }\FunctionTok{aes}\NormalTok{(}\AttributeTok{x =}\NormalTok{ time)) }\SpecialCharTok{+}
  \FunctionTok{geom\_ribbon}\NormalTok{(}\FunctionTok{aes}\NormalTok{(}\AttributeTok{ymin =}\NormalTok{ lower, }\AttributeTok{ymax =}\NormalTok{ upper), }\AttributeTok{alpha =} \FloatTok{0.3}\NormalTok{, }\AttributeTok{fill =} \StringTok{"orange"}\NormalTok{) }\SpecialCharTok{+}
  \FunctionTok{geom\_line}\NormalTok{(}\FunctionTok{aes}\NormalTok{(}\AttributeTok{y =}\NormalTok{ estimate), }\AttributeTok{color =} \StringTok{"darkorange"}\NormalTok{, }\AttributeTok{size =} \FloatTok{1.2}\NormalTok{) }\SpecialCharTok{+}
  \FunctionTok{geom\_hline}\NormalTok{(}\AttributeTok{yintercept =}\NormalTok{ Fmsy[}\DecValTok{1}\NormalTok{], }\AttributeTok{linetype =} \StringTok{"dashed"}\NormalTok{, }\AttributeTok{color =} \StringTok{"red"}\NormalTok{) }\SpecialCharTok{+}
  \FunctionTok{labs}\NormalTok{(}\AttributeTok{title =} \StringTok{"B. Промысловая смертность"}\NormalTok{, }\AttributeTok{x =} \StringTok{"Год"}\NormalTok{, }\AttributeTok{y =} \StringTok{"F (год⁻¹)"}\NormalTok{) }\SpecialCharTok{+}
  \FunctionTok{theme\_minimal}\NormalTok{()}

\CommentTok{\# График 3: B/Bmsy}
\NormalTok{g3 }\OtherTok{\textless{}{-}} \FunctionTok{ggplot}\NormalTok{(df\_BBmsy, }\FunctionTok{aes}\NormalTok{(}\AttributeTok{x =}\NormalTok{ time)) }\SpecialCharTok{+}
  \FunctionTok{geom\_ribbon}\NormalTok{(}\FunctionTok{aes}\NormalTok{(}\AttributeTok{ymin =}\NormalTok{ lower, }\AttributeTok{ymax =}\NormalTok{ upper), }\AttributeTok{alpha =} \FloatTok{0.3}\NormalTok{, }\AttributeTok{fill =} \StringTok{"green"}\NormalTok{) }\SpecialCharTok{+}
  \FunctionTok{geom\_line}\NormalTok{(}\FunctionTok{aes}\NormalTok{(}\AttributeTok{y =}\NormalTok{ estimate), }\AttributeTok{color =} \StringTok{"darkgreen"}\NormalTok{, }\AttributeTok{size =} \FloatTok{1.2}\NormalTok{) }\SpecialCharTok{+}
  \FunctionTok{geom\_hline}\NormalTok{(}\AttributeTok{yintercept =} \DecValTok{1}\NormalTok{, }\AttributeTok{linetype =} \StringTok{"dashed"}\NormalTok{, }\AttributeTok{color =} \StringTok{"red"}\NormalTok{) }\SpecialCharTok{+}
  \FunctionTok{geom\_hline}\NormalTok{(}\AttributeTok{yintercept =} \FloatTok{0.5}\NormalTok{, }\AttributeTok{linetype =} \StringTok{"dotted"}\NormalTok{, }\AttributeTok{color =} \StringTok{"orange"}\NormalTok{) }\SpecialCharTok{+}
  \FunctionTok{labs}\NormalTok{(}\AttributeTok{title =} \StringTok{"C. Относительная биомасса"}\NormalTok{, }\AttributeTok{x =} \StringTok{"Год"}\NormalTok{, }\AttributeTok{y =} \StringTok{"B/Bmsy"}\NormalTok{) }\SpecialCharTok{+}
  \FunctionTok{theme\_minimal}\NormalTok{()}

\CommentTok{\# График 4: F/Fmsy}
\NormalTok{g4 }\OtherTok{\textless{}{-}} \FunctionTok{ggplot}\NormalTok{(df\_FFmsy, }\FunctionTok{aes}\NormalTok{(}\AttributeTok{x =}\NormalTok{ time)) }\SpecialCharTok{+}
  \FunctionTok{geom\_ribbon}\NormalTok{(}\FunctionTok{aes}\NormalTok{(}\AttributeTok{ymin =}\NormalTok{ lower, }\AttributeTok{ymax =}\NormalTok{ upper), }\AttributeTok{alpha =} \FloatTok{0.3}\NormalTok{, }\AttributeTok{fill =} \StringTok{"purple"}\NormalTok{) }\SpecialCharTok{+}
  \FunctionTok{geom\_line}\NormalTok{(}\FunctionTok{aes}\NormalTok{(}\AttributeTok{y =}\NormalTok{ estimate), }\AttributeTok{color =} \StringTok{"darkviolet"}\NormalTok{, }\AttributeTok{size =} \FloatTok{1.2}\NormalTok{) }\SpecialCharTok{+}
  \FunctionTok{geom\_hline}\NormalTok{(}\AttributeTok{yintercept =} \DecValTok{1}\NormalTok{, }\AttributeTok{linetype =} \StringTok{"dashed"}\NormalTok{, }\AttributeTok{color =} \StringTok{"red"}\NormalTok{) }\SpecialCharTok{+}
  \FunctionTok{labs}\NormalTok{(}\AttributeTok{title =} \StringTok{"D. Относительная смертность"}\NormalTok{, }\AttributeTok{x =} \StringTok{"Год"}\NormalTok{, }\AttributeTok{y =} \StringTok{"F/Fmsy"}\NormalTok{) }\SpecialCharTok{+}
  \FunctionTok{theme\_minimal}\NormalTok{()}

\CommentTok{\# Объединяем графики}
\NormalTok{grid\_plot }\OtherTok{\textless{}{-}} \FunctionTok{grid.arrange}\NormalTok{(g1, g2, g3, g4, }\AttributeTok{ncol =} \DecValTok{2}\NormalTok{,}
                         \AttributeTok{top =} \StringTok{"Временные ряды ключевых параметров"}\NormalTok{)}
\end{Highlighting}
\end{Shaded}

\begin{verbatim}
Warning in grid.Call(C_textBounds, as.graphicsAnnot(x$label), x$x, x$y, :
неизвестна ширина символа 0xc2 в кодировке CP1251
\end{verbatim}

\begin{verbatim}
Warning in grid.Call(C_textBounds, as.graphicsAnnot(x$label), x$x, x$y, :
неизвестна ширина символа 0xf0 в кодировке CP1251
\end{verbatim}

\begin{verbatim}
Warning in grid.Call(C_textBounds, as.graphicsAnnot(x$label), x$x, x$y, :
неизвестна ширина символа 0xe5 в кодировке CP1251
\end{verbatim}

\begin{verbatim}
Warning in grid.Call(C_textBounds, as.graphicsAnnot(x$label), x$x, x$y, :
неизвестна ширина символа 0xec в кодировке CP1251
\end{verbatim}

\begin{verbatim}
Warning in grid.Call(C_textBounds, as.graphicsAnnot(x$label), x$x, x$y, :
неизвестна ширина символа 0xe5 в кодировке CP1251
\end{verbatim}

\begin{verbatim}
Warning in grid.Call(C_textBounds, as.graphicsAnnot(x$label), x$x, x$y, :
неизвестна ширина символа 0xed в кодировке CP1251
Warning in grid.Call(C_textBounds, as.graphicsAnnot(x$label), x$x, x$y, :
неизвестна ширина символа 0xed в кодировке CP1251
\end{verbatim}

\begin{verbatim}
Warning in grid.Call(C_textBounds, as.graphicsAnnot(x$label), x$x, x$y, :
неизвестна ширина символа 0xfb в кодировке CP1251
\end{verbatim}

\begin{verbatim}
Warning in grid.Call(C_textBounds, as.graphicsAnnot(x$label), x$x, x$y, :
неизвестна ширина символа 0xe5 в кодировке CP1251
\end{verbatim}

\begin{verbatim}
Warning in grid.Call(C_textBounds, as.graphicsAnnot(x$label), x$x, x$y, :
неизвестна ширина символа 0xf0 в кодировке CP1251
\end{verbatim}

\begin{verbatim}
Warning in grid.Call(C_textBounds, as.graphicsAnnot(x$label), x$x, x$y, :
неизвестна ширина символа 0xff в кодировке CP1251
\end{verbatim}

\begin{verbatim}
Warning in grid.Call(C_textBounds, as.graphicsAnnot(x$label), x$x, x$y, :
неизвестна ширина символа 0xe4 в кодировке CP1251
\end{verbatim}

\begin{verbatim}
Warning in grid.Call(C_textBounds, as.graphicsAnnot(x$label), x$x, x$y, :
неизвестна ширина символа 0xfb в кодировке CP1251
\end{verbatim}

\begin{verbatim}
Warning in grid.Call(C_textBounds, as.graphicsAnnot(x$label), x$x, x$y, :
неизвестна ширина символа 0xea в кодировке CP1251
\end{verbatim}

\begin{verbatim}
Warning in grid.Call(C_textBounds, as.graphicsAnnot(x$label), x$x, x$y, :
неизвестна ширина символа 0xeb в кодировке CP1251
\end{verbatim}

\begin{verbatim}
Warning in grid.Call(C_textBounds, as.graphicsAnnot(x$label), x$x, x$y, :
неизвестна ширина символа 0xfe в кодировке CP1251
\end{verbatim}

\begin{verbatim}
Warning in grid.Call(C_textBounds, as.graphicsAnnot(x$label), x$x, x$y, :
неизвестна ширина символа 0xf7 в кодировке CP1251
\end{verbatim}

\begin{verbatim}
Warning in grid.Call(C_textBounds, as.graphicsAnnot(x$label), x$x, x$y, :
неизвестна ширина символа 0xe5 в кодировке CP1251
\end{verbatim}

\begin{verbatim}
Warning in grid.Call(C_textBounds, as.graphicsAnnot(x$label), x$x, x$y, :
неизвестна ширина символа 0xe2 в кодировке CP1251
\end{verbatim}

\begin{verbatim}
Warning in grid.Call(C_textBounds, as.graphicsAnnot(x$label), x$x, x$y, :
неизвестна ширина символа 0xfb в кодировке CP1251
\end{verbatim}

\begin{verbatim}
Warning in grid.Call(C_textBounds, as.graphicsAnnot(x$label), x$x, x$y, :
неизвестна ширина символа 0xf5 в кодировке CP1251
\end{verbatim}

\begin{verbatim}
Warning in grid.Call(C_textBounds, as.graphicsAnnot(x$label), x$x, x$y, :
неизвестна ширина символа 0xef в кодировке CP1251
\end{verbatim}

\begin{verbatim}
Warning in grid.Call(C_textBounds, as.graphicsAnnot(x$label), x$x, x$y, :
неизвестна ширина символа 0xe0 в кодировке CP1251
\end{verbatim}

\begin{verbatim}
Warning in grid.Call(C_textBounds, as.graphicsAnnot(x$label), x$x, x$y, :
неизвестна ширина символа 0xf0 в кодировке CP1251
\end{verbatim}

\begin{verbatim}
Warning in grid.Call(C_textBounds, as.graphicsAnnot(x$label), x$x, x$y, :
неизвестна ширина символа 0xe0 в кодировке CP1251
\end{verbatim}

\begin{verbatim}
Warning in grid.Call(C_textBounds, as.graphicsAnnot(x$label), x$x, x$y, :
неизвестна ширина символа 0xec в кодировке CP1251
\end{verbatim}

\begin{verbatim}
Warning in grid.Call(C_textBounds, as.graphicsAnnot(x$label), x$x, x$y, :
неизвестна ширина символа 0xe5 в кодировке CP1251
\end{verbatim}

\begin{verbatim}
Warning in grid.Call(C_textBounds, as.graphicsAnnot(x$label), x$x, x$y, :
неизвестна ширина символа 0xf2 в кодировке CP1251
\end{verbatim}

\begin{verbatim}
Warning in grid.Call(C_textBounds, as.graphicsAnnot(x$label), x$x, x$y, :
неизвестна ширина символа 0xf0 в кодировке CP1251
\end{verbatim}

\begin{verbatim}
Warning in grid.Call(C_textBounds, as.graphicsAnnot(x$label), x$x, x$y, :
неизвестна ширина символа 0xee в кодировке CP1251
\end{verbatim}

\begin{verbatim}
Warning in grid.Call(C_textBounds, as.graphicsAnnot(x$label), x$x, x$y, :
неизвестна ширина символа 0xe2 в кодировке CP1251
\end{verbatim}

\begin{verbatim}
Warning in grid.Call(C_textBounds, as.graphicsAnnot(x$label), x$x, x$y, :
неизвестна ширина символа 0xc2 в кодировке CP1251
\end{verbatim}

\begin{verbatim}
Warning in grid.Call(C_textBounds, as.graphicsAnnot(x$label), x$x, x$y, :
неизвестна ширина символа 0xf0 в кодировке CP1251
\end{verbatim}

\begin{verbatim}
Warning in grid.Call(C_textBounds, as.graphicsAnnot(x$label), x$x, x$y, :
неизвестна ширина символа 0xe5 в кодировке CP1251
\end{verbatim}

\begin{verbatim}
Warning in grid.Call(C_textBounds, as.graphicsAnnot(x$label), x$x, x$y, :
неизвестна ширина символа 0xec в кодировке CP1251
\end{verbatim}

\begin{verbatim}
Warning in grid.Call(C_textBounds, as.graphicsAnnot(x$label), x$x, x$y, :
неизвестна ширина символа 0xe5 в кодировке CP1251
\end{verbatim}

\begin{verbatim}
Warning in grid.Call(C_textBounds, as.graphicsAnnot(x$label), x$x, x$y, :
неизвестна ширина символа 0xed в кодировке CP1251
Warning in grid.Call(C_textBounds, as.graphicsAnnot(x$label), x$x, x$y, :
неизвестна ширина символа 0xed в кодировке CP1251
\end{verbatim}

\begin{verbatim}
Warning in grid.Call(C_textBounds, as.graphicsAnnot(x$label), x$x, x$y, :
неизвестна ширина символа 0xfb в кодировке CP1251
\end{verbatim}

\begin{verbatim}
Warning in grid.Call(C_textBounds, as.graphicsAnnot(x$label), x$x, x$y, :
неизвестна ширина символа 0xe5 в кодировке CP1251
\end{verbatim}

\begin{verbatim}
Warning in grid.Call(C_textBounds, as.graphicsAnnot(x$label), x$x, x$y, :
неизвестна ширина символа 0xf0 в кодировке CP1251
\end{verbatim}

\begin{verbatim}
Warning in grid.Call(C_textBounds, as.graphicsAnnot(x$label), x$x, x$y, :
неизвестна ширина символа 0xff в кодировке CP1251
\end{verbatim}

\begin{verbatim}
Warning in grid.Call(C_textBounds, as.graphicsAnnot(x$label), x$x, x$y, :
неизвестна ширина символа 0xe4 в кодировке CP1251
\end{verbatim}

\begin{verbatim}
Warning in grid.Call(C_textBounds, as.graphicsAnnot(x$label), x$x, x$y, :
неизвестна ширина символа 0xfb в кодировке CP1251
\end{verbatim}

\begin{verbatim}
Warning in grid.Call(C_textBounds, as.graphicsAnnot(x$label), x$x, x$y, :
неизвестна ширина символа 0xea в кодировке CP1251
\end{verbatim}

\begin{verbatim}
Warning in grid.Call(C_textBounds, as.graphicsAnnot(x$label), x$x, x$y, :
неизвестна ширина символа 0xeb в кодировке CP1251
\end{verbatim}

\begin{verbatim}
Warning in grid.Call(C_textBounds, as.graphicsAnnot(x$label), x$x, x$y, :
неизвестна ширина символа 0xfe в кодировке CP1251
\end{verbatim}

\begin{verbatim}
Warning in grid.Call(C_textBounds, as.graphicsAnnot(x$label), x$x, x$y, :
неизвестна ширина символа 0xf7 в кодировке CP1251
\end{verbatim}

\begin{verbatim}
Warning in grid.Call(C_textBounds, as.graphicsAnnot(x$label), x$x, x$y, :
неизвестна ширина символа 0xe5 в кодировке CP1251
\end{verbatim}

\begin{verbatim}
Warning in grid.Call(C_textBounds, as.graphicsAnnot(x$label), x$x, x$y, :
неизвестна ширина символа 0xe2 в кодировке CP1251
\end{verbatim}

\begin{verbatim}
Warning in grid.Call(C_textBounds, as.graphicsAnnot(x$label), x$x, x$y, :
неизвестна ширина символа 0xfb в кодировке CP1251
\end{verbatim}

\begin{verbatim}
Warning in grid.Call(C_textBounds, as.graphicsAnnot(x$label), x$x, x$y, :
неизвестна ширина символа 0xf5 в кодировке CP1251
\end{verbatim}

\begin{verbatim}
Warning in grid.Call(C_textBounds, as.graphicsAnnot(x$label), x$x, x$y, :
неизвестна ширина символа 0xef в кодировке CP1251
\end{verbatim}

\begin{verbatim}
Warning in grid.Call(C_textBounds, as.graphicsAnnot(x$label), x$x, x$y, :
неизвестна ширина символа 0xe0 в кодировке CP1251
\end{verbatim}

\begin{verbatim}
Warning in grid.Call(C_textBounds, as.graphicsAnnot(x$label), x$x, x$y, :
неизвестна ширина символа 0xf0 в кодировке CP1251
\end{verbatim}

\begin{verbatim}
Warning in grid.Call(C_textBounds, as.graphicsAnnot(x$label), x$x, x$y, :
неизвестна ширина символа 0xe0 в кодировке CP1251
\end{verbatim}

\begin{verbatim}
Warning in grid.Call(C_textBounds, as.graphicsAnnot(x$label), x$x, x$y, :
неизвестна ширина символа 0xec в кодировке CP1251
\end{verbatim}

\begin{verbatim}
Warning in grid.Call(C_textBounds, as.graphicsAnnot(x$label), x$x, x$y, :
неизвестна ширина символа 0xe5 в кодировке CP1251
\end{verbatim}

\begin{verbatim}
Warning in grid.Call(C_textBounds, as.graphicsAnnot(x$label), x$x, x$y, :
неизвестна ширина символа 0xf2 в кодировке CP1251
\end{verbatim}

\begin{verbatim}
Warning in grid.Call(C_textBounds, as.graphicsAnnot(x$label), x$x, x$y, :
неизвестна ширина символа 0xf0 в кодировке CP1251
\end{verbatim}

\begin{verbatim}
Warning in grid.Call(C_textBounds, as.graphicsAnnot(x$label), x$x, x$y, :
неизвестна ширина символа 0xee в кодировке CP1251
\end{verbatim}

\begin{verbatim}
Warning in grid.Call(C_textBounds, as.graphicsAnnot(x$label), x$x, x$y, :
неизвестна ширина символа 0xe2 в кодировке CP1251
\end{verbatim}

\begin{verbatim}
Warning in grid.Call(C_textBounds, as.graphicsAnnot(x$label), x$x, x$y, :
неизвестна ширина символа 0xc1 в кодировке CP1251
\end{verbatim}

\begin{verbatim}
Warning in grid.Call(C_textBounds, as.graphicsAnnot(x$label), x$x, x$y, :
неизвестна ширина символа 0xe8 в кодировке CP1251
\end{verbatim}

\begin{verbatim}
Warning in grid.Call(C_textBounds, as.graphicsAnnot(x$label), x$x, x$y, :
неизвестна ширина символа 0xee в кодировке CP1251
\end{verbatim}

\begin{verbatim}
Warning in grid.Call(C_textBounds, as.graphicsAnnot(x$label), x$x, x$y, :
неизвестна ширина символа 0xec в кодировке CP1251
\end{verbatim}

\begin{verbatim}
Warning in grid.Call(C_textBounds, as.graphicsAnnot(x$label), x$x, x$y, :
неизвестна ширина символа 0xe0 в кодировке CP1251
\end{verbatim}

\begin{verbatim}
Warning in grid.Call(C_textBounds, as.graphicsAnnot(x$label), x$x, x$y, :
неизвестна ширина символа 0xf1 в кодировке CP1251
Warning in grid.Call(C_textBounds, as.graphicsAnnot(x$label), x$x, x$y, :
неизвестна ширина символа 0xf1 в кодировке CP1251
\end{verbatim}

\begin{verbatim}
Warning in grid.Call(C_textBounds, as.graphicsAnnot(x$label), x$x, x$y, :
неизвестна ширина символа 0xe0 в кодировке CP1251
\end{verbatim}

\begin{verbatim}
Warning in grid.Call(C_textBounds, as.graphicsAnnot(x$label), x$x, x$y, :
неизвестна ширина символа 0xf2 в кодировке CP1251
\end{verbatim}

\begin{verbatim}
Warning in grid.Call(C_textBounds, as.graphicsAnnot(x$label), x$x, x$y, :
неизвестна ширина символа 0xfb в кодировке CP1251
\end{verbatim}

\begin{verbatim}
Warning in grid.Call(C_textBounds, as.graphicsAnnot(x$label), x$x, x$y, :
неизвестна ширина символа 0xf1 в кодировке CP1251
\end{verbatim}

\begin{verbatim}
Warning in grid.Call(C_textBounds, as.graphicsAnnot(x$label), x$x, x$y, :
неизвестна ширина символа 0xf2 в кодировке CP1251
\end{verbatim}

\begin{verbatim}
Warning in grid.Call(C_textBounds, as.graphicsAnnot(x$label), x$x, x$y, :
неизвестна ширина символа 0xc1 в кодировке CP1251
\end{verbatim}

\begin{verbatim}
Warning in grid.Call(C_textBounds, as.graphicsAnnot(x$label), x$x, x$y, :
неизвестна ширина символа 0xe8 в кодировке CP1251
\end{verbatim}

\begin{verbatim}
Warning in grid.Call(C_textBounds, as.graphicsAnnot(x$label), x$x, x$y, :
неизвестна ширина символа 0xee в кодировке CP1251
\end{verbatim}

\begin{verbatim}
Warning in grid.Call(C_textBounds, as.graphicsAnnot(x$label), x$x, x$y, :
неизвестна ширина символа 0xec в кодировке CP1251
\end{verbatim}

\begin{verbatim}
Warning in grid.Call(C_textBounds, as.graphicsAnnot(x$label), x$x, x$y, :
неизвестна ширина символа 0xe0 в кодировке CP1251
\end{verbatim}

\begin{verbatim}
Warning in grid.Call(C_textBounds, as.graphicsAnnot(x$label), x$x, x$y, :
неизвестна ширина символа 0xf1 в кодировке CP1251
Warning in grid.Call(C_textBounds, as.graphicsAnnot(x$label), x$x, x$y, :
неизвестна ширина символа 0xf1 в кодировке CP1251
\end{verbatim}

\begin{verbatim}
Warning in grid.Call(C_textBounds, as.graphicsAnnot(x$label), x$x, x$y, :
неизвестна ширина символа 0xe0 в кодировке CP1251
\end{verbatim}

\begin{verbatim}
Warning in grid.Call(C_textBounds, as.graphicsAnnot(x$label), x$x, x$y, :
неизвестна ширина символа 0xc3 в кодировке CP1251
\end{verbatim}

\begin{verbatim}
Warning in grid.Call(C_textBounds, as.graphicsAnnot(x$label), x$x, x$y, :
неизвестна ширина символа 0xee в кодировке CP1251
\end{verbatim}

\begin{verbatim}
Warning in grid.Call(C_textBounds, as.graphicsAnnot(x$label), x$x, x$y, :
неизвестна ширина символа 0xe4 в кодировке CP1251
\end{verbatim}

\begin{verbatim}
Warning in grid.Call.graphics(C_text, as.graphicsAnnot(x$label), x$x, x$y, :
неизвестна ширина символа 0xc3 в кодировке CP1251
\end{verbatim}

\begin{verbatim}
Warning in grid.Call.graphics(C_text, as.graphicsAnnot(x$label), x$x, x$y, :
неизвестна ширина символа 0xee в кодировке CP1251
\end{verbatim}

\begin{verbatim}
Warning in grid.Call.graphics(C_text, as.graphicsAnnot(x$label), x$x, x$y, :
неизвестна ширина символа 0xe4 в кодировке CP1251
\end{verbatim}

\begin{verbatim}
Warning in grid.Call.graphics(C_text, as.graphicsAnnot(x$label), x$x, x$y, :
неизвестна ширина символа 0xc1 в кодировке CP1251
\end{verbatim}

\begin{verbatim}
Warning in grid.Call.graphics(C_text, as.graphicsAnnot(x$label), x$x, x$y, :
неизвестна ширина символа 0xe8 в кодировке CP1251
\end{verbatim}

\begin{verbatim}
Warning in grid.Call.graphics(C_text, as.graphicsAnnot(x$label), x$x, x$y, :
неизвестна ширина символа 0xee в кодировке CP1251
\end{verbatim}

\begin{verbatim}
Warning in grid.Call.graphics(C_text, as.graphicsAnnot(x$label), x$x, x$y, :
неизвестна ширина символа 0xec в кодировке CP1251
\end{verbatim}

\begin{verbatim}
Warning in grid.Call.graphics(C_text, as.graphicsAnnot(x$label), x$x, x$y, :
неизвестна ширина символа 0xe0 в кодировке CP1251
\end{verbatim}

\begin{verbatim}
Warning in grid.Call.graphics(C_text, as.graphicsAnnot(x$label), x$x, x$y, :
неизвестна ширина символа 0xf1 в кодировке CP1251
Warning in grid.Call.graphics(C_text, as.graphicsAnnot(x$label), x$x, x$y, :
неизвестна ширина символа 0xf1 в кодировке CP1251
\end{verbatim}

\begin{verbatim}
Warning in grid.Call.graphics(C_text, as.graphicsAnnot(x$label), x$x, x$y, :
неизвестна ширина символа 0xe0 в кодировке CP1251
\end{verbatim}

\begin{verbatim}
Warning in grid.Call.graphics(C_text, as.graphicsAnnot(x$label), x$x, x$y, :
неизвестна ширина символа 0xf2 в кодировке CP1251
\end{verbatim}

\begin{verbatim}
Warning in grid.Call.graphics(C_text, as.graphicsAnnot(x$label), x$x, x$y, :
неизвестна ширина символа 0xfb в кодировке CP1251
\end{verbatim}

\begin{verbatim}
Warning in grid.Call.graphics(C_text, as.graphicsAnnot(x$label), x$x, x$y, :
неизвестна ширина символа 0xf1 в кодировке CP1251
\end{verbatim}

\begin{verbatim}
Warning in grid.Call.graphics(C_text, as.graphicsAnnot(x$label), x$x, x$y, :
неизвестна ширина символа 0xf2 в кодировке CP1251
\end{verbatim}

\begin{verbatim}
Warning in grid.Call.graphics(C_text, as.graphicsAnnot(x$label), x$x, x$y, :
неизвестна ширина символа 0xc1 в кодировке CP1251
\end{verbatim}

\begin{verbatim}
Warning in grid.Call.graphics(C_text, as.graphicsAnnot(x$label), x$x, x$y, :
неизвестна ширина символа 0xe8 в кодировке CP1251
\end{verbatim}

\begin{verbatim}
Warning in grid.Call.graphics(C_text, as.graphicsAnnot(x$label), x$x, x$y, :
неизвестна ширина символа 0xee в кодировке CP1251
\end{verbatim}

\begin{verbatim}
Warning in grid.Call.graphics(C_text, as.graphicsAnnot(x$label), x$x, x$y, :
неизвестна ширина символа 0xec в кодировке CP1251
\end{verbatim}

\begin{verbatim}
Warning in grid.Call.graphics(C_text, as.graphicsAnnot(x$label), x$x, x$y, :
неизвестна ширина символа 0xe0 в кодировке CP1251
\end{verbatim}

\begin{verbatim}
Warning in grid.Call.graphics(C_text, as.graphicsAnnot(x$label), x$x, x$y, :
неизвестна ширина символа 0xf1 в кодировке CP1251
Warning in grid.Call.graphics(C_text, as.graphicsAnnot(x$label), x$x, x$y, :
неизвестна ширина символа 0xf1 в кодировке CP1251
\end{verbatim}

\begin{verbatim}
Warning in grid.Call.graphics(C_text, as.graphicsAnnot(x$label), x$x, x$y, :
неизвестна ширина символа 0xe0 в кодировке CP1251
\end{verbatim}

\begin{verbatim}
Warning in grid.Call(C_textBounds, as.graphicsAnnot(x$label), x$x, x$y, :
неизвестна ширина символа 0xe3 в кодировке CP1251
\end{verbatim}

\begin{verbatim}
Warning in grid.Call(C_textBounds, as.graphicsAnnot(x$label), x$x, x$y, :
неизвестна ширина символа 0xee в кодировке CP1251
\end{verbatim}

\begin{verbatim}
Warning in grid.Call(C_textBounds, as.graphicsAnnot(x$label), x$x, x$y, :
неизвестна ширина символа 0xe4 в кодировке CP1251
\end{verbatim}

\begin{verbatim}
Warning in grid.Call(C_textBounds, as.graphicsAnnot(x$label), x$x, x$y, :
неизвестна ширина символа 0xcf в кодировке CP1251
\end{verbatim}

\begin{verbatim}
Warning in grid.Call(C_textBounds, as.graphicsAnnot(x$label), x$x, x$y, :
неизвестна ширина символа 0xf0 в кодировке CP1251
\end{verbatim}

\begin{verbatim}
Warning in grid.Call(C_textBounds, as.graphicsAnnot(x$label), x$x, x$y, :
неизвестна ширина символа 0xee в кодировке CP1251
\end{verbatim}

\begin{verbatim}
Warning in grid.Call(C_textBounds, as.graphicsAnnot(x$label), x$x, x$y, :
неизвестна ширина символа 0xec в кодировке CP1251
\end{verbatim}

\begin{verbatim}
Warning in grid.Call(C_textBounds, as.graphicsAnnot(x$label), x$x, x$y, :
неизвестна ширина символа 0xfb в кодировке CP1251
\end{verbatim}

\begin{verbatim}
Warning in grid.Call(C_textBounds, as.graphicsAnnot(x$label), x$x, x$y, :
неизвестна ширина символа 0xf1 в кодировке CP1251
\end{verbatim}

\begin{verbatim}
Warning in grid.Call(C_textBounds, as.graphicsAnnot(x$label), x$x, x$y, :
неизвестна ширина символа 0xeb в кодировке CP1251
\end{verbatim}

\begin{verbatim}
Warning in grid.Call(C_textBounds, as.graphicsAnnot(x$label), x$x, x$y, :
неизвестна ширина символа 0xee в кодировке CP1251
\end{verbatim}

\begin{verbatim}
Warning in grid.Call(C_textBounds, as.graphicsAnnot(x$label), x$x, x$y, :
неизвестна ширина символа 0xe2 в кодировке CP1251
\end{verbatim}

\begin{verbatim}
Warning in grid.Call(C_textBounds, as.graphicsAnnot(x$label), x$x, x$y, :
неизвестна ширина символа 0xe0 в кодировке CP1251
\end{verbatim}

\begin{verbatim}
Warning in grid.Call(C_textBounds, as.graphicsAnnot(x$label), x$x, x$y, :
неизвестна ширина символа 0xff в кодировке CP1251
\end{verbatim}

\begin{verbatim}
Warning in grid.Call(C_textBounds, as.graphicsAnnot(x$label), x$x, x$y, :
неизвестна ширина символа 0xf1 в кодировке CP1251
\end{verbatim}

\begin{verbatim}
Warning in grid.Call(C_textBounds, as.graphicsAnnot(x$label), x$x, x$y, :
неизвестна ширина символа 0xec в кодировке CP1251
\end{verbatim}

\begin{verbatim}
Warning in grid.Call(C_textBounds, as.graphicsAnnot(x$label), x$x, x$y, :
неизвестна ширина символа 0xe5 в кодировке CP1251
\end{verbatim}

\begin{verbatim}
Warning in grid.Call(C_textBounds, as.graphicsAnnot(x$label), x$x, x$y, :
неизвестна ширина символа 0xf0 в кодировке CP1251
\end{verbatim}

\begin{verbatim}
Warning in grid.Call(C_textBounds, as.graphicsAnnot(x$label), x$x, x$y, :
неизвестна ширина символа 0xf2 в кодировке CP1251
\end{verbatim}

\begin{verbatim}
Warning in grid.Call(C_textBounds, as.graphicsAnnot(x$label), x$x, x$y, :
неизвестна ширина символа 0xed в кодировке CP1251
\end{verbatim}

\begin{verbatim}
Warning in grid.Call(C_textBounds, as.graphicsAnnot(x$label), x$x, x$y, :
неизвестна ширина символа 0xee в кодировке CP1251
\end{verbatim}

\begin{verbatim}
Warning in grid.Call(C_textBounds, as.graphicsAnnot(x$label), x$x, x$y, :
неизвестна ширина символа 0xf1 в кодировке CP1251
\end{verbatim}

\begin{verbatim}
Warning in grid.Call(C_textBounds, as.graphicsAnnot(x$label), x$x, x$y, :
неизвестна ширина символа 0xf2 в кодировке CP1251
\end{verbatim}

\begin{verbatim}
Warning in grid.Call(C_textBounds, as.graphicsAnnot(x$label), x$x, x$y, :
неизвестна ширина символа 0xfc в кодировке CP1251
\end{verbatim}

\begin{verbatim}
Warning in grid.Call(C_textBounds, as.graphicsAnnot(x$label), x$x, x$y, :
неизвестна ширина символа 0xc3 в кодировке CP1251
\end{verbatim}

\begin{verbatim}
Warning in grid.Call(C_textBounds, as.graphicsAnnot(x$label), x$x, x$y, :
неизвестна ширина символа 0xee в кодировке CP1251
\end{verbatim}

\begin{verbatim}
Warning in grid.Call(C_textBounds, as.graphicsAnnot(x$label), x$x, x$y, :
неизвестна ширина символа 0xe4 в кодировке CP1251
\end{verbatim}

\begin{verbatim}
Warning in grid.Call.graphics(C_text, as.graphicsAnnot(x$label), x$x, x$y, :
неизвестна ширина символа 0xc3 в кодировке CP1251
\end{verbatim}

\begin{verbatim}
Warning in grid.Call.graphics(C_text, as.graphicsAnnot(x$label), x$x, x$y, :
неизвестна ширина символа 0xee в кодировке CP1251
\end{verbatim}

\begin{verbatim}
Warning in grid.Call.graphics(C_text, as.graphicsAnnot(x$label), x$x, x$y, :
неизвестна ширина символа 0xe4 в кодировке CP1251
\end{verbatim}

\begin{verbatim}
Warning in grid.Call.graphics(C_text, as.graphicsAnnot(x$label), x$x, x$y, :
неизвестна ширина символа 0xe3 в кодировке CP1251
\end{verbatim}

\begin{verbatim}
Warning in grid.Call.graphics(C_text, as.graphicsAnnot(x$label), x$x, x$y, :
неизвестна ширина символа 0xee в кодировке CP1251
\end{verbatim}

\begin{verbatim}
Warning in grid.Call.graphics(C_text, as.graphicsAnnot(x$label), x$x, x$y, :
неизвестна ширина символа 0xe4 в кодировке CP1251
\end{verbatim}

\begin{verbatim}
Warning in grid.Call.graphics(C_text, as.graphicsAnnot(x$label), x$x, x$y, :
неизвестна ширина символа 0xcf в кодировке CP1251
\end{verbatim}

\begin{verbatim}
Warning in grid.Call.graphics(C_text, as.graphicsAnnot(x$label), x$x, x$y, :
неизвестна ширина символа 0xf0 в кодировке CP1251
\end{verbatim}

\begin{verbatim}
Warning in grid.Call.graphics(C_text, as.graphicsAnnot(x$label), x$x, x$y, :
неизвестна ширина символа 0xee в кодировке CP1251
\end{verbatim}

\begin{verbatim}
Warning in grid.Call.graphics(C_text, as.graphicsAnnot(x$label), x$x, x$y, :
неизвестна ширина символа 0xec в кодировке CP1251
\end{verbatim}

\begin{verbatim}
Warning in grid.Call.graphics(C_text, as.graphicsAnnot(x$label), x$x, x$y, :
неизвестна ширина символа 0xfb в кодировке CP1251
\end{verbatim}

\begin{verbatim}
Warning in grid.Call.graphics(C_text, as.graphicsAnnot(x$label), x$x, x$y, :
неизвестна ширина символа 0xf1 в кодировке CP1251
\end{verbatim}

\begin{verbatim}
Warning in grid.Call.graphics(C_text, as.graphicsAnnot(x$label), x$x, x$y, :
неизвестна ширина символа 0xeb в кодировке CP1251
\end{verbatim}

\begin{verbatim}
Warning in grid.Call.graphics(C_text, as.graphicsAnnot(x$label), x$x, x$y, :
неизвестна ширина символа 0xee в кодировке CP1251
\end{verbatim}

\begin{verbatim}
Warning in grid.Call.graphics(C_text, as.graphicsAnnot(x$label), x$x, x$y, :
неизвестна ширина символа 0xe2 в кодировке CP1251
\end{verbatim}

\begin{verbatim}
Warning in grid.Call.graphics(C_text, as.graphicsAnnot(x$label), x$x, x$y, :
неизвестна ширина символа 0xe0 в кодировке CP1251
\end{verbatim}

\begin{verbatim}
Warning in grid.Call.graphics(C_text, as.graphicsAnnot(x$label), x$x, x$y, :
неизвестна ширина символа 0xff в кодировке CP1251
\end{verbatim}

\begin{verbatim}
Warning in grid.Call.graphics(C_text, as.graphicsAnnot(x$label), x$x, x$y, :
неизвестна ширина символа 0xf1 в кодировке CP1251
\end{verbatim}

\begin{verbatim}
Warning in grid.Call.graphics(C_text, as.graphicsAnnot(x$label), x$x, x$y, :
неизвестна ширина символа 0xec в кодировке CP1251
\end{verbatim}

\begin{verbatim}
Warning in grid.Call.graphics(C_text, as.graphicsAnnot(x$label), x$x, x$y, :
неизвестна ширина символа 0xe5 в кодировке CP1251
\end{verbatim}

\begin{verbatim}
Warning in grid.Call.graphics(C_text, as.graphicsAnnot(x$label), x$x, x$y, :
неизвестна ширина символа 0xf0 в кодировке CP1251
\end{verbatim}

\begin{verbatim}
Warning in grid.Call.graphics(C_text, as.graphicsAnnot(x$label), x$x, x$y, :
неизвестна ширина символа 0xf2 в кодировке CP1251
\end{verbatim}

\begin{verbatim}
Warning in grid.Call.graphics(C_text, as.graphicsAnnot(x$label), x$x, x$y, :
неизвестна ширина символа 0xed в кодировке CP1251
\end{verbatim}

\begin{verbatim}
Warning in grid.Call.graphics(C_text, as.graphicsAnnot(x$label), x$x, x$y, :
неизвестна ширина символа 0xee в кодировке CP1251
\end{verbatim}

\begin{verbatim}
Warning in grid.Call.graphics(C_text, as.graphicsAnnot(x$label), x$x, x$y, :
неизвестна ширина символа 0xf1 в кодировке CP1251
\end{verbatim}

\begin{verbatim}
Warning in grid.Call.graphics(C_text, as.graphicsAnnot(x$label), x$x, x$y, :
неизвестна ширина символа 0xf2 в кодировке CP1251
\end{verbatim}

\begin{verbatim}
Warning in grid.Call.graphics(C_text, as.graphicsAnnot(x$label), x$x, x$y, :
неизвестна ширина символа 0xfc в кодировке CP1251
\end{verbatim}

\begin{verbatim}
Warning in grid.Call(C_textBounds, as.graphicsAnnot(x$label), x$x, x$y, :
неизвестна ширина символа 0xce в кодировке CP1251
\end{verbatim}

\begin{verbatim}
Warning in grid.Call(C_textBounds, as.graphicsAnnot(x$label), x$x, x$y, :
неизвестна ширина символа 0xf2 в кодировке CP1251
\end{verbatim}

\begin{verbatim}
Warning in grid.Call(C_textBounds, as.graphicsAnnot(x$label), x$x, x$y, :
неизвестна ширина символа 0xed в кодировке CP1251
\end{verbatim}

\begin{verbatim}
Warning in grid.Call(C_textBounds, as.graphicsAnnot(x$label), x$x, x$y, :
неизвестна ширина символа 0xee в кодировке CP1251
\end{verbatim}

\begin{verbatim}
Warning in grid.Call(C_textBounds, as.graphicsAnnot(x$label), x$x, x$y, :
неизвестна ширина символа 0xf1 в кодировке CP1251
\end{verbatim}

\begin{verbatim}
Warning in grid.Call(C_textBounds, as.graphicsAnnot(x$label), x$x, x$y, :
неизвестна ширина символа 0xe8 в кодировке CP1251
\end{verbatim}

\begin{verbatim}
Warning in grid.Call(C_textBounds, as.graphicsAnnot(x$label), x$x, x$y, :
неизвестна ширина символа 0xf2 в кодировке CP1251
\end{verbatim}

\begin{verbatim}
Warning in grid.Call(C_textBounds, as.graphicsAnnot(x$label), x$x, x$y, :
неизвестна ширина символа 0xe5 в кодировке CP1251
\end{verbatim}

\begin{verbatim}
Warning in grid.Call(C_textBounds, as.graphicsAnnot(x$label), x$x, x$y, :
неизвестна ширина символа 0xeb в кодировке CP1251
\end{verbatim}

\begin{verbatim}
Warning in grid.Call(C_textBounds, as.graphicsAnnot(x$label), x$x, x$y, :
неизвестна ширина символа 0xfc в кодировке CP1251
\end{verbatim}

\begin{verbatim}
Warning in grid.Call(C_textBounds, as.graphicsAnnot(x$label), x$x, x$y, :
неизвестна ширина символа 0xed в кодировке CP1251
\end{verbatim}

\begin{verbatim}
Warning in grid.Call(C_textBounds, as.graphicsAnnot(x$label), x$x, x$y, :
неизвестна ширина символа 0xe0 в кодировке CP1251
\end{verbatim}

\begin{verbatim}
Warning in grid.Call(C_textBounds, as.graphicsAnnot(x$label), x$x, x$y, :
неизвестна ширина символа 0xff в кодировке CP1251
\end{verbatim}

\begin{verbatim}
Warning in grid.Call(C_textBounds, as.graphicsAnnot(x$label), x$x, x$y, :
неизвестна ширина символа 0xe1 в кодировке CP1251
\end{verbatim}

\begin{verbatim}
Warning in grid.Call(C_textBounds, as.graphicsAnnot(x$label), x$x, x$y, :
неизвестна ширина символа 0xe8 в кодировке CP1251
\end{verbatim}

\begin{verbatim}
Warning in grid.Call(C_textBounds, as.graphicsAnnot(x$label), x$x, x$y, :
неизвестна ширина символа 0xee в кодировке CP1251
\end{verbatim}

\begin{verbatim}
Warning in grid.Call(C_textBounds, as.graphicsAnnot(x$label), x$x, x$y, :
неизвестна ширина символа 0xec в кодировке CP1251
\end{verbatim}

\begin{verbatim}
Warning in grid.Call(C_textBounds, as.graphicsAnnot(x$label), x$x, x$y, :
неизвестна ширина символа 0xe0 в кодировке CP1251
\end{verbatim}

\begin{verbatim}
Warning in grid.Call(C_textBounds, as.graphicsAnnot(x$label), x$x, x$y, :
неизвестна ширина символа 0xf1 в кодировке CP1251
Warning in grid.Call(C_textBounds, as.graphicsAnnot(x$label), x$x, x$y, :
неизвестна ширина символа 0xf1 в кодировке CP1251
\end{verbatim}

\begin{verbatim}
Warning in grid.Call(C_textBounds, as.graphicsAnnot(x$label), x$x, x$y, :
неизвестна ширина символа 0xe0 в кодировке CP1251
\end{verbatim}

\begin{verbatim}
Warning in grid.Call(C_textBounds, as.graphicsAnnot(x$label), x$x, x$y, :
неизвестна ширина символа 0xc3 в кодировке CP1251
\end{verbatim}

\begin{verbatim}
Warning in grid.Call(C_textBounds, as.graphicsAnnot(x$label), x$x, x$y, :
неизвестна ширина символа 0xee в кодировке CP1251
\end{verbatim}

\begin{verbatim}
Warning in grid.Call(C_textBounds, as.graphicsAnnot(x$label), x$x, x$y, :
неизвестна ширина символа 0xe4 в кодировке CP1251
\end{verbatim}

\begin{verbatim}
Warning in grid.Call.graphics(C_text, as.graphicsAnnot(x$label), x$x, x$y, :
неизвестна ширина символа 0xc3 в кодировке CP1251
\end{verbatim}

\begin{verbatim}
Warning in grid.Call.graphics(C_text, as.graphicsAnnot(x$label), x$x, x$y, :
неизвестна ширина символа 0xee в кодировке CP1251
\end{verbatim}

\begin{verbatim}
Warning in grid.Call.graphics(C_text, as.graphicsAnnot(x$label), x$x, x$y, :
неизвестна ширина символа 0xe4 в кодировке CP1251
\end{verbatim}

\begin{verbatim}
Warning in grid.Call.graphics(C_text, as.graphicsAnnot(x$label), x$x, x$y, :
неизвестна ширина символа 0xce в кодировке CP1251
\end{verbatim}

\begin{verbatim}
Warning in grid.Call.graphics(C_text, as.graphicsAnnot(x$label), x$x, x$y, :
неизвестна ширина символа 0xf2 в кодировке CP1251
\end{verbatim}

\begin{verbatim}
Warning in grid.Call.graphics(C_text, as.graphicsAnnot(x$label), x$x, x$y, :
неизвестна ширина символа 0xed в кодировке CP1251
\end{verbatim}

\begin{verbatim}
Warning in grid.Call.graphics(C_text, as.graphicsAnnot(x$label), x$x, x$y, :
неизвестна ширина символа 0xee в кодировке CP1251
\end{verbatim}

\begin{verbatim}
Warning in grid.Call.graphics(C_text, as.graphicsAnnot(x$label), x$x, x$y, :
неизвестна ширина символа 0xf1 в кодировке CP1251
\end{verbatim}

\begin{verbatim}
Warning in grid.Call.graphics(C_text, as.graphicsAnnot(x$label), x$x, x$y, :
неизвестна ширина символа 0xe8 в кодировке CP1251
\end{verbatim}

\begin{verbatim}
Warning in grid.Call.graphics(C_text, as.graphicsAnnot(x$label), x$x, x$y, :
неизвестна ширина символа 0xf2 в кодировке CP1251
\end{verbatim}

\begin{verbatim}
Warning in grid.Call.graphics(C_text, as.graphicsAnnot(x$label), x$x, x$y, :
неизвестна ширина символа 0xe5 в кодировке CP1251
\end{verbatim}

\begin{verbatim}
Warning in grid.Call.graphics(C_text, as.graphicsAnnot(x$label), x$x, x$y, :
неизвестна ширина символа 0xeb в кодировке CP1251
\end{verbatim}

\begin{verbatim}
Warning in grid.Call.graphics(C_text, as.graphicsAnnot(x$label), x$x, x$y, :
неизвестна ширина символа 0xfc в кодировке CP1251
\end{verbatim}

\begin{verbatim}
Warning in grid.Call.graphics(C_text, as.graphicsAnnot(x$label), x$x, x$y, :
неизвестна ширина символа 0xed в кодировке CP1251
\end{verbatim}

\begin{verbatim}
Warning in grid.Call.graphics(C_text, as.graphicsAnnot(x$label), x$x, x$y, :
неизвестна ширина символа 0xe0 в кодировке CP1251
\end{verbatim}

\begin{verbatim}
Warning in grid.Call.graphics(C_text, as.graphicsAnnot(x$label), x$x, x$y, :
неизвестна ширина символа 0xff в кодировке CP1251
\end{verbatim}

\begin{verbatim}
Warning in grid.Call.graphics(C_text, as.graphicsAnnot(x$label), x$x, x$y, :
неизвестна ширина символа 0xe1 в кодировке CP1251
\end{verbatim}

\begin{verbatim}
Warning in grid.Call.graphics(C_text, as.graphicsAnnot(x$label), x$x, x$y, :
неизвестна ширина символа 0xe8 в кодировке CP1251
\end{verbatim}

\begin{verbatim}
Warning in grid.Call.graphics(C_text, as.graphicsAnnot(x$label), x$x, x$y, :
неизвестна ширина символа 0xee в кодировке CP1251
\end{verbatim}

\begin{verbatim}
Warning in grid.Call.graphics(C_text, as.graphicsAnnot(x$label), x$x, x$y, :
неизвестна ширина символа 0xec в кодировке CP1251
\end{verbatim}

\begin{verbatim}
Warning in grid.Call.graphics(C_text, as.graphicsAnnot(x$label), x$x, x$y, :
неизвестна ширина символа 0xe0 в кодировке CP1251
\end{verbatim}

\begin{verbatim}
Warning in grid.Call.graphics(C_text, as.graphicsAnnot(x$label), x$x, x$y, :
неизвестна ширина символа 0xf1 в кодировке CP1251
Warning in grid.Call.graphics(C_text, as.graphicsAnnot(x$label), x$x, x$y, :
неизвестна ширина символа 0xf1 в кодировке CP1251
\end{verbatim}

\begin{verbatim}
Warning in grid.Call.graphics(C_text, as.graphicsAnnot(x$label), x$x, x$y, :
неизвестна ширина символа 0xe0 в кодировке CP1251
\end{verbatim}

\begin{verbatim}
Warning in grid.Call(C_textBounds, as.graphicsAnnot(x$label), x$x, x$y, :
неизвестна ширина символа 0xce в кодировке CP1251
\end{verbatim}

\begin{verbatim}
Warning in grid.Call(C_textBounds, as.graphicsAnnot(x$label), x$x, x$y, :
неизвестна ширина символа 0xf2 в кодировке CP1251
\end{verbatim}

\begin{verbatim}
Warning in grid.Call(C_textBounds, as.graphicsAnnot(x$label), x$x, x$y, :
неизвестна ширина символа 0xed в кодировке CP1251
\end{verbatim}

\begin{verbatim}
Warning in grid.Call(C_textBounds, as.graphicsAnnot(x$label), x$x, x$y, :
неизвестна ширина символа 0xee в кодировке CP1251
\end{verbatim}

\begin{verbatim}
Warning in grid.Call(C_textBounds, as.graphicsAnnot(x$label), x$x, x$y, :
неизвестна ширина символа 0xf1 в кодировке CP1251
\end{verbatim}

\begin{verbatim}
Warning in grid.Call(C_textBounds, as.graphicsAnnot(x$label), x$x, x$y, :
неизвестна ширина символа 0xe8 в кодировке CP1251
\end{verbatim}

\begin{verbatim}
Warning in grid.Call(C_textBounds, as.graphicsAnnot(x$label), x$x, x$y, :
неизвестна ширина символа 0xf2 в кодировке CP1251
\end{verbatim}

\begin{verbatim}
Warning in grid.Call(C_textBounds, as.graphicsAnnot(x$label), x$x, x$y, :
неизвестна ширина символа 0xe5 в кодировке CP1251
\end{verbatim}

\begin{verbatim}
Warning in grid.Call(C_textBounds, as.graphicsAnnot(x$label), x$x, x$y, :
неизвестна ширина символа 0xeb в кодировке CP1251
\end{verbatim}

\begin{verbatim}
Warning in grid.Call(C_textBounds, as.graphicsAnnot(x$label), x$x, x$y, :
неизвестна ширина символа 0xfc в кодировке CP1251
\end{verbatim}

\begin{verbatim}
Warning in grid.Call(C_textBounds, as.graphicsAnnot(x$label), x$x, x$y, :
неизвестна ширина символа 0xed в кодировке CP1251
\end{verbatim}

\begin{verbatim}
Warning in grid.Call(C_textBounds, as.graphicsAnnot(x$label), x$x, x$y, :
неизвестна ширина символа 0xe0 в кодировке CP1251
\end{verbatim}

\begin{verbatim}
Warning in grid.Call(C_textBounds, as.graphicsAnnot(x$label), x$x, x$y, :
неизвестна ширина символа 0xff в кодировке CP1251
\end{verbatim}

\begin{verbatim}
Warning in grid.Call(C_textBounds, as.graphicsAnnot(x$label), x$x, x$y, :
неизвестна ширина символа 0xf1 в кодировке CP1251
\end{verbatim}

\begin{verbatim}
Warning in grid.Call(C_textBounds, as.graphicsAnnot(x$label), x$x, x$y, :
неизвестна ширина символа 0xec в кодировке CP1251
\end{verbatim}

\begin{verbatim}
Warning in grid.Call(C_textBounds, as.graphicsAnnot(x$label), x$x, x$y, :
неизвестна ширина символа 0xe5 в кодировке CP1251
\end{verbatim}

\begin{verbatim}
Warning in grid.Call(C_textBounds, as.graphicsAnnot(x$label), x$x, x$y, :
неизвестна ширина символа 0xf0 в кодировке CP1251
\end{verbatim}

\begin{verbatim}
Warning in grid.Call(C_textBounds, as.graphicsAnnot(x$label), x$x, x$y, :
неизвестна ширина символа 0xf2 в кодировке CP1251
\end{verbatim}

\begin{verbatim}
Warning in grid.Call(C_textBounds, as.graphicsAnnot(x$label), x$x, x$y, :
неизвестна ширина символа 0xed в кодировке CP1251
\end{verbatim}

\begin{verbatim}
Warning in grid.Call(C_textBounds, as.graphicsAnnot(x$label), x$x, x$y, :
неизвестна ширина символа 0xee в кодировке CP1251
\end{verbatim}

\begin{verbatim}
Warning in grid.Call(C_textBounds, as.graphicsAnnot(x$label), x$x, x$y, :
неизвестна ширина символа 0xf1 в кодировке CP1251
\end{verbatim}

\begin{verbatim}
Warning in grid.Call(C_textBounds, as.graphicsAnnot(x$label), x$x, x$y, :
неизвестна ширина символа 0xf2 в кодировке CP1251
\end{verbatim}

\begin{verbatim}
Warning in grid.Call(C_textBounds, as.graphicsAnnot(x$label), x$x, x$y, :
неизвестна ширина символа 0xfc в кодировке CP1251
\end{verbatim}

\begin{verbatim}
Warning in grid.Call(C_textBounds, as.graphicsAnnot(x$label), x$x, x$y, :
неизвестна ширина символа 0xc3 в кодировке CP1251
\end{verbatim}

\begin{verbatim}
Warning in grid.Call(C_textBounds, as.graphicsAnnot(x$label), x$x, x$y, :
неизвестна ширина символа 0xee в кодировке CP1251
\end{verbatim}

\begin{verbatim}
Warning in grid.Call(C_textBounds, as.graphicsAnnot(x$label), x$x, x$y, :
неизвестна ширина символа 0xe4 в кодировке CP1251
\end{verbatim}

\begin{verbatim}
Warning in grid.Call.graphics(C_text, as.graphicsAnnot(x$label), x$x, x$y, :
неизвестна ширина символа 0xc3 в кодировке CP1251
\end{verbatim}

\begin{verbatim}
Warning in grid.Call.graphics(C_text, as.graphicsAnnot(x$label), x$x, x$y, :
неизвестна ширина символа 0xee в кодировке CP1251
\end{verbatim}

\begin{verbatim}
Warning in grid.Call.graphics(C_text, as.graphicsAnnot(x$label), x$x, x$y, :
неизвестна ширина символа 0xe4 в кодировке CP1251
\end{verbatim}

\begin{verbatim}
Warning in grid.Call.graphics(C_text, as.graphicsAnnot(x$label), x$x, x$y, :
неизвестна ширина символа 0xce в кодировке CP1251
\end{verbatim}

\begin{verbatim}
Warning in grid.Call.graphics(C_text, as.graphicsAnnot(x$label), x$x, x$y, :
неизвестна ширина символа 0xf2 в кодировке CP1251
\end{verbatim}

\begin{verbatim}
Warning in grid.Call.graphics(C_text, as.graphicsAnnot(x$label), x$x, x$y, :
неизвестна ширина символа 0xed в кодировке CP1251
\end{verbatim}

\begin{verbatim}
Warning in grid.Call.graphics(C_text, as.graphicsAnnot(x$label), x$x, x$y, :
неизвестна ширина символа 0xee в кодировке CP1251
\end{verbatim}

\begin{verbatim}
Warning in grid.Call.graphics(C_text, as.graphicsAnnot(x$label), x$x, x$y, :
неизвестна ширина символа 0xf1 в кодировке CP1251
\end{verbatim}

\begin{verbatim}
Warning in grid.Call.graphics(C_text, as.graphicsAnnot(x$label), x$x, x$y, :
неизвестна ширина символа 0xe8 в кодировке CP1251
\end{verbatim}

\begin{verbatim}
Warning in grid.Call.graphics(C_text, as.graphicsAnnot(x$label), x$x, x$y, :
неизвестна ширина символа 0xf2 в кодировке CP1251
\end{verbatim}

\begin{verbatim}
Warning in grid.Call.graphics(C_text, as.graphicsAnnot(x$label), x$x, x$y, :
неизвестна ширина символа 0xe5 в кодировке CP1251
\end{verbatim}

\begin{verbatim}
Warning in grid.Call.graphics(C_text, as.graphicsAnnot(x$label), x$x, x$y, :
неизвестна ширина символа 0xeb в кодировке CP1251
\end{verbatim}

\begin{verbatim}
Warning in grid.Call.graphics(C_text, as.graphicsAnnot(x$label), x$x, x$y, :
неизвестна ширина символа 0xfc в кодировке CP1251
\end{verbatim}

\begin{verbatim}
Warning in grid.Call.graphics(C_text, as.graphicsAnnot(x$label), x$x, x$y, :
неизвестна ширина символа 0xed в кодировке CP1251
\end{verbatim}

\begin{verbatim}
Warning in grid.Call.graphics(C_text, as.graphicsAnnot(x$label), x$x, x$y, :
неизвестна ширина символа 0xe0 в кодировке CP1251
\end{verbatim}

\begin{verbatim}
Warning in grid.Call.graphics(C_text, as.graphicsAnnot(x$label), x$x, x$y, :
неизвестна ширина символа 0xff в кодировке CP1251
\end{verbatim}

\begin{verbatim}
Warning in grid.Call.graphics(C_text, as.graphicsAnnot(x$label), x$x, x$y, :
неизвестна ширина символа 0xf1 в кодировке CP1251
\end{verbatim}

\begin{verbatim}
Warning in grid.Call.graphics(C_text, as.graphicsAnnot(x$label), x$x, x$y, :
неизвестна ширина символа 0xec в кодировке CP1251
\end{verbatim}

\begin{verbatim}
Warning in grid.Call.graphics(C_text, as.graphicsAnnot(x$label), x$x, x$y, :
неизвестна ширина символа 0xe5 в кодировке CP1251
\end{verbatim}

\begin{verbatim}
Warning in grid.Call.graphics(C_text, as.graphicsAnnot(x$label), x$x, x$y, :
неизвестна ширина символа 0xf0 в кодировке CP1251
\end{verbatim}

\begin{verbatim}
Warning in grid.Call.graphics(C_text, as.graphicsAnnot(x$label), x$x, x$y, :
неизвестна ширина символа 0xf2 в кодировке CP1251
\end{verbatim}

\begin{verbatim}
Warning in grid.Call.graphics(C_text, as.graphicsAnnot(x$label), x$x, x$y, :
неизвестна ширина символа 0xed в кодировке CP1251
\end{verbatim}

\begin{verbatim}
Warning in grid.Call.graphics(C_text, as.graphicsAnnot(x$label), x$x, x$y, :
неизвестна ширина символа 0xee в кодировке CP1251
\end{verbatim}

\begin{verbatim}
Warning in grid.Call.graphics(C_text, as.graphicsAnnot(x$label), x$x, x$y, :
неизвестна ширина символа 0xf1 в кодировке CP1251
\end{verbatim}

\begin{verbatim}
Warning in grid.Call.graphics(C_text, as.graphicsAnnot(x$label), x$x, x$y, :
неизвестна ширина символа 0xf2 в кодировке CP1251
\end{verbatim}

\begin{verbatim}
Warning in grid.Call.graphics(C_text, as.graphicsAnnot(x$label), x$x, x$y, :
неизвестна ширина символа 0xfc в кодировке CP1251
\end{verbatim}

\begin{verbatim}
Warning in grid.Call.graphics(C_text, as.graphicsAnnot(x$label), x$x, x$y, :
неизвестна ширина символа 0xc2 в кодировке CP1251
\end{verbatim}

\begin{verbatim}
Warning in grid.Call.graphics(C_text, as.graphicsAnnot(x$label), x$x, x$y, :
неизвестна ширина символа 0xf0 в кодировке CP1251
\end{verbatim}

\begin{verbatim}
Warning in grid.Call.graphics(C_text, as.graphicsAnnot(x$label), x$x, x$y, :
неизвестна ширина символа 0xe5 в кодировке CP1251
\end{verbatim}

\begin{verbatim}
Warning in grid.Call.graphics(C_text, as.graphicsAnnot(x$label), x$x, x$y, :
неизвестна ширина символа 0xec в кодировке CP1251
\end{verbatim}

\begin{verbatim}
Warning in grid.Call.graphics(C_text, as.graphicsAnnot(x$label), x$x, x$y, :
неизвестна ширина символа 0xe5 в кодировке CP1251
\end{verbatim}

\begin{verbatim}
Warning in grid.Call.graphics(C_text, as.graphicsAnnot(x$label), x$x, x$y, :
неизвестна ширина символа 0xed в кодировке CP1251
Warning in grid.Call.graphics(C_text, as.graphicsAnnot(x$label), x$x, x$y, :
неизвестна ширина символа 0xed в кодировке CP1251
\end{verbatim}

\begin{verbatim}
Warning in grid.Call.graphics(C_text, as.graphicsAnnot(x$label), x$x, x$y, :
неизвестна ширина символа 0xfb в кодировке CP1251
\end{verbatim}

\begin{verbatim}
Warning in grid.Call.graphics(C_text, as.graphicsAnnot(x$label), x$x, x$y, :
неизвестна ширина символа 0xe5 в кодировке CP1251
\end{verbatim}

\begin{verbatim}
Warning in grid.Call.graphics(C_text, as.graphicsAnnot(x$label), x$x, x$y, :
неизвестна ширина символа 0xf0 в кодировке CP1251
\end{verbatim}

\begin{verbatim}
Warning in grid.Call.graphics(C_text, as.graphicsAnnot(x$label), x$x, x$y, :
неизвестна ширина символа 0xff в кодировке CP1251
\end{verbatim}

\begin{verbatim}
Warning in grid.Call.graphics(C_text, as.graphicsAnnot(x$label), x$x, x$y, :
неизвестна ширина символа 0xe4 в кодировке CP1251
\end{verbatim}

\begin{verbatim}
Warning in grid.Call.graphics(C_text, as.graphicsAnnot(x$label), x$x, x$y, :
неизвестна ширина символа 0xfb в кодировке CP1251
\end{verbatim}

\begin{verbatim}
Warning in grid.Call.graphics(C_text, as.graphicsAnnot(x$label), x$x, x$y, :
неизвестна ширина символа 0xea в кодировке CP1251
\end{verbatim}

\begin{verbatim}
Warning in grid.Call.graphics(C_text, as.graphicsAnnot(x$label), x$x, x$y, :
неизвестна ширина символа 0xeb в кодировке CP1251
\end{verbatim}

\begin{verbatim}
Warning in grid.Call.graphics(C_text, as.graphicsAnnot(x$label), x$x, x$y, :
неизвестна ширина символа 0xfe в кодировке CP1251
\end{verbatim}

\begin{verbatim}
Warning in grid.Call.graphics(C_text, as.graphicsAnnot(x$label), x$x, x$y, :
неизвестна ширина символа 0xf7 в кодировке CP1251
\end{verbatim}

\begin{verbatim}
Warning in grid.Call.graphics(C_text, as.graphicsAnnot(x$label), x$x, x$y, :
неизвестна ширина символа 0xe5 в кодировке CP1251
\end{verbatim}

\begin{verbatim}
Warning in grid.Call.graphics(C_text, as.graphicsAnnot(x$label), x$x, x$y, :
неизвестна ширина символа 0xe2 в кодировке CP1251
\end{verbatim}

\begin{verbatim}
Warning in grid.Call.graphics(C_text, as.graphicsAnnot(x$label), x$x, x$y, :
неизвестна ширина символа 0xfb в кодировке CP1251
\end{verbatim}

\begin{verbatim}
Warning in grid.Call.graphics(C_text, as.graphicsAnnot(x$label), x$x, x$y, :
неизвестна ширина символа 0xf5 в кодировке CP1251
\end{verbatim}

\begin{verbatim}
Warning in grid.Call.graphics(C_text, as.graphicsAnnot(x$label), x$x, x$y, :
неизвестна ширина символа 0xef в кодировке CP1251
\end{verbatim}

\begin{verbatim}
Warning in grid.Call.graphics(C_text, as.graphicsAnnot(x$label), x$x, x$y, :
неизвестна ширина символа 0xe0 в кодировке CP1251
\end{verbatim}

\begin{verbatim}
Warning in grid.Call.graphics(C_text, as.graphicsAnnot(x$label), x$x, x$y, :
неизвестна ширина символа 0xf0 в кодировке CP1251
\end{verbatim}

\begin{verbatim}
Warning in grid.Call.graphics(C_text, as.graphicsAnnot(x$label), x$x, x$y, :
неизвестна ширина символа 0xe0 в кодировке CP1251
\end{verbatim}

\begin{verbatim}
Warning in grid.Call.graphics(C_text, as.graphicsAnnot(x$label), x$x, x$y, :
неизвестна ширина символа 0xec в кодировке CP1251
\end{verbatim}

\begin{verbatim}
Warning in grid.Call.graphics(C_text, as.graphicsAnnot(x$label), x$x, x$y, :
неизвестна ширина символа 0xe5 в кодировке CP1251
\end{verbatim}

\begin{verbatim}
Warning in grid.Call.graphics(C_text, as.graphicsAnnot(x$label), x$x, x$y, :
неизвестна ширина символа 0xf2 в кодировке CP1251
\end{verbatim}

\begin{verbatim}
Warning in grid.Call.graphics(C_text, as.graphicsAnnot(x$label), x$x, x$y, :
неизвестна ширина символа 0xf0 в кодировке CP1251
\end{verbatim}

\begin{verbatim}
Warning in grid.Call.graphics(C_text, as.graphicsAnnot(x$label), x$x, x$y, :
неизвестна ширина символа 0xee в кодировке CP1251
\end{verbatim}

\begin{verbatim}
Warning in grid.Call.graphics(C_text, as.graphicsAnnot(x$label), x$x, x$y, :
неизвестна ширина символа 0xe2 в кодировке CP1251
\end{verbatim}

\pandocbounded{\includegraphics[keepaspectratio]{chapter13_files/figure-pdf/unnamed-chunk-5-7.pdf}}

\begin{Shaded}
\begin{Highlighting}[]
\CommentTok{\# Сохраняем}
\FunctionTok{ggsave}\NormalTok{(}\StringTok{"SPiCT\_time\_series.png"}\NormalTok{, grid\_plot, }\AttributeTok{width =} \DecValTok{12}\NormalTok{, }\AttributeTok{height =} \DecValTok{10}\NormalTok{, }\AttributeTok{dpi =} \DecValTok{300}\NormalTok{)}
\FunctionTok{cat}\NormalTok{(}\StringTok{"График временных рядов сохранен как \textquotesingle{}SPiCT\_time\_series.png\textquotesingle{}}\SpecialCharTok{\textbackslash{}n}\StringTok{"}\NormalTok{)}
\end{Highlighting}
\end{Shaded}

\begin{verbatim}
График временных рядов сохранен как 'SPiCT_time_series.png'
\end{verbatim}

\begin{Shaded}
\begin{Highlighting}[]
\CommentTok{\# {-}{-}{-}{-}{-}{-}{-}{-}{-}{-}{-}{-}{-}{-}{-}{-}{-}{-}{-} 9. СОХРАНЕНИЕ РЕЗУЛЬТАТОВ {-}{-}{-}{-}{-}{-}{-}{-}{-}{-}{-}{-}{-}{-}{-}{-}{-}{-}{-}{-}}

\FunctionTok{cat}\NormalTok{(}\StringTok{"}\SpecialCharTok{\textbackslash{}n}\StringTok{========== СОХРАНЕНИЕ РЕЗУЛЬТАТОВ ==========}\SpecialCharTok{\textbackslash{}n}\StringTok{"}\NormalTok{)}
\end{Highlighting}
\end{Shaded}

\begin{verbatim}

========== СОХРАНЕНИЕ РЕЗУЛЬТАТОВ ==========
\end{verbatim}

\begin{Shaded}
\begin{Highlighting}[]
\DocumentationTok{\#\# 9.1 Сохранение объекта модели}
\CommentTok{\#saveRDS(fit, "spict\_model\_fit.rds")}
\CommentTok{\#cat("Модель сохранена в \textquotesingle{}spict\_model\_fit.rds\textquotesingle{}\textbackslash{}n")}

\DocumentationTok{\#\# 9.2 Экспорт таблицы с результатами}
\NormalTok{results\_table }\OtherTok{\textless{}{-}} \FunctionTok{data.frame}\NormalTok{(}
  \AttributeTok{Parameter =} \FunctionTok{c}\NormalTok{(}\StringTok{"r"}\NormalTok{, }\StringTok{"K"}\NormalTok{, }\StringTok{"MSY"}\NormalTok{, }\StringTok{"Bmsy"}\NormalTok{, }\StringTok{"Fmsy"}\NormalTok{, }
                \StringTok{"B\_current"}\NormalTok{, }\StringTok{"F\_current"}\NormalTok{, }\StringTok{"B/Bmsy"}\NormalTok{, }\StringTok{"F/Fmsy"}\NormalTok{),}
  \AttributeTok{Estimate =} \FunctionTok{c}\NormalTok{(r\_est[}\DecValTok{1}\NormalTok{], K\_est[}\DecValTok{1}\NormalTok{], MSY[}\DecValTok{1}\NormalTok{], Bmsy[}\DecValTok{1}\NormalTok{], Fmsy[}\DecValTok{1}\NormalTok{],}
\NormalTok{               B\_current[}\DecValTok{1}\NormalTok{], F\_current[}\DecValTok{1}\NormalTok{], B\_Bmsy[}\DecValTok{1}\NormalTok{], F\_Fmsy[}\DecValTok{1}\NormalTok{]),}
  \AttributeTok{Lower\_CI =} \FunctionTok{c}\NormalTok{(r\_est[}\DecValTok{2}\NormalTok{], K\_est[}\DecValTok{2}\NormalTok{], MSY[}\DecValTok{2}\NormalTok{], Bmsy[}\DecValTok{2}\NormalTok{], Fmsy[}\DecValTok{2}\NormalTok{],}
\NormalTok{               B\_current[}\DecValTok{2}\NormalTok{], F\_current[}\DecValTok{2}\NormalTok{], B\_Bmsy[}\DecValTok{2}\NormalTok{], F\_Fmsy[}\DecValTok{2}\NormalTok{]),}
  \AttributeTok{Upper\_CI =} \FunctionTok{c}\NormalTok{(r\_est[}\DecValTok{3}\NormalTok{], K\_est[}\DecValTok{3}\NormalTok{], MSY[}\DecValTok{3}\NormalTok{], Bmsy[}\DecValTok{3}\NormalTok{], Fmsy[}\DecValTok{3}\NormalTok{],}
\NormalTok{               B\_current[}\DecValTok{3}\NormalTok{], F\_current[}\DecValTok{3}\NormalTok{], B\_Bmsy[}\DecValTok{3}\NormalTok{], F\_Fmsy[}\DecValTok{3}\NormalTok{])}
\NormalTok{)}

\FunctionTok{write.csv}\NormalTok{(results\_table, }\StringTok{"spict\_results.csv"}\NormalTok{, }\AttributeTok{row.names =} \ConstantTok{FALSE}\NormalTok{)}
\FunctionTok{cat}\NormalTok{(}\StringTok{"Таблица результатов сохранена в \textquotesingle{}spict\_results.csv\textquotesingle{}}\SpecialCharTok{\textbackslash{}n}\StringTok{"}\NormalTok{)}
\end{Highlighting}
\end{Shaded}

\begin{verbatim}
Таблица результатов сохранена в 'spict_results.csv'
\end{verbatim}

\begin{Shaded}
\begin{Highlighting}[]
\DocumentationTok{\#\# 9.3 Создание отчета}
\CommentTok{\#cat("\textbackslash{}n========== ИТОГОВЫЙ ОТЧЕТ ==========\textbackslash{}n")}
\CommentTok{\#cat("Дата анализа:", format(Sys.Date(), "\%d.\%m.\%Y"), "\textbackslash{}n")}
\CommentTok{\#cat("Версия SPiCT:", packageVersion("spict"), "\textbackslash{}n")}
\CommentTok{\#cat("Период данных:", min(inp$timeC), "{-}", max(inp$timeC), "\textbackslash{}n")}
\CommentTok{\#cat("Количество наблюдений:", length(inp$obsC), "\textbackslash{}n")}
\CommentTok{\#cat("Сходимость модели:", ifelse(fit$opt$convergence == 0, "Да", "Нет"), "\textbackslash{}n")}
\CommentTok{\#cat("\textbackslash{}nОСНОВНЫЕ ВЫВОДЫ:\textbackslash{}n")}
\CommentTok{\#cat("1. Текущая биомасса составляет", round(B\_Bmsy[1]*100), "\% от Bmsy\textbackslash{}n")}
\CommentTok{\#cat("2. Промысловая смертность составляет", round(F\_Fmsy[1]*100), "\% от Fmsy\textbackslash{}n")}
\CommentTok{\#cat("3. Рекомендуемый вылов (при F=Fmsy):", round(Fmsy[1] * B\_current[1], 1), "тыс. т\textbackslash{}n")}

\CommentTok{\#cat("\textbackslash{}n=============== КОНЕЦ АНАЛИЗА ===============\textbackslash{}n")}
\end{Highlighting}
\end{Shaded}

\section{Результаты выполнения первого
скрипта}\label{ux440ux435ux437ux443ux43bux44cux442ux430ux442ux44b-ux432ux44bux43fux43eux43bux43dux435ux43dux438ux44f-ux43fux435ux440ux432ux43eux433ux43e-ux441ux43aux440ux438ux43fux442ux430}

Анализ выполнен для условного запаса с данными за 2005--2024 годы. В
качестве индексов биомассы использовались промысловый индекс CPUE
(собираемый в середине года) и индекс научной съемки BESS (четвертый
квартал). Визуализация исходных данных показала ожидаемую динамику: рост
вылова до 2014 года с последующим снижением, что согласуется с
снижающимися трендами в индексах биомассы.

Модель успешно сошлась (код сходимости 0), что уже можно считать
небольшим достижением --- статистические модели не всегда сходятся с
первого раза, особенно при работе с реальными, а не идеализированными
данными. Диагностика остатков не выявила серьезных проблем: тесты на
нормальность (Шапиро-Уилк) и автокорреляцию (Льюнг-Бокс) для наблюдений
вылова и обоих индексов дали незначимые p-значения (p \textgreater{}
0.05). Это означает, что остатки модели ведут себя предсказуемо и не
нарушают ключевых предположений метода.

Ретроспективный анализ показал исключительно низкие значения Mohn's rho
(ρ для B/Bmsy = -0.0002, для F/Fmsy = 0.0029), что свидетельствует о
высочайшей стабильности оценок модели и отсутствии ретроспективного
смещения --- редкий и прекрасный результат на реальных данных.

Оценки параметров популяции оказались биологически правдоподобными: темп
роста r = 0.377 {[}0.289--0.491{]} год⁻¹, что характерно для видов со
средней продуктивностью. Несущая способность среды K оценена в 189.6
{[}153.8--233.6{]} тыс. тонн. Максимальный устойчивый вылов (MSY)
составил 17.8 {[}16.3--19.6{]} тыс. тонн в год.

Ключевой результат --- оценка текущего состояния запаса. По состоянию на
конец 2024 года запас находится в хорошем состоянии: относительная
биомасса B/Bmsy = 1.17 {[}1.16--1.18{]}, а промысловая смертность F/Fmsy
= 0.10 {[}0.10--0.11{]}. Это означает, что запас превышает целевой
уровень Bmsy, а промысловое давление существенно ниже пределього уровня
Fmsy. Говоря управленческим языком, запас находится в «зеленой зоне» и
имеет значительную устойчивость к потенциальным ошибкам управления.

\section{Заключение по первому этапу
анализа}\label{ux437ux430ux43aux43bux44eux447ux435ux43dux438ux435-ux43fux43e-ux43fux435ux440ux432ux43eux43cux443-ux44dux442ux430ux43fux443-ux430ux43dux430ux43bux438ux437ux430}

Первый скрипт успешно выполнил свою задачу: мы перешли от сырых данных к
параметризованной модели, дающей количественные оценки состояния запаса.
Важно подчеркнуть, что это не конец работы, а только начало. Полученные
оценки --- это не истина в последней инстанции, а наиболее вероятное
состояние системы given the data and the model.

Тот факт, что модель сошлась, прошла диагностику и дала биологически
осмысленные результаты с узкими доверительными интервалами, вселяет
осторожный оптимизм. Однако следует помнить, что модель --- это
упрощение. Например, мы зафиксировали параметр формы продукционной
кривой (n = 2, модель Шефера), хотя в реальности он может отличаться. Мы
использовали информативные априорные распределения для K и начальной
биомассы, что помогло стабилизировать оценки, но одновременно внесло в
анализ субъективный элемент.

Полученная картина благополучного состояния запаса выглядит
правдоподобно на фоне снижения выловов в последнее десятилетие. Низкая
промысловая смертность позволяет запасу восстанавливаться. Теперь, имея
на руках оцененную модель, мы можем переходить к следующему шагу ---
исследованию различных сценариев управления и расчету потенциальных
выловов, что и является содержанием последующих скриптов. По иронии
судьбы, самая сложная часть --- получение надежной модели --- часто
оказывается проще, чем последующее принятие управленческих решений на ее
основе.

\bookmarksetup{startatroot}

\chapter{II. SPiCT: ПРП и
ОДУ}\label{ii.-spict-ux43fux440ux43f-ux438-ux43eux434ux443}

\section{Правила управления промыслом: концепция и
классификация}\label{ux43fux440ux430ux432ux438ux43bux430-ux443ux43fux440ux430ux432ux43bux435ux43dux438ux44f-ux43fux440ux43eux43cux44bux441ux43bux43eux43c-ux43aux43eux43dux446ux435ux43fux446ux438ux44f-ux438-ux43aux43bux430ux441ux441ux438ux444ux438ux43aux430ux446ux438ux44f}

Модель --- это про то, «как устроен мир», а управление --- про то, «что
делать завтра». Правило регулирования промысла (ПРП, на англ. Harvest
Control Rule, HCR) --- та самая мостовая плита между оценкой и решением:
простая и воспроизводимая функция, которая из состояния запаса делает
рекомендацию по смертности или вылову. Здесь интуиция особенно любит
крайности: «держать \emph{F} постоянной --- стабильно», «держать вылов
постоянным --- надёжно». На практике устойчивость промысла и риск для
запаса меняются местами, как минимум потому, что природа и данные шумят.

Что такое HCR по делу. Это карта вида «если
\emph{B/B\textsubscript{msy}} такая-то, то \emph{F} --- такая-то», с
предосторожными точками (\emph{B\textsubscript{lim}},
\emph{B\textsubscript{pa}}), с ограничителями на межгодовое изменение
TAC и с явным «буфером» на неопределённость (P*‑подход, нижние квантили
биомассы). Важно, что правило --- не случайная кривая, а договор о
компромиссе между тремя силами: средним выловом, риском перелова и
стабильностью межгодовой динамики.

Распространённые формы --- и их характер. - Постоянная \emph{F}
(\emph{F\textsubscript{constant}}): отзывчива к состоянию, но даёт
переменный вылов; чувствительна к ошибкам в \emph{B}. - Постоянный вылов
(\emph{C\textsubscript{constant}}): сглаживает экономику, но рискует
«надавить» на слабый запас. - «Хоккейная клюшка» (hockey stick):
\emph{F} линейно снижается от \emph{B\textsubscript{msy}} к
\emph{B\textsubscript{lim}}, при
\emph{B}\textless{}\emph{B\textsubscript{lim}} --- закрытие; понятна,
наглядна, управляемо‑предосторожна. - 40--10 (дефолт на Тихом океане
США): \emph{F}=\emph{F\textsubscript{target}} при
\emph{B}/\emph{B\textsubscript{0}}≥40\%, линейно к нулю до 10\%;
работает, когда известна исходная биомасса \emph{B\textsubscript{0}}. -
Советы ICES: \emph{F\textsubscript{pa}} и \emph{B\textsubscript{pa}} как
предосторожные точки, плюс ограничения ±15--20\% к ОДУ; про
устойчивость, а не про «максимум любой ценой».

\subsection{\texorpdfstring{\textbf{Типовые
ловушки}}{Типовые ловушки}}\label{ux442ux438ux43fux43eux432ux44bux435-ux43bux43eux432ux443ux448ux43aux438}

Это ошибки при проектировании или реализации HCR, которые сводят на нет
их эффективность и предосторожность.

\begin{enumerate}
\def\labelenumi{\arabic{enumi}.}
\item
  \textbf{Ловушка «Нож-на-ребре» (Knife-edge)}

  \begin{itemize}
  \item
    \textbf{Суть:} Использование жестких, абсолютных порогов без плавных
    переходов или ``буферных зон''.
  \item
    \textbf{Проблема:} Незначительное колебание оценки запаса вокруг
    жесткого порога (например, \emph{B\textsubscript{lim}}) приводит к
    резким, скачкообразным изменениям в ОДУ: сегодня промысел открыт с
    хорошим квотой, завтра оценка чуть ниже --- и полное закрытие. Это
    создает ``пилу'' в рекомендациях, что разрушительно для экономики.
  \item
    \textbf{Решение:} Внедрение полосы нечувствительности и плавного
    ``склона''. Например, как в правиле ``хоккейная клюшка'', где
    снижение \emph{F} начинается не резко, а линейно от
    \emph{B\textsubscript{pa}} до \emph{B\textsubscript{lim}}.
  \end{itemize}
\item
  \textbf{Игнорирование неопределенности}

  \begin{itemize}
  \item
    \textbf{Суть:} Разработка правила, основанного исключительно на
    точечных оценках (``среднее по больице''), без учета ошибок и
    неопределенности в данных.
  \item
    \textbf{Проблема:} В реальности наша оценка запаса --- это не точное
    число, а вероятностное распределение. Если советовать вылов исходя
    из средней оценки, но реальное состояние запаса находится в
    ``хвосте'' распределения (т.е. хуже, чем мы думаем), это приведет к
    перелову.
  \item
    \textbf{Решение:} Использование P*-подхода (подхода, основанного на
    вероятности). Управленческое решение принимается так, чтобы
    вероятность падения запаса ниже \emph{B\textsubscript{lim}} не
    превышала заранее выбранного низкого уровня (напр., 5\%).
    Фактически, мы рассчитываем ОДУ для пессимистичного сценария.
  \end{itemize}
\item
  \textbf{Оптимизация на средний (исторический) сценарий}

  \begin{itemize}
  \item
    \textbf{Суть:} Настройка правила так, чтобы оно идеально работало на
    исторических данных.
  \item
    \textbf{Проблема:} Такое правило оказывается ``заточено'' под
    прошлые условия и оказывается хрупким. Первое же непредвиденное
    изменение в системе (например, дрейф уловистости --- когда
    эффективность промысла меняется из-за погоды, поведения рыбы или
    технологий) опрокинет его.
  \item
    \textbf{Решение:} Проверка (тестирование) правила не на истории, а в
    рамках оценки стратегии управления (MSE), где модель ``генерирует''
    тысячи возможных будущих сценариев, включая ошибки и изменения.
    Правило должно быть \emph{робастным} (устойчивым) к ним.
  \end{itemize}
\item
  \textbf{Слишком узкие «тормоза» на ОДУ}

  \begin{itemize}
  \item
    \textbf{Суть:} Введение очень жестких ограничений на межгодовое
    изменение ОДУ (например, не более ±5-10\% в год) исключительно ради
    стабильности.
  \item
    \textbf{Проблема:} Если запас объективно начал снижаться, такое
    правило не позволит быстро и адекватно сократить вылов. Неизбежное
    снижение ОДУ ``откладывается'' на годы, в течение которых промысел
    продолжается на чрезмерно высоком уровне, усугубляя проблему и
    увеличивая совокупный риск для запаса.
  \item
    \textbf{Решение:} Поиск баланса. Ограничения на изменение TAC нужны
    (напр., ±20\%), но они не должны быть слишком узкими, чтобы не
    мешать необходимой адаптации.
  \end{itemize}
\end{enumerate}

Что делаем в этой главе. - Формализуем HCR как функцию от \emph{B},
\emph{B\textsubscript{msy}}, \emph{F\textsubscript{msy}} (и при
необходимости \emph{B\textsubscript{0}}, \emph{B\textsubscript{lim}},
\emph{B\textsubscript{pa}}), обсуждаем буферы и квантильные ОДУ.

- Переводим F‑совет в ОДУ с учётом последнего года и его дисперсии;
вводим ограничения на межгодовые изменения.

\begin{itemize}
\item
  Сравниваем формы правил на одних и тех же оценках (SPiCT): где
  «клюшка» лучше, чем постоянная F, и чем советы ICES отличаются от
  40--10.
\item
  Говорим о таблицах решений: прозрачные матрицы «если B/Bmsy в этом
  диапазоне --- TAC такой», с колонкой риска.
\end{itemize}

- Подготавливаем правила к испытаниям в MSE: параметры правила --- не
«выбраны», а «настроены» под цели и риски.

Хорошее ПРП --- это не «среднее арифметическое пожеланий», а записанный
на одной странице договор о том, как мы обмениваем краткосрочную выгоду
на долгосрочную устойчивость. Оно тем лучше, чем легче объясняется и
воспроизводится, и тем надёжнее, чем честнее обращается с
неопределённостью. Каждое из таких правил имеет свою область применения
и обоснование. Выбор конкретного ПРП зависит от множества факторов:
состояния запаса, качества данных, социально-экономического контекста и,
что немаловажно, управленческой философии. Как заметил бы Довлатов,
``управление рыболовством --- это искусство находить компромисс между
тем, что нужно рыбе, и тем, что хотят рыбаки''.

Полный скрипт можно скачать по
\href{https://mombus.github.io/cRab/data/SPICT_TAC.R}{ссылке}. Ниже
приводится исполнение скрипта и пояснения еще ниже.

\begin{Shaded}
\begin{Highlighting}[]
\CommentTok{\# ===============================================================}
\CommentTok{\#     СКРИПТ 2: ПРАВИЛА УПРАВЛЕНИЯ (HCR) И ОПРЕДЕЛЕНИЕ ОДУ}
\CommentTok{\#     Курс: Оценка водных биоресурсов при недостатке данных в R}
\CommentTok{\#     Автор: [Ваше имя]}
\CommentTok{\#     Дата создания: 2024}
\CommentTok{\# ===============================================================}

\CommentTok{\# ======================= ВВЕДЕНИЕ =============================}
\CommentTok{\# Этот скрипт демонстрирует различные правила управления промыслом}
\CommentTok{\# (Harvest Control Rules {-} HCR) и методы определения общего }
\CommentTok{\# допустимого улова (ОДУ/TAC) на основе результатов SPiCT}

\CommentTok{\# {-}{-}{-}{-}{-}{-}{-}{-}{-}{-}{-}{-}{-}{-}{-}{-}{-}{-}{-} 1. ПОДГОТОВКА СРЕДЫ {-}{-}{-}{-}{-}{-}{-}{-}{-}{-}{-}{-}{-}{-}{-}{-}{-}{-}{-}{-}}

\DocumentationTok{\#\# 1.1 Очистка среды и загрузка библиотек}
\FunctionTok{rm}\NormalTok{(}\AttributeTok{list =} \FunctionTok{ls}\NormalTok{())}
\FunctionTok{library}\NormalTok{(spict)}
\end{Highlighting}
\end{Shaded}

\begin{verbatim}
Загрузка требуемого пакета: TMB
\end{verbatim}

\begin{verbatim}
Welcome to spict_v1.3.8@107a32
\end{verbatim}

\begin{Shaded}
\begin{Highlighting}[]
\FunctionTok{library}\NormalTok{(tidyverse)}
\end{Highlighting}
\end{Shaded}

\begin{verbatim}
-- Attaching core tidyverse packages ------------------------ tidyverse 2.0.0 --
v dplyr     1.1.4     v readr     2.1.5
v forcats   1.0.0     v stringr   1.5.2
v ggplot2   4.0.0     v tibble    3.2.1
v lubridate 1.9.4     v tidyr     1.3.1
v purrr     1.0.4     
\end{verbatim}

\begin{verbatim}
-- Conflicts ------------------------------------------ tidyverse_conflicts() --
x dplyr::filter() masks stats::filter()
x dplyr::lag()    masks stats::lag()
i Use the conflicted package (<http://conflicted.r-lib.org/>) to force all conflicts to become errors
\end{verbatim}

\begin{Shaded}
\begin{Highlighting}[]
\FunctionTok{library}\NormalTok{(ggplot2)}
\FunctionTok{library}\NormalTok{(gridExtra)}
\end{Highlighting}
\end{Shaded}

\begin{verbatim}

Присоединяю пакет: 'gridExtra'

Следующий объект скрыт от 'package:dplyr':

    combine
\end{verbatim}

\begin{Shaded}
\begin{Highlighting}[]
\DocumentationTok{\#\# 1.2 Загрузка результатов из первого скрипта}
\CommentTok{\# Предполагаем, что модель уже подогнана и сохранена}
\FunctionTok{setwd}\NormalTok{(}\StringTok{"C:/SPICT"}\NormalTok{)}
\NormalTok{fit }\OtherTok{\textless{}{-}} \FunctionTok{readRDS}\NormalTok{(}\StringTok{"spict\_model\_fit.rds"}\NormalTok{)}

\FunctionTok{cat}\NormalTok{(}\StringTok{"}\SpecialCharTok{\textbackslash{}n}\StringTok{========== АНАЛИЗ ПРАВИЛ УПРАВЛЕНИЯ (HCR) ==========}\SpecialCharTok{\textbackslash{}n}\StringTok{"}\NormalTok{)}
\end{Highlighting}
\end{Shaded}

\begin{verbatim}

========== АНАЛИЗ ПРАВИЛ УПРАВЛЕНИЯ (HCR) ==========
\end{verbatim}

\begin{Shaded}
\begin{Highlighting}[]
\FunctionTok{cat}\NormalTok{(}\StringTok{"Модель загружена успешно}\SpecialCharTok{\textbackslash{}n}\StringTok{"}\NormalTok{)}
\end{Highlighting}
\end{Shaded}

\begin{verbatim}
Модель загружена успешно
\end{verbatim}

\begin{Shaded}
\begin{Highlighting}[]
\CommentTok{\# {-}{-}{-}{-}{-}{-}{-}{-}{-}{-}{-}{-}{-}{-}{-}{-}{-}{-}{-} 2. ОБЗОР СТАНДАРТНЫХ СЦЕНАРИЕВ {-}{-}{-}{-}{-}{-}{-}{-}{-}{-}{-}{-}{-}{-}{-}{-}{-}{-}{-}{-}}

\FunctionTok{cat}\NormalTok{(}\StringTok{"}\SpecialCharTok{\textbackslash{}n}\StringTok{========== СТАНДАРТНЫЕ СЦЕНАРИИ УПРАВЛЕНИЯ ==========}\SpecialCharTok{\textbackslash{}n}\StringTok{"}\NormalTok{)}
\end{Highlighting}
\end{Shaded}

\begin{verbatim}

========== СТАНДАРТНЫЕ СЦЕНАРИИ УПРАВЛЕНИЯ ==========
\end{verbatim}

\begin{Shaded}
\begin{Highlighting}[]
\DocumentationTok{\#\# 2.1 Использование функции sumspict.manage()}
\CommentTok{\# Эта функция рассчитывает прогнозы для различных сценариев управления}
\CommentTok{\# include.unc = TRUE включает оценку неопределенности}

\NormalTok{fit }\OtherTok{\textless{}{-}} \FunctionTok{manage}\NormalTok{(fit)}
\end{Highlighting}
\end{Shaded}

\begin{verbatim}
Selected scenario(s):  currentCatch, currentF, Fmsy, noF, reduceF25, increaseF25, msyHockeyStick, ices  
\end{verbatim}

\begin{Shaded}
\begin{Highlighting}[]
\DocumentationTok{\#\# 2.2 Вывод результатов}
\FunctionTok{cat}\NormalTok{(}\StringTok{"}\SpecialCharTok{\textbackslash{}n}\StringTok{Доступные сценарии управления:}\SpecialCharTok{\textbackslash{}n}\StringTok{"}\NormalTok{)}
\end{Highlighting}
\end{Shaded}

\begin{verbatim}

Доступные сценарии управления:
\end{verbatim}

\begin{Shaded}
\begin{Highlighting}[]
\NormalTok{manage\_results }\OtherTok{\textless{}{-}} \FunctionTok{sumspict.manage}\NormalTok{(fit, }\AttributeTok{include.unc =} \ConstantTok{TRUE}\NormalTok{)}
\end{Highlighting}
\end{Shaded}

\begin{verbatim}
SPiCT timeline:
                                                  
      Observations              Management        
    2005.00 - 2025.00        2025.00 - 2026.00    
 |-----------------------| ----------------------|

Management evaluation: 2026.00

Predicted catch for management period and states at management evaluation time:

                            C B/Bmsy F/Fmsy
1. Keep current catch    11.8   1.31   0.52
2. Keep current F        12.5   1.30   0.55
3. Fish at Fmsy          22.0   1.20   1.00
4. No fishing             0.0   1.42   0.00
5. Reduce F by 25%        9.5   1.33   0.41
6. Increase F by 25%     15.4   1.27   0.68
7. MSY hockey-stick rule 22.0   1.20   1.00
8. ICES advice rule      19.8   1.23   0.90

95% confidence intervals for states:

                         B/Bmsy.lo B/Bmsy.hi F/Fmsy.lo F/Fmsy.hi
1. Keep current catch         1.12      1.52      0.23      1.15
2. Keep current F             1.11      1.51      0.25      1.22
3. Fish at Fmsy               1.00      1.44      0.45      2.23
4. No fishing                 1.25      1.63      0.00      0.00
5. Reduce F by 25%            1.15      1.54      0.18      0.92
6. Increase F by 25%          1.08      1.49      0.31      1.53
7. MSY hockey-stick rule      1.00      1.44      0.45      2.23
8. ICES advice rule           1.03      1.46      0.40      2.00
\end{verbatim}

\begin{Shaded}
\begin{Highlighting}[]
\DocumentationTok{\#\# 2.3 Детальный разбор сценариев}
\FunctionTok{cat}\NormalTok{(}\StringTok{"}\SpecialCharTok{\textbackslash{}n}\StringTok{{-}{-}{-} ОПИСАНИЕ СЦЕНАРИЕВ {-}{-}{-}}\SpecialCharTok{\textbackslash{}n}\StringTok{"}\NormalTok{)}
\end{Highlighting}
\end{Shaded}

\begin{verbatim}

--- ОПИСАНИЕ СЦЕНАРИЕВ ---
\end{verbatim}

\begin{Shaded}
\begin{Highlighting}[]
\FunctionTok{cat}\NormalTok{(}\StringTok{"1. Keep current catch: Сохранить текущий уровень вылова}\SpecialCharTok{\textbackslash{}n}\StringTok{"}\NormalTok{)}
\end{Highlighting}
\end{Shaded}

\begin{verbatim}
1. Keep current catch: Сохранить текущий уровень вылова
\end{verbatim}

\begin{Shaded}
\begin{Highlighting}[]
\FunctionTok{cat}\NormalTok{(}\StringTok{"2. Keep current F: Сохранить текущую промысловую смертность}\SpecialCharTok{\textbackslash{}n}\StringTok{"}\NormalTok{)}
\end{Highlighting}
\end{Shaded}

\begin{verbatim}
2. Keep current F: Сохранить текущую промысловую смертность
\end{verbatim}

\begin{Shaded}
\begin{Highlighting}[]
\FunctionTok{cat}\NormalTok{(}\StringTok{"3. Fish at Fmsy: Промысел на уровне Fmsy (оптимальный)}\SpecialCharTok{\textbackslash{}n}\StringTok{"}\NormalTok{)}
\end{Highlighting}
\end{Shaded}

\begin{verbatim}
3. Fish at Fmsy: Промысел на уровне Fmsy (оптимальный)
\end{verbatim}

\begin{Shaded}
\begin{Highlighting}[]
\FunctionTok{cat}\NormalTok{(}\StringTok{"4. No fishing: Полное закрытие промысла}\SpecialCharTok{\textbackslash{}n}\StringTok{"}\NormalTok{)}
\end{Highlighting}
\end{Shaded}

\begin{verbatim}
4. No fishing: Полное закрытие промысла
\end{verbatim}

\begin{Shaded}
\begin{Highlighting}[]
\FunctionTok{cat}\NormalTok{(}\StringTok{"5. Reduce F by 25\%: Снизить F на 25\%}\SpecialCharTok{\textbackslash{}n}\StringTok{"}\NormalTok{)}
\end{Highlighting}
\end{Shaded}

\begin{verbatim}
5. Reduce F by 25%: Снизить F на 25%
\end{verbatim}

\begin{Shaded}
\begin{Highlighting}[]
\FunctionTok{cat}\NormalTok{(}\StringTok{"6. Increase F by 25\%: Увеличить F на 25\%}\SpecialCharTok{\textbackslash{}n}\StringTok{"}\NormalTok{)}
\end{Highlighting}
\end{Shaded}

\begin{verbatim}
6. Increase F by 25%: Увеличить F на 25%
\end{verbatim}

\begin{Shaded}
\begin{Highlighting}[]
\FunctionTok{cat}\NormalTok{(}\StringTok{"7. MSY hockey{-}stick: Правило хоккейной клюшки}\SpecialCharTok{\textbackslash{}n}\StringTok{"}\NormalTok{)}
\end{Highlighting}
\end{Shaded}

\begin{verbatim}
7. MSY hockey-stick: Правило хоккейной клюшки
\end{verbatim}

\begin{Shaded}
\begin{Highlighting}[]
\FunctionTok{cat}\NormalTok{(}\StringTok{"8. ICES advice rule: Правило ICES для научных рекомендаций}\SpecialCharTok{\textbackslash{}n}\StringTok{"}\NormalTok{)}
\end{Highlighting}
\end{Shaded}

\begin{verbatim}
8. ICES advice rule: Правило ICES для научных рекомендаций
\end{verbatim}

\begin{Shaded}
\begin{Highlighting}[]
\CommentTok{\# {-}{-}{-}{-}{-}{-}{-}{-}{-}{-}{-}{-}{-}{-}{-}{-}{-}{-}{-} 3. ДЕТАЛЬНЫЙ АНАЛИЗ КАЖДОГО СЦЕНАРИЯ {-}{-}{-}{-}{-}{-}{-}{-}{-}{-}{-}{-}{-}{-}{-}{-}{-}{-}{-}{-}}

\FunctionTok{cat}\NormalTok{(}\StringTok{"}\SpecialCharTok{\textbackslash{}n}\StringTok{========== ДЕТАЛЬНЫЙ АНАЛИЗ СЦЕНАРИЕВ ==========}\SpecialCharTok{\textbackslash{}n}\StringTok{"}\NormalTok{)}
\end{Highlighting}
\end{Shaded}

\begin{verbatim}

========== ДЕТАЛЬНЫЙ АНАЛИЗ СЦЕНАРИЕВ ==========
\end{verbatim}

\begin{Shaded}
\begin{Highlighting}[]
\DocumentationTok{\#\# 3.1 Функция для расчета детальных прогнозов}
\NormalTok{calculate\_scenario\_details }\OtherTok{\textless{}{-}} \ControlFlowTok{function}\NormalTok{(fit, scenario\_name, }\AttributeTok{F\_multiplier =} \FloatTok{1.0}\NormalTok{, }
                                      \AttributeTok{years\_ahead =} \DecValTok{5}\NormalTok{) \{}
  
  \CommentTok{\# Извлекаем текущие параметры}
\NormalTok{  B\_current }\OtherTok{\textless{}{-}} \FunctionTok{get.par}\NormalTok{(}\StringTok{"logB"}\NormalTok{, fit, }\AttributeTok{exp =} \ConstantTok{TRUE}\NormalTok{)[}\DecValTok{1}\NormalTok{]}
\NormalTok{  F\_current }\OtherTok{\textless{}{-}} \FunctionTok{get.par}\NormalTok{(}\StringTok{"logF"}\NormalTok{, fit, }\AttributeTok{exp =} \ConstantTok{TRUE}\NormalTok{)[}\DecValTok{1}\NormalTok{]}
\NormalTok{  r }\OtherTok{\textless{}{-}} \FunctionTok{get.par}\NormalTok{(}\StringTok{"logr"}\NormalTok{, fit, }\AttributeTok{exp =} \ConstantTok{TRUE}\NormalTok{)[}\DecValTok{1}\NormalTok{]}
\NormalTok{  K }\OtherTok{\textless{}{-}} \FunctionTok{get.par}\NormalTok{(}\StringTok{"logK"}\NormalTok{, fit, }\AttributeTok{exp =} \ConstantTok{TRUE}\NormalTok{)[}\DecValTok{1}\NormalTok{]}
\NormalTok{  Bmsy }\OtherTok{\textless{}{-}} \FunctionTok{get.par}\NormalTok{(}\StringTok{"logBmsy"}\NormalTok{, fit, }\AttributeTok{exp =} \ConstantTok{TRUE}\NormalTok{)[}\DecValTok{1}\NormalTok{]}
\NormalTok{  Fmsy }\OtherTok{\textless{}{-}} \FunctionTok{get.par}\NormalTok{(}\StringTok{"logFmsy"}\NormalTok{, fit, }\AttributeTok{exp =} \ConstantTok{TRUE}\NormalTok{)[}\DecValTok{1}\NormalTok{]}
  
  \CommentTok{\# Применяем сценарий}
\NormalTok{  F\_scenario }\OtherTok{\textless{}{-}}\NormalTok{ F\_current }\SpecialCharTok{*}\NormalTok{ F\_multiplier}
  
  \CommentTok{\# Прогнозирование}
\NormalTok{  B\_forecast }\OtherTok{\textless{}{-}} \FunctionTok{numeric}\NormalTok{(years\_ahead }\SpecialCharTok{+} \DecValTok{1}\NormalTok{)}
\NormalTok{  C\_forecast }\OtherTok{\textless{}{-}} \FunctionTok{numeric}\NormalTok{(years\_ahead)}
\NormalTok{  B\_forecast[}\DecValTok{1}\NormalTok{] }\OtherTok{\textless{}{-}}\NormalTok{ B\_current}
  
  \ControlFlowTok{for}\NormalTok{ (i }\ControlFlowTok{in} \DecValTok{1}\SpecialCharTok{:}\NormalTok{years\_ahead) \{}
    \CommentTok{\# Продукционная модель Шефера}
\NormalTok{    surplus }\OtherTok{\textless{}{-}}\NormalTok{ r }\SpecialCharTok{*}\NormalTok{ B\_forecast[i] }\SpecialCharTok{*}\NormalTok{ (}\DecValTok{1} \SpecialCharTok{{-}}\NormalTok{ B\_forecast[i]}\SpecialCharTok{/}\NormalTok{K)}
\NormalTok{    C\_forecast[i] }\OtherTok{\textless{}{-}}\NormalTok{ F\_scenario }\SpecialCharTok{*}\NormalTok{ B\_forecast[i]}
\NormalTok{    B\_forecast[i}\SpecialCharTok{+}\DecValTok{1}\NormalTok{] }\OtherTok{\textless{}{-}}\NormalTok{ B\_forecast[i] }\SpecialCharTok{+}\NormalTok{ surplus }\SpecialCharTok{{-}}\NormalTok{ C\_forecast[i]}
\NormalTok{    B\_forecast[i}\SpecialCharTok{+}\DecValTok{1}\NormalTok{] }\OtherTok{\textless{}{-}} \FunctionTok{max}\NormalTok{(B\_forecast[i}\SpecialCharTok{+}\DecValTok{1}\NormalTok{], }\FloatTok{0.01}\NormalTok{) }\CommentTok{\# Минимальная биомасса}
\NormalTok{  \}}
  
  \CommentTok{\# Результаты}
\NormalTok{  results }\OtherTok{\textless{}{-}} \FunctionTok{data.frame}\NormalTok{(}
    \AttributeTok{Year =} \DecValTok{0}\SpecialCharTok{:}\NormalTok{years\_ahead,}
    \AttributeTok{Scenario =}\NormalTok{ scenario\_name,}
    \AttributeTok{Biomass =}\NormalTok{ B\_forecast,}
    \AttributeTok{Catch =} \FunctionTok{c}\NormalTok{(}\ConstantTok{NA}\NormalTok{, C\_forecast),}
    \AttributeTok{B\_Bmsy =}\NormalTok{ B\_forecast }\SpecialCharTok{/}\NormalTok{ Bmsy,}
    \AttributeTok{F =} \FunctionTok{c}\NormalTok{(}\FunctionTok{rep}\NormalTok{(F\_scenario, years\_ahead }\SpecialCharTok{+} \DecValTok{1}\NormalTok{)),}
    \AttributeTok{F\_Fmsy =}\NormalTok{ F\_scenario }\SpecialCharTok{/}\NormalTok{ Fmsy}
\NormalTok{  )}
  
  \FunctionTok{return}\NormalTok{(results)}
\NormalTok{\}}

\DocumentationTok{\#\# 3.2 Расчет для всех базовых сценариев}
\NormalTok{scenarios\_list }\OtherTok{\textless{}{-}} \FunctionTok{list}\NormalTok{(}
  \StringTok{"Current F"} \OtherTok{=} \FloatTok{1.0}\NormalTok{,}
  \StringTok{"Fmsy"} \OtherTok{=} \FunctionTok{get.par}\NormalTok{(}\StringTok{"logFmsy"}\NormalTok{, fit, }\AttributeTok{exp =} \ConstantTok{TRUE}\NormalTok{)[}\DecValTok{1}\NormalTok{] }\SpecialCharTok{/} 
           \FunctionTok{get.par}\NormalTok{(}\StringTok{"logF"}\NormalTok{, fit, }\AttributeTok{exp =} \ConstantTok{TRUE}\NormalTok{)[}\DecValTok{1}\NormalTok{],}
  \StringTok{"No fishing"} \OtherTok{=} \DecValTok{0}\NormalTok{,}
  \StringTok{"Reduce F 25\%"} \OtherTok{=} \FloatTok{0.75}\NormalTok{,}
  \StringTok{"Increase F 25\%"} \OtherTok{=} \FloatTok{1.25}
\NormalTok{)}

\NormalTok{all\_scenarios }\OtherTok{\textless{}{-}} \FunctionTok{bind\_rows}\NormalTok{(}
  \FunctionTok{lapply}\NormalTok{(}\FunctionTok{names}\NormalTok{(scenarios\_list), }\ControlFlowTok{function}\NormalTok{(name) \{}
    \FunctionTok{calculate\_scenario\_details}\NormalTok{(fit, name, scenarios\_list[[name]], }\AttributeTok{years\_ahead =} \DecValTok{10}\NormalTok{)}
\NormalTok{  \})}
\NormalTok{)}

\DocumentationTok{\#\# 3.3 Визуализация сценариев}
\NormalTok{p\_scenarios }\OtherTok{\textless{}{-}} \FunctionTok{ggplot}\NormalTok{(all\_scenarios, }\FunctionTok{aes}\NormalTok{(}\AttributeTok{x =}\NormalTok{ Year, }\AttributeTok{y =}\NormalTok{ Biomass, }
                                         \AttributeTok{color =}\NormalTok{ Scenario, }\AttributeTok{linetype =}\NormalTok{ Scenario)) }\SpecialCharTok{+}
  \FunctionTok{geom\_line}\NormalTok{(}\AttributeTok{linewidth =} \FloatTok{1.2}\NormalTok{) }\SpecialCharTok{+}
  \FunctionTok{geom\_hline}\NormalTok{(}\AttributeTok{yintercept =} \FunctionTok{get.par}\NormalTok{(}\StringTok{"logBmsy"}\NormalTok{, fit, }\AttributeTok{exp =} \ConstantTok{TRUE}\NormalTok{)[}\DecValTok{1}\NormalTok{], }
             \AttributeTok{linetype =} \StringTok{"dashed"}\NormalTok{, }\AttributeTok{alpha =} \FloatTok{0.5}\NormalTok{) }\SpecialCharTok{+}
  \FunctionTok{labs}\NormalTok{(}\AttributeTok{title =} \StringTok{"Прогноз биомассы при различных сценариях управления"}\NormalTok{,}
       \AttributeTok{x =} \StringTok{"Годы вперед"}\NormalTok{, }\AttributeTok{y =} \StringTok{"Биомасса (тыс. т)"}\NormalTok{) }\SpecialCharTok{+}
  \FunctionTok{theme\_minimal}\NormalTok{() }\SpecialCharTok{+}
  \FunctionTok{scale\_color\_brewer}\NormalTok{(}\AttributeTok{palette =} \StringTok{"Set1"}\NormalTok{)}

\FunctionTok{print}\NormalTok{(p\_scenarios)}
\end{Highlighting}
\end{Shaded}

\begin{verbatim}
Warning in grid.Call(C_textBounds, as.graphicsAnnot(x$label), x$x, x$y, :
неизвестна ширина символа 0xc1 в кодировке CP1251
\end{verbatim}

\begin{verbatim}
Warning in grid.Call(C_textBounds, as.graphicsAnnot(x$label), x$x, x$y, :
неизвестна ширина символа 0xe8 в кодировке CP1251
\end{verbatim}

\begin{verbatim}
Warning in grid.Call(C_textBounds, as.graphicsAnnot(x$label), x$x, x$y, :
неизвестна ширина символа 0xee в кодировке CP1251
\end{verbatim}

\begin{verbatim}
Warning in grid.Call(C_textBounds, as.graphicsAnnot(x$label), x$x, x$y, :
неизвестна ширина символа 0xec в кодировке CP1251
\end{verbatim}

\begin{verbatim}
Warning in grid.Call(C_textBounds, as.graphicsAnnot(x$label), x$x, x$y, :
неизвестна ширина символа 0xe0 в кодировке CP1251
\end{verbatim}

\begin{verbatim}
Warning in grid.Call(C_textBounds, as.graphicsAnnot(x$label), x$x, x$y, :
неизвестна ширина символа 0xf1 в кодировке CP1251
Warning in grid.Call(C_textBounds, as.graphicsAnnot(x$label), x$x, x$y, :
неизвестна ширина символа 0xf1 в кодировке CP1251
\end{verbatim}

\begin{verbatim}
Warning in grid.Call(C_textBounds, as.graphicsAnnot(x$label), x$x, x$y, :
неизвестна ширина символа 0xe0 в кодировке CP1251
\end{verbatim}

\begin{verbatim}
Warning in grid.Call(C_textBounds, as.graphicsAnnot(x$label), x$x, x$y, :
неизвестна ширина символа 0xf2 в кодировке CP1251
\end{verbatim}

\begin{verbatim}
Warning in grid.Call(C_textBounds, as.graphicsAnnot(x$label), x$x, x$y, :
неизвестна ширина символа 0xfb в кодировке CP1251
\end{verbatim}

\begin{verbatim}
Warning in grid.Call(C_textBounds, as.graphicsAnnot(x$label), x$x, x$y, :
неизвестна ширина символа 0xf1 в кодировке CP1251
\end{verbatim}

\begin{verbatim}
Warning in grid.Call(C_textBounds, as.graphicsAnnot(x$label), x$x, x$y, :
неизвестна ширина символа 0xf2 в кодировке CP1251
\end{verbatim}

\begin{verbatim}
Warning in grid.Call(C_textBounds, as.graphicsAnnot(x$label), x$x, x$y, :
неизвестна ширина символа 0xcf в кодировке CP1251
\end{verbatim}

\begin{verbatim}
Warning in grid.Call(C_textBounds, as.graphicsAnnot(x$label), x$x, x$y, :
неизвестна ширина символа 0xf0 в кодировке CP1251
\end{verbatim}

\begin{verbatim}
Warning in grid.Call(C_textBounds, as.graphicsAnnot(x$label), x$x, x$y, :
неизвестна ширина символа 0xee в кодировке CP1251
\end{verbatim}

\begin{verbatim}
Warning in grid.Call(C_textBounds, as.graphicsAnnot(x$label), x$x, x$y, :
неизвестна ширина символа 0xe3 в кодировке CP1251
\end{verbatim}

\begin{verbatim}
Warning in grid.Call(C_textBounds, as.graphicsAnnot(x$label), x$x, x$y, :
неизвестна ширина символа 0xed в кодировке CP1251
\end{verbatim}

\begin{verbatim}
Warning in grid.Call(C_textBounds, as.graphicsAnnot(x$label), x$x, x$y, :
неизвестна ширина символа 0xee в кодировке CP1251
\end{verbatim}

\begin{verbatim}
Warning in grid.Call(C_textBounds, as.graphicsAnnot(x$label), x$x, x$y, :
неизвестна ширина символа 0xe7 в кодировке CP1251
\end{verbatim}

\begin{verbatim}
Warning in grid.Call(C_textBounds, as.graphicsAnnot(x$label), x$x, x$y, :
неизвестна ширина символа 0xe1 в кодировке CP1251
\end{verbatim}

\begin{verbatim}
Warning in grid.Call(C_textBounds, as.graphicsAnnot(x$label), x$x, x$y, :
неизвестна ширина символа 0xe8 в кодировке CP1251
\end{verbatim}

\begin{verbatim}
Warning in grid.Call(C_textBounds, as.graphicsAnnot(x$label), x$x, x$y, :
неизвестна ширина символа 0xee в кодировке CP1251
\end{verbatim}

\begin{verbatim}
Warning in grid.Call(C_textBounds, as.graphicsAnnot(x$label), x$x, x$y, :
неизвестна ширина символа 0xec в кодировке CP1251
\end{verbatim}

\begin{verbatim}
Warning in grid.Call(C_textBounds, as.graphicsAnnot(x$label), x$x, x$y, :
неизвестна ширина символа 0xe0 в кодировке CP1251
\end{verbatim}

\begin{verbatim}
Warning in grid.Call(C_textBounds, as.graphicsAnnot(x$label), x$x, x$y, :
неизвестна ширина символа 0xf1 в кодировке CP1251
Warning in grid.Call(C_textBounds, as.graphicsAnnot(x$label), x$x, x$y, :
неизвестна ширина символа 0xf1 в кодировке CP1251
\end{verbatim}

\begin{verbatim}
Warning in grid.Call(C_textBounds, as.graphicsAnnot(x$label), x$x, x$y, :
неизвестна ширина символа 0xfb в кодировке CP1251
\end{verbatim}

\begin{verbatim}
Warning in grid.Call(C_textBounds, as.graphicsAnnot(x$label), x$x, x$y, :
неизвестна ширина символа 0xef в кодировке CP1251
\end{verbatim}

\begin{verbatim}
Warning in grid.Call(C_textBounds, as.graphicsAnnot(x$label), x$x, x$y, :
неизвестна ширина символа 0xf0 в кодировке CP1251
\end{verbatim}

\begin{verbatim}
Warning in grid.Call(C_textBounds, as.graphicsAnnot(x$label), x$x, x$y, :
неизвестна ширина символа 0xe8 в кодировке CP1251
\end{verbatim}

\begin{verbatim}
Warning in grid.Call(C_textBounds, as.graphicsAnnot(x$label), x$x, x$y, :
неизвестна ширина символа 0xf0 в кодировке CP1251
\end{verbatim}

\begin{verbatim}
Warning in grid.Call(C_textBounds, as.graphicsAnnot(x$label), x$x, x$y, :
неизвестна ширина символа 0xe0 в кодировке CP1251
\end{verbatim}

\begin{verbatim}
Warning in grid.Call(C_textBounds, as.graphicsAnnot(x$label), x$x, x$y, :
неизвестна ширина символа 0xe7 в кодировке CP1251
\end{verbatim}

\begin{verbatim}
Warning in grid.Call(C_textBounds, as.graphicsAnnot(x$label), x$x, x$y, :
неизвестна ширина символа 0xeb в кодировке CP1251
\end{verbatim}

\begin{verbatim}
Warning in grid.Call(C_textBounds, as.graphicsAnnot(x$label), x$x, x$y, :
неизвестна ширина символа 0xe8 в кодировке CP1251
\end{verbatim}

\begin{verbatim}
Warning in grid.Call(C_textBounds, as.graphicsAnnot(x$label), x$x, x$y, :
неизвестна ширина символа 0xf7 в кодировке CP1251
\end{verbatim}

\begin{verbatim}
Warning in grid.Call(C_textBounds, as.graphicsAnnot(x$label), x$x, x$y, :
неизвестна ширина символа 0xed в кодировке CP1251
\end{verbatim}

\begin{verbatim}
Warning in grid.Call(C_textBounds, as.graphicsAnnot(x$label), x$x, x$y, :
неизвестна ширина символа 0xfb в кодировке CP1251
\end{verbatim}

\begin{verbatim}
Warning in grid.Call(C_textBounds, as.graphicsAnnot(x$label), x$x, x$y, :
неизвестна ширина символа 0xf5 в кодировке CP1251
\end{verbatim}

\begin{verbatim}
Warning in grid.Call(C_textBounds, as.graphicsAnnot(x$label), x$x, x$y, :
неизвестна ширина символа 0xf1 в кодировке CP1251
\end{verbatim}

\begin{verbatim}
Warning in grid.Call(C_textBounds, as.graphicsAnnot(x$label), x$x, x$y, :
неизвестна ширина символа 0xf6 в кодировке CP1251
\end{verbatim}

\begin{verbatim}
Warning in grid.Call(C_textBounds, as.graphicsAnnot(x$label), x$x, x$y, :
неизвестна ширина символа 0xe5 в кодировке CP1251
\end{verbatim}

\begin{verbatim}
Warning in grid.Call(C_textBounds, as.graphicsAnnot(x$label), x$x, x$y, :
неизвестна ширина символа 0xed в кодировке CP1251
\end{verbatim}

\begin{verbatim}
Warning in grid.Call(C_textBounds, as.graphicsAnnot(x$label), x$x, x$y, :
неизвестна ширина символа 0xe0 в кодировке CP1251
\end{verbatim}

\begin{verbatim}
Warning in grid.Call(C_textBounds, as.graphicsAnnot(x$label), x$x, x$y, :
неизвестна ширина символа 0xf0 в кодировке CP1251
\end{verbatim}

\begin{verbatim}
Warning in grid.Call(C_textBounds, as.graphicsAnnot(x$label), x$x, x$y, :
неизвестна ширина символа 0xe8 в кодировке CP1251
\end{verbatim}

\begin{verbatim}
Warning in grid.Call(C_textBounds, as.graphicsAnnot(x$label), x$x, x$y, :
неизвестна ширина символа 0xff в кодировке CP1251
\end{verbatim}

\begin{verbatim}
Warning in grid.Call(C_textBounds, as.graphicsAnnot(x$label), x$x, x$y, :
неизвестна ширина символа 0xf5 в кодировке CP1251
\end{verbatim}

\begin{verbatim}
Warning in grid.Call(C_textBounds, as.graphicsAnnot(x$label), x$x, x$y, :
неизвестна ширина символа 0xf3 в кодировке CP1251
\end{verbatim}

\begin{verbatim}
Warning in grid.Call(C_textBounds, as.graphicsAnnot(x$label), x$x, x$y, :
неизвестна ширина символа 0xef в кодировке CP1251
\end{verbatim}

\begin{verbatim}
Warning in grid.Call(C_textBounds, as.graphicsAnnot(x$label), x$x, x$y, :
неизвестна ширина символа 0xf0 в кодировке CP1251
\end{verbatim}

\begin{verbatim}
Warning in grid.Call(C_textBounds, as.graphicsAnnot(x$label), x$x, x$y, :
неизвестна ширина символа 0xe0 в кодировке CP1251
\end{verbatim}

\begin{verbatim}
Warning in grid.Call(C_textBounds, as.graphicsAnnot(x$label), x$x, x$y, :
неизвестна ширина символа 0xe2 в кодировке CP1251
\end{verbatim}

\begin{verbatim}
Warning in grid.Call(C_textBounds, as.graphicsAnnot(x$label), x$x, x$y, :
неизвестна ширина символа 0xeb в кодировке CP1251
\end{verbatim}

\begin{verbatim}
Warning in grid.Call(C_textBounds, as.graphicsAnnot(x$label), x$x, x$y, :
неизвестна ширина символа 0xe5 в кодировке CP1251
\end{verbatim}

\begin{verbatim}
Warning in grid.Call(C_textBounds, as.graphicsAnnot(x$label), x$x, x$y, :
неизвестна ширина символа 0xed в кодировке CP1251
\end{verbatim}

\begin{verbatim}
Warning in grid.Call(C_textBounds, as.graphicsAnnot(x$label), x$x, x$y, :
неизвестна ширина символа 0xe8 в кодировке CP1251
\end{verbatim}

\begin{verbatim}
Warning in grid.Call(C_textBounds, as.graphicsAnnot(x$label), x$x, x$y, :
неизвестна ширина символа 0xff в кодировке CP1251
\end{verbatim}

\begin{verbatim}
Warning in grid.Call(C_textBounds, as.graphicsAnnot(x$label), x$x, x$y, :
неизвестна ширина символа 0xc3 в кодировке CP1251
\end{verbatim}

\begin{verbatim}
Warning in grid.Call(C_textBounds, as.graphicsAnnot(x$label), x$x, x$y, :
неизвестна ширина символа 0xee в кодировке CP1251
\end{verbatim}

\begin{verbatim}
Warning in grid.Call(C_textBounds, as.graphicsAnnot(x$label), x$x, x$y, :
неизвестна ширина символа 0xe4 в кодировке CP1251
\end{verbatim}

\begin{verbatim}
Warning in grid.Call(C_textBounds, as.graphicsAnnot(x$label), x$x, x$y, :
неизвестна ширина символа 0xfb в кодировке CP1251
\end{verbatim}

\begin{verbatim}
Warning in grid.Call(C_textBounds, as.graphicsAnnot(x$label), x$x, x$y, :
неизвестна ширина символа 0xe2 в кодировке CP1251
\end{verbatim}

\begin{verbatim}
Warning in grid.Call(C_textBounds, as.graphicsAnnot(x$label), x$x, x$y, :
неизвестна ширина символа 0xef в кодировке CP1251
\end{verbatim}

\begin{verbatim}
Warning in grid.Call(C_textBounds, as.graphicsAnnot(x$label), x$x, x$y, :
неизвестна ширина символа 0xe5 в кодировке CP1251
\end{verbatim}

\begin{verbatim}
Warning in grid.Call(C_textBounds, as.graphicsAnnot(x$label), x$x, x$y, :
неизвестна ширина символа 0xf0 в кодировке CP1251
\end{verbatim}

\begin{verbatim}
Warning in grid.Call(C_textBounds, as.graphicsAnnot(x$label), x$x, x$y, :
неизвестна ширина символа 0xe5 в кодировке CP1251
\end{verbatim}

\begin{verbatim}
Warning in grid.Call(C_textBounds, as.graphicsAnnot(x$label), x$x, x$y, :
неизвестна ширина символа 0xe4 в кодировке CP1251
\end{verbatim}

\begin{verbatim}
Warning in grid.Call.graphics(C_text, as.graphicsAnnot(x$label), x$x, x$y, :
неизвестна ширина символа 0xc3 в кодировке CP1251
\end{verbatim}

\begin{verbatim}
Warning in grid.Call.graphics(C_text, as.graphicsAnnot(x$label), x$x, x$y, :
неизвестна ширина символа 0xee в кодировке CP1251
\end{verbatim}

\begin{verbatim}
Warning in grid.Call.graphics(C_text, as.graphicsAnnot(x$label), x$x, x$y, :
неизвестна ширина символа 0xe4 в кодировке CP1251
\end{verbatim}

\begin{verbatim}
Warning in grid.Call.graphics(C_text, as.graphicsAnnot(x$label), x$x, x$y, :
неизвестна ширина символа 0xfb в кодировке CP1251
\end{verbatim}

\begin{verbatim}
Warning in grid.Call.graphics(C_text, as.graphicsAnnot(x$label), x$x, x$y, :
неизвестна ширина символа 0xe2 в кодировке CP1251
\end{verbatim}

\begin{verbatim}
Warning in grid.Call.graphics(C_text, as.graphicsAnnot(x$label), x$x, x$y, :
неизвестна ширина символа 0xef в кодировке CP1251
\end{verbatim}

\begin{verbatim}
Warning in grid.Call.graphics(C_text, as.graphicsAnnot(x$label), x$x, x$y, :
неизвестна ширина символа 0xe5 в кодировке CP1251
\end{verbatim}

\begin{verbatim}
Warning in grid.Call.graphics(C_text, as.graphicsAnnot(x$label), x$x, x$y, :
неизвестна ширина символа 0xf0 в кодировке CP1251
\end{verbatim}

\begin{verbatim}
Warning in grid.Call.graphics(C_text, as.graphicsAnnot(x$label), x$x, x$y, :
неизвестна ширина символа 0xe5 в кодировке CP1251
\end{verbatim}

\begin{verbatim}
Warning in grid.Call.graphics(C_text, as.graphicsAnnot(x$label), x$x, x$y, :
неизвестна ширина символа 0xe4 в кодировке CP1251
\end{verbatim}

\begin{verbatim}
Warning in grid.Call.graphics(C_text, as.graphicsAnnot(x$label), x$x, x$y, :
неизвестна ширина символа 0xc1 в кодировке CP1251
\end{verbatim}

\begin{verbatim}
Warning in grid.Call.graphics(C_text, as.graphicsAnnot(x$label), x$x, x$y, :
неизвестна ширина символа 0xe8 в кодировке CP1251
\end{verbatim}

\begin{verbatim}
Warning in grid.Call.graphics(C_text, as.graphicsAnnot(x$label), x$x, x$y, :
неизвестна ширина символа 0xee в кодировке CP1251
\end{verbatim}

\begin{verbatim}
Warning in grid.Call.graphics(C_text, as.graphicsAnnot(x$label), x$x, x$y, :
неизвестна ширина символа 0xec в кодировке CP1251
\end{verbatim}

\begin{verbatim}
Warning in grid.Call.graphics(C_text, as.graphicsAnnot(x$label), x$x, x$y, :
неизвестна ширина символа 0xe0 в кодировке CP1251
\end{verbatim}

\begin{verbatim}
Warning in grid.Call.graphics(C_text, as.graphicsAnnot(x$label), x$x, x$y, :
неизвестна ширина символа 0xf1 в кодировке CP1251
Warning in grid.Call.graphics(C_text, as.graphicsAnnot(x$label), x$x, x$y, :
неизвестна ширина символа 0xf1 в кодировке CP1251
\end{verbatim}

\begin{verbatim}
Warning in grid.Call.graphics(C_text, as.graphicsAnnot(x$label), x$x, x$y, :
неизвестна ширина символа 0xe0 в кодировке CP1251
\end{verbatim}

\begin{verbatim}
Warning in grid.Call.graphics(C_text, as.graphicsAnnot(x$label), x$x, x$y, :
неизвестна ширина символа 0xf2 в кодировке CP1251
\end{verbatim}

\begin{verbatim}
Warning in grid.Call.graphics(C_text, as.graphicsAnnot(x$label), x$x, x$y, :
неизвестна ширина символа 0xfb в кодировке CP1251
\end{verbatim}

\begin{verbatim}
Warning in grid.Call.graphics(C_text, as.graphicsAnnot(x$label), x$x, x$y, :
неизвестна ширина символа 0xf1 в кодировке CP1251
\end{verbatim}

\begin{verbatim}
Warning in grid.Call.graphics(C_text, as.graphicsAnnot(x$label), x$x, x$y, :
неизвестна ширина символа 0xf2 в кодировке CP1251
\end{verbatim}

\begin{verbatim}
Warning in grid.Call.graphics(C_text, as.graphicsAnnot(x$label), x$x, x$y, :
неизвестна ширина символа 0xcf в кодировке CP1251
\end{verbatim}

\begin{verbatim}
Warning in grid.Call.graphics(C_text, as.graphicsAnnot(x$label), x$x, x$y, :
неизвестна ширина символа 0xf0 в кодировке CP1251
\end{verbatim}

\begin{verbatim}
Warning in grid.Call.graphics(C_text, as.graphicsAnnot(x$label), x$x, x$y, :
неизвестна ширина символа 0xee в кодировке CP1251
\end{verbatim}

\begin{verbatim}
Warning in grid.Call.graphics(C_text, as.graphicsAnnot(x$label), x$x, x$y, :
неизвестна ширина символа 0xe3 в кодировке CP1251
\end{verbatim}

\begin{verbatim}
Warning in grid.Call.graphics(C_text, as.graphicsAnnot(x$label), x$x, x$y, :
неизвестна ширина символа 0xed в кодировке CP1251
\end{verbatim}

\begin{verbatim}
Warning in grid.Call.graphics(C_text, as.graphicsAnnot(x$label), x$x, x$y, :
неизвестна ширина символа 0xee в кодировке CP1251
\end{verbatim}

\begin{verbatim}
Warning in grid.Call.graphics(C_text, as.graphicsAnnot(x$label), x$x, x$y, :
неизвестна ширина символа 0xe7 в кодировке CP1251
\end{verbatim}

\begin{verbatim}
Warning in grid.Call.graphics(C_text, as.graphicsAnnot(x$label), x$x, x$y, :
неизвестна ширина символа 0xe1 в кодировке CP1251
\end{verbatim}

\begin{verbatim}
Warning in grid.Call.graphics(C_text, as.graphicsAnnot(x$label), x$x, x$y, :
неизвестна ширина символа 0xe8 в кодировке CP1251
\end{verbatim}

\begin{verbatim}
Warning in grid.Call.graphics(C_text, as.graphicsAnnot(x$label), x$x, x$y, :
неизвестна ширина символа 0xee в кодировке CP1251
\end{verbatim}

\begin{verbatim}
Warning in grid.Call.graphics(C_text, as.graphicsAnnot(x$label), x$x, x$y, :
неизвестна ширина символа 0xec в кодировке CP1251
\end{verbatim}

\begin{verbatim}
Warning in grid.Call.graphics(C_text, as.graphicsAnnot(x$label), x$x, x$y, :
неизвестна ширина символа 0xe0 в кодировке CP1251
\end{verbatim}

\begin{verbatim}
Warning in grid.Call.graphics(C_text, as.graphicsAnnot(x$label), x$x, x$y, :
неизвестна ширина символа 0xf1 в кодировке CP1251
Warning in grid.Call.graphics(C_text, as.graphicsAnnot(x$label), x$x, x$y, :
неизвестна ширина символа 0xf1 в кодировке CP1251
\end{verbatim}

\begin{verbatim}
Warning in grid.Call.graphics(C_text, as.graphicsAnnot(x$label), x$x, x$y, :
неизвестна ширина символа 0xfb в кодировке CP1251
\end{verbatim}

\begin{verbatim}
Warning in grid.Call.graphics(C_text, as.graphicsAnnot(x$label), x$x, x$y, :
неизвестна ширина символа 0xef в кодировке CP1251
\end{verbatim}

\begin{verbatim}
Warning in grid.Call.graphics(C_text, as.graphicsAnnot(x$label), x$x, x$y, :
неизвестна ширина символа 0xf0 в кодировке CP1251
\end{verbatim}

\begin{verbatim}
Warning in grid.Call.graphics(C_text, as.graphicsAnnot(x$label), x$x, x$y, :
неизвестна ширина символа 0xe8 в кодировке CP1251
\end{verbatim}

\begin{verbatim}
Warning in grid.Call.graphics(C_text, as.graphicsAnnot(x$label), x$x, x$y, :
неизвестна ширина символа 0xf0 в кодировке CP1251
\end{verbatim}

\begin{verbatim}
Warning in grid.Call.graphics(C_text, as.graphicsAnnot(x$label), x$x, x$y, :
неизвестна ширина символа 0xe0 в кодировке CP1251
\end{verbatim}

\begin{verbatim}
Warning in grid.Call.graphics(C_text, as.graphicsAnnot(x$label), x$x, x$y, :
неизвестна ширина символа 0xe7 в кодировке CP1251
\end{verbatim}

\begin{verbatim}
Warning in grid.Call.graphics(C_text, as.graphicsAnnot(x$label), x$x, x$y, :
неизвестна ширина символа 0xeb в кодировке CP1251
\end{verbatim}

\begin{verbatim}
Warning in grid.Call.graphics(C_text, as.graphicsAnnot(x$label), x$x, x$y, :
неизвестна ширина символа 0xe8 в кодировке CP1251
\end{verbatim}

\begin{verbatim}
Warning in grid.Call.graphics(C_text, as.graphicsAnnot(x$label), x$x, x$y, :
неизвестна ширина символа 0xf7 в кодировке CP1251
\end{verbatim}

\begin{verbatim}
Warning in grid.Call.graphics(C_text, as.graphicsAnnot(x$label), x$x, x$y, :
неизвестна ширина символа 0xed в кодировке CP1251
\end{verbatim}

\begin{verbatim}
Warning in grid.Call.graphics(C_text, as.graphicsAnnot(x$label), x$x, x$y, :
неизвестна ширина символа 0xfb в кодировке CP1251
\end{verbatim}

\begin{verbatim}
Warning in grid.Call.graphics(C_text, as.graphicsAnnot(x$label), x$x, x$y, :
неизвестна ширина символа 0xf5 в кодировке CP1251
\end{verbatim}

\begin{verbatim}
Warning in grid.Call.graphics(C_text, as.graphicsAnnot(x$label), x$x, x$y, :
неизвестна ширина символа 0xf1 в кодировке CP1251
\end{verbatim}

\begin{verbatim}
Warning in grid.Call.graphics(C_text, as.graphicsAnnot(x$label), x$x, x$y, :
неизвестна ширина символа 0xf6 в кодировке CP1251
\end{verbatim}

\begin{verbatim}
Warning in grid.Call.graphics(C_text, as.graphicsAnnot(x$label), x$x, x$y, :
неизвестна ширина символа 0xe5 в кодировке CP1251
\end{verbatim}

\begin{verbatim}
Warning in grid.Call.graphics(C_text, as.graphicsAnnot(x$label), x$x, x$y, :
неизвестна ширина символа 0xed в кодировке CP1251
\end{verbatim}

\begin{verbatim}
Warning in grid.Call.graphics(C_text, as.graphicsAnnot(x$label), x$x, x$y, :
неизвестна ширина символа 0xe0 в кодировке CP1251
\end{verbatim}

\begin{verbatim}
Warning in grid.Call.graphics(C_text, as.graphicsAnnot(x$label), x$x, x$y, :
неизвестна ширина символа 0xf0 в кодировке CP1251
\end{verbatim}

\begin{verbatim}
Warning in grid.Call.graphics(C_text, as.graphicsAnnot(x$label), x$x, x$y, :
неизвестна ширина символа 0xe8 в кодировке CP1251
\end{verbatim}

\begin{verbatim}
Warning in grid.Call.graphics(C_text, as.graphicsAnnot(x$label), x$x, x$y, :
неизвестна ширина символа 0xff в кодировке CP1251
\end{verbatim}

\begin{verbatim}
Warning in grid.Call.graphics(C_text, as.graphicsAnnot(x$label), x$x, x$y, :
неизвестна ширина символа 0xf5 в кодировке CP1251
\end{verbatim}

\begin{verbatim}
Warning in grid.Call.graphics(C_text, as.graphicsAnnot(x$label), x$x, x$y, :
неизвестна ширина символа 0xf3 в кодировке CP1251
\end{verbatim}

\begin{verbatim}
Warning in grid.Call.graphics(C_text, as.graphicsAnnot(x$label), x$x, x$y, :
неизвестна ширина символа 0xef в кодировке CP1251
\end{verbatim}

\begin{verbatim}
Warning in grid.Call.graphics(C_text, as.graphicsAnnot(x$label), x$x, x$y, :
неизвестна ширина символа 0xf0 в кодировке CP1251
\end{verbatim}

\begin{verbatim}
Warning in grid.Call.graphics(C_text, as.graphicsAnnot(x$label), x$x, x$y, :
неизвестна ширина символа 0xe0 в кодировке CP1251
\end{verbatim}

\begin{verbatim}
Warning in grid.Call.graphics(C_text, as.graphicsAnnot(x$label), x$x, x$y, :
неизвестна ширина символа 0xe2 в кодировке CP1251
\end{verbatim}

\begin{verbatim}
Warning in grid.Call.graphics(C_text, as.graphicsAnnot(x$label), x$x, x$y, :
неизвестна ширина символа 0xeb в кодировке CP1251
\end{verbatim}

\begin{verbatim}
Warning in grid.Call.graphics(C_text, as.graphicsAnnot(x$label), x$x, x$y, :
неизвестна ширина символа 0xe5 в кодировке CP1251
\end{verbatim}

\begin{verbatim}
Warning in grid.Call.graphics(C_text, as.graphicsAnnot(x$label), x$x, x$y, :
неизвестна ширина символа 0xed в кодировке CP1251
\end{verbatim}

\begin{verbatim}
Warning in grid.Call.graphics(C_text, as.graphicsAnnot(x$label), x$x, x$y, :
неизвестна ширина символа 0xe8 в кодировке CP1251
\end{verbatim}

\begin{verbatim}
Warning in grid.Call.graphics(C_text, as.graphicsAnnot(x$label), x$x, x$y, :
неизвестна ширина символа 0xff в кодировке CP1251
\end{verbatim}

\pandocbounded{\includegraphics[keepaspectratio]{chapter14_files/figure-pdf/unnamed-chunk-1-1.pdf}}

\begin{Shaded}
\begin{Highlighting}[]
\CommentTok{\# {-}{-}{-}{-}{-}{-}{-}{-}{-}{-}{-}{-}{-}{-}{-}{-}{-}{-}{-} 4. ПРАВИЛО ХОККЕЙНОЙ КЛЮШКИ (MSY HOCKEY{-}STICK) {-}{-}{-}{-}{-}{-}{-}{-}{-}{-}{-}{-}{-}{-}{-}{-}{-}{-}{-}{-}}

\FunctionTok{cat}\NormalTok{(}\StringTok{"}\SpecialCharTok{\textbackslash{}n}\StringTok{========== MSY HOCKEY{-}STICK ПРАВИЛО ==========}\SpecialCharTok{\textbackslash{}n}\StringTok{"}\NormalTok{)}
\end{Highlighting}
\end{Shaded}

\begin{verbatim}

========== MSY HOCKEY-STICK ПРАВИЛО ==========
\end{verbatim}

\begin{Shaded}
\begin{Highlighting}[]
\DocumentationTok{\#\# 4.1 Описание правила}
\FunctionTok{cat}\NormalTok{(}\StringTok{"}\SpecialCharTok{\textbackslash{}n}\StringTok{Правило хоккейной клюшки:}\SpecialCharTok{\textbackslash{}n}\StringTok{"}\NormalTok{)}
\end{Highlighting}
\end{Shaded}

\begin{verbatim}

Правило хоккейной клюшки:
\end{verbatim}

\begin{Shaded}
\begin{Highlighting}[]
\FunctionTok{cat}\NormalTok{(}\StringTok{"{-} Если B \textgreater{}= Bmsy, то F = Fmsy}\SpecialCharTok{\textbackslash{}n}\StringTok{"}\NormalTok{)}
\end{Highlighting}
\end{Shaded}

\begin{verbatim}
- Если B >= Bmsy, то F = Fmsy
\end{verbatim}

\begin{Shaded}
\begin{Highlighting}[]
\FunctionTok{cat}\NormalTok{(}\StringTok{"{-} Если B \textless{} Bmsy, то F = Fmsy * (B/Bmsy)}\SpecialCharTok{\textbackslash{}n}\StringTok{"}\NormalTok{)}
\end{Highlighting}
\end{Shaded}

\begin{verbatim}
- Если B < Bmsy, то F = Fmsy * (B/Bmsy)
\end{verbatim}

\begin{Shaded}
\begin{Highlighting}[]
\FunctionTok{cat}\NormalTok{(}\StringTok{"Это правило обеспечивает линейное снижение F при снижении биомассы}\SpecialCharTok{\textbackslash{}n}\StringTok{"}\NormalTok{)}
\end{Highlighting}
\end{Shaded}

\begin{verbatim}
Это правило обеспечивает линейное снижение F при снижении биомассы
\end{verbatim}

\begin{Shaded}
\begin{Highlighting}[]
\DocumentationTok{\#\# 4.2 Функция для hockey{-}stick HCR}
\NormalTok{hockey\_stick\_HCR }\OtherTok{\textless{}{-}} \ControlFlowTok{function}\NormalTok{(B, Bmsy, Fmsy, }\AttributeTok{Blim =} \ConstantTok{NULL}\NormalTok{) \{}
  \CommentTok{\# Blim {-} предельная биомасса (по умолчанию 0.5*Bmsy)}
  \ControlFlowTok{if}\NormalTok{ (}\FunctionTok{is.null}\NormalTok{(Blim)) Blim }\OtherTok{\textless{}{-}} \FloatTok{0.5} \SpecialCharTok{*}\NormalTok{ Bmsy}
  
\NormalTok{  B\_Bmsy }\OtherTok{\textless{}{-}}\NormalTok{ B }\SpecialCharTok{/}\NormalTok{ Bmsy}
  
  \ControlFlowTok{if}\NormalTok{ (B }\SpecialCharTok{\textless{}=}\NormalTok{ Blim) \{}
    \CommentTok{\# Критически низкая биомасса {-} закрытие промысла}
\NormalTok{    F\_advice }\OtherTok{\textless{}{-}} \DecValTok{0}
\NormalTok{  \} }\ControlFlowTok{else} \ControlFlowTok{if}\NormalTok{ (B }\SpecialCharTok{\textgreater{}}\NormalTok{ Blim }\SpecialCharTok{\&}\NormalTok{ B }\SpecialCharTok{\textless{}}\NormalTok{ Bmsy) \{}
    \CommentTok{\# Линейное снижение F}
\NormalTok{    F\_advice }\OtherTok{\textless{}{-}}\NormalTok{ Fmsy }\SpecialCharTok{*}\NormalTok{ (B }\SpecialCharTok{{-}}\NormalTok{ Blim) }\SpecialCharTok{/}\NormalTok{ (Bmsy }\SpecialCharTok{{-}}\NormalTok{ Blim)}
\NormalTok{  \} }\ControlFlowTok{else}\NormalTok{ \{}
    \CommentTok{\# Биомасса выше Bmsy {-} полный промысел}
\NormalTok{    F\_advice }\OtherTok{\textless{}{-}}\NormalTok{ Fmsy}
\NormalTok{  \}}
  
  \FunctionTok{return}\NormalTok{(F\_advice)}
\NormalTok{\}}

\DocumentationTok{\#\# 4.3 Демонстрация правила}
\NormalTok{B\_range }\OtherTok{\textless{}{-}} \FunctionTok{seq}\NormalTok{(}\DecValTok{0}\NormalTok{, }\DecValTok{2} \SpecialCharTok{*} \FunctionTok{get.par}\NormalTok{(}\StringTok{"logBmsy"}\NormalTok{, fit, }\AttributeTok{exp =} \ConstantTok{TRUE}\NormalTok{)[}\DecValTok{1}\NormalTok{], }\AttributeTok{length.out =} \DecValTok{100}\NormalTok{)}
\NormalTok{Bmsy }\OtherTok{\textless{}{-}} \FunctionTok{get.par}\NormalTok{(}\StringTok{"logBmsy"}\NormalTok{, fit, }\AttributeTok{exp =} \ConstantTok{TRUE}\NormalTok{)[}\DecValTok{1}\NormalTok{]}
\NormalTok{Fmsy }\OtherTok{\textless{}{-}} \FunctionTok{get.par}\NormalTok{(}\StringTok{"logFmsy"}\NormalTok{, fit, }\AttributeTok{exp =} \ConstantTok{TRUE}\NormalTok{)[}\DecValTok{1}\NormalTok{]}

\NormalTok{F\_hockey }\OtherTok{\textless{}{-}} \FunctionTok{sapply}\NormalTok{(B\_range, }\ControlFlowTok{function}\NormalTok{(b) }\FunctionTok{hockey\_stick\_HCR}\NormalTok{(b, Bmsy, Fmsy))}

\NormalTok{df\_hockey }\OtherTok{\textless{}{-}} \FunctionTok{data.frame}\NormalTok{(}
  \AttributeTok{Biomass =}\NormalTok{ B\_range,}
  \AttributeTok{F\_advice =}\NormalTok{ F\_hockey,}
  \AttributeTok{B\_Bmsy =}\NormalTok{ B\_range }\SpecialCharTok{/}\NormalTok{ Bmsy}
\NormalTok{)}

\NormalTok{p\_hockey }\OtherTok{\textless{}{-}} \FunctionTok{ggplot}\NormalTok{(df\_hockey, }\FunctionTok{aes}\NormalTok{(}\AttributeTok{x =}\NormalTok{ B\_Bmsy, }\AttributeTok{y =}\NormalTok{ F\_advice)) }\SpecialCharTok{+}
  \FunctionTok{geom\_line}\NormalTok{(}\AttributeTok{linewidth =} \FloatTok{1.5}\NormalTok{, }\AttributeTok{color =} \StringTok{"darkblue"}\NormalTok{) }\SpecialCharTok{+}
  \FunctionTok{geom\_vline}\NormalTok{(}\AttributeTok{xintercept =} \DecValTok{1}\NormalTok{, }\AttributeTok{linetype =} \StringTok{"dashed"}\NormalTok{, }\AttributeTok{color =} \StringTok{"red"}\NormalTok{) }\SpecialCharTok{+}
  \FunctionTok{geom\_vline}\NormalTok{(}\AttributeTok{xintercept =} \FloatTok{0.5}\NormalTok{, }\AttributeTok{linetype =} \StringTok{"dotted"}\NormalTok{, }\AttributeTok{color =} \StringTok{"orange"}\NormalTok{) }\SpecialCharTok{+}
  \FunctionTok{geom\_hline}\NormalTok{(}\AttributeTok{yintercept =}\NormalTok{ Fmsy, }\AttributeTok{linetype =} \StringTok{"dashed"}\NormalTok{, }\AttributeTok{color =} \StringTok{"green"}\NormalTok{) }\SpecialCharTok{+}
  \FunctionTok{labs}\NormalTok{(}\AttributeTok{title =} \StringTok{"MSY Hockey{-}Stick правило управления"}\NormalTok{,}
       \AttributeTok{x =} \StringTok{"B/Bmsy"}\NormalTok{, }\AttributeTok{y =} \StringTok{"F рекомендованное"}\NormalTok{) }\SpecialCharTok{+}
  \FunctionTok{theme\_minimal}\NormalTok{() }\SpecialCharTok{+}
  \FunctionTok{annotate}\NormalTok{(}\StringTok{"text"}\NormalTok{, }\AttributeTok{x =} \DecValTok{1}\NormalTok{, }\AttributeTok{y =} \DecValTok{0}\NormalTok{, }\AttributeTok{label =} \StringTok{"Bmsy"}\NormalTok{, }\AttributeTok{vjust =} \SpecialCharTok{{-}}\DecValTok{1}\NormalTok{) }\SpecialCharTok{+}
  \FunctionTok{annotate}\NormalTok{(}\StringTok{"text"}\NormalTok{, }\AttributeTok{x =} \FloatTok{0.5}\NormalTok{, }\AttributeTok{y =} \DecValTok{0}\NormalTok{, }\AttributeTok{label =} \StringTok{"Blim"}\NormalTok{, }\AttributeTok{vjust =} \SpecialCharTok{{-}}\DecValTok{1}\NormalTok{)}

\FunctionTok{print}\NormalTok{(p\_hockey)}
\end{Highlighting}
\end{Shaded}

\begin{verbatim}
Warning in grid.Call(C_textBounds, as.graphicsAnnot(x$label), x$x, x$y, :
неизвестна ширина символа 0xf0 в кодировке CP1251
\end{verbatim}

\begin{verbatim}
Warning in grid.Call(C_textBounds, as.graphicsAnnot(x$label), x$x, x$y, :
неизвестна ширина символа 0xe5 в кодировке CP1251
\end{verbatim}

\begin{verbatim}
Warning in grid.Call(C_textBounds, as.graphicsAnnot(x$label), x$x, x$y, :
неизвестна ширина символа 0xea в кодировке CP1251
\end{verbatim}

\begin{verbatim}
Warning in grid.Call(C_textBounds, as.graphicsAnnot(x$label), x$x, x$y, :
неизвестна ширина символа 0xee в кодировке CP1251
\end{verbatim}

\begin{verbatim}
Warning in grid.Call(C_textBounds, as.graphicsAnnot(x$label), x$x, x$y, :
неизвестна ширина символа 0xec в кодировке CP1251
\end{verbatim}

\begin{verbatim}
Warning in grid.Call(C_textBounds, as.graphicsAnnot(x$label), x$x, x$y, :
неизвестна ширина символа 0xe5 в кодировке CP1251
\end{verbatim}

\begin{verbatim}
Warning in grid.Call(C_textBounds, as.graphicsAnnot(x$label), x$x, x$y, :
неизвестна ширина символа 0xed в кодировке CP1251
\end{verbatim}

\begin{verbatim}
Warning in grid.Call(C_textBounds, as.graphicsAnnot(x$label), x$x, x$y, :
неизвестна ширина символа 0xe4 в кодировке CP1251
\end{verbatim}

\begin{verbatim}
Warning in grid.Call(C_textBounds, as.graphicsAnnot(x$label), x$x, x$y, :
неизвестна ширина символа 0xee в кодировке CP1251
\end{verbatim}

\begin{verbatim}
Warning in grid.Call(C_textBounds, as.graphicsAnnot(x$label), x$x, x$y, :
неизвестна ширина символа 0xe2 в кодировке CP1251
\end{verbatim}

\begin{verbatim}
Warning in grid.Call(C_textBounds, as.graphicsAnnot(x$label), x$x, x$y, :
неизвестна ширина символа 0xe0 в кодировке CP1251
\end{verbatim}

\begin{verbatim}
Warning in grid.Call(C_textBounds, as.graphicsAnnot(x$label), x$x, x$y, :
неизвестна ширина символа 0xed в кодировке CP1251
Warning in grid.Call(C_textBounds, as.graphicsAnnot(x$label), x$x, x$y, :
неизвестна ширина символа 0xed в кодировке CP1251
\end{verbatim}

\begin{verbatim}
Warning in grid.Call(C_textBounds, as.graphicsAnnot(x$label), x$x, x$y, :
неизвестна ширина символа 0xee в кодировке CP1251
\end{verbatim}

\begin{verbatim}
Warning in grid.Call(C_textBounds, as.graphicsAnnot(x$label), x$x, x$y, :
неизвестна ширина символа 0xe5 в кодировке CP1251
\end{verbatim}

\begin{verbatim}
Warning in grid.Call(C_textBounds, as.graphicsAnnot(x$label), x$x, x$y, :
неизвестна ширина символа 0xef в кодировке CP1251
\end{verbatim}

\begin{verbatim}
Warning in grid.Call(C_textBounds, as.graphicsAnnot(x$label), x$x, x$y, :
неизвестна ширина символа 0xf0 в кодировке CP1251
\end{verbatim}

\begin{verbatim}
Warning in grid.Call(C_textBounds, as.graphicsAnnot(x$label), x$x, x$y, :
неизвестна ширина символа 0xe0 в кодировке CP1251
\end{verbatim}

\begin{verbatim}
Warning in grid.Call(C_textBounds, as.graphicsAnnot(x$label), x$x, x$y, :
неизвестна ширина символа 0xe2 в кодировке CP1251
\end{verbatim}

\begin{verbatim}
Warning in grid.Call(C_textBounds, as.graphicsAnnot(x$label), x$x, x$y, :
неизвестна ширина символа 0xe8 в кодировке CP1251
\end{verbatim}

\begin{verbatim}
Warning in grid.Call(C_textBounds, as.graphicsAnnot(x$label), x$x, x$y, :
неизвестна ширина символа 0xeb в кодировке CP1251
\end{verbatim}

\begin{verbatim}
Warning in grid.Call(C_textBounds, as.graphicsAnnot(x$label), x$x, x$y, :
неизвестна ширина символа 0xee в кодировке CP1251
\end{verbatim}

\begin{verbatim}
Warning in grid.Call(C_textBounds, as.graphicsAnnot(x$label), x$x, x$y, :
неизвестна ширина символа 0xf3 в кодировке CP1251
\end{verbatim}

\begin{verbatim}
Warning in grid.Call(C_textBounds, as.graphicsAnnot(x$label), x$x, x$y, :
неизвестна ширина символа 0xef в кодировке CP1251
\end{verbatim}

\begin{verbatim}
Warning in grid.Call(C_textBounds, as.graphicsAnnot(x$label), x$x, x$y, :
неизвестна ширина символа 0xf0 в кодировке CP1251
\end{verbatim}

\begin{verbatim}
Warning in grid.Call(C_textBounds, as.graphicsAnnot(x$label), x$x, x$y, :
неизвестна ширина символа 0xe0 в кодировке CP1251
\end{verbatim}

\begin{verbatim}
Warning in grid.Call(C_textBounds, as.graphicsAnnot(x$label), x$x, x$y, :
неизвестна ширина символа 0xe2 в кодировке CP1251
\end{verbatim}

\begin{verbatim}
Warning in grid.Call(C_textBounds, as.graphicsAnnot(x$label), x$x, x$y, :
неизвестна ширина символа 0xeb в кодировке CP1251
\end{verbatim}

\begin{verbatim}
Warning in grid.Call(C_textBounds, as.graphicsAnnot(x$label), x$x, x$y, :
неизвестна ширина символа 0xe5 в кодировке CP1251
\end{verbatim}

\begin{verbatim}
Warning in grid.Call(C_textBounds, as.graphicsAnnot(x$label), x$x, x$y, :
неизвестна ширина символа 0xed в кодировке CP1251
\end{verbatim}

\begin{verbatim}
Warning in grid.Call(C_textBounds, as.graphicsAnnot(x$label), x$x, x$y, :
неизвестна ширина символа 0xe8 в кодировке CP1251
\end{verbatim}

\begin{verbatim}
Warning in grid.Call(C_textBounds, as.graphicsAnnot(x$label), x$x, x$y, :
неизвестна ширина символа 0xff в кодировке CP1251
\end{verbatim}

\begin{verbatim}
Warning in grid.Call.graphics(C_text, as.graphicsAnnot(x$label), x$x, x$y, :
неизвестна ширина символа 0xf0 в кодировке CP1251
\end{verbatim}

\begin{verbatim}
Warning in grid.Call.graphics(C_text, as.graphicsAnnot(x$label), x$x, x$y, :
неизвестна ширина символа 0xe5 в кодировке CP1251
\end{verbatim}

\begin{verbatim}
Warning in grid.Call.graphics(C_text, as.graphicsAnnot(x$label), x$x, x$y, :
неизвестна ширина символа 0xea в кодировке CP1251
\end{verbatim}

\begin{verbatim}
Warning in grid.Call.graphics(C_text, as.graphicsAnnot(x$label), x$x, x$y, :
неизвестна ширина символа 0xee в кодировке CP1251
\end{verbatim}

\begin{verbatim}
Warning in grid.Call.graphics(C_text, as.graphicsAnnot(x$label), x$x, x$y, :
неизвестна ширина символа 0xec в кодировке CP1251
\end{verbatim}

\begin{verbatim}
Warning in grid.Call.graphics(C_text, as.graphicsAnnot(x$label), x$x, x$y, :
неизвестна ширина символа 0xe5 в кодировке CP1251
\end{verbatim}

\begin{verbatim}
Warning in grid.Call.graphics(C_text, as.graphicsAnnot(x$label), x$x, x$y, :
неизвестна ширина символа 0xed в кодировке CP1251
\end{verbatim}

\begin{verbatim}
Warning in grid.Call.graphics(C_text, as.graphicsAnnot(x$label), x$x, x$y, :
неизвестна ширина символа 0xe4 в кодировке CP1251
\end{verbatim}

\begin{verbatim}
Warning in grid.Call.graphics(C_text, as.graphicsAnnot(x$label), x$x, x$y, :
неизвестна ширина символа 0xee в кодировке CP1251
\end{verbatim}

\begin{verbatim}
Warning in grid.Call.graphics(C_text, as.graphicsAnnot(x$label), x$x, x$y, :
неизвестна ширина символа 0xe2 в кодировке CP1251
\end{verbatim}

\begin{verbatim}
Warning in grid.Call.graphics(C_text, as.graphicsAnnot(x$label), x$x, x$y, :
неизвестна ширина символа 0xe0 в кодировке CP1251
\end{verbatim}

\begin{verbatim}
Warning in grid.Call.graphics(C_text, as.graphicsAnnot(x$label), x$x, x$y, :
неизвестна ширина символа 0xed в кодировке CP1251
Warning in grid.Call.graphics(C_text, as.graphicsAnnot(x$label), x$x, x$y, :
неизвестна ширина символа 0xed в кодировке CP1251
\end{verbatim}

\begin{verbatim}
Warning in grid.Call.graphics(C_text, as.graphicsAnnot(x$label), x$x, x$y, :
неизвестна ширина символа 0xee в кодировке CP1251
\end{verbatim}

\begin{verbatim}
Warning in grid.Call.graphics(C_text, as.graphicsAnnot(x$label), x$x, x$y, :
неизвестна ширина символа 0xe5 в кодировке CP1251
\end{verbatim}

\begin{verbatim}
Warning in grid.Call.graphics(C_text, as.graphicsAnnot(x$label), x$x, x$y, :
неизвестна ширина символа 0xef в кодировке CP1251
\end{verbatim}

\begin{verbatim}
Warning in grid.Call.graphics(C_text, as.graphicsAnnot(x$label), x$x, x$y, :
неизвестна ширина символа 0xf0 в кодировке CP1251
\end{verbatim}

\begin{verbatim}
Warning in grid.Call.graphics(C_text, as.graphicsAnnot(x$label), x$x, x$y, :
неизвестна ширина символа 0xe0 в кодировке CP1251
\end{verbatim}

\begin{verbatim}
Warning in grid.Call.graphics(C_text, as.graphicsAnnot(x$label), x$x, x$y, :
неизвестна ширина символа 0xe2 в кодировке CP1251
\end{verbatim}

\begin{verbatim}
Warning in grid.Call.graphics(C_text, as.graphicsAnnot(x$label), x$x, x$y, :
неизвестна ширина символа 0xe8 в кодировке CP1251
\end{verbatim}

\begin{verbatim}
Warning in grid.Call.graphics(C_text, as.graphicsAnnot(x$label), x$x, x$y, :
неизвестна ширина символа 0xeb в кодировке CP1251
\end{verbatim}

\begin{verbatim}
Warning in grid.Call.graphics(C_text, as.graphicsAnnot(x$label), x$x, x$y, :
неизвестна ширина символа 0xee в кодировке CP1251
\end{verbatim}

\begin{verbatim}
Warning in grid.Call.graphics(C_text, as.graphicsAnnot(x$label), x$x, x$y, :
неизвестна ширина символа 0xf3 в кодировке CP1251
\end{verbatim}

\begin{verbatim}
Warning in grid.Call.graphics(C_text, as.graphicsAnnot(x$label), x$x, x$y, :
неизвестна ширина символа 0xef в кодировке CP1251
\end{verbatim}

\begin{verbatim}
Warning in grid.Call.graphics(C_text, as.graphicsAnnot(x$label), x$x, x$y, :
неизвестна ширина символа 0xf0 в кодировке CP1251
\end{verbatim}

\begin{verbatim}
Warning in grid.Call.graphics(C_text, as.graphicsAnnot(x$label), x$x, x$y, :
неизвестна ширина символа 0xe0 в кодировке CP1251
\end{verbatim}

\begin{verbatim}
Warning in grid.Call.graphics(C_text, as.graphicsAnnot(x$label), x$x, x$y, :
неизвестна ширина символа 0xe2 в кодировке CP1251
\end{verbatim}

\begin{verbatim}
Warning in grid.Call.graphics(C_text, as.graphicsAnnot(x$label), x$x, x$y, :
неизвестна ширина символа 0xeb в кодировке CP1251
\end{verbatim}

\begin{verbatim}
Warning in grid.Call.graphics(C_text, as.graphicsAnnot(x$label), x$x, x$y, :
неизвестна ширина символа 0xe5 в кодировке CP1251
\end{verbatim}

\begin{verbatim}
Warning in grid.Call.graphics(C_text, as.graphicsAnnot(x$label), x$x, x$y, :
неизвестна ширина символа 0xed в кодировке CP1251
\end{verbatim}

\begin{verbatim}
Warning in grid.Call.graphics(C_text, as.graphicsAnnot(x$label), x$x, x$y, :
неизвестна ширина символа 0xe8 в кодировке CP1251
\end{verbatim}

\begin{verbatim}
Warning in grid.Call.graphics(C_text, as.graphicsAnnot(x$label), x$x, x$y, :
неизвестна ширина символа 0xff в кодировке CP1251
\end{verbatim}

\pandocbounded{\includegraphics[keepaspectratio]{chapter14_files/figure-pdf/unnamed-chunk-1-2.pdf}}

\begin{Shaded}
\begin{Highlighting}[]
\CommentTok{\# {-}{-}{-}{-}{-}{-}{-}{-}{-}{-}{-}{-}{-}{-}{-}{-}{-}{-}{-} 5. ПРАВИЛО ICES {-}{-}{-}{-}{-}{-}{-}{-}{-}{-}{-}{-}{-}{-}{-}{-}{-}{-}{-}{-}}

\FunctionTok{cat}\NormalTok{(}\StringTok{"}\SpecialCharTok{\textbackslash{}n}\StringTok{========== ПРАВИЛО ICES ==========}\SpecialCharTok{\textbackslash{}n}\StringTok{"}\NormalTok{)}
\end{Highlighting}
\end{Shaded}

\begin{verbatim}

========== ПРАВИЛО ICES ==========
\end{verbatim}

\begin{Shaded}
\begin{Highlighting}[]
\DocumentationTok{\#\# 5.1 Описание правила ICES}
\FunctionTok{cat}\NormalTok{(}\StringTok{"}\SpecialCharTok{\textbackslash{}n}\StringTok{Правило ICES (упрощенное):}\SpecialCharTok{\textbackslash{}n}\StringTok{"}\NormalTok{)}
\end{Highlighting}
\end{Shaded}

\begin{verbatim}

Правило ICES (упрощенное):
\end{verbatim}

\begin{Shaded}
\begin{Highlighting}[]
\FunctionTok{cat}\NormalTok{(}\StringTok{"{-} Основано на предосторожном подходе}\SpecialCharTok{\textbackslash{}n}\StringTok{"}\NormalTok{)}
\end{Highlighting}
\end{Shaded}

\begin{verbatim}
- Основано на предосторожном подходе
\end{verbatim}

\begin{Shaded}
\begin{Highlighting}[]
\FunctionTok{cat}\NormalTok{(}\StringTok{"{-} Использует Bpa (предосторожная биомасса) и Fpa}\SpecialCharTok{\textbackslash{}n}\StringTok{"}\NormalTok{)}
\end{Highlighting}
\end{Shaded}

\begin{verbatim}
- Использует Bpa (предосторожная биомасса) и Fpa
\end{verbatim}

\begin{Shaded}
\begin{Highlighting}[]
\FunctionTok{cat}\NormalTok{(}\StringTok{"{-} Включает ограничения на межгодовые изменения TAC}\SpecialCharTok{\textbackslash{}n}\StringTok{"}\NormalTok{)}
\end{Highlighting}
\end{Shaded}

\begin{verbatim}
- Включает ограничения на межгодовые изменения TAC
\end{verbatim}

\begin{Shaded}
\begin{Highlighting}[]
\DocumentationTok{\#\# 5.2 Функция правила ICES}
\NormalTok{ICES\_advice\_rule }\OtherTok{\textless{}{-}} \ControlFlowTok{function}\NormalTok{(B, Bmsy, Fmsy, }\AttributeTok{previous\_TAC =} \ConstantTok{NULL}\NormalTok{,}
                             \AttributeTok{Bpa\_multiplier =} \FloatTok{1.4}\NormalTok{, }\AttributeTok{Fpa\_multiplier =} \FloatTok{0.85}\NormalTok{,}
                             \AttributeTok{max\_change =} \FloatTok{0.2}\NormalTok{) \{}
  
  \CommentTok{\# Расчет предосторожных референсных точек}
\NormalTok{  Bpa }\OtherTok{\textless{}{-}}\NormalTok{ Bmsy }\SpecialCharTok{/}\NormalTok{ Bpa\_multiplier  }\CommentTok{\# Предосторожная биомасса}
\NormalTok{  Blim }\OtherTok{\textless{}{-}}\NormalTok{ Bpa }\SpecialCharTok{/} \FloatTok{1.4}              \CommentTok{\# Предельная биомасса}
\NormalTok{  Fpa }\OtherTok{\textless{}{-}}\NormalTok{ Fmsy }\SpecialCharTok{*}\NormalTok{ Fpa\_multiplier  }\CommentTok{\# Предосторожная F}
  
  \CommentTok{\# Определение F в зависимости от состояния запаса}
  \ControlFlowTok{if}\NormalTok{ (B }\SpecialCharTok{\textless{}}\NormalTok{ Blim) \{}
\NormalTok{    F\_advice }\OtherTok{\textless{}{-}} \DecValTok{0}  \CommentTok{\# Закрытие промысла}
\NormalTok{    status }\OtherTok{\textless{}{-}} \StringTok{"Критическое"}
\NormalTok{  \} }\ControlFlowTok{else} \ControlFlowTok{if}\NormalTok{ (B }\SpecialCharTok{\textgreater{}=}\NormalTok{ Blim }\SpecialCharTok{\&}\NormalTok{ B }\SpecialCharTok{\textless{}}\NormalTok{ Bpa) \{}
\NormalTok{    F\_advice }\OtherTok{\textless{}{-}}\NormalTok{ Fpa }\SpecialCharTok{*}\NormalTok{ (B }\SpecialCharTok{{-}}\NormalTok{ Blim) }\SpecialCharTok{/}\NormalTok{ (Bpa }\SpecialCharTok{{-}}\NormalTok{ Blim)}
\NormalTok{    status }\OtherTok{\textless{}{-}} \StringTok{"Восстановление"}
\NormalTok{  \} }\ControlFlowTok{else}\NormalTok{ \{}
\NormalTok{    F\_advice }\OtherTok{\textless{}{-}}\NormalTok{ Fpa}
\NormalTok{    status }\OtherTok{\textless{}{-}} \StringTok{"Устойчивое"}
\NormalTok{  \}}
  
  \CommentTok{\# Расчет TAC}
\NormalTok{  TAC }\OtherTok{\textless{}{-}}\NormalTok{ F\_advice }\SpecialCharTok{*}\NormalTok{ B}
  
  \CommentTok{\# Ограничение межгодовых изменений (если есть предыдущий TAC)}
  \ControlFlowTok{if}\NormalTok{ (}\SpecialCharTok{!}\FunctionTok{is.null}\NormalTok{(previous\_TAC) }\SpecialCharTok{\&\&}\NormalTok{ previous\_TAC }\SpecialCharTok{\textgreater{}} \DecValTok{0}\NormalTok{) \{}
\NormalTok{    max\_increase }\OtherTok{\textless{}{-}}\NormalTok{ previous\_TAC }\SpecialCharTok{*}\NormalTok{ (}\DecValTok{1} \SpecialCharTok{+}\NormalTok{ max\_change)}
\NormalTok{    max\_decrease }\OtherTok{\textless{}{-}}\NormalTok{ previous\_TAC }\SpecialCharTok{*}\NormalTok{ (}\DecValTok{1} \SpecialCharTok{{-}}\NormalTok{ max\_change)}
\NormalTok{    TAC }\OtherTok{\textless{}{-}} \FunctionTok{min}\NormalTok{(}\FunctionTok{max}\NormalTok{(TAC, max\_decrease), max\_increase)}
\NormalTok{  \}}
  
  \FunctionTok{return}\NormalTok{(}\FunctionTok{list}\NormalTok{(}
    \AttributeTok{F\_advice =}\NormalTok{ F\_advice,}
    \AttributeTok{TAC =}\NormalTok{ TAC,}
    \AttributeTok{status =}\NormalTok{ status,}
    \AttributeTok{Bpa =}\NormalTok{ Bpa,}
    \AttributeTok{Blim =}\NormalTok{ Blim,}
    \AttributeTok{Fpa =}\NormalTok{ Fpa}
\NormalTok{  ))}
\NormalTok{\}}

\DocumentationTok{\#\# 5.3 Применение правила ICES}
\NormalTok{B\_current }\OtherTok{\textless{}{-}} \FunctionTok{get.par}\NormalTok{(}\StringTok{"logB"}\NormalTok{, fit, }\AttributeTok{exp =} \ConstantTok{TRUE}\NormalTok{)[}\DecValTok{1}\NormalTok{]}
\NormalTok{ices\_result }\OtherTok{\textless{}{-}} \FunctionTok{ICES\_advice\_rule}\NormalTok{(B\_current, Bmsy, Fmsy, }\AttributeTok{previous\_TAC =} \DecValTok{12}\NormalTok{)}

\FunctionTok{cat}\NormalTok{(}\StringTok{"}\SpecialCharTok{\textbackslash{}n}\StringTok{{-}{-}{-} Результаты применения правила ICES {-}{-}{-}}\SpecialCharTok{\textbackslash{}n}\StringTok{"}\NormalTok{)}
\end{Highlighting}
\end{Shaded}

\begin{verbatim}

--- Результаты применения правила ICES ---
\end{verbatim}

\begin{Shaded}
\begin{Highlighting}[]
\FunctionTok{cat}\NormalTok{(}\StringTok{"Текущая биомасса:"}\NormalTok{, }\FunctionTok{round}\NormalTok{(B\_current, }\DecValTok{1}\NormalTok{), }\StringTok{"тыс. т}\SpecialCharTok{\textbackslash{}n}\StringTok{"}\NormalTok{)}
\end{Highlighting}
\end{Shaded}

\begin{verbatim}
Текущая биомасса: 103 тыс. т
\end{verbatim}

\begin{Shaded}
\begin{Highlighting}[]
\FunctionTok{cat}\NormalTok{(}\StringTok{"Статус запаса:"}\NormalTok{, ices\_result}\SpecialCharTok{$}\NormalTok{status, }\StringTok{"}\SpecialCharTok{\textbackslash{}n}\StringTok{"}\NormalTok{)}
\end{Highlighting}
\end{Shaded}

\begin{verbatim}
Статус запаса: Устойчивое 
\end{verbatim}

\begin{Shaded}
\begin{Highlighting}[]
\FunctionTok{cat}\NormalTok{(}\StringTok{"Bpa:"}\NormalTok{, }\FunctionTok{round}\NormalTok{(ices\_result}\SpecialCharTok{$}\NormalTok{Bpa, }\DecValTok{1}\NormalTok{), }\StringTok{"тыс. т}\SpecialCharTok{\textbackslash{}n}\StringTok{"}\NormalTok{)}
\end{Highlighting}
\end{Shaded}

\begin{verbatim}
Bpa: 54.9 тыс. т
\end{verbatim}

\begin{Shaded}
\begin{Highlighting}[]
\FunctionTok{cat}\NormalTok{(}\StringTok{"Blim:"}\NormalTok{, }\FunctionTok{round}\NormalTok{(ices\_result}\SpecialCharTok{$}\NormalTok{Blim, }\DecValTok{1}\NormalTok{), }\StringTok{"тыс. т}\SpecialCharTok{\textbackslash{}n}\StringTok{"}\NormalTok{)}
\end{Highlighting}
\end{Shaded}

\begin{verbatim}
Blim: 39.2 тыс. т
\end{verbatim}

\begin{Shaded}
\begin{Highlighting}[]
\FunctionTok{cat}\NormalTok{(}\StringTok{"Fpa:"}\NormalTok{, }\FunctionTok{round}\NormalTok{(ices\_result}\SpecialCharTok{$}\NormalTok{Fpa, }\DecValTok{3}\NormalTok{), }\StringTok{"}\SpecialCharTok{\textbackslash{}n}\StringTok{"}\NormalTok{)}
\end{Highlighting}
\end{Shaded}

\begin{verbatim}
Fpa: 0.123 
\end{verbatim}

\begin{Shaded}
\begin{Highlighting}[]
\FunctionTok{cat}\NormalTok{(}\StringTok{"Рекомендованная F:"}\NormalTok{, }\FunctionTok{round}\NormalTok{(ices\_result}\SpecialCharTok{$}\NormalTok{F\_advice, }\DecValTok{3}\NormalTok{), }\StringTok{"}\SpecialCharTok{\textbackslash{}n}\StringTok{"}\NormalTok{)}
\end{Highlighting}
\end{Shaded}

\begin{verbatim}
Рекомендованная F: 0.123 
\end{verbatim}

\begin{Shaded}
\begin{Highlighting}[]
\FunctionTok{cat}\NormalTok{(}\StringTok{"Рекомендованный TAC:"}\NormalTok{, }\FunctionTok{round}\NormalTok{(ices\_result}\SpecialCharTok{$}\NormalTok{TAC, }\DecValTok{1}\NormalTok{), }\StringTok{"тыс. т}\SpecialCharTok{\textbackslash{}n}\StringTok{"}\NormalTok{)}
\end{Highlighting}
\end{Shaded}

\begin{verbatim}
Рекомендованный TAC: 12.7 тыс. т
\end{verbatim}

\begin{Shaded}
\begin{Highlighting}[]
\CommentTok{\# {-}{-}{-}{-}{-}{-}{-}{-}{-}{-}{-}{-}{-}{-}{-}{-}{-}{-}{-} 6. ПРАВИЛО 40{-}10 {-}{-}{-}{-}{-}{-}{-}{-}{-}{-}{-}{-}{-}{-}{-}{-}{-}{-}{-}{-}}

\FunctionTok{cat}\NormalTok{(}\StringTok{"}\SpecialCharTok{\textbackslash{}n}\StringTok{========== ПРАВИЛО 40{-}10 ==========}\SpecialCharTok{\textbackslash{}n}\StringTok{"}\NormalTok{)}
\end{Highlighting}
\end{Shaded}

\begin{verbatim}

========== ПРАВИЛО 40-10 ==========
\end{verbatim}

\begin{Shaded}
\begin{Highlighting}[]
\DocumentationTok{\#\# 6.1 Описание правила}
\FunctionTok{cat}\NormalTok{(}\StringTok{"}\SpecialCharTok{\textbackslash{}n}\StringTok{Правило 40{-}10 (используется в США):}\SpecialCharTok{\textbackslash{}n}\StringTok{"}\NormalTok{)}
\end{Highlighting}
\end{Shaded}

\begin{verbatim}

Правило 40-10 (используется в США):
\end{verbatim}

\begin{Shaded}
\begin{Highlighting}[]
\FunctionTok{cat}\NormalTok{(}\StringTok{"{-} Если B/B0 \textgreater{}= 40\%, то F = Ftarget}\SpecialCharTok{\textbackslash{}n}\StringTok{"}\NormalTok{)}
\end{Highlighting}
\end{Shaded}

\begin{verbatim}
- Если B/B0 >= 40%, то F = Ftarget
\end{verbatim}

\begin{Shaded}
\begin{Highlighting}[]
\FunctionTok{cat}\NormalTok{(}\StringTok{"{-} Если 10\% \textless{} B/B0 \textless{} 40\%, то F линейно снижается}\SpecialCharTok{\textbackslash{}n}\StringTok{"}\NormalTok{)}
\end{Highlighting}
\end{Shaded}

\begin{verbatim}
- Если 10% < B/B0 < 40%, то F линейно снижается
\end{verbatim}

\begin{Shaded}
\begin{Highlighting}[]
\FunctionTok{cat}\NormalTok{(}\StringTok{"{-} Если B/B0 \textless{}= 10\%, то F = 0 (закрытие промысла)}\SpecialCharTok{\textbackslash{}n}\StringTok{"}\NormalTok{)}
\end{Highlighting}
\end{Shaded}

\begin{verbatim}
- Если B/B0 <= 10%, то F = 0 (закрытие промысла)
\end{verbatim}

\begin{Shaded}
\begin{Highlighting}[]
\DocumentationTok{\#\# 6.2 Функция правила 40{-}10}
\NormalTok{rule\_40\_10 }\OtherTok{\textless{}{-}} \ControlFlowTok{function}\NormalTok{(B, B0, Ftarget) \{}
\NormalTok{  depletion }\OtherTok{\textless{}{-}}\NormalTok{ B }\SpecialCharTok{/}\NormalTok{ B0}
  
  \ControlFlowTok{if}\NormalTok{ (depletion }\SpecialCharTok{\textless{}=} \FloatTok{0.10}\NormalTok{) \{}
\NormalTok{    F\_advice }\OtherTok{\textless{}{-}} \DecValTok{0}
\NormalTok{    status }\OtherTok{\textless{}{-}} \StringTok{"Закрыт"}
\NormalTok{  \} }\ControlFlowTok{else} \ControlFlowTok{if}\NormalTok{ (depletion }\SpecialCharTok{\textgreater{}} \FloatTok{0.10} \SpecialCharTok{\&}\NormalTok{ depletion }\SpecialCharTok{\textless{}} \FloatTok{0.40}\NormalTok{) \{}
\NormalTok{    F\_advice }\OtherTok{\textless{}{-}}\NormalTok{ Ftarget }\SpecialCharTok{*}\NormalTok{ (depletion }\SpecialCharTok{{-}} \FloatTok{0.10}\NormalTok{) }\SpecialCharTok{/}\NormalTok{ (}\FloatTok{0.40} \SpecialCharTok{{-}} \FloatTok{0.10}\NormalTok{)}
\NormalTok{    status }\OtherTok{\textless{}{-}} \StringTok{"Ограничен"}
\NormalTok{  \} }\ControlFlowTok{else}\NormalTok{ \{}
\NormalTok{    F\_advice }\OtherTok{\textless{}{-}}\NormalTok{ Ftarget}
\NormalTok{    status }\OtherTok{\textless{}{-}} \StringTok{"Полный промысел"}
\NormalTok{  \}}
  
  \FunctionTok{return}\NormalTok{(}\FunctionTok{list}\NormalTok{(}\AttributeTok{F\_advice =}\NormalTok{ F\_advice, }\AttributeTok{status =}\NormalTok{ status, }\AttributeTok{depletion =}\NormalTok{ depletion))}
\NormalTok{\}}

\DocumentationTok{\#\# 6.3 Применение правила 40{-}10}
\NormalTok{K }\OtherTok{\textless{}{-}} \FunctionTok{get.par}\NormalTok{(}\StringTok{"logK"}\NormalTok{, fit, }\AttributeTok{exp =} \ConstantTok{TRUE}\NormalTok{)[}\DecValTok{1}\NormalTok{]  }\CommentTok{\# K примерно равно B0}
\NormalTok{result\_40\_10 }\OtherTok{\textless{}{-}} \FunctionTok{rule\_40\_10}\NormalTok{(B\_current, K, Fmsy }\SpecialCharTok{*} \FloatTok{0.75}\NormalTok{)}

\FunctionTok{cat}\NormalTok{(}\StringTok{"}\SpecialCharTok{\textbackslash{}n}\StringTok{{-}{-}{-} Результаты правила 40{-}10 {-}{-}{-}}\SpecialCharTok{\textbackslash{}n}\StringTok{"}\NormalTok{)}
\end{Highlighting}
\end{Shaded}

\begin{verbatim}

--- Результаты правила 40-10 ---
\end{verbatim}

\begin{Shaded}
\begin{Highlighting}[]
\FunctionTok{cat}\NormalTok{(}\StringTok{"Истощение запаса (B/B0):"}\NormalTok{, }\FunctionTok{round}\NormalTok{(result\_40\_10}\SpecialCharTok{$}\NormalTok{depletion }\SpecialCharTok{*} \DecValTok{100}\NormalTok{, }\DecValTok{1}\NormalTok{), }\StringTok{"\%}\SpecialCharTok{\textbackslash{}n}\StringTok{"}\NormalTok{)}
\end{Highlighting}
\end{Shaded}

\begin{verbatim}
Истощение запаса (B/B0): 66.9 %
\end{verbatim}

\begin{Shaded}
\begin{Highlighting}[]
\FunctionTok{cat}\NormalTok{(}\StringTok{"Статус:"}\NormalTok{, result\_40\_10}\SpecialCharTok{$}\NormalTok{status, }\StringTok{"}\SpecialCharTok{\textbackslash{}n}\StringTok{"}\NormalTok{)}
\end{Highlighting}
\end{Shaded}

\begin{verbatim}
Статус: Полный промысел 
\end{verbatim}

\begin{Shaded}
\begin{Highlighting}[]
\FunctionTok{cat}\NormalTok{(}\StringTok{"Рекомендованная F:"}\NormalTok{, }\FunctionTok{round}\NormalTok{(result\_40\_10}\SpecialCharTok{$}\NormalTok{F\_advice, }\DecValTok{3}\NormalTok{), }\StringTok{"}\SpecialCharTok{\textbackslash{}n}\StringTok{"}\NormalTok{)}
\end{Highlighting}
\end{Shaded}

\begin{verbatim}
Рекомендованная F: 0.109 
\end{verbatim}

\begin{Shaded}
\begin{Highlighting}[]
\FunctionTok{cat}\NormalTok{(}\StringTok{"Рекомендованный вылов:"}\NormalTok{, }\FunctionTok{round}\NormalTok{(result\_40\_10}\SpecialCharTok{$}\NormalTok{F\_advice }\SpecialCharTok{*}\NormalTok{ B\_current, }\DecValTok{1}\NormalTok{), }\StringTok{"тыс. т}\SpecialCharTok{\textbackslash{}n}\StringTok{"}\NormalTok{)}
\end{Highlighting}
\end{Shaded}

\begin{verbatim}
Рекомендованный вылов: 11.2 тыс. т
\end{verbatim}

\begin{Shaded}
\begin{Highlighting}[]
\CommentTok{\# {-}{-}{-}{-}{-}{-}{-}{-}{-}{-}{-}{-}{-}{-}{-}{-}{-}{-}{-} 7. СРАВНЕНИЕ ВСЕХ ПРАВИЛ {-}{-}{-}{-}{-}{-}{-}{-}{-}{-}{-}{-}{-}{-}{-}{-}{-}{-}{-}{-}}

\FunctionTok{cat}\NormalTok{(}\StringTok{"}\SpecialCharTok{\textbackslash{}n}\StringTok{========== СРАВНЕНИЕ ПРАВИЛ УПРАВЛЕНИЯ ==========}\SpecialCharTok{\textbackslash{}n}\StringTok{"}\NormalTok{)}
\end{Highlighting}
\end{Shaded}

\begin{verbatim}

========== СРАВНЕНИЕ ПРАВИЛ УПРАВЛЕНИЯ ==========
\end{verbatim}

\begin{Shaded}
\begin{Highlighting}[]
\DocumentationTok{\#\# 7.1 Функция для сравнения правил при разных уровнях биомассы}
\NormalTok{compare\_HCRs }\OtherTok{\textless{}{-}} \ControlFlowTok{function}\NormalTok{(B\_range, Bmsy, Fmsy, K) \{}
  
\NormalTok{  results }\OtherTok{\textless{}{-}} \FunctionTok{data.frame}\NormalTok{(}
    \AttributeTok{Biomass =} \FunctionTok{rep}\NormalTok{(B\_range, }\DecValTok{4}\NormalTok{),}
    \AttributeTok{B\_Bmsy =} \FunctionTok{rep}\NormalTok{(B\_range }\SpecialCharTok{/}\NormalTok{ Bmsy, }\DecValTok{4}\NormalTok{),}
    \AttributeTok{HCR =} \FunctionTok{rep}\NormalTok{(}\FunctionTok{c}\NormalTok{(}\StringTok{"Hockey{-}stick"}\NormalTok{, }\StringTok{"ICES"}\NormalTok{, }\StringTok{"40{-}10"}\NormalTok{, }\StringTok{"Constant F"}\NormalTok{), }
              \AttributeTok{each =} \FunctionTok{length}\NormalTok{(B\_range)),}
    \AttributeTok{F\_advice =} \ConstantTok{NA}
\NormalTok{  )}
  
  \ControlFlowTok{for}\NormalTok{ (i }\ControlFlowTok{in} \DecValTok{1}\SpecialCharTok{:}\FunctionTok{length}\NormalTok{(B\_range)) \{}
\NormalTok{    B }\OtherTok{\textless{}{-}}\NormalTok{ B\_range[i]}
    
    \CommentTok{\# Hockey{-}stick}
\NormalTok{    results}\SpecialCharTok{$}\NormalTok{F\_advice[i] }\OtherTok{\textless{}{-}} \FunctionTok{hockey\_stick\_HCR}\NormalTok{(B, Bmsy, Fmsy)}
    
    \CommentTok{\# ICES}
\NormalTok{    ices }\OtherTok{\textless{}{-}} \FunctionTok{ICES\_advice\_rule}\NormalTok{(B, Bmsy, Fmsy)}
\NormalTok{    results}\SpecialCharTok{$}\NormalTok{F\_advice[i }\SpecialCharTok{+} \FunctionTok{length}\NormalTok{(B\_range)] }\OtherTok{\textless{}{-}}\NormalTok{ ices}\SpecialCharTok{$}\NormalTok{F\_advice}
    
    \CommentTok{\# 40{-}10}
\NormalTok{    rule40 }\OtherTok{\textless{}{-}} \FunctionTok{rule\_40\_10}\NormalTok{(B, K, Fmsy }\SpecialCharTok{*} \FloatTok{0.75}\NormalTok{)}
\NormalTok{    results}\SpecialCharTok{$}\NormalTok{F\_advice[i }\SpecialCharTok{+} \DecValTok{2}\SpecialCharTok{*}\FunctionTok{length}\NormalTok{(B\_range)] }\OtherTok{\textless{}{-}}\NormalTok{ rule40}\SpecialCharTok{$}\NormalTok{F\_advice}
    
    \CommentTok{\# Constant F}
\NormalTok{    results}\SpecialCharTok{$}\NormalTok{F\_advice[i }\SpecialCharTok{+} \DecValTok{3}\SpecialCharTok{*}\FunctionTok{length}\NormalTok{(B\_range)] }\OtherTok{\textless{}{-}}\NormalTok{ Fmsy}
\NormalTok{  \}}
  
  \FunctionTok{return}\NormalTok{(results)}
\NormalTok{\}}

\DocumentationTok{\#\# 7.2 Создание сравнительного графика}
\NormalTok{B\_test\_range }\OtherTok{\textless{}{-}} \FunctionTok{seq}\NormalTok{(}\FloatTok{0.1}\NormalTok{, }\DecValTok{2} \SpecialCharTok{*}\NormalTok{ Bmsy, }\AttributeTok{length.out =} \DecValTok{100}\NormalTok{)}
\NormalTok{comparison\_df }\OtherTok{\textless{}{-}} \FunctionTok{compare\_HCRs}\NormalTok{(B\_test\_range, Bmsy, Fmsy, K)}

\NormalTok{p\_comparison }\OtherTok{\textless{}{-}} \FunctionTok{ggplot}\NormalTok{(comparison\_df, }\FunctionTok{aes}\NormalTok{(}\AttributeTok{x =}\NormalTok{ B\_Bmsy, }\AttributeTok{y =}\NormalTok{ F\_advice, }
                                          \AttributeTok{color =}\NormalTok{ HCR, }\AttributeTok{linetype =}\NormalTok{ HCR)) }\SpecialCharTok{+}
  \FunctionTok{geom\_line}\NormalTok{(}\AttributeTok{linewidth =} \FloatTok{1.2}\NormalTok{) }\SpecialCharTok{+}
  \FunctionTok{geom\_vline}\NormalTok{(}\AttributeTok{xintercept =} \DecValTok{1}\NormalTok{, }\AttributeTok{linetype =} \StringTok{"dashed"}\NormalTok{, }\AttributeTok{alpha =} \FloatTok{0.5}\NormalTok{) }\SpecialCharTok{+}
  \FunctionTok{geom\_hline}\NormalTok{(}\AttributeTok{yintercept =}\NormalTok{ Fmsy, }\AttributeTok{linetype =} \StringTok{"dashed"}\NormalTok{, }\AttributeTok{alpha =} \FloatTok{0.5}\NormalTok{) }\SpecialCharTok{+}
  \FunctionTok{labs}\NormalTok{(}\AttributeTok{title =} \StringTok{"Сравнение различных правил управления (HCR)"}\NormalTok{,}
       \AttributeTok{x =} \StringTok{"B/Bmsy"}\NormalTok{, }\AttributeTok{y =} \StringTok{"Рекомендованная F"}\NormalTok{) }\SpecialCharTok{+}
  \FunctionTok{theme\_minimal}\NormalTok{() }\SpecialCharTok{+}
  \FunctionTok{scale\_color\_brewer}\NormalTok{(}\AttributeTok{palette =} \StringTok{"Set2"}\NormalTok{) }\SpecialCharTok{+}
  \FunctionTok{annotate}\NormalTok{(}\StringTok{"text"}\NormalTok{, }\AttributeTok{x =} \DecValTok{1}\NormalTok{, }\AttributeTok{y =} \DecValTok{0}\NormalTok{, }\AttributeTok{label =} \StringTok{"Bmsy"}\NormalTok{, }\AttributeTok{vjust =} \SpecialCharTok{{-}}\DecValTok{1}\NormalTok{)}

\FunctionTok{print}\NormalTok{(p\_comparison)}
\end{Highlighting}
\end{Shaded}

\begin{verbatim}
Warning in grid.Call(C_textBounds, as.graphicsAnnot(x$label), x$x, x$y, :
неизвестна ширина символа 0xd0 в кодировке CP1251
\end{verbatim}

\begin{verbatim}
Warning in grid.Call(C_textBounds, as.graphicsAnnot(x$label), x$x, x$y, :
неизвестна ширина символа 0xe5 в кодировке CP1251
\end{verbatim}

\begin{verbatim}
Warning in grid.Call(C_textBounds, as.graphicsAnnot(x$label), x$x, x$y, :
неизвестна ширина символа 0xea в кодировке CP1251
\end{verbatim}

\begin{verbatim}
Warning in grid.Call(C_textBounds, as.graphicsAnnot(x$label), x$x, x$y, :
неизвестна ширина символа 0xee в кодировке CP1251
\end{verbatim}

\begin{verbatim}
Warning in grid.Call(C_textBounds, as.graphicsAnnot(x$label), x$x, x$y, :
неизвестна ширина символа 0xec в кодировке CP1251
\end{verbatim}

\begin{verbatim}
Warning in grid.Call(C_textBounds, as.graphicsAnnot(x$label), x$x, x$y, :
неизвестна ширина символа 0xe5 в кодировке CP1251
\end{verbatim}

\begin{verbatim}
Warning in grid.Call(C_textBounds, as.graphicsAnnot(x$label), x$x, x$y, :
неизвестна ширина символа 0xed в кодировке CP1251
\end{verbatim}

\begin{verbatim}
Warning in grid.Call(C_textBounds, as.graphicsAnnot(x$label), x$x, x$y, :
неизвестна ширина символа 0xe4 в кодировке CP1251
\end{verbatim}

\begin{verbatim}
Warning in grid.Call(C_textBounds, as.graphicsAnnot(x$label), x$x, x$y, :
неизвестна ширина символа 0xee в кодировке CP1251
\end{verbatim}

\begin{verbatim}
Warning in grid.Call(C_textBounds, as.graphicsAnnot(x$label), x$x, x$y, :
неизвестна ширина символа 0xe2 в кодировке CP1251
\end{verbatim}

\begin{verbatim}
Warning in grid.Call(C_textBounds, as.graphicsAnnot(x$label), x$x, x$y, :
неизвестна ширина символа 0xe0 в кодировке CP1251
\end{verbatim}

\begin{verbatim}
Warning in grid.Call(C_textBounds, as.graphicsAnnot(x$label), x$x, x$y, :
неизвестна ширина символа 0xed в кодировке CP1251
Warning in grid.Call(C_textBounds, as.graphicsAnnot(x$label), x$x, x$y, :
неизвестна ширина символа 0xed в кодировке CP1251
\end{verbatim}

\begin{verbatim}
Warning in grid.Call(C_textBounds, as.graphicsAnnot(x$label), x$x, x$y, :
неизвестна ширина символа 0xe0 в кодировке CP1251
\end{verbatim}

\begin{verbatim}
Warning in grid.Call(C_textBounds, as.graphicsAnnot(x$label), x$x, x$y, :
неизвестна ширина символа 0xff в кодировке CP1251
\end{verbatim}

\begin{verbatim}
Warning in grid.Call(C_textBounds, as.graphicsAnnot(x$label), x$x, x$y, :
неизвестна ширина символа 0xd1 в кодировке CP1251
\end{verbatim}

\begin{verbatim}
Warning in grid.Call(C_textBounds, as.graphicsAnnot(x$label), x$x, x$y, :
неизвестна ширина символа 0xf0 в кодировке CP1251
\end{verbatim}

\begin{verbatim}
Warning in grid.Call(C_textBounds, as.graphicsAnnot(x$label), x$x, x$y, :
неизвестна ширина символа 0xe0 в кодировке CP1251
\end{verbatim}

\begin{verbatim}
Warning in grid.Call(C_textBounds, as.graphicsAnnot(x$label), x$x, x$y, :
неизвестна ширина символа 0xe2 в кодировке CP1251
\end{verbatim}

\begin{verbatim}
Warning in grid.Call(C_textBounds, as.graphicsAnnot(x$label), x$x, x$y, :
неизвестна ширина символа 0xed в кодировке CP1251
\end{verbatim}

\begin{verbatim}
Warning in grid.Call(C_textBounds, as.graphicsAnnot(x$label), x$x, x$y, :
неизвестна ширина символа 0xe5 в кодировке CP1251
\end{verbatim}

\begin{verbatim}
Warning in grid.Call(C_textBounds, as.graphicsAnnot(x$label), x$x, x$y, :
неизвестна ширина символа 0xed в кодировке CP1251
\end{verbatim}

\begin{verbatim}
Warning in grid.Call(C_textBounds, as.graphicsAnnot(x$label), x$x, x$y, :
неизвестна ширина символа 0xe8 в кодировке CP1251
\end{verbatim}

\begin{verbatim}
Warning in grid.Call(C_textBounds, as.graphicsAnnot(x$label), x$x, x$y, :
неизвестна ширина символа 0xe5 в кодировке CP1251
\end{verbatim}

\begin{verbatim}
Warning in grid.Call(C_textBounds, as.graphicsAnnot(x$label), x$x, x$y, :
неизвестна ширина символа 0xf0 в кодировке CP1251
\end{verbatim}

\begin{verbatim}
Warning in grid.Call(C_textBounds, as.graphicsAnnot(x$label), x$x, x$y, :
неизвестна ширина символа 0xe0 в кодировке CP1251
\end{verbatim}

\begin{verbatim}
Warning in grid.Call(C_textBounds, as.graphicsAnnot(x$label), x$x, x$y, :
неизвестна ширина символа 0xe7 в кодировке CP1251
\end{verbatim}

\begin{verbatim}
Warning in grid.Call(C_textBounds, as.graphicsAnnot(x$label), x$x, x$y, :
неизвестна ширина символа 0xeb в кодировке CP1251
\end{verbatim}

\begin{verbatim}
Warning in grid.Call(C_textBounds, as.graphicsAnnot(x$label), x$x, x$y, :
неизвестна ширина символа 0xe8 в кодировке CP1251
\end{verbatim}

\begin{verbatim}
Warning in grid.Call(C_textBounds, as.graphicsAnnot(x$label), x$x, x$y, :
неизвестна ширина символа 0xf7 в кодировке CP1251
\end{verbatim}

\begin{verbatim}
Warning in grid.Call(C_textBounds, as.graphicsAnnot(x$label), x$x, x$y, :
неизвестна ширина символа 0xed в кодировке CP1251
\end{verbatim}

\begin{verbatim}
Warning in grid.Call(C_textBounds, as.graphicsAnnot(x$label), x$x, x$y, :
неизвестна ширина символа 0xfb в кодировке CP1251
\end{verbatim}

\begin{verbatim}
Warning in grid.Call(C_textBounds, as.graphicsAnnot(x$label), x$x, x$y, :
неизвестна ширина символа 0xf5 в кодировке CP1251
\end{verbatim}

\begin{verbatim}
Warning in grid.Call(C_textBounds, as.graphicsAnnot(x$label), x$x, x$y, :
неизвестна ширина символа 0xef в кодировке CP1251
\end{verbatim}

\begin{verbatim}
Warning in grid.Call(C_textBounds, as.graphicsAnnot(x$label), x$x, x$y, :
неизвестна ширина символа 0xf0 в кодировке CP1251
\end{verbatim}

\begin{verbatim}
Warning in grid.Call(C_textBounds, as.graphicsAnnot(x$label), x$x, x$y, :
неизвестна ширина символа 0xe0 в кодировке CP1251
\end{verbatim}

\begin{verbatim}
Warning in grid.Call(C_textBounds, as.graphicsAnnot(x$label), x$x, x$y, :
неизвестна ширина символа 0xe2 в кодировке CP1251
\end{verbatim}

\begin{verbatim}
Warning in grid.Call(C_textBounds, as.graphicsAnnot(x$label), x$x, x$y, :
неизвестна ширина символа 0xe8 в кодировке CP1251
\end{verbatim}

\begin{verbatim}
Warning in grid.Call(C_textBounds, as.graphicsAnnot(x$label), x$x, x$y, :
неизвестна ширина символа 0xeb в кодировке CP1251
\end{verbatim}

\begin{verbatim}
Warning in grid.Call(C_textBounds, as.graphicsAnnot(x$label), x$x, x$y, :
неизвестна ширина символа 0xf3 в кодировке CP1251
\end{verbatim}

\begin{verbatim}
Warning in grid.Call(C_textBounds, as.graphicsAnnot(x$label), x$x, x$y, :
неизвестна ширина символа 0xef в кодировке CP1251
\end{verbatim}

\begin{verbatim}
Warning in grid.Call(C_textBounds, as.graphicsAnnot(x$label), x$x, x$y, :
неизвестна ширина символа 0xf0 в кодировке CP1251
\end{verbatim}

\begin{verbatim}
Warning in grid.Call(C_textBounds, as.graphicsAnnot(x$label), x$x, x$y, :
неизвестна ширина символа 0xe0 в кодировке CP1251
\end{verbatim}

\begin{verbatim}
Warning in grid.Call(C_textBounds, as.graphicsAnnot(x$label), x$x, x$y, :
неизвестна ширина символа 0xe2 в кодировке CP1251
\end{verbatim}

\begin{verbatim}
Warning in grid.Call(C_textBounds, as.graphicsAnnot(x$label), x$x, x$y, :
неизвестна ширина символа 0xeb в кодировке CP1251
\end{verbatim}

\begin{verbatim}
Warning in grid.Call(C_textBounds, as.graphicsAnnot(x$label), x$x, x$y, :
неизвестна ширина символа 0xe5 в кодировке CP1251
\end{verbatim}

\begin{verbatim}
Warning in grid.Call(C_textBounds, as.graphicsAnnot(x$label), x$x, x$y, :
неизвестна ширина символа 0xed в кодировке CP1251
\end{verbatim}

\begin{verbatim}
Warning in grid.Call(C_textBounds, as.graphicsAnnot(x$label), x$x, x$y, :
неизвестна ширина символа 0xe8 в кодировке CP1251
\end{verbatim}

\begin{verbatim}
Warning in grid.Call(C_textBounds, as.graphicsAnnot(x$label), x$x, x$y, :
неизвестна ширина символа 0xff в кодировке CP1251
\end{verbatim}

\begin{verbatim}
Warning in grid.Call.graphics(C_text, as.graphicsAnnot(x$label), x$x, x$y, :
неизвестна ширина символа 0xd0 в кодировке CP1251
\end{verbatim}

\begin{verbatim}
Warning in grid.Call.graphics(C_text, as.graphicsAnnot(x$label), x$x, x$y, :
неизвестна ширина символа 0xe5 в кодировке CP1251
\end{verbatim}

\begin{verbatim}
Warning in grid.Call.graphics(C_text, as.graphicsAnnot(x$label), x$x, x$y, :
неизвестна ширина символа 0xea в кодировке CP1251
\end{verbatim}

\begin{verbatim}
Warning in grid.Call.graphics(C_text, as.graphicsAnnot(x$label), x$x, x$y, :
неизвестна ширина символа 0xee в кодировке CP1251
\end{verbatim}

\begin{verbatim}
Warning in grid.Call.graphics(C_text, as.graphicsAnnot(x$label), x$x, x$y, :
неизвестна ширина символа 0xec в кодировке CP1251
\end{verbatim}

\begin{verbatim}
Warning in grid.Call.graphics(C_text, as.graphicsAnnot(x$label), x$x, x$y, :
неизвестна ширина символа 0xe5 в кодировке CP1251
\end{verbatim}

\begin{verbatim}
Warning in grid.Call.graphics(C_text, as.graphicsAnnot(x$label), x$x, x$y, :
неизвестна ширина символа 0xed в кодировке CP1251
\end{verbatim}

\begin{verbatim}
Warning in grid.Call.graphics(C_text, as.graphicsAnnot(x$label), x$x, x$y, :
неизвестна ширина символа 0xe4 в кодировке CP1251
\end{verbatim}

\begin{verbatim}
Warning in grid.Call.graphics(C_text, as.graphicsAnnot(x$label), x$x, x$y, :
неизвестна ширина символа 0xee в кодировке CP1251
\end{verbatim}

\begin{verbatim}
Warning in grid.Call.graphics(C_text, as.graphicsAnnot(x$label), x$x, x$y, :
неизвестна ширина символа 0xe2 в кодировке CP1251
\end{verbatim}

\begin{verbatim}
Warning in grid.Call.graphics(C_text, as.graphicsAnnot(x$label), x$x, x$y, :
неизвестна ширина символа 0xe0 в кодировке CP1251
\end{verbatim}

\begin{verbatim}
Warning in grid.Call.graphics(C_text, as.graphicsAnnot(x$label), x$x, x$y, :
неизвестна ширина символа 0xed в кодировке CP1251
Warning in grid.Call.graphics(C_text, as.graphicsAnnot(x$label), x$x, x$y, :
неизвестна ширина символа 0xed в кодировке CP1251
\end{verbatim}

\begin{verbatim}
Warning in grid.Call.graphics(C_text, as.graphicsAnnot(x$label), x$x, x$y, :
неизвестна ширина символа 0xe0 в кодировке CP1251
\end{verbatim}

\begin{verbatim}
Warning in grid.Call.graphics(C_text, as.graphicsAnnot(x$label), x$x, x$y, :
неизвестна ширина символа 0xff в кодировке CP1251
\end{verbatim}

\begin{verbatim}
Warning in grid.Call.graphics(C_text, as.graphicsAnnot(x$label), x$x, x$y, :
неизвестна ширина символа 0xd1 в кодировке CP1251
\end{verbatim}

\begin{verbatim}
Warning in grid.Call.graphics(C_text, as.graphicsAnnot(x$label), x$x, x$y, :
неизвестна ширина символа 0xf0 в кодировке CP1251
\end{verbatim}

\begin{verbatim}
Warning in grid.Call.graphics(C_text, as.graphicsAnnot(x$label), x$x, x$y, :
неизвестна ширина символа 0xe0 в кодировке CP1251
\end{verbatim}

\begin{verbatim}
Warning in grid.Call.graphics(C_text, as.graphicsAnnot(x$label), x$x, x$y, :
неизвестна ширина символа 0xe2 в кодировке CP1251
\end{verbatim}

\begin{verbatim}
Warning in grid.Call.graphics(C_text, as.graphicsAnnot(x$label), x$x, x$y, :
неизвестна ширина символа 0xed в кодировке CP1251
\end{verbatim}

\begin{verbatim}
Warning in grid.Call.graphics(C_text, as.graphicsAnnot(x$label), x$x, x$y, :
неизвестна ширина символа 0xe5 в кодировке CP1251
\end{verbatim}

\begin{verbatim}
Warning in grid.Call.graphics(C_text, as.graphicsAnnot(x$label), x$x, x$y, :
неизвестна ширина символа 0xed в кодировке CP1251
\end{verbatim}

\begin{verbatim}
Warning in grid.Call.graphics(C_text, as.graphicsAnnot(x$label), x$x, x$y, :
неизвестна ширина символа 0xe8 в кодировке CP1251
\end{verbatim}

\begin{verbatim}
Warning in grid.Call.graphics(C_text, as.graphicsAnnot(x$label), x$x, x$y, :
неизвестна ширина символа 0xe5 в кодировке CP1251
\end{verbatim}

\begin{verbatim}
Warning in grid.Call.graphics(C_text, as.graphicsAnnot(x$label), x$x, x$y, :
неизвестна ширина символа 0xf0 в кодировке CP1251
\end{verbatim}

\begin{verbatim}
Warning in grid.Call.graphics(C_text, as.graphicsAnnot(x$label), x$x, x$y, :
неизвестна ширина символа 0xe0 в кодировке CP1251
\end{verbatim}

\begin{verbatim}
Warning in grid.Call.graphics(C_text, as.graphicsAnnot(x$label), x$x, x$y, :
неизвестна ширина символа 0xe7 в кодировке CP1251
\end{verbatim}

\begin{verbatim}
Warning in grid.Call.graphics(C_text, as.graphicsAnnot(x$label), x$x, x$y, :
неизвестна ширина символа 0xeb в кодировке CP1251
\end{verbatim}

\begin{verbatim}
Warning in grid.Call.graphics(C_text, as.graphicsAnnot(x$label), x$x, x$y, :
неизвестна ширина символа 0xe8 в кодировке CP1251
\end{verbatim}

\begin{verbatim}
Warning in grid.Call.graphics(C_text, as.graphicsAnnot(x$label), x$x, x$y, :
неизвестна ширина символа 0xf7 в кодировке CP1251
\end{verbatim}

\begin{verbatim}
Warning in grid.Call.graphics(C_text, as.graphicsAnnot(x$label), x$x, x$y, :
неизвестна ширина символа 0xed в кодировке CP1251
\end{verbatim}

\begin{verbatim}
Warning in grid.Call.graphics(C_text, as.graphicsAnnot(x$label), x$x, x$y, :
неизвестна ширина символа 0xfb в кодировке CP1251
\end{verbatim}

\begin{verbatim}
Warning in grid.Call.graphics(C_text, as.graphicsAnnot(x$label), x$x, x$y, :
неизвестна ширина символа 0xf5 в кодировке CP1251
\end{verbatim}

\begin{verbatim}
Warning in grid.Call.graphics(C_text, as.graphicsAnnot(x$label), x$x, x$y, :
неизвестна ширина символа 0xef в кодировке CP1251
\end{verbatim}

\begin{verbatim}
Warning in grid.Call.graphics(C_text, as.graphicsAnnot(x$label), x$x, x$y, :
неизвестна ширина символа 0xf0 в кодировке CP1251
\end{verbatim}

\begin{verbatim}
Warning in grid.Call.graphics(C_text, as.graphicsAnnot(x$label), x$x, x$y, :
неизвестна ширина символа 0xe0 в кодировке CP1251
\end{verbatim}

\begin{verbatim}
Warning in grid.Call.graphics(C_text, as.graphicsAnnot(x$label), x$x, x$y, :
неизвестна ширина символа 0xe2 в кодировке CP1251
\end{verbatim}

\begin{verbatim}
Warning in grid.Call.graphics(C_text, as.graphicsAnnot(x$label), x$x, x$y, :
неизвестна ширина символа 0xe8 в кодировке CP1251
\end{verbatim}

\begin{verbatim}
Warning in grid.Call.graphics(C_text, as.graphicsAnnot(x$label), x$x, x$y, :
неизвестна ширина символа 0xeb в кодировке CP1251
\end{verbatim}

\begin{verbatim}
Warning in grid.Call.graphics(C_text, as.graphicsAnnot(x$label), x$x, x$y, :
неизвестна ширина символа 0xf3 в кодировке CP1251
\end{verbatim}

\begin{verbatim}
Warning in grid.Call.graphics(C_text, as.graphicsAnnot(x$label), x$x, x$y, :
неизвестна ширина символа 0xef в кодировке CP1251
\end{verbatim}

\begin{verbatim}
Warning in grid.Call.graphics(C_text, as.graphicsAnnot(x$label), x$x, x$y, :
неизвестна ширина символа 0xf0 в кодировке CP1251
\end{verbatim}

\begin{verbatim}
Warning in grid.Call.graphics(C_text, as.graphicsAnnot(x$label), x$x, x$y, :
неизвестна ширина символа 0xe0 в кодировке CP1251
\end{verbatim}

\begin{verbatim}
Warning in grid.Call.graphics(C_text, as.graphicsAnnot(x$label), x$x, x$y, :
неизвестна ширина символа 0xe2 в кодировке CP1251
\end{verbatim}

\begin{verbatim}
Warning in grid.Call.graphics(C_text, as.graphicsAnnot(x$label), x$x, x$y, :
неизвестна ширина символа 0xeb в кодировке CP1251
\end{verbatim}

\begin{verbatim}
Warning in grid.Call.graphics(C_text, as.graphicsAnnot(x$label), x$x, x$y, :
неизвестна ширина символа 0xe5 в кодировке CP1251
\end{verbatim}

\begin{verbatim}
Warning in grid.Call.graphics(C_text, as.graphicsAnnot(x$label), x$x, x$y, :
неизвестна ширина символа 0xed в кодировке CP1251
\end{verbatim}

\begin{verbatim}
Warning in grid.Call.graphics(C_text, as.graphicsAnnot(x$label), x$x, x$y, :
неизвестна ширина символа 0xe8 в кодировке CP1251
\end{verbatim}

\begin{verbatim}
Warning in grid.Call.graphics(C_text, as.graphicsAnnot(x$label), x$x, x$y, :
неизвестна ширина символа 0xff в кодировке CP1251
\end{verbatim}

\pandocbounded{\includegraphics[keepaspectratio]{chapter14_files/figure-pdf/unnamed-chunk-1-3.pdf}}

\begin{Shaded}
\begin{Highlighting}[]
\CommentTok{\# {-}{-}{-}{-}{-}{-}{-}{-}{-}{-}{-}{-}{-}{-}{-}{-}{-}{-}{-} 8. РАСЧЕТ ОДУ С УЧЕТОМ НЕОПРЕДЕЛЕННОСТИ {-}{-}{-}{-}{-}{-}{-}{-}{-}{-}{-}{-}{-}{-}{-}{-}{-}{-}{-}{-}}

\FunctionTok{cat}\NormalTok{(}\StringTok{"}\SpecialCharTok{\textbackslash{}n}\StringTok{========== РАСЧЕТ ОДУ С НЕОПРЕДЕЛЕННОСТЬЮ ==========}\SpecialCharTok{\textbackslash{}n}\StringTok{"}\NormalTok{)}
\end{Highlighting}
\end{Shaded}

\begin{verbatim}

========== РАСЧЕТ ОДУ С НЕОПРЕДЕЛЕННОСТЬЮ ==========
\end{verbatim}

\begin{Shaded}
\begin{Highlighting}[]
\DocumentationTok{\#\# 8.1 Функция для расчета ОДУ с доверительными интервалами}
\NormalTok{calculate\_TAC\_uncertainty }\OtherTok{\textless{}{-}} \ControlFlowTok{function}\NormalTok{(fit, }\AttributeTok{F\_target\_ratio =} \FloatTok{1.0}\NormalTok{, }
                                     \AttributeTok{confidence\_level =} \FloatTok{0.95}\NormalTok{) \{}
  
  \CommentTok{\# Извлечение параметров с неопределенностью}
\NormalTok{  B\_est }\OtherTok{\textless{}{-}} \FunctionTok{get.par}\NormalTok{(}\StringTok{"logB"}\NormalTok{, fit, }\AttributeTok{exp =} \ConstantTok{TRUE}\NormalTok{)}
\NormalTok{  F\_current }\OtherTok{\textless{}{-}} \FunctionTok{get.par}\NormalTok{(}\StringTok{"logF"}\NormalTok{, fit, }\AttributeTok{exp =} \ConstantTok{TRUE}\NormalTok{)}
\NormalTok{  Fmsy }\OtherTok{\textless{}{-}} \FunctionTok{get.par}\NormalTok{(}\StringTok{"logFmsy"}\NormalTok{, fit, }\AttributeTok{exp =} \ConstantTok{TRUE}\NormalTok{)}
  
  \CommentTok{\# Целевая F}
\NormalTok{  F\_target }\OtherTok{\textless{}{-}}\NormalTok{ Fmsy[}\DecValTok{1}\NormalTok{] }\SpecialCharTok{*}\NormalTok{ F\_target\_ratio}
  
  \CommentTok{\# Расчет TAC с квантилями}
\NormalTok{  TAC\_median }\OtherTok{\textless{}{-}}\NormalTok{ F\_target }\SpecialCharTok{*}\NormalTok{ B\_est[}\DecValTok{1}\NormalTok{]}
\NormalTok{  TAC\_lower }\OtherTok{\textless{}{-}}\NormalTok{ F\_target }\SpecialCharTok{*}\NormalTok{ B\_est[}\DecValTok{2}\NormalTok{]  }\CommentTok{\# Нижняя граница}
\NormalTok{  TAC\_upper }\OtherTok{\textless{}{-}}\NormalTok{ F\_target }\SpecialCharTok{*}\NormalTok{ B\_est[}\DecValTok{3}\NormalTok{]  }\CommentTok{\# Верхняя граница}
  
  \CommentTok{\# Предосторожный подход {-} используем нижний квантиль биомассы}
\NormalTok{  TAC\_precautionary }\OtherTok{\textless{}{-}}\NormalTok{ F\_target }\SpecialCharTok{*}\NormalTok{ B\_est[}\DecValTok{2}\NormalTok{]}
  
  \FunctionTok{return}\NormalTok{(}\FunctionTok{list}\NormalTok{(}
    \AttributeTok{TAC\_median =}\NormalTok{ TAC\_median,}
    \AttributeTok{TAC\_lower =}\NormalTok{ TAC\_lower,}
    \AttributeTok{TAC\_upper =}\NormalTok{ TAC\_upper,}
    \AttributeTok{TAC\_precautionary =}\NormalTok{ TAC\_precautionary,}
    \AttributeTok{F\_target =}\NormalTok{ F\_target,}
    \AttributeTok{B\_estimate =}\NormalTok{ B\_est}
\NormalTok{  ))}
\NormalTok{\}}

\DocumentationTok{\#\# 8.2 Расчет TAC для разных сценариев}
\NormalTok{scenarios\_TAC }\OtherTok{\textless{}{-}} \FunctionTok{list}\NormalTok{(}
  \StringTok{"F = Fmsy"} \OtherTok{=} \FloatTok{1.0}\NormalTok{,}
  \StringTok{"F = 0.75*Fmsy"} \OtherTok{=} \FloatTok{0.75}\NormalTok{,}
  \StringTok{"F = 0.5*Fmsy"} \OtherTok{=} \FloatTok{0.5}\NormalTok{,}
  \StringTok{"F = 1.25*Fmsy"} \OtherTok{=} \FloatTok{1.25}
\NormalTok{)}

\FunctionTok{cat}\NormalTok{(}\StringTok{"}\SpecialCharTok{\textbackslash{}n}\StringTok{{-}{-}{-} Расчет ОДУ для различных целевых F {-}{-}{-}}\SpecialCharTok{\textbackslash{}n}\StringTok{"}\NormalTok{)}
\end{Highlighting}
\end{Shaded}

\begin{verbatim}

--- Расчет ОДУ для различных целевых F ---
\end{verbatim}

\begin{Shaded}
\begin{Highlighting}[]
\ControlFlowTok{for}\NormalTok{ (scenario }\ControlFlowTok{in} \FunctionTok{names}\NormalTok{(scenarios\_TAC)) \{}
\NormalTok{  tac\_result }\OtherTok{\textless{}{-}} \FunctionTok{calculate\_TAC\_uncertainty}\NormalTok{(fit, scenarios\_TAC[[scenario]])}
  \FunctionTok{cat}\NormalTok{(}\FunctionTok{sprintf}\NormalTok{(}\StringTok{"}\SpecialCharTok{\textbackslash{}n}\StringTok{\%s:}\SpecialCharTok{\textbackslash{}n}\StringTok{"}\NormalTok{, scenario))}
  \FunctionTok{cat}\NormalTok{(}\FunctionTok{sprintf}\NormalTok{(}\StringTok{"  TAC (медиана): \%.1f тыс. т}\SpecialCharTok{\textbackslash{}n}\StringTok{"}\NormalTok{, tac\_result}\SpecialCharTok{$}\NormalTok{TAC\_median))}
  \FunctionTok{cat}\NormalTok{(}\FunctionTok{sprintf}\NormalTok{(}\StringTok{"  TAC (95\%\% CI): [\%.1f {-} \%.1f] тыс. т}\SpecialCharTok{\textbackslash{}n}\StringTok{"}\NormalTok{, }
\NormalTok{              tac\_result}\SpecialCharTok{$}\NormalTok{TAC\_lower, tac\_result}\SpecialCharTok{$}\NormalTok{TAC\_upper))}
  \FunctionTok{cat}\NormalTok{(}\FunctionTok{sprintf}\NormalTok{(}\StringTok{"  TAC (предосторожный): \%.1f тыс. т}\SpecialCharTok{\textbackslash{}n}\StringTok{"}\NormalTok{, }
\NormalTok{              tac\_result}\SpecialCharTok{$}\NormalTok{TAC\_precautionary))}
\NormalTok{\}}
\end{Highlighting}
\end{Shaded}

\begin{verbatim}

F = Fmsy:
  TAC (медиана): 14.9 тыс. т
  TAC (95% CI): [15.0 - 15.1] тыс. т
  TAC (предосторожный): 15.0 тыс. т

F = 0.75*Fmsy:
  TAC (медиана): 11.2 тыс. т
  TAC (95% CI): [11.3 - 11.3] тыс. т
  TAC (предосторожный): 11.3 тыс. т

F = 0.5*Fmsy:
  TAC (медиана): 7.4 тыс. т
  TAC (95% CI): [7.5 - 7.6] тыс. т
  TAC (предосторожный): 7.5 тыс. т

F = 1.25*Fmsy:
  TAC (медиана): 18.6 тыс. т
  TAC (95% CI): [18.8 - 18.9] тыс. т
  TAC (предосторожный): 18.8 тыс. т
\end{verbatim}

\begin{Shaded}
\begin{Highlighting}[]
\CommentTok{\# {-}{-}{-}{-}{-}{-}{-}{-}{-}{-}{-}{-}{-}{-}{-}{-}{-}{-}{-} 9. МНОГОЛЕТНИЙ ПЛАН УПРАВЛЕНИЯ {-}{-}{-}{-}{-}{-}{-}{-}{-}{-}{-}{-}{-}{-}{-}{-}{-}{-}{-}{-}}

\FunctionTok{cat}\NormalTok{(}\StringTok{"}\SpecialCharTok{\textbackslash{}n}\StringTok{========== МНОГОЛЕТНИЙ ПЛАН УПРАВЛЕНИЯ ==========}\SpecialCharTok{\textbackslash{}n}\StringTok{"}\NormalTok{)}
\end{Highlighting}
\end{Shaded}

\begin{verbatim}

========== МНОГОЛЕТНИЙ ПЛАН УПРАВЛЕНИЯ ==========
\end{verbatim}

\begin{Shaded}
\begin{Highlighting}[]
\DocumentationTok{\#\# 9.1 Функция для создания многолетнего плана}
\NormalTok{create\_management\_plan }\OtherTok{\textless{}{-}} \ControlFlowTok{function}\NormalTok{(fit, }\AttributeTok{n\_years =} \DecValTok{5}\NormalTok{, }
                                  \AttributeTok{target\_B\_Bmsy =} \FloatTok{1.2}\NormalTok{,}
                                  \AttributeTok{recovery\_period =} \DecValTok{10}\NormalTok{) \{}
  
  \CommentTok{\# Текущее состояние}
\NormalTok{  B\_current }\OtherTok{\textless{}{-}} \FunctionTok{get.par}\NormalTok{(}\StringTok{"logB"}\NormalTok{, fit, }\AttributeTok{exp =} \ConstantTok{TRUE}\NormalTok{)[}\DecValTok{1}\NormalTok{]}
\NormalTok{  Bmsy }\OtherTok{\textless{}{-}} \FunctionTok{get.par}\NormalTok{(}\StringTok{"logBmsy"}\NormalTok{, fit, }\AttributeTok{exp =} \ConstantTok{TRUE}\NormalTok{)[}\DecValTok{1}\NormalTok{]}
\NormalTok{  Fmsy }\OtherTok{\textless{}{-}} \FunctionTok{get.par}\NormalTok{(}\StringTok{"logFmsy"}\NormalTok{, fit, }\AttributeTok{exp =} \ConstantTok{TRUE}\NormalTok{)[}\DecValTok{1}\NormalTok{]}
\NormalTok{  r }\OtherTok{\textless{}{-}} \FunctionTok{get.par}\NormalTok{(}\StringTok{"logr"}\NormalTok{, fit, }\AttributeTok{exp =} \ConstantTok{TRUE}\NormalTok{)[}\DecValTok{1}\NormalTok{]}
\NormalTok{  K }\OtherTok{\textless{}{-}} \FunctionTok{get.par}\NormalTok{(}\StringTok{"logK"}\NormalTok{, fit, }\AttributeTok{exp =} \ConstantTok{TRUE}\NormalTok{)[}\DecValTok{1}\NormalTok{]}
  
\NormalTok{  B\_Bmsy\_current }\OtherTok{\textless{}{-}}\NormalTok{ B\_current }\SpecialCharTok{/}\NormalTok{ Bmsy}
  
  \CommentTok{\# Определение стратегии}
  \ControlFlowTok{if}\NormalTok{ (B\_Bmsy\_current }\SpecialCharTok{\textless{}} \FloatTok{0.5}\NormalTok{) \{}
\NormalTok{    strategy }\OtherTok{\textless{}{-}} \StringTok{"Восстановление запаса"}
\NormalTok{    F\_multiplier\_start }\OtherTok{\textless{}{-}} \FloatTok{0.3}
\NormalTok{    F\_multiplier\_end }\OtherTok{\textless{}{-}} \FloatTok{0.7}
\NormalTok{  \} }\ControlFlowTok{else} \ControlFlowTok{if}\NormalTok{ (B\_Bmsy\_current }\SpecialCharTok{\textless{}} \FloatTok{1.0}\NormalTok{) \{}
\NormalTok{    strategy }\OtherTok{\textless{}{-}} \StringTok{"Умеренное восстановление"}
\NormalTok{    F\_multiplier\_start }\OtherTok{\textless{}{-}} \FloatTok{0.7}
\NormalTok{    F\_multiplier\_end }\OtherTok{\textless{}{-}} \FloatTok{0.9}
\NormalTok{  \} }\ControlFlowTok{else}\NormalTok{ \{}
\NormalTok{    strategy }\OtherTok{\textless{}{-}} \StringTok{"Устойчивый промысел"}
\NormalTok{    F\_multiplier\_start }\OtherTok{\textless{}{-}} \FloatTok{1.0}
\NormalTok{    F\_multiplier\_end }\OtherTok{\textless{}{-}} \FloatTok{1.0}
\NormalTok{  \}}
  
  \CommentTok{\# План по годам}
\NormalTok{  plan }\OtherTok{\textless{}{-}} \FunctionTok{data.frame}\NormalTok{(}
    \AttributeTok{Year =} \DecValTok{1}\SpecialCharTok{:}\NormalTok{n\_years,}
    \AttributeTok{Strategy =}\NormalTok{ strategy,}
    \AttributeTok{F\_multiplier =} \FunctionTok{seq}\NormalTok{(F\_multiplier\_start, F\_multiplier\_end, }
                      \AttributeTok{length.out =}\NormalTok{ n\_years),}
    \AttributeTok{F\_target =} \ConstantTok{NA}\NormalTok{,}
    \AttributeTok{Expected\_B =} \ConstantTok{NA}\NormalTok{,}
    \AttributeTok{Expected\_B\_Bmsy =} \ConstantTok{NA}\NormalTok{,}
    \AttributeTok{TAC =} \ConstantTok{NA}
\NormalTok{  )}
  
  \CommentTok{\# Прогнозирование}
\NormalTok{  B\_proj }\OtherTok{\textless{}{-}}\NormalTok{ B\_current}
  \ControlFlowTok{for}\NormalTok{ (i }\ControlFlowTok{in} \DecValTok{1}\SpecialCharTok{:}\NormalTok{n\_years) \{}
\NormalTok{    plan}\SpecialCharTok{$}\NormalTok{F\_target[i] }\OtherTok{\textless{}{-}}\NormalTok{ Fmsy }\SpecialCharTok{*}\NormalTok{ plan}\SpecialCharTok{$}\NormalTok{F\_multiplier[i]}
\NormalTok{    plan}\SpecialCharTok{$}\NormalTok{Expected\_B[i] }\OtherTok{\textless{}{-}}\NormalTok{ B\_proj}
\NormalTok{    plan}\SpecialCharTok{$}\NormalTok{Expected\_B\_Bmsy[i] }\OtherTok{\textless{}{-}}\NormalTok{ B\_proj }\SpecialCharTok{/}\NormalTok{ Bmsy}
\NormalTok{    plan}\SpecialCharTok{$}\NormalTok{TAC[i] }\OtherTok{\textless{}{-}}\NormalTok{ plan}\SpecialCharTok{$}\NormalTok{F\_target[i] }\SpecialCharTok{*}\NormalTok{ B\_proj}
    
    \CommentTok{\# Обновление биомассы}
\NormalTok{    surplus }\OtherTok{\textless{}{-}}\NormalTok{ r }\SpecialCharTok{*}\NormalTok{ B\_proj }\SpecialCharTok{*}\NormalTok{ (}\DecValTok{1} \SpecialCharTok{{-}}\NormalTok{ B\_proj}\SpecialCharTok{/}\NormalTok{K)}
\NormalTok{    catch }\OtherTok{\textless{}{-}}\NormalTok{ plan}\SpecialCharTok{$}\NormalTok{F\_target[i] }\SpecialCharTok{*}\NormalTok{ B\_proj}
\NormalTok{    B\_proj }\OtherTok{\textless{}{-}}\NormalTok{ B\_proj }\SpecialCharTok{+}\NormalTok{ surplus }\SpecialCharTok{{-}}\NormalTok{ catch}
\NormalTok{  \}}
  
  \FunctionTok{return}\NormalTok{(plan)}
\NormalTok{\}}

\DocumentationTok{\#\# 9.2 Создание и вывод плана}
\NormalTok{management\_plan }\OtherTok{\textless{}{-}} \FunctionTok{create\_management\_plan}\NormalTok{(fit, }\AttributeTok{n\_years =} \DecValTok{15}\NormalTok{)}

\FunctionTok{cat}\NormalTok{(}\StringTok{"}\SpecialCharTok{\textbackslash{}n}\StringTok{{-}{-}{-} 15{-}летний план управления {-}{-}{-}}\SpecialCharTok{\textbackslash{}n}\StringTok{"}\NormalTok{)}
\end{Highlighting}
\end{Shaded}

\begin{verbatim}

--- 15-летний план управления ---
\end{verbatim}

\begin{Shaded}
\begin{Highlighting}[]
\FunctionTok{print}\NormalTok{(management\_plan, }\AttributeTok{digits =} \DecValTok{2}\NormalTok{)}
\end{Highlighting}
\end{Shaded}

\begin{verbatim}
   Year            Strategy F_multiplier F_target Expected_B Expected_B_Bmsy
1     1 Устойчивый промысел            1     0.14        103             1.3
2     2 Устойчивый промысел            1     0.14         98             1.3
3     3 Устойчивый промысел            1     0.14         94             1.2
4     4 Устойчивый промысел            1     0.14         91             1.2
5     5 Устойчивый промысел            1     0.14         89             1.2
6     6 Устойчивый промысел            1     0.14         87             1.1
7     7 Устойчивый промысел            1     0.14         85             1.1
8     8 Устойчивый промысел            1     0.14         84             1.1
9     9 Устойчивый промысел            1     0.14         83             1.1
10   10 Устойчивый промысел            1     0.14         82             1.1
11   11 Устойчивый промысел            1     0.14         81             1.1
12   12 Устойчивый промысел            1     0.14         80             1.0
13   13 Устойчивый промысел            1     0.14         80             1.0
14   14 Устойчивый промысел            1     0.14         79             1.0
15   15 Устойчивый промысел            1     0.14         79             1.0
   TAC
1   15
2   14
3   14
4   13
5   13
6   13
7   12
8   12
9   12
10  12
11  12
12  12
13  12
14  11
15  11
\end{verbatim}

\begin{Shaded}
\begin{Highlighting}[]
\DocumentationTok{\#\# 9.3 Визуализация плана}
\NormalTok{p\_plan }\OtherTok{\textless{}{-}} \FunctionTok{ggplot}\NormalTok{(management\_plan, }\FunctionTok{aes}\NormalTok{(}\AttributeTok{x =}\NormalTok{ Year)) }\SpecialCharTok{+}
  \FunctionTok{geom\_line}\NormalTok{(}\FunctionTok{aes}\NormalTok{(}\AttributeTok{y =}\NormalTok{ TAC), }\AttributeTok{color =} \StringTok{"blue"}\NormalTok{, }\AttributeTok{size =} \FloatTok{1.2}\NormalTok{) }\SpecialCharTok{+}
  \FunctionTok{geom\_point}\NormalTok{(}\FunctionTok{aes}\NormalTok{(}\AttributeTok{y =}\NormalTok{ TAC), }\AttributeTok{color =} \StringTok{"blue"}\NormalTok{, }\AttributeTok{size =} \DecValTok{3}\NormalTok{) }\SpecialCharTok{+}
  \FunctionTok{geom\_text}\NormalTok{(}\FunctionTok{aes}\NormalTok{(}\AttributeTok{y =}\NormalTok{ TAC, }\AttributeTok{label =} \FunctionTok{round}\NormalTok{(TAC, }\DecValTok{1}\NormalTok{)), }\AttributeTok{vjust =} \SpecialCharTok{{-}}\DecValTok{1}\NormalTok{) }\SpecialCharTok{+}
  \FunctionTok{labs}\NormalTok{(}\AttributeTok{title =} \FunctionTok{paste}\NormalTok{(}\StringTok{"План управления:"}\NormalTok{, management\_plan}\SpecialCharTok{$}\NormalTok{Strategy[}\DecValTok{1}\NormalTok{]),}
       \AttributeTok{x =} \StringTok{"Год"}\NormalTok{, }\AttributeTok{y =} \StringTok{"ОДУ (тыс. т)"}\NormalTok{) }\SpecialCharTok{+}
  \FunctionTok{theme\_minimal}\NormalTok{()}
\end{Highlighting}
\end{Shaded}

\begin{verbatim}
Warning: Using `size` aesthetic for lines was deprecated in ggplot2 3.4.0.
i Please use `linewidth` instead.
\end{verbatim}

\begin{Shaded}
\begin{Highlighting}[]
\FunctionTok{print}\NormalTok{(p\_plan)}
\end{Highlighting}
\end{Shaded}

\begin{verbatim}
Warning in grid.Call(C_textBounds, as.graphicsAnnot(x$label), x$x, x$y, :
неизвестна ширина символа 0xce в кодировке CP1251
\end{verbatim}

\begin{verbatim}
Warning in grid.Call(C_textBounds, as.graphicsAnnot(x$label), x$x, x$y, :
неизвестна ширина символа 0xc4 в кодировке CP1251
\end{verbatim}

\begin{verbatim}
Warning in grid.Call(C_textBounds, as.graphicsAnnot(x$label), x$x, x$y, :
неизвестна ширина символа 0xd3 в кодировке CP1251
\end{verbatim}

\begin{verbatim}
Warning in grid.Call(C_textBounds, as.graphicsAnnot(x$label), x$x, x$y, :
неизвестна ширина символа 0xf2 в кодировке CP1251
\end{verbatim}

\begin{verbatim}
Warning in grid.Call(C_textBounds, as.graphicsAnnot(x$label), x$x, x$y, :
неизвестна ширина символа 0xfb в кодировке CP1251
\end{verbatim}

\begin{verbatim}
Warning in grid.Call(C_textBounds, as.graphicsAnnot(x$label), x$x, x$y, :
неизвестна ширина символа 0xf1 в кодировке CP1251
\end{verbatim}

\begin{verbatim}
Warning in grid.Call(C_textBounds, as.graphicsAnnot(x$label), x$x, x$y, :
неизвестна ширина символа 0xf2 в кодировке CP1251
\end{verbatim}

\begin{verbatim}
Warning in grid.Call(C_textBounds, as.graphicsAnnot(x$label), x$x, x$y, :
неизвестна ширина символа 0xcf в кодировке CP1251
\end{verbatim}

\begin{verbatim}
Warning in grid.Call(C_textBounds, as.graphicsAnnot(x$label), x$x, x$y, :
неизвестна ширина символа 0xeb в кодировке CP1251
\end{verbatim}

\begin{verbatim}
Warning in grid.Call(C_textBounds, as.graphicsAnnot(x$label), x$x, x$y, :
неизвестна ширина символа 0xe0 в кодировке CP1251
\end{verbatim}

\begin{verbatim}
Warning in grid.Call(C_textBounds, as.graphicsAnnot(x$label), x$x, x$y, :
неизвестна ширина символа 0xed в кодировке CP1251
\end{verbatim}

\begin{verbatim}
Warning in grid.Call(C_textBounds, as.graphicsAnnot(x$label), x$x, x$y, :
неизвестна ширина символа 0xf3 в кодировке CP1251
\end{verbatim}

\begin{verbatim}
Warning in grid.Call(C_textBounds, as.graphicsAnnot(x$label), x$x, x$y, :
неизвестна ширина символа 0xef в кодировке CP1251
\end{verbatim}

\begin{verbatim}
Warning in grid.Call(C_textBounds, as.graphicsAnnot(x$label), x$x, x$y, :
неизвестна ширина символа 0xf0 в кодировке CP1251
\end{verbatim}

\begin{verbatim}
Warning in grid.Call(C_textBounds, as.graphicsAnnot(x$label), x$x, x$y, :
неизвестна ширина символа 0xe0 в кодировке CP1251
\end{verbatim}

\begin{verbatim}
Warning in grid.Call(C_textBounds, as.graphicsAnnot(x$label), x$x, x$y, :
неизвестна ширина символа 0xe2 в кодировке CP1251
\end{verbatim}

\begin{verbatim}
Warning in grid.Call(C_textBounds, as.graphicsAnnot(x$label), x$x, x$y, :
неизвестна ширина символа 0xeb в кодировке CP1251
\end{verbatim}

\begin{verbatim}
Warning in grid.Call(C_textBounds, as.graphicsAnnot(x$label), x$x, x$y, :
неизвестна ширина символа 0xe5 в кодировке CP1251
\end{verbatim}

\begin{verbatim}
Warning in grid.Call(C_textBounds, as.graphicsAnnot(x$label), x$x, x$y, :
неизвестна ширина символа 0xed в кодировке CP1251
\end{verbatim}

\begin{verbatim}
Warning in grid.Call(C_textBounds, as.graphicsAnnot(x$label), x$x, x$y, :
неизвестна ширина символа 0xe8 в кодировке CP1251
\end{verbatim}

\begin{verbatim}
Warning in grid.Call(C_textBounds, as.graphicsAnnot(x$label), x$x, x$y, :
неизвестна ширина символа 0xff в кодировке CP1251
\end{verbatim}

\begin{verbatim}
Warning in grid.Call(C_textBounds, as.graphicsAnnot(x$label), x$x, x$y, :
неизвестна ширина символа 0xd3 в кодировке CP1251
\end{verbatim}

\begin{verbatim}
Warning in grid.Call(C_textBounds, as.graphicsAnnot(x$label), x$x, x$y, :
неизвестна ширина символа 0xf1 в кодировке CP1251
\end{verbatim}

\begin{verbatim}
Warning in grid.Call(C_textBounds, as.graphicsAnnot(x$label), x$x, x$y, :
неизвестна ширина символа 0xf2 в кодировке CP1251
\end{verbatim}

\begin{verbatim}
Warning in grid.Call(C_textBounds, as.graphicsAnnot(x$label), x$x, x$y, :
неизвестна ширина символа 0xee в кодировке CP1251
\end{verbatim}

\begin{verbatim}
Warning in grid.Call(C_textBounds, as.graphicsAnnot(x$label), x$x, x$y, :
неизвестна ширина символа 0xe9 в кодировке CP1251
\end{verbatim}

\begin{verbatim}
Warning in grid.Call(C_textBounds, as.graphicsAnnot(x$label), x$x, x$y, :
неизвестна ширина символа 0xf7 в кодировке CP1251
\end{verbatim}

\begin{verbatim}
Warning in grid.Call(C_textBounds, as.graphicsAnnot(x$label), x$x, x$y, :
неизвестна ширина символа 0xe8 в кодировке CP1251
\end{verbatim}

\begin{verbatim}
Warning in grid.Call(C_textBounds, as.graphicsAnnot(x$label), x$x, x$y, :
неизвестна ширина символа 0xe2 в кодировке CP1251
\end{verbatim}

\begin{verbatim}
Warning in grid.Call(C_textBounds, as.graphicsAnnot(x$label), x$x, x$y, :
неизвестна ширина символа 0xfb в кодировке CP1251
\end{verbatim}

\begin{verbatim}
Warning in grid.Call(C_textBounds, as.graphicsAnnot(x$label), x$x, x$y, :
неизвестна ширина символа 0xe9 в кодировке CP1251
\end{verbatim}

\begin{verbatim}
Warning in grid.Call(C_textBounds, as.graphicsAnnot(x$label), x$x, x$y, :
неизвестна ширина символа 0xef в кодировке CP1251
\end{verbatim}

\begin{verbatim}
Warning in grid.Call(C_textBounds, as.graphicsAnnot(x$label), x$x, x$y, :
неизвестна ширина символа 0xf0 в кодировке CP1251
\end{verbatim}

\begin{verbatim}
Warning in grid.Call(C_textBounds, as.graphicsAnnot(x$label), x$x, x$y, :
неизвестна ширина символа 0xee в кодировке CP1251
\end{verbatim}

\begin{verbatim}
Warning in grid.Call(C_textBounds, as.graphicsAnnot(x$label), x$x, x$y, :
неизвестна ширина символа 0xec в кодировке CP1251
\end{verbatim}

\begin{verbatim}
Warning in grid.Call(C_textBounds, as.graphicsAnnot(x$label), x$x, x$y, :
неизвестна ширина символа 0xfb в кодировке CP1251
\end{verbatim}

\begin{verbatim}
Warning in grid.Call(C_textBounds, as.graphicsAnnot(x$label), x$x, x$y, :
неизвестна ширина символа 0xf1 в кодировке CP1251
\end{verbatim}

\begin{verbatim}
Warning in grid.Call(C_textBounds, as.graphicsAnnot(x$label), x$x, x$y, :
неизвестна ширина символа 0xe5 в кодировке CP1251
\end{verbatim}

\begin{verbatim}
Warning in grid.Call(C_textBounds, as.graphicsAnnot(x$label), x$x, x$y, :
неизвестна ширина символа 0xeb в кодировке CP1251
\end{verbatim}

\begin{verbatim}
Warning in grid.Call(C_textBounds, as.graphicsAnnot(x$label), x$x, x$y, :
неизвестна ширина символа 0xc3 в кодировке CP1251
\end{verbatim}

\begin{verbatim}
Warning in grid.Call(C_textBounds, as.graphicsAnnot(x$label), x$x, x$y, :
неизвестна ширина символа 0xee в кодировке CP1251
\end{verbatim}

\begin{verbatim}
Warning in grid.Call(C_textBounds, as.graphicsAnnot(x$label), x$x, x$y, :
неизвестна ширина символа 0xe4 в кодировке CP1251
\end{verbatim}

\begin{verbatim}
Warning in grid.Call.graphics(C_text, as.graphicsAnnot(x$label), x$x, x$y, :
неизвестна ширина символа 0xc3 в кодировке CP1251
\end{verbatim}

\begin{verbatim}
Warning in grid.Call.graphics(C_text, as.graphicsAnnot(x$label), x$x, x$y, :
неизвестна ширина символа 0xee в кодировке CP1251
\end{verbatim}

\begin{verbatim}
Warning in grid.Call.graphics(C_text, as.graphicsAnnot(x$label), x$x, x$y, :
неизвестна ширина символа 0xe4 в кодировке CP1251
\end{verbatim}

\begin{verbatim}
Warning in grid.Call.graphics(C_text, as.graphicsAnnot(x$label), x$x, x$y, :
неизвестна ширина символа 0xce в кодировке CP1251
\end{verbatim}

\begin{verbatim}
Warning in grid.Call.graphics(C_text, as.graphicsAnnot(x$label), x$x, x$y, :
неизвестна ширина символа 0xc4 в кодировке CP1251
\end{verbatim}

\begin{verbatim}
Warning in grid.Call.graphics(C_text, as.graphicsAnnot(x$label), x$x, x$y, :
неизвестна ширина символа 0xd3 в кодировке CP1251
\end{verbatim}

\begin{verbatim}
Warning in grid.Call.graphics(C_text, as.graphicsAnnot(x$label), x$x, x$y, :
неизвестна ширина символа 0xf2 в кодировке CP1251
\end{verbatim}

\begin{verbatim}
Warning in grid.Call.graphics(C_text, as.graphicsAnnot(x$label), x$x, x$y, :
неизвестна ширина символа 0xfb в кодировке CP1251
\end{verbatim}

\begin{verbatim}
Warning in grid.Call.graphics(C_text, as.graphicsAnnot(x$label), x$x, x$y, :
неизвестна ширина символа 0xf1 в кодировке CP1251
\end{verbatim}

\begin{verbatim}
Warning in grid.Call.graphics(C_text, as.graphicsAnnot(x$label), x$x, x$y, :
неизвестна ширина символа 0xf2 в кодировке CP1251
\end{verbatim}

\begin{verbatim}
Warning in grid.Call.graphics(C_text, as.graphicsAnnot(x$label), x$x, x$y, :
неизвестна ширина символа 0xcf в кодировке CP1251
\end{verbatim}

\begin{verbatim}
Warning in grid.Call.graphics(C_text, as.graphicsAnnot(x$label), x$x, x$y, :
неизвестна ширина символа 0xeb в кодировке CP1251
\end{verbatim}

\begin{verbatim}
Warning in grid.Call.graphics(C_text, as.graphicsAnnot(x$label), x$x, x$y, :
неизвестна ширина символа 0xe0 в кодировке CP1251
\end{verbatim}

\begin{verbatim}
Warning in grid.Call.graphics(C_text, as.graphicsAnnot(x$label), x$x, x$y, :
неизвестна ширина символа 0xed в кодировке CP1251
\end{verbatim}

\begin{verbatim}
Warning in grid.Call.graphics(C_text, as.graphicsAnnot(x$label), x$x, x$y, :
неизвестна ширина символа 0xf3 в кодировке CP1251
\end{verbatim}

\begin{verbatim}
Warning in grid.Call.graphics(C_text, as.graphicsAnnot(x$label), x$x, x$y, :
неизвестна ширина символа 0xef в кодировке CP1251
\end{verbatim}

\begin{verbatim}
Warning in grid.Call.graphics(C_text, as.graphicsAnnot(x$label), x$x, x$y, :
неизвестна ширина символа 0xf0 в кодировке CP1251
\end{verbatim}

\begin{verbatim}
Warning in grid.Call.graphics(C_text, as.graphicsAnnot(x$label), x$x, x$y, :
неизвестна ширина символа 0xe0 в кодировке CP1251
\end{verbatim}

\begin{verbatim}
Warning in grid.Call.graphics(C_text, as.graphicsAnnot(x$label), x$x, x$y, :
неизвестна ширина символа 0xe2 в кодировке CP1251
\end{verbatim}

\begin{verbatim}
Warning in grid.Call.graphics(C_text, as.graphicsAnnot(x$label), x$x, x$y, :
неизвестна ширина символа 0xeb в кодировке CP1251
\end{verbatim}

\begin{verbatim}
Warning in grid.Call.graphics(C_text, as.graphicsAnnot(x$label), x$x, x$y, :
неизвестна ширина символа 0xe5 в кодировке CP1251
\end{verbatim}

\begin{verbatim}
Warning in grid.Call.graphics(C_text, as.graphicsAnnot(x$label), x$x, x$y, :
неизвестна ширина символа 0xed в кодировке CP1251
\end{verbatim}

\begin{verbatim}
Warning in grid.Call.graphics(C_text, as.graphicsAnnot(x$label), x$x, x$y, :
неизвестна ширина символа 0xe8 в кодировке CP1251
\end{verbatim}

\begin{verbatim}
Warning in grid.Call.graphics(C_text, as.graphicsAnnot(x$label), x$x, x$y, :
неизвестна ширина символа 0xff в кодировке CP1251
\end{verbatim}

\begin{verbatim}
Warning in grid.Call.graphics(C_text, as.graphicsAnnot(x$label), x$x, x$y, :
неизвестна ширина символа 0xd3 в кодировке CP1251
\end{verbatim}

\begin{verbatim}
Warning in grid.Call.graphics(C_text, as.graphicsAnnot(x$label), x$x, x$y, :
неизвестна ширина символа 0xf1 в кодировке CP1251
\end{verbatim}

\begin{verbatim}
Warning in grid.Call.graphics(C_text, as.graphicsAnnot(x$label), x$x, x$y, :
неизвестна ширина символа 0xf2 в кодировке CP1251
\end{verbatim}

\begin{verbatim}
Warning in grid.Call.graphics(C_text, as.graphicsAnnot(x$label), x$x, x$y, :
неизвестна ширина символа 0xee в кодировке CP1251
\end{verbatim}

\begin{verbatim}
Warning in grid.Call.graphics(C_text, as.graphicsAnnot(x$label), x$x, x$y, :
неизвестна ширина символа 0xe9 в кодировке CP1251
\end{verbatim}

\begin{verbatim}
Warning in grid.Call.graphics(C_text, as.graphicsAnnot(x$label), x$x, x$y, :
неизвестна ширина символа 0xf7 в кодировке CP1251
\end{verbatim}

\begin{verbatim}
Warning in grid.Call.graphics(C_text, as.graphicsAnnot(x$label), x$x, x$y, :
неизвестна ширина символа 0xe8 в кодировке CP1251
\end{verbatim}

\begin{verbatim}
Warning in grid.Call.graphics(C_text, as.graphicsAnnot(x$label), x$x, x$y, :
неизвестна ширина символа 0xe2 в кодировке CP1251
\end{verbatim}

\begin{verbatim}
Warning in grid.Call.graphics(C_text, as.graphicsAnnot(x$label), x$x, x$y, :
неизвестна ширина символа 0xfb в кодировке CP1251
\end{verbatim}

\begin{verbatim}
Warning in grid.Call.graphics(C_text, as.graphicsAnnot(x$label), x$x, x$y, :
неизвестна ширина символа 0xe9 в кодировке CP1251
\end{verbatim}

\begin{verbatim}
Warning in grid.Call.graphics(C_text, as.graphicsAnnot(x$label), x$x, x$y, :
неизвестна ширина символа 0xef в кодировке CP1251
\end{verbatim}

\begin{verbatim}
Warning in grid.Call.graphics(C_text, as.graphicsAnnot(x$label), x$x, x$y, :
неизвестна ширина символа 0xf0 в кодировке CP1251
\end{verbatim}

\begin{verbatim}
Warning in grid.Call.graphics(C_text, as.graphicsAnnot(x$label), x$x, x$y, :
неизвестна ширина символа 0xee в кодировке CP1251
\end{verbatim}

\begin{verbatim}
Warning in grid.Call.graphics(C_text, as.graphicsAnnot(x$label), x$x, x$y, :
неизвестна ширина символа 0xec в кодировке CP1251
\end{verbatim}

\begin{verbatim}
Warning in grid.Call.graphics(C_text, as.graphicsAnnot(x$label), x$x, x$y, :
неизвестна ширина символа 0xfb в кодировке CP1251
\end{verbatim}

\begin{verbatim}
Warning in grid.Call.graphics(C_text, as.graphicsAnnot(x$label), x$x, x$y, :
неизвестна ширина символа 0xf1 в кодировке CP1251
\end{verbatim}

\begin{verbatim}
Warning in grid.Call.graphics(C_text, as.graphicsAnnot(x$label), x$x, x$y, :
неизвестна ширина символа 0xe5 в кодировке CP1251
\end{verbatim}

\begin{verbatim}
Warning in grid.Call.graphics(C_text, as.graphicsAnnot(x$label), x$x, x$y, :
неизвестна ширина символа 0xeb в кодировке CP1251
\end{verbatim}

\pandocbounded{\includegraphics[keepaspectratio]{chapter14_files/figure-pdf/unnamed-chunk-1-4.pdf}}

\begin{Shaded}
\begin{Highlighting}[]
\CommentTok{\# {-}{-}{-}{-}{-}{-}{-}{-}{-}{-}{-}{-}{-}{-}{-}{-}{-}{-}{-} 10. ЭКСПОРТ РЕЗУЛЬТАТОВ {-}{-}{-}{-}{-}{-}{-}{-}{-}{-}{-}{-}{-}{-}{-}{-}{-}{-}{-}{-}}

\CommentTok{\#cat("\textbackslash{}n========== СОХРАНЕНИЕ РЕЗУЛЬТАТОВ ==========\textbackslash{}n")}

\DocumentationTok{\#\# 10.1 Сохранение всех графиков}
\CommentTok{\#pdf("HCR\_analysis.pdf", width = 12, height = 8)}
\CommentTok{\#print(p\_scenarios)}
\CommentTok{\#print(p\_hockey)}
\CommentTok{\#print(p\_comparison)}
\CommentTok{\#print(p\_plan)}
\CommentTok{\#dev.off()}
\CommentTok{\#cat("Графики сохранены в \textquotesingle{}HCR\_analysis.pdf\textquotesingle{}\textbackslash{}n")}

\DocumentationTok{\#\# 10.2 Экспорт таблицы с рекомендациями}
\NormalTok{recommendations }\OtherTok{\textless{}{-}} \FunctionTok{data.frame}\NormalTok{(}
  \AttributeTok{Rule =} \FunctionTok{c}\NormalTok{(}\StringTok{"Hockey{-}stick"}\NormalTok{, }\StringTok{"ICES"}\NormalTok{, }\StringTok{"40{-}10"}\NormalTok{, }\StringTok{"Current F"}\NormalTok{, }\StringTok{"Fmsy"}\NormalTok{),}
  \AttributeTok{F\_recommended =} \FunctionTok{c}\NormalTok{(}
    \FunctionTok{hockey\_stick\_HCR}\NormalTok{(B\_current, Bmsy, Fmsy),}
\NormalTok{    ices\_result}\SpecialCharTok{$}\NormalTok{F\_advice,}
\NormalTok{    result\_40\_10}\SpecialCharTok{$}\NormalTok{F\_advice,}
    \FunctionTok{get.par}\NormalTok{(}\StringTok{"logF"}\NormalTok{, fit, }\AttributeTok{exp =} \ConstantTok{TRUE}\NormalTok{)[}\DecValTok{1}\NormalTok{],}
\NormalTok{    Fmsy}
\NormalTok{  ),}
  \AttributeTok{TAC\_recommended =} \ConstantTok{NA}\NormalTok{,}
  \AttributeTok{Status =} \FunctionTok{c}\NormalTok{(}
    \StringTok{"MSY{-}based"}\NormalTok{,}
\NormalTok{    ices\_result}\SpecialCharTok{$}\NormalTok{status,}
\NormalTok{    result\_40\_10}\SpecialCharTok{$}\NormalTok{status,}
    \StringTok{"Status quo"}\NormalTok{,}
    \StringTok{"Optimal"}
\NormalTok{  )}
\NormalTok{)}

\NormalTok{recommendations}\SpecialCharTok{$}\NormalTok{TAC\_recommended }\OtherTok{\textless{}{-}}\NormalTok{ recommendations}\SpecialCharTok{$}\NormalTok{F\_recommended }\SpecialCharTok{*}\NormalTok{ B\_current}

\FunctionTok{write.csv}\NormalTok{(recommendations, }\StringTok{"TAC\_recommendations.csv"}\NormalTok{, }\AttributeTok{row.names =} \ConstantTok{FALSE}\NormalTok{)}
\FunctionTok{cat}\NormalTok{(}\StringTok{"Рекомендации сохранены в \textquotesingle{}TAC\_recommendations.csv\textquotesingle{}}\SpecialCharTok{\textbackslash{}n}\StringTok{"}\NormalTok{)}
\end{Highlighting}
\end{Shaded}

\begin{verbatim}
Рекомендации сохранены в 'TAC_recommendations.csv'
\end{verbatim}

\begin{Shaded}
\begin{Highlighting}[]
\DocumentationTok{\#\# 10.3 Создание итогового отчета}
\FunctionTok{sink}\NormalTok{(}\StringTok{"HCR\_report.txt"}\NormalTok{)}
\FunctionTok{cat}\NormalTok{(}\StringTok{"ОТЧЕТ ПО ПРАВИЛАМ УПРАВЛЕНИЯ И ОПРЕДЕЛЕНИЮ ОДУ}\SpecialCharTok{\textbackslash{}n}\StringTok{"}\NormalTok{)}
\FunctionTok{cat}\NormalTok{(}\StringTok{"="}\NormalTok{ , }\FunctionTok{strrep}\NormalTok{(}\StringTok{"="}\NormalTok{, }\DecValTok{50}\NormalTok{), }\StringTok{"}\SpecialCharTok{\textbackslash{}n}\StringTok{"}\NormalTok{)}
\FunctionTok{cat}\NormalTok{(}\StringTok{"Дата:"}\NormalTok{, }\FunctionTok{format}\NormalTok{(}\FunctionTok{Sys.Date}\NormalTok{(), }\StringTok{"\%d.\%m.\%Y"}\NormalTok{), }\StringTok{"}\SpecialCharTok{\textbackslash{}n\textbackslash{}n}\StringTok{"}\NormalTok{)}

\FunctionTok{cat}\NormalTok{(}\StringTok{"ТЕКУЩЕЕ СОСТОЯНИЕ ЗАПАСА:}\SpecialCharTok{\textbackslash{}n}\StringTok{"}\NormalTok{)}
\FunctionTok{cat}\NormalTok{(}\StringTok{"Биомасса:"}\NormalTok{, }\FunctionTok{round}\NormalTok{(B\_current, }\DecValTok{1}\NormalTok{), }\StringTok{"тыс. т}\SpecialCharTok{\textbackslash{}n}\StringTok{"}\NormalTok{)}
\FunctionTok{cat}\NormalTok{(}\StringTok{"B/Bmsy:"}\NormalTok{, }\FunctionTok{round}\NormalTok{(B\_current}\SpecialCharTok{/}\NormalTok{Bmsy, }\DecValTok{2}\NormalTok{), }\StringTok{"}\SpecialCharTok{\textbackslash{}n}\StringTok{"}\NormalTok{)}
\FunctionTok{cat}\NormalTok{(}\StringTok{"F/Fmsy:"}\NormalTok{, }\FunctionTok{round}\NormalTok{(}\FunctionTok{get.par}\NormalTok{(}\StringTok{"logF"}\NormalTok{, fit, }\AttributeTok{exp =} \ConstantTok{TRUE}\NormalTok{)[}\DecValTok{1}\NormalTok{]}\SpecialCharTok{/}\NormalTok{Fmsy, }\DecValTok{2}\NormalTok{), }\StringTok{"}\SpecialCharTok{\textbackslash{}n\textbackslash{}n}\StringTok{"}\NormalTok{)}

\FunctionTok{cat}\NormalTok{(}\StringTok{"РЕКОМЕНДАЦИИ ПО ОДУ:}\SpecialCharTok{\textbackslash{}n}\StringTok{"}\NormalTok{)}
\FunctionTok{print}\NormalTok{(recommendations, }\AttributeTok{digits =} \DecValTok{2}\NormalTok{)}

\FunctionTok{cat}\NormalTok{(}\StringTok{"}\SpecialCharTok{\textbackslash{}n\textbackslash{}n}\StringTok{ПЯТИЛЕТНИЙ ПЛАН:}\SpecialCharTok{\textbackslash{}n}\StringTok{"}\NormalTok{)}
\FunctionTok{print}\NormalTok{(management\_plan, }\AttributeTok{digits =} \DecValTok{2}\NormalTok{)}
\FunctionTok{sink}\NormalTok{()}

\FunctionTok{cat}\NormalTok{(}\StringTok{"Отчет сохранен в \textquotesingle{}HCR\_report.txt\textquotesingle{}}\SpecialCharTok{\textbackslash{}n}\StringTok{"}\NormalTok{)}
\end{Highlighting}
\end{Shaded}

\begin{verbatim}
Отчет сохранен в 'HCR_report.txt'
\end{verbatim}

\begin{Shaded}
\begin{Highlighting}[]
\FunctionTok{cat}\NormalTok{(}\StringTok{"}\SpecialCharTok{\textbackslash{}n}\StringTok{=============== АНАЛИЗ ЗАВЕРШЕН ===============}\SpecialCharTok{\textbackslash{}n}\StringTok{"}\NormalTok{)}
\end{Highlighting}
\end{Shaded}

\begin{verbatim}

=============== АНАЛИЗ ЗАВЕРШЕН ===============
\end{verbatim}

\section{Результаты применения правил управления: анализ второго
скрипта}\label{ux440ux435ux437ux443ux43bux44cux442ux430ux442ux44b-ux43fux440ux438ux43cux435ux43dux435ux43dux438ux44f-ux43fux440ux430ux432ux438ux43b-ux443ux43fux440ux430ux432ux43bux435ux43dux438ux44f-ux430ux43dux430ux43bux438ux437-ux432ux442ux43eux440ux43eux433ux43e-ux441ux43aux440ux438ux43fux442ux430}

Выполнение второго скрипта наглядно демонстрирует, как теоретические
правила управления превращаются в конкретные цифры вылова. Анализ восьми
сценариев показывает существенный разброс в возможных последствиях
управленческих решений.

Правило статус-кво (``Сохранить текущий уровень вылова'') предсказывает
улов 11.8 тыс. тонн при относительной биомассе 1.31 B/Bmsy. Любопытно,
что этот консервативный подход, вопреки ожиданиям, показывает не лучшие
результаты по стабильности промысла. Широкие доверительные интервалы
{[}1.12-1.52{]} для B/Bmsy свидетельствуют о значительной
неопределенности.

Сценарий ``Fish at Fmsy'' дает максимальный предсказанный вылов --- 22.0
тыс. тонн, но достигается это ценой снижения биомассы до 1.20 B/Bmsy.
Примечательно, что доверительный интервал для F/Fmsy {[}0.45-2.23{]}
превышает целевой уровень, что указывает на риск непреднамеренного
перелова.

Полное закрытие промысла, как и ожидалось, приводит к восстановлению
запаса до 1.42 B/Bmsy, но с очевидными социально-экономическими
последствиями. Этот сценарий служит скорее теоретическим ориентиром, чем
практическим решением.

Корректирующие правила показывают предсказуемую динамику: снижение F на
25\% уменьшает вылов до 9.5 тыс. тонн при росте биомассы до 1.33 B/Bmsy,
тогда увеличение F на 25\% дает вылов 15.4 тыс. тонн, но снижает
биомассу до 1.27 B/Bmsy.

Наиболее интересны современные адаптивные правила. MSY hockey-stick
демонстрирует тот же вылов, что и Fmsy подход (22.0 тыс. тонн), но с
лучшими показателями стабильности. Правило ICES показывает разумный
компромисс --- вылов 19.8 тыс. тонн при биомассе 1.23 B/Bmsy и
значительно более узких доверительных интервалах.

Анализ доверительных интервалов заслуживает отдельного внимания. Широкие
интервалы для всех сценариев (особенно для F/Fmsy) подчеркивают
фундаментальную неопределенность в управлении рыболовством. Как мог бы
заметить Довлатов, ``рыболовство --- это искусство принимать решения на
основе неточных данных с точными последствиями''.

Результаты наглядно показывают, что не существует идеального правила ---
каждый выбор представляет собой компромисс между величиной вылова,
стабильностью промысла и риском для запаса. Практическая ценность такого
анализа заключается в возможности количественно оценить последствия
каждого управленческого решения до его реализации.

\bookmarksetup{startatroot}

\chapter{III. SPiCT: MSE - оценка стратегии
управления}\label{iii.-spict-mse---ux43eux446ux435ux43dux43aux430-ux441ux442ux440ux430ux442ux435ux433ux438ux438-ux443ux43fux440ux430ux432ux43bux435ux43dux438ux44f}

\section{Концепция Management Strategy Evaluation (MSE): Искусство
управления в условиях
неопределенности}\label{ux43aux43eux43dux446ux435ux43fux446ux438ux44f-management-strategy-evaluation-mse-ux438ux441ux43aux443ux441ux441ux442ux432ux43e-ux443ux43fux440ux430ux432ux43bux435ux43dux438ux44f-ux432-ux443ux441ux43bux43eux432ux438ux44fux445-ux43dux435ux43eux43fux440ux435ux434ux435ux43bux435ux43dux43dux43eux441ux442ux438}

Если SPiCT --- это гипотеза о состоянии запаса, а HCR --- рецепт, то MSE
--- испытательный стенд, на котором рецепт готовят в турбулентной кухне
реальности. Это «аэродинамическая труба» для управления: мы создаём
правдоподобный мир (операционную модель), пропускаем через него шум
наблюдений, оценку, задержки внедрения, ошибки исполнения --- и смотрим,
как ведут себя разные стратегии в тысячах альтернативных будущих. Не
чтобы найти «идеально оптимальную» --- её нет, --- а чтобы выбрать
наименее хрупкую.

Из чего состоит MSE.

- Операционная модель (OM): «истинная» динамика запаса и флота, с
вариабельностью рекрутинга, автокорреляцией, возможным дрейфом
уловистости, изменчивостью среды.

- Наблюдательная модель: как рождаются данные (ошибки индексов, задержки
отчётности, недоучёт).

- Оценка (например, SPiCT): периодическая, с неизбежными смещениями и
сходимостями.

- Правило управления (HCR): превращает оценку в совет по \emph{F}/ОДУ, с
ограничителями и буферами.

- Исполнение: реализация с ошибками (implementation error), реакция
флота, соблюдение ограничений, лаги решений.

Что считаем успехом.

- Биология: вероятность \emph{B}\textless{}\emph{B\textsubscript{lim}},
доля времени в «зелёной зоне», средний
\emph{B}/\emph{B\textsubscript{msy}}.

- Экономика/социальная стабильность: средний вылов, межгодовая
изменчивость ОДУ, доля «закрытых» лет, длительность восстановления.

- Робастность: чувствительность к «что‑если» (дрейф \emph{q},
структурная ошибка модели, неблагоприятные серии лет со слабым
пополнением).

- Компромисс: фронт Парето вместо одного числа; подход «минимакс
сожаления» --- минимум проигрыша в плохих сценариях.

Чем MSE спасает от иллюзий.

- Мы перестаём путать модель с реальностью: оценки систематически
ошибаются, и это встроено в эксперименты.

- Широкие сценарии важнее глубины одной «идеальной» модели: узкие
допущения делают стратегию хрупкой.

- Поведение людей имеет значение: соблюдение квот, перераспределение
усилия, экономические стимулы --- часть петли обратной связи.

Что делаем в этой главе.

- Формулируем цели и метрики: явно, измеримо, с порогами риска.

- Строим набор OM: несколько правдоподобных «миров» (варианты \emph{r,}
продуктивности, пополнения, \emph{q}‑дрейфа, среды).

- Задаём наблюдательный и оценочный контур (в т. ч. SPiCT), календарь
пересмотров, лаги и ошибки исполнения.

- Сравниваем HCR‑кандидаты, настраиваем их параметры под целевые риски,
считаем фронт компромиссов.

- Проводим стресс‑тесты: «плохие годы подряд», резкий технологический
скачок, смещение индекса, неполное соблюдение.

- Готовим «панель управления» для решений: таблицы производительности,
графики риска/вылова/стабильности, понятные без кода.

Границы и дисциплина. MSE не отвечает на вопрос «что истина?» --- он
отвечает «что работает чаще и не ломается в плохую погоду». Он требует
прозрачного кода, фиксированных зерён случайности и версионирования
сценариев. И, как хорошие инженерные мосты, лучшие MSE‑решения реже
всего «самые умные» --- они просто переживают больше штормов.

Полный скрипт можно скачать по
\href{https://mombus.github.io/cRab/data/SPICT_MSE.R}{ссылке}. Ниже
приводится исполнение скрипта.

\begin{Shaded}
\begin{Highlighting}[]
\CommentTok{\# ===============================================================}
\CommentTok{\#     СКРИПТ 3: MANAGEMENT STRATEGY EVALUATION (MSE){-} ОЦЕНКА СТРАТЕГИИ УПРАВЛЕНИЯ}
\CommentTok{\#     Сравнение трех ключевых стратегий управления}
\CommentTok{\#     Курс: Оценка водных биоресурсов при недостатке данных в R}
\CommentTok{\#     Автор: Баканёв С.В.}
\CommentTok{\#     Дата создания: 28.08.2025}
\CommentTok{\# ===============================================================}

\CommentTok{\# ======================= ВВЕДЕНИЕ =============================}
\CommentTok{\# В этом скрипте сравниваются три основные стратегии управления:}
\CommentTok{\# 1. Fish at Fmsy {-} промысел на уровне оптимальной смертности}
\CommentTok{\# 2. MSY hockey{-}stick rule {-} адаптивное правило с защитой запаса}
\CommentTok{\# 3. ICES advice rule {-} предосторожный подход ICES}
\CommentTok{\#}
\CommentTok{\# MSE позволяет тестировать эти стратегии с учетом всех}
\CommentTok{\# источников неопределенности в системе промысел{-}запас}

\CommentTok{\# {-}{-}{-}{-}{-}{-}{-}{-}{-}{-}{-}{-}{-}{-}{-}{-}{-}{-}{-} 1. ПОДГОТОВКА СРЕДЫ {-}{-}{-}{-}{-}{-}{-}{-}{-}{-}{-}{-}{-}{-}{-}{-}{-}{-}{-}{-}}

\DocumentationTok{\#\# 1.1 Очистка рабочей среды и настройка}
\FunctionTok{rm}\NormalTok{(}\AttributeTok{list =} \FunctionTok{ls}\NormalTok{())}
\FunctionTok{set.seed}\NormalTok{(}\DecValTok{123}\NormalTok{)  }\CommentTok{\# Для воспроизводимости результатов}

\DocumentationTok{\#\# 1.2 Загрузка необходимых библиотек}
\FunctionTok{library}\NormalTok{(spict)       }\CommentTok{\# Для работы с моделью SPiCT}
\end{Highlighting}
\end{Shaded}

\begin{verbatim}
Загрузка требуемого пакета: TMB
\end{verbatim}

\begin{verbatim}
Welcome to spict_v1.3.8@107a32
\end{verbatim}

\begin{Shaded}
\begin{Highlighting}[]
\FunctionTok{library}\NormalTok{(tidyverse)   }\CommentTok{\# Обработка данных}
\end{Highlighting}
\end{Shaded}

\begin{verbatim}
-- Attaching core tidyverse packages ------------------------ tidyverse 2.0.0 --
v dplyr     1.1.4     v readr     2.1.5
v forcats   1.0.0     v stringr   1.5.2
v ggplot2   4.0.0     v tibble    3.2.1
v lubridate 1.9.4     v tidyr     1.3.1
v purrr     1.0.4     
\end{verbatim}

\begin{verbatim}
-- Conflicts ------------------------------------------ tidyverse_conflicts() --
x dplyr::filter() masks stats::filter()
x dplyr::lag()    masks stats::lag()
i Use the conflicted package (<http://conflicted.r-lib.org/>) to force all conflicts to become errors
\end{verbatim}

\begin{Shaded}
\begin{Highlighting}[]
\FunctionTok{library}\NormalTok{(ggplot2)     }\CommentTok{\# Визуализация}
\FunctionTok{library}\NormalTok{(viridis)     }\CommentTok{\# Цветовые схемы}
\end{Highlighting}
\end{Shaded}

\begin{verbatim}
Загрузка требуемого пакета: viridisLite
\end{verbatim}

\begin{Shaded}
\begin{Highlighting}[]
\FunctionTok{library}\NormalTok{(patchwork)   }\CommentTok{\# Компоновка графиков}
\FunctionTok{library}\NormalTok{(gridExtra)   }\CommentTok{\# Дополнительные возможности компоновки}
\end{Highlighting}
\end{Shaded}

\begin{verbatim}

Присоединяю пакет: 'gridExtra'

Следующий объект скрыт от 'package:dplyr':

    combine
\end{verbatim}

\begin{Shaded}
\begin{Highlighting}[]
\DocumentationTok{\#\# 1.3 Установка рабочей директории и загрузка модели}
\FunctionTok{setwd}\NormalTok{(}\StringTok{"C:/SPICT"}\NormalTok{)}

\CommentTok{\# Загружаем подогнанную модель из первого скрипта}
\ControlFlowTok{if}\NormalTok{ (}\FunctionTok{file.exists}\NormalTok{(}\StringTok{"spict\_model\_fit.rds"}\NormalTok{)) \{}
\NormalTok{  fit }\OtherTok{\textless{}{-}} \FunctionTok{readRDS}\NormalTok{(}\StringTok{"spict\_model\_fit.rds"}\NormalTok{)}
  \FunctionTok{cat}\NormalTok{(}\StringTok{"}\SpecialCharTok{\textbackslash{}n}\StringTok{✓ Модель SPiCT успешно загружена}\SpecialCharTok{\textbackslash{}n}\StringTok{"}\NormalTok{)}
\NormalTok{\} }\ControlFlowTok{else}\NormalTok{ \{}
  \FunctionTok{stop}\NormalTok{(}\StringTok{"Файл модели не найден. Запустите первый скрипт для подгонки модели."}\NormalTok{)}
\NormalTok{\}}
\end{Highlighting}
\end{Shaded}

\begin{verbatim}

<U+2713> Модель SPiCT успешно загружена
\end{verbatim}

\begin{Shaded}
\begin{Highlighting}[]
\FunctionTok{cat}\NormalTok{(}\StringTok{"}\SpecialCharTok{\textbackslash{}n}\StringTok{"}\NormalTok{ , }\FunctionTok{strrep}\NormalTok{(}\StringTok{"="}\NormalTok{, }\DecValTok{60}\NormalTok{), }\StringTok{"}\SpecialCharTok{\textbackslash{}n}\StringTok{"}\NormalTok{)}
\end{Highlighting}
\end{Shaded}

\begin{verbatim}

 ============================================================ 
\end{verbatim}

\begin{Shaded}
\begin{Highlighting}[]
\FunctionTok{cat}\NormalTok{(}\StringTok{"      MANAGEMENT STRATEGY EVALUATION (MSE)}\SpecialCharTok{\textbackslash{}n}\StringTok{"}\NormalTok{)}
\end{Highlighting}
\end{Shaded}

\begin{verbatim}
      MANAGEMENT STRATEGY EVALUATION (MSE)
\end{verbatim}

\begin{Shaded}
\begin{Highlighting}[]
\FunctionTok{cat}\NormalTok{(}\StringTok{"   Сравнение трех стратегий управления промыслом}\SpecialCharTok{\textbackslash{}n}\StringTok{"}\NormalTok{)}
\end{Highlighting}
\end{Shaded}

\begin{verbatim}
   Сравнение трех стратегий управления промыслом
\end{verbatim}

\begin{Shaded}
\begin{Highlighting}[]
\FunctionTok{cat}\NormalTok{(}\FunctionTok{strrep}\NormalTok{(}\StringTok{"="}\NormalTok{, }\DecValTok{60}\NormalTok{), }\StringTok{"}\SpecialCharTok{\textbackslash{}n}\StringTok{"}\NormalTok{)}
\end{Highlighting}
\end{Shaded}

\begin{verbatim}
============================================================ 
\end{verbatim}

\begin{Shaded}
\begin{Highlighting}[]
\CommentTok{\# {-}{-}{-}{-}{-}{-}{-}{-}{-}{-}{-}{-}{-}{-}{-}{-}{-}{-}{-} 2. ИЗВЛЕЧЕНИЕ ПАРАМЕТРОВ ИЗ МОДЕЛИ {-}{-}{-}{-}{-}{-}{-}{-}{-}{-}{-}{-}{-}{-}{-}{-}{-}{-}{-}{-}}

\FunctionTok{cat}\NormalTok{(}\StringTok{"}\SpecialCharTok{\textbackslash{}n}\StringTok{========== ПАРАМЕТРЫ ОПЕРАЦИОННОЙ МОДЕЛИ ==========}\SpecialCharTok{\textbackslash{}n}\StringTok{"}\NormalTok{)}
\end{Highlighting}
\end{Shaded}

\begin{verbatim}

========== ПАРАМЕТРЫ ОПЕРАЦИОННОЙ МОДЕЛИ ==========
\end{verbatim}

\begin{Shaded}
\begin{Highlighting}[]
\DocumentationTok{\#\# 2.1 Извлечение оценок параметров популяционной динамики}
\CommentTok{\# Эти параметры представляют "истинное" состояние в операционной модели}
\NormalTok{r\_true }\OtherTok{\textless{}{-}} \FunctionTok{get.par}\NormalTok{(}\StringTok{"logr"}\NormalTok{, fit, }\AttributeTok{exp =} \ConstantTok{TRUE}\NormalTok{)[}\DecValTok{1}\NormalTok{]          }\CommentTok{\# Внутренний темп роста}
\NormalTok{K\_true }\OtherTok{\textless{}{-}} \FunctionTok{get.par}\NormalTok{(}\StringTok{"logK"}\NormalTok{, fit, }\AttributeTok{exp =} \ConstantTok{TRUE}\NormalTok{)[}\DecValTok{1}\NormalTok{]          }\CommentTok{\# Ёмкость среды}
\NormalTok{Bmsy\_true }\OtherTok{\textless{}{-}} \FunctionTok{get.par}\NormalTok{(}\StringTok{"logBmsy"}\NormalTok{, fit, }\AttributeTok{exp =} \ConstantTok{TRUE}\NormalTok{)[}\DecValTok{1}\NormalTok{]    }\CommentTok{\# Биомасса MSY}
\NormalTok{Fmsy\_true }\OtherTok{\textless{}{-}} \FunctionTok{get.par}\NormalTok{(}\StringTok{"logFmsy"}\NormalTok{, fit, }\AttributeTok{exp =} \ConstantTok{TRUE}\NormalTok{)[}\DecValTok{1}\NormalTok{]    }\CommentTok{\# Промысловая смертность MSY}
\NormalTok{B\_current }\OtherTok{\textless{}{-}} \FunctionTok{get.par}\NormalTok{(}\StringTok{"logB"}\NormalTok{, fit, }\AttributeTok{exp =} \ConstantTok{TRUE}\NormalTok{)[}\DecValTok{1}\NormalTok{]       }\CommentTok{\# Текущая биомасса}
\NormalTok{F\_current }\OtherTok{\textless{}{-}} \FunctionTok{get.par}\NormalTok{(}\StringTok{"logF"}\NormalTok{, fit, }\AttributeTok{exp =} \ConstantTok{TRUE}\NormalTok{)[}\DecValTok{1}\NormalTok{]       }\CommentTok{\# Текущая пром. смертность}

\CommentTok{\# Параметр улавливаемости для индекса CPUE}
\NormalTok{q\_cpue }\OtherTok{\textless{}{-}} \FunctionTok{get.par}\NormalTok{(}\StringTok{"logq"}\NormalTok{, fit, }\AttributeTok{exp =} \ConstantTok{TRUE}\NormalTok{)[}\DecValTok{1}\NormalTok{]}

\DocumentationTok{\#\# 2.2 Вывод ключевых параметров}
\FunctionTok{cat}\NormalTok{(}\StringTok{"}\SpecialCharTok{\textbackslash{}n}\StringTok{Извлеченные параметры модели:}\SpecialCharTok{\textbackslash{}n}\StringTok{"}\NormalTok{)}
\end{Highlighting}
\end{Shaded}

\begin{verbatim}

Извлеченные параметры модели:
\end{verbatim}

\begin{Shaded}
\begin{Highlighting}[]
\FunctionTok{cat}\NormalTok{(}\FunctionTok{sprintf}\NormalTok{(}\StringTok{"r (темп роста): \%.3f год⁻¹}\SpecialCharTok{\textbackslash{}n}\StringTok{"}\NormalTok{, r\_true))}
\end{Highlighting}
\end{Shaded}

\begin{verbatim}
r (темп роста): 0.289 год<U+207B><U+00B9>
\end{verbatim}

\begin{Shaded}
\begin{Highlighting}[]
\FunctionTok{cat}\NormalTok{(}\FunctionTok{sprintf}\NormalTok{(}\StringTok{"K (ёмкость средыь): \%.1f тыс. т}\SpecialCharTok{\textbackslash{}n}\StringTok{"}\NormalTok{, K\_true))}
\end{Highlighting}
\end{Shaded}

\begin{verbatim}
K (ёмкость средыь): 153.8 тыс. т
\end{verbatim}

\begin{Shaded}
\begin{Highlighting}[]
\FunctionTok{cat}\NormalTok{(}\FunctionTok{sprintf}\NormalTok{(}\StringTok{"Bmsy: \%.1f тыс. т}\SpecialCharTok{\textbackslash{}n}\StringTok{"}\NormalTok{, Bmsy\_true))}
\end{Highlighting}
\end{Shaded}

\begin{verbatim}
Bmsy: 76.9 тыс. т
\end{verbatim}

\begin{Shaded}
\begin{Highlighting}[]
\FunctionTok{cat}\NormalTok{(}\FunctionTok{sprintf}\NormalTok{(}\StringTok{"Fmsy: \%.3f год⁻¹}\SpecialCharTok{\textbackslash{}n}\StringTok{"}\NormalTok{, Fmsy\_true))}
\end{Highlighting}
\end{Shaded}

\begin{verbatim}
Fmsy: 0.145 год<U+207B><U+00B9>
\end{verbatim}

\begin{Shaded}
\begin{Highlighting}[]
\FunctionTok{cat}\NormalTok{(}\FunctionTok{sprintf}\NormalTok{(}\StringTok{"Текущая биомасса: \%.1f тыс. т}\SpecialCharTok{\textbackslash{}n}\StringTok{"}\NormalTok{, B\_current))}
\end{Highlighting}
\end{Shaded}

\begin{verbatim}
Текущая биомасса: 103.0 тыс. т
\end{verbatim}

\begin{Shaded}
\begin{Highlighting}[]
\FunctionTok{cat}\NormalTok{(}\FunctionTok{sprintf}\NormalTok{(}\StringTok{"Текущее B/Bmsy: \%.2f}\SpecialCharTok{\textbackslash{}n}\StringTok{"}\NormalTok{, B\_current}\SpecialCharTok{/}\NormalTok{Bmsy\_true))}
\end{Highlighting}
\end{Shaded}

\begin{verbatim}
Текущее B/Bmsy: 1.34
\end{verbatim}

\begin{Shaded}
\begin{Highlighting}[]
\FunctionTok{cat}\NormalTok{(}\FunctionTok{sprintf}\NormalTok{(}\StringTok{"Текущее F/Fmsy: \%.2f}\SpecialCharTok{\textbackslash{}n}\StringTok{"}\NormalTok{, F\_current}\SpecialCharTok{/}\NormalTok{Fmsy\_true))}
\end{Highlighting}
\end{Shaded}

\begin{verbatim}
Текущее F/Fmsy: 0.13
\end{verbatim}

\begin{Shaded}
\begin{Highlighting}[]
\CommentTok{\# {-}{-}{-}{-}{-}{-}{-}{-}{-}{-}{-}{-}{-}{-}{-}{-}{-}{-}{-} 3. НАСТРОЙКИ СИМУЛЯЦИИ {-}{-}{-}{-}{-}{-}{-}{-}{-}{-}{-}{-}{-}{-}{-}{-}{-}{-}{-}{-}}

\FunctionTok{cat}\NormalTok{(}\StringTok{"}\SpecialCharTok{\textbackslash{}n}\StringTok{========== НАСТРОЙКИ MSE ==========}\SpecialCharTok{\textbackslash{}n}\StringTok{"}\NormalTok{)}
\end{Highlighting}
\end{Shaded}

\begin{verbatim}

========== НАСТРОЙКИ MSE ==========
\end{verbatim}

\begin{Shaded}
\begin{Highlighting}[]
\DocumentationTok{\#\# 3.1 Основные параметры симуляции}
\NormalTok{n\_sim }\OtherTok{\textless{}{-}} \DecValTok{500}              \CommentTok{\# Количество симуляций (реализаций)}
\NormalTok{n\_years }\OtherTok{\textless{}{-}} \DecValTok{100}             \CommentTok{\# Период прогнозирования (лет)}
\NormalTok{assessment\_interval }\OtherTok{\textless{}{-}} \DecValTok{2}  \CommentTok{\# Интервал между оценками запаса (лет)}

\FunctionTok{cat}\NormalTok{(}\FunctionTok{sprintf}\NormalTok{(}\StringTok{"Количество симуляций: \%d}\SpecialCharTok{\textbackslash{}n}\StringTok{"}\NormalTok{, n\_sim))}
\end{Highlighting}
\end{Shaded}

\begin{verbatim}
Количество симуляций: 500
\end{verbatim}

\begin{Shaded}
\begin{Highlighting}[]
\FunctionTok{cat}\NormalTok{(}\FunctionTok{sprintf}\NormalTok{(}\StringTok{"Горизонт прогнозирования: \%d лет}\SpecialCharTok{\textbackslash{}n}\StringTok{"}\NormalTok{, n\_years))}
\end{Highlighting}
\end{Shaded}

\begin{verbatim}
Горизонт прогнозирования: 100 лет
\end{verbatim}

\begin{Shaded}
\begin{Highlighting}[]
\FunctionTok{cat}\NormalTok{(}\FunctionTok{sprintf}\NormalTok{(}\StringTok{"Частота оценки запаса: каждые \%d года}\SpecialCharTok{\textbackslash{}n}\StringTok{"}\NormalTok{, assessment\_interval))}
\end{Highlighting}
\end{Shaded}

\begin{verbatim}
Частота оценки запаса: каждые 2 года
\end{verbatim}

\begin{Shaded}
\begin{Highlighting}[]
\DocumentationTok{\#\# 3.2 Параметры неопределенности}
\CommentTok{\# Все источники неопределенности в системе}

\CommentTok{\# Процессная ошибка {-} естественная изменчивость в динамике популяции}
\NormalTok{process\_error\_cv }\OtherTok{\textless{}{-}} \FloatTok{0.15}  

\CommentTok{\# Ошибка наблюдения {-} неточность в индексах биомассы}
\NormalTok{observation\_error\_cv }\OtherTok{\textless{}{-}} \FloatTok{0.20}  

\CommentTok{\# Ошибка оценки {-} неточность в оценке состояния запаса}
\NormalTok{assessment\_bias\_cv }\OtherTok{\textless{}{-}} \FloatTok{0.15}  

\CommentTok{\# Ошибка реализации {-} разница между рекомендованным и фактическим выловом}
\NormalTok{implementation\_error\_cv }\OtherTok{\textless{}{-}} \FloatTok{0.10}  

\FunctionTok{cat}\NormalTok{(}\StringTok{"}\SpecialCharTok{\textbackslash{}n}\StringTok{Параметры неопределенности:}\SpecialCharTok{\textbackslash{}n}\StringTok{"}\NormalTok{)}
\end{Highlighting}
\end{Shaded}

\begin{verbatim}

Параметры неопределенности:
\end{verbatim}

\begin{Shaded}
\begin{Highlighting}[]
\FunctionTok{cat}\NormalTok{(}\FunctionTok{sprintf}\NormalTok{(}\StringTok{"CV процессной ошибки: \%.0f\%\%}\SpecialCharTok{\textbackslash{}n}\StringTok{"}\NormalTok{, process\_error\_cv }\SpecialCharTok{*} \DecValTok{100}\NormalTok{))}
\end{Highlighting}
\end{Shaded}

\begin{verbatim}
CV процессной ошибки: 15%
\end{verbatim}

\begin{Shaded}
\begin{Highlighting}[]
\FunctionTok{cat}\NormalTok{(}\FunctionTok{sprintf}\NormalTok{(}\StringTok{"CV ошибки наблюдения: \%.0f\%\%}\SpecialCharTok{\textbackslash{}n}\StringTok{"}\NormalTok{, observation\_error\_cv }\SpecialCharTok{*} \DecValTok{100}\NormalTok{))}
\end{Highlighting}
\end{Shaded}

\begin{verbatim}
CV ошибки наблюдения: 20%
\end{verbatim}

\begin{Shaded}
\begin{Highlighting}[]
\FunctionTok{cat}\NormalTok{(}\FunctionTok{sprintf}\NormalTok{(}\StringTok{"CV смещения в оценках: \%.0f\%\%}\SpecialCharTok{\textbackslash{}n}\StringTok{"}\NormalTok{, assessment\_bias\_cv }\SpecialCharTok{*} \DecValTok{100}\NormalTok{))}
\end{Highlighting}
\end{Shaded}

\begin{verbatim}
CV смещения в оценках: 15%
\end{verbatim}

\begin{Shaded}
\begin{Highlighting}[]
\FunctionTok{cat}\NormalTok{(}\FunctionTok{sprintf}\NormalTok{(}\StringTok{"CV ошибки реализации: \%.0f\%\%}\SpecialCharTok{\textbackslash{}n}\StringTok{"}\NormalTok{, implementation\_error\_cv }\SpecialCharTok{*} \DecValTok{100}\NormalTok{))}
\end{Highlighting}
\end{Shaded}

\begin{verbatim}
CV ошибки реализации: 10%
\end{verbatim}

\begin{Shaded}
\begin{Highlighting}[]
\CommentTok{\# {-}{-}{-}{-}{-}{-}{-}{-}{-}{-}{-}{-}{-}{-}{-}{-}{-}{-}{-} 4. ОПЕРАЦИОННАЯ МОДЕЛЬ {-}{-}{-}{-}{-}{-}{-}{-}{-}{-}{-}{-}{-}{-}{-}{-}{-}{-}{-}{-}}

\FunctionTok{cat}\NormalTok{(}\StringTok{"}\SpecialCharTok{\textbackslash{}n}\StringTok{========== ОПРЕДЕЛЕНИЕ ОПЕРАЦИОННОЙ МОДЕЛИ ==========}\SpecialCharTok{\textbackslash{}n}\StringTok{"}\NormalTok{)}
\end{Highlighting}
\end{Shaded}

\begin{verbatim}

========== ОПРЕДЕЛЕНИЕ ОПЕРАЦИОННОЙ МОДЕЛИ ==========
\end{verbatim}

\begin{Shaded}
\begin{Highlighting}[]
\DocumentationTok{\#\# 4.1 Функция истинной динамики популяции}
\CommentTok{\# Модель Шефера с процессной стохастичностью}
\NormalTok{simulate\_population\_dynamics }\OtherTok{\textless{}{-}} \ControlFlowTok{function}\NormalTok{(B\_t, F\_t, r, K, }\AttributeTok{process\_cv =} \FloatTok{0.15}\NormalTok{) \{}
  
  \CommentTok{\# Расчет прибавочной продукции (модель Шефера)}
\NormalTok{  surplus\_production }\OtherTok{\textless{}{-}}\NormalTok{ r }\SpecialCharTok{*}\NormalTok{ B\_t }\SpecialCharTok{*}\NormalTok{ (}\DecValTok{1} \SpecialCharTok{{-}}\NormalTok{ B\_t}\SpecialCharTok{/}\NormalTok{K)}
  
  \CommentTok{\# Расчет вылова}
\NormalTok{  catch }\OtherTok{\textless{}{-}}\NormalTok{ F\_t }\SpecialCharTok{*}\NormalTok{ B\_t}
  
  \CommentTok{\# Обновление биомассы с учетом процессной ошибки}
  \CommentTok{\# Используем лог{-}нормальное распределение для мультипликативной ошибки}
\NormalTok{  process\_error }\OtherTok{\textless{}{-}} \FunctionTok{rlnorm}\NormalTok{(}\DecValTok{1}\NormalTok{, }\AttributeTok{meanlog =} \SpecialCharTok{{-}}\NormalTok{process\_cv}\SpecialCharTok{\^{}}\DecValTok{2}\SpecialCharTok{/}\DecValTok{2}\NormalTok{, }\AttributeTok{sdlog =}\NormalTok{ process\_cv)}
  
\NormalTok{  B\_next }\OtherTok{\textless{}{-}}\NormalTok{ (B\_t }\SpecialCharTok{+}\NormalTok{ surplus\_production }\SpecialCharTok{{-}}\NormalTok{ catch) }\SpecialCharTok{*}\NormalTok{ process\_error}
  
  \CommentTok{\# Ограничения для реалистичности}
\NormalTok{  B\_next }\OtherTok{\textless{}{-}} \FunctionTok{max}\NormalTok{(B\_next, }\FloatTok{0.001} \SpecialCharTok{*}\NormalTok{ K)  }\CommentTok{\# Минимум 0.1\% от K}
\NormalTok{  B\_next }\OtherTok{\textless{}{-}} \FunctionTok{min}\NormalTok{(B\_next, }\FloatTok{1.5} \SpecialCharTok{*}\NormalTok{ K)    }\CommentTok{\# Максимум 150\% от K}
  
  \FunctionTok{return}\NormalTok{(}\FunctionTok{list}\NormalTok{(}
    \AttributeTok{B\_next =}\NormalTok{ B\_next,}
    \AttributeTok{catch\_realized =}\NormalTok{ catch,}
    \AttributeTok{surplus =}\NormalTok{ surplus\_production,}
    \AttributeTok{process\_multiplier =}\NormalTok{ process\_error}
\NormalTok{  ))}
\NormalTok{\}}

\DocumentationTok{\#\# 4.2 Функция генерации наблюдаемого индекса}
\CommentTok{\# Имитирует процесс сбора данных с ошибками}
\NormalTok{generate\_index\_observation }\OtherTok{\textless{}{-}} \ControlFlowTok{function}\NormalTok{(B\_true, q, }\AttributeTok{obs\_cv =} \FloatTok{0.20}\NormalTok{) \{}
  
  \CommentTok{\# Истинный индекс пропорционален биомассе}
\NormalTok{  true\_index }\OtherTok{\textless{}{-}}\NormalTok{ q }\SpecialCharTok{*}\NormalTok{ B\_true}
  
  \CommentTok{\# Добавление ошибки наблюдения}
\NormalTok{  obs\_error }\OtherTok{\textless{}{-}} \FunctionTok{rlnorm}\NormalTok{(}\DecValTok{1}\NormalTok{, }\AttributeTok{meanlog =} \SpecialCharTok{{-}}\NormalTok{obs\_cv}\SpecialCharTok{\^{}}\DecValTok{2}\SpecialCharTok{/}\DecValTok{2}\NormalTok{, }\AttributeTok{sdlog =}\NormalTok{ obs\_cv)}
\NormalTok{  observed\_index }\OtherTok{\textless{}{-}}\NormalTok{ true\_index }\SpecialCharTok{*}\NormalTok{ obs\_error}
  
  \FunctionTok{return}\NormalTok{(observed\_index)}
\NormalTok{\}}

\DocumentationTok{\#\# 4.3 Функция оценки состояния запаса}
\CommentTok{\# Упрощенная процедура оценки (в реальности здесь бы запускалась полная модель)}
\NormalTok{assess\_stock\_status }\OtherTok{\textless{}{-}} \ControlFlowTok{function}\NormalTok{(index\_history, catch\_history, }
\NormalTok{                               true\_q, true\_Bmsy, true\_Fmsy,}
                               \AttributeTok{assessment\_cv =} \FloatTok{0.15}\NormalTok{) \{}
  
  \CommentTok{\# Оценка текущей биомассы по последним наблюдениям индекса}
  \CommentTok{\# Используем среднее за последние 3 года для сглаживания}
\NormalTok{  recent\_indices }\OtherTok{\textless{}{-}} \FunctionTok{tail}\NormalTok{(index\_history[}\SpecialCharTok{!}\FunctionTok{is.na}\NormalTok{(index\_history)], }\DecValTok{3}\NormalTok{)}
\NormalTok{  mean\_recent\_index }\OtherTok{\textless{}{-}} \FunctionTok{mean}\NormalTok{(recent\_indices, }\AttributeTok{na.rm =} \ConstantTok{TRUE}\NormalTok{)}
  
  \CommentTok{\# Оценка биомассы с учетом смещения}
\NormalTok{  assessment\_bias }\OtherTok{\textless{}{-}} \FunctionTok{rlnorm}\NormalTok{(}\DecValTok{1}\NormalTok{, }\AttributeTok{meanlog =} \SpecialCharTok{{-}}\NormalTok{assessment\_cv}\SpecialCharTok{\^{}}\DecValTok{2}\SpecialCharTok{/}\DecValTok{2}\NormalTok{, }\AttributeTok{sdlog =}\NormalTok{ assessment\_cv)}
\NormalTok{  B\_estimated }\OtherTok{\textless{}{-}}\NormalTok{ (mean\_recent\_index }\SpecialCharTok{/}\NormalTok{ true\_q) }\SpecialCharTok{*}\NormalTok{ assessment\_bias}
  
  \CommentTok{\# Оценка текущей F из последнего вылова}
\NormalTok{  recent\_catch }\OtherTok{\textless{}{-}} \FunctionTok{tail}\NormalTok{(catch\_history[catch\_history }\SpecialCharTok{\textgreater{}} \DecValTok{0}\NormalTok{], }\DecValTok{1}\NormalTok{)}
\NormalTok{  F\_estimated }\OtherTok{\textless{}{-}} \FunctionTok{ifelse}\NormalTok{(}\FunctionTok{length}\NormalTok{(recent\_catch) }\SpecialCharTok{\textgreater{}} \DecValTok{0} \SpecialCharTok{\&\&}\NormalTok{ B\_estimated }\SpecialCharTok{\textgreater{}} \DecValTok{0}\NormalTok{,}
\NormalTok{                        recent\_catch }\SpecialCharTok{/}\NormalTok{ B\_estimated,}
                        \FloatTok{0.1}\NormalTok{)  }\CommentTok{\# Значение по умолчанию}
  
  \CommentTok{\# Оценки референсных точек также содержат неопределенность}
\NormalTok{  Bmsy\_estimated }\OtherTok{\textless{}{-}}\NormalTok{ true\_Bmsy }\SpecialCharTok{*} \FunctionTok{rlnorm}\NormalTok{(}\DecValTok{1}\NormalTok{, }\DecValTok{0}\NormalTok{, assessment\_cv}\SpecialCharTok{/}\DecValTok{2}\NormalTok{)}
\NormalTok{  Fmsy\_estimated }\OtherTok{\textless{}{-}}\NormalTok{ true\_Fmsy }\SpecialCharTok{*} \FunctionTok{rlnorm}\NormalTok{(}\DecValTok{1}\NormalTok{, }\DecValTok{0}\NormalTok{, assessment\_cv}\SpecialCharTok{/}\DecValTok{2}\NormalTok{)}
  
  \FunctionTok{return}\NormalTok{(}\FunctionTok{list}\NormalTok{(}
    \AttributeTok{B =}\NormalTok{ B\_estimated,}
    \AttributeTok{F =}\NormalTok{ F\_estimated,}
    \AttributeTok{Bmsy =}\NormalTok{ Bmsy\_estimated,}
    \AttributeTok{Fmsy =}\NormalTok{ Fmsy\_estimated,}
    \AttributeTok{B\_Bmsy =}\NormalTok{ B\_estimated }\SpecialCharTok{/}\NormalTok{ Bmsy\_estimated,}
    \AttributeTok{F\_Fmsy =}\NormalTok{ F\_estimated }\SpecialCharTok{/}\NormalTok{ Fmsy\_estimated}
\NormalTok{  ))}
\NormalTok{\}}

\CommentTok{\# {-}{-}{-}{-}{-}{-}{-}{-}{-}{-}{-}{-}{-}{-}{-}{-}{-}{-}{-} 5. ПРАВИЛА УПРАВЛЕНИЯ (HCR) {-}{-}{-}{-}{-}{-}{-}{-}{-}{-}{-}{-}{-}{-}{-}{-}{-}{-}{-}{-}}

\FunctionTok{cat}\NormalTok{(}\StringTok{"}\SpecialCharTok{\textbackslash{}n}\StringTok{========== ОПРЕДЕЛЕНИЕ ПРАВИЛ УПРАВЛЕНИЯ ==========}\SpecialCharTok{\textbackslash{}n}\StringTok{"}\NormalTok{)}
\end{Highlighting}
\end{Shaded}

\begin{verbatim}

========== ОПРЕДЕЛЕНИЕ ПРАВИЛ УПРАВЛЕНИЯ ==========
\end{verbatim}

\begin{Shaded}
\begin{Highlighting}[]
\DocumentationTok{\#\# 5.1 Стратегия 1: Fish at Fmsy}
\CommentTok{\# Простейшее правило {-} всегда промысел на уровне Fmsy}
\NormalTok{HCR\_Fmsy }\OtherTok{\textless{}{-}} \ControlFlowTok{function}\NormalTok{(assessment, }\AttributeTok{previous\_TAC =} \ConstantTok{NULL}\NormalTok{) \{}
  
  \CommentTok{\# Рекомендация по F}
\NormalTok{  F\_advice }\OtherTok{\textless{}{-}}\NormalTok{ assessment}\SpecialCharTok{$}\NormalTok{Fmsy}
  
  \CommentTok{\# Расчет TAC}
\NormalTok{  TAC }\OtherTok{\textless{}{-}}\NormalTok{ F\_advice }\SpecialCharTok{*}\NormalTok{ assessment}\SpecialCharTok{$}\NormalTok{B}
  
  \CommentTok{\# Обеспечиваем неотрицательность}
\NormalTok{  TAC }\OtherTok{\textless{}{-}} \FunctionTok{max}\NormalTok{(TAC, }\DecValTok{0}\NormalTok{)}
  
  \FunctionTok{return}\NormalTok{(}\FunctionTok{list}\NormalTok{(}
    \AttributeTok{F\_advice =}\NormalTok{ F\_advice,}
    \AttributeTok{TAC =}\NormalTok{ TAC,}
    \AttributeTok{rule\_name =} \StringTok{"Fish at Fmsy"}\NormalTok{,}
    \AttributeTok{status =} \FunctionTok{ifelse}\NormalTok{(assessment}\SpecialCharTok{$}\NormalTok{B\_Bmsy }\SpecialCharTok{\textless{}} \FloatTok{0.5}\NormalTok{, }\StringTok{"Риск истощения"}\NormalTok{, }\StringTok{"Стандартный промысел"}\NormalTok{)}
\NormalTok{  ))}
\NormalTok{\}}

\FunctionTok{cat}\NormalTok{(}\StringTok{"✓ Стратегия 1: Fish at Fmsy}\SpecialCharTok{\textbackslash{}n}\StringTok{"}\NormalTok{)}
\end{Highlighting}
\end{Shaded}

\begin{verbatim}
<U+2713> Стратегия 1: Fish at Fmsy
\end{verbatim}

\begin{Shaded}
\begin{Highlighting}[]
\FunctionTok{cat}\NormalTok{(}\StringTok{"  {-} Постоянный промысел на уровне Fmsy}\SpecialCharTok{\textbackslash{}n}\StringTok{"}\NormalTok{)}
\end{Highlighting}
\end{Shaded}

\begin{verbatim}
  - Постоянный промысел на уровне Fmsy
\end{verbatim}

\begin{Shaded}
\begin{Highlighting}[]
\FunctionTok{cat}\NormalTok{(}\StringTok{"  {-} Не учитывает состояние запаса}\SpecialCharTok{\textbackslash{}n}\StringTok{"}\NormalTok{)}
\end{Highlighting}
\end{Shaded}

\begin{verbatim}
  - Не учитывает состояние запаса
\end{verbatim}

\begin{Shaded}
\begin{Highlighting}[]
\FunctionTok{cat}\NormalTok{(}\StringTok{"  {-} Простое в применении правило}\SpecialCharTok{\textbackslash{}n}\StringTok{"}\NormalTok{)}
\end{Highlighting}
\end{Shaded}

\begin{verbatim}
  - Простое в применении правило
\end{verbatim}

\begin{Shaded}
\begin{Highlighting}[]
\DocumentationTok{\#\# 5.2 Стратегия 2: MSY Hockey{-}stick Rule}
\CommentTok{\# Адаптивное правило с защитой при низкой биомассе}
\NormalTok{HCR\_hockey\_stick }\OtherTok{\textless{}{-}} \ControlFlowTok{function}\NormalTok{(assessment, }\AttributeTok{previous\_TAC =} \ConstantTok{NULL}\NormalTok{,}
                            \AttributeTok{Blim\_fraction =} \FloatTok{0.5}\NormalTok{) \{}
  
  \CommentTok{\# Определение предельной биомассы}
\NormalTok{  Blim }\OtherTok{\textless{}{-}}\NormalTok{ assessment}\SpecialCharTok{$}\NormalTok{Bmsy }\SpecialCharTok{*}\NormalTok{ Blim\_fraction}
  
  \CommentTok{\# Применение правила hockey{-}stick}
  \ControlFlowTok{if}\NormalTok{ (assessment}\SpecialCharTok{$}\NormalTok{B }\SpecialCharTok{\textless{}=}\NormalTok{ Blim) \{}
    \CommentTok{\# Линейное снижение F при B \textless{} Blim}
\NormalTok{    F\_multiplier }\OtherTok{\textless{}{-}}\NormalTok{ assessment}\SpecialCharTok{$}\NormalTok{B }\SpecialCharTok{/}\NormalTok{ Blim}
\NormalTok{    F\_advice }\OtherTok{\textless{}{-}}\NormalTok{ assessment}\SpecialCharTok{$}\NormalTok{Fmsy }\SpecialCharTok{*}\NormalTok{ F\_multiplier}
\NormalTok{    status }\OtherTok{\textless{}{-}} \StringTok{"Снижение промысла"}
\NormalTok{  \} }\ControlFlowTok{else}\NormalTok{ \{}
    \CommentTok{\# Полный промысел при B \textgreater{}= Blim}
\NormalTok{    F\_advice }\OtherTok{\textless{}{-}}\NormalTok{ assessment}\SpecialCharTok{$}\NormalTok{Fmsy}
\NormalTok{    status }\OtherTok{\textless{}{-}} \StringTok{"Полный промысел"}
\NormalTok{  \}}
  
  \CommentTok{\# Расчет TAC}
\NormalTok{  TAC }\OtherTok{\textless{}{-}}\NormalTok{ F\_advice }\SpecialCharTok{*}\NormalTok{ assessment}\SpecialCharTok{$}\NormalTok{B}
\NormalTok{  TAC }\OtherTok{\textless{}{-}} \FunctionTok{max}\NormalTok{(TAC, }\DecValTok{0}\NormalTok{)}
  
  \CommentTok{\# Полное закрытие при критически низкой биомассе}
  \ControlFlowTok{if}\NormalTok{ (assessment}\SpecialCharTok{$}\NormalTok{B\_Bmsy }\SpecialCharTok{\textless{}} \FloatTok{0.2}\NormalTok{) \{}
\NormalTok{    TAC }\OtherTok{\textless{}{-}} \DecValTok{0}
\NormalTok{    F\_advice }\OtherTok{\textless{}{-}} \DecValTok{0}
\NormalTok{    status }\OtherTok{\textless{}{-}} \StringTok{"Промысел закрыт"}
\NormalTok{  \}}
  
  \FunctionTok{return}\NormalTok{(}\FunctionTok{list}\NormalTok{(}
    \AttributeTok{F\_advice =}\NormalTok{ F\_advice,}
    \AttributeTok{TAC =}\NormalTok{ TAC,}
    \AttributeTok{rule\_name =} \StringTok{"MSY Hockey{-}stick"}\NormalTok{,}
    \AttributeTok{status =}\NormalTok{ status,}
    \AttributeTok{Blim =}\NormalTok{ Blim}
\NormalTok{  ))}
\NormalTok{\}}

\FunctionTok{cat}\NormalTok{(}\StringTok{"}\SpecialCharTok{\textbackslash{}n}\StringTok{✓ Стратегия 2: MSY Hockey{-}stick Rule}\SpecialCharTok{\textbackslash{}n}\StringTok{"}\NormalTok{)}
\end{Highlighting}
\end{Shaded}

\begin{verbatim}

<U+2713> Стратегия 2: MSY Hockey-stick Rule
\end{verbatim}

\begin{Shaded}
\begin{Highlighting}[]
\FunctionTok{cat}\NormalTok{(}\StringTok{"  {-} Адаптивное управление в зависимости от состояния}\SpecialCharTok{\textbackslash{}n}\StringTok{"}\NormalTok{)}
\end{Highlighting}
\end{Shaded}

\begin{verbatim}
  - Адаптивное управление в зависимости от состояния
\end{verbatim}

\begin{Shaded}
\begin{Highlighting}[]
\FunctionTok{cat}\NormalTok{(}\StringTok{"  {-} Линейное снижение F при B \textless{} 0.5*Bmsy}\SpecialCharTok{\textbackslash{}n}\StringTok{"}\NormalTok{)}
\end{Highlighting}
\end{Shaded}

\begin{verbatim}
  - Линейное снижение F при B < 0.5*Bmsy
\end{verbatim}

\begin{Shaded}
\begin{Highlighting}[]
\FunctionTok{cat}\NormalTok{(}\StringTok{"  {-} Полное закрытие при B \textless{} 0.2*Bmsy}\SpecialCharTok{\textbackslash{}n}\StringTok{"}\NormalTok{)}
\end{Highlighting}
\end{Shaded}

\begin{verbatim}
  - Полное закрытие при B < 0.2*Bmsy
\end{verbatim}

\begin{Shaded}
\begin{Highlighting}[]
\DocumentationTok{\#\# 5.3 Стратегия 3: ICES Advice Rule}
\CommentTok{\# Предосторожный подход с ограничениями межгодовых изменений}
\NormalTok{HCR\_ICES }\OtherTok{\textless{}{-}} \ControlFlowTok{function}\NormalTok{(assessment, }\AttributeTok{previous\_TAC =} \ConstantTok{NULL}\NormalTok{,}
                    \AttributeTok{Bpa\_multiplier =} \FloatTok{1.4}\NormalTok{,}
                    \AttributeTok{Fpa\_multiplier =} \FloatTok{0.85}\NormalTok{,}
                    \AttributeTok{max\_TAC\_change =} \FloatTok{0.20}\NormalTok{) \{}
  
  \CommentTok{\# Расчет предосторожных ориентиров управления}
\NormalTok{  Bpa }\OtherTok{\textless{}{-}}\NormalTok{ assessment}\SpecialCharTok{$}\NormalTok{Bmsy }\SpecialCharTok{/}\NormalTok{ Bpa\_multiplier   }\CommentTok{\# Предосторожная биомасса}
\NormalTok{  Blim }\OtherTok{\textless{}{-}}\NormalTok{ Bpa }\SpecialCharTok{/} \FloatTok{1.4}                         \CommentTok{\# Предельная биомасса  }
\NormalTok{  Fpa }\OtherTok{\textless{}{-}}\NormalTok{ assessment}\SpecialCharTok{$}\NormalTok{Fmsy }\SpecialCharTok{*}\NormalTok{ Fpa\_multiplier   }\CommentTok{\# Предосторожная F}
  
  \CommentTok{\# Определение F по правилу ICES}
  \ControlFlowTok{if}\NormalTok{ (assessment}\SpecialCharTok{$}\NormalTok{B }\SpecialCharTok{\textless{}}\NormalTok{ Blim) \{}
    \CommentTok{\# Критическое состояние {-} закрытие промысла}
\NormalTok{    F\_advice }\OtherTok{\textless{}{-}} \DecValTok{0}
\NormalTok{    status }\OtherTok{\textless{}{-}} \StringTok{"Критическое {-} промысел закрыт"}
    
\NormalTok{  \} }\ControlFlowTok{else} \ControlFlowTok{if}\NormalTok{ (assessment}\SpecialCharTok{$}\NormalTok{B }\SpecialCharTok{\textgreater{}=}\NormalTok{ Blim }\SpecialCharTok{\&\&}\NormalTok{ assessment}\SpecialCharTok{$}\NormalTok{B }\SpecialCharTok{\textless{}}\NormalTok{ Bpa) \{}
    \CommentTok{\# Восстановление {-} пропорциональное снижение F}
\NormalTok{    F\_multiplier }\OtherTok{\textless{}{-}}\NormalTok{ (assessment}\SpecialCharTok{$}\NormalTok{B }\SpecialCharTok{{-}}\NormalTok{ Blim) }\SpecialCharTok{/}\NormalTok{ (Bpa }\SpecialCharTok{{-}}\NormalTok{ Blim)}
\NormalTok{    F\_advice }\OtherTok{\textless{}{-}}\NormalTok{ Fpa }\SpecialCharTok{*}\NormalTok{ F\_multiplier}
\NormalTok{    status }\OtherTok{\textless{}{-}} \StringTok{"Восстановление запаса"}
    
\NormalTok{  \} }\ControlFlowTok{else}\NormalTok{ \{}
    \CommentTok{\# Нормальное состояние {-} предосторожный промысел}
\NormalTok{    F\_advice }\OtherTok{\textless{}{-}}\NormalTok{ Fpa}
\NormalTok{    status }\OtherTok{\textless{}{-}} \StringTok{"Устойчивый промысел"}
\NormalTok{  \}}
  
  \CommentTok{\# Расчет TAC}
\NormalTok{  TAC }\OtherTok{\textless{}{-}}\NormalTok{ F\_advice }\SpecialCharTok{*}\NormalTok{ assessment}\SpecialCharTok{$}\NormalTok{B}
\NormalTok{  TAC }\OtherTok{\textless{}{-}} \FunctionTok{max}\NormalTok{(TAC, }\DecValTok{0}\NormalTok{)}
  
  \CommentTok{\# Ограничение межгодовых изменений TAC (стабильность для промышленности)}
  \ControlFlowTok{if}\NormalTok{ (}\SpecialCharTok{!}\FunctionTok{is.null}\NormalTok{(previous\_TAC) }\SpecialCharTok{\&\&}\NormalTok{ previous\_TAC }\SpecialCharTok{\textgreater{}} \DecValTok{0} \SpecialCharTok{\&\&}\NormalTok{ TAC }\SpecialCharTok{\textgreater{}} \DecValTok{0}\NormalTok{) \{}
\NormalTok{    max\_increase }\OtherTok{\textless{}{-}}\NormalTok{ previous\_TAC }\SpecialCharTok{*}\NormalTok{ (}\DecValTok{1} \SpecialCharTok{+}\NormalTok{ max\_TAC\_change)}
\NormalTok{    max\_decrease }\OtherTok{\textless{}{-}}\NormalTok{ previous\_TAC }\SpecialCharTok{*}\NormalTok{ (}\DecValTok{1} \SpecialCharTok{{-}}\NormalTok{ max\_TAC\_change)}
\NormalTok{    TAC\_constrained }\OtherTok{\textless{}{-}} \FunctionTok{min}\NormalTok{(}\FunctionTok{max}\NormalTok{(TAC, max\_decrease), max\_increase)}
    
    \ControlFlowTok{if}\NormalTok{ (TAC }\SpecialCharTok{!=}\NormalTok{ TAC\_constrained) \{}
\NormalTok{      status }\OtherTok{\textless{}{-}} \FunctionTok{paste}\NormalTok{(status, }\StringTok{"(TAC ограничен)"}\NormalTok{)}
\NormalTok{    \}}
\NormalTok{    TAC }\OtherTok{\textless{}{-}}\NormalTok{ TAC\_constrained}
\NormalTok{  \}}
  
  \FunctionTok{return}\NormalTok{(}\FunctionTok{list}\NormalTok{(}
    \AttributeTok{F\_advice =}\NormalTok{ F\_advice,}
    \AttributeTok{TAC =}\NormalTok{ TAC,}
    \AttributeTok{rule\_name =} \StringTok{"ICES Advice Rule"}\NormalTok{,}
    \AttributeTok{status =}\NormalTok{ status,}
    \AttributeTok{Bpa =}\NormalTok{ Bpa,}
    \AttributeTok{Blim =}\NormalTok{ Blim,}
    \AttributeTok{Fpa =}\NormalTok{ Fpa}
\NormalTok{  ))}
\NormalTok{\}}

\FunctionTok{cat}\NormalTok{(}\StringTok{"}\SpecialCharTok{\textbackslash{}n}\StringTok{✓ Стратегия 3: ICES Advice Rule}\SpecialCharTok{\textbackslash{}n}\StringTok{"}\NormalTok{)}
\end{Highlighting}
\end{Shaded}

\begin{verbatim}

<U+2713> Стратегия 3: ICES Advice Rule
\end{verbatim}

\begin{Shaded}
\begin{Highlighting}[]
\FunctionTok{cat}\NormalTok{(}\StringTok{"  {-} Предосторожный подход (Fpa = 0.85*Fmsy)}\SpecialCharTok{\textbackslash{}n}\StringTok{"}\NormalTok{)}
\end{Highlighting}
\end{Shaded}

\begin{verbatim}
  - Предосторожный подход (Fpa = 0.85*Fmsy)
\end{verbatim}

\begin{Shaded}
\begin{Highlighting}[]
\FunctionTok{cat}\NormalTok{(}\StringTok{"  {-} Двухуровневая система пороговых значений}\SpecialCharTok{\textbackslash{}n}\StringTok{"}\NormalTok{)}
\end{Highlighting}
\end{Shaded}

\begin{verbatim}
  - Двухуровневая система пороговых значений
\end{verbatim}

\begin{Shaded}
\begin{Highlighting}[]
\FunctionTok{cat}\NormalTok{(}\StringTok{"  {-} Ограничение межгодовых изменений TAC (±20\%)}\SpecialCharTok{\textbackslash{}n}\StringTok{"}\NormalTok{)}
\end{Highlighting}
\end{Shaded}

\begin{verbatim}
  - Ограничение межгодовых изменений TAC (±20%)
\end{verbatim}

\begin{Shaded}
\begin{Highlighting}[]
\CommentTok{\# {-}{-}{-}{-}{-}{-}{-}{-}{-}{-}{-}{-}{-}{-}{-}{-}{-}{-}{-} 6. ФУНКЦИЯ MSE ДЛЯ ОДНОЙ СИМУЛЯЦИИ {-}{-}{-}{-}{-}{-}{-}{-}{-}{-}{-}{-}{-}{-}{-}{-}{-}{-}{-}{-}}

\DocumentationTok{\#\# 6.1 Основная функция симуляции}
\NormalTok{run\_mse\_simulation }\OtherTok{\textless{}{-}} \ControlFlowTok{function}\NormalTok{(sim\_id, HCR\_function, params, settings) \{}
  
  \CommentTok{\# Инициализация с небольшой вариацией начальных условий}
\NormalTok{  B\_sim }\OtherTok{\textless{}{-}}\NormalTok{ params}\SpecialCharTok{$}\NormalTok{B\_initial }\SpecialCharTok{*} \FunctionTok{rlnorm}\NormalTok{(}\DecValTok{1}\NormalTok{, }\DecValTok{0}\NormalTok{, }\FloatTok{0.05}\NormalTok{)}
  
  \CommentTok{\# Вариация в "истинных" параметрах (представляет неопределенность в природе)}
\NormalTok{  r\_sim }\OtherTok{\textless{}{-}}\NormalTok{ params}\SpecialCharTok{$}\NormalTok{r }\SpecialCharTok{*} \FunctionTok{rlnorm}\NormalTok{(}\DecValTok{1}\NormalTok{, }\DecValTok{0}\NormalTok{, }\FloatTok{0.05}\NormalTok{)}
\NormalTok{  K\_sim }\OtherTok{\textless{}{-}}\NormalTok{ params}\SpecialCharTok{$}\NormalTok{K }\SpecialCharTok{*} \FunctionTok{rlnorm}\NormalTok{(}\DecValTok{1}\NormalTok{, }\DecValTok{0}\NormalTok{, }\FloatTok{0.05}\NormalTok{)}
  
  \CommentTok{\# Массивы для хранения результатов}
\NormalTok{  n\_years }\OtherTok{\textless{}{-}}\NormalTok{ settings}\SpecialCharTok{$}\NormalTok{n\_years}
\NormalTok{  results }\OtherTok{\textless{}{-}} \FunctionTok{data.frame}\NormalTok{(}
    \AttributeTok{sim\_id =}\NormalTok{ sim\_id,}
    \AttributeTok{year =} \DecValTok{1}\SpecialCharTok{:}\NormalTok{n\_years,}
    \AttributeTok{B\_true =} \FunctionTok{numeric}\NormalTok{(n\_years),}
    \AttributeTok{B\_estimated =} \FunctionTok{numeric}\NormalTok{(n\_years),}
    \AttributeTok{F\_advice =} \FunctionTok{numeric}\NormalTok{(n\_years),}
    \AttributeTok{F\_realized =} \FunctionTok{numeric}\NormalTok{(n\_years),}
    \AttributeTok{catch =} \FunctionTok{numeric}\NormalTok{(n\_years),}
    \AttributeTok{TAC =} \FunctionTok{numeric}\NormalTok{(n\_years),}
    \AttributeTok{index\_obs =} \FunctionTok{numeric}\NormalTok{(n\_years),}
    \AttributeTok{B\_Bmsy\_true =} \FunctionTok{numeric}\NormalTok{(n\_years),}
    \AttributeTok{F\_Fmsy\_true =} \FunctionTok{numeric}\NormalTok{(n\_years),}
    \AttributeTok{HCR\_status =} \FunctionTok{character}\NormalTok{(n\_years),}
    \AttributeTok{stringsAsFactors =} \ConstantTok{FALSE}
\NormalTok{  )}
  
  \CommentTok{\# История наблюдений для оценки}
\NormalTok{  index\_history }\OtherTok{\textless{}{-}} \FunctionTok{numeric}\NormalTok{()}
\NormalTok{  catch\_history }\OtherTok{\textless{}{-}} \FunctionTok{numeric}\NormalTok{()}
\NormalTok{  previous\_TAC }\OtherTok{\textless{}{-}} \ConstantTok{NULL}
\NormalTok{  last\_assessment }\OtherTok{\textless{}{-}} \ConstantTok{NULL}
  
  \CommentTok{\# Основной цикл симуляции по годам}
  \ControlFlowTok{for}\NormalTok{ (t }\ControlFlowTok{in} \DecValTok{1}\SpecialCharTok{:}\NormalTok{n\_years) \{}
    
    \CommentTok{\# 1. Генерация наблюдений (индекс биомассы)}
\NormalTok{    index\_obs }\OtherTok{\textless{}{-}} \FunctionTok{generate\_index\_observation}\NormalTok{(}
\NormalTok{      B\_sim, }
\NormalTok{      params}\SpecialCharTok{$}\NormalTok{q, }
\NormalTok{      settings}\SpecialCharTok{$}\NormalTok{observation\_error\_cv}
\NormalTok{    )}
\NormalTok{    index\_history }\OtherTok{\textless{}{-}} \FunctionTok{c}\NormalTok{(index\_history, index\_obs)}
    
    \CommentTok{\# 2. Оценка запаса (периодически или в первый год)}
    \ControlFlowTok{if}\NormalTok{ (t }\SpecialCharTok{==} \DecValTok{1} \SpecialCharTok{||}\NormalTok{ (t }\SpecialCharTok{{-}} \DecValTok{1}\NormalTok{) }\SpecialCharTok{\%\%}\NormalTok{ settings}\SpecialCharTok{$}\NormalTok{assessment\_interval }\SpecialCharTok{==} \DecValTok{0}\NormalTok{) \{}
      
\NormalTok{      last\_assessment }\OtherTok{\textless{}{-}} \FunctionTok{assess\_stock\_status}\NormalTok{(}
\NormalTok{        index\_history,}
\NormalTok{        catch\_history,}
\NormalTok{        params}\SpecialCharTok{$}\NormalTok{q,}
\NormalTok{        params}\SpecialCharTok{$}\NormalTok{Bmsy,}
\NormalTok{        params}\SpecialCharTok{$}\NormalTok{Fmsy,}
\NormalTok{        settings}\SpecialCharTok{$}\NormalTok{assessment\_bias\_cv}
\NormalTok{      )}
      
      \CommentTok{\# 3. Применение правила управления (HCR)}
\NormalTok{      advice }\OtherTok{\textless{}{-}} \FunctionTok{HCR\_function}\NormalTok{(last\_assessment, previous\_TAC)}
\NormalTok{      F\_advice }\OtherTok{\textless{}{-}}\NormalTok{ advice}\SpecialCharTok{$}\NormalTok{F\_advice}
\NormalTok{      TAC }\OtherTok{\textless{}{-}}\NormalTok{ advice}\SpecialCharTok{$}\NormalTok{TAC}
\NormalTok{      previous\_TAC }\OtherTok{\textless{}{-}}\NormalTok{ TAC}
\NormalTok{      HCR\_status }\OtherTok{\textless{}{-}}\NormalTok{ advice}\SpecialCharTok{$}\NormalTok{status}
      
\NormalTok{    \} }\ControlFlowTok{else}\NormalTok{ \{}
      \CommentTok{\# Используем прошлогодние рекомендации}
\NormalTok{      F\_advice }\OtherTok{\textless{}{-}}\NormalTok{ results}\SpecialCharTok{$}\NormalTok{F\_advice[t}\DecValTok{{-}1}\NormalTok{]}
\NormalTok{      TAC }\OtherTok{\textless{}{-}}\NormalTok{ results}\SpecialCharTok{$}\NormalTok{TAC[t}\DecValTok{{-}1}\NormalTok{]}
\NormalTok{      HCR\_status }\OtherTok{\textless{}{-}}\NormalTok{ results}\SpecialCharTok{$}\NormalTok{HCR\_status[t}\DecValTok{{-}1}\NormalTok{]}
\NormalTok{    \}}
    
    \CommentTok{\# 4. Реализация промысла с ошибкой implementation}
\NormalTok{    implementation\_error }\OtherTok{\textless{}{-}} \FunctionTok{rlnorm}\NormalTok{(}
      \DecValTok{1}\NormalTok{, }
      \AttributeTok{meanlog =} \SpecialCharTok{{-}}\NormalTok{settings}\SpecialCharTok{$}\NormalTok{implementation\_error\_cv}\SpecialCharTok{\^{}}\DecValTok{2}\SpecialCharTok{/}\DecValTok{2}\NormalTok{,}
      \AttributeTok{sdlog =}\NormalTok{ settings}\SpecialCharTok{$}\NormalTok{implementation\_error\_cv}
\NormalTok{    )}
    
\NormalTok{    F\_realized }\OtherTok{\textless{}{-}}\NormalTok{ F\_advice }\SpecialCharTok{*}\NormalTok{ implementation\_error}
\NormalTok{    F\_realized }\OtherTok{\textless{}{-}} \FunctionTok{min}\NormalTok{(F\_realized, }\FloatTok{2.0}\NormalTok{)  }\CommentTok{\# Ограничение максимальной F}
    
    \CommentTok{\# 5. Расчет динамики популяции}
\NormalTok{    pop\_update }\OtherTok{\textless{}{-}} \FunctionTok{simulate\_population\_dynamics}\NormalTok{(}
\NormalTok{      B\_sim, }
\NormalTok{      F\_realized,}
\NormalTok{      r\_sim,}
\NormalTok{      K\_sim,}
\NormalTok{      settings}\SpecialCharTok{$}\NormalTok{process\_error\_cv}
\NormalTok{    )}
    
    \CommentTok{\# 6. Сохранение результатов текущего года}
\NormalTok{    results}\SpecialCharTok{$}\NormalTok{B\_true[t] }\OtherTok{\textless{}{-}}\NormalTok{ B\_sim}
\NormalTok{    results}\SpecialCharTok{$}\NormalTok{B\_estimated[t] }\OtherTok{\textless{}{-}} \FunctionTok{ifelse}\NormalTok{(}\SpecialCharTok{!}\FunctionTok{is.null}\NormalTok{(last\_assessment), }
\NormalTok{                                     last\_assessment}\SpecialCharTok{$}\NormalTok{B, }\ConstantTok{NA}\NormalTok{)}
\NormalTok{    results}\SpecialCharTok{$}\NormalTok{F\_advice[t] }\OtherTok{\textless{}{-}}\NormalTok{ F\_advice}
\NormalTok{    results}\SpecialCharTok{$}\NormalTok{F\_realized[t] }\OtherTok{\textless{}{-}}\NormalTok{ F\_realized}
\NormalTok{    results}\SpecialCharTok{$}\NormalTok{catch[t] }\OtherTok{\textless{}{-}}\NormalTok{ pop\_update}\SpecialCharTok{$}\NormalTok{catch\_realized}
\NormalTok{    results}\SpecialCharTok{$}\NormalTok{TAC[t] }\OtherTok{\textless{}{-}}\NormalTok{ TAC}
\NormalTok{    results}\SpecialCharTok{$}\NormalTok{index\_obs[t] }\OtherTok{\textless{}{-}}\NormalTok{ index\_obs}
\NormalTok{    results}\SpecialCharTok{$}\NormalTok{B\_Bmsy\_true[t] }\OtherTok{\textless{}{-}}\NormalTok{ B\_sim }\SpecialCharTok{/}\NormalTok{ params}\SpecialCharTok{$}\NormalTok{Bmsy}
\NormalTok{    results}\SpecialCharTok{$}\NormalTok{F\_Fmsy\_true[t] }\OtherTok{\textless{}{-}}\NormalTok{ F\_realized }\SpecialCharTok{/}\NormalTok{ params}\SpecialCharTok{$}\NormalTok{Fmsy}
\NormalTok{    results}\SpecialCharTok{$}\NormalTok{HCR\_status[t] }\OtherTok{\textless{}{-}}\NormalTok{ HCR\_status}
    
    \CommentTok{\# 7. Обновление состояния для следующего года}
\NormalTok{    B\_sim }\OtherTok{\textless{}{-}}\NormalTok{ pop\_update}\SpecialCharTok{$}\NormalTok{B\_next}
\NormalTok{    catch\_history }\OtherTok{\textless{}{-}} \FunctionTok{c}\NormalTok{(catch\_history, pop\_update}\SpecialCharTok{$}\NormalTok{catch\_realized)}
\NormalTok{  \}}
  
  \FunctionTok{return}\NormalTok{(results)}
\NormalTok{\}}

\CommentTok{\# {-}{-}{-}{-}{-}{-}{-}{-}{-}{-}{-}{-}{-}{-}{-}{-}{-}{-}{-} 7. ЗАПУСК MSE ДЛЯ ВСЕХ СТРАТЕГИЙ {-}{-}{-}{-}{-}{-}{-}{-}{-}{-}{-}{-}{-}{-}{-}{-}{-}{-}{-}{-}}

\FunctionTok{cat}\NormalTok{(}\StringTok{"}\SpecialCharTok{\textbackslash{}n}\StringTok{========== ЗАПУСК СИМУЛЯЦИЙ MSE ==========}\SpecialCharTok{\textbackslash{}n}\StringTok{"}\NormalTok{)}
\end{Highlighting}
\end{Shaded}

\begin{verbatim}

========== ЗАПУСК СИМУЛЯЦИЙ MSE ==========
\end{verbatim}

\begin{Shaded}
\begin{Highlighting}[]
\DocumentationTok{\#\# 7.1 Подготовка параметров для симуляций}
\NormalTok{sim\_params }\OtherTok{\textless{}{-}} \FunctionTok{list}\NormalTok{(}
  \AttributeTok{r =}\NormalTok{ r\_true,}
  \AttributeTok{K =}\NormalTok{ K\_true,}
  \AttributeTok{Bmsy =}\NormalTok{ Bmsy\_true,}
  \AttributeTok{Fmsy =}\NormalTok{ Fmsy\_true,}
  \AttributeTok{B\_initial =}\NormalTok{ B\_current,}
  \AttributeTok{q =}\NormalTok{ q\_cpue}
\NormalTok{)}

\NormalTok{sim\_settings }\OtherTok{\textless{}{-}} \FunctionTok{list}\NormalTok{(}
  \AttributeTok{n\_years =}\NormalTok{ n\_years,}
  \AttributeTok{assessment\_interval =}\NormalTok{ assessment\_interval,}
  \AttributeTok{process\_error\_cv =}\NormalTok{ process\_error\_cv,}
  \AttributeTok{observation\_error\_cv =}\NormalTok{ observation\_error\_cv,}
  \AttributeTok{assessment\_bias\_cv =}\NormalTok{ assessment\_bias\_cv,}
  \AttributeTok{implementation\_error\_cv =}\NormalTok{ implementation\_error\_cv}
\NormalTok{)}

\DocumentationTok{\#\# 7.2 Список стратегий для сравнения}
\NormalTok{strategies }\OtherTok{\textless{}{-}} \FunctionTok{list}\NormalTok{(}
  \StringTok{"Fish at Fmsy"} \OtherTok{=}\NormalTok{ HCR\_Fmsy,}
  \StringTok{"MSY Hockey{-}stick"} \OtherTok{=}\NormalTok{ HCR\_hockey\_stick,}
  \StringTok{"ICES Advice Rule"} \OtherTok{=}\NormalTok{ HCR\_ICES}
\NormalTok{)}

\DocumentationTok{\#\# 7.3 Запуск симуляций для каждой стратегии}
\NormalTok{all\_results }\OtherTok{\textless{}{-}} \FunctionTok{list}\NormalTok{()}

\ControlFlowTok{for}\NormalTok{ (strategy\_name }\ControlFlowTok{in} \FunctionTok{names}\NormalTok{(strategies)) \{}
  
  \FunctionTok{cat}\NormalTok{(}\FunctionTok{sprintf}\NormalTok{(}\StringTok{"}\SpecialCharTok{\textbackslash{}n}\StringTok{Запуск стратегии: \%s}\SpecialCharTok{\textbackslash{}n}\StringTok{"}\NormalTok{, strategy\_name))}
  \FunctionTok{cat}\NormalTok{(}\StringTok{"Прогресс: "}\NormalTok{)}
  
  \CommentTok{\# Запуск n\_sim симуляций для текущей стратегии}
\NormalTok{  strategy\_results }\OtherTok{\textless{}{-}} \FunctionTok{list}\NormalTok{()}
  
  \CommentTok{\# Прогресс{-}индикатор}
\NormalTok{  pb }\OtherTok{\textless{}{-}} \FunctionTok{txtProgressBar}\NormalTok{(}\AttributeTok{min =} \DecValTok{0}\NormalTok{, }\AttributeTok{max =}\NormalTok{ n\_sim, }\AttributeTok{style =} \DecValTok{3}\NormalTok{, }\AttributeTok{width =} \DecValTok{50}\NormalTok{)}
  
  \ControlFlowTok{for}\NormalTok{ (i }\ControlFlowTok{in} \DecValTok{1}\SpecialCharTok{:}\NormalTok{n\_sim) \{}
\NormalTok{    sim\_result }\OtherTok{\textless{}{-}} \FunctionTok{run\_mse\_simulation}\NormalTok{(}
      \AttributeTok{sim\_id =}\NormalTok{ i,}
      \AttributeTok{HCR\_function =}\NormalTok{ strategies[[strategy\_name]],}
      \AttributeTok{params =}\NormalTok{ sim\_params,}
      \AttributeTok{settings =}\NormalTok{ sim\_settings}
\NormalTok{    )}
\NormalTok{    sim\_result}\SpecialCharTok{$}\NormalTok{strategy }\OtherTok{\textless{}{-}}\NormalTok{ strategy\_name}
\NormalTok{    strategy\_results[[i]] }\OtherTok{\textless{}{-}}\NormalTok{ sim\_result}
    
    \FunctionTok{setTxtProgressBar}\NormalTok{(pb, i)}
\NormalTok{  \}}
  
  \FunctionTok{close}\NormalTok{(pb)}
  
  \CommentTok{\# Объединение результатов симуляций}
\NormalTok{  all\_results[[strategy\_name]] }\OtherTok{\textless{}{-}} \FunctionTok{bind\_rows}\NormalTok{(strategy\_results)}
  
  \FunctionTok{cat}\NormalTok{(}\FunctionTok{sprintf}\NormalTok{(}\StringTok{"}\SpecialCharTok{\textbackslash{}n}\StringTok{✓ Завершено \%d симуляций для \%s}\SpecialCharTok{\textbackslash{}n}\StringTok{"}\NormalTok{, n\_sim, strategy\_name))}
\NormalTok{\}}
\end{Highlighting}
\end{Shaded}

\begin{verbatim}

Запуск стратегии: Fish at Fmsy
Прогресс: 
  |                                                        
  |                                                  |   0%
  |                                                        
  |                                                  |   1%
  |                                                        
  |=                                                 |   1%
  |                                                        
  |=                                                 |   2%
  |                                                        
  |=                                                 |   3%
  |                                                        
  |==                                                |   3%
  |                                                        
  |==                                                |   4%
  |                                                        
  |==                                                |   5%
  |                                                        
  |===                                               |   5%
  |                                                        
  |===                                               |   6%
  |                                                        
  |===                                               |   7%
  |                                                        
  |====                                              |   7%
  |                                                        
  |====                                              |   8%
  |                                                        
  |====                                              |   9%
  |                                                        
  |=====                                             |   9%
  |                                                        
  |=====                                             |  10%
  |                                                        
  |=====                                             |  11%
  |                                                        
  |======                                            |  11%
  |                                                        
  |======                                            |  12%
  |                                                        
  |======                                            |  13%
  |                                                        
  |=======                                           |  13%
  |                                                        
  |=======                                           |  14%
  |                                                        
  |=======                                           |  15%
  |                                                        
  |========                                          |  15%
  |                                                        
  |========                                          |  16%
  |                                                        
  |========                                          |  17%
  |                                                        
  |=========                                         |  17%
  |                                                        
  |=========                                         |  18%
  |                                                        
  |=========                                         |  19%
  |                                                        
  |==========                                        |  19%
  |                                                        
  |==========                                        |  20%
  |                                                        
  |==========                                        |  21%
  |                                                        
  |===========                                       |  21%
  |                                                        
  |===========                                       |  22%
  |                                                        
  |===========                                       |  23%
  |                                                        
  |============                                      |  23%
  |                                                        
  |============                                      |  24%
  |                                                        
  |============                                      |  25%
  |                                                        
  |=============                                     |  25%
  |                                                        
  |=============                                     |  26%
  |                                                        
  |=============                                     |  27%
  |                                                        
  |==============                                    |  27%
  |                                                        
  |==============                                    |  28%
  |                                                        
  |==============                                    |  29%
  |                                                        
  |===============                                   |  29%
  |                                                        
  |===============                                   |  30%
  |                                                        
  |===============                                   |  31%
  |                                                        
  |================                                  |  31%
  |                                                        
  |================                                  |  32%
  |                                                        
  |================                                  |  33%
  |                                                        
  |=================                                 |  33%
  |                                                        
  |=================                                 |  34%
  |                                                        
  |=================                                 |  35%
  |                                                        
  |==================                                |  35%
  |                                                        
  |==================                                |  36%
  |                                                        
  |==================                                |  37%
  |                                                        
  |===================                               |  37%
  |                                                        
  |===================                               |  38%
  |                                                        
  |===================                               |  39%
  |                                                        
  |====================                              |  39%
  |                                                        
  |====================                              |  40%
  |                                                        
  |====================                              |  41%
  |                                                        
  |=====================                             |  41%
  |                                                        
  |=====================                             |  42%
  |                                                        
  |=====================                             |  43%
  |                                                        
  |======================                            |  43%
  |                                                        
  |======================                            |  44%
  |                                                        
  |======================                            |  45%
  |                                                        
  |=======================                           |  45%
  |                                                        
  |=======================                           |  46%
  |                                                        
  |=======================                           |  47%
  |                                                        
  |========================                          |  47%
  |                                                        
  |========================                          |  48%
  |                                                        
  |========================                          |  49%
  |                                                        
  |=========================                         |  49%
  |                                                        
  |=========================                         |  50%
  |                                                        
  |=========================                         |  51%
  |                                                        
  |==========================                        |  51%
  |                                                        
  |==========================                        |  52%
  |                                                        
  |==========================                        |  53%
  |                                                        
  |===========================                       |  53%
  |                                                        
  |===========================                       |  54%
  |                                                        
  |===========================                       |  55%
  |                                                        
  |============================                      |  55%
  |                                                        
  |============================                      |  56%
  |                                                        
  |============================                      |  57%
  |                                                        
  |=============================                     |  57%
  |                                                        
  |=============================                     |  58%
  |                                                        
  |=============================                     |  59%
  |                                                        
  |==============================                    |  59%
  |                                                        
  |==============================                    |  60%
  |                                                        
  |==============================                    |  61%
  |                                                        
  |===============================                   |  61%
  |                                                        
  |===============================                   |  62%
  |                                                        
  |===============================                   |  63%
  |                                                        
  |================================                  |  63%
  |                                                        
  |================================                  |  64%
  |                                                        
  |================================                  |  65%
  |                                                        
  |=================================                 |  65%
  |                                                        
  |=================================                 |  66%
  |                                                        
  |=================================                 |  67%
  |                                                        
  |==================================                |  67%
  |                                                        
  |==================================                |  68%
  |                                                        
  |==================================                |  69%
  |                                                        
  |===================================               |  69%
  |                                                        
  |===================================               |  70%
  |                                                        
  |===================================               |  71%
  |                                                        
  |====================================              |  71%
  |                                                        
  |====================================              |  72%
  |                                                        
  |====================================              |  73%
  |                                                        
  |=====================================             |  73%
  |                                                        
  |=====================================             |  74%
  |                                                        
  |=====================================             |  75%
  |                                                        
  |======================================            |  75%
  |                                                        
  |======================================            |  76%
  |                                                        
  |======================================            |  77%
  |                                                        
  |=======================================           |  77%
  |                                                        
  |=======================================           |  78%
  |                                                        
  |=======================================           |  79%
  |                                                        
  |========================================          |  79%
  |                                                        
  |========================================          |  80%
  |                                                        
  |========================================          |  81%
  |                                                        
  |=========================================         |  81%
  |                                                        
  |=========================================         |  82%
  |                                                        
  |=========================================         |  83%
  |                                                        
  |==========================================        |  83%
  |                                                        
  |==========================================        |  84%
  |                                                        
  |==========================================        |  85%
  |                                                        
  |===========================================       |  85%
  |                                                        
  |===========================================       |  86%
  |                                                        
  |===========================================       |  87%
  |                                                        
  |============================================      |  87%
  |                                                        
  |============================================      |  88%
  |                                                        
  |============================================      |  89%
  |                                                        
  |=============================================     |  89%
  |                                                        
  |=============================================     |  90%
  |                                                        
  |=============================================     |  91%
  |                                                        
  |==============================================    |  91%
  |                                                        
  |==============================================    |  92%
  |                                                        
  |==============================================    |  93%
  |                                                        
  |===============================================   |  93%
  |                                                        
  |===============================================   |  94%
  |                                                        
  |===============================================   |  95%
  |                                                        
  |================================================  |  95%
  |                                                        
  |================================================  |  96%
  |                                                        
  |================================================  |  97%
  |                                                        
  |================================================= |  97%
  |                                                        
  |================================================= |  98%
  |                                                        
  |================================================= |  99%
  |                                                        
  |==================================================|  99%
  |                                                        
  |==================================================| 100%

<U+2713> Завершено 500 симуляций для Fish at Fmsy

Запуск стратегии: MSY Hockey-stick
Прогресс: 
  |                                                        
  |                                                  |   0%
  |                                                        
  |                                                  |   1%
  |                                                        
  |=                                                 |   1%
  |                                                        
  |=                                                 |   2%
  |                                                        
  |=                                                 |   3%
  |                                                        
  |==                                                |   3%
  |                                                        
  |==                                                |   4%
  |                                                        
  |==                                                |   5%
  |                                                        
  |===                                               |   5%
  |                                                        
  |===                                               |   6%
  |                                                        
  |===                                               |   7%
  |                                                        
  |====                                              |   7%
  |                                                        
  |====                                              |   8%
  |                                                        
  |====                                              |   9%
  |                                                        
  |=====                                             |   9%
  |                                                        
  |=====                                             |  10%
  |                                                        
  |=====                                             |  11%
  |                                                        
  |======                                            |  11%
  |                                                        
  |======                                            |  12%
  |                                                        
  |======                                            |  13%
  |                                                        
  |=======                                           |  13%
  |                                                        
  |=======                                           |  14%
  |                                                        
  |=======                                           |  15%
  |                                                        
  |========                                          |  15%
  |                                                        
  |========                                          |  16%
  |                                                        
  |========                                          |  17%
  |                                                        
  |=========                                         |  17%
  |                                                        
  |=========                                         |  18%
  |                                                        
  |=========                                         |  19%
  |                                                        
  |==========                                        |  19%
  |                                                        
  |==========                                        |  20%
  |                                                        
  |==========                                        |  21%
  |                                                        
  |===========                                       |  21%
  |                                                        
  |===========                                       |  22%
  |                                                        
  |===========                                       |  23%
  |                                                        
  |============                                      |  23%
  |                                                        
  |============                                      |  24%
  |                                                        
  |============                                      |  25%
  |                                                        
  |=============                                     |  25%
  |                                                        
  |=============                                     |  26%
  |                                                        
  |=============                                     |  27%
  |                                                        
  |==============                                    |  27%
  |                                                        
  |==============                                    |  28%
  |                                                        
  |==============                                    |  29%
  |                                                        
  |===============                                   |  29%
  |                                                        
  |===============                                   |  30%
  |                                                        
  |===============                                   |  31%
  |                                                        
  |================                                  |  31%
  |                                                        
  |================                                  |  32%
  |                                                        
  |================                                  |  33%
  |                                                        
  |=================                                 |  33%
  |                                                        
  |=================                                 |  34%
  |                                                        
  |=================                                 |  35%
  |                                                        
  |==================                                |  35%
  |                                                        
  |==================                                |  36%
  |                                                        
  |==================                                |  37%
  |                                                        
  |===================                               |  37%
  |                                                        
  |===================                               |  38%
  |                                                        
  |===================                               |  39%
  |                                                        
  |====================                              |  39%
  |                                                        
  |====================                              |  40%
  |                                                        
  |====================                              |  41%
  |                                                        
  |=====================                             |  41%
  |                                                        
  |=====================                             |  42%
  |                                                        
  |=====================                             |  43%
  |                                                        
  |======================                            |  43%
  |                                                        
  |======================                            |  44%
  |                                                        
  |======================                            |  45%
  |                                                        
  |=======================                           |  45%
  |                                                        
  |=======================                           |  46%
  |                                                        
  |=======================                           |  47%
  |                                                        
  |========================                          |  47%
  |                                                        
  |========================                          |  48%
  |                                                        
  |========================                          |  49%
  |                                                        
  |=========================                         |  49%
  |                                                        
  |=========================                         |  50%
  |                                                        
  |=========================                         |  51%
  |                                                        
  |==========================                        |  51%
  |                                                        
  |==========================                        |  52%
  |                                                        
  |==========================                        |  53%
  |                                                        
  |===========================                       |  53%
  |                                                        
  |===========================                       |  54%
  |                                                        
  |===========================                       |  55%
  |                                                        
  |============================                      |  55%
  |                                                        
  |============================                      |  56%
  |                                                        
  |============================                      |  57%
  |                                                        
  |=============================                     |  57%
  |                                                        
  |=============================                     |  58%
  |                                                        
  |=============================                     |  59%
  |                                                        
  |==============================                    |  59%
  |                                                        
  |==============================                    |  60%
  |                                                        
  |==============================                    |  61%
  |                                                        
  |===============================                   |  61%
  |                                                        
  |===============================                   |  62%
  |                                                        
  |===============================                   |  63%
  |                                                        
  |================================                  |  63%
  |                                                        
  |================================                  |  64%
  |                                                        
  |================================                  |  65%
  |                                                        
  |=================================                 |  65%
  |                                                        
  |=================================                 |  66%
  |                                                        
  |=================================                 |  67%
  |                                                        
  |==================================                |  67%
  |                                                        
  |==================================                |  68%
  |                                                        
  |==================================                |  69%
  |                                                        
  |===================================               |  69%
  |                                                        
  |===================================               |  70%
  |                                                        
  |===================================               |  71%
  |                                                        
  |====================================              |  71%
  |                                                        
  |====================================              |  72%
  |                                                        
  |====================================              |  73%
  |                                                        
  |=====================================             |  73%
  |                                                        
  |=====================================             |  74%
  |                                                        
  |=====================================             |  75%
  |                                                        
  |======================================            |  75%
  |                                                        
  |======================================            |  76%
  |                                                        
  |======================================            |  77%
  |                                                        
  |=======================================           |  77%
  |                                                        
  |=======================================           |  78%
  |                                                        
  |=======================================           |  79%
  |                                                        
  |========================================          |  79%
  |                                                        
  |========================================          |  80%
  |                                                        
  |========================================          |  81%
  |                                                        
  |=========================================         |  81%
  |                                                        
  |=========================================         |  82%
  |                                                        
  |=========================================         |  83%
  |                                                        
  |==========================================        |  83%
  |                                                        
  |==========================================        |  84%
  |                                                        
  |==========================================        |  85%
  |                                                        
  |===========================================       |  85%
  |                                                        
  |===========================================       |  86%
  |                                                        
  |===========================================       |  87%
  |                                                        
  |============================================      |  87%
  |                                                        
  |============================================      |  88%
  |                                                        
  |============================================      |  89%
  |                                                        
  |=============================================     |  89%
  |                                                        
  |=============================================     |  90%
  |                                                        
  |=============================================     |  91%
  |                                                        
  |==============================================    |  91%
  |                                                        
  |==============================================    |  92%
  |                                                        
  |==============================================    |  93%
  |                                                        
  |===============================================   |  93%
  |                                                        
  |===============================================   |  94%
  |                                                        
  |===============================================   |  95%
  |                                                        
  |================================================  |  95%
  |                                                        
  |================================================  |  96%
  |                                                        
  |================================================  |  97%
  |                                                        
  |================================================= |  97%
  |                                                        
  |================================================= |  98%
  |                                                        
  |================================================= |  99%
  |                                                        
  |==================================================|  99%
  |                                                        
  |==================================================| 100%

<U+2713> Завершено 500 симуляций для MSY Hockey-stick

Запуск стратегии: ICES Advice Rule
Прогресс: 
  |                                                        
  |                                                  |   0%
  |                                                        
  |                                                  |   1%
  |                                                        
  |=                                                 |   1%
  |                                                        
  |=                                                 |   2%
  |                                                        
  |=                                                 |   3%
  |                                                        
  |==                                                |   3%
  |                                                        
  |==                                                |   4%
  |                                                        
  |==                                                |   5%
  |                                                        
  |===                                               |   5%
  |                                                        
  |===                                               |   6%
  |                                                        
  |===                                               |   7%
  |                                                        
  |====                                              |   7%
  |                                                        
  |====                                              |   8%
  |                                                        
  |====                                              |   9%
  |                                                        
  |=====                                             |   9%
  |                                                        
  |=====                                             |  10%
  |                                                        
  |=====                                             |  11%
  |                                                        
  |======                                            |  11%
  |                                                        
  |======                                            |  12%
  |                                                        
  |======                                            |  13%
  |                                                        
  |=======                                           |  13%
  |                                                        
  |=======                                           |  14%
  |                                                        
  |=======                                           |  15%
  |                                                        
  |========                                          |  15%
  |                                                        
  |========                                          |  16%
  |                                                        
  |========                                          |  17%
  |                                                        
  |=========                                         |  17%
  |                                                        
  |=========                                         |  18%
  |                                                        
  |=========                                         |  19%
  |                                                        
  |==========                                        |  19%
  |                                                        
  |==========                                        |  20%
  |                                                        
  |==========                                        |  21%
  |                                                        
  |===========                                       |  21%
  |                                                        
  |===========                                       |  22%
  |                                                        
  |===========                                       |  23%
  |                                                        
  |============                                      |  23%
  |                                                        
  |============                                      |  24%
  |                                                        
  |============                                      |  25%
  |                                                        
  |=============                                     |  25%
  |                                                        
  |=============                                     |  26%
  |                                                        
  |=============                                     |  27%
  |                                                        
  |==============                                    |  27%
  |                                                        
  |==============                                    |  28%
  |                                                        
  |==============                                    |  29%
  |                                                        
  |===============                                   |  29%
  |                                                        
  |===============                                   |  30%
  |                                                        
  |===============                                   |  31%
  |                                                        
  |================                                  |  31%
  |                                                        
  |================                                  |  32%
  |                                                        
  |================                                  |  33%
  |                                                        
  |=================                                 |  33%
  |                                                        
  |=================                                 |  34%
  |                                                        
  |=================                                 |  35%
  |                                                        
  |==================                                |  35%
  |                                                        
  |==================                                |  36%
  |                                                        
  |==================                                |  37%
  |                                                        
  |===================                               |  37%
  |                                                        
  |===================                               |  38%
  |                                                        
  |===================                               |  39%
  |                                                        
  |====================                              |  39%
  |                                                        
  |====================                              |  40%
  |                                                        
  |====================                              |  41%
  |                                                        
  |=====================                             |  41%
  |                                                        
  |=====================                             |  42%
  |                                                        
  |=====================                             |  43%
  |                                                        
  |======================                            |  43%
  |                                                        
  |======================                            |  44%
  |                                                        
  |======================                            |  45%
  |                                                        
  |=======================                           |  45%
  |                                                        
  |=======================                           |  46%
  |                                                        
  |=======================                           |  47%
  |                                                        
  |========================                          |  47%
  |                                                        
  |========================                          |  48%
  |                                                        
  |========================                          |  49%
  |                                                        
  |=========================                         |  49%
  |                                                        
  |=========================                         |  50%
  |                                                        
  |=========================                         |  51%
  |                                                        
  |==========================                        |  51%
  |                                                        
  |==========================                        |  52%
  |                                                        
  |==========================                        |  53%
  |                                                        
  |===========================                       |  53%
  |                                                        
  |===========================                       |  54%
  |                                                        
  |===========================                       |  55%
  |                                                        
  |============================                      |  55%
  |                                                        
  |============================                      |  56%
  |                                                        
  |============================                      |  57%
  |                                                        
  |=============================                     |  57%
  |                                                        
  |=============================                     |  58%
  |                                                        
  |=============================                     |  59%
  |                                                        
  |==============================                    |  59%
  |                                                        
  |==============================                    |  60%
  |                                                        
  |==============================                    |  61%
  |                                                        
  |===============================                   |  61%
  |                                                        
  |===============================                   |  62%
  |                                                        
  |===============================                   |  63%
  |                                                        
  |================================                  |  63%
  |                                                        
  |================================                  |  64%
  |                                                        
  |================================                  |  65%
  |                                                        
  |=================================                 |  65%
  |                                                        
  |=================================                 |  66%
  |                                                        
  |=================================                 |  67%
  |                                                        
  |==================================                |  67%
  |                                                        
  |==================================                |  68%
  |                                                        
  |==================================                |  69%
  |                                                        
  |===================================               |  69%
  |                                                        
  |===================================               |  70%
  |                                                        
  |===================================               |  71%
  |                                                        
  |====================================              |  71%
  |                                                        
  |====================================              |  72%
  |                                                        
  |====================================              |  73%
  |                                                        
  |=====================================             |  73%
  |                                                        
  |=====================================             |  74%
  |                                                        
  |=====================================             |  75%
  |                                                        
  |======================================            |  75%
  |                                                        
  |======================================            |  76%
  |                                                        
  |======================================            |  77%
  |                                                        
  |=======================================           |  77%
  |                                                        
  |=======================================           |  78%
  |                                                        
  |=======================================           |  79%
  |                                                        
  |========================================          |  79%
  |                                                        
  |========================================          |  80%
  |                                                        
  |========================================          |  81%
  |                                                        
  |=========================================         |  81%
  |                                                        
  |=========================================         |  82%
  |                                                        
  |=========================================         |  83%
  |                                                        
  |==========================================        |  83%
  |                                                        
  |==========================================        |  84%
  |                                                        
  |==========================================        |  85%
  |                                                        
  |===========================================       |  85%
  |                                                        
  |===========================================       |  86%
  |                                                        
  |===========================================       |  87%
  |                                                        
  |============================================      |  87%
  |                                                        
  |============================================      |  88%
  |                                                        
  |============================================      |  89%
  |                                                        
  |=============================================     |  89%
  |                                                        
  |=============================================     |  90%
  |                                                        
  |=============================================     |  91%
  |                                                        
  |==============================================    |  91%
  |                                                        
  |==============================================    |  92%
  |                                                        
  |==============================================    |  93%
  |                                                        
  |===============================================   |  93%
  |                                                        
  |===============================================   |  94%
  |                                                        
  |===============================================   |  95%
  |                                                        
  |================================================  |  95%
  |                                                        
  |================================================  |  96%
  |                                                        
  |================================================  |  97%
  |                                                        
  |================================================= |  97%
  |                                                        
  |================================================= |  98%
  |                                                        
  |================================================= |  99%
  |                                                        
  |==================================================|  99%
  |                                                        
  |==================================================| 100%

<U+2713> Завершено 500 симуляций для ICES Advice Rule
\end{verbatim}

\begin{Shaded}
\begin{Highlighting}[]
\DocumentationTok{\#\# 7.4 Объединение всех результатов в один датафрейм}
\NormalTok{mse\_results }\OtherTok{\textless{}{-}} \FunctionTok{bind\_rows}\NormalTok{(all\_results)}

\FunctionTok{cat}\NormalTok{(}\StringTok{"}\SpecialCharTok{\textbackslash{}n}\StringTok{✓ Всего выполнено:"}\NormalTok{, n\_sim }\SpecialCharTok{*} \FunctionTok{length}\NormalTok{(strategies), }\StringTok{"симуляций}\SpecialCharTok{\textbackslash{}n}\StringTok{"}\NormalTok{)}
\end{Highlighting}
\end{Shaded}

\begin{verbatim}

<U+2713> Всего выполнено: 1500 симуляций
\end{verbatim}

\begin{Shaded}
\begin{Highlighting}[]
\CommentTok{\# {-}{-}{-}{-}{-}{-}{-}{-}{-}{-}{-}{-}{-}{-}{-}{-}{-}{-}{-} 8. РАСЧЕТ РИСК{-}МЕТРИК {-}{-}{-}{-}{-}{-}{-}{-}{-}{-}{-}{-}{-}{-}{-}{-}{-}{-}{-}{-}}

\FunctionTok{cat}\NormalTok{(}\StringTok{"}\SpecialCharTok{\textbackslash{}n}\StringTok{========== РАСЧЕТ РИСК{-}МЕТРИК ==========}\SpecialCharTok{\textbackslash{}n}\StringTok{"}\NormalTok{)}
\end{Highlighting}
\end{Shaded}

\begin{verbatim}

========== РАСЧЕТ РИСК-МЕТРИК ==========
\end{verbatim}

\begin{Shaded}
\begin{Highlighting}[]
\DocumentationTok{\#\# 8.1 Функция для расчета комплексных метрик производительности}
\NormalTok{calculate\_performance\_metrics }\OtherTok{\textless{}{-}} \ControlFlowTok{function}\NormalTok{(results) \{}
  
\NormalTok{  metrics }\OtherTok{\textless{}{-}}\NormalTok{ results }\SpecialCharTok{\%\textgreater{}\%}
    \FunctionTok{group\_by}\NormalTok{(strategy) }\SpecialCharTok{\%\textgreater{}\%}
    \FunctionTok{summarise}\NormalTok{(}
      
      \CommentTok{\# === БИОЛОГИЧЕСКИЕ МЕТРИКИ ===}
      \CommentTok{\# Вероятность перелова (F \textgreater{} Fmsy)}
      \AttributeTok{prob\_overfishing =} \FunctionTok{mean}\NormalTok{(F\_Fmsy\_true }\SpecialCharTok{\textgreater{}} \DecValTok{1}\NormalTok{, }\AttributeTok{na.rm =} \ConstantTok{TRUE}\NormalTok{),}
      
      \CommentTok{\# Вероятность истощения запаса (B \textless{} 0.5*Bmsy)}
      \AttributeTok{prob\_overfished =} \FunctionTok{mean}\NormalTok{(B\_Bmsy\_true }\SpecialCharTok{\textless{}} \FloatTok{0.5}\NormalTok{, }\AttributeTok{na.rm =} \ConstantTok{TRUE}\NormalTok{),}
      
      \CommentTok{\# Вероятность коллапса (B \textless{} 0.2*Bmsy)}
      \AttributeTok{prob\_collapsed =} \FunctionTok{mean}\NormalTok{(B\_Bmsy\_true }\SpecialCharTok{\textless{}} \FloatTok{0.2}\NormalTok{, }\AttributeTok{na.rm =} \ConstantTok{TRUE}\NormalTok{),}
      
      \CommentTok{\# Вероятность нахождения в "зеленой зоне" Кобе}
      \AttributeTok{prob\_green\_zone =} \FunctionTok{mean}\NormalTok{(B\_Bmsy\_true }\SpecialCharTok{\textgreater{}} \DecValTok{1} \SpecialCharTok{\&}\NormalTok{ F\_Fmsy\_true }\SpecialCharTok{\textless{}} \DecValTok{1}\NormalTok{, }\AttributeTok{na.rm =} \ConstantTok{TRUE}\NormalTok{),}
      
      \CommentTok{\# Средние показатели за последние 5 лет}
      \AttributeTok{mean\_B\_Bmsy\_final =} \FunctionTok{mean}\NormalTok{(B\_Bmsy\_true[year }\SpecialCharTok{\textgreater{}}\NormalTok{ (}\FunctionTok{max}\NormalTok{(year) }\SpecialCharTok{{-}} \DecValTok{5}\NormalTok{)], }\AttributeTok{na.rm =} \ConstantTok{TRUE}\NormalTok{),}
      \AttributeTok{mean\_F\_Fmsy\_final =} \FunctionTok{mean}\NormalTok{(F\_Fmsy\_true[year }\SpecialCharTok{\textgreater{}}\NormalTok{ (}\FunctionTok{max}\NormalTok{(year) }\SpecialCharTok{{-}} \DecValTok{5}\NormalTok{)], }\AttributeTok{na.rm =} \ConstantTok{TRUE}\NormalTok{),}
      
      \CommentTok{\# === ЭКОНОМИЧЕСКИЕ МЕТРИКИ ===}
      \CommentTok{\# Средний вылов за весь период}
      \AttributeTok{mean\_catch =} \FunctionTok{mean}\NormalTok{(catch, }\AttributeTok{na.rm =} \ConstantTok{TRUE}\NormalTok{),}
      
      \CommentTok{\# Суммарный вылов за период}
      \AttributeTok{total\_catch =} \FunctionTok{sum}\NormalTok{(catch) }\SpecialCharTok{/} \FunctionTok{n\_distinct}\NormalTok{(sim\_id),}
      
      \CommentTok{\# Стабильность вылова (обратная величина коэффициента вариации)}
      \AttributeTok{catch\_stability =}\NormalTok{ \{}
\NormalTok{        annual\_catch }\OtherTok{\textless{}{-}} \FunctionTok{aggregate}\NormalTok{(catch, }\AttributeTok{by =} \FunctionTok{list}\NormalTok{(year), mean)}\SpecialCharTok{$}\NormalTok{x}
        \DecValTok{1} \SpecialCharTok{{-}}\NormalTok{ (}\FunctionTok{sd}\NormalTok{(annual\_catch, }\AttributeTok{na.rm =} \ConstantTok{TRUE}\NormalTok{) }\SpecialCharTok{/} \FunctionTok{mean}\NormalTok{(annual\_catch, }\AttributeTok{na.rm =} \ConstantTok{TRUE}\NormalTok{))}
\NormalTok{      \},}
      
      \CommentTok{\# Средняя межгодовая изменчивость вылова (AAV)}
      \AttributeTok{catch\_aav =}\NormalTok{ \{}
\NormalTok{        annual\_catch }\OtherTok{\textless{}{-}} \FunctionTok{aggregate}\NormalTok{(catch, }\AttributeTok{by =} \FunctionTok{list}\NormalTok{(year), mean)}\SpecialCharTok{$}\NormalTok{x}
        \FunctionTok{mean}\NormalTok{(}\FunctionTok{abs}\NormalTok{(}\FunctionTok{diff}\NormalTok{(annual\_catch)) }\SpecialCharTok{/}\NormalTok{ annual\_catch[}\SpecialCharTok{{-}}\FunctionTok{length}\NormalTok{(annual\_catch)], }\AttributeTok{na.rm =} \ConstantTok{TRUE}\NormalTok{)}
\NormalTok{      \},}
      
      \CommentTok{\# === УПРАВЛЕНЧЕСКИЕ МЕТРИКИ ===}
      \CommentTok{\# Вероятность закрытия промысла}
      \AttributeTok{prob\_closure =} \FunctionTok{mean}\NormalTok{(TAC }\SpecialCharTok{==} \DecValTok{0}\NormalTok{, }\AttributeTok{na.rm =} \ConstantTok{TRUE}\NormalTok{),}
      
      \CommentTok{\# Частота изменения рекомендаций}
      \AttributeTok{advice\_changes =}\NormalTok{ \{}
\NormalTok{        changes }\OtherTok{\textless{}{-}} \FunctionTok{aggregate}\NormalTok{(F\_advice, }\AttributeTok{by =} \FunctionTok{list}\NormalTok{(sim\_id), }
                           \ControlFlowTok{function}\NormalTok{(x) }\FunctionTok{sum}\NormalTok{(}\FunctionTok{diff}\NormalTok{(x) }\SpecialCharTok{!=} \DecValTok{0}\NormalTok{))}\SpecialCharTok{$}\NormalTok{x}
        \FunctionTok{mean}\NormalTok{(changes) }\SpecialCharTok{/}\NormalTok{ (}\FunctionTok{max}\NormalTok{(year) }\SpecialCharTok{{-}} \DecValTok{1}\NormalTok{)}
\NormalTok{      \},}
      
      \AttributeTok{.groups =} \StringTok{"drop"}
\NormalTok{    )}
  
  \FunctionTok{return}\NormalTok{(metrics)}
\NormalTok{\}}

\DocumentationTok{\#\# 8.2 Расчет метрик для каждой стратегии}
\NormalTok{performance\_metrics }\OtherTok{\textless{}{-}} \FunctionTok{calculate\_performance\_metrics}\NormalTok{(mse\_results)}

\DocumentationTok{\#\# 8.3 Добавление комплексных оценок}
\NormalTok{performance\_metrics }\OtherTok{\textless{}{-}}\NormalTok{ performance\_metrics }\SpecialCharTok{\%\textgreater{}\%}
  \FunctionTok{mutate}\NormalTok{(}
    \CommentTok{\# Биологический индекс (0{-}1, где 1 {-} лучше)}
    \AttributeTok{bio\_score =} \FloatTok{0.4} \SpecialCharTok{*}\NormalTok{ (}\DecValTok{1} \SpecialCharTok{{-}}\NormalTok{ prob\_overfished) }\SpecialCharTok{+} 
                \FloatTok{0.3} \SpecialCharTok{*}\NormalTok{ (}\DecValTok{1} \SpecialCharTok{{-}}\NormalTok{ prob\_overfishing) }\SpecialCharTok{+} 
                \FloatTok{0.3} \SpecialCharTok{*}\NormalTok{ prob\_green\_zone,}
    
    \CommentTok{\# Экономический индекс (0{-}1, где 1 {-} лучше)}
    \AttributeTok{econ\_score =} \FloatTok{0.5} \SpecialCharTok{*}\NormalTok{ (mean\_catch }\SpecialCharTok{/} \FunctionTok{max}\NormalTok{(mean\_catch)) }\SpecialCharTok{+} 
                 \FloatTok{0.3} \SpecialCharTok{*}\NormalTok{ catch\_stability }\SpecialCharTok{+}
                 \FloatTok{0.2} \SpecialCharTok{*}\NormalTok{ (}\DecValTok{1} \SpecialCharTok{{-}}\NormalTok{ catch\_aav),}
    
    \CommentTok{\# Общий индекс производительности}
    \AttributeTok{overall\_score =} \FloatTok{0.6} \SpecialCharTok{*}\NormalTok{ bio\_score }\SpecialCharTok{+} \FloatTok{0.4} \SpecialCharTok{*}\NormalTok{ econ\_score}
\NormalTok{  ) }\SpecialCharTok{\%\textgreater{}\%}
  \FunctionTok{arrange}\NormalTok{(}\FunctionTok{desc}\NormalTok{(overall\_score))}

\DocumentationTok{\#\# 8.4 Вывод таблицы метрик}
\FunctionTok{cat}\NormalTok{(}\StringTok{"}\SpecialCharTok{\textbackslash{}n}\StringTok{{-}{-}{-} ТАБЛИЦА РИСК{-}МЕТРИК {-}{-}{-}}\SpecialCharTok{\textbackslash{}n\textbackslash{}n}\StringTok{"}\NormalTok{)}
\end{Highlighting}
\end{Shaded}

\begin{verbatim}

--- ТАБЛИЦА РИСК-МЕТРИК ---
\end{verbatim}

\begin{Shaded}
\begin{Highlighting}[]
\CommentTok{\# Форматированный вывод ключевых метрик}
\ControlFlowTok{for}\NormalTok{ (strat }\ControlFlowTok{in} \FunctionTok{unique}\NormalTok{(performance\_metrics}\SpecialCharTok{$}\NormalTok{strategy)) \{}
\NormalTok{  metrics\_row }\OtherTok{\textless{}{-}}\NormalTok{ performance\_metrics[performance\_metrics}\SpecialCharTok{$}\NormalTok{strategy }\SpecialCharTok{==}\NormalTok{ strat, ]}
  
  \FunctionTok{cat}\NormalTok{(}\FunctionTok{sprintf}\NormalTok{(}\StringTok{"СТРАТЕГИЯ: \%s}\SpecialCharTok{\textbackslash{}n}\StringTok{"}\NormalTok{, strat))}
  \FunctionTok{cat}\NormalTok{(}\FunctionTok{strrep}\NormalTok{(}\StringTok{"{-}"}\NormalTok{, }\DecValTok{40}\NormalTok{), }\StringTok{"}\SpecialCharTok{\textbackslash{}n}\StringTok{"}\NormalTok{)}
  \FunctionTok{cat}\NormalTok{(}\FunctionTok{sprintf}\NormalTok{(}\StringTok{"Биологические риски:}\SpecialCharTok{\textbackslash{}n}\StringTok{"}\NormalTok{))}
  \FunctionTok{cat}\NormalTok{(}\FunctionTok{sprintf}\NormalTok{(}\StringTok{"  P(перелов): \%.1f\%\%}\SpecialCharTok{\textbackslash{}n}\StringTok{"}\NormalTok{, metrics\_row}\SpecialCharTok{$}\NormalTok{prob\_overfishing }\SpecialCharTok{*} \DecValTok{100}\NormalTok{))}
  \FunctionTok{cat}\NormalTok{(}\FunctionTok{sprintf}\NormalTok{(}\StringTok{"  P(истощение): \%.1f\%\%}\SpecialCharTok{\textbackslash{}n}\StringTok{"}\NormalTok{, metrics\_row}\SpecialCharTok{$}\NormalTok{prob\_overfished }\SpecialCharTok{*} \DecValTok{100}\NormalTok{))}
  \FunctionTok{cat}\NormalTok{(}\FunctionTok{sprintf}\NormalTok{(}\StringTok{"  P(коллапс): \%.1f\%\%}\SpecialCharTok{\textbackslash{}n}\StringTok{"}\NormalTok{, metrics\_row}\SpecialCharTok{$}\NormalTok{prob\_collapsed }\SpecialCharTok{*} \DecValTok{100}\NormalTok{))}
  \FunctionTok{cat}\NormalTok{(}\FunctionTok{sprintf}\NormalTok{(}\StringTok{"  P(зеленая зона): \%.1f\%\%}\SpecialCharTok{\textbackslash{}n}\StringTok{"}\NormalTok{, metrics\_row}\SpecialCharTok{$}\NormalTok{prob\_green\_zone }\SpecialCharTok{*} \DecValTok{100}\NormalTok{))}
  \FunctionTok{cat}\NormalTok{(}\FunctionTok{sprintf}\NormalTok{(}\StringTok{"Экономические показатели:}\SpecialCharTok{\textbackslash{}n}\StringTok{"}\NormalTok{))}
  \FunctionTok{cat}\NormalTok{(}\FunctionTok{sprintf}\NormalTok{(}\StringTok{"  Средний вылов: \%.1f тыс. т}\SpecialCharTok{\textbackslash{}n}\StringTok{"}\NormalTok{, metrics\_row}\SpecialCharTok{$}\NormalTok{mean\_catch))}
  \FunctionTok{cat}\NormalTok{(}\FunctionTok{sprintf}\NormalTok{(}\StringTok{"  Стабильность: \%.2f}\SpecialCharTok{\textbackslash{}n}\StringTok{"}\NormalTok{, metrics\_row}\SpecialCharTok{$}\NormalTok{catch\_stability))}
  \FunctionTok{cat}\NormalTok{(}\FunctionTok{sprintf}\NormalTok{(}\StringTok{"  Межгодовая изменчивость: \%.1f\%\%}\SpecialCharTok{\textbackslash{}n}\StringTok{"}\NormalTok{, metrics\_row}\SpecialCharTok{$}\NormalTok{catch\_aav }\SpecialCharTok{*} \DecValTok{100}\NormalTok{))}
  \FunctionTok{cat}\NormalTok{(}\FunctionTok{sprintf}\NormalTok{(}\StringTok{"Итоговые оценки:}\SpecialCharTok{\textbackslash{}n}\StringTok{"}\NormalTok{))}
  \FunctionTok{cat}\NormalTok{(}\FunctionTok{sprintf}\NormalTok{(}\StringTok{"  Биологический индекс: \%.2f}\SpecialCharTok{\textbackslash{}n}\StringTok{"}\NormalTok{, metrics\_row}\SpecialCharTok{$}\NormalTok{bio\_score))}
  \FunctionTok{cat}\NormalTok{(}\FunctionTok{sprintf}\NormalTok{(}\StringTok{"  Экономический индекс: \%.2f}\SpecialCharTok{\textbackslash{}n}\StringTok{"}\NormalTok{, metrics\_row}\SpecialCharTok{$}\NormalTok{econ\_score))}
  \FunctionTok{cat}\NormalTok{(}\FunctionTok{sprintf}\NormalTok{(}\StringTok{"  ОБЩИЙ ИНДЕКС: \%.2f}\SpecialCharTok{\textbackslash{}n}\StringTok{"}\NormalTok{, metrics\_row}\SpecialCharTok{$}\NormalTok{overall\_score))}
  \FunctionTok{cat}\NormalTok{(}\StringTok{"}\SpecialCharTok{\textbackslash{}n}\StringTok{"}\NormalTok{)}
\NormalTok{\}}
\end{Highlighting}
\end{Shaded}

\begin{verbatim}
СТРАТЕГИЯ: ICES Advice Rule
---------------------------------------- 
Биологические риски:
  P(перелов): 8.3%
  P(истощение): 0.2%
  P(коллапс): 0.0%
  P(зеленая зона): 57.6%
Экономические показатели:
  Средний вылов: 10.3 тыс. т
  Стабильность: 0.95
  Межгодовая изменчивость: 1.3%
Итоговые оценки:
  Биологический индекс: 0.85
  Экономический индекс: 0.97
  ОБЩИЙ ИНДЕКС: 0.90

СТРАТЕГИЯ: MSY Hockey-stick
---------------------------------------- 
Биологические риски:
  P(перелов): 46.3%
  P(истощение): 3.7%
  P(коллапс): 0.0%
  P(зеленая зона): 20.2%
Экономические показатели:
  Средний вылов: 10.5 тыс. т
  Стабильность: 0.92
  Межгодовая изменчивость: 1.1%
Итоговые оценки:
  Биологический индекс: 0.61
  Экономический индекс: 0.97
  ОБЩИЙ ИНДЕКС: 0.75

СТРАТЕГИЯ: Fish at Fmsy
---------------------------------------- 
Биологические риски:
  P(перелов): 48.5%
  P(истощение): 4.7%
  P(коллапс): 0.0%
  P(зеленая зона): 20.0%
Экономические показатели:
  Средний вылов: 10.5 тыс. т
  Стабильность: 0.92
  Межгодовая изменчивость: 1.0%
Итоговые оценки:
  Биологический индекс: 0.60
  Экономический индекс: 0.97
  ОБЩИЙ ИНДЕКС: 0.75
\end{verbatim}

\begin{Shaded}
\begin{Highlighting}[]
\CommentTok{\# {-}{-}{-}{-}{-}{-}{-}{-}{-}{-}{-}{-}{-}{-}{-}{-}{-}{-}{-} 9. ВИЗУАЛИЗАЦИЯ РЕЗУЛЬТАТОВ {-}{-}{-}{-}{-}{-}{-}{-}{-}{-}{-}{-}{-}{-}{-}{-}{-}{-}{-}{-}}

\FunctionTok{cat}\NormalTok{(}\StringTok{"========== СОЗДАНИЕ ГРАФИКОВ ==========}\SpecialCharTok{\textbackslash{}n}\StringTok{"}\NormalTok{)}
\end{Highlighting}
\end{Shaded}

\begin{verbatim}
========== СОЗДАНИЕ ГРАФИКОВ ==========
\end{verbatim}

\begin{Shaded}
\begin{Highlighting}[]
\DocumentationTok{\#\# 9.1 Настройка темы для графиков}
\NormalTok{theme\_mse }\OtherTok{\textless{}{-}} \FunctionTok{theme\_minimal}\NormalTok{() }\SpecialCharTok{+}
  \FunctionTok{theme}\NormalTok{(}
    \AttributeTok{plot.title =} \FunctionTok{element\_text}\NormalTok{(}\AttributeTok{size =} \DecValTok{14}\NormalTok{, }\AttributeTok{face =} \StringTok{"bold"}\NormalTok{),}
    \AttributeTok{plot.subtitle =} \FunctionTok{element\_text}\NormalTok{(}\AttributeTok{size =} \DecValTok{11}\NormalTok{),}
    \AttributeTok{strip.text =} \FunctionTok{element\_text}\NormalTok{(}\AttributeTok{size =} \DecValTok{11}\NormalTok{, }\AttributeTok{face =} \StringTok{"bold"}\NormalTok{),}
    \AttributeTok{legend.position =} \StringTok{"bottom"}\NormalTok{,}
    \AttributeTok{panel.grid.minor =} \FunctionTok{element\_blank}\NormalTok{()}
\NormalTok{  )}

\DocumentationTok{\#\# 9.2 Подготовка сводных данных для визуализации}
\NormalTok{summary\_data }\OtherTok{\textless{}{-}}\NormalTok{ mse\_results }\SpecialCharTok{\%\textgreater{}\%}
  \FunctionTok{group\_by}\NormalTok{(strategy, year) }\SpecialCharTok{\%\textgreater{}\%}
  \FunctionTok{summarise}\NormalTok{(}
    \CommentTok{\# Квантили для биомассы}
    \AttributeTok{B\_median =} \FunctionTok{median}\NormalTok{(B\_true, }\AttributeTok{na.rm =} \ConstantTok{TRUE}\NormalTok{),}
    \AttributeTok{B\_q25 =} \FunctionTok{quantile}\NormalTok{(B\_true, }\FloatTok{0.25}\NormalTok{, }\AttributeTok{na.rm =} \ConstantTok{TRUE}\NormalTok{),}
    \AttributeTok{B\_q75 =} \FunctionTok{quantile}\NormalTok{(B\_true, }\FloatTok{0.75}\NormalTok{, }\AttributeTok{na.rm =} \ConstantTok{TRUE}\NormalTok{),}
    \AttributeTok{B\_q05 =} \FunctionTok{quantile}\NormalTok{(B\_true, }\FloatTok{0.05}\NormalTok{, }\AttributeTok{na.rm =} \ConstantTok{TRUE}\NormalTok{),}
    \AttributeTok{B\_q95 =} \FunctionTok{quantile}\NormalTok{(B\_true, }\FloatTok{0.95}\NormalTok{, }\AttributeTok{na.rm =} \ConstantTok{TRUE}\NormalTok{),}
    
    \CommentTok{\# Квантили для B/Bmsy}
    \AttributeTok{B\_Bmsy\_median =} \FunctionTok{median}\NormalTok{(B\_Bmsy\_true, }\AttributeTok{na.rm =} \ConstantTok{TRUE}\NormalTok{),}
    \AttributeTok{B\_Bmsy\_q25 =} \FunctionTok{quantile}\NormalTok{(B\_Bmsy\_true, }\FloatTok{0.25}\NormalTok{, }\AttributeTok{na.rm =} \ConstantTok{TRUE}\NormalTok{),}
    \AttributeTok{B\_Bmsy\_q75 =} \FunctionTok{quantile}\NormalTok{(B\_Bmsy\_true, }\FloatTok{0.75}\NormalTok{, }\AttributeTok{na.rm =} \ConstantTok{TRUE}\NormalTok{),}
    
    \CommentTok{\# Квантили для F/Fmsy}
    \AttributeTok{F\_Fmsy\_median =} \FunctionTok{median}\NormalTok{(F\_Fmsy\_true, }\AttributeTok{na.rm =} \ConstantTok{TRUE}\NormalTok{),}
    \AttributeTok{F\_Fmsy\_q25 =} \FunctionTok{quantile}\NormalTok{(F\_Fmsy\_true, }\FloatTok{0.25}\NormalTok{, }\AttributeTok{na.rm =} \ConstantTok{TRUE}\NormalTok{),}
    \AttributeTok{F\_Fmsy\_q75 =} \FunctionTok{quantile}\NormalTok{(F\_Fmsy\_true, }\FloatTok{0.75}\NormalTok{, }\AttributeTok{na.rm =} \ConstantTok{TRUE}\NormalTok{),}
    
    \CommentTok{\# Квантили для вылова}
    \AttributeTok{catch\_median =} \FunctionTok{median}\NormalTok{(catch, }\AttributeTok{na.rm =} \ConstantTok{TRUE}\NormalTok{),}
    \AttributeTok{catch\_q25 =} \FunctionTok{quantile}\NormalTok{(catch, }\FloatTok{0.25}\NormalTok{, }\AttributeTok{na.rm =} \ConstantTok{TRUE}\NormalTok{),}
    \AttributeTok{catch\_q75 =} \FunctionTok{quantile}\NormalTok{(catch, }\FloatTok{0.75}\NormalTok{, }\AttributeTok{na.rm =} \ConstantTok{TRUE}\NormalTok{),}
    
    \AttributeTok{.groups =} \StringTok{"drop"}
\NormalTok{  )}

\DocumentationTok{\#\# 9.3 График 1: Траектории B/Bmsy}
\NormalTok{p1\_depletion }\OtherTok{\textless{}{-}} \FunctionTok{ggplot}\NormalTok{(summary\_data, }\FunctionTok{aes}\NormalTok{(}\AttributeTok{x =}\NormalTok{ year, }\AttributeTok{color =}\NormalTok{ strategy, }\AttributeTok{fill =}\NormalTok{ strategy)) }\SpecialCharTok{+}
  \FunctionTok{geom\_ribbon}\NormalTok{(}\FunctionTok{aes}\NormalTok{(}\AttributeTok{ymin =}\NormalTok{ B\_Bmsy\_q25, }\AttributeTok{ymax =}\NormalTok{ B\_Bmsy\_q75), }\AttributeTok{alpha =} \FloatTok{0.3}\NormalTok{, }\AttributeTok{color =} \ConstantTok{NA}\NormalTok{) }\SpecialCharTok{+}
  \FunctionTok{geom\_line}\NormalTok{(}\FunctionTok{aes}\NormalTok{(}\AttributeTok{y =}\NormalTok{ B\_Bmsy\_median), }\AttributeTok{size =} \FloatTok{1.5}\NormalTok{) }\SpecialCharTok{+}
  \FunctionTok{geom\_hline}\NormalTok{(}\AttributeTok{yintercept =} \DecValTok{1}\NormalTok{, }\AttributeTok{linetype =} \StringTok{"dashed"}\NormalTok{, }\AttributeTok{color =} \StringTok{"black"}\NormalTok{, }\AttributeTok{alpha =} \FloatTok{0.5}\NormalTok{) }\SpecialCharTok{+}
  \FunctionTok{geom\_hline}\NormalTok{(}\AttributeTok{yintercept =} \FloatTok{0.5}\NormalTok{, }\AttributeTok{linetype =} \StringTok{"dotted"}\NormalTok{, }\AttributeTok{color =} \StringTok{"red"}\NormalTok{, }\AttributeTok{alpha =} \FloatTok{0.5}\NormalTok{) }\SpecialCharTok{+}
  \FunctionTok{scale\_color\_manual}\NormalTok{(}\AttributeTok{values =} \FunctionTok{c}\NormalTok{(}\StringTok{"Fish at Fmsy"} \OtherTok{=} \StringTok{"\#E41A1C"}\NormalTok{, }
                               \StringTok{"MSY Hockey{-}stick"} \OtherTok{=} \StringTok{"\#377EB8"}\NormalTok{,}
                               \StringTok{"ICES Advice Rule"} \OtherTok{=} \StringTok{"\#4DAF4A"}\NormalTok{)) }\SpecialCharTok{+}
  \FunctionTok{scale\_fill\_manual}\NormalTok{(}\AttributeTok{values =} \FunctionTok{c}\NormalTok{(}\StringTok{"Fish at Fmsy"} \OtherTok{=} \StringTok{"\#E41A1C"}\NormalTok{, }
                              \StringTok{"MSY Hockey{-}stick"} \OtherTok{=} \StringTok{"\#377EB8"}\NormalTok{,}
                              \StringTok{"ICES Advice Rule"} \OtherTok{=} \StringTok{"\#4DAF4A"}\NormalTok{)) }\SpecialCharTok{+}
  \FunctionTok{labs}\NormalTok{(}\AttributeTok{title =} \StringTok{"Динамика относительной биомассы (B/Bmsy)"}\NormalTok{,}
       \AttributeTok{subtitle =} \StringTok{"Медиана и межквартильный размах"}\NormalTok{,}
       \AttributeTok{x =} \StringTok{"Год прогноза"}\NormalTok{, }
       \AttributeTok{y =} \StringTok{"B/Bmsy"}\NormalTok{,}
       \AttributeTok{color =} \StringTok{"Стратегия"}\NormalTok{,}
       \AttributeTok{fill =} \StringTok{"Стратегия"}\NormalTok{) }\SpecialCharTok{+}
\NormalTok{  theme\_mse }\SpecialCharTok{+}
  \FunctionTok{annotate}\NormalTok{(}\StringTok{"text"}\NormalTok{, }\AttributeTok{x =} \FunctionTok{max}\NormalTok{(summary\_data}\SpecialCharTok{$}\NormalTok{year), }\AttributeTok{y =} \FloatTok{1.02}\NormalTok{, }
           \AttributeTok{label =} \StringTok{"Bmsy"}\NormalTok{, }\AttributeTok{hjust =} \DecValTok{1}\NormalTok{, }\AttributeTok{vjust =} \DecValTok{0}\NormalTok{, }\AttributeTok{size =} \DecValTok{3}\NormalTok{) }\SpecialCharTok{+}
  \FunctionTok{annotate}\NormalTok{(}\StringTok{"text"}\NormalTok{, }\AttributeTok{x =} \FunctionTok{max}\NormalTok{(summary\_data}\SpecialCharTok{$}\NormalTok{year), }\AttributeTok{y =} \FloatTok{0.52}\NormalTok{, }
           \AttributeTok{label =} \StringTok{"0.5 Bmsy"}\NormalTok{, }\AttributeTok{hjust =} \DecValTok{1}\NormalTok{, }\AttributeTok{vjust =} \DecValTok{0}\NormalTok{, }\AttributeTok{size =} \DecValTok{3}\NormalTok{, }\AttributeTok{color =} \StringTok{"red"}\NormalTok{)}
\end{Highlighting}
\end{Shaded}

\begin{verbatim}
Warning: Using `size` aesthetic for lines was deprecated in ggplot2 3.4.0.
i Please use `linewidth` instead.
\end{verbatim}

\begin{Shaded}
\begin{Highlighting}[]
\NormalTok{p1\_depletion}
\end{Highlighting}
\end{Shaded}

\begin{verbatim}
Warning in grid.Call(C_textBounds, as.graphicsAnnot(x$label), x$x, x$y, :
неизвестна ширина символа 0xd1 в кодировке CP1251
\end{verbatim}

\begin{verbatim}
Warning in grid.Call(C_textBounds, as.graphicsAnnot(x$label), x$x, x$y, :
неизвестна ширина символа 0xf2 в кодировке CP1251
\end{verbatim}

\begin{verbatim}
Warning in grid.Call(C_textBounds, as.graphicsAnnot(x$label), x$x, x$y, :
неизвестна ширина символа 0xf0 в кодировке CP1251
\end{verbatim}

\begin{verbatim}
Warning in grid.Call(C_textBounds, as.graphicsAnnot(x$label), x$x, x$y, :
неизвестна ширина символа 0xe0 в кодировке CP1251
\end{verbatim}

\begin{verbatim}
Warning in grid.Call(C_textBounds, as.graphicsAnnot(x$label), x$x, x$y, :
неизвестна ширина символа 0xf2 в кодировке CP1251
\end{verbatim}

\begin{verbatim}
Warning in grid.Call(C_textBounds, as.graphicsAnnot(x$label), x$x, x$y, :
неизвестна ширина символа 0xe5 в кодировке CP1251
\end{verbatim}

\begin{verbatim}
Warning in grid.Call(C_textBounds, as.graphicsAnnot(x$label), x$x, x$y, :
неизвестна ширина символа 0xe3 в кодировке CP1251
\end{verbatim}

\begin{verbatim}
Warning in grid.Call(C_textBounds, as.graphicsAnnot(x$label), x$x, x$y, :
неизвестна ширина символа 0xe8 в кодировке CP1251
\end{verbatim}

\begin{verbatim}
Warning in grid.Call(C_textBounds, as.graphicsAnnot(x$label), x$x, x$y, :
неизвестна ширина символа 0xff в кодировке CP1251
\end{verbatim}

\begin{verbatim}
Warning in grid.Call(C_textBounds, as.graphicsAnnot(x$label), x$x, x$y, :
неизвестна ширина символа 0xd1 в кодировке CP1251
\end{verbatim}

\begin{verbatim}
Warning in grid.Call(C_textBounds, as.graphicsAnnot(x$label), x$x, x$y, :
неизвестна ширина символа 0xf2 в кодировке CP1251
\end{verbatim}

\begin{verbatim}
Warning in grid.Call(C_textBounds, as.graphicsAnnot(x$label), x$x, x$y, :
неизвестна ширина символа 0xf0 в кодировке CP1251
\end{verbatim}

\begin{verbatim}
Warning in grid.Call(C_textBounds, as.graphicsAnnot(x$label), x$x, x$y, :
неизвестна ширина символа 0xe0 в кодировке CP1251
\end{verbatim}

\begin{verbatim}
Warning in grid.Call(C_textBounds, as.graphicsAnnot(x$label), x$x, x$y, :
неизвестна ширина символа 0xf2 в кодировке CP1251
\end{verbatim}

\begin{verbatim}
Warning in grid.Call(C_textBounds, as.graphicsAnnot(x$label), x$x, x$y, :
неизвестна ширина символа 0xe5 в кодировке CP1251
\end{verbatim}

\begin{verbatim}
Warning in grid.Call(C_textBounds, as.graphicsAnnot(x$label), x$x, x$y, :
неизвестна ширина символа 0xe3 в кодировке CP1251
\end{verbatim}

\begin{verbatim}
Warning in grid.Call(C_textBounds, as.graphicsAnnot(x$label), x$x, x$y, :
неизвестна ширина символа 0xe8 в кодировке CP1251
\end{verbatim}

\begin{verbatim}
Warning in grid.Call(C_textBounds, as.graphicsAnnot(x$label), x$x, x$y, :
неизвестна ширина символа 0xff в кодировке CP1251
\end{verbatim}

\begin{verbatim}
Warning in grid.Call(C_textBounds, as.graphicsAnnot(x$label), x$x, x$y, :
неизвестна ширина символа 0xc4 в кодировке CP1251
\end{verbatim}

\begin{verbatim}
Warning in grid.Call(C_textBounds, as.graphicsAnnot(x$label), x$x, x$y, :
неизвестна ширина символа 0xe8 в кодировке CP1251
\end{verbatim}

\begin{verbatim}
Warning in grid.Call(C_textBounds, as.graphicsAnnot(x$label), x$x, x$y, :
неизвестна ширина символа 0xed в кодировке CP1251
\end{verbatim}

\begin{verbatim}
Warning in grid.Call(C_textBounds, as.graphicsAnnot(x$label), x$x, x$y, :
неизвестна ширина символа 0xe0 в кодировке CP1251
\end{verbatim}

\begin{verbatim}
Warning in grid.Call(C_textBounds, as.graphicsAnnot(x$label), x$x, x$y, :
неизвестна ширина символа 0xec в кодировке CP1251
\end{verbatim}

\begin{verbatim}
Warning in grid.Call(C_textBounds, as.graphicsAnnot(x$label), x$x, x$y, :
неизвестна ширина символа 0xe8 в кодировке CP1251
\end{verbatim}

\begin{verbatim}
Warning in grid.Call(C_textBounds, as.graphicsAnnot(x$label), x$x, x$y, :
неизвестна ширина символа 0xea в кодировке CP1251
\end{verbatim}

\begin{verbatim}
Warning in grid.Call(C_textBounds, as.graphicsAnnot(x$label), x$x, x$y, :
неизвестна ширина символа 0xe0 в кодировке CP1251
\end{verbatim}

\begin{verbatim}
Warning in grid.Call(C_textBounds, as.graphicsAnnot(x$label), x$x, x$y, :
неизвестна ширина символа 0xee в кодировке CP1251
\end{verbatim}

\begin{verbatim}
Warning in grid.Call(C_textBounds, as.graphicsAnnot(x$label), x$x, x$y, :
неизвестна ширина символа 0xf2 в кодировке CP1251
\end{verbatim}

\begin{verbatim}
Warning in grid.Call(C_textBounds, as.graphicsAnnot(x$label), x$x, x$y, :
неизвестна ширина символа 0xed в кодировке CP1251
\end{verbatim}

\begin{verbatim}
Warning in grid.Call(C_textBounds, as.graphicsAnnot(x$label), x$x, x$y, :
неизвестна ширина символа 0xee в кодировке CP1251
\end{verbatim}

\begin{verbatim}
Warning in grid.Call(C_textBounds, as.graphicsAnnot(x$label), x$x, x$y, :
неизвестна ширина символа 0xf1 в кодировке CP1251
\end{verbatim}

\begin{verbatim}
Warning in grid.Call(C_textBounds, as.graphicsAnnot(x$label), x$x, x$y, :
неизвестна ширина символа 0xe8 в кодировке CP1251
\end{verbatim}

\begin{verbatim}
Warning in grid.Call(C_textBounds, as.graphicsAnnot(x$label), x$x, x$y, :
неизвестна ширина символа 0xf2 в кодировке CP1251
\end{verbatim}

\begin{verbatim}
Warning in grid.Call(C_textBounds, as.graphicsAnnot(x$label), x$x, x$y, :
неизвестна ширина символа 0xe5 в кодировке CP1251
\end{verbatim}

\begin{verbatim}
Warning in grid.Call(C_textBounds, as.graphicsAnnot(x$label), x$x, x$y, :
неизвестна ширина символа 0xeb в кодировке CP1251
\end{verbatim}

\begin{verbatim}
Warning in grid.Call(C_textBounds, as.graphicsAnnot(x$label), x$x, x$y, :
неизвестна ширина символа 0xfc в кодировке CP1251
\end{verbatim}

\begin{verbatim}
Warning in grid.Call(C_textBounds, as.graphicsAnnot(x$label), x$x, x$y, :
неизвестна ширина символа 0xed в кодировке CP1251
\end{verbatim}

\begin{verbatim}
Warning in grid.Call(C_textBounds, as.graphicsAnnot(x$label), x$x, x$y, :
неизвестна ширина символа 0xee в кодировке CP1251
\end{verbatim}

\begin{verbatim}
Warning in grid.Call(C_textBounds, as.graphicsAnnot(x$label), x$x, x$y, :
неизвестна ширина символа 0xe9 в кодировке CP1251
\end{verbatim}

\begin{verbatim}
Warning in grid.Call(C_textBounds, as.graphicsAnnot(x$label), x$x, x$y, :
неизвестна ширина символа 0xe1 в кодировке CP1251
\end{verbatim}

\begin{verbatim}
Warning in grid.Call(C_textBounds, as.graphicsAnnot(x$label), x$x, x$y, :
неизвестна ширина символа 0xe8 в кодировке CP1251
\end{verbatim}

\begin{verbatim}
Warning in grid.Call(C_textBounds, as.graphicsAnnot(x$label), x$x, x$y, :
неизвестна ширина символа 0xee в кодировке CP1251
\end{verbatim}

\begin{verbatim}
Warning in grid.Call(C_textBounds, as.graphicsAnnot(x$label), x$x, x$y, :
неизвестна ширина символа 0xec в кодировке CP1251
\end{verbatim}

\begin{verbatim}
Warning in grid.Call(C_textBounds, as.graphicsAnnot(x$label), x$x, x$y, :
неизвестна ширина символа 0xe0 в кодировке CP1251
\end{verbatim}

\begin{verbatim}
Warning in grid.Call(C_textBounds, as.graphicsAnnot(x$label), x$x, x$y, :
неизвестна ширина символа 0xf1 в кодировке CP1251
Warning in grid.Call(C_textBounds, as.graphicsAnnot(x$label), x$x, x$y, :
неизвестна ширина символа 0xf1 в кодировке CP1251
\end{verbatim}

\begin{verbatim}
Warning in grid.Call(C_textBounds, as.graphicsAnnot(x$label), x$x, x$y, :
неизвестна ширина символа 0xfb в кодировке CP1251
\end{verbatim}

\begin{verbatim}
Warning in grid.Call(C_textBounds, as.graphicsAnnot(x$label), x$x, x$y, :
неизвестна ширина символа 0xcc в кодировке CP1251
\end{verbatim}

\begin{verbatim}
Warning in grid.Call(C_textBounds, as.graphicsAnnot(x$label), x$x, x$y, :
неизвестна ширина символа 0xe5 в кодировке CP1251
\end{verbatim}

\begin{verbatim}
Warning in grid.Call(C_textBounds, as.graphicsAnnot(x$label), x$x, x$y, :
неизвестна ширина символа 0xe4 в кодировке CP1251
\end{verbatim}

\begin{verbatim}
Warning in grid.Call(C_textBounds, as.graphicsAnnot(x$label), x$x, x$y, :
неизвестна ширина символа 0xe8 в кодировке CP1251
\end{verbatim}

\begin{verbatim}
Warning in grid.Call(C_textBounds, as.graphicsAnnot(x$label), x$x, x$y, :
неизвестна ширина символа 0xe0 в кодировке CP1251
\end{verbatim}

\begin{verbatim}
Warning in grid.Call(C_textBounds, as.graphicsAnnot(x$label), x$x, x$y, :
неизвестна ширина символа 0xed в кодировке CP1251
\end{verbatim}

\begin{verbatim}
Warning in grid.Call(C_textBounds, as.graphicsAnnot(x$label), x$x, x$y, :
неизвестна ширина символа 0xe0 в кодировке CP1251
\end{verbatim}

\begin{verbatim}
Warning in grid.Call(C_textBounds, as.graphicsAnnot(x$label), x$x, x$y, :
неизвестна ширина символа 0xe8 в кодировке CP1251
\end{verbatim}

\begin{verbatim}
Warning in grid.Call(C_textBounds, as.graphicsAnnot(x$label), x$x, x$y, :
неизвестна ширина символа 0xec в кодировке CP1251
\end{verbatim}

\begin{verbatim}
Warning in grid.Call(C_textBounds, as.graphicsAnnot(x$label), x$x, x$y, :
неизвестна ширина символа 0xe5 в кодировке CP1251
\end{verbatim}

\begin{verbatim}
Warning in grid.Call(C_textBounds, as.graphicsAnnot(x$label), x$x, x$y, :
неизвестна ширина символа 0xe6 в кодировке CP1251
\end{verbatim}

\begin{verbatim}
Warning in grid.Call(C_textBounds, as.graphicsAnnot(x$label), x$x, x$y, :
неизвестна ширина символа 0xea в кодировке CP1251
\end{verbatim}

\begin{verbatim}
Warning in grid.Call(C_textBounds, as.graphicsAnnot(x$label), x$x, x$y, :
неизвестна ширина символа 0xe2 в кодировке CP1251
\end{verbatim}

\begin{verbatim}
Warning in grid.Call(C_textBounds, as.graphicsAnnot(x$label), x$x, x$y, :
неизвестна ширина символа 0xe0 в кодировке CP1251
\end{verbatim}

\begin{verbatim}
Warning in grid.Call(C_textBounds, as.graphicsAnnot(x$label), x$x, x$y, :
неизвестна ширина символа 0xf0 в кодировке CP1251
\end{verbatim}

\begin{verbatim}
Warning in grid.Call(C_textBounds, as.graphicsAnnot(x$label), x$x, x$y, :
неизвестна ширина символа 0xf2 в кодировке CP1251
\end{verbatim}

\begin{verbatim}
Warning in grid.Call(C_textBounds, as.graphicsAnnot(x$label), x$x, x$y, :
неизвестна ширина символа 0xe8 в кодировке CP1251
\end{verbatim}

\begin{verbatim}
Warning in grid.Call(C_textBounds, as.graphicsAnnot(x$label), x$x, x$y, :
неизвестна ширина символа 0xeb в кодировке CP1251
\end{verbatim}

\begin{verbatim}
Warning in grid.Call(C_textBounds, as.graphicsAnnot(x$label), x$x, x$y, :
неизвестна ширина символа 0xfc в кодировке CP1251
\end{verbatim}

\begin{verbatim}
Warning in grid.Call(C_textBounds, as.graphicsAnnot(x$label), x$x, x$y, :
неизвестна ширина символа 0xed в кодировке CP1251
\end{verbatim}

\begin{verbatim}
Warning in grid.Call(C_textBounds, as.graphicsAnnot(x$label), x$x, x$y, :
неизвестна ширина символа 0xfb в кодировке CP1251
\end{verbatim}

\begin{verbatim}
Warning in grid.Call(C_textBounds, as.graphicsAnnot(x$label), x$x, x$y, :
неизвестна ширина символа 0xe9 в кодировке CP1251
\end{verbatim}

\begin{verbatim}
Warning in grid.Call(C_textBounds, as.graphicsAnnot(x$label), x$x, x$y, :
неизвестна ширина символа 0xf0 в кодировке CP1251
\end{verbatim}

\begin{verbatim}
Warning in grid.Call(C_textBounds, as.graphicsAnnot(x$label), x$x, x$y, :
неизвестна ширина символа 0xe0 в кодировке CP1251
\end{verbatim}

\begin{verbatim}
Warning in grid.Call(C_textBounds, as.graphicsAnnot(x$label), x$x, x$y, :
неизвестна ширина символа 0xe7 в кодировке CP1251
\end{verbatim}

\begin{verbatim}
Warning in grid.Call(C_textBounds, as.graphicsAnnot(x$label), x$x, x$y, :
неизвестна ширина символа 0xec в кодировке CP1251
\end{verbatim}

\begin{verbatim}
Warning in grid.Call(C_textBounds, as.graphicsAnnot(x$label), x$x, x$y, :
неизвестна ширина символа 0xe0 в кодировке CP1251
\end{verbatim}

\begin{verbatim}
Warning in grid.Call(C_textBounds, as.graphicsAnnot(x$label), x$x, x$y, :
неизвестна ширина символа 0xf5 в кодировке CP1251
\end{verbatim}

\begin{verbatim}
Warning in grid.Call(C_textBounds, as.graphicsAnnot(x$label), x$x, x$y, :
неизвестна ширина символа 0xc3 в кодировке CP1251
\end{verbatim}

\begin{verbatim}
Warning in grid.Call(C_textBounds, as.graphicsAnnot(x$label), x$x, x$y, :
неизвестна ширина символа 0xee в кодировке CP1251
\end{verbatim}

\begin{verbatim}
Warning in grid.Call(C_textBounds, as.graphicsAnnot(x$label), x$x, x$y, :
неизвестна ширина символа 0xe4 в кодировке CP1251
\end{verbatim}

\begin{verbatim}
Warning in grid.Call(C_textBounds, as.graphicsAnnot(x$label), x$x, x$y, :
неизвестна ширина символа 0xef в кодировке CP1251
\end{verbatim}

\begin{verbatim}
Warning in grid.Call(C_textBounds, as.graphicsAnnot(x$label), x$x, x$y, :
неизвестна ширина символа 0xf0 в кодировке CP1251
\end{verbatim}

\begin{verbatim}
Warning in grid.Call(C_textBounds, as.graphicsAnnot(x$label), x$x, x$y, :
неизвестна ширина символа 0xee в кодировке CP1251
\end{verbatim}

\begin{verbatim}
Warning in grid.Call(C_textBounds, as.graphicsAnnot(x$label), x$x, x$y, :
неизвестна ширина символа 0xe3 в кодировке CP1251
\end{verbatim}

\begin{verbatim}
Warning in grid.Call(C_textBounds, as.graphicsAnnot(x$label), x$x, x$y, :
неизвестна ширина символа 0xed в кодировке CP1251
\end{verbatim}

\begin{verbatim}
Warning in grid.Call(C_textBounds, as.graphicsAnnot(x$label), x$x, x$y, :
неизвестна ширина символа 0xee в кодировке CP1251
\end{verbatim}

\begin{verbatim}
Warning in grid.Call(C_textBounds, as.graphicsAnnot(x$label), x$x, x$y, :
неизвестна ширина символа 0xe7 в кодировке CP1251
\end{verbatim}

\begin{verbatim}
Warning in grid.Call(C_textBounds, as.graphicsAnnot(x$label), x$x, x$y, :
неизвестна ширина символа 0xe0 в кодировке CP1251
\end{verbatim}

\begin{verbatim}
Warning in grid.Call.graphics(C_text, as.graphicsAnnot(x$label), x$x, x$y, :
неизвестна ширина символа 0xc3 в кодировке CP1251
\end{verbatim}

\begin{verbatim}
Warning in grid.Call.graphics(C_text, as.graphicsAnnot(x$label), x$x, x$y, :
неизвестна ширина символа 0xee в кодировке CP1251
\end{verbatim}

\begin{verbatim}
Warning in grid.Call.graphics(C_text, as.graphicsAnnot(x$label), x$x, x$y, :
неизвестна ширина символа 0xe4 в кодировке CP1251
\end{verbatim}

\begin{verbatim}
Warning in grid.Call.graphics(C_text, as.graphicsAnnot(x$label), x$x, x$y, :
неизвестна ширина символа 0xef в кодировке CP1251
\end{verbatim}

\begin{verbatim}
Warning in grid.Call.graphics(C_text, as.graphicsAnnot(x$label), x$x, x$y, :
неизвестна ширина символа 0xf0 в кодировке CP1251
\end{verbatim}

\begin{verbatim}
Warning in grid.Call.graphics(C_text, as.graphicsAnnot(x$label), x$x, x$y, :
неизвестна ширина символа 0xee в кодировке CP1251
\end{verbatim}

\begin{verbatim}
Warning in grid.Call.graphics(C_text, as.graphicsAnnot(x$label), x$x, x$y, :
неизвестна ширина символа 0xe3 в кодировке CP1251
\end{verbatim}

\begin{verbatim}
Warning in grid.Call.graphics(C_text, as.graphicsAnnot(x$label), x$x, x$y, :
неизвестна ширина символа 0xed в кодировке CP1251
\end{verbatim}

\begin{verbatim}
Warning in grid.Call.graphics(C_text, as.graphicsAnnot(x$label), x$x, x$y, :
неизвестна ширина символа 0xee в кодировке CP1251
\end{verbatim}

\begin{verbatim}
Warning in grid.Call.graphics(C_text, as.graphicsAnnot(x$label), x$x, x$y, :
неизвестна ширина символа 0xe7 в кодировке CP1251
\end{verbatim}

\begin{verbatim}
Warning in grid.Call.graphics(C_text, as.graphicsAnnot(x$label), x$x, x$y, :
неизвестна ширина символа 0xe0 в кодировке CP1251
\end{verbatim}

\begin{verbatim}
Warning in grid.Call.graphics(C_text, as.graphicsAnnot(x$label), x$x, x$y, :
неизвестна ширина символа 0xd1 в кодировке CP1251
\end{verbatim}

\begin{verbatim}
Warning in grid.Call.graphics(C_text, as.graphicsAnnot(x$label), x$x, x$y, :
неизвестна ширина символа 0xf2 в кодировке CP1251
\end{verbatim}

\begin{verbatim}
Warning in grid.Call.graphics(C_text, as.graphicsAnnot(x$label), x$x, x$y, :
неизвестна ширина символа 0xf0 в кодировке CP1251
\end{verbatim}

\begin{verbatim}
Warning in grid.Call.graphics(C_text, as.graphicsAnnot(x$label), x$x, x$y, :
неизвестна ширина символа 0xe0 в кодировке CP1251
\end{verbatim}

\begin{verbatim}
Warning in grid.Call.graphics(C_text, as.graphicsAnnot(x$label), x$x, x$y, :
неизвестна ширина символа 0xf2 в кодировке CP1251
\end{verbatim}

\begin{verbatim}
Warning in grid.Call.graphics(C_text, as.graphicsAnnot(x$label), x$x, x$y, :
неизвестна ширина символа 0xe5 в кодировке CP1251
\end{verbatim}

\begin{verbatim}
Warning in grid.Call.graphics(C_text, as.graphicsAnnot(x$label), x$x, x$y, :
неизвестна ширина символа 0xe3 в кодировке CP1251
\end{verbatim}

\begin{verbatim}
Warning in grid.Call.graphics(C_text, as.graphicsAnnot(x$label), x$x, x$y, :
неизвестна ширина символа 0xe8 в кодировке CP1251
\end{verbatim}

\begin{verbatim}
Warning in grid.Call.graphics(C_text, as.graphicsAnnot(x$label), x$x, x$y, :
неизвестна ширина символа 0xff в кодировке CP1251
\end{verbatim}

\begin{verbatim}
Warning in grid.Call.graphics(C_text, as.graphicsAnnot(x$label), x$x, x$y, :
неизвестна ширина символа 0xcc в кодировке CP1251
\end{verbatim}

\begin{verbatim}
Warning in grid.Call.graphics(C_text, as.graphicsAnnot(x$label), x$x, x$y, :
неизвестна ширина символа 0xe5 в кодировке CP1251
\end{verbatim}

\begin{verbatim}
Warning in grid.Call.graphics(C_text, as.graphicsAnnot(x$label), x$x, x$y, :
неизвестна ширина символа 0xe4 в кодировке CP1251
\end{verbatim}

\begin{verbatim}
Warning in grid.Call.graphics(C_text, as.graphicsAnnot(x$label), x$x, x$y, :
неизвестна ширина символа 0xe8 в кодировке CP1251
\end{verbatim}

\begin{verbatim}
Warning in grid.Call.graphics(C_text, as.graphicsAnnot(x$label), x$x, x$y, :
неизвестна ширина символа 0xe0 в кодировке CP1251
\end{verbatim}

\begin{verbatim}
Warning in grid.Call.graphics(C_text, as.graphicsAnnot(x$label), x$x, x$y, :
неизвестна ширина символа 0xed в кодировке CP1251
\end{verbatim}

\begin{verbatim}
Warning in grid.Call.graphics(C_text, as.graphicsAnnot(x$label), x$x, x$y, :
неизвестна ширина символа 0xe0 в кодировке CP1251
\end{verbatim}

\begin{verbatim}
Warning in grid.Call.graphics(C_text, as.graphicsAnnot(x$label), x$x, x$y, :
неизвестна ширина символа 0xe8 в кодировке CP1251
\end{verbatim}

\begin{verbatim}
Warning in grid.Call.graphics(C_text, as.graphicsAnnot(x$label), x$x, x$y, :
неизвестна ширина символа 0xec в кодировке CP1251
\end{verbatim}

\begin{verbatim}
Warning in grid.Call.graphics(C_text, as.graphicsAnnot(x$label), x$x, x$y, :
неизвестна ширина символа 0xe5 в кодировке CP1251
\end{verbatim}

\begin{verbatim}
Warning in grid.Call.graphics(C_text, as.graphicsAnnot(x$label), x$x, x$y, :
неизвестна ширина символа 0xe6 в кодировке CP1251
\end{verbatim}

\begin{verbatim}
Warning in grid.Call.graphics(C_text, as.graphicsAnnot(x$label), x$x, x$y, :
неизвестна ширина символа 0xea в кодировке CP1251
\end{verbatim}

\begin{verbatim}
Warning in grid.Call.graphics(C_text, as.graphicsAnnot(x$label), x$x, x$y, :
неизвестна ширина символа 0xe2 в кодировке CP1251
\end{verbatim}

\begin{verbatim}
Warning in grid.Call.graphics(C_text, as.graphicsAnnot(x$label), x$x, x$y, :
неизвестна ширина символа 0xe0 в кодировке CP1251
\end{verbatim}

\begin{verbatim}
Warning in grid.Call.graphics(C_text, as.graphicsAnnot(x$label), x$x, x$y, :
неизвестна ширина символа 0xf0 в кодировке CP1251
\end{verbatim}

\begin{verbatim}
Warning in grid.Call.graphics(C_text, as.graphicsAnnot(x$label), x$x, x$y, :
неизвестна ширина символа 0xf2 в кодировке CP1251
\end{verbatim}

\begin{verbatim}
Warning in grid.Call.graphics(C_text, as.graphicsAnnot(x$label), x$x, x$y, :
неизвестна ширина символа 0xe8 в кодировке CP1251
\end{verbatim}

\begin{verbatim}
Warning in grid.Call.graphics(C_text, as.graphicsAnnot(x$label), x$x, x$y, :
неизвестна ширина символа 0xeb в кодировке CP1251
\end{verbatim}

\begin{verbatim}
Warning in grid.Call.graphics(C_text, as.graphicsAnnot(x$label), x$x, x$y, :
неизвестна ширина символа 0xfc в кодировке CP1251
\end{verbatim}

\begin{verbatim}
Warning in grid.Call.graphics(C_text, as.graphicsAnnot(x$label), x$x, x$y, :
неизвестна ширина символа 0xed в кодировке CP1251
\end{verbatim}

\begin{verbatim}
Warning in grid.Call.graphics(C_text, as.graphicsAnnot(x$label), x$x, x$y, :
неизвестна ширина символа 0xfb в кодировке CP1251
\end{verbatim}

\begin{verbatim}
Warning in grid.Call.graphics(C_text, as.graphicsAnnot(x$label), x$x, x$y, :
неизвестна ширина символа 0xe9 в кодировке CP1251
\end{verbatim}

\begin{verbatim}
Warning in grid.Call.graphics(C_text, as.graphicsAnnot(x$label), x$x, x$y, :
неизвестна ширина символа 0xf0 в кодировке CP1251
\end{verbatim}

\begin{verbatim}
Warning in grid.Call.graphics(C_text, as.graphicsAnnot(x$label), x$x, x$y, :
неизвестна ширина символа 0xe0 в кодировке CP1251
\end{verbatim}

\begin{verbatim}
Warning in grid.Call.graphics(C_text, as.graphicsAnnot(x$label), x$x, x$y, :
неизвестна ширина символа 0xe7 в кодировке CP1251
\end{verbatim}

\begin{verbatim}
Warning in grid.Call.graphics(C_text, as.graphicsAnnot(x$label), x$x, x$y, :
неизвестна ширина символа 0xec в кодировке CP1251
\end{verbatim}

\begin{verbatim}
Warning in grid.Call.graphics(C_text, as.graphicsAnnot(x$label), x$x, x$y, :
неизвестна ширина символа 0xe0 в кодировке CP1251
\end{verbatim}

\begin{verbatim}
Warning in grid.Call.graphics(C_text, as.graphicsAnnot(x$label), x$x, x$y, :
неизвестна ширина символа 0xf5 в кодировке CP1251
\end{verbatim}

\begin{verbatim}
Warning in grid.Call.graphics(C_text, as.graphicsAnnot(x$label), x$x, x$y, :
неизвестна ширина символа 0xc4 в кодировке CP1251
\end{verbatim}

\begin{verbatim}
Warning in grid.Call.graphics(C_text, as.graphicsAnnot(x$label), x$x, x$y, :
неизвестна ширина символа 0xe8 в кодировке CP1251
\end{verbatim}

\begin{verbatim}
Warning in grid.Call.graphics(C_text, as.graphicsAnnot(x$label), x$x, x$y, :
неизвестна ширина символа 0xed в кодировке CP1251
\end{verbatim}

\begin{verbatim}
Warning in grid.Call.graphics(C_text, as.graphicsAnnot(x$label), x$x, x$y, :
неизвестна ширина символа 0xe0 в кодировке CP1251
\end{verbatim}

\begin{verbatim}
Warning in grid.Call.graphics(C_text, as.graphicsAnnot(x$label), x$x, x$y, :
неизвестна ширина символа 0xec в кодировке CP1251
\end{verbatim}

\begin{verbatim}
Warning in grid.Call.graphics(C_text, as.graphicsAnnot(x$label), x$x, x$y, :
неизвестна ширина символа 0xe8 в кодировке CP1251
\end{verbatim}

\begin{verbatim}
Warning in grid.Call.graphics(C_text, as.graphicsAnnot(x$label), x$x, x$y, :
неизвестна ширина символа 0xea в кодировке CP1251
\end{verbatim}

\begin{verbatim}
Warning in grid.Call.graphics(C_text, as.graphicsAnnot(x$label), x$x, x$y, :
неизвестна ширина символа 0xe0 в кодировке CP1251
\end{verbatim}

\begin{verbatim}
Warning in grid.Call.graphics(C_text, as.graphicsAnnot(x$label), x$x, x$y, :
неизвестна ширина символа 0xee в кодировке CP1251
\end{verbatim}

\begin{verbatim}
Warning in grid.Call.graphics(C_text, as.graphicsAnnot(x$label), x$x, x$y, :
неизвестна ширина символа 0xf2 в кодировке CP1251
\end{verbatim}

\begin{verbatim}
Warning in grid.Call.graphics(C_text, as.graphicsAnnot(x$label), x$x, x$y, :
неизвестна ширина символа 0xed в кодировке CP1251
\end{verbatim}

\begin{verbatim}
Warning in grid.Call.graphics(C_text, as.graphicsAnnot(x$label), x$x, x$y, :
неизвестна ширина символа 0xee в кодировке CP1251
\end{verbatim}

\begin{verbatim}
Warning in grid.Call.graphics(C_text, as.graphicsAnnot(x$label), x$x, x$y, :
неизвестна ширина символа 0xf1 в кодировке CP1251
\end{verbatim}

\begin{verbatim}
Warning in grid.Call.graphics(C_text, as.graphicsAnnot(x$label), x$x, x$y, :
неизвестна ширина символа 0xe8 в кодировке CP1251
\end{verbatim}

\begin{verbatim}
Warning in grid.Call.graphics(C_text, as.graphicsAnnot(x$label), x$x, x$y, :
неизвестна ширина символа 0xf2 в кодировке CP1251
\end{verbatim}

\begin{verbatim}
Warning in grid.Call.graphics(C_text, as.graphicsAnnot(x$label), x$x, x$y, :
неизвестна ширина символа 0xe5 в кодировке CP1251
\end{verbatim}

\begin{verbatim}
Warning in grid.Call.graphics(C_text, as.graphicsAnnot(x$label), x$x, x$y, :
неизвестна ширина символа 0xeb в кодировке CP1251
\end{verbatim}

\begin{verbatim}
Warning in grid.Call.graphics(C_text, as.graphicsAnnot(x$label), x$x, x$y, :
неизвестна ширина символа 0xfc в кодировке CP1251
\end{verbatim}

\begin{verbatim}
Warning in grid.Call.graphics(C_text, as.graphicsAnnot(x$label), x$x, x$y, :
неизвестна ширина символа 0xed в кодировке CP1251
\end{verbatim}

\begin{verbatim}
Warning in grid.Call.graphics(C_text, as.graphicsAnnot(x$label), x$x, x$y, :
неизвестна ширина символа 0xee в кодировке CP1251
\end{verbatim}

\begin{verbatim}
Warning in grid.Call.graphics(C_text, as.graphicsAnnot(x$label), x$x, x$y, :
неизвестна ширина символа 0xe9 в кодировке CP1251
\end{verbatim}

\begin{verbatim}
Warning in grid.Call.graphics(C_text, as.graphicsAnnot(x$label), x$x, x$y, :
неизвестна ширина символа 0xe1 в кодировке CP1251
\end{verbatim}

\begin{verbatim}
Warning in grid.Call.graphics(C_text, as.graphicsAnnot(x$label), x$x, x$y, :
неизвестна ширина символа 0xe8 в кодировке CP1251
\end{verbatim}

\begin{verbatim}
Warning in grid.Call.graphics(C_text, as.graphicsAnnot(x$label), x$x, x$y, :
неизвестна ширина символа 0xee в кодировке CP1251
\end{verbatim}

\begin{verbatim}
Warning in grid.Call.graphics(C_text, as.graphicsAnnot(x$label), x$x, x$y, :
неизвестна ширина символа 0xec в кодировке CP1251
\end{verbatim}

\begin{verbatim}
Warning in grid.Call.graphics(C_text, as.graphicsAnnot(x$label), x$x, x$y, :
неизвестна ширина символа 0xe0 в кодировке CP1251
\end{verbatim}

\begin{verbatim}
Warning in grid.Call.graphics(C_text, as.graphicsAnnot(x$label), x$x, x$y, :
неизвестна ширина символа 0xf1 в кодировке CP1251
Warning in grid.Call.graphics(C_text, as.graphicsAnnot(x$label), x$x, x$y, :
неизвестна ширина символа 0xf1 в кодировке CP1251
\end{verbatim}

\begin{verbatim}
Warning in grid.Call.graphics(C_text, as.graphicsAnnot(x$label), x$x, x$y, :
неизвестна ширина символа 0xfb в кодировке CP1251
\end{verbatim}

\pandocbounded{\includegraphics[keepaspectratio]{chapter15_files/figure-pdf/unnamed-chunk-1-1.pdf}}

\begin{Shaded}
\begin{Highlighting}[]
\DocumentationTok{\#\# 9.4 График 2: Траектории F/Fmsy}
\NormalTok{p2\_fishing }\OtherTok{\textless{}{-}} \FunctionTok{ggplot}\NormalTok{(summary\_data, }\FunctionTok{aes}\NormalTok{(}\AttributeTok{x =}\NormalTok{ year, }\AttributeTok{color =}\NormalTok{ strategy, }\AttributeTok{fill =}\NormalTok{ strategy)) }\SpecialCharTok{+}
  \FunctionTok{geom\_ribbon}\NormalTok{(}\FunctionTok{aes}\NormalTok{(}\AttributeTok{ymin =}\NormalTok{ F\_Fmsy\_q25, }\AttributeTok{ymax =}\NormalTok{ F\_Fmsy\_q75), }\AttributeTok{alpha =} \FloatTok{0.3}\NormalTok{, }\AttributeTok{color =} \ConstantTok{NA}\NormalTok{) }\SpecialCharTok{+}
  \FunctionTok{geom\_line}\NormalTok{(}\FunctionTok{aes}\NormalTok{(}\AttributeTok{y =}\NormalTok{ F\_Fmsy\_median), }\AttributeTok{size =} \FloatTok{1.5}\NormalTok{) }\SpecialCharTok{+}
  \FunctionTok{geom\_hline}\NormalTok{(}\AttributeTok{yintercept =} \DecValTok{1}\NormalTok{, }\AttributeTok{linetype =} \StringTok{"dashed"}\NormalTok{, }\AttributeTok{color =} \StringTok{"black"}\NormalTok{, }\AttributeTok{alpha =} \FloatTok{0.5}\NormalTok{) }\SpecialCharTok{+}
  \FunctionTok{scale\_color\_manual}\NormalTok{(}\AttributeTok{values =} \FunctionTok{c}\NormalTok{(}\StringTok{"Fish at Fmsy"} \OtherTok{=} \StringTok{"\#E41A1C"}\NormalTok{, }
                               \StringTok{"MSY Hockey{-}stick"} \OtherTok{=} \StringTok{"\#377EB8"}\NormalTok{,}
                               \StringTok{"ICES Advice Rule"} \OtherTok{=} \StringTok{"\#4DAF4A"}\NormalTok{)) }\SpecialCharTok{+}
  \FunctionTok{scale\_fill\_manual}\NormalTok{(}\AttributeTok{values =} \FunctionTok{c}\NormalTok{(}\StringTok{"Fish at Fmsy"} \OtherTok{=} \StringTok{"\#E41A1C"}\NormalTok{, }
                              \StringTok{"MSY Hockey{-}stick"} \OtherTok{=} \StringTok{"\#377EB8"}\NormalTok{,}
                              \StringTok{"ICES Advice Rule"} \OtherTok{=} \StringTok{"\#4DAF4A"}\NormalTok{)) }\SpecialCharTok{+}
  \FunctionTok{labs}\NormalTok{(}\AttributeTok{title =} \StringTok{"Динамика промысловой смертности (F/Fmsy)"}\NormalTok{,}
       \AttributeTok{subtitle =} \StringTok{"Медиана и межквартильный размах"}\NormalTok{,}
       \AttributeTok{x =} \StringTok{"Год прогноза"}\NormalTok{, }
       \AttributeTok{y =} \StringTok{"F/Fmsy"}\NormalTok{,}
       \AttributeTok{color =} \StringTok{"Стратегия"}\NormalTok{,}
       \AttributeTok{fill =} \StringTok{"Стратегия"}\NormalTok{) }\SpecialCharTok{+}
\NormalTok{  theme\_mse }\SpecialCharTok{+}
  \FunctionTok{coord\_cartesian}\NormalTok{(}\AttributeTok{ylim =} \FunctionTok{c}\NormalTok{(}\DecValTok{0}\NormalTok{, }\DecValTok{2}\NormalTok{)) }\SpecialCharTok{+}
  \FunctionTok{annotate}\NormalTok{(}\StringTok{"text"}\NormalTok{, }\AttributeTok{x =} \FunctionTok{max}\NormalTok{(summary\_data}\SpecialCharTok{$}\NormalTok{year), }\AttributeTok{y =} \FloatTok{1.02}\NormalTok{, }
           \AttributeTok{label =} \StringTok{"Fmsy"}\NormalTok{, }\AttributeTok{hjust =} \DecValTok{1}\NormalTok{, }\AttributeTok{vjust =} \DecValTok{0}\NormalTok{, }\AttributeTok{size =} \DecValTok{3}\NormalTok{)}

\NormalTok{p2\_fishing}
\end{Highlighting}
\end{Shaded}

\begin{verbatim}
Warning in grid.Call(C_textBounds, as.graphicsAnnot(x$label), x$x, x$y, :
неизвестна ширина символа 0xd1 в кодировке CP1251
\end{verbatim}

\begin{verbatim}
Warning in grid.Call(C_textBounds, as.graphicsAnnot(x$label), x$x, x$y, :
неизвестна ширина символа 0xf2 в кодировке CP1251
\end{verbatim}

\begin{verbatim}
Warning in grid.Call(C_textBounds, as.graphicsAnnot(x$label), x$x, x$y, :
неизвестна ширина символа 0xf0 в кодировке CP1251
\end{verbatim}

\begin{verbatim}
Warning in grid.Call(C_textBounds, as.graphicsAnnot(x$label), x$x, x$y, :
неизвестна ширина символа 0xe0 в кодировке CP1251
\end{verbatim}

\begin{verbatim}
Warning in grid.Call(C_textBounds, as.graphicsAnnot(x$label), x$x, x$y, :
неизвестна ширина символа 0xf2 в кодировке CP1251
\end{verbatim}

\begin{verbatim}
Warning in grid.Call(C_textBounds, as.graphicsAnnot(x$label), x$x, x$y, :
неизвестна ширина символа 0xe5 в кодировке CP1251
\end{verbatim}

\begin{verbatim}
Warning in grid.Call(C_textBounds, as.graphicsAnnot(x$label), x$x, x$y, :
неизвестна ширина символа 0xe3 в кодировке CP1251
\end{verbatim}

\begin{verbatim}
Warning in grid.Call(C_textBounds, as.graphicsAnnot(x$label), x$x, x$y, :
неизвестна ширина символа 0xe8 в кодировке CP1251
\end{verbatim}

\begin{verbatim}
Warning in grid.Call(C_textBounds, as.graphicsAnnot(x$label), x$x, x$y, :
неизвестна ширина символа 0xff в кодировке CP1251
\end{verbatim}

\begin{verbatim}
Warning in grid.Call(C_textBounds, as.graphicsAnnot(x$label), x$x, x$y, :
неизвестна ширина символа 0xd1 в кодировке CP1251
\end{verbatim}

\begin{verbatim}
Warning in grid.Call(C_textBounds, as.graphicsAnnot(x$label), x$x, x$y, :
неизвестна ширина символа 0xf2 в кодировке CP1251
\end{verbatim}

\begin{verbatim}
Warning in grid.Call(C_textBounds, as.graphicsAnnot(x$label), x$x, x$y, :
неизвестна ширина символа 0xf0 в кодировке CP1251
\end{verbatim}

\begin{verbatim}
Warning in grid.Call(C_textBounds, as.graphicsAnnot(x$label), x$x, x$y, :
неизвестна ширина символа 0xe0 в кодировке CP1251
\end{verbatim}

\begin{verbatim}
Warning in grid.Call(C_textBounds, as.graphicsAnnot(x$label), x$x, x$y, :
неизвестна ширина символа 0xf2 в кодировке CP1251
\end{verbatim}

\begin{verbatim}
Warning in grid.Call(C_textBounds, as.graphicsAnnot(x$label), x$x, x$y, :
неизвестна ширина символа 0xe5 в кодировке CP1251
\end{verbatim}

\begin{verbatim}
Warning in grid.Call(C_textBounds, as.graphicsAnnot(x$label), x$x, x$y, :
неизвестна ширина символа 0xe3 в кодировке CP1251
\end{verbatim}

\begin{verbatim}
Warning in grid.Call(C_textBounds, as.graphicsAnnot(x$label), x$x, x$y, :
неизвестна ширина символа 0xe8 в кодировке CP1251
\end{verbatim}

\begin{verbatim}
Warning in grid.Call(C_textBounds, as.graphicsAnnot(x$label), x$x, x$y, :
неизвестна ширина символа 0xff в кодировке CP1251
\end{verbatim}

\begin{verbatim}
Warning in grid.Call(C_textBounds, as.graphicsAnnot(x$label), x$x, x$y, :
неизвестна ширина символа 0xc4 в кодировке CP1251
\end{verbatim}

\begin{verbatim}
Warning in grid.Call(C_textBounds, as.graphicsAnnot(x$label), x$x, x$y, :
неизвестна ширина символа 0xe8 в кодировке CP1251
\end{verbatim}

\begin{verbatim}
Warning in grid.Call(C_textBounds, as.graphicsAnnot(x$label), x$x, x$y, :
неизвестна ширина символа 0xed в кодировке CP1251
\end{verbatim}

\begin{verbatim}
Warning in grid.Call(C_textBounds, as.graphicsAnnot(x$label), x$x, x$y, :
неизвестна ширина символа 0xe0 в кодировке CP1251
\end{verbatim}

\begin{verbatim}
Warning in grid.Call(C_textBounds, as.graphicsAnnot(x$label), x$x, x$y, :
неизвестна ширина символа 0xec в кодировке CP1251
\end{verbatim}

\begin{verbatim}
Warning in grid.Call(C_textBounds, as.graphicsAnnot(x$label), x$x, x$y, :
неизвестна ширина символа 0xe8 в кодировке CP1251
\end{verbatim}

\begin{verbatim}
Warning in grid.Call(C_textBounds, as.graphicsAnnot(x$label), x$x, x$y, :
неизвестна ширина символа 0xea в кодировке CP1251
\end{verbatim}

\begin{verbatim}
Warning in grid.Call(C_textBounds, as.graphicsAnnot(x$label), x$x, x$y, :
неизвестна ширина символа 0xe0 в кодировке CP1251
\end{verbatim}

\begin{verbatim}
Warning in grid.Call(C_textBounds, as.graphicsAnnot(x$label), x$x, x$y, :
неизвестна ширина символа 0xef в кодировке CP1251
\end{verbatim}

\begin{verbatim}
Warning in grid.Call(C_textBounds, as.graphicsAnnot(x$label), x$x, x$y, :
неизвестна ширина символа 0xf0 в кодировке CP1251
\end{verbatim}

\begin{verbatim}
Warning in grid.Call(C_textBounds, as.graphicsAnnot(x$label), x$x, x$y, :
неизвестна ширина символа 0xee в кодировке CP1251
\end{verbatim}

\begin{verbatim}
Warning in grid.Call(C_textBounds, as.graphicsAnnot(x$label), x$x, x$y, :
неизвестна ширина символа 0xec в кодировке CP1251
\end{verbatim}

\begin{verbatim}
Warning in grid.Call(C_textBounds, as.graphicsAnnot(x$label), x$x, x$y, :
неизвестна ширина символа 0xfb в кодировке CP1251
\end{verbatim}

\begin{verbatim}
Warning in grid.Call(C_textBounds, as.graphicsAnnot(x$label), x$x, x$y, :
неизвестна ширина символа 0xf1 в кодировке CP1251
\end{verbatim}

\begin{verbatim}
Warning in grid.Call(C_textBounds, as.graphicsAnnot(x$label), x$x, x$y, :
неизвестна ширина символа 0xeb в кодировке CP1251
\end{verbatim}

\begin{verbatim}
Warning in grid.Call(C_textBounds, as.graphicsAnnot(x$label), x$x, x$y, :
неизвестна ширина символа 0xee в кодировке CP1251
\end{verbatim}

\begin{verbatim}
Warning in grid.Call(C_textBounds, as.graphicsAnnot(x$label), x$x, x$y, :
неизвестна ширина символа 0xe2 в кодировке CP1251
\end{verbatim}

\begin{verbatim}
Warning in grid.Call(C_textBounds, as.graphicsAnnot(x$label), x$x, x$y, :
неизвестна ширина символа 0xee в кодировке CP1251
\end{verbatim}

\begin{verbatim}
Warning in grid.Call(C_textBounds, as.graphicsAnnot(x$label), x$x, x$y, :
неизвестна ширина символа 0xe9 в кодировке CP1251
\end{verbatim}

\begin{verbatim}
Warning in grid.Call(C_textBounds, as.graphicsAnnot(x$label), x$x, x$y, :
неизвестна ширина символа 0xf1 в кодировке CP1251
\end{verbatim}

\begin{verbatim}
Warning in grid.Call(C_textBounds, as.graphicsAnnot(x$label), x$x, x$y, :
неизвестна ширина символа 0xec в кодировке CP1251
\end{verbatim}

\begin{verbatim}
Warning in grid.Call(C_textBounds, as.graphicsAnnot(x$label), x$x, x$y, :
неизвестна ширина символа 0xe5 в кодировке CP1251
\end{verbatim}

\begin{verbatim}
Warning in grid.Call(C_textBounds, as.graphicsAnnot(x$label), x$x, x$y, :
неизвестна ширина символа 0xf0 в кодировке CP1251
\end{verbatim}

\begin{verbatim}
Warning in grid.Call(C_textBounds, as.graphicsAnnot(x$label), x$x, x$y, :
неизвестна ширина символа 0xf2 в кодировке CP1251
\end{verbatim}

\begin{verbatim}
Warning in grid.Call(C_textBounds, as.graphicsAnnot(x$label), x$x, x$y, :
неизвестна ширина символа 0xed в кодировке CP1251
\end{verbatim}

\begin{verbatim}
Warning in grid.Call(C_textBounds, as.graphicsAnnot(x$label), x$x, x$y, :
неизвестна ширина символа 0xee в кодировке CP1251
\end{verbatim}

\begin{verbatim}
Warning in grid.Call(C_textBounds, as.graphicsAnnot(x$label), x$x, x$y, :
неизвестна ширина символа 0xf1 в кодировке CP1251
\end{verbatim}

\begin{verbatim}
Warning in grid.Call(C_textBounds, as.graphicsAnnot(x$label), x$x, x$y, :
неизвестна ширина символа 0xf2 в кодировке CP1251
\end{verbatim}

\begin{verbatim}
Warning in grid.Call(C_textBounds, as.graphicsAnnot(x$label), x$x, x$y, :
неизвестна ширина символа 0xe8 в кодировке CP1251
\end{verbatim}

\begin{verbatim}
Warning in grid.Call(C_textBounds, as.graphicsAnnot(x$label), x$x, x$y, :
неизвестна ширина символа 0xcc в кодировке CP1251
\end{verbatim}

\begin{verbatim}
Warning in grid.Call(C_textBounds, as.graphicsAnnot(x$label), x$x, x$y, :
неизвестна ширина символа 0xe5 в кодировке CP1251
\end{verbatim}

\begin{verbatim}
Warning in grid.Call(C_textBounds, as.graphicsAnnot(x$label), x$x, x$y, :
неизвестна ширина символа 0xe4 в кодировке CP1251
\end{verbatim}

\begin{verbatim}
Warning in grid.Call(C_textBounds, as.graphicsAnnot(x$label), x$x, x$y, :
неизвестна ширина символа 0xe8 в кодировке CP1251
\end{verbatim}

\begin{verbatim}
Warning in grid.Call(C_textBounds, as.graphicsAnnot(x$label), x$x, x$y, :
неизвестна ширина символа 0xe0 в кодировке CP1251
\end{verbatim}

\begin{verbatim}
Warning in grid.Call(C_textBounds, as.graphicsAnnot(x$label), x$x, x$y, :
неизвестна ширина символа 0xed в кодировке CP1251
\end{verbatim}

\begin{verbatim}
Warning in grid.Call(C_textBounds, as.graphicsAnnot(x$label), x$x, x$y, :
неизвестна ширина символа 0xe0 в кодировке CP1251
\end{verbatim}

\begin{verbatim}
Warning in grid.Call(C_textBounds, as.graphicsAnnot(x$label), x$x, x$y, :
неизвестна ширина символа 0xe8 в кодировке CP1251
\end{verbatim}

\begin{verbatim}
Warning in grid.Call(C_textBounds, as.graphicsAnnot(x$label), x$x, x$y, :
неизвестна ширина символа 0xec в кодировке CP1251
\end{verbatim}

\begin{verbatim}
Warning in grid.Call(C_textBounds, as.graphicsAnnot(x$label), x$x, x$y, :
неизвестна ширина символа 0xe5 в кодировке CP1251
\end{verbatim}

\begin{verbatim}
Warning in grid.Call(C_textBounds, as.graphicsAnnot(x$label), x$x, x$y, :
неизвестна ширина символа 0xe6 в кодировке CP1251
\end{verbatim}

\begin{verbatim}
Warning in grid.Call(C_textBounds, as.graphicsAnnot(x$label), x$x, x$y, :
неизвестна ширина символа 0xea в кодировке CP1251
\end{verbatim}

\begin{verbatim}
Warning in grid.Call(C_textBounds, as.graphicsAnnot(x$label), x$x, x$y, :
неизвестна ширина символа 0xe2 в кодировке CP1251
\end{verbatim}

\begin{verbatim}
Warning in grid.Call(C_textBounds, as.graphicsAnnot(x$label), x$x, x$y, :
неизвестна ширина символа 0xe0 в кодировке CP1251
\end{verbatim}

\begin{verbatim}
Warning in grid.Call(C_textBounds, as.graphicsAnnot(x$label), x$x, x$y, :
неизвестна ширина символа 0xf0 в кодировке CP1251
\end{verbatim}

\begin{verbatim}
Warning in grid.Call(C_textBounds, as.graphicsAnnot(x$label), x$x, x$y, :
неизвестна ширина символа 0xf2 в кодировке CP1251
\end{verbatim}

\begin{verbatim}
Warning in grid.Call(C_textBounds, as.graphicsAnnot(x$label), x$x, x$y, :
неизвестна ширина символа 0xe8 в кодировке CP1251
\end{verbatim}

\begin{verbatim}
Warning in grid.Call(C_textBounds, as.graphicsAnnot(x$label), x$x, x$y, :
неизвестна ширина символа 0xeb в кодировке CP1251
\end{verbatim}

\begin{verbatim}
Warning in grid.Call(C_textBounds, as.graphicsAnnot(x$label), x$x, x$y, :
неизвестна ширина символа 0xfc в кодировке CP1251
\end{verbatim}

\begin{verbatim}
Warning in grid.Call(C_textBounds, as.graphicsAnnot(x$label), x$x, x$y, :
неизвестна ширина символа 0xed в кодировке CP1251
\end{verbatim}

\begin{verbatim}
Warning in grid.Call(C_textBounds, as.graphicsAnnot(x$label), x$x, x$y, :
неизвестна ширина символа 0xfb в кодировке CP1251
\end{verbatim}

\begin{verbatim}
Warning in grid.Call(C_textBounds, as.graphicsAnnot(x$label), x$x, x$y, :
неизвестна ширина символа 0xe9 в кодировке CP1251
\end{verbatim}

\begin{verbatim}
Warning in grid.Call(C_textBounds, as.graphicsAnnot(x$label), x$x, x$y, :
неизвестна ширина символа 0xf0 в кодировке CP1251
\end{verbatim}

\begin{verbatim}
Warning in grid.Call(C_textBounds, as.graphicsAnnot(x$label), x$x, x$y, :
неизвестна ширина символа 0xe0 в кодировке CP1251
\end{verbatim}

\begin{verbatim}
Warning in grid.Call(C_textBounds, as.graphicsAnnot(x$label), x$x, x$y, :
неизвестна ширина символа 0xe7 в кодировке CP1251
\end{verbatim}

\begin{verbatim}
Warning in grid.Call(C_textBounds, as.graphicsAnnot(x$label), x$x, x$y, :
неизвестна ширина символа 0xec в кодировке CP1251
\end{verbatim}

\begin{verbatim}
Warning in grid.Call(C_textBounds, as.graphicsAnnot(x$label), x$x, x$y, :
неизвестна ширина символа 0xe0 в кодировке CP1251
\end{verbatim}

\begin{verbatim}
Warning in grid.Call(C_textBounds, as.graphicsAnnot(x$label), x$x, x$y, :
неизвестна ширина символа 0xf5 в кодировке CP1251
\end{verbatim}

\begin{verbatim}
Warning in grid.Call(C_textBounds, as.graphicsAnnot(x$label), x$x, x$y, :
неизвестна ширина символа 0xc3 в кодировке CP1251
\end{verbatim}

\begin{verbatim}
Warning in grid.Call(C_textBounds, as.graphicsAnnot(x$label), x$x, x$y, :
неизвестна ширина символа 0xee в кодировке CP1251
\end{verbatim}

\begin{verbatim}
Warning in grid.Call(C_textBounds, as.graphicsAnnot(x$label), x$x, x$y, :
неизвестна ширина символа 0xe4 в кодировке CP1251
\end{verbatim}

\begin{verbatim}
Warning in grid.Call(C_textBounds, as.graphicsAnnot(x$label), x$x, x$y, :
неизвестна ширина символа 0xef в кодировке CP1251
\end{verbatim}

\begin{verbatim}
Warning in grid.Call(C_textBounds, as.graphicsAnnot(x$label), x$x, x$y, :
неизвестна ширина символа 0xf0 в кодировке CP1251
\end{verbatim}

\begin{verbatim}
Warning in grid.Call(C_textBounds, as.graphicsAnnot(x$label), x$x, x$y, :
неизвестна ширина символа 0xee в кодировке CP1251
\end{verbatim}

\begin{verbatim}
Warning in grid.Call(C_textBounds, as.graphicsAnnot(x$label), x$x, x$y, :
неизвестна ширина символа 0xe3 в кодировке CP1251
\end{verbatim}

\begin{verbatim}
Warning in grid.Call(C_textBounds, as.graphicsAnnot(x$label), x$x, x$y, :
неизвестна ширина символа 0xed в кодировке CP1251
\end{verbatim}

\begin{verbatim}
Warning in grid.Call(C_textBounds, as.graphicsAnnot(x$label), x$x, x$y, :
неизвестна ширина символа 0xee в кодировке CP1251
\end{verbatim}

\begin{verbatim}
Warning in grid.Call(C_textBounds, as.graphicsAnnot(x$label), x$x, x$y, :
неизвестна ширина символа 0xe7 в кодировке CP1251
\end{verbatim}

\begin{verbatim}
Warning in grid.Call(C_textBounds, as.graphicsAnnot(x$label), x$x, x$y, :
неизвестна ширина символа 0xe0 в кодировке CP1251
\end{verbatim}

\begin{verbatim}
Warning in grid.Call.graphics(C_text, as.graphicsAnnot(x$label), x$x, x$y, :
неизвестна ширина символа 0xc3 в кодировке CP1251
\end{verbatim}

\begin{verbatim}
Warning in grid.Call.graphics(C_text, as.graphicsAnnot(x$label), x$x, x$y, :
неизвестна ширина символа 0xee в кодировке CP1251
\end{verbatim}

\begin{verbatim}
Warning in grid.Call.graphics(C_text, as.graphicsAnnot(x$label), x$x, x$y, :
неизвестна ширина символа 0xe4 в кодировке CP1251
\end{verbatim}

\begin{verbatim}
Warning in grid.Call.graphics(C_text, as.graphicsAnnot(x$label), x$x, x$y, :
неизвестна ширина символа 0xef в кодировке CP1251
\end{verbatim}

\begin{verbatim}
Warning in grid.Call.graphics(C_text, as.graphicsAnnot(x$label), x$x, x$y, :
неизвестна ширина символа 0xf0 в кодировке CP1251
\end{verbatim}

\begin{verbatim}
Warning in grid.Call.graphics(C_text, as.graphicsAnnot(x$label), x$x, x$y, :
неизвестна ширина символа 0xee в кодировке CP1251
\end{verbatim}

\begin{verbatim}
Warning in grid.Call.graphics(C_text, as.graphicsAnnot(x$label), x$x, x$y, :
неизвестна ширина символа 0xe3 в кодировке CP1251
\end{verbatim}

\begin{verbatim}
Warning in grid.Call.graphics(C_text, as.graphicsAnnot(x$label), x$x, x$y, :
неизвестна ширина символа 0xed в кодировке CP1251
\end{verbatim}

\begin{verbatim}
Warning in grid.Call.graphics(C_text, as.graphicsAnnot(x$label), x$x, x$y, :
неизвестна ширина символа 0xee в кодировке CP1251
\end{verbatim}

\begin{verbatim}
Warning in grid.Call.graphics(C_text, as.graphicsAnnot(x$label), x$x, x$y, :
неизвестна ширина символа 0xe7 в кодировке CP1251
\end{verbatim}

\begin{verbatim}
Warning in grid.Call.graphics(C_text, as.graphicsAnnot(x$label), x$x, x$y, :
неизвестна ширина символа 0xe0 в кодировке CP1251
\end{verbatim}

\begin{verbatim}
Warning in grid.Call.graphics(C_text, as.graphicsAnnot(x$label), x$x, x$y, :
неизвестна ширина символа 0xd1 в кодировке CP1251
\end{verbatim}

\begin{verbatim}
Warning in grid.Call.graphics(C_text, as.graphicsAnnot(x$label), x$x, x$y, :
неизвестна ширина символа 0xf2 в кодировке CP1251
\end{verbatim}

\begin{verbatim}
Warning in grid.Call.graphics(C_text, as.graphicsAnnot(x$label), x$x, x$y, :
неизвестна ширина символа 0xf0 в кодировке CP1251
\end{verbatim}

\begin{verbatim}
Warning in grid.Call.graphics(C_text, as.graphicsAnnot(x$label), x$x, x$y, :
неизвестна ширина символа 0xe0 в кодировке CP1251
\end{verbatim}

\begin{verbatim}
Warning in grid.Call.graphics(C_text, as.graphicsAnnot(x$label), x$x, x$y, :
неизвестна ширина символа 0xf2 в кодировке CP1251
\end{verbatim}

\begin{verbatim}
Warning in grid.Call.graphics(C_text, as.graphicsAnnot(x$label), x$x, x$y, :
неизвестна ширина символа 0xe5 в кодировке CP1251
\end{verbatim}

\begin{verbatim}
Warning in grid.Call.graphics(C_text, as.graphicsAnnot(x$label), x$x, x$y, :
неизвестна ширина символа 0xe3 в кодировке CP1251
\end{verbatim}

\begin{verbatim}
Warning in grid.Call.graphics(C_text, as.graphicsAnnot(x$label), x$x, x$y, :
неизвестна ширина символа 0xe8 в кодировке CP1251
\end{verbatim}

\begin{verbatim}
Warning in grid.Call.graphics(C_text, as.graphicsAnnot(x$label), x$x, x$y, :
неизвестна ширина символа 0xff в кодировке CP1251
\end{verbatim}

\begin{verbatim}
Warning in grid.Call.graphics(C_text, as.graphicsAnnot(x$label), x$x, x$y, :
неизвестна ширина символа 0xcc в кодировке CP1251
\end{verbatim}

\begin{verbatim}
Warning in grid.Call.graphics(C_text, as.graphicsAnnot(x$label), x$x, x$y, :
неизвестна ширина символа 0xe5 в кодировке CP1251
\end{verbatim}

\begin{verbatim}
Warning in grid.Call.graphics(C_text, as.graphicsAnnot(x$label), x$x, x$y, :
неизвестна ширина символа 0xe4 в кодировке CP1251
\end{verbatim}

\begin{verbatim}
Warning in grid.Call.graphics(C_text, as.graphicsAnnot(x$label), x$x, x$y, :
неизвестна ширина символа 0xe8 в кодировке CP1251
\end{verbatim}

\begin{verbatim}
Warning in grid.Call.graphics(C_text, as.graphicsAnnot(x$label), x$x, x$y, :
неизвестна ширина символа 0xe0 в кодировке CP1251
\end{verbatim}

\begin{verbatim}
Warning in grid.Call.graphics(C_text, as.graphicsAnnot(x$label), x$x, x$y, :
неизвестна ширина символа 0xed в кодировке CP1251
\end{verbatim}

\begin{verbatim}
Warning in grid.Call.graphics(C_text, as.graphicsAnnot(x$label), x$x, x$y, :
неизвестна ширина символа 0xe0 в кодировке CP1251
\end{verbatim}

\begin{verbatim}
Warning in grid.Call.graphics(C_text, as.graphicsAnnot(x$label), x$x, x$y, :
неизвестна ширина символа 0xe8 в кодировке CP1251
\end{verbatim}

\begin{verbatim}
Warning in grid.Call.graphics(C_text, as.graphicsAnnot(x$label), x$x, x$y, :
неизвестна ширина символа 0xec в кодировке CP1251
\end{verbatim}

\begin{verbatim}
Warning in grid.Call.graphics(C_text, as.graphicsAnnot(x$label), x$x, x$y, :
неизвестна ширина символа 0xe5 в кодировке CP1251
\end{verbatim}

\begin{verbatim}
Warning in grid.Call.graphics(C_text, as.graphicsAnnot(x$label), x$x, x$y, :
неизвестна ширина символа 0xe6 в кодировке CP1251
\end{verbatim}

\begin{verbatim}
Warning in grid.Call.graphics(C_text, as.graphicsAnnot(x$label), x$x, x$y, :
неизвестна ширина символа 0xea в кодировке CP1251
\end{verbatim}

\begin{verbatim}
Warning in grid.Call.graphics(C_text, as.graphicsAnnot(x$label), x$x, x$y, :
неизвестна ширина символа 0xe2 в кодировке CP1251
\end{verbatim}

\begin{verbatim}
Warning in grid.Call.graphics(C_text, as.graphicsAnnot(x$label), x$x, x$y, :
неизвестна ширина символа 0xe0 в кодировке CP1251
\end{verbatim}

\begin{verbatim}
Warning in grid.Call.graphics(C_text, as.graphicsAnnot(x$label), x$x, x$y, :
неизвестна ширина символа 0xf0 в кодировке CP1251
\end{verbatim}

\begin{verbatim}
Warning in grid.Call.graphics(C_text, as.graphicsAnnot(x$label), x$x, x$y, :
неизвестна ширина символа 0xf2 в кодировке CP1251
\end{verbatim}

\begin{verbatim}
Warning in grid.Call.graphics(C_text, as.graphicsAnnot(x$label), x$x, x$y, :
неизвестна ширина символа 0xe8 в кодировке CP1251
\end{verbatim}

\begin{verbatim}
Warning in grid.Call.graphics(C_text, as.graphicsAnnot(x$label), x$x, x$y, :
неизвестна ширина символа 0xeb в кодировке CP1251
\end{verbatim}

\begin{verbatim}
Warning in grid.Call.graphics(C_text, as.graphicsAnnot(x$label), x$x, x$y, :
неизвестна ширина символа 0xfc в кодировке CP1251
\end{verbatim}

\begin{verbatim}
Warning in grid.Call.graphics(C_text, as.graphicsAnnot(x$label), x$x, x$y, :
неизвестна ширина символа 0xed в кодировке CP1251
\end{verbatim}

\begin{verbatim}
Warning in grid.Call.graphics(C_text, as.graphicsAnnot(x$label), x$x, x$y, :
неизвестна ширина символа 0xfb в кодировке CP1251
\end{verbatim}

\begin{verbatim}
Warning in grid.Call.graphics(C_text, as.graphicsAnnot(x$label), x$x, x$y, :
неизвестна ширина символа 0xe9 в кодировке CP1251
\end{verbatim}

\begin{verbatim}
Warning in grid.Call.graphics(C_text, as.graphicsAnnot(x$label), x$x, x$y, :
неизвестна ширина символа 0xf0 в кодировке CP1251
\end{verbatim}

\begin{verbatim}
Warning in grid.Call.graphics(C_text, as.graphicsAnnot(x$label), x$x, x$y, :
неизвестна ширина символа 0xe0 в кодировке CP1251
\end{verbatim}

\begin{verbatim}
Warning in grid.Call.graphics(C_text, as.graphicsAnnot(x$label), x$x, x$y, :
неизвестна ширина символа 0xe7 в кодировке CP1251
\end{verbatim}

\begin{verbatim}
Warning in grid.Call.graphics(C_text, as.graphicsAnnot(x$label), x$x, x$y, :
неизвестна ширина символа 0xec в кодировке CP1251
\end{verbatim}

\begin{verbatim}
Warning in grid.Call.graphics(C_text, as.graphicsAnnot(x$label), x$x, x$y, :
неизвестна ширина символа 0xe0 в кодировке CP1251
\end{verbatim}

\begin{verbatim}
Warning in grid.Call.graphics(C_text, as.graphicsAnnot(x$label), x$x, x$y, :
неизвестна ширина символа 0xf5 в кодировке CP1251
\end{verbatim}

\begin{verbatim}
Warning in grid.Call.graphics(C_text, as.graphicsAnnot(x$label), x$x, x$y, :
неизвестна ширина символа 0xc4 в кодировке CP1251
\end{verbatim}

\begin{verbatim}
Warning in grid.Call.graphics(C_text, as.graphicsAnnot(x$label), x$x, x$y, :
неизвестна ширина символа 0xe8 в кодировке CP1251
\end{verbatim}

\begin{verbatim}
Warning in grid.Call.graphics(C_text, as.graphicsAnnot(x$label), x$x, x$y, :
неизвестна ширина символа 0xed в кодировке CP1251
\end{verbatim}

\begin{verbatim}
Warning in grid.Call.graphics(C_text, as.graphicsAnnot(x$label), x$x, x$y, :
неизвестна ширина символа 0xe0 в кодировке CP1251
\end{verbatim}

\begin{verbatim}
Warning in grid.Call.graphics(C_text, as.graphicsAnnot(x$label), x$x, x$y, :
неизвестна ширина символа 0xec в кодировке CP1251
\end{verbatim}

\begin{verbatim}
Warning in grid.Call.graphics(C_text, as.graphicsAnnot(x$label), x$x, x$y, :
неизвестна ширина символа 0xe8 в кодировке CP1251
\end{verbatim}

\begin{verbatim}
Warning in grid.Call.graphics(C_text, as.graphicsAnnot(x$label), x$x, x$y, :
неизвестна ширина символа 0xea в кодировке CP1251
\end{verbatim}

\begin{verbatim}
Warning in grid.Call.graphics(C_text, as.graphicsAnnot(x$label), x$x, x$y, :
неизвестна ширина символа 0xe0 в кодировке CP1251
\end{verbatim}

\begin{verbatim}
Warning in grid.Call.graphics(C_text, as.graphicsAnnot(x$label), x$x, x$y, :
неизвестна ширина символа 0xef в кодировке CP1251
\end{verbatim}

\begin{verbatim}
Warning in grid.Call.graphics(C_text, as.graphicsAnnot(x$label), x$x, x$y, :
неизвестна ширина символа 0xf0 в кодировке CP1251
\end{verbatim}

\begin{verbatim}
Warning in grid.Call.graphics(C_text, as.graphicsAnnot(x$label), x$x, x$y, :
неизвестна ширина символа 0xee в кодировке CP1251
\end{verbatim}

\begin{verbatim}
Warning in grid.Call.graphics(C_text, as.graphicsAnnot(x$label), x$x, x$y, :
неизвестна ширина символа 0xec в кодировке CP1251
\end{verbatim}

\begin{verbatim}
Warning in grid.Call.graphics(C_text, as.graphicsAnnot(x$label), x$x, x$y, :
неизвестна ширина символа 0xfb в кодировке CP1251
\end{verbatim}

\begin{verbatim}
Warning in grid.Call.graphics(C_text, as.graphicsAnnot(x$label), x$x, x$y, :
неизвестна ширина символа 0xf1 в кодировке CP1251
\end{verbatim}

\begin{verbatim}
Warning in grid.Call.graphics(C_text, as.graphicsAnnot(x$label), x$x, x$y, :
неизвестна ширина символа 0xeb в кодировке CP1251
\end{verbatim}

\begin{verbatim}
Warning in grid.Call.graphics(C_text, as.graphicsAnnot(x$label), x$x, x$y, :
неизвестна ширина символа 0xee в кодировке CP1251
\end{verbatim}

\begin{verbatim}
Warning in grid.Call.graphics(C_text, as.graphicsAnnot(x$label), x$x, x$y, :
неизвестна ширина символа 0xe2 в кодировке CP1251
\end{verbatim}

\begin{verbatim}
Warning in grid.Call.graphics(C_text, as.graphicsAnnot(x$label), x$x, x$y, :
неизвестна ширина символа 0xee в кодировке CP1251
\end{verbatim}

\begin{verbatim}
Warning in grid.Call.graphics(C_text, as.graphicsAnnot(x$label), x$x, x$y, :
неизвестна ширина символа 0xe9 в кодировке CP1251
\end{verbatim}

\begin{verbatim}
Warning in grid.Call.graphics(C_text, as.graphicsAnnot(x$label), x$x, x$y, :
неизвестна ширина символа 0xf1 в кодировке CP1251
\end{verbatim}

\begin{verbatim}
Warning in grid.Call.graphics(C_text, as.graphicsAnnot(x$label), x$x, x$y, :
неизвестна ширина символа 0xec в кодировке CP1251
\end{verbatim}

\begin{verbatim}
Warning in grid.Call.graphics(C_text, as.graphicsAnnot(x$label), x$x, x$y, :
неизвестна ширина символа 0xe5 в кодировке CP1251
\end{verbatim}

\begin{verbatim}
Warning in grid.Call.graphics(C_text, as.graphicsAnnot(x$label), x$x, x$y, :
неизвестна ширина символа 0xf0 в кодировке CP1251
\end{verbatim}

\begin{verbatim}
Warning in grid.Call.graphics(C_text, as.graphicsAnnot(x$label), x$x, x$y, :
неизвестна ширина символа 0xf2 в кодировке CP1251
\end{verbatim}

\begin{verbatim}
Warning in grid.Call.graphics(C_text, as.graphicsAnnot(x$label), x$x, x$y, :
неизвестна ширина символа 0xed в кодировке CP1251
\end{verbatim}

\begin{verbatim}
Warning in grid.Call.graphics(C_text, as.graphicsAnnot(x$label), x$x, x$y, :
неизвестна ширина символа 0xee в кодировке CP1251
\end{verbatim}

\begin{verbatim}
Warning in grid.Call.graphics(C_text, as.graphicsAnnot(x$label), x$x, x$y, :
неизвестна ширина символа 0xf1 в кодировке CP1251
\end{verbatim}

\begin{verbatim}
Warning in grid.Call.graphics(C_text, as.graphicsAnnot(x$label), x$x, x$y, :
неизвестна ширина символа 0xf2 в кодировке CP1251
\end{verbatim}

\begin{verbatim}
Warning in grid.Call.graphics(C_text, as.graphicsAnnot(x$label), x$x, x$y, :
неизвестна ширина символа 0xe8 в кодировке CP1251
\end{verbatim}

\pandocbounded{\includegraphics[keepaspectratio]{chapter15_files/figure-pdf/unnamed-chunk-1-2.pdf}}

\begin{Shaded}
\begin{Highlighting}[]
\DocumentationTok{\#\# 9.5 График 3: Динамика вылова}
\NormalTok{p3\_catch }\OtherTok{\textless{}{-}} \FunctionTok{ggplot}\NormalTok{(summary\_data, }\FunctionTok{aes}\NormalTok{(}\AttributeTok{x =}\NormalTok{ year, }\AttributeTok{color =}\NormalTok{ strategy, }\AttributeTok{fill =}\NormalTok{ strategy)) }\SpecialCharTok{+}
  \FunctionTok{geom\_ribbon}\NormalTok{(}\FunctionTok{aes}\NormalTok{(}\AttributeTok{ymin =}\NormalTok{ catch\_q25, }\AttributeTok{ymax =}\NormalTok{ catch\_q75), }\AttributeTok{alpha =} \FloatTok{0.3}\NormalTok{, }\AttributeTok{color =} \ConstantTok{NA}\NormalTok{) }\SpecialCharTok{+}
  \FunctionTok{geom\_line}\NormalTok{(}\FunctionTok{aes}\NormalTok{(}\AttributeTok{y =}\NormalTok{ catch\_median), }\AttributeTok{size =} \FloatTok{1.5}\NormalTok{) }\SpecialCharTok{+}
  \FunctionTok{scale\_color\_manual}\NormalTok{(}\AttributeTok{values =} \FunctionTok{c}\NormalTok{(}\StringTok{"Fish at Fmsy"} \OtherTok{=} \StringTok{"\#E41A1C"}\NormalTok{, }
                               \StringTok{"MSY Hockey{-}stick"} \OtherTok{=} \StringTok{"\#377EB8"}\NormalTok{,}
                               \StringTok{"ICES Advice Rule"} \OtherTok{=} \StringTok{"\#4DAF4A"}\NormalTok{)) }\SpecialCharTok{+}
  \FunctionTok{scale\_fill\_manual}\NormalTok{(}\AttributeTok{values =} \FunctionTok{c}\NormalTok{(}\StringTok{"Fish at Fmsy"} \OtherTok{=} \StringTok{"\#E41A1C"}\NormalTok{, }
                              \StringTok{"MSY Hockey{-}stick"} \OtherTok{=} \StringTok{"\#377EB8"}\NormalTok{,}
                              \StringTok{"ICES Advice Rule"} \OtherTok{=} \StringTok{"\#4DAF4A"}\NormalTok{)) }\SpecialCharTok{+}
  \FunctionTok{labs}\NormalTok{(}\AttributeTok{title =} \StringTok{"Динамика вылова"}\NormalTok{,}
       \AttributeTok{subtitle =} \StringTok{"Медиана и межквартильный размах"}\NormalTok{,}
       \AttributeTok{x =} \StringTok{"Год прогноза"}\NormalTok{, }
       \AttributeTok{y =} \StringTok{"Вылов (тыс. т)"}\NormalTok{,}
       \AttributeTok{color =} \StringTok{"Стратегия"}\NormalTok{,}
       \AttributeTok{fill =} \StringTok{"Стратегия"}\NormalTok{) }\SpecialCharTok{+}
\NormalTok{  theme\_mse}

\NormalTok{p3\_catch }
\end{Highlighting}
\end{Shaded}

\begin{verbatim}
Warning in grid.Call(C_textBounds, as.graphicsAnnot(x$label), x$x, x$y, :
неизвестна ширина символа 0xd1 в кодировке CP1251
\end{verbatim}

\begin{verbatim}
Warning in grid.Call(C_textBounds, as.graphicsAnnot(x$label), x$x, x$y, :
неизвестна ширина символа 0xf2 в кодировке CP1251
\end{verbatim}

\begin{verbatim}
Warning in grid.Call(C_textBounds, as.graphicsAnnot(x$label), x$x, x$y, :
неизвестна ширина символа 0xf0 в кодировке CP1251
\end{verbatim}

\begin{verbatim}
Warning in grid.Call(C_textBounds, as.graphicsAnnot(x$label), x$x, x$y, :
неизвестна ширина символа 0xe0 в кодировке CP1251
\end{verbatim}

\begin{verbatim}
Warning in grid.Call(C_textBounds, as.graphicsAnnot(x$label), x$x, x$y, :
неизвестна ширина символа 0xf2 в кодировке CP1251
\end{verbatim}

\begin{verbatim}
Warning in grid.Call(C_textBounds, as.graphicsAnnot(x$label), x$x, x$y, :
неизвестна ширина символа 0xe5 в кодировке CP1251
\end{verbatim}

\begin{verbatim}
Warning in grid.Call(C_textBounds, as.graphicsAnnot(x$label), x$x, x$y, :
неизвестна ширина символа 0xe3 в кодировке CP1251
\end{verbatim}

\begin{verbatim}
Warning in grid.Call(C_textBounds, as.graphicsAnnot(x$label), x$x, x$y, :
неизвестна ширина символа 0xe8 в кодировке CP1251
\end{verbatim}

\begin{verbatim}
Warning in grid.Call(C_textBounds, as.graphicsAnnot(x$label), x$x, x$y, :
неизвестна ширина символа 0xff в кодировке CP1251
\end{verbatim}

\begin{verbatim}
Warning in grid.Call(C_textBounds, as.graphicsAnnot(x$label), x$x, x$y, :
неизвестна ширина символа 0xd1 в кодировке CP1251
\end{verbatim}

\begin{verbatim}
Warning in grid.Call(C_textBounds, as.graphicsAnnot(x$label), x$x, x$y, :
неизвестна ширина символа 0xf2 в кодировке CP1251
\end{verbatim}

\begin{verbatim}
Warning in grid.Call(C_textBounds, as.graphicsAnnot(x$label), x$x, x$y, :
неизвестна ширина символа 0xf0 в кодировке CP1251
\end{verbatim}

\begin{verbatim}
Warning in grid.Call(C_textBounds, as.graphicsAnnot(x$label), x$x, x$y, :
неизвестна ширина символа 0xe0 в кодировке CP1251
\end{verbatim}

\begin{verbatim}
Warning in grid.Call(C_textBounds, as.graphicsAnnot(x$label), x$x, x$y, :
неизвестна ширина символа 0xf2 в кодировке CP1251
\end{verbatim}

\begin{verbatim}
Warning in grid.Call(C_textBounds, as.graphicsAnnot(x$label), x$x, x$y, :
неизвестна ширина символа 0xe5 в кодировке CP1251
\end{verbatim}

\begin{verbatim}
Warning in grid.Call(C_textBounds, as.graphicsAnnot(x$label), x$x, x$y, :
неизвестна ширина символа 0xe3 в кодировке CP1251
\end{verbatim}

\begin{verbatim}
Warning in grid.Call(C_textBounds, as.graphicsAnnot(x$label), x$x, x$y, :
неизвестна ширина символа 0xe8 в кодировке CP1251
\end{verbatim}

\begin{verbatim}
Warning in grid.Call(C_textBounds, as.graphicsAnnot(x$label), x$x, x$y, :
неизвестна ширина символа 0xff в кодировке CP1251
\end{verbatim}

\begin{verbatim}
Warning in grid.Call(C_textBounds, as.graphicsAnnot(x$label), x$x, x$y, :
неизвестна ширина символа 0xc2 в кодировке CP1251
\end{verbatim}

\begin{verbatim}
Warning in grid.Call(C_textBounds, as.graphicsAnnot(x$label), x$x, x$y, :
неизвестна ширина символа 0xfb в кодировке CP1251
\end{verbatim}

\begin{verbatim}
Warning in grid.Call(C_textBounds, as.graphicsAnnot(x$label), x$x, x$y, :
неизвестна ширина символа 0xeb в кодировке CP1251
\end{verbatim}

\begin{verbatim}
Warning in grid.Call(C_textBounds, as.graphicsAnnot(x$label), x$x, x$y, :
неизвестна ширина символа 0xee в кодировке CP1251
\end{verbatim}

\begin{verbatim}
Warning in grid.Call(C_textBounds, as.graphicsAnnot(x$label), x$x, x$y, :
неизвестна ширина символа 0xe2 в кодировке CP1251
\end{verbatim}

\begin{verbatim}
Warning in grid.Call(C_textBounds, as.graphicsAnnot(x$label), x$x, x$y, :
неизвестна ширина символа 0xf2 в кодировке CP1251
\end{verbatim}

\begin{verbatim}
Warning in grid.Call(C_textBounds, as.graphicsAnnot(x$label), x$x, x$y, :
неизвестна ширина символа 0xfb в кодировке CP1251
\end{verbatim}

\begin{verbatim}
Warning in grid.Call(C_textBounds, as.graphicsAnnot(x$label), x$x, x$y, :
неизвестна ширина символа 0xf1 в кодировке CP1251
\end{verbatim}

\begin{verbatim}
Warning in grid.Call(C_textBounds, as.graphicsAnnot(x$label), x$x, x$y, :
неизвестна ширина символа 0xf2 в кодировке CP1251
\end{verbatim}

\begin{verbatim}
Warning in grid.Call(C_textBounds, as.graphicsAnnot(x$label), x$x, x$y, :
неизвестна ширина символа 0xc4 в кодировке CP1251
\end{verbatim}

\begin{verbatim}
Warning in grid.Call(C_textBounds, as.graphicsAnnot(x$label), x$x, x$y, :
неизвестна ширина символа 0xe8 в кодировке CP1251
\end{verbatim}

\begin{verbatim}
Warning in grid.Call(C_textBounds, as.graphicsAnnot(x$label), x$x, x$y, :
неизвестна ширина символа 0xed в кодировке CP1251
\end{verbatim}

\begin{verbatim}
Warning in grid.Call(C_textBounds, as.graphicsAnnot(x$label), x$x, x$y, :
неизвестна ширина символа 0xe0 в кодировке CP1251
\end{verbatim}

\begin{verbatim}
Warning in grid.Call(C_textBounds, as.graphicsAnnot(x$label), x$x, x$y, :
неизвестна ширина символа 0xec в кодировке CP1251
\end{verbatim}

\begin{verbatim}
Warning in grid.Call(C_textBounds, as.graphicsAnnot(x$label), x$x, x$y, :
неизвестна ширина символа 0xe8 в кодировке CP1251
\end{verbatim}

\begin{verbatim}
Warning in grid.Call(C_textBounds, as.graphicsAnnot(x$label), x$x, x$y, :
неизвестна ширина символа 0xea в кодировке CP1251
\end{verbatim}

\begin{verbatim}
Warning in grid.Call(C_textBounds, as.graphicsAnnot(x$label), x$x, x$y, :
неизвестна ширина символа 0xe0 в кодировке CP1251
\end{verbatim}

\begin{verbatim}
Warning in grid.Call(C_textBounds, as.graphicsAnnot(x$label), x$x, x$y, :
неизвестна ширина символа 0xe2 в кодировке CP1251
\end{verbatim}

\begin{verbatim}
Warning in grid.Call(C_textBounds, as.graphicsAnnot(x$label), x$x, x$y, :
неизвестна ширина символа 0xfb в кодировке CP1251
\end{verbatim}

\begin{verbatim}
Warning in grid.Call(C_textBounds, as.graphicsAnnot(x$label), x$x, x$y, :
неизвестна ширина символа 0xeb в кодировке CP1251
\end{verbatim}

\begin{verbatim}
Warning in grid.Call(C_textBounds, as.graphicsAnnot(x$label), x$x, x$y, :
неизвестна ширина символа 0xee в кодировке CP1251
\end{verbatim}

\begin{verbatim}
Warning in grid.Call(C_textBounds, as.graphicsAnnot(x$label), x$x, x$y, :
неизвестна ширина символа 0xe2 в кодировке CP1251
\end{verbatim}

\begin{verbatim}
Warning in grid.Call(C_textBounds, as.graphicsAnnot(x$label), x$x, x$y, :
неизвестна ширина символа 0xe0 в кодировке CP1251
\end{verbatim}

\begin{verbatim}
Warning in grid.Call(C_textBounds, as.graphicsAnnot(x$label), x$x, x$y, :
неизвестна ширина символа 0xcc в кодировке CP1251
\end{verbatim}

\begin{verbatim}
Warning in grid.Call(C_textBounds, as.graphicsAnnot(x$label), x$x, x$y, :
неизвестна ширина символа 0xe5 в кодировке CP1251
\end{verbatim}

\begin{verbatim}
Warning in grid.Call(C_textBounds, as.graphicsAnnot(x$label), x$x, x$y, :
неизвестна ширина символа 0xe4 в кодировке CP1251
\end{verbatim}

\begin{verbatim}
Warning in grid.Call(C_textBounds, as.graphicsAnnot(x$label), x$x, x$y, :
неизвестна ширина символа 0xe8 в кодировке CP1251
\end{verbatim}

\begin{verbatim}
Warning in grid.Call(C_textBounds, as.graphicsAnnot(x$label), x$x, x$y, :
неизвестна ширина символа 0xe0 в кодировке CP1251
\end{verbatim}

\begin{verbatim}
Warning in grid.Call(C_textBounds, as.graphicsAnnot(x$label), x$x, x$y, :
неизвестна ширина символа 0xed в кодировке CP1251
\end{verbatim}

\begin{verbatim}
Warning in grid.Call(C_textBounds, as.graphicsAnnot(x$label), x$x, x$y, :
неизвестна ширина символа 0xe0 в кодировке CP1251
\end{verbatim}

\begin{verbatim}
Warning in grid.Call(C_textBounds, as.graphicsAnnot(x$label), x$x, x$y, :
неизвестна ширина символа 0xe8 в кодировке CP1251
\end{verbatim}

\begin{verbatim}
Warning in grid.Call(C_textBounds, as.graphicsAnnot(x$label), x$x, x$y, :
неизвестна ширина символа 0xec в кодировке CP1251
\end{verbatim}

\begin{verbatim}
Warning in grid.Call(C_textBounds, as.graphicsAnnot(x$label), x$x, x$y, :
неизвестна ширина символа 0xe5 в кодировке CP1251
\end{verbatim}

\begin{verbatim}
Warning in grid.Call(C_textBounds, as.graphicsAnnot(x$label), x$x, x$y, :
неизвестна ширина символа 0xe6 в кодировке CP1251
\end{verbatim}

\begin{verbatim}
Warning in grid.Call(C_textBounds, as.graphicsAnnot(x$label), x$x, x$y, :
неизвестна ширина символа 0xea в кодировке CP1251
\end{verbatim}

\begin{verbatim}
Warning in grid.Call(C_textBounds, as.graphicsAnnot(x$label), x$x, x$y, :
неизвестна ширина символа 0xe2 в кодировке CP1251
\end{verbatim}

\begin{verbatim}
Warning in grid.Call(C_textBounds, as.graphicsAnnot(x$label), x$x, x$y, :
неизвестна ширина символа 0xe0 в кодировке CP1251
\end{verbatim}

\begin{verbatim}
Warning in grid.Call(C_textBounds, as.graphicsAnnot(x$label), x$x, x$y, :
неизвестна ширина символа 0xf0 в кодировке CP1251
\end{verbatim}

\begin{verbatim}
Warning in grid.Call(C_textBounds, as.graphicsAnnot(x$label), x$x, x$y, :
неизвестна ширина символа 0xf2 в кодировке CP1251
\end{verbatim}

\begin{verbatim}
Warning in grid.Call(C_textBounds, as.graphicsAnnot(x$label), x$x, x$y, :
неизвестна ширина символа 0xe8 в кодировке CP1251
\end{verbatim}

\begin{verbatim}
Warning in grid.Call(C_textBounds, as.graphicsAnnot(x$label), x$x, x$y, :
неизвестна ширина символа 0xeb в кодировке CP1251
\end{verbatim}

\begin{verbatim}
Warning in grid.Call(C_textBounds, as.graphicsAnnot(x$label), x$x, x$y, :
неизвестна ширина символа 0xfc в кодировке CP1251
\end{verbatim}

\begin{verbatim}
Warning in grid.Call(C_textBounds, as.graphicsAnnot(x$label), x$x, x$y, :
неизвестна ширина символа 0xed в кодировке CP1251
\end{verbatim}

\begin{verbatim}
Warning in grid.Call(C_textBounds, as.graphicsAnnot(x$label), x$x, x$y, :
неизвестна ширина символа 0xfb в кодировке CP1251
\end{verbatim}

\begin{verbatim}
Warning in grid.Call(C_textBounds, as.graphicsAnnot(x$label), x$x, x$y, :
неизвестна ширина символа 0xe9 в кодировке CP1251
\end{verbatim}

\begin{verbatim}
Warning in grid.Call(C_textBounds, as.graphicsAnnot(x$label), x$x, x$y, :
неизвестна ширина символа 0xf0 в кодировке CP1251
\end{verbatim}

\begin{verbatim}
Warning in grid.Call(C_textBounds, as.graphicsAnnot(x$label), x$x, x$y, :
неизвестна ширина символа 0xe0 в кодировке CP1251
\end{verbatim}

\begin{verbatim}
Warning in grid.Call(C_textBounds, as.graphicsAnnot(x$label), x$x, x$y, :
неизвестна ширина символа 0xe7 в кодировке CP1251
\end{verbatim}

\begin{verbatim}
Warning in grid.Call(C_textBounds, as.graphicsAnnot(x$label), x$x, x$y, :
неизвестна ширина символа 0xec в кодировке CP1251
\end{verbatim}

\begin{verbatim}
Warning in grid.Call(C_textBounds, as.graphicsAnnot(x$label), x$x, x$y, :
неизвестна ширина символа 0xe0 в кодировке CP1251
\end{verbatim}

\begin{verbatim}
Warning in grid.Call(C_textBounds, as.graphicsAnnot(x$label), x$x, x$y, :
неизвестна ширина символа 0xf5 в кодировке CP1251
\end{verbatim}

\begin{verbatim}
Warning in grid.Call(C_textBounds, as.graphicsAnnot(x$label), x$x, x$y, :
неизвестна ширина символа 0xc3 в кодировке CP1251
\end{verbatim}

\begin{verbatim}
Warning in grid.Call(C_textBounds, as.graphicsAnnot(x$label), x$x, x$y, :
неизвестна ширина символа 0xee в кодировке CP1251
\end{verbatim}

\begin{verbatim}
Warning in grid.Call(C_textBounds, as.graphicsAnnot(x$label), x$x, x$y, :
неизвестна ширина символа 0xe4 в кодировке CP1251
\end{verbatim}

\begin{verbatim}
Warning in grid.Call(C_textBounds, as.graphicsAnnot(x$label), x$x, x$y, :
неизвестна ширина символа 0xef в кодировке CP1251
\end{verbatim}

\begin{verbatim}
Warning in grid.Call(C_textBounds, as.graphicsAnnot(x$label), x$x, x$y, :
неизвестна ширина символа 0xf0 в кодировке CP1251
\end{verbatim}

\begin{verbatim}
Warning in grid.Call(C_textBounds, as.graphicsAnnot(x$label), x$x, x$y, :
неизвестна ширина символа 0xee в кодировке CP1251
\end{verbatim}

\begin{verbatim}
Warning in grid.Call(C_textBounds, as.graphicsAnnot(x$label), x$x, x$y, :
неизвестна ширина символа 0xe3 в кодировке CP1251
\end{verbatim}

\begin{verbatim}
Warning in grid.Call(C_textBounds, as.graphicsAnnot(x$label), x$x, x$y, :
неизвестна ширина символа 0xed в кодировке CP1251
\end{verbatim}

\begin{verbatim}
Warning in grid.Call(C_textBounds, as.graphicsAnnot(x$label), x$x, x$y, :
неизвестна ширина символа 0xee в кодировке CP1251
\end{verbatim}

\begin{verbatim}
Warning in grid.Call(C_textBounds, as.graphicsAnnot(x$label), x$x, x$y, :
неизвестна ширина символа 0xe7 в кодировке CP1251
\end{verbatim}

\begin{verbatim}
Warning in grid.Call(C_textBounds, as.graphicsAnnot(x$label), x$x, x$y, :
неизвестна ширина символа 0xe0 в кодировке CP1251
\end{verbatim}

\begin{verbatim}
Warning in grid.Call.graphics(C_text, as.graphicsAnnot(x$label), x$x, x$y, :
неизвестна ширина символа 0xc3 в кодировке CP1251
\end{verbatim}

\begin{verbatim}
Warning in grid.Call.graphics(C_text, as.graphicsAnnot(x$label), x$x, x$y, :
неизвестна ширина символа 0xee в кодировке CP1251
\end{verbatim}

\begin{verbatim}
Warning in grid.Call.graphics(C_text, as.graphicsAnnot(x$label), x$x, x$y, :
неизвестна ширина символа 0xe4 в кодировке CP1251
\end{verbatim}

\begin{verbatim}
Warning in grid.Call.graphics(C_text, as.graphicsAnnot(x$label), x$x, x$y, :
неизвестна ширина символа 0xef в кодировке CP1251
\end{verbatim}

\begin{verbatim}
Warning in grid.Call.graphics(C_text, as.graphicsAnnot(x$label), x$x, x$y, :
неизвестна ширина символа 0xf0 в кодировке CP1251
\end{verbatim}

\begin{verbatim}
Warning in grid.Call.graphics(C_text, as.graphicsAnnot(x$label), x$x, x$y, :
неизвестна ширина символа 0xee в кодировке CP1251
\end{verbatim}

\begin{verbatim}
Warning in grid.Call.graphics(C_text, as.graphicsAnnot(x$label), x$x, x$y, :
неизвестна ширина символа 0xe3 в кодировке CP1251
\end{verbatim}

\begin{verbatim}
Warning in grid.Call.graphics(C_text, as.graphicsAnnot(x$label), x$x, x$y, :
неизвестна ширина символа 0xed в кодировке CP1251
\end{verbatim}

\begin{verbatim}
Warning in grid.Call.graphics(C_text, as.graphicsAnnot(x$label), x$x, x$y, :
неизвестна ширина символа 0xee в кодировке CP1251
\end{verbatim}

\begin{verbatim}
Warning in grid.Call.graphics(C_text, as.graphicsAnnot(x$label), x$x, x$y, :
неизвестна ширина символа 0xe7 в кодировке CP1251
\end{verbatim}

\begin{verbatim}
Warning in grid.Call.graphics(C_text, as.graphicsAnnot(x$label), x$x, x$y, :
неизвестна ширина символа 0xe0 в кодировке CP1251
\end{verbatim}

\begin{verbatim}
Warning in grid.Call.graphics(C_text, as.graphicsAnnot(x$label), x$x, x$y, :
неизвестна ширина символа 0xc2 в кодировке CP1251
\end{verbatim}

\begin{verbatim}
Warning in grid.Call.graphics(C_text, as.graphicsAnnot(x$label), x$x, x$y, :
неизвестна ширина символа 0xfb в кодировке CP1251
\end{verbatim}

\begin{verbatim}
Warning in grid.Call.graphics(C_text, as.graphicsAnnot(x$label), x$x, x$y, :
неизвестна ширина символа 0xeb в кодировке CP1251
\end{verbatim}

\begin{verbatim}
Warning in grid.Call.graphics(C_text, as.graphicsAnnot(x$label), x$x, x$y, :
неизвестна ширина символа 0xee в кодировке CP1251
\end{verbatim}

\begin{verbatim}
Warning in grid.Call.graphics(C_text, as.graphicsAnnot(x$label), x$x, x$y, :
неизвестна ширина символа 0xe2 в кодировке CP1251
\end{verbatim}

\begin{verbatim}
Warning in grid.Call.graphics(C_text, as.graphicsAnnot(x$label), x$x, x$y, :
неизвестна ширина символа 0xf2 в кодировке CP1251
\end{verbatim}

\begin{verbatim}
Warning in grid.Call.graphics(C_text, as.graphicsAnnot(x$label), x$x, x$y, :
неизвестна ширина символа 0xfb в кодировке CP1251
\end{verbatim}

\begin{verbatim}
Warning in grid.Call.graphics(C_text, as.graphicsAnnot(x$label), x$x, x$y, :
неизвестна ширина символа 0xf1 в кодировке CP1251
\end{verbatim}

\begin{verbatim}
Warning in grid.Call.graphics(C_text, as.graphicsAnnot(x$label), x$x, x$y, :
неизвестна ширина символа 0xf2 в кодировке CP1251
\end{verbatim}

\begin{verbatim}
Warning in grid.Call.graphics(C_text, as.graphicsAnnot(x$label), x$x, x$y, :
неизвестна ширина символа 0xd1 в кодировке CP1251
\end{verbatim}

\begin{verbatim}
Warning in grid.Call.graphics(C_text, as.graphicsAnnot(x$label), x$x, x$y, :
неизвестна ширина символа 0xf2 в кодировке CP1251
\end{verbatim}

\begin{verbatim}
Warning in grid.Call.graphics(C_text, as.graphicsAnnot(x$label), x$x, x$y, :
неизвестна ширина символа 0xf0 в кодировке CP1251
\end{verbatim}

\begin{verbatim}
Warning in grid.Call.graphics(C_text, as.graphicsAnnot(x$label), x$x, x$y, :
неизвестна ширина символа 0xe0 в кодировке CP1251
\end{verbatim}

\begin{verbatim}
Warning in grid.Call.graphics(C_text, as.graphicsAnnot(x$label), x$x, x$y, :
неизвестна ширина символа 0xf2 в кодировке CP1251
\end{verbatim}

\begin{verbatim}
Warning in grid.Call.graphics(C_text, as.graphicsAnnot(x$label), x$x, x$y, :
неизвестна ширина символа 0xe5 в кодировке CP1251
\end{verbatim}

\begin{verbatim}
Warning in grid.Call.graphics(C_text, as.graphicsAnnot(x$label), x$x, x$y, :
неизвестна ширина символа 0xe3 в кодировке CP1251
\end{verbatim}

\begin{verbatim}
Warning in grid.Call.graphics(C_text, as.graphicsAnnot(x$label), x$x, x$y, :
неизвестна ширина символа 0xe8 в кодировке CP1251
\end{verbatim}

\begin{verbatim}
Warning in grid.Call.graphics(C_text, as.graphicsAnnot(x$label), x$x, x$y, :
неизвестна ширина символа 0xff в кодировке CP1251
\end{verbatim}

\begin{verbatim}
Warning in grid.Call.graphics(C_text, as.graphicsAnnot(x$label), x$x, x$y, :
неизвестна ширина символа 0xcc в кодировке CP1251
\end{verbatim}

\begin{verbatim}
Warning in grid.Call.graphics(C_text, as.graphicsAnnot(x$label), x$x, x$y, :
неизвестна ширина символа 0xe5 в кодировке CP1251
\end{verbatim}

\begin{verbatim}
Warning in grid.Call.graphics(C_text, as.graphicsAnnot(x$label), x$x, x$y, :
неизвестна ширина символа 0xe4 в кодировке CP1251
\end{verbatim}

\begin{verbatim}
Warning in grid.Call.graphics(C_text, as.graphicsAnnot(x$label), x$x, x$y, :
неизвестна ширина символа 0xe8 в кодировке CP1251
\end{verbatim}

\begin{verbatim}
Warning in grid.Call.graphics(C_text, as.graphicsAnnot(x$label), x$x, x$y, :
неизвестна ширина символа 0xe0 в кодировке CP1251
\end{verbatim}

\begin{verbatim}
Warning in grid.Call.graphics(C_text, as.graphicsAnnot(x$label), x$x, x$y, :
неизвестна ширина символа 0xed в кодировке CP1251
\end{verbatim}

\begin{verbatim}
Warning in grid.Call.graphics(C_text, as.graphicsAnnot(x$label), x$x, x$y, :
неизвестна ширина символа 0xe0 в кодировке CP1251
\end{verbatim}

\begin{verbatim}
Warning in grid.Call.graphics(C_text, as.graphicsAnnot(x$label), x$x, x$y, :
неизвестна ширина символа 0xe8 в кодировке CP1251
\end{verbatim}

\begin{verbatim}
Warning in grid.Call.graphics(C_text, as.graphicsAnnot(x$label), x$x, x$y, :
неизвестна ширина символа 0xec в кодировке CP1251
\end{verbatim}

\begin{verbatim}
Warning in grid.Call.graphics(C_text, as.graphicsAnnot(x$label), x$x, x$y, :
неизвестна ширина символа 0xe5 в кодировке CP1251
\end{verbatim}

\begin{verbatim}
Warning in grid.Call.graphics(C_text, as.graphicsAnnot(x$label), x$x, x$y, :
неизвестна ширина символа 0xe6 в кодировке CP1251
\end{verbatim}

\begin{verbatim}
Warning in grid.Call.graphics(C_text, as.graphicsAnnot(x$label), x$x, x$y, :
неизвестна ширина символа 0xea в кодировке CP1251
\end{verbatim}

\begin{verbatim}
Warning in grid.Call.graphics(C_text, as.graphicsAnnot(x$label), x$x, x$y, :
неизвестна ширина символа 0xe2 в кодировке CP1251
\end{verbatim}

\begin{verbatim}
Warning in grid.Call.graphics(C_text, as.graphicsAnnot(x$label), x$x, x$y, :
неизвестна ширина символа 0xe0 в кодировке CP1251
\end{verbatim}

\begin{verbatim}
Warning in grid.Call.graphics(C_text, as.graphicsAnnot(x$label), x$x, x$y, :
неизвестна ширина символа 0xf0 в кодировке CP1251
\end{verbatim}

\begin{verbatim}
Warning in grid.Call.graphics(C_text, as.graphicsAnnot(x$label), x$x, x$y, :
неизвестна ширина символа 0xf2 в кодировке CP1251
\end{verbatim}

\begin{verbatim}
Warning in grid.Call.graphics(C_text, as.graphicsAnnot(x$label), x$x, x$y, :
неизвестна ширина символа 0xe8 в кодировке CP1251
\end{verbatim}

\begin{verbatim}
Warning in grid.Call.graphics(C_text, as.graphicsAnnot(x$label), x$x, x$y, :
неизвестна ширина символа 0xeb в кодировке CP1251
\end{verbatim}

\begin{verbatim}
Warning in grid.Call.graphics(C_text, as.graphicsAnnot(x$label), x$x, x$y, :
неизвестна ширина символа 0xfc в кодировке CP1251
\end{verbatim}

\begin{verbatim}
Warning in grid.Call.graphics(C_text, as.graphicsAnnot(x$label), x$x, x$y, :
неизвестна ширина символа 0xed в кодировке CP1251
\end{verbatim}

\begin{verbatim}
Warning in grid.Call.graphics(C_text, as.graphicsAnnot(x$label), x$x, x$y, :
неизвестна ширина символа 0xfb в кодировке CP1251
\end{verbatim}

\begin{verbatim}
Warning in grid.Call.graphics(C_text, as.graphicsAnnot(x$label), x$x, x$y, :
неизвестна ширина символа 0xe9 в кодировке CP1251
\end{verbatim}

\begin{verbatim}
Warning in grid.Call.graphics(C_text, as.graphicsAnnot(x$label), x$x, x$y, :
неизвестна ширина символа 0xf0 в кодировке CP1251
\end{verbatim}

\begin{verbatim}
Warning in grid.Call.graphics(C_text, as.graphicsAnnot(x$label), x$x, x$y, :
неизвестна ширина символа 0xe0 в кодировке CP1251
\end{verbatim}

\begin{verbatim}
Warning in grid.Call.graphics(C_text, as.graphicsAnnot(x$label), x$x, x$y, :
неизвестна ширина символа 0xe7 в кодировке CP1251
\end{verbatim}

\begin{verbatim}
Warning in grid.Call.graphics(C_text, as.graphicsAnnot(x$label), x$x, x$y, :
неизвестна ширина символа 0xec в кодировке CP1251
\end{verbatim}

\begin{verbatim}
Warning in grid.Call.graphics(C_text, as.graphicsAnnot(x$label), x$x, x$y, :
неизвестна ширина символа 0xe0 в кодировке CP1251
\end{verbatim}

\begin{verbatim}
Warning in grid.Call.graphics(C_text, as.graphicsAnnot(x$label), x$x, x$y, :
неизвестна ширина символа 0xf5 в кодировке CP1251
\end{verbatim}

\begin{verbatim}
Warning in grid.Call.graphics(C_text, as.graphicsAnnot(x$label), x$x, x$y, :
неизвестна ширина символа 0xc4 в кодировке CP1251
\end{verbatim}

\begin{verbatim}
Warning in grid.Call.graphics(C_text, as.graphicsAnnot(x$label), x$x, x$y, :
неизвестна ширина символа 0xe8 в кодировке CP1251
\end{verbatim}

\begin{verbatim}
Warning in grid.Call.graphics(C_text, as.graphicsAnnot(x$label), x$x, x$y, :
неизвестна ширина символа 0xed в кодировке CP1251
\end{verbatim}

\begin{verbatim}
Warning in grid.Call.graphics(C_text, as.graphicsAnnot(x$label), x$x, x$y, :
неизвестна ширина символа 0xe0 в кодировке CP1251
\end{verbatim}

\begin{verbatim}
Warning in grid.Call.graphics(C_text, as.graphicsAnnot(x$label), x$x, x$y, :
неизвестна ширина символа 0xec в кодировке CP1251
\end{verbatim}

\begin{verbatim}
Warning in grid.Call.graphics(C_text, as.graphicsAnnot(x$label), x$x, x$y, :
неизвестна ширина символа 0xe8 в кодировке CP1251
\end{verbatim}

\begin{verbatim}
Warning in grid.Call.graphics(C_text, as.graphicsAnnot(x$label), x$x, x$y, :
неизвестна ширина символа 0xea в кодировке CP1251
\end{verbatim}

\begin{verbatim}
Warning in grid.Call.graphics(C_text, as.graphicsAnnot(x$label), x$x, x$y, :
неизвестна ширина символа 0xe0 в кодировке CP1251
\end{verbatim}

\begin{verbatim}
Warning in grid.Call.graphics(C_text, as.graphicsAnnot(x$label), x$x, x$y, :
неизвестна ширина символа 0xe2 в кодировке CP1251
\end{verbatim}

\begin{verbatim}
Warning in grid.Call.graphics(C_text, as.graphicsAnnot(x$label), x$x, x$y, :
неизвестна ширина символа 0xfb в кодировке CP1251
\end{verbatim}

\begin{verbatim}
Warning in grid.Call.graphics(C_text, as.graphicsAnnot(x$label), x$x, x$y, :
неизвестна ширина символа 0xeb в кодировке CP1251
\end{verbatim}

\begin{verbatim}
Warning in grid.Call.graphics(C_text, as.graphicsAnnot(x$label), x$x, x$y, :
неизвестна ширина символа 0xee в кодировке CP1251
\end{verbatim}

\begin{verbatim}
Warning in grid.Call.graphics(C_text, as.graphicsAnnot(x$label), x$x, x$y, :
неизвестна ширина символа 0xe2 в кодировке CP1251
\end{verbatim}

\begin{verbatim}
Warning in grid.Call.graphics(C_text, as.graphicsAnnot(x$label), x$x, x$y, :
неизвестна ширина символа 0xe0 в кодировке CP1251
\end{verbatim}

\pandocbounded{\includegraphics[keepaspectratio]{chapter15_files/figure-pdf/unnamed-chunk-1-3.pdf}}

\begin{Shaded}
\begin{Highlighting}[]
\DocumentationTok{\#\# 9.6 График 4: Фазовая диаграмма Кобе для последнего года}
\NormalTok{kobe\_data }\OtherTok{\textless{}{-}}\NormalTok{ mse\_results }\SpecialCharTok{\%\textgreater{}\%}
  \FunctionTok{filter}\NormalTok{(year }\SpecialCharTok{==} \FunctionTok{max}\NormalTok{(year)) }\SpecialCharTok{\%\textgreater{}\%}
  \FunctionTok{group\_by}\NormalTok{(strategy) }\SpecialCharTok{\%\textgreater{}\%}
  \FunctionTok{sample\_n}\NormalTok{(}\FunctionTok{min}\NormalTok{(}\DecValTok{100}\NormalTok{, }\FunctionTok{n}\NormalTok{()))  }\CommentTok{\# Максимум 100 точек для читаемости}

\NormalTok{p4\_kobe }\OtherTok{\textless{}{-}} \FunctionTok{ggplot}\NormalTok{(kobe\_data, }\FunctionTok{aes}\NormalTok{(}\AttributeTok{x =}\NormalTok{ B\_Bmsy\_true, }\AttributeTok{y =}\NormalTok{ F\_Fmsy\_true)) }\SpecialCharTok{+}
  \CommentTok{\# Зоны Кобе}
  \FunctionTok{annotate}\NormalTok{(}\StringTok{"rect"}\NormalTok{, }\AttributeTok{xmin =} \DecValTok{0}\NormalTok{, }\AttributeTok{xmax =} \DecValTok{1}\NormalTok{, }\AttributeTok{ymin =} \DecValTok{1}\NormalTok{, }\AttributeTok{ymax =} \DecValTok{3}\NormalTok{,}
           \AttributeTok{fill =} \StringTok{"red"}\NormalTok{, }\AttributeTok{alpha =} \FloatTok{0.15}\NormalTok{) }\SpecialCharTok{+}        \CommentTok{\# Красная зона}
  \FunctionTok{annotate}\NormalTok{(}\StringTok{"rect"}\NormalTok{, }\AttributeTok{xmin =} \DecValTok{1}\NormalTok{, }\AttributeTok{xmax =} \DecValTok{3}\NormalTok{, }\AttributeTok{ymin =} \DecValTok{1}\NormalTok{, }\AttributeTok{ymax =} \DecValTok{3}\NormalTok{,}
           \AttributeTok{fill =} \StringTok{"\#FFA500"}\NormalTok{, }\AttributeTok{alpha =} \FloatTok{0.15}\NormalTok{) }\SpecialCharTok{+}    \CommentTok{\# Оранжевая зона  }
  \FunctionTok{annotate}\NormalTok{(}\StringTok{"rect"}\NormalTok{, }\AttributeTok{xmin =} \DecValTok{0}\NormalTok{, }\AttributeTok{xmax =} \DecValTok{1}\NormalTok{, }\AttributeTok{ymin =} \DecValTok{0}\NormalTok{, }\AttributeTok{ymax =} \DecValTok{1}\NormalTok{,}
           \AttributeTok{fill =} \StringTok{"\#FFFF00"}\NormalTok{, }\AttributeTok{alpha =} \FloatTok{0.15}\NormalTok{) }\SpecialCharTok{+}    \CommentTok{\# Желтая зона}
  \FunctionTok{annotate}\NormalTok{(}\StringTok{"rect"}\NormalTok{, }\AttributeTok{xmin =} \DecValTok{1}\NormalTok{, }\AttributeTok{xmax =} \DecValTok{3}\NormalTok{, }\AttributeTok{ymin =} \DecValTok{0}\NormalTok{, }\AttributeTok{ymax =} \DecValTok{1}\NormalTok{,}
           \AttributeTok{fill =} \StringTok{"green"}\NormalTok{, }\AttributeTok{alpha =} \FloatTok{0.15}\NormalTok{) }\SpecialCharTok{+}      \CommentTok{\# Зеленая зона}
  
  \CommentTok{\# Точки симуляций}
  \FunctionTok{geom\_point}\NormalTok{(}\FunctionTok{aes}\NormalTok{(}\AttributeTok{color =}\NormalTok{ strategy), }\AttributeTok{alpha =} \FloatTok{0.5}\NormalTok{, }\AttributeTok{size =} \FloatTok{1.5}\NormalTok{) }\SpecialCharTok{+}
  
  \CommentTok{\# Референсные линии}
  \FunctionTok{geom\_vline}\NormalTok{(}\AttributeTok{xintercept =} \DecValTok{1}\NormalTok{, }\AttributeTok{linetype =} \StringTok{"solid"}\NormalTok{, }\AttributeTok{size =} \FloatTok{0.8}\NormalTok{) }\SpecialCharTok{+}
  \FunctionTok{geom\_hline}\NormalTok{(}\AttributeTok{yintercept =} \DecValTok{1}\NormalTok{, }\AttributeTok{linetype =} \StringTok{"solid"}\NormalTok{, }\AttributeTok{size =} \FloatTok{0.8}\NormalTok{) }\SpecialCharTok{+}
  
  \CommentTok{\# Настройки}
  \FunctionTok{facet\_wrap}\NormalTok{(}\SpecialCharTok{\textasciitilde{}}\NormalTok{ strategy, }\AttributeTok{ncol =} \DecValTok{3}\NormalTok{) }\SpecialCharTok{+}
  \FunctionTok{scale\_color\_manual}\NormalTok{(}\AttributeTok{values =} \FunctionTok{c}\NormalTok{(}\StringTok{"Fish at Fmsy"} \OtherTok{=} \StringTok{"\#E41A1C"}\NormalTok{, }
                               \StringTok{"MSY Hockey{-}stick"} \OtherTok{=} \StringTok{"\#377EB8"}\NormalTok{,}
                               \StringTok{"ICES Advice Rule"} \OtherTok{=} \StringTok{"\#4DAF4A"}\NormalTok{)) }\SpecialCharTok{+}
  \FunctionTok{labs}\NormalTok{(}\AttributeTok{title =} \StringTok{"Фазовая диаграмма Кобе (год 20)"}\NormalTok{,}
       \AttributeTok{subtitle =} \StringTok{"Распределение конечных состояний"}\NormalTok{,}
       \AttributeTok{x =} \StringTok{"B/Bmsy"}\NormalTok{, }
       \AttributeTok{y =} \StringTok{"F/Fmsy"}\NormalTok{) }\SpecialCharTok{+}
\NormalTok{  theme\_mse }\SpecialCharTok{+}
  \FunctionTok{theme}\NormalTok{(}\AttributeTok{legend.position =} \StringTok{"none"}\NormalTok{) }\SpecialCharTok{+}
  \FunctionTok{coord\_cartesian}\NormalTok{(}\AttributeTok{xlim =} \FunctionTok{c}\NormalTok{(}\DecValTok{0}\NormalTok{, }\DecValTok{2}\NormalTok{), }\AttributeTok{ylim =} \FunctionTok{c}\NormalTok{(}\DecValTok{0}\NormalTok{, }\DecValTok{2}\NormalTok{))}

\NormalTok{p4\_kobe}
\end{Highlighting}
\end{Shaded}

\begin{verbatim}
Warning in grid.Call(C_textBounds, as.graphicsAnnot(x$label), x$x, x$y, :
неизвестна ширина символа 0xd4 в кодировке CP1251
\end{verbatim}

\begin{verbatim}
Warning in grid.Call(C_textBounds, as.graphicsAnnot(x$label), x$x, x$y, :
неизвестна ширина символа 0xe0 в кодировке CP1251
\end{verbatim}

\begin{verbatim}
Warning in grid.Call(C_textBounds, as.graphicsAnnot(x$label), x$x, x$y, :
неизвестна ширина символа 0xe7 в кодировке CP1251
\end{verbatim}

\begin{verbatim}
Warning in grid.Call(C_textBounds, as.graphicsAnnot(x$label), x$x, x$y, :
неизвестна ширина символа 0xee в кодировке CP1251
\end{verbatim}

\begin{verbatim}
Warning in grid.Call(C_textBounds, as.graphicsAnnot(x$label), x$x, x$y, :
неизвестна ширина символа 0xe2 в кодировке CP1251
\end{verbatim}

\begin{verbatim}
Warning in grid.Call(C_textBounds, as.graphicsAnnot(x$label), x$x, x$y, :
неизвестна ширина символа 0xe0 в кодировке CP1251
\end{verbatim}

\begin{verbatim}
Warning in grid.Call(C_textBounds, as.graphicsAnnot(x$label), x$x, x$y, :
неизвестна ширина символа 0xff в кодировке CP1251
\end{verbatim}

\begin{verbatim}
Warning in grid.Call(C_textBounds, as.graphicsAnnot(x$label), x$x, x$y, :
неизвестна ширина символа 0xe4 в кодировке CP1251
\end{verbatim}

\begin{verbatim}
Warning in grid.Call(C_textBounds, as.graphicsAnnot(x$label), x$x, x$y, :
неизвестна ширина символа 0xe8 в кодировке CP1251
\end{verbatim}

\begin{verbatim}
Warning in grid.Call(C_textBounds, as.graphicsAnnot(x$label), x$x, x$y, :
неизвестна ширина символа 0xe0 в кодировке CP1251
\end{verbatim}

\begin{verbatim}
Warning in grid.Call(C_textBounds, as.graphicsAnnot(x$label), x$x, x$y, :
неизвестна ширина символа 0xe3 в кодировке CP1251
\end{verbatim}

\begin{verbatim}
Warning in grid.Call(C_textBounds, as.graphicsAnnot(x$label), x$x, x$y, :
неизвестна ширина символа 0xf0 в кодировке CP1251
\end{verbatim}

\begin{verbatim}
Warning in grid.Call(C_textBounds, as.graphicsAnnot(x$label), x$x, x$y, :
неизвестна ширина символа 0xe0 в кодировке CP1251
\end{verbatim}

\begin{verbatim}
Warning in grid.Call(C_textBounds, as.graphicsAnnot(x$label), x$x, x$y, :
неизвестна ширина символа 0xec в кодировке CP1251
Warning in grid.Call(C_textBounds, as.graphicsAnnot(x$label), x$x, x$y, :
неизвестна ширина символа 0xec в кодировке CP1251
\end{verbatim}

\begin{verbatim}
Warning in grid.Call(C_textBounds, as.graphicsAnnot(x$label), x$x, x$y, :
неизвестна ширина символа 0xe0 в кодировке CP1251
\end{verbatim}

\begin{verbatim}
Warning in grid.Call(C_textBounds, as.graphicsAnnot(x$label), x$x, x$y, :
неизвестна ширина символа 0xca в кодировке CP1251
\end{verbatim}

\begin{verbatim}
Warning in grid.Call(C_textBounds, as.graphicsAnnot(x$label), x$x, x$y, :
неизвестна ширина символа 0xee в кодировке CP1251
\end{verbatim}

\begin{verbatim}
Warning in grid.Call(C_textBounds, as.graphicsAnnot(x$label), x$x, x$y, :
неизвестна ширина символа 0xe1 в кодировке CP1251
\end{verbatim}

\begin{verbatim}
Warning in grid.Call(C_textBounds, as.graphicsAnnot(x$label), x$x, x$y, :
неизвестна ширина символа 0xe5 в кодировке CP1251
\end{verbatim}

\begin{verbatim}
Warning in grid.Call(C_textBounds, as.graphicsAnnot(x$label), x$x, x$y, :
неизвестна ширина символа 0xe3 в кодировке CP1251
\end{verbatim}

\begin{verbatim}
Warning in grid.Call(C_textBounds, as.graphicsAnnot(x$label), x$x, x$y, :
неизвестна ширина символа 0xee в кодировке CP1251
\end{verbatim}

\begin{verbatim}
Warning in grid.Call(C_textBounds, as.graphicsAnnot(x$label), x$x, x$y, :
неизвестна ширина символа 0xe4 в кодировке CP1251
\end{verbatim}

\begin{verbatim}
Warning in grid.Call(C_textBounds, as.graphicsAnnot(x$label), x$x, x$y, :
неизвестна ширина символа 0xd0 в кодировке CP1251
\end{verbatim}

\begin{verbatim}
Warning in grid.Call(C_textBounds, as.graphicsAnnot(x$label), x$x, x$y, :
неизвестна ширина символа 0xe0 в кодировке CP1251
\end{verbatim}

\begin{verbatim}
Warning in grid.Call(C_textBounds, as.graphicsAnnot(x$label), x$x, x$y, :
неизвестна ширина символа 0xf1 в кодировке CP1251
\end{verbatim}

\begin{verbatim}
Warning in grid.Call(C_textBounds, as.graphicsAnnot(x$label), x$x, x$y, :
неизвестна ширина символа 0xef в кодировке CP1251
\end{verbatim}

\begin{verbatim}
Warning in grid.Call(C_textBounds, as.graphicsAnnot(x$label), x$x, x$y, :
неизвестна ширина символа 0xf0 в кодировке CP1251
\end{verbatim}

\begin{verbatim}
Warning in grid.Call(C_textBounds, as.graphicsAnnot(x$label), x$x, x$y, :
неизвестна ширина символа 0xe5 в кодировке CP1251
\end{verbatim}

\begin{verbatim}
Warning in grid.Call(C_textBounds, as.graphicsAnnot(x$label), x$x, x$y, :
неизвестна ширина символа 0xe4 в кодировке CP1251
\end{verbatim}

\begin{verbatim}
Warning in grid.Call(C_textBounds, as.graphicsAnnot(x$label), x$x, x$y, :
неизвестна ширина символа 0xe5 в кодировке CP1251
\end{verbatim}

\begin{verbatim}
Warning in grid.Call(C_textBounds, as.graphicsAnnot(x$label), x$x, x$y, :
неизвестна ширина символа 0xeb в кодировке CP1251
\end{verbatim}

\begin{verbatim}
Warning in grid.Call(C_textBounds, as.graphicsAnnot(x$label), x$x, x$y, :
неизвестна ширина символа 0xe5 в кодировке CP1251
\end{verbatim}

\begin{verbatim}
Warning in grid.Call(C_textBounds, as.graphicsAnnot(x$label), x$x, x$y, :
неизвестна ширина символа 0xed в кодировке CP1251
\end{verbatim}

\begin{verbatim}
Warning in grid.Call(C_textBounds, as.graphicsAnnot(x$label), x$x, x$y, :
неизвестна ширина символа 0xe8 в кодировке CP1251
\end{verbatim}

\begin{verbatim}
Warning in grid.Call(C_textBounds, as.graphicsAnnot(x$label), x$x, x$y, :
неизвестна ширина символа 0xe5 в кодировке CP1251
\end{verbatim}

\begin{verbatim}
Warning in grid.Call(C_textBounds, as.graphicsAnnot(x$label), x$x, x$y, :
неизвестна ширина символа 0xea в кодировке CP1251
\end{verbatim}

\begin{verbatim}
Warning in grid.Call(C_textBounds, as.graphicsAnnot(x$label), x$x, x$y, :
неизвестна ширина символа 0xee в кодировке CP1251
\end{verbatim}

\begin{verbatim}
Warning in grid.Call(C_textBounds, as.graphicsAnnot(x$label), x$x, x$y, :
неизвестна ширина символа 0xed в кодировке CP1251
\end{verbatim}

\begin{verbatim}
Warning in grid.Call(C_textBounds, as.graphicsAnnot(x$label), x$x, x$y, :
неизвестна ширина символа 0xe5 в кодировке CP1251
\end{verbatim}

\begin{verbatim}
Warning in grid.Call(C_textBounds, as.graphicsAnnot(x$label), x$x, x$y, :
неизвестна ширина символа 0xf7 в кодировке CP1251
\end{verbatim}

\begin{verbatim}
Warning in grid.Call(C_textBounds, as.graphicsAnnot(x$label), x$x, x$y, :
неизвестна ширина символа 0xed в кодировке CP1251
\end{verbatim}

\begin{verbatim}
Warning in grid.Call(C_textBounds, as.graphicsAnnot(x$label), x$x, x$y, :
неизвестна ширина символа 0xfb в кодировке CP1251
\end{verbatim}

\begin{verbatim}
Warning in grid.Call(C_textBounds, as.graphicsAnnot(x$label), x$x, x$y, :
неизвестна ширина символа 0xf5 в кодировке CP1251
\end{verbatim}

\begin{verbatim}
Warning in grid.Call(C_textBounds, as.graphicsAnnot(x$label), x$x, x$y, :
неизвестна ширина символа 0xf1 в кодировке CP1251
\end{verbatim}

\begin{verbatim}
Warning in grid.Call(C_textBounds, as.graphicsAnnot(x$label), x$x, x$y, :
неизвестна ширина символа 0xee в кодировке CP1251
\end{verbatim}

\begin{verbatim}
Warning in grid.Call(C_textBounds, as.graphicsAnnot(x$label), x$x, x$y, :
неизвестна ширина символа 0xf1 в кодировке CP1251
\end{verbatim}

\begin{verbatim}
Warning in grid.Call(C_textBounds, as.graphicsAnnot(x$label), x$x, x$y, :
неизвестна ширина символа 0xf2 в кодировке CP1251
\end{verbatim}

\begin{verbatim}
Warning in grid.Call(C_textBounds, as.graphicsAnnot(x$label), x$x, x$y, :
неизвестна ширина символа 0xee в кодировке CP1251
\end{verbatim}

\begin{verbatim}
Warning in grid.Call(C_textBounds, as.graphicsAnnot(x$label), x$x, x$y, :
неизвестна ширина символа 0xff в кодировке CP1251
\end{verbatim}

\begin{verbatim}
Warning in grid.Call(C_textBounds, as.graphicsAnnot(x$label), x$x, x$y, :
неизвестна ширина символа 0xed в кодировке CP1251
\end{verbatim}

\begin{verbatim}
Warning in grid.Call(C_textBounds, as.graphicsAnnot(x$label), x$x, x$y, :
неизвестна ширина символа 0xe8 в кодировке CP1251
\end{verbatim}

\begin{verbatim}
Warning in grid.Call(C_textBounds, as.graphicsAnnot(x$label), x$x, x$y, :
неизвестна ширина символа 0xe9 в кодировке CP1251
\end{verbatim}

\begin{verbatim}
Warning in grid.Call.graphics(C_text, as.graphicsAnnot(x$label), x$x, x$y, :
неизвестна ширина символа 0xd0 в кодировке CP1251
\end{verbatim}

\begin{verbatim}
Warning in grid.Call.graphics(C_text, as.graphicsAnnot(x$label), x$x, x$y, :
неизвестна ширина символа 0xe0 в кодировке CP1251
\end{verbatim}

\begin{verbatim}
Warning in grid.Call.graphics(C_text, as.graphicsAnnot(x$label), x$x, x$y, :
неизвестна ширина символа 0xf1 в кодировке CP1251
\end{verbatim}

\begin{verbatim}
Warning in grid.Call.graphics(C_text, as.graphicsAnnot(x$label), x$x, x$y, :
неизвестна ширина символа 0xef в кодировке CP1251
\end{verbatim}

\begin{verbatim}
Warning in grid.Call.graphics(C_text, as.graphicsAnnot(x$label), x$x, x$y, :
неизвестна ширина символа 0xf0 в кодировке CP1251
\end{verbatim}

\begin{verbatim}
Warning in grid.Call.graphics(C_text, as.graphicsAnnot(x$label), x$x, x$y, :
неизвестна ширина символа 0xe5 в кодировке CP1251
\end{verbatim}

\begin{verbatim}
Warning in grid.Call.graphics(C_text, as.graphicsAnnot(x$label), x$x, x$y, :
неизвестна ширина символа 0xe4 в кодировке CP1251
\end{verbatim}

\begin{verbatim}
Warning in grid.Call.graphics(C_text, as.graphicsAnnot(x$label), x$x, x$y, :
неизвестна ширина символа 0xe5 в кодировке CP1251
\end{verbatim}

\begin{verbatim}
Warning in grid.Call.graphics(C_text, as.graphicsAnnot(x$label), x$x, x$y, :
неизвестна ширина символа 0xeb в кодировке CP1251
\end{verbatim}

\begin{verbatim}
Warning in grid.Call.graphics(C_text, as.graphicsAnnot(x$label), x$x, x$y, :
неизвестна ширина символа 0xe5 в кодировке CP1251
\end{verbatim}

\begin{verbatim}
Warning in grid.Call.graphics(C_text, as.graphicsAnnot(x$label), x$x, x$y, :
неизвестна ширина символа 0xed в кодировке CP1251
\end{verbatim}

\begin{verbatim}
Warning in grid.Call.graphics(C_text, as.graphicsAnnot(x$label), x$x, x$y, :
неизвестна ширина символа 0xe8 в кодировке CP1251
\end{verbatim}

\begin{verbatim}
Warning in grid.Call.graphics(C_text, as.graphicsAnnot(x$label), x$x, x$y, :
неизвестна ширина символа 0xe5 в кодировке CP1251
\end{verbatim}

\begin{verbatim}
Warning in grid.Call.graphics(C_text, as.graphicsAnnot(x$label), x$x, x$y, :
неизвестна ширина символа 0xea в кодировке CP1251
\end{verbatim}

\begin{verbatim}
Warning in grid.Call.graphics(C_text, as.graphicsAnnot(x$label), x$x, x$y, :
неизвестна ширина символа 0xee в кодировке CP1251
\end{verbatim}

\begin{verbatim}
Warning in grid.Call.graphics(C_text, as.graphicsAnnot(x$label), x$x, x$y, :
неизвестна ширина символа 0xed в кодировке CP1251
\end{verbatim}

\begin{verbatim}
Warning in grid.Call.graphics(C_text, as.graphicsAnnot(x$label), x$x, x$y, :
неизвестна ширина символа 0xe5 в кодировке CP1251
\end{verbatim}

\begin{verbatim}
Warning in grid.Call.graphics(C_text, as.graphicsAnnot(x$label), x$x, x$y, :
неизвестна ширина символа 0xf7 в кодировке CP1251
\end{verbatim}

\begin{verbatim}
Warning in grid.Call.graphics(C_text, as.graphicsAnnot(x$label), x$x, x$y, :
неизвестна ширина символа 0xed в кодировке CP1251
\end{verbatim}

\begin{verbatim}
Warning in grid.Call.graphics(C_text, as.graphicsAnnot(x$label), x$x, x$y, :
неизвестна ширина символа 0xfb в кодировке CP1251
\end{verbatim}

\begin{verbatim}
Warning in grid.Call.graphics(C_text, as.graphicsAnnot(x$label), x$x, x$y, :
неизвестна ширина символа 0xf5 в кодировке CP1251
\end{verbatim}

\begin{verbatim}
Warning in grid.Call.graphics(C_text, as.graphicsAnnot(x$label), x$x, x$y, :
неизвестна ширина символа 0xf1 в кодировке CP1251
\end{verbatim}

\begin{verbatim}
Warning in grid.Call.graphics(C_text, as.graphicsAnnot(x$label), x$x, x$y, :
неизвестна ширина символа 0xee в кодировке CP1251
\end{verbatim}

\begin{verbatim}
Warning in grid.Call.graphics(C_text, as.graphicsAnnot(x$label), x$x, x$y, :
неизвестна ширина символа 0xf1 в кодировке CP1251
\end{verbatim}

\begin{verbatim}
Warning in grid.Call.graphics(C_text, as.graphicsAnnot(x$label), x$x, x$y, :
неизвестна ширина символа 0xf2 в кодировке CP1251
\end{verbatim}

\begin{verbatim}
Warning in grid.Call.graphics(C_text, as.graphicsAnnot(x$label), x$x, x$y, :
неизвестна ширина символа 0xee в кодировке CP1251
\end{verbatim}

\begin{verbatim}
Warning in grid.Call.graphics(C_text, as.graphicsAnnot(x$label), x$x, x$y, :
неизвестна ширина символа 0xff в кодировке CP1251
\end{verbatim}

\begin{verbatim}
Warning in grid.Call.graphics(C_text, as.graphicsAnnot(x$label), x$x, x$y, :
неизвестна ширина символа 0xed в кодировке CP1251
\end{verbatim}

\begin{verbatim}
Warning in grid.Call.graphics(C_text, as.graphicsAnnot(x$label), x$x, x$y, :
неизвестна ширина символа 0xe8 в кодировке CP1251
\end{verbatim}

\begin{verbatim}
Warning in grid.Call.graphics(C_text, as.graphicsAnnot(x$label), x$x, x$y, :
неизвестна ширина символа 0xe9 в кодировке CP1251
\end{verbatim}

\begin{verbatim}
Warning in grid.Call.graphics(C_text, as.graphicsAnnot(x$label), x$x, x$y, :
неизвестна ширина символа 0xd4 в кодировке CP1251
\end{verbatim}

\begin{verbatim}
Warning in grid.Call.graphics(C_text, as.graphicsAnnot(x$label), x$x, x$y, :
неизвестна ширина символа 0xe0 в кодировке CP1251
\end{verbatim}

\begin{verbatim}
Warning in grid.Call.graphics(C_text, as.graphicsAnnot(x$label), x$x, x$y, :
неизвестна ширина символа 0xe7 в кодировке CP1251
\end{verbatim}

\begin{verbatim}
Warning in grid.Call.graphics(C_text, as.graphicsAnnot(x$label), x$x, x$y, :
неизвестна ширина символа 0xee в кодировке CP1251
\end{verbatim}

\begin{verbatim}
Warning in grid.Call.graphics(C_text, as.graphicsAnnot(x$label), x$x, x$y, :
неизвестна ширина символа 0xe2 в кодировке CP1251
\end{verbatim}

\begin{verbatim}
Warning in grid.Call.graphics(C_text, as.graphicsAnnot(x$label), x$x, x$y, :
неизвестна ширина символа 0xe0 в кодировке CP1251
\end{verbatim}

\begin{verbatim}
Warning in grid.Call.graphics(C_text, as.graphicsAnnot(x$label), x$x, x$y, :
неизвестна ширина символа 0xff в кодировке CP1251
\end{verbatim}

\begin{verbatim}
Warning in grid.Call.graphics(C_text, as.graphicsAnnot(x$label), x$x, x$y, :
неизвестна ширина символа 0xe4 в кодировке CP1251
\end{verbatim}

\begin{verbatim}
Warning in grid.Call.graphics(C_text, as.graphicsAnnot(x$label), x$x, x$y, :
неизвестна ширина символа 0xe8 в кодировке CP1251
\end{verbatim}

\begin{verbatim}
Warning in grid.Call.graphics(C_text, as.graphicsAnnot(x$label), x$x, x$y, :
неизвестна ширина символа 0xe0 в кодировке CP1251
\end{verbatim}

\begin{verbatim}
Warning in grid.Call.graphics(C_text, as.graphicsAnnot(x$label), x$x, x$y, :
неизвестна ширина символа 0xe3 в кодировке CP1251
\end{verbatim}

\begin{verbatim}
Warning in grid.Call.graphics(C_text, as.graphicsAnnot(x$label), x$x, x$y, :
неизвестна ширина символа 0xf0 в кодировке CP1251
\end{verbatim}

\begin{verbatim}
Warning in grid.Call.graphics(C_text, as.graphicsAnnot(x$label), x$x, x$y, :
неизвестна ширина символа 0xe0 в кодировке CP1251
\end{verbatim}

\begin{verbatim}
Warning in grid.Call.graphics(C_text, as.graphicsAnnot(x$label), x$x, x$y, :
неизвестна ширина символа 0xec в кодировке CP1251
Warning in grid.Call.graphics(C_text, as.graphicsAnnot(x$label), x$x, x$y, :
неизвестна ширина символа 0xec в кодировке CP1251
\end{verbatim}

\begin{verbatim}
Warning in grid.Call.graphics(C_text, as.graphicsAnnot(x$label), x$x, x$y, :
неизвестна ширина символа 0xe0 в кодировке CP1251
\end{verbatim}

\begin{verbatim}
Warning in grid.Call.graphics(C_text, as.graphicsAnnot(x$label), x$x, x$y, :
неизвестна ширина символа 0xca в кодировке CP1251
\end{verbatim}

\begin{verbatim}
Warning in grid.Call.graphics(C_text, as.graphicsAnnot(x$label), x$x, x$y, :
неизвестна ширина символа 0xee в кодировке CP1251
\end{verbatim}

\begin{verbatim}
Warning in grid.Call.graphics(C_text, as.graphicsAnnot(x$label), x$x, x$y, :
неизвестна ширина символа 0xe1 в кодировке CP1251
\end{verbatim}

\begin{verbatim}
Warning in grid.Call.graphics(C_text, as.graphicsAnnot(x$label), x$x, x$y, :
неизвестна ширина символа 0xe5 в кодировке CP1251
\end{verbatim}

\begin{verbatim}
Warning in grid.Call.graphics(C_text, as.graphicsAnnot(x$label), x$x, x$y, :
неизвестна ширина символа 0xe3 в кодировке CP1251
\end{verbatim}

\begin{verbatim}
Warning in grid.Call.graphics(C_text, as.graphicsAnnot(x$label), x$x, x$y, :
неизвестна ширина символа 0xee в кодировке CP1251
\end{verbatim}

\begin{verbatim}
Warning in grid.Call.graphics(C_text, as.graphicsAnnot(x$label), x$x, x$y, :
неизвестна ширина символа 0xe4 в кодировке CP1251
\end{verbatim}

\pandocbounded{\includegraphics[keepaspectratio]{chapter15_files/figure-pdf/unnamed-chunk-1-4.pdf}}

\begin{Shaded}
\begin{Highlighting}[]
\DocumentationTok{\#\# 9.7 График 5: Сравнение риск{-}метрик (барплот)}
\NormalTok{metrics\_long }\OtherTok{\textless{}{-}}\NormalTok{ performance\_metrics }\SpecialCharTok{\%\textgreater{}\%}
  \FunctionTok{select}\NormalTok{(strategy, prob\_overfishing, prob\_overfished, }
\NormalTok{         prob\_collapsed, prob\_green\_zone) }\SpecialCharTok{\%\textgreater{}\%}
  \FunctionTok{pivot\_longer}\NormalTok{(}\AttributeTok{cols =} \SpecialCharTok{{-}}\NormalTok{strategy, }\AttributeTok{names\_to =} \StringTok{"metric"}\NormalTok{, }\AttributeTok{values\_to =} \StringTok{"probability"}\NormalTok{)}

\CommentTok{\# Переименование метрик для графика}
\NormalTok{metric\_labels }\OtherTok{\textless{}{-}} \FunctionTok{c}\NormalTok{(}
  \AttributeTok{prob\_overfishing =} \StringTok{"Перелов}\SpecialCharTok{\textbackslash{}n}\StringTok{(F \textgreater{} Fmsy)"}\NormalTok{,}
  \AttributeTok{prob\_overfished =} \StringTok{"Истощение}\SpecialCharTok{\textbackslash{}n}\StringTok{(B \textless{} 0.5 Bmsy)"}\NormalTok{,}
  \AttributeTok{prob\_collapsed =} \StringTok{"Коллапс}\SpecialCharTok{\textbackslash{}n}\StringTok{(B \textless{} 0.2 Bmsy)"}\NormalTok{,}
  \AttributeTok{prob\_green\_zone =} \StringTok{"Зеленая зона}\SpecialCharTok{\textbackslash{}n}\StringTok{(B \textgreater{} Bmsy, F \textless{} Fmsy)"}
\NormalTok{)}

\NormalTok{p5\_risks }\OtherTok{\textless{}{-}} \FunctionTok{ggplot}\NormalTok{(metrics\_long, }\FunctionTok{aes}\NormalTok{(}\AttributeTok{x =}\NormalTok{ metric, }\AttributeTok{y =}\NormalTok{ probability, }\AttributeTok{fill =}\NormalTok{ strategy)) }\SpecialCharTok{+}
  \FunctionTok{geom\_bar}\NormalTok{(}\AttributeTok{stat =} \StringTok{"identity"}\NormalTok{, }\AttributeTok{position =} \FunctionTok{position\_dodge}\NormalTok{(}\AttributeTok{width =} \FloatTok{0.8}\NormalTok{), }\AttributeTok{width =} \FloatTok{0.7}\NormalTok{) }\SpecialCharTok{+}
  \FunctionTok{geom\_text}\NormalTok{(}\FunctionTok{aes}\NormalTok{(}\AttributeTok{label =} \FunctionTok{sprintf}\NormalTok{(}\StringTok{"\%.1f\%\%"}\NormalTok{, probability }\SpecialCharTok{*} \DecValTok{100}\NormalTok{)),}
            \AttributeTok{position =} \FunctionTok{position\_dodge}\NormalTok{(}\AttributeTok{width =} \FloatTok{0.8}\NormalTok{), }
            \AttributeTok{vjust =} \SpecialCharTok{{-}}\FloatTok{0.5}\NormalTok{, }\AttributeTok{size =} \DecValTok{3}\NormalTok{) }\SpecialCharTok{+}
  \FunctionTok{scale\_fill\_manual}\NormalTok{(}\AttributeTok{values =} \FunctionTok{c}\NormalTok{(}\StringTok{"Fish at Fmsy"} \OtherTok{=} \StringTok{"\#E41A1C"}\NormalTok{, }
                              \StringTok{"MSY Hockey{-}stick"} \OtherTok{=} \StringTok{"\#377EB8"}\NormalTok{,}
                              \StringTok{"ICES Advice Rule"} \OtherTok{=} \StringTok{"\#4DAF4A"}\NormalTok{)) }\SpecialCharTok{+}
  \FunctionTok{scale\_x\_discrete}\NormalTok{(}\AttributeTok{labels =}\NormalTok{ metric\_labels) }\SpecialCharTok{+}
  \FunctionTok{scale\_y\_continuous}\NormalTok{(}\AttributeTok{labels =}\NormalTok{ scales}\SpecialCharTok{::}\NormalTok{percent, }\AttributeTok{limits =} \FunctionTok{c}\NormalTok{(}\DecValTok{0}\NormalTok{, }\FunctionTok{max}\NormalTok{(metrics\_long}\SpecialCharTok{$}\NormalTok{probability) }\SpecialCharTok{*} \FloatTok{1.1}\NormalTok{)) }\SpecialCharTok{+}
  \FunctionTok{labs}\NormalTok{(}\AttributeTok{title =} \StringTok{"Сравнение вероятностей риска"}\NormalTok{,}
       \AttributeTok{subtitle =} \StringTok{"По всему периоду симуляции"}\NormalTok{,}
       \AttributeTok{x =} \StringTok{""}\NormalTok{, }
       \AttributeTok{y =} \StringTok{"Вероятность"}\NormalTok{,}
       \AttributeTok{fill =} \StringTok{"Стратегия"}\NormalTok{) }\SpecialCharTok{+}
\NormalTok{  theme\_mse}

\NormalTok{p5\_risks}
\end{Highlighting}
\end{Shaded}

\begin{verbatim}
Warning in grid.Call(C_textBounds, as.graphicsAnnot(x$label), x$x, x$y, :
неизвестна ширина символа 0xca в кодировке CP1251
\end{verbatim}

\begin{verbatim}
Warning in grid.Call(C_textBounds, as.graphicsAnnot(x$label), x$x, x$y, :
неизвестна ширина символа 0xee в кодировке CP1251
\end{verbatim}

\begin{verbatim}
Warning in grid.Call(C_textBounds, as.graphicsAnnot(x$label), x$x, x$y, :
неизвестна ширина символа 0xeb в кодировке CP1251
Warning in grid.Call(C_textBounds, as.graphicsAnnot(x$label), x$x, x$y, :
неизвестна ширина символа 0xeb в кодировке CP1251
\end{verbatim}

\begin{verbatim}
Warning in grid.Call(C_textBounds, as.graphicsAnnot(x$label), x$x, x$y, :
неизвестна ширина символа 0xe0 в кодировке CP1251
\end{verbatim}

\begin{verbatim}
Warning in grid.Call(C_textBounds, as.graphicsAnnot(x$label), x$x, x$y, :
неизвестна ширина символа 0xef в кодировке CP1251
\end{verbatim}

\begin{verbatim}
Warning in grid.Call(C_textBounds, as.graphicsAnnot(x$label), x$x, x$y, :
неизвестна ширина символа 0xf1 в кодировке CP1251
\end{verbatim}

\begin{verbatim}
Warning in grid.Call(C_textBounds, as.graphicsAnnot(x$label), x$x, x$y, :
неизвестна ширина символа 0xc7 в кодировке CP1251
\end{verbatim}

\begin{verbatim}
Warning in grid.Call(C_textBounds, as.graphicsAnnot(x$label), x$x, x$y, :
неизвестна ширина символа 0xe5 в кодировке CP1251
\end{verbatim}

\begin{verbatim}
Warning in grid.Call(C_textBounds, as.graphicsAnnot(x$label), x$x, x$y, :
неизвестна ширина символа 0xeb в кодировке CP1251
\end{verbatim}

\begin{verbatim}
Warning in grid.Call(C_textBounds, as.graphicsAnnot(x$label), x$x, x$y, :
неизвестна ширина символа 0xe5 в кодировке CP1251
\end{verbatim}

\begin{verbatim}
Warning in grid.Call(C_textBounds, as.graphicsAnnot(x$label), x$x, x$y, :
неизвестна ширина символа 0xed в кодировке CP1251
\end{verbatim}

\begin{verbatim}
Warning in grid.Call(C_textBounds, as.graphicsAnnot(x$label), x$x, x$y, :
неизвестна ширина символа 0xe0 в кодировке CP1251
\end{verbatim}

\begin{verbatim}
Warning in grid.Call(C_textBounds, as.graphicsAnnot(x$label), x$x, x$y, :
неизвестна ширина символа 0xff в кодировке CP1251
\end{verbatim}

\begin{verbatim}
Warning in grid.Call(C_textBounds, as.graphicsAnnot(x$label), x$x, x$y, :
неизвестна ширина символа 0xe7 в кодировке CP1251
\end{verbatim}

\begin{verbatim}
Warning in grid.Call(C_textBounds, as.graphicsAnnot(x$label), x$x, x$y, :
неизвестна ширина символа 0xee в кодировке CP1251
\end{verbatim}

\begin{verbatim}
Warning in grid.Call(C_textBounds, as.graphicsAnnot(x$label), x$x, x$y, :
неизвестна ширина символа 0xed в кодировке CP1251
\end{verbatim}

\begin{verbatim}
Warning in grid.Call(C_textBounds, as.graphicsAnnot(x$label), x$x, x$y, :
неизвестна ширина символа 0xe0 в кодировке CP1251
\end{verbatim}

\begin{verbatim}
Warning in grid.Call(C_textBounds, as.graphicsAnnot(x$label), x$x, x$y, :
неизвестна ширина символа 0xc8 в кодировке CP1251
\end{verbatim}

\begin{verbatim}
Warning in grid.Call(C_textBounds, as.graphicsAnnot(x$label), x$x, x$y, :
неизвестна ширина символа 0xf1 в кодировке CP1251
\end{verbatim}

\begin{verbatim}
Warning in grid.Call(C_textBounds, as.graphicsAnnot(x$label), x$x, x$y, :
неизвестна ширина символа 0xf2 в кодировке CP1251
\end{verbatim}

\begin{verbatim}
Warning in grid.Call(C_textBounds, as.graphicsAnnot(x$label), x$x, x$y, :
неизвестна ширина символа 0xee в кодировке CP1251
\end{verbatim}

\begin{verbatim}
Warning in grid.Call(C_textBounds, as.graphicsAnnot(x$label), x$x, x$y, :
неизвестна ширина символа 0xf9 в кодировке CP1251
\end{verbatim}

\begin{verbatim}
Warning in grid.Call(C_textBounds, as.graphicsAnnot(x$label), x$x, x$y, :
неизвестна ширина символа 0xe5 в кодировке CP1251
\end{verbatim}

\begin{verbatim}
Warning in grid.Call(C_textBounds, as.graphicsAnnot(x$label), x$x, x$y, :
неизвестна ширина символа 0xed в кодировке CP1251
\end{verbatim}

\begin{verbatim}
Warning in grid.Call(C_textBounds, as.graphicsAnnot(x$label), x$x, x$y, :
неизвестна ширина символа 0xe8 в кодировке CP1251
\end{verbatim}

\begin{verbatim}
Warning in grid.Call(C_textBounds, as.graphicsAnnot(x$label), x$x, x$y, :
неизвестна ширина символа 0xe5 в кодировке CP1251
\end{verbatim}

\begin{verbatim}
Warning in grid.Call(C_textBounds, as.graphicsAnnot(x$label), x$x, x$y, :
неизвестна ширина символа 0xcf в кодировке CP1251
\end{verbatim}

\begin{verbatim}
Warning in grid.Call(C_textBounds, as.graphicsAnnot(x$label), x$x, x$y, :
неизвестна ширина символа 0xe5 в кодировке CP1251
\end{verbatim}

\begin{verbatim}
Warning in grid.Call(C_textBounds, as.graphicsAnnot(x$label), x$x, x$y, :
неизвестна ширина символа 0xf0 в кодировке CP1251
\end{verbatim}

\begin{verbatim}
Warning in grid.Call(C_textBounds, as.graphicsAnnot(x$label), x$x, x$y, :
неизвестна ширина символа 0xe5 в кодировке CP1251
\end{verbatim}

\begin{verbatim}
Warning in grid.Call(C_textBounds, as.graphicsAnnot(x$label), x$x, x$y, :
неизвестна ширина символа 0xeb в кодировке CP1251
\end{verbatim}

\begin{verbatim}
Warning in grid.Call(C_textBounds, as.graphicsAnnot(x$label), x$x, x$y, :
неизвестна ширина символа 0xee в кодировке CP1251
\end{verbatim}

\begin{verbatim}
Warning in grid.Call(C_textBounds, as.graphicsAnnot(x$label), x$x, x$y, :
неизвестна ширина символа 0xe2 в кодировке CP1251
\end{verbatim}

\begin{verbatim}
Warning in grid.Call(C_textBounds, as.graphicsAnnot(x$label), x$x, x$y, :
неизвестна ширина символа 0xd1 в кодировке CP1251
\end{verbatim}

\begin{verbatim}
Warning in grid.Call(C_textBounds, as.graphicsAnnot(x$label), x$x, x$y, :
неизвестна ширина символа 0xf2 в кодировке CP1251
\end{verbatim}

\begin{verbatim}
Warning in grid.Call(C_textBounds, as.graphicsAnnot(x$label), x$x, x$y, :
неизвестна ширина символа 0xf0 в кодировке CP1251
\end{verbatim}

\begin{verbatim}
Warning in grid.Call(C_textBounds, as.graphicsAnnot(x$label), x$x, x$y, :
неизвестна ширина символа 0xe0 в кодировке CP1251
\end{verbatim}

\begin{verbatim}
Warning in grid.Call(C_textBounds, as.graphicsAnnot(x$label), x$x, x$y, :
неизвестна ширина символа 0xf2 в кодировке CP1251
\end{verbatim}

\begin{verbatim}
Warning in grid.Call(C_textBounds, as.graphicsAnnot(x$label), x$x, x$y, :
неизвестна ширина символа 0xe5 в кодировке CP1251
\end{verbatim}

\begin{verbatim}
Warning in grid.Call(C_textBounds, as.graphicsAnnot(x$label), x$x, x$y, :
неизвестна ширина символа 0xe3 в кодировке CP1251
\end{verbatim}

\begin{verbatim}
Warning in grid.Call(C_textBounds, as.graphicsAnnot(x$label), x$x, x$y, :
неизвестна ширина символа 0xe8 в кодировке CP1251
\end{verbatim}

\begin{verbatim}
Warning in grid.Call(C_textBounds, as.graphicsAnnot(x$label), x$x, x$y, :
неизвестна ширина символа 0xff в кодировке CP1251
\end{verbatim}

\begin{verbatim}
Warning in grid.Call(C_textBounds, as.graphicsAnnot(x$label), x$x, x$y, :
неизвестна ширина символа 0xd1 в кодировке CP1251
\end{verbatim}

\begin{verbatim}
Warning in grid.Call(C_textBounds, as.graphicsAnnot(x$label), x$x, x$y, :
неизвестна ширина символа 0xf2 в кодировке CP1251
\end{verbatim}

\begin{verbatim}
Warning in grid.Call(C_textBounds, as.graphicsAnnot(x$label), x$x, x$y, :
неизвестна ширина символа 0xf0 в кодировке CP1251
\end{verbatim}

\begin{verbatim}
Warning in grid.Call(C_textBounds, as.graphicsAnnot(x$label), x$x, x$y, :
неизвестна ширина символа 0xe0 в кодировке CP1251
\end{verbatim}

\begin{verbatim}
Warning in grid.Call(C_textBounds, as.graphicsAnnot(x$label), x$x, x$y, :
неизвестна ширина символа 0xf2 в кодировке CP1251
\end{verbatim}

\begin{verbatim}
Warning in grid.Call(C_textBounds, as.graphicsAnnot(x$label), x$x, x$y, :
неизвестна ширина символа 0xe5 в кодировке CP1251
\end{verbatim}

\begin{verbatim}
Warning in grid.Call(C_textBounds, as.graphicsAnnot(x$label), x$x, x$y, :
неизвестна ширина символа 0xe3 в кодировке CP1251
\end{verbatim}

\begin{verbatim}
Warning in grid.Call(C_textBounds, as.graphicsAnnot(x$label), x$x, x$y, :
неизвестна ширина символа 0xe8 в кодировке CP1251
\end{verbatim}

\begin{verbatim}
Warning in grid.Call(C_textBounds, as.graphicsAnnot(x$label), x$x, x$y, :
неизвестна ширина символа 0xff в кодировке CP1251
\end{verbatim}

\begin{verbatim}
Warning in grid.Call(C_textBounds, as.graphicsAnnot(x$label), x$x, x$y, :
неизвестна ширина символа 0xc2 в кодировке CP1251
\end{verbatim}

\begin{verbatim}
Warning in grid.Call(C_textBounds, as.graphicsAnnot(x$label), x$x, x$y, :
неизвестна ширина символа 0xe5 в кодировке CP1251
\end{verbatim}

\begin{verbatim}
Warning in grid.Call(C_textBounds, as.graphicsAnnot(x$label), x$x, x$y, :
неизвестна ширина символа 0xf0 в кодировке CP1251
\end{verbatim}

\begin{verbatim}
Warning in grid.Call(C_textBounds, as.graphicsAnnot(x$label), x$x, x$y, :
неизвестна ширина символа 0xee в кодировке CP1251
\end{verbatim}

\begin{verbatim}
Warning in grid.Call(C_textBounds, as.graphicsAnnot(x$label), x$x, x$y, :
неизвестна ширина символа 0xff в кодировке CP1251
\end{verbatim}

\begin{verbatim}
Warning in grid.Call(C_textBounds, as.graphicsAnnot(x$label), x$x, x$y, :
неизвестна ширина символа 0xf2 в кодировке CP1251
\end{verbatim}

\begin{verbatim}
Warning in grid.Call(C_textBounds, as.graphicsAnnot(x$label), x$x, x$y, :
неизвестна ширина символа 0xed в кодировке CP1251
\end{verbatim}

\begin{verbatim}
Warning in grid.Call(C_textBounds, as.graphicsAnnot(x$label), x$x, x$y, :
неизвестна ширина символа 0xee в кодировке CP1251
\end{verbatim}

\begin{verbatim}
Warning in grid.Call(C_textBounds, as.graphicsAnnot(x$label), x$x, x$y, :
неизвестна ширина символа 0xf1 в кодировке CP1251
\end{verbatim}

\begin{verbatim}
Warning in grid.Call(C_textBounds, as.graphicsAnnot(x$label), x$x, x$y, :
неизвестна ширина символа 0xf2 в кодировке CP1251
\end{verbatim}

\begin{verbatim}
Warning in grid.Call(C_textBounds, as.graphicsAnnot(x$label), x$x, x$y, :
неизвестна ширина символа 0xfc в кодировке CP1251
\end{verbatim}

\begin{verbatim}
Warning in grid.Call(C_textBounds, as.graphicsAnnot(x$label), x$x, x$y, :
неизвестна ширина символа 0xd1 в кодировке CP1251
\end{verbatim}

\begin{verbatim}
Warning in grid.Call(C_textBounds, as.graphicsAnnot(x$label), x$x, x$y, :
неизвестна ширина символа 0xf0 в кодировке CP1251
\end{verbatim}

\begin{verbatim}
Warning in grid.Call(C_textBounds, as.graphicsAnnot(x$label), x$x, x$y, :
неизвестна ширина символа 0xe0 в кодировке CP1251
\end{verbatim}

\begin{verbatim}
Warning in grid.Call(C_textBounds, as.graphicsAnnot(x$label), x$x, x$y, :
неизвестна ширина символа 0xe2 в кодировке CP1251
\end{verbatim}

\begin{verbatim}
Warning in grid.Call(C_textBounds, as.graphicsAnnot(x$label), x$x, x$y, :
неизвестна ширина символа 0xed в кодировке CP1251
\end{verbatim}

\begin{verbatim}
Warning in grid.Call(C_textBounds, as.graphicsAnnot(x$label), x$x, x$y, :
неизвестна ширина символа 0xe5 в кодировке CP1251
\end{verbatim}

\begin{verbatim}
Warning in grid.Call(C_textBounds, as.graphicsAnnot(x$label), x$x, x$y, :
неизвестна ширина символа 0xed в кодировке CP1251
\end{verbatim}

\begin{verbatim}
Warning in grid.Call(C_textBounds, as.graphicsAnnot(x$label), x$x, x$y, :
неизвестна ширина символа 0xe8 в кодировке CP1251
\end{verbatim}

\begin{verbatim}
Warning in grid.Call(C_textBounds, as.graphicsAnnot(x$label), x$x, x$y, :
неизвестна ширина символа 0xe5 в кодировке CP1251
\end{verbatim}

\begin{verbatim}
Warning in grid.Call(C_textBounds, as.graphicsAnnot(x$label), x$x, x$y, :
неизвестна ширина символа 0xe2 в кодировке CP1251
\end{verbatim}

\begin{verbatim}
Warning in grid.Call(C_textBounds, as.graphicsAnnot(x$label), x$x, x$y, :
неизвестна ширина символа 0xe5 в кодировке CP1251
\end{verbatim}

\begin{verbatim}
Warning in grid.Call(C_textBounds, as.graphicsAnnot(x$label), x$x, x$y, :
неизвестна ширина символа 0xf0 в кодировке CP1251
\end{verbatim}

\begin{verbatim}
Warning in grid.Call(C_textBounds, as.graphicsAnnot(x$label), x$x, x$y, :
неизвестна ширина символа 0xee в кодировке CP1251
\end{verbatim}

\begin{verbatim}
Warning in grid.Call(C_textBounds, as.graphicsAnnot(x$label), x$x, x$y, :
неизвестна ширина символа 0xff в кодировке CP1251
\end{verbatim}

\begin{verbatim}
Warning in grid.Call(C_textBounds, as.graphicsAnnot(x$label), x$x, x$y, :
неизвестна ширина символа 0xf2 в кодировке CP1251
\end{verbatim}

\begin{verbatim}
Warning in grid.Call(C_textBounds, as.graphicsAnnot(x$label), x$x, x$y, :
неизвестна ширина символа 0xed в кодировке CP1251
\end{verbatim}

\begin{verbatim}
Warning in grid.Call(C_textBounds, as.graphicsAnnot(x$label), x$x, x$y, :
неизвестна ширина символа 0xee в кодировке CP1251
\end{verbatim}

\begin{verbatim}
Warning in grid.Call(C_textBounds, as.graphicsAnnot(x$label), x$x, x$y, :
неизвестна ширина символа 0xf1 в кодировке CP1251
\end{verbatim}

\begin{verbatim}
Warning in grid.Call(C_textBounds, as.graphicsAnnot(x$label), x$x, x$y, :
неизвестна ширина символа 0xf2 в кодировке CP1251
\end{verbatim}

\begin{verbatim}
Warning in grid.Call(C_textBounds, as.graphicsAnnot(x$label), x$x, x$y, :
неизвестна ширина символа 0xe5 в кодировке CP1251
\end{verbatim}

\begin{verbatim}
Warning in grid.Call(C_textBounds, as.graphicsAnnot(x$label), x$x, x$y, :
неизвестна ширина символа 0xe9 в кодировке CP1251
\end{verbatim}

\begin{verbatim}
Warning in grid.Call(C_textBounds, as.graphicsAnnot(x$label), x$x, x$y, :
неизвестна ширина символа 0xf0 в кодировке CP1251
\end{verbatim}

\begin{verbatim}
Warning in grid.Call(C_textBounds, as.graphicsAnnot(x$label), x$x, x$y, :
неизвестна ширина символа 0xe8 в кодировке CP1251
\end{verbatim}

\begin{verbatim}
Warning in grid.Call(C_textBounds, as.graphicsAnnot(x$label), x$x, x$y, :
неизвестна ширина символа 0xf1 в кодировке CP1251
\end{verbatim}

\begin{verbatim}
Warning in grid.Call(C_textBounds, as.graphicsAnnot(x$label), x$x, x$y, :
неизвестна ширина символа 0xea в кодировке CP1251
\end{verbatim}

\begin{verbatim}
Warning in grid.Call(C_textBounds, as.graphicsAnnot(x$label), x$x, x$y, :
неизвестна ширина символа 0xe0 в кодировке CP1251
\end{verbatim}

\begin{verbatim}
Warning in grid.Call(C_textBounds, as.graphicsAnnot(x$label), x$x, x$y, :
неизвестна ширина символа 0xcf в кодировке CP1251
\end{verbatim}

\begin{verbatim}
Warning in grid.Call(C_textBounds, as.graphicsAnnot(x$label), x$x, x$y, :
неизвестна ширина символа 0xee в кодировке CP1251
\end{verbatim}

\begin{verbatim}
Warning in grid.Call(C_textBounds, as.graphicsAnnot(x$label), x$x, x$y, :
неизвестна ширина символа 0xe2 в кодировке CP1251
\end{verbatim}

\begin{verbatim}
Warning in grid.Call(C_textBounds, as.graphicsAnnot(x$label), x$x, x$y, :
неизвестна ширина символа 0xf1 в кодировке CP1251
\end{verbatim}

\begin{verbatim}
Warning in grid.Call(C_textBounds, as.graphicsAnnot(x$label), x$x, x$y, :
неизвестна ширина символа 0xe5 в кодировке CP1251
\end{verbatim}

\begin{verbatim}
Warning in grid.Call(C_textBounds, as.graphicsAnnot(x$label), x$x, x$y, :
неизвестна ширина символа 0xec в кодировке CP1251
\end{verbatim}

\begin{verbatim}
Warning in grid.Call(C_textBounds, as.graphicsAnnot(x$label), x$x, x$y, :
неизвестна ширина символа 0xf3 в кодировке CP1251
\end{verbatim}

\begin{verbatim}
Warning in grid.Call(C_textBounds, as.graphicsAnnot(x$label), x$x, x$y, :
неизвестна ширина символа 0xef в кодировке CP1251
\end{verbatim}

\begin{verbatim}
Warning in grid.Call(C_textBounds, as.graphicsAnnot(x$label), x$x, x$y, :
неизвестна ширина символа 0xe5 в кодировке CP1251
\end{verbatim}

\begin{verbatim}
Warning in grid.Call(C_textBounds, as.graphicsAnnot(x$label), x$x, x$y, :
неизвестна ширина символа 0xf0 в кодировке CP1251
\end{verbatim}

\begin{verbatim}
Warning in grid.Call(C_textBounds, as.graphicsAnnot(x$label), x$x, x$y, :
неизвестна ширина символа 0xe8 в кодировке CP1251
\end{verbatim}

\begin{verbatim}
Warning in grid.Call(C_textBounds, as.graphicsAnnot(x$label), x$x, x$y, :
неизвестна ширина символа 0xee в кодировке CP1251
\end{verbatim}

\begin{verbatim}
Warning in grid.Call(C_textBounds, as.graphicsAnnot(x$label), x$x, x$y, :
неизвестна ширина символа 0xe4 в кодировке CP1251
\end{verbatim}

\begin{verbatim}
Warning in grid.Call(C_textBounds, as.graphicsAnnot(x$label), x$x, x$y, :
неизвестна ширина символа 0xf3 в кодировке CP1251
\end{verbatim}

\begin{verbatim}
Warning in grid.Call(C_textBounds, as.graphicsAnnot(x$label), x$x, x$y, :
неизвестна ширина символа 0xf1 в кодировке CP1251
\end{verbatim}

\begin{verbatim}
Warning in grid.Call(C_textBounds, as.graphicsAnnot(x$label), x$x, x$y, :
неизвестна ширина символа 0xe8 в кодировке CP1251
\end{verbatim}

\begin{verbatim}
Warning in grid.Call(C_textBounds, as.graphicsAnnot(x$label), x$x, x$y, :
неизвестна ширина символа 0xec в кодировке CP1251
\end{verbatim}

\begin{verbatim}
Warning in grid.Call(C_textBounds, as.graphicsAnnot(x$label), x$x, x$y, :
неизвестна ширина символа 0xf3 в кодировке CP1251
\end{verbatim}

\begin{verbatim}
Warning in grid.Call(C_textBounds, as.graphicsAnnot(x$label), x$x, x$y, :
неизвестна ширина символа 0xeb в кодировке CP1251
\end{verbatim}

\begin{verbatim}
Warning in grid.Call(C_textBounds, as.graphicsAnnot(x$label), x$x, x$y, :
неизвестна ширина символа 0xff в кодировке CP1251
\end{verbatim}

\begin{verbatim}
Warning in grid.Call(C_textBounds, as.graphicsAnnot(x$label), x$x, x$y, :
неизвестна ширина символа 0xf6 в кодировке CP1251
\end{verbatim}

\begin{verbatim}
Warning in grid.Call(C_textBounds, as.graphicsAnnot(x$label), x$x, x$y, :
неизвестна ширина символа 0xe8 в кодировке CP1251
Warning in grid.Call(C_textBounds, as.graphicsAnnot(x$label), x$x, x$y, :
неизвестна ширина символа 0xe8 в кодировке CP1251
\end{verbatim}

\begin{verbatim}
Warning in grid.Call.graphics(C_text, as.graphicsAnnot(x$label), x$x, x$y, :
неизвестна ширина символа 0xca в кодировке CP1251
\end{verbatim}

\begin{verbatim}
Warning in grid.Call.graphics(C_text, as.graphicsAnnot(x$label), x$x, x$y, :
неизвестна ширина символа 0xee в кодировке CP1251
\end{verbatim}

\begin{verbatim}
Warning in grid.Call.graphics(C_text, as.graphicsAnnot(x$label), x$x, x$y, :
неизвестна ширина символа 0xeb в кодировке CP1251
Warning in grid.Call.graphics(C_text, as.graphicsAnnot(x$label), x$x, x$y, :
неизвестна ширина символа 0xeb в кодировке CP1251
\end{verbatim}

\begin{verbatim}
Warning in grid.Call.graphics(C_text, as.graphicsAnnot(x$label), x$x, x$y, :
неизвестна ширина символа 0xe0 в кодировке CP1251
\end{verbatim}

\begin{verbatim}
Warning in grid.Call.graphics(C_text, as.graphicsAnnot(x$label), x$x, x$y, :
неизвестна ширина символа 0xef в кодировке CP1251
\end{verbatim}

\begin{verbatim}
Warning in grid.Call.graphics(C_text, as.graphicsAnnot(x$label), x$x, x$y, :
неизвестна ширина символа 0xf1 в кодировке CP1251
\end{verbatim}

\begin{verbatim}
Warning in grid.Call.graphics(C_text, as.graphicsAnnot(x$label), x$x, x$y, :
неизвестна ширина символа 0xc7 в кодировке CP1251
\end{verbatim}

\begin{verbatim}
Warning in grid.Call.graphics(C_text, as.graphicsAnnot(x$label), x$x, x$y, :
неизвестна ширина символа 0xe5 в кодировке CP1251
\end{verbatim}

\begin{verbatim}
Warning in grid.Call.graphics(C_text, as.graphicsAnnot(x$label), x$x, x$y, :
неизвестна ширина символа 0xeb в кодировке CP1251
\end{verbatim}

\begin{verbatim}
Warning in grid.Call.graphics(C_text, as.graphicsAnnot(x$label), x$x, x$y, :
неизвестна ширина символа 0xe5 в кодировке CP1251
\end{verbatim}

\begin{verbatim}
Warning in grid.Call.graphics(C_text, as.graphicsAnnot(x$label), x$x, x$y, :
неизвестна ширина символа 0xed в кодировке CP1251
\end{verbatim}

\begin{verbatim}
Warning in grid.Call.graphics(C_text, as.graphicsAnnot(x$label), x$x, x$y, :
неизвестна ширина символа 0xe0 в кодировке CP1251
\end{verbatim}

\begin{verbatim}
Warning in grid.Call.graphics(C_text, as.graphicsAnnot(x$label), x$x, x$y, :
неизвестна ширина символа 0xff в кодировке CP1251
\end{verbatim}

\begin{verbatim}
Warning in grid.Call.graphics(C_text, as.graphicsAnnot(x$label), x$x, x$y, :
неизвестна ширина символа 0xe7 в кодировке CP1251
\end{verbatim}

\begin{verbatim}
Warning in grid.Call.graphics(C_text, as.graphicsAnnot(x$label), x$x, x$y, :
неизвестна ширина символа 0xee в кодировке CP1251
\end{verbatim}

\begin{verbatim}
Warning in grid.Call.graphics(C_text, as.graphicsAnnot(x$label), x$x, x$y, :
неизвестна ширина символа 0xed в кодировке CP1251
\end{verbatim}

\begin{verbatim}
Warning in grid.Call.graphics(C_text, as.graphicsAnnot(x$label), x$x, x$y, :
неизвестна ширина символа 0xe0 в кодировке CP1251
\end{verbatim}

\begin{verbatim}
Warning in grid.Call.graphics(C_text, as.graphicsAnnot(x$label), x$x, x$y, :
неизвестна ширина символа 0xc8 в кодировке CP1251
\end{verbatim}

\begin{verbatim}
Warning in grid.Call.graphics(C_text, as.graphicsAnnot(x$label), x$x, x$y, :
неизвестна ширина символа 0xf1 в кодировке CP1251
\end{verbatim}

\begin{verbatim}
Warning in grid.Call.graphics(C_text, as.graphicsAnnot(x$label), x$x, x$y, :
неизвестна ширина символа 0xf2 в кодировке CP1251
\end{verbatim}

\begin{verbatim}
Warning in grid.Call.graphics(C_text, as.graphicsAnnot(x$label), x$x, x$y, :
неизвестна ширина символа 0xee в кодировке CP1251
\end{verbatim}

\begin{verbatim}
Warning in grid.Call.graphics(C_text, as.graphicsAnnot(x$label), x$x, x$y, :
неизвестна ширина символа 0xf9 в кодировке CP1251
\end{verbatim}

\begin{verbatim}
Warning in grid.Call.graphics(C_text, as.graphicsAnnot(x$label), x$x, x$y, :
неизвестна ширина символа 0xe5 в кодировке CP1251
\end{verbatim}

\begin{verbatim}
Warning in grid.Call.graphics(C_text, as.graphicsAnnot(x$label), x$x, x$y, :
неизвестна ширина символа 0xed в кодировке CP1251
\end{verbatim}

\begin{verbatim}
Warning in grid.Call.graphics(C_text, as.graphicsAnnot(x$label), x$x, x$y, :
неизвестна ширина символа 0xe8 в кодировке CP1251
\end{verbatim}

\begin{verbatim}
Warning in grid.Call.graphics(C_text, as.graphicsAnnot(x$label), x$x, x$y, :
неизвестна ширина символа 0xe5 в кодировке CP1251
\end{verbatim}

\begin{verbatim}
Warning in grid.Call.graphics(C_text, as.graphicsAnnot(x$label), x$x, x$y, :
неизвестна ширина символа 0xcf в кодировке CP1251
\end{verbatim}

\begin{verbatim}
Warning in grid.Call.graphics(C_text, as.graphicsAnnot(x$label), x$x, x$y, :
неизвестна ширина символа 0xe5 в кодировке CP1251
\end{verbatim}

\begin{verbatim}
Warning in grid.Call.graphics(C_text, as.graphicsAnnot(x$label), x$x, x$y, :
неизвестна ширина символа 0xf0 в кодировке CP1251
\end{verbatim}

\begin{verbatim}
Warning in grid.Call.graphics(C_text, as.graphicsAnnot(x$label), x$x, x$y, :
неизвестна ширина символа 0xe5 в кодировке CP1251
\end{verbatim}

\begin{verbatim}
Warning in grid.Call.graphics(C_text, as.graphicsAnnot(x$label), x$x, x$y, :
неизвестна ширина символа 0xeb в кодировке CP1251
\end{verbatim}

\begin{verbatim}
Warning in grid.Call.graphics(C_text, as.graphicsAnnot(x$label), x$x, x$y, :
неизвестна ширина символа 0xee в кодировке CP1251
\end{verbatim}

\begin{verbatim}
Warning in grid.Call.graphics(C_text, as.graphicsAnnot(x$label), x$x, x$y, :
неизвестна ширина символа 0xe2 в кодировке CP1251
\end{verbatim}

\begin{verbatim}
Warning in grid.Call.graphics(C_text, as.graphicsAnnot(x$label), x$x, x$y, :
неизвестна ширина символа 0xc2 в кодировке CP1251
\end{verbatim}

\begin{verbatim}
Warning in grid.Call.graphics(C_text, as.graphicsAnnot(x$label), x$x, x$y, :
неизвестна ширина символа 0xe5 в кодировке CP1251
\end{verbatim}

\begin{verbatim}
Warning in grid.Call.graphics(C_text, as.graphicsAnnot(x$label), x$x, x$y, :
неизвестна ширина символа 0xf0 в кодировке CP1251
\end{verbatim}

\begin{verbatim}
Warning in grid.Call.graphics(C_text, as.graphicsAnnot(x$label), x$x, x$y, :
неизвестна ширина символа 0xee в кодировке CP1251
\end{verbatim}

\begin{verbatim}
Warning in grid.Call.graphics(C_text, as.graphicsAnnot(x$label), x$x, x$y, :
неизвестна ширина символа 0xff в кодировке CP1251
\end{verbatim}

\begin{verbatim}
Warning in grid.Call.graphics(C_text, as.graphicsAnnot(x$label), x$x, x$y, :
неизвестна ширина символа 0xf2 в кодировке CP1251
\end{verbatim}

\begin{verbatim}
Warning in grid.Call.graphics(C_text, as.graphicsAnnot(x$label), x$x, x$y, :
неизвестна ширина символа 0xed в кодировке CP1251
\end{verbatim}

\begin{verbatim}
Warning in grid.Call.graphics(C_text, as.graphicsAnnot(x$label), x$x, x$y, :
неизвестна ширина символа 0xee в кодировке CP1251
\end{verbatim}

\begin{verbatim}
Warning in grid.Call.graphics(C_text, as.graphicsAnnot(x$label), x$x, x$y, :
неизвестна ширина символа 0xf1 в кодировке CP1251
\end{verbatim}

\begin{verbatim}
Warning in grid.Call.graphics(C_text, as.graphicsAnnot(x$label), x$x, x$y, :
неизвестна ширина символа 0xf2 в кодировке CP1251
\end{verbatim}

\begin{verbatim}
Warning in grid.Call.graphics(C_text, as.graphicsAnnot(x$label), x$x, x$y, :
неизвестна ширина символа 0xfc в кодировке CP1251
\end{verbatim}

\begin{verbatim}
Warning in grid.Call.graphics(C_text, as.graphicsAnnot(x$label), x$x, x$y, :
неизвестна ширина символа 0xd1 в кодировке CP1251
\end{verbatim}

\begin{verbatim}
Warning in grid.Call.graphics(C_text, as.graphicsAnnot(x$label), x$x, x$y, :
неизвестна ширина символа 0xf2 в кодировке CP1251
\end{verbatim}

\begin{verbatim}
Warning in grid.Call.graphics(C_text, as.graphicsAnnot(x$label), x$x, x$y, :
неизвестна ширина символа 0xf0 в кодировке CP1251
\end{verbatim}

\begin{verbatim}
Warning in grid.Call.graphics(C_text, as.graphicsAnnot(x$label), x$x, x$y, :
неизвестна ширина символа 0xe0 в кодировке CP1251
\end{verbatim}

\begin{verbatim}
Warning in grid.Call.graphics(C_text, as.graphicsAnnot(x$label), x$x, x$y, :
неизвестна ширина символа 0xf2 в кодировке CP1251
\end{verbatim}

\begin{verbatim}
Warning in grid.Call.graphics(C_text, as.graphicsAnnot(x$label), x$x, x$y, :
неизвестна ширина символа 0xe5 в кодировке CP1251
\end{verbatim}

\begin{verbatim}
Warning in grid.Call.graphics(C_text, as.graphicsAnnot(x$label), x$x, x$y, :
неизвестна ширина символа 0xe3 в кодировке CP1251
\end{verbatim}

\begin{verbatim}
Warning in grid.Call.graphics(C_text, as.graphicsAnnot(x$label), x$x, x$y, :
неизвестна ширина символа 0xe8 в кодировке CP1251
\end{verbatim}

\begin{verbatim}
Warning in grid.Call.graphics(C_text, as.graphicsAnnot(x$label), x$x, x$y, :
неизвестна ширина символа 0xff в кодировке CP1251
\end{verbatim}

\begin{verbatim}
Warning in grid.Call.graphics(C_text, as.graphicsAnnot(x$label), x$x, x$y, :
неизвестна ширина символа 0xcf в кодировке CP1251
\end{verbatim}

\begin{verbatim}
Warning in grid.Call.graphics(C_text, as.graphicsAnnot(x$label), x$x, x$y, :
неизвестна ширина символа 0xee в кодировке CP1251
\end{verbatim}

\begin{verbatim}
Warning in grid.Call.graphics(C_text, as.graphicsAnnot(x$label), x$x, x$y, :
неизвестна ширина символа 0xe2 в кодировке CP1251
\end{verbatim}

\begin{verbatim}
Warning in grid.Call.graphics(C_text, as.graphicsAnnot(x$label), x$x, x$y, :
неизвестна ширина символа 0xf1 в кодировке CP1251
\end{verbatim}

\begin{verbatim}
Warning in grid.Call.graphics(C_text, as.graphicsAnnot(x$label), x$x, x$y, :
неизвестна ширина символа 0xe5 в кодировке CP1251
\end{verbatim}

\begin{verbatim}
Warning in grid.Call.graphics(C_text, as.graphicsAnnot(x$label), x$x, x$y, :
неизвестна ширина символа 0xec в кодировке CP1251
\end{verbatim}

\begin{verbatim}
Warning in grid.Call.graphics(C_text, as.graphicsAnnot(x$label), x$x, x$y, :
неизвестна ширина символа 0xf3 в кодировке CP1251
\end{verbatim}

\begin{verbatim}
Warning in grid.Call.graphics(C_text, as.graphicsAnnot(x$label), x$x, x$y, :
неизвестна ширина символа 0xef в кодировке CP1251
\end{verbatim}

\begin{verbatim}
Warning in grid.Call.graphics(C_text, as.graphicsAnnot(x$label), x$x, x$y, :
неизвестна ширина символа 0xe5 в кодировке CP1251
\end{verbatim}

\begin{verbatim}
Warning in grid.Call.graphics(C_text, as.graphicsAnnot(x$label), x$x, x$y, :
неизвестна ширина символа 0xf0 в кодировке CP1251
\end{verbatim}

\begin{verbatim}
Warning in grid.Call.graphics(C_text, as.graphicsAnnot(x$label), x$x, x$y, :
неизвестна ширина символа 0xe8 в кодировке CP1251
\end{verbatim}

\begin{verbatim}
Warning in grid.Call.graphics(C_text, as.graphicsAnnot(x$label), x$x, x$y, :
неизвестна ширина символа 0xee в кодировке CP1251
\end{verbatim}

\begin{verbatim}
Warning in grid.Call.graphics(C_text, as.graphicsAnnot(x$label), x$x, x$y, :
неизвестна ширина символа 0xe4 в кодировке CP1251
\end{verbatim}

\begin{verbatim}
Warning in grid.Call.graphics(C_text, as.graphicsAnnot(x$label), x$x, x$y, :
неизвестна ширина символа 0xf3 в кодировке CP1251
\end{verbatim}

\begin{verbatim}
Warning in grid.Call.graphics(C_text, as.graphicsAnnot(x$label), x$x, x$y, :
неизвестна ширина символа 0xf1 в кодировке CP1251
\end{verbatim}

\begin{verbatim}
Warning in grid.Call.graphics(C_text, as.graphicsAnnot(x$label), x$x, x$y, :
неизвестна ширина символа 0xe8 в кодировке CP1251
\end{verbatim}

\begin{verbatim}
Warning in grid.Call.graphics(C_text, as.graphicsAnnot(x$label), x$x, x$y, :
неизвестна ширина символа 0xec в кодировке CP1251
\end{verbatim}

\begin{verbatim}
Warning in grid.Call.graphics(C_text, as.graphicsAnnot(x$label), x$x, x$y, :
неизвестна ширина символа 0xf3 в кодировке CP1251
\end{verbatim}

\begin{verbatim}
Warning in grid.Call.graphics(C_text, as.graphicsAnnot(x$label), x$x, x$y, :
неизвестна ширина символа 0xeb в кодировке CP1251
\end{verbatim}

\begin{verbatim}
Warning in grid.Call.graphics(C_text, as.graphicsAnnot(x$label), x$x, x$y, :
неизвестна ширина символа 0xff в кодировке CP1251
\end{verbatim}

\begin{verbatim}
Warning in grid.Call.graphics(C_text, as.graphicsAnnot(x$label), x$x, x$y, :
неизвестна ширина символа 0xf6 в кодировке CP1251
\end{verbatim}

\begin{verbatim}
Warning in grid.Call.graphics(C_text, as.graphicsAnnot(x$label), x$x, x$y, :
неизвестна ширина символа 0xe8 в кодировке CP1251
Warning in grid.Call.graphics(C_text, as.graphicsAnnot(x$label), x$x, x$y, :
неизвестна ширина символа 0xe8 в кодировке CP1251
\end{verbatim}

\begin{verbatim}
Warning in grid.Call.graphics(C_text, as.graphicsAnnot(x$label), x$x, x$y, :
неизвестна ширина символа 0xd1 в кодировке CP1251
\end{verbatim}

\begin{verbatim}
Warning in grid.Call.graphics(C_text, as.graphicsAnnot(x$label), x$x, x$y, :
неизвестна ширина символа 0xf0 в кодировке CP1251
\end{verbatim}

\begin{verbatim}
Warning in grid.Call.graphics(C_text, as.graphicsAnnot(x$label), x$x, x$y, :
неизвестна ширина символа 0xe0 в кодировке CP1251
\end{verbatim}

\begin{verbatim}
Warning in grid.Call.graphics(C_text, as.graphicsAnnot(x$label), x$x, x$y, :
неизвестна ширина символа 0xe2 в кодировке CP1251
\end{verbatim}

\begin{verbatim}
Warning in grid.Call.graphics(C_text, as.graphicsAnnot(x$label), x$x, x$y, :
неизвестна ширина символа 0xed в кодировке CP1251
\end{verbatim}

\begin{verbatim}
Warning in grid.Call.graphics(C_text, as.graphicsAnnot(x$label), x$x, x$y, :
неизвестна ширина символа 0xe5 в кодировке CP1251
\end{verbatim}

\begin{verbatim}
Warning in grid.Call.graphics(C_text, as.graphicsAnnot(x$label), x$x, x$y, :
неизвестна ширина символа 0xed в кодировке CP1251
\end{verbatim}

\begin{verbatim}
Warning in grid.Call.graphics(C_text, as.graphicsAnnot(x$label), x$x, x$y, :
неизвестна ширина символа 0xe8 в кодировке CP1251
\end{verbatim}

\begin{verbatim}
Warning in grid.Call.graphics(C_text, as.graphicsAnnot(x$label), x$x, x$y, :
неизвестна ширина символа 0xe5 в кодировке CP1251
\end{verbatim}

\begin{verbatim}
Warning in grid.Call.graphics(C_text, as.graphicsAnnot(x$label), x$x, x$y, :
неизвестна ширина символа 0xe2 в кодировке CP1251
\end{verbatim}

\begin{verbatim}
Warning in grid.Call.graphics(C_text, as.graphicsAnnot(x$label), x$x, x$y, :
неизвестна ширина символа 0xe5 в кодировке CP1251
\end{verbatim}

\begin{verbatim}
Warning in grid.Call.graphics(C_text, as.graphicsAnnot(x$label), x$x, x$y, :
неизвестна ширина символа 0xf0 в кодировке CP1251
\end{verbatim}

\begin{verbatim}
Warning in grid.Call.graphics(C_text, as.graphicsAnnot(x$label), x$x, x$y, :
неизвестна ширина символа 0xee в кодировке CP1251
\end{verbatim}

\begin{verbatim}
Warning in grid.Call.graphics(C_text, as.graphicsAnnot(x$label), x$x, x$y, :
неизвестна ширина символа 0xff в кодировке CP1251
\end{verbatim}

\begin{verbatim}
Warning in grid.Call.graphics(C_text, as.graphicsAnnot(x$label), x$x, x$y, :
неизвестна ширина символа 0xf2 в кодировке CP1251
\end{verbatim}

\begin{verbatim}
Warning in grid.Call.graphics(C_text, as.graphicsAnnot(x$label), x$x, x$y, :
неизвестна ширина символа 0xed в кодировке CP1251
\end{verbatim}

\begin{verbatim}
Warning in grid.Call.graphics(C_text, as.graphicsAnnot(x$label), x$x, x$y, :
неизвестна ширина символа 0xee в кодировке CP1251
\end{verbatim}

\begin{verbatim}
Warning in grid.Call.graphics(C_text, as.graphicsAnnot(x$label), x$x, x$y, :
неизвестна ширина символа 0xf1 в кодировке CP1251
\end{verbatim}

\begin{verbatim}
Warning in grid.Call.graphics(C_text, as.graphicsAnnot(x$label), x$x, x$y, :
неизвестна ширина символа 0xf2 в кодировке CP1251
\end{verbatim}

\begin{verbatim}
Warning in grid.Call.graphics(C_text, as.graphicsAnnot(x$label), x$x, x$y, :
неизвестна ширина символа 0xe5 в кодировке CP1251
\end{verbatim}

\begin{verbatim}
Warning in grid.Call.graphics(C_text, as.graphicsAnnot(x$label), x$x, x$y, :
неизвестна ширина символа 0xe9 в кодировке CP1251
\end{verbatim}

\begin{verbatim}
Warning in grid.Call.graphics(C_text, as.graphicsAnnot(x$label), x$x, x$y, :
неизвестна ширина символа 0xf0 в кодировке CP1251
\end{verbatim}

\begin{verbatim}
Warning in grid.Call.graphics(C_text, as.graphicsAnnot(x$label), x$x, x$y, :
неизвестна ширина символа 0xe8 в кодировке CP1251
\end{verbatim}

\begin{verbatim}
Warning in grid.Call.graphics(C_text, as.graphicsAnnot(x$label), x$x, x$y, :
неизвестна ширина символа 0xf1 в кодировке CP1251
\end{verbatim}

\begin{verbatim}
Warning in grid.Call.graphics(C_text, as.graphicsAnnot(x$label), x$x, x$y, :
неизвестна ширина символа 0xea в кодировке CP1251
\end{verbatim}

\begin{verbatim}
Warning in grid.Call.graphics(C_text, as.graphicsAnnot(x$label), x$x, x$y, :
неизвестна ширина символа 0xe0 в кодировке CP1251
\end{verbatim}

\pandocbounded{\includegraphics[keepaspectratio]{chapter15_files/figure-pdf/unnamed-chunk-1-5.pdf}}

\begin{Shaded}
\begin{Highlighting}[]
\DocumentationTok{\#\# 9.8 График 6: Компромиссы (trade{-}offs)}
\NormalTok{p6\_tradeoff }\OtherTok{\textless{}{-}} \FunctionTok{ggplot}\NormalTok{(performance\_metrics, }
                      \FunctionTok{aes}\NormalTok{(}\AttributeTok{x =}\NormalTok{ mean\_catch, }\AttributeTok{y =} \DecValTok{1} \SpecialCharTok{{-}}\NormalTok{ prob\_overfished)) }\SpecialCharTok{+}
  \FunctionTok{geom\_point}\NormalTok{(}\FunctionTok{aes}\NormalTok{(}\AttributeTok{color =}\NormalTok{ strategy, }\AttributeTok{size =}\NormalTok{ catch\_stability), }\AttributeTok{alpha =} \FloatTok{0.7}\NormalTok{) }\SpecialCharTok{+}
  \FunctionTok{geom\_text}\NormalTok{(}\FunctionTok{aes}\NormalTok{(}\AttributeTok{label =}\NormalTok{ strategy), }\AttributeTok{vjust =} \SpecialCharTok{{-}}\FloatTok{1.5}\NormalTok{, }\AttributeTok{size =} \FloatTok{3.5}\NormalTok{) }\SpecialCharTok{+}
  \FunctionTok{scale\_color\_manual}\NormalTok{(}\AttributeTok{values =} \FunctionTok{c}\NormalTok{(}\StringTok{"Fish at Fmsy"} \OtherTok{=} \StringTok{"\#E41A1C"}\NormalTok{, }
                               \StringTok{"MSY Hockey{-}stick"} \OtherTok{=} \StringTok{"\#377EB8"}\NormalTok{,}
                               \StringTok{"ICES Advice Rule"} \OtherTok{=} \StringTok{"\#4DAF4A"}\NormalTok{)) }\SpecialCharTok{+}
  \FunctionTok{scale\_size\_continuous}\NormalTok{(}\AttributeTok{range =} \FunctionTok{c}\NormalTok{(}\DecValTok{8}\NormalTok{, }\DecValTok{15}\NormalTok{), }\AttributeTok{name =} \StringTok{"Стабильность}\SpecialCharTok{\textbackslash{}n}\StringTok{вылова"}\NormalTok{) }\SpecialCharTok{+}
  \FunctionTok{labs}\NormalTok{(}\AttributeTok{title =} \StringTok{"Анализ компромиссов между целями управления"}\NormalTok{,}
       \AttributeTok{subtitle =} \StringTok{"Размер точки = стабильность вылова"}\NormalTok{,}
       \AttributeTok{x =} \StringTok{"Средний вылов (тыс. т)"}\NormalTok{, }
       \AttributeTok{y =} \StringTok{"Вероятность избежать истощения"}\NormalTok{,}
       \AttributeTok{color =} \StringTok{"Стратегия"}\NormalTok{) }\SpecialCharTok{+}
\NormalTok{  theme\_mse }\SpecialCharTok{+}
  \FunctionTok{theme}\NormalTok{(}\AttributeTok{legend.position =} \StringTok{"right"}\NormalTok{)}

\NormalTok{p6\_tradeoff}
\end{Highlighting}
\end{Shaded}

\begin{verbatim}
Warning in grid.Call(C_textBounds, as.graphicsAnnot(x$label), x$x, x$y, :
неизвестна ширина символа 0xd1 в кодировке CP1251
\end{verbatim}

\begin{verbatim}
Warning in grid.Call(C_textBounds, as.graphicsAnnot(x$label), x$x, x$y, :
неизвестна ширина символа 0xf2 в кодировке CP1251
\end{verbatim}

\begin{verbatim}
Warning in grid.Call(C_textBounds, as.graphicsAnnot(x$label), x$x, x$y, :
неизвестна ширина символа 0xf0 в кодировке CP1251
\end{verbatim}

\begin{verbatim}
Warning in grid.Call(C_textBounds, as.graphicsAnnot(x$label), x$x, x$y, :
неизвестна ширина символа 0xe0 в кодировке CP1251
\end{verbatim}

\begin{verbatim}
Warning in grid.Call(C_textBounds, as.graphicsAnnot(x$label), x$x, x$y, :
неизвестна ширина символа 0xf2 в кодировке CP1251
\end{verbatim}

\begin{verbatim}
Warning in grid.Call(C_textBounds, as.graphicsAnnot(x$label), x$x, x$y, :
неизвестна ширина символа 0xe5 в кодировке CP1251
\end{verbatim}

\begin{verbatim}
Warning in grid.Call(C_textBounds, as.graphicsAnnot(x$label), x$x, x$y, :
неизвестна ширина символа 0xe3 в кодировке CP1251
\end{verbatim}

\begin{verbatim}
Warning in grid.Call(C_textBounds, as.graphicsAnnot(x$label), x$x, x$y, :
неизвестна ширина символа 0xe8 в кодировке CP1251
\end{verbatim}

\begin{verbatim}
Warning in grid.Call(C_textBounds, as.graphicsAnnot(x$label), x$x, x$y, :
неизвестна ширина символа 0xff в кодировке CP1251
\end{verbatim}

\begin{verbatim}
Warning in grid.Call(C_textBounds, as.graphicsAnnot(x$label), x$x, x$y, :
неизвестна ширина символа 0xd1 в кодировке CP1251
\end{verbatim}

\begin{verbatim}
Warning in grid.Call(C_textBounds, as.graphicsAnnot(x$label), x$x, x$y, :
неизвестна ширина символа 0xf2 в кодировке CP1251
\end{verbatim}

\begin{verbatim}
Warning in grid.Call(C_textBounds, as.graphicsAnnot(x$label), x$x, x$y, :
неизвестна ширина символа 0xf0 в кодировке CP1251
\end{verbatim}

\begin{verbatim}
Warning in grid.Call(C_textBounds, as.graphicsAnnot(x$label), x$x, x$y, :
неизвестна ширина символа 0xe0 в кодировке CP1251
\end{verbatim}

\begin{verbatim}
Warning in grid.Call(C_textBounds, as.graphicsAnnot(x$label), x$x, x$y, :
неизвестна ширина символа 0xf2 в кодировке CP1251
\end{verbatim}

\begin{verbatim}
Warning in grid.Call(C_textBounds, as.graphicsAnnot(x$label), x$x, x$y, :
неизвестна ширина символа 0xe5 в кодировке CP1251
\end{verbatim}

\begin{verbatim}
Warning in grid.Call(C_textBounds, as.graphicsAnnot(x$label), x$x, x$y, :
неизвестна ширина символа 0xe3 в кодировке CP1251
\end{verbatim}

\begin{verbatim}
Warning in grid.Call(C_textBounds, as.graphicsAnnot(x$label), x$x, x$y, :
неизвестна ширина символа 0xe8 в кодировке CP1251
\end{verbatim}

\begin{verbatim}
Warning in grid.Call(C_textBounds, as.graphicsAnnot(x$label), x$x, x$y, :
неизвестна ширина символа 0xff в кодировке CP1251
\end{verbatim}

\begin{verbatim}
Warning in grid.Call(C_textBounds, as.graphicsAnnot(x$label), x$x, x$y, :
неизвестна ширина символа 0xd1 в кодировке CP1251
\end{verbatim}

\begin{verbatim}
Warning in grid.Call(C_textBounds, as.graphicsAnnot(x$label), x$x, x$y, :
неизвестна ширина символа 0xf2 в кодировке CP1251
\end{verbatim}

\begin{verbatim}
Warning in grid.Call(C_textBounds, as.graphicsAnnot(x$label), x$x, x$y, :
неизвестна ширина символа 0xe0 в кодировке CP1251
\end{verbatim}

\begin{verbatim}
Warning in grid.Call(C_textBounds, as.graphicsAnnot(x$label), x$x, x$y, :
неизвестна ширина символа 0xe1 в кодировке CP1251
\end{verbatim}

\begin{verbatim}
Warning in grid.Call(C_textBounds, as.graphicsAnnot(x$label), x$x, x$y, :
неизвестна ширина символа 0xe8 в кодировке CP1251
\end{verbatim}

\begin{verbatim}
Warning in grid.Call(C_textBounds, as.graphicsAnnot(x$label), x$x, x$y, :
неизвестна ширина символа 0xeb в кодировке CP1251
\end{verbatim}

\begin{verbatim}
Warning in grid.Call(C_textBounds, as.graphicsAnnot(x$label), x$x, x$y, :
неизвестна ширина символа 0xfc в кодировке CP1251
\end{verbatim}

\begin{verbatim}
Warning in grid.Call(C_textBounds, as.graphicsAnnot(x$label), x$x, x$y, :
неизвестна ширина символа 0xed в кодировке CP1251
\end{verbatim}

\begin{verbatim}
Warning in grid.Call(C_textBounds, as.graphicsAnnot(x$label), x$x, x$y, :
неизвестна ширина символа 0xee в кодировке CP1251
\end{verbatim}

\begin{verbatim}
Warning in grid.Call(C_textBounds, as.graphicsAnnot(x$label), x$x, x$y, :
неизвестна ширина символа 0xf1 в кодировке CP1251
\end{verbatim}

\begin{verbatim}
Warning in grid.Call(C_textBounds, as.graphicsAnnot(x$label), x$x, x$y, :
неизвестна ширина символа 0xf2 в кодировке CP1251
\end{verbatim}

\begin{verbatim}
Warning in grid.Call(C_textBounds, as.graphicsAnnot(x$label), x$x, x$y, :
неизвестна ширина символа 0xfc в кодировке CP1251
\end{verbatim}

\begin{verbatim}
Warning in grid.Call(C_textBounds, as.graphicsAnnot(x$label), x$x, x$y, :
неизвестна ширина символа 0xe2 в кодировке CP1251
\end{verbatim}

\begin{verbatim}
Warning in grid.Call(C_textBounds, as.graphicsAnnot(x$label), x$x, x$y, :
неизвестна ширина символа 0xfb в кодировке CP1251
\end{verbatim}

\begin{verbatim}
Warning in grid.Call(C_textBounds, as.graphicsAnnot(x$label), x$x, x$y, :
неизвестна ширина символа 0xeb в кодировке CP1251
\end{verbatim}

\begin{verbatim}
Warning in grid.Call(C_textBounds, as.graphicsAnnot(x$label), x$x, x$y, :
неизвестна ширина символа 0xee в кодировке CP1251
\end{verbatim}

\begin{verbatim}
Warning in grid.Call(C_textBounds, as.graphicsAnnot(x$label), x$x, x$y, :
неизвестна ширина символа 0xe2 в кодировке CP1251
\end{verbatim}

\begin{verbatim}
Warning in grid.Call(C_textBounds, as.graphicsAnnot(x$label), x$x, x$y, :
неизвестна ширина символа 0xe0 в кодировке CP1251
\end{verbatim}

\begin{verbatim}
Warning in grid.Call(C_textBounds, as.graphicsAnnot(x$label), x$x, x$y, :
неизвестна ширина символа 0xd1 в кодировке CP1251
\end{verbatim}

\begin{verbatim}
Warning in grid.Call(C_textBounds, as.graphicsAnnot(x$label), x$x, x$y, :
неизвестна ширина символа 0xf2 в кодировке CP1251
\end{verbatim}

\begin{verbatim}
Warning in grid.Call(C_textBounds, as.graphicsAnnot(x$label), x$x, x$y, :
неизвестна ширина символа 0xe0 в кодировке CP1251
\end{verbatim}

\begin{verbatim}
Warning in grid.Call(C_textBounds, as.graphicsAnnot(x$label), x$x, x$y, :
неизвестна ширина символа 0xe1 в кодировке CP1251
\end{verbatim}

\begin{verbatim}
Warning in grid.Call(C_textBounds, as.graphicsAnnot(x$label), x$x, x$y, :
неизвестна ширина символа 0xe8 в кодировке CP1251
\end{verbatim}

\begin{verbatim}
Warning in grid.Call(C_textBounds, as.graphicsAnnot(x$label), x$x, x$y, :
неизвестна ширина символа 0xeb в кодировке CP1251
\end{verbatim}

\begin{verbatim}
Warning in grid.Call(C_textBounds, as.graphicsAnnot(x$label), x$x, x$y, :
неизвестна ширина символа 0xfc в кодировке CP1251
\end{verbatim}

\begin{verbatim}
Warning in grid.Call(C_textBounds, as.graphicsAnnot(x$label), x$x, x$y, :
неизвестна ширина символа 0xed в кодировке CP1251
\end{verbatim}

\begin{verbatim}
Warning in grid.Call(C_textBounds, as.graphicsAnnot(x$label), x$x, x$y, :
неизвестна ширина символа 0xee в кодировке CP1251
\end{verbatim}

\begin{verbatim}
Warning in grid.Call(C_textBounds, as.graphicsAnnot(x$label), x$x, x$y, :
неизвестна ширина символа 0xf1 в кодировке CP1251
\end{verbatim}

\begin{verbatim}
Warning in grid.Call(C_textBounds, as.graphicsAnnot(x$label), x$x, x$y, :
неизвестна ширина символа 0xf2 в кодировке CP1251
\end{verbatim}

\begin{verbatim}
Warning in grid.Call(C_textBounds, as.graphicsAnnot(x$label), x$x, x$y, :
неизвестна ширина символа 0xfc в кодировке CP1251
\end{verbatim}

\begin{verbatim}
Warning in grid.Call(C_textBounds, as.graphicsAnnot(x$label), x$x, x$y, :
неизвестна ширина символа 0xe2 в кодировке CP1251
\end{verbatim}

\begin{verbatim}
Warning in grid.Call(C_textBounds, as.graphicsAnnot(x$label), x$x, x$y, :
неизвестна ширина символа 0xfb в кодировке CP1251
\end{verbatim}

\begin{verbatim}
Warning in grid.Call(C_textBounds, as.graphicsAnnot(x$label), x$x, x$y, :
неизвестна ширина символа 0xeb в кодировке CP1251
\end{verbatim}

\begin{verbatim}
Warning in grid.Call(C_textBounds, as.graphicsAnnot(x$label), x$x, x$y, :
неизвестна ширина символа 0xee в кодировке CP1251
\end{verbatim}

\begin{verbatim}
Warning in grid.Call(C_textBounds, as.graphicsAnnot(x$label), x$x, x$y, :
неизвестна ширина символа 0xe2 в кодировке CP1251
\end{verbatim}

\begin{verbatim}
Warning in grid.Call(C_textBounds, as.graphicsAnnot(x$label), x$x, x$y, :
неизвестна ширина символа 0xe0 в кодировке CP1251
\end{verbatim}

\begin{verbatim}
Warning in grid.Call(C_textBounds, as.graphicsAnnot(x$label), x$x, x$y, :
неизвестна ширина символа 0xc2 в кодировке CP1251
\end{verbatim}

\begin{verbatim}
Warning in grid.Call(C_textBounds, as.graphicsAnnot(x$label), x$x, x$y, :
неизвестна ширина символа 0xe5 в кодировке CP1251
\end{verbatim}

\begin{verbatim}
Warning in grid.Call(C_textBounds, as.graphicsAnnot(x$label), x$x, x$y, :
неизвестна ширина символа 0xf0 в кодировке CP1251
\end{verbatim}

\begin{verbatim}
Warning in grid.Call(C_textBounds, as.graphicsAnnot(x$label), x$x, x$y, :
неизвестна ширина символа 0xee в кодировке CP1251
\end{verbatim}

\begin{verbatim}
Warning in grid.Call(C_textBounds, as.graphicsAnnot(x$label), x$x, x$y, :
неизвестна ширина символа 0xff в кодировке CP1251
\end{verbatim}

\begin{verbatim}
Warning in grid.Call(C_textBounds, as.graphicsAnnot(x$label), x$x, x$y, :
неизвестна ширина символа 0xf2 в кодировке CP1251
\end{verbatim}

\begin{verbatim}
Warning in grid.Call(C_textBounds, as.graphicsAnnot(x$label), x$x, x$y, :
неизвестна ширина символа 0xed в кодировке CP1251
\end{verbatim}

\begin{verbatim}
Warning in grid.Call(C_textBounds, as.graphicsAnnot(x$label), x$x, x$y, :
неизвестна ширина символа 0xee в кодировке CP1251
\end{verbatim}

\begin{verbatim}
Warning in grid.Call(C_textBounds, as.graphicsAnnot(x$label), x$x, x$y, :
неизвестна ширина символа 0xf1 в кодировке CP1251
\end{verbatim}

\begin{verbatim}
Warning in grid.Call(C_textBounds, as.graphicsAnnot(x$label), x$x, x$y, :
неизвестна ширина символа 0xf2 в кодировке CP1251
\end{verbatim}

\begin{verbatim}
Warning in grid.Call(C_textBounds, as.graphicsAnnot(x$label), x$x, x$y, :
неизвестна ширина символа 0xfc в кодировке CP1251
\end{verbatim}

\begin{verbatim}
Warning in grid.Call(C_textBounds, as.graphicsAnnot(x$label), x$x, x$y, :
неизвестна ширина символа 0xe8 в кодировке CP1251
\end{verbatim}

\begin{verbatim}
Warning in grid.Call(C_textBounds, as.graphicsAnnot(x$label), x$x, x$y, :
неизвестна ширина символа 0xe7 в кодировке CP1251
\end{verbatim}

\begin{verbatim}
Warning in grid.Call(C_textBounds, as.graphicsAnnot(x$label), x$x, x$y, :
неизвестна ширина символа 0xe1 в кодировке CP1251
\end{verbatim}

\begin{verbatim}
Warning in grid.Call(C_textBounds, as.graphicsAnnot(x$label), x$x, x$y, :
неизвестна ширина символа 0xe5 в кодировке CP1251
\end{verbatim}

\begin{verbatim}
Warning in grid.Call(C_textBounds, as.graphicsAnnot(x$label), x$x, x$y, :
неизвестна ширина символа 0xe6 в кодировке CP1251
\end{verbatim}

\begin{verbatim}
Warning in grid.Call(C_textBounds, as.graphicsAnnot(x$label), x$x, x$y, :
неизвестна ширина символа 0xe0 в кодировке CP1251
\end{verbatim}

\begin{verbatim}
Warning in grid.Call(C_textBounds, as.graphicsAnnot(x$label), x$x, x$y, :
неизвестна ширина символа 0xf2 в кодировке CP1251
\end{verbatim}

\begin{verbatim}
Warning in grid.Call(C_textBounds, as.graphicsAnnot(x$label), x$x, x$y, :
неизвестна ширина символа 0xfc в кодировке CP1251
\end{verbatim}

\begin{verbatim}
Warning in grid.Call(C_textBounds, as.graphicsAnnot(x$label), x$x, x$y, :
неизвестна ширина символа 0xe8 в кодировке CP1251
\end{verbatim}

\begin{verbatim}
Warning in grid.Call(C_textBounds, as.graphicsAnnot(x$label), x$x, x$y, :
неизвестна ширина символа 0xf1 в кодировке CP1251
\end{verbatim}

\begin{verbatim}
Warning in grid.Call(C_textBounds, as.graphicsAnnot(x$label), x$x, x$y, :
неизвестна ширина символа 0xf2 в кодировке CP1251
\end{verbatim}

\begin{verbatim}
Warning in grid.Call(C_textBounds, as.graphicsAnnot(x$label), x$x, x$y, :
неизвестна ширина символа 0xee в кодировке CP1251
\end{verbatim}

\begin{verbatim}
Warning in grid.Call(C_textBounds, as.graphicsAnnot(x$label), x$x, x$y, :
неизвестна ширина символа 0xf9 в кодировке CP1251
\end{verbatim}

\begin{verbatim}
Warning in grid.Call(C_textBounds, as.graphicsAnnot(x$label), x$x, x$y, :
неизвестна ширина символа 0xe5 в кодировке CP1251
\end{verbatim}

\begin{verbatim}
Warning in grid.Call(C_textBounds, as.graphicsAnnot(x$label), x$x, x$y, :
неизвестна ширина символа 0xed в кодировке CP1251
\end{verbatim}

\begin{verbatim}
Warning in grid.Call(C_textBounds, as.graphicsAnnot(x$label), x$x, x$y, :
неизвестна ширина символа 0xe8 в кодировке CP1251
\end{verbatim}

\begin{verbatim}
Warning in grid.Call(C_textBounds, as.graphicsAnnot(x$label), x$x, x$y, :
неизвестна ширина символа 0xff в кодировке CP1251
\end{verbatim}

\begin{verbatim}
Warning in grid.Call(C_textBounds, as.graphicsAnnot(x$label), x$x, x$y, :
неизвестна ширина символа 0xc0 в кодировке CP1251
\end{verbatim}

\begin{verbatim}
Warning in grid.Call(C_textBounds, as.graphicsAnnot(x$label), x$x, x$y, :
неизвестна ширина символа 0xed в кодировке CP1251
\end{verbatim}

\begin{verbatim}
Warning in grid.Call(C_textBounds, as.graphicsAnnot(x$label), x$x, x$y, :
неизвестна ширина символа 0xe0 в кодировке CP1251
\end{verbatim}

\begin{verbatim}
Warning in grid.Call(C_textBounds, as.graphicsAnnot(x$label), x$x, x$y, :
неизвестна ширина символа 0xeb в кодировке CP1251
\end{verbatim}

\begin{verbatim}
Warning in grid.Call(C_textBounds, as.graphicsAnnot(x$label), x$x, x$y, :
неизвестна ширина символа 0xe8 в кодировке CP1251
\end{verbatim}

\begin{verbatim}
Warning in grid.Call(C_textBounds, as.graphicsAnnot(x$label), x$x, x$y, :
неизвестна ширина символа 0xe7 в кодировке CP1251
\end{verbatim}

\begin{verbatim}
Warning in grid.Call(C_textBounds, as.graphicsAnnot(x$label), x$x, x$y, :
неизвестна ширина символа 0xea в кодировке CP1251
\end{verbatim}

\begin{verbatim}
Warning in grid.Call(C_textBounds, as.graphicsAnnot(x$label), x$x, x$y, :
неизвестна ширина символа 0xee в кодировке CP1251
\end{verbatim}

\begin{verbatim}
Warning in grid.Call(C_textBounds, as.graphicsAnnot(x$label), x$x, x$y, :
неизвестна ширина символа 0xec в кодировке CP1251
\end{verbatim}

\begin{verbatim}
Warning in grid.Call(C_textBounds, as.graphicsAnnot(x$label), x$x, x$y, :
неизвестна ширина символа 0xef в кодировке CP1251
\end{verbatim}

\begin{verbatim}
Warning in grid.Call(C_textBounds, as.graphicsAnnot(x$label), x$x, x$y, :
неизвестна ширина символа 0xf0 в кодировке CP1251
\end{verbatim}

\begin{verbatim}
Warning in grid.Call(C_textBounds, as.graphicsAnnot(x$label), x$x, x$y, :
неизвестна ширина символа 0xee в кодировке CP1251
\end{verbatim}

\begin{verbatim}
Warning in grid.Call(C_textBounds, as.graphicsAnnot(x$label), x$x, x$y, :
неизвестна ширина символа 0xec в кодировке CP1251
\end{verbatim}

\begin{verbatim}
Warning in grid.Call(C_textBounds, as.graphicsAnnot(x$label), x$x, x$y, :
неизвестна ширина символа 0xe8 в кодировке CP1251
\end{verbatim}

\begin{verbatim}
Warning in grid.Call(C_textBounds, as.graphicsAnnot(x$label), x$x, x$y, :
неизвестна ширина символа 0xf1 в кодировке CP1251
Warning in grid.Call(C_textBounds, as.graphicsAnnot(x$label), x$x, x$y, :
неизвестна ширина символа 0xf1 в кодировке CP1251
\end{verbatim}

\begin{verbatim}
Warning in grid.Call(C_textBounds, as.graphicsAnnot(x$label), x$x, x$y, :
неизвестна ширина символа 0xee в кодировке CP1251
\end{verbatim}

\begin{verbatim}
Warning in grid.Call(C_textBounds, as.graphicsAnnot(x$label), x$x, x$y, :
неизвестна ширина символа 0xe2 в кодировке CP1251
\end{verbatim}

\begin{verbatim}
Warning in grid.Call(C_textBounds, as.graphicsAnnot(x$label), x$x, x$y, :
неизвестна ширина символа 0xec в кодировке CP1251
\end{verbatim}

\begin{verbatim}
Warning in grid.Call(C_textBounds, as.graphicsAnnot(x$label), x$x, x$y, :
неизвестна ширина символа 0xe5 в кодировке CP1251
\end{verbatim}

\begin{verbatim}
Warning in grid.Call(C_textBounds, as.graphicsAnnot(x$label), x$x, x$y, :
неизвестна ширина символа 0xe6 в кодировке CP1251
\end{verbatim}

\begin{verbatim}
Warning in grid.Call(C_textBounds, as.graphicsAnnot(x$label), x$x, x$y, :
неизвестна ширина символа 0xe4 в кодировке CP1251
\end{verbatim}

\begin{verbatim}
Warning in grid.Call(C_textBounds, as.graphicsAnnot(x$label), x$x, x$y, :
неизвестна ширина символа 0xf3 в кодировке CP1251
\end{verbatim}

\begin{verbatim}
Warning in grid.Call(C_textBounds, as.graphicsAnnot(x$label), x$x, x$y, :
неизвестна ширина символа 0xf6 в кодировке CP1251
\end{verbatim}

\begin{verbatim}
Warning in grid.Call(C_textBounds, as.graphicsAnnot(x$label), x$x, x$y, :
неизвестна ширина символа 0xe5 в кодировке CP1251
\end{verbatim}

\begin{verbatim}
Warning in grid.Call(C_textBounds, as.graphicsAnnot(x$label), x$x, x$y, :
неизвестна ширина символа 0xeb в кодировке CP1251
\end{verbatim}

\begin{verbatim}
Warning in grid.Call(C_textBounds, as.graphicsAnnot(x$label), x$x, x$y, :
неизвестна ширина символа 0xff в кодировке CP1251
\end{verbatim}

\begin{verbatim}
Warning in grid.Call(C_textBounds, as.graphicsAnnot(x$label), x$x, x$y, :
неизвестна ширина символа 0xec в кодировке CP1251
\end{verbatim}

\begin{verbatim}
Warning in grid.Call(C_textBounds, as.graphicsAnnot(x$label), x$x, x$y, :
неизвестна ширина символа 0xe8 в кодировке CP1251
\end{verbatim}

\begin{verbatim}
Warning in grid.Call(C_textBounds, as.graphicsAnnot(x$label), x$x, x$y, :
неизвестна ширина символа 0xf3 в кодировке CP1251
\end{verbatim}

\begin{verbatim}
Warning in grid.Call(C_textBounds, as.graphicsAnnot(x$label), x$x, x$y, :
неизвестна ширина символа 0xef в кодировке CP1251
\end{verbatim}

\begin{verbatim}
Warning in grid.Call(C_textBounds, as.graphicsAnnot(x$label), x$x, x$y, :
неизвестна ширина символа 0xf0 в кодировке CP1251
\end{verbatim}

\begin{verbatim}
Warning in grid.Call(C_textBounds, as.graphicsAnnot(x$label), x$x, x$y, :
неизвестна ширина символа 0xe0 в кодировке CP1251
\end{verbatim}

\begin{verbatim}
Warning in grid.Call(C_textBounds, as.graphicsAnnot(x$label), x$x, x$y, :
неизвестна ширина символа 0xe2 в кодировке CP1251
\end{verbatim}

\begin{verbatim}
Warning in grid.Call(C_textBounds, as.graphicsAnnot(x$label), x$x, x$y, :
неизвестна ширина символа 0xeb в кодировке CP1251
\end{verbatim}

\begin{verbatim}
Warning in grid.Call(C_textBounds, as.graphicsAnnot(x$label), x$x, x$y, :
неизвестна ширина символа 0xe5 в кодировке CP1251
\end{verbatim}

\begin{verbatim}
Warning in grid.Call(C_textBounds, as.graphicsAnnot(x$label), x$x, x$y, :
неизвестна ширина символа 0xed в кодировке CP1251
\end{verbatim}

\begin{verbatim}
Warning in grid.Call(C_textBounds, as.graphicsAnnot(x$label), x$x, x$y, :
неизвестна ширина символа 0xe8 в кодировке CP1251
\end{verbatim}

\begin{verbatim}
Warning in grid.Call(C_textBounds, as.graphicsAnnot(x$label), x$x, x$y, :
неизвестна ширина символа 0xff в кодировке CP1251
\end{verbatim}

\begin{verbatim}
Warning in grid.Call(C_textBounds, as.graphicsAnnot(x$label), x$x, x$y, :
неизвестна ширина символа 0xd0 в кодировке CP1251
\end{verbatim}

\begin{verbatim}
Warning in grid.Call(C_textBounds, as.graphicsAnnot(x$label), x$x, x$y, :
неизвестна ширина символа 0xe0 в кодировке CP1251
\end{verbatim}

\begin{verbatim}
Warning in grid.Call(C_textBounds, as.graphicsAnnot(x$label), x$x, x$y, :
неизвестна ширина символа 0xe7 в кодировке CP1251
\end{verbatim}

\begin{verbatim}
Warning in grid.Call(C_textBounds, as.graphicsAnnot(x$label), x$x, x$y, :
неизвестна ширина символа 0xec в кодировке CP1251
\end{verbatim}

\begin{verbatim}
Warning in grid.Call(C_textBounds, as.graphicsAnnot(x$label), x$x, x$y, :
неизвестна ширина символа 0xe5 в кодировке CP1251
\end{verbatim}

\begin{verbatim}
Warning in grid.Call(C_textBounds, as.graphicsAnnot(x$label), x$x, x$y, :
неизвестна ширина символа 0xf0 в кодировке CP1251
\end{verbatim}

\begin{verbatim}
Warning in grid.Call(C_textBounds, as.graphicsAnnot(x$label), x$x, x$y, :
неизвестна ширина символа 0xf2 в кодировке CP1251
\end{verbatim}

\begin{verbatim}
Warning in grid.Call(C_textBounds, as.graphicsAnnot(x$label), x$x, x$y, :
неизвестна ширина символа 0xee в кодировке CP1251
\end{verbatim}

\begin{verbatim}
Warning in grid.Call(C_textBounds, as.graphicsAnnot(x$label), x$x, x$y, :
неизвестна ширина символа 0xf7 в кодировке CP1251
\end{verbatim}

\begin{verbatim}
Warning in grid.Call(C_textBounds, as.graphicsAnnot(x$label), x$x, x$y, :
неизвестна ширина символа 0xea в кодировке CP1251
\end{verbatim}

\begin{verbatim}
Warning in grid.Call(C_textBounds, as.graphicsAnnot(x$label), x$x, x$y, :
неизвестна ширина символа 0xe8 в кодировке CP1251
\end{verbatim}

\begin{verbatim}
Warning in grid.Call(C_textBounds, as.graphicsAnnot(x$label), x$x, x$y, :
неизвестна ширина символа 0xf1 в кодировке CP1251
\end{verbatim}

\begin{verbatim}
Warning in grid.Call(C_textBounds, as.graphicsAnnot(x$label), x$x, x$y, :
неизвестна ширина символа 0xf2 в кодировке CP1251
\end{verbatim}

\begin{verbatim}
Warning in grid.Call(C_textBounds, as.graphicsAnnot(x$label), x$x, x$y, :
неизвестна ширина символа 0xe0 в кодировке CP1251
\end{verbatim}

\begin{verbatim}
Warning in grid.Call(C_textBounds, as.graphicsAnnot(x$label), x$x, x$y, :
неизвестна ширина символа 0xe1 в кодировке CP1251
\end{verbatim}

\begin{verbatim}
Warning in grid.Call(C_textBounds, as.graphicsAnnot(x$label), x$x, x$y, :
неизвестна ширина символа 0xe8 в кодировке CP1251
\end{verbatim}

\begin{verbatim}
Warning in grid.Call(C_textBounds, as.graphicsAnnot(x$label), x$x, x$y, :
неизвестна ширина символа 0xeb в кодировке CP1251
\end{verbatim}

\begin{verbatim}
Warning in grid.Call(C_textBounds, as.graphicsAnnot(x$label), x$x, x$y, :
неизвестна ширина символа 0xfc в кодировке CP1251
\end{verbatim}

\begin{verbatim}
Warning in grid.Call(C_textBounds, as.graphicsAnnot(x$label), x$x, x$y, :
неизвестна ширина символа 0xed в кодировке CP1251
\end{verbatim}

\begin{verbatim}
Warning in grid.Call(C_textBounds, as.graphicsAnnot(x$label), x$x, x$y, :
неизвестна ширина символа 0xee в кодировке CP1251
\end{verbatim}

\begin{verbatim}
Warning in grid.Call(C_textBounds, as.graphicsAnnot(x$label), x$x, x$y, :
неизвестна ширина символа 0xf1 в кодировке CP1251
\end{verbatim}

\begin{verbatim}
Warning in grid.Call(C_textBounds, as.graphicsAnnot(x$label), x$x, x$y, :
неизвестна ширина символа 0xf2 в кодировке CP1251
\end{verbatim}

\begin{verbatim}
Warning in grid.Call(C_textBounds, as.graphicsAnnot(x$label), x$x, x$y, :
неизвестна ширина символа 0xfc в кодировке CP1251
\end{verbatim}

\begin{verbatim}
Warning in grid.Call(C_textBounds, as.graphicsAnnot(x$label), x$x, x$y, :
неизвестна ширина символа 0xe2 в кодировке CP1251
\end{verbatim}

\begin{verbatim}
Warning in grid.Call(C_textBounds, as.graphicsAnnot(x$label), x$x, x$y, :
неизвестна ширина символа 0xfb в кодировке CP1251
\end{verbatim}

\begin{verbatim}
Warning in grid.Call(C_textBounds, as.graphicsAnnot(x$label), x$x, x$y, :
неизвестна ширина символа 0xeb в кодировке CP1251
\end{verbatim}

\begin{verbatim}
Warning in grid.Call(C_textBounds, as.graphicsAnnot(x$label), x$x, x$y, :
неизвестна ширина символа 0xee в кодировке CP1251
\end{verbatim}

\begin{verbatim}
Warning in grid.Call(C_textBounds, as.graphicsAnnot(x$label), x$x, x$y, :
неизвестна ширина символа 0xe2 в кодировке CP1251
\end{verbatim}

\begin{verbatim}
Warning in grid.Call(C_textBounds, as.graphicsAnnot(x$label), x$x, x$y, :
неизвестна ширина символа 0xe0 в кодировке CP1251
\end{verbatim}

\begin{verbatim}
Warning in grid.Call(C_textBounds, as.graphicsAnnot(x$label), x$x, x$y, :
неизвестна ширина символа 0xd1 в кодировке CP1251
\end{verbatim}

\begin{verbatim}
Warning in grid.Call(C_textBounds, as.graphicsAnnot(x$label), x$x, x$y, :
неизвестна ширина символа 0xf0 в кодировке CP1251
\end{verbatim}

\begin{verbatim}
Warning in grid.Call(C_textBounds, as.graphicsAnnot(x$label), x$x, x$y, :
неизвестна ширина символа 0xe5 в кодировке CP1251
\end{verbatim}

\begin{verbatim}
Warning in grid.Call(C_textBounds, as.graphicsAnnot(x$label), x$x, x$y, :
неизвестна ширина символа 0xe4 в кодировке CP1251
\end{verbatim}

\begin{verbatim}
Warning in grid.Call(C_textBounds, as.graphicsAnnot(x$label), x$x, x$y, :
неизвестна ширина символа 0xed в кодировке CP1251
\end{verbatim}

\begin{verbatim}
Warning in grid.Call(C_textBounds, as.graphicsAnnot(x$label), x$x, x$y, :
неизвестна ширина символа 0xe8 в кодировке CP1251
\end{verbatim}

\begin{verbatim}
Warning in grid.Call(C_textBounds, as.graphicsAnnot(x$label), x$x, x$y, :
неизвестна ширина символа 0xe9 в кодировке CP1251
\end{verbatim}

\begin{verbatim}
Warning in grid.Call(C_textBounds, as.graphicsAnnot(x$label), x$x, x$y, :
неизвестна ширина символа 0xe2 в кодировке CP1251
\end{verbatim}

\begin{verbatim}
Warning in grid.Call(C_textBounds, as.graphicsAnnot(x$label), x$x, x$y, :
неизвестна ширина символа 0xfb в кодировке CP1251
\end{verbatim}

\begin{verbatim}
Warning in grid.Call(C_textBounds, as.graphicsAnnot(x$label), x$x, x$y, :
неизвестна ширина символа 0xeb в кодировке CP1251
\end{verbatim}

\begin{verbatim}
Warning in grid.Call(C_textBounds, as.graphicsAnnot(x$label), x$x, x$y, :
неизвестна ширина символа 0xee в кодировке CP1251
\end{verbatim}

\begin{verbatim}
Warning in grid.Call(C_textBounds, as.graphicsAnnot(x$label), x$x, x$y, :
неизвестна ширина символа 0xe2 в кодировке CP1251
\end{verbatim}

\begin{verbatim}
Warning in grid.Call(C_textBounds, as.graphicsAnnot(x$label), x$x, x$y, :
неизвестна ширина символа 0xf2 в кодировке CP1251
\end{verbatim}

\begin{verbatim}
Warning in grid.Call(C_textBounds, as.graphicsAnnot(x$label), x$x, x$y, :
неизвестна ширина символа 0xfb в кодировке CP1251
\end{verbatim}

\begin{verbatim}
Warning in grid.Call(C_textBounds, as.graphicsAnnot(x$label), x$x, x$y, :
неизвестна ширина символа 0xf1 в кодировке CP1251
\end{verbatim}

\begin{verbatim}
Warning in grid.Call(C_textBounds, as.graphicsAnnot(x$label), x$x, x$y, :
неизвестна ширина символа 0xf2 в кодировке CP1251
\end{verbatim}

\begin{verbatim}
Warning in grid.Call.graphics(C_text, as.graphicsAnnot(x$label), x$x, x$y, :
неизвестна ширина символа 0xd1 в кодировке CP1251
\end{verbatim}

\begin{verbatim}
Warning in grid.Call.graphics(C_text, as.graphicsAnnot(x$label), x$x, x$y, :
неизвестна ширина символа 0xf0 в кодировке CP1251
\end{verbatim}

\begin{verbatim}
Warning in grid.Call.graphics(C_text, as.graphicsAnnot(x$label), x$x, x$y, :
неизвестна ширина символа 0xe5 в кодировке CP1251
\end{verbatim}

\begin{verbatim}
Warning in grid.Call.graphics(C_text, as.graphicsAnnot(x$label), x$x, x$y, :
неизвестна ширина символа 0xe4 в кодировке CP1251
\end{verbatim}

\begin{verbatim}
Warning in grid.Call.graphics(C_text, as.graphicsAnnot(x$label), x$x, x$y, :
неизвестна ширина символа 0xed в кодировке CP1251
\end{verbatim}

\begin{verbatim}
Warning in grid.Call.graphics(C_text, as.graphicsAnnot(x$label), x$x, x$y, :
неизвестна ширина символа 0xe8 в кодировке CP1251
\end{verbatim}

\begin{verbatim}
Warning in grid.Call.graphics(C_text, as.graphicsAnnot(x$label), x$x, x$y, :
неизвестна ширина символа 0xe9 в кодировке CP1251
\end{verbatim}

\begin{verbatim}
Warning in grid.Call.graphics(C_text, as.graphicsAnnot(x$label), x$x, x$y, :
неизвестна ширина символа 0xe2 в кодировке CP1251
\end{verbatim}

\begin{verbatim}
Warning in grid.Call.graphics(C_text, as.graphicsAnnot(x$label), x$x, x$y, :
неизвестна ширина символа 0xfb в кодировке CP1251
\end{verbatim}

\begin{verbatim}
Warning in grid.Call.graphics(C_text, as.graphicsAnnot(x$label), x$x, x$y, :
неизвестна ширина символа 0xeb в кодировке CP1251
\end{verbatim}

\begin{verbatim}
Warning in grid.Call.graphics(C_text, as.graphicsAnnot(x$label), x$x, x$y, :
неизвестна ширина символа 0xee в кодировке CP1251
\end{verbatim}

\begin{verbatim}
Warning in grid.Call.graphics(C_text, as.graphicsAnnot(x$label), x$x, x$y, :
неизвестна ширина символа 0xe2 в кодировке CP1251
\end{verbatim}

\begin{verbatim}
Warning in grid.Call.graphics(C_text, as.graphicsAnnot(x$label), x$x, x$y, :
неизвестна ширина символа 0xf2 в кодировке CP1251
\end{verbatim}

\begin{verbatim}
Warning in grid.Call.graphics(C_text, as.graphicsAnnot(x$label), x$x, x$y, :
неизвестна ширина символа 0xfb в кодировке CP1251
\end{verbatim}

\begin{verbatim}
Warning in grid.Call.graphics(C_text, as.graphicsAnnot(x$label), x$x, x$y, :
неизвестна ширина символа 0xf1 в кодировке CP1251
\end{verbatim}

\begin{verbatim}
Warning in grid.Call.graphics(C_text, as.graphicsAnnot(x$label), x$x, x$y, :
неизвестна ширина символа 0xf2 в кодировке CP1251
\end{verbatim}

\begin{verbatim}
Warning in grid.Call.graphics(C_text, as.graphicsAnnot(x$label), x$x, x$y, :
неизвестна ширина символа 0xc2 в кодировке CP1251
\end{verbatim}

\begin{verbatim}
Warning in grid.Call.graphics(C_text, as.graphicsAnnot(x$label), x$x, x$y, :
неизвестна ширина символа 0xe5 в кодировке CP1251
\end{verbatim}

\begin{verbatim}
Warning in grid.Call.graphics(C_text, as.graphicsAnnot(x$label), x$x, x$y, :
неизвестна ширина символа 0xf0 в кодировке CP1251
\end{verbatim}

\begin{verbatim}
Warning in grid.Call.graphics(C_text, as.graphicsAnnot(x$label), x$x, x$y, :
неизвестна ширина символа 0xee в кодировке CP1251
\end{verbatim}

\begin{verbatim}
Warning in grid.Call.graphics(C_text, as.graphicsAnnot(x$label), x$x, x$y, :
неизвестна ширина символа 0xff в кодировке CP1251
\end{verbatim}

\begin{verbatim}
Warning in grid.Call.graphics(C_text, as.graphicsAnnot(x$label), x$x, x$y, :
неизвестна ширина символа 0xf2 в кодировке CP1251
\end{verbatim}

\begin{verbatim}
Warning in grid.Call.graphics(C_text, as.graphicsAnnot(x$label), x$x, x$y, :
неизвестна ширина символа 0xed в кодировке CP1251
\end{verbatim}

\begin{verbatim}
Warning in grid.Call.graphics(C_text, as.graphicsAnnot(x$label), x$x, x$y, :
неизвестна ширина символа 0xee в кодировке CP1251
\end{verbatim}

\begin{verbatim}
Warning in grid.Call.graphics(C_text, as.graphicsAnnot(x$label), x$x, x$y, :
неизвестна ширина символа 0xf1 в кодировке CP1251
\end{verbatim}

\begin{verbatim}
Warning in grid.Call.graphics(C_text, as.graphicsAnnot(x$label), x$x, x$y, :
неизвестна ширина символа 0xf2 в кодировке CP1251
\end{verbatim}

\begin{verbatim}
Warning in grid.Call.graphics(C_text, as.graphicsAnnot(x$label), x$x, x$y, :
неизвестна ширина символа 0xfc в кодировке CP1251
\end{verbatim}

\begin{verbatim}
Warning in grid.Call.graphics(C_text, as.graphicsAnnot(x$label), x$x, x$y, :
неизвестна ширина символа 0xe8 в кодировке CP1251
\end{verbatim}

\begin{verbatim}
Warning in grid.Call.graphics(C_text, as.graphicsAnnot(x$label), x$x, x$y, :
неизвестна ширина символа 0xe7 в кодировке CP1251
\end{verbatim}

\begin{verbatim}
Warning in grid.Call.graphics(C_text, as.graphicsAnnot(x$label), x$x, x$y, :
неизвестна ширина символа 0xe1 в кодировке CP1251
\end{verbatim}

\begin{verbatim}
Warning in grid.Call.graphics(C_text, as.graphicsAnnot(x$label), x$x, x$y, :
неизвестна ширина символа 0xe5 в кодировке CP1251
\end{verbatim}

\begin{verbatim}
Warning in grid.Call.graphics(C_text, as.graphicsAnnot(x$label), x$x, x$y, :
неизвестна ширина символа 0xe6 в кодировке CP1251
\end{verbatim}

\begin{verbatim}
Warning in grid.Call.graphics(C_text, as.graphicsAnnot(x$label), x$x, x$y, :
неизвестна ширина символа 0xe0 в кодировке CP1251
\end{verbatim}

\begin{verbatim}
Warning in grid.Call.graphics(C_text, as.graphicsAnnot(x$label), x$x, x$y, :
неизвестна ширина символа 0xf2 в кодировке CP1251
\end{verbatim}

\begin{verbatim}
Warning in grid.Call.graphics(C_text, as.graphicsAnnot(x$label), x$x, x$y, :
неизвестна ширина символа 0xfc в кодировке CP1251
\end{verbatim}

\begin{verbatim}
Warning in grid.Call.graphics(C_text, as.graphicsAnnot(x$label), x$x, x$y, :
неизвестна ширина символа 0xe8 в кодировке CP1251
\end{verbatim}

\begin{verbatim}
Warning in grid.Call.graphics(C_text, as.graphicsAnnot(x$label), x$x, x$y, :
неизвестна ширина символа 0xf1 в кодировке CP1251
\end{verbatim}

\begin{verbatim}
Warning in grid.Call.graphics(C_text, as.graphicsAnnot(x$label), x$x, x$y, :
неизвестна ширина символа 0xf2 в кодировке CP1251
\end{verbatim}

\begin{verbatim}
Warning in grid.Call.graphics(C_text, as.graphicsAnnot(x$label), x$x, x$y, :
неизвестна ширина символа 0xee в кодировке CP1251
\end{verbatim}

\begin{verbatim}
Warning in grid.Call.graphics(C_text, as.graphicsAnnot(x$label), x$x, x$y, :
неизвестна ширина символа 0xf9 в кодировке CP1251
\end{verbatim}

\begin{verbatim}
Warning in grid.Call.graphics(C_text, as.graphicsAnnot(x$label), x$x, x$y, :
неизвестна ширина символа 0xe5 в кодировке CP1251
\end{verbatim}

\begin{verbatim}
Warning in grid.Call.graphics(C_text, as.graphicsAnnot(x$label), x$x, x$y, :
неизвестна ширина символа 0xed в кодировке CP1251
\end{verbatim}

\begin{verbatim}
Warning in grid.Call.graphics(C_text, as.graphicsAnnot(x$label), x$x, x$y, :
неизвестна ширина символа 0xe8 в кодировке CP1251
\end{verbatim}

\begin{verbatim}
Warning in grid.Call.graphics(C_text, as.graphicsAnnot(x$label), x$x, x$y, :
неизвестна ширина символа 0xff в кодировке CP1251
\end{verbatim}

\begin{verbatim}
Warning in grid.Call.graphics(C_text, as.graphicsAnnot(x$label), x$x, x$y, :
неизвестна ширина символа 0xd1 в кодировке CP1251
\end{verbatim}

\begin{verbatim}
Warning in grid.Call.graphics(C_text, as.graphicsAnnot(x$label), x$x, x$y, :
неизвестна ширина символа 0xf2 в кодировке CP1251
\end{verbatim}

\begin{verbatim}
Warning in grid.Call.graphics(C_text, as.graphicsAnnot(x$label), x$x, x$y, :
неизвестна ширина символа 0xf0 в кодировке CP1251
\end{verbatim}

\begin{verbatim}
Warning in grid.Call.graphics(C_text, as.graphicsAnnot(x$label), x$x, x$y, :
неизвестна ширина символа 0xe0 в кодировке CP1251
\end{verbatim}

\begin{verbatim}
Warning in grid.Call.graphics(C_text, as.graphicsAnnot(x$label), x$x, x$y, :
неизвестна ширина символа 0xf2 в кодировке CP1251
\end{verbatim}

\begin{verbatim}
Warning in grid.Call.graphics(C_text, as.graphicsAnnot(x$label), x$x, x$y, :
неизвестна ширина символа 0xe5 в кодировке CP1251
\end{verbatim}

\begin{verbatim}
Warning in grid.Call.graphics(C_text, as.graphicsAnnot(x$label), x$x, x$y, :
неизвестна ширина символа 0xe3 в кодировке CP1251
\end{verbatim}

\begin{verbatim}
Warning in grid.Call.graphics(C_text, as.graphicsAnnot(x$label), x$x, x$y, :
неизвестна ширина символа 0xe8 в кодировке CP1251
\end{verbatim}

\begin{verbatim}
Warning in grid.Call.graphics(C_text, as.graphicsAnnot(x$label), x$x, x$y, :
неизвестна ширина символа 0xff в кодировке CP1251
\end{verbatim}

\begin{verbatim}
Warning in grid.Call.graphics(C_text, as.graphicsAnnot(x$label), x$x, x$y, :
неизвестна ширина символа 0xd1 в кодировке CP1251
\end{verbatim}

\begin{verbatim}
Warning in grid.Call.graphics(C_text, as.graphicsAnnot(x$label), x$x, x$y, :
неизвестна ширина символа 0xf2 в кодировке CP1251
\end{verbatim}

\begin{verbatim}
Warning in grid.Call.graphics(C_text, as.graphicsAnnot(x$label), x$x, x$y, :
неизвестна ширина символа 0xe0 в кодировке CP1251
\end{verbatim}

\begin{verbatim}
Warning in grid.Call.graphics(C_text, as.graphicsAnnot(x$label), x$x, x$y, :
неизвестна ширина символа 0xe1 в кодировке CP1251
\end{verbatim}

\begin{verbatim}
Warning in grid.Call.graphics(C_text, as.graphicsAnnot(x$label), x$x, x$y, :
неизвестна ширина символа 0xe8 в кодировке CP1251
\end{verbatim}

\begin{verbatim}
Warning in grid.Call.graphics(C_text, as.graphicsAnnot(x$label), x$x, x$y, :
неизвестна ширина символа 0xeb в кодировке CP1251
\end{verbatim}

\begin{verbatim}
Warning in grid.Call.graphics(C_text, as.graphicsAnnot(x$label), x$x, x$y, :
неизвестна ширина символа 0xfc в кодировке CP1251
\end{verbatim}

\begin{verbatim}
Warning in grid.Call.graphics(C_text, as.graphicsAnnot(x$label), x$x, x$y, :
неизвестна ширина символа 0xed в кодировке CP1251
\end{verbatim}

\begin{verbatim}
Warning in grid.Call.graphics(C_text, as.graphicsAnnot(x$label), x$x, x$y, :
неизвестна ширина символа 0xee в кодировке CP1251
\end{verbatim}

\begin{verbatim}
Warning in grid.Call.graphics(C_text, as.graphicsAnnot(x$label), x$x, x$y, :
неизвестна ширина символа 0xf1 в кодировке CP1251
\end{verbatim}

\begin{verbatim}
Warning in grid.Call.graphics(C_text, as.graphicsAnnot(x$label), x$x, x$y, :
неизвестна ширина символа 0xf2 в кодировке CP1251
\end{verbatim}

\begin{verbatim}
Warning in grid.Call.graphics(C_text, as.graphicsAnnot(x$label), x$x, x$y, :
неизвестна ширина символа 0xfc в кодировке CP1251
\end{verbatim}

\begin{verbatim}
Warning in grid.Call.graphics(C_text, as.graphicsAnnot(x$label), x$x, x$y, :
неизвестна ширина символа 0xe2 в кодировке CP1251
\end{verbatim}

\begin{verbatim}
Warning in grid.Call.graphics(C_text, as.graphicsAnnot(x$label), x$x, x$y, :
неизвестна ширина символа 0xfb в кодировке CP1251
\end{verbatim}

\begin{verbatim}
Warning in grid.Call.graphics(C_text, as.graphicsAnnot(x$label), x$x, x$y, :
неизвестна ширина символа 0xeb в кодировке CP1251
\end{verbatim}

\begin{verbatim}
Warning in grid.Call.graphics(C_text, as.graphicsAnnot(x$label), x$x, x$y, :
неизвестна ширина символа 0xee в кодировке CP1251
\end{verbatim}

\begin{verbatim}
Warning in grid.Call.graphics(C_text, as.graphicsAnnot(x$label), x$x, x$y, :
неизвестна ширина символа 0xe2 в кодировке CP1251
\end{verbatim}

\begin{verbatim}
Warning in grid.Call.graphics(C_text, as.graphicsAnnot(x$label), x$x, x$y, :
неизвестна ширина символа 0xe0 в кодировке CP1251
\end{verbatim}

\begin{verbatim}
Warning in grid.Call.graphics(C_text, as.graphicsAnnot(x$label), x$x, x$y, :
неизвестна ширина символа 0xd0 в кодировке CP1251
\end{verbatim}

\begin{verbatim}
Warning in grid.Call.graphics(C_text, as.graphicsAnnot(x$label), x$x, x$y, :
неизвестна ширина символа 0xe0 в кодировке CP1251
\end{verbatim}

\begin{verbatim}
Warning in grid.Call.graphics(C_text, as.graphicsAnnot(x$label), x$x, x$y, :
неизвестна ширина символа 0xe7 в кодировке CP1251
\end{verbatim}

\begin{verbatim}
Warning in grid.Call.graphics(C_text, as.graphicsAnnot(x$label), x$x, x$y, :
неизвестна ширина символа 0xec в кодировке CP1251
\end{verbatim}

\begin{verbatim}
Warning in grid.Call.graphics(C_text, as.graphicsAnnot(x$label), x$x, x$y, :
неизвестна ширина символа 0xe5 в кодировке CP1251
\end{verbatim}

\begin{verbatim}
Warning in grid.Call.graphics(C_text, as.graphicsAnnot(x$label), x$x, x$y, :
неизвестна ширина символа 0xf0 в кодировке CP1251
\end{verbatim}

\begin{verbatim}
Warning in grid.Call.graphics(C_text, as.graphicsAnnot(x$label), x$x, x$y, :
неизвестна ширина символа 0xf2 в кодировке CP1251
\end{verbatim}

\begin{verbatim}
Warning in grid.Call.graphics(C_text, as.graphicsAnnot(x$label), x$x, x$y, :
неизвестна ширина символа 0xee в кодировке CP1251
\end{verbatim}

\begin{verbatim}
Warning in grid.Call.graphics(C_text, as.graphicsAnnot(x$label), x$x, x$y, :
неизвестна ширина символа 0xf7 в кодировке CP1251
\end{verbatim}

\begin{verbatim}
Warning in grid.Call.graphics(C_text, as.graphicsAnnot(x$label), x$x, x$y, :
неизвестна ширина символа 0xea в кодировке CP1251
\end{verbatim}

\begin{verbatim}
Warning in grid.Call.graphics(C_text, as.graphicsAnnot(x$label), x$x, x$y, :
неизвестна ширина символа 0xe8 в кодировке CP1251
\end{verbatim}

\begin{verbatim}
Warning in grid.Call.graphics(C_text, as.graphicsAnnot(x$label), x$x, x$y, :
неизвестна ширина символа 0xf1 в кодировке CP1251
\end{verbatim}

\begin{verbatim}
Warning in grid.Call.graphics(C_text, as.graphicsAnnot(x$label), x$x, x$y, :
неизвестна ширина символа 0xf2 в кодировке CP1251
\end{verbatim}

\begin{verbatim}
Warning in grid.Call.graphics(C_text, as.graphicsAnnot(x$label), x$x, x$y, :
неизвестна ширина символа 0xe0 в кодировке CP1251
\end{verbatim}

\begin{verbatim}
Warning in grid.Call.graphics(C_text, as.graphicsAnnot(x$label), x$x, x$y, :
неизвестна ширина символа 0xe1 в кодировке CP1251
\end{verbatim}

\begin{verbatim}
Warning in grid.Call.graphics(C_text, as.graphicsAnnot(x$label), x$x, x$y, :
неизвестна ширина символа 0xe8 в кодировке CP1251
\end{verbatim}

\begin{verbatim}
Warning in grid.Call.graphics(C_text, as.graphicsAnnot(x$label), x$x, x$y, :
неизвестна ширина символа 0xeb в кодировке CP1251
\end{verbatim}

\begin{verbatim}
Warning in grid.Call.graphics(C_text, as.graphicsAnnot(x$label), x$x, x$y, :
неизвестна ширина символа 0xfc в кодировке CP1251
\end{verbatim}

\begin{verbatim}
Warning in grid.Call.graphics(C_text, as.graphicsAnnot(x$label), x$x, x$y, :
неизвестна ширина символа 0xed в кодировке CP1251
\end{verbatim}

\begin{verbatim}
Warning in grid.Call.graphics(C_text, as.graphicsAnnot(x$label), x$x, x$y, :
неизвестна ширина символа 0xee в кодировке CP1251
\end{verbatim}

\begin{verbatim}
Warning in grid.Call.graphics(C_text, as.graphicsAnnot(x$label), x$x, x$y, :
неизвестна ширина символа 0xf1 в кодировке CP1251
\end{verbatim}

\begin{verbatim}
Warning in grid.Call.graphics(C_text, as.graphicsAnnot(x$label), x$x, x$y, :
неизвестна ширина символа 0xf2 в кодировке CP1251
\end{verbatim}

\begin{verbatim}
Warning in grid.Call.graphics(C_text, as.graphicsAnnot(x$label), x$x, x$y, :
неизвестна ширина символа 0xfc в кодировке CP1251
\end{verbatim}

\begin{verbatim}
Warning in grid.Call.graphics(C_text, as.graphicsAnnot(x$label), x$x, x$y, :
неизвестна ширина символа 0xe2 в кодировке CP1251
\end{verbatim}

\begin{verbatim}
Warning in grid.Call.graphics(C_text, as.graphicsAnnot(x$label), x$x, x$y, :
неизвестна ширина символа 0xfb в кодировке CP1251
\end{verbatim}

\begin{verbatim}
Warning in grid.Call.graphics(C_text, as.graphicsAnnot(x$label), x$x, x$y, :
неизвестна ширина символа 0xeb в кодировке CP1251
\end{verbatim}

\begin{verbatim}
Warning in grid.Call.graphics(C_text, as.graphicsAnnot(x$label), x$x, x$y, :
неизвестна ширина символа 0xee в кодировке CP1251
\end{verbatim}

\begin{verbatim}
Warning in grid.Call.graphics(C_text, as.graphicsAnnot(x$label), x$x, x$y, :
неизвестна ширина символа 0xe2 в кодировке CP1251
\end{verbatim}

\begin{verbatim}
Warning in grid.Call.graphics(C_text, as.graphicsAnnot(x$label), x$x, x$y, :
неизвестна ширина символа 0xe0 в кодировке CP1251
\end{verbatim}

\begin{verbatim}
Warning in grid.Call.graphics(C_text, as.graphicsAnnot(x$label), x$x, x$y, :
неизвестна ширина символа 0xc0 в кодировке CP1251
\end{verbatim}

\begin{verbatim}
Warning in grid.Call.graphics(C_text, as.graphicsAnnot(x$label), x$x, x$y, :
неизвестна ширина символа 0xed в кодировке CP1251
\end{verbatim}

\begin{verbatim}
Warning in grid.Call.graphics(C_text, as.graphicsAnnot(x$label), x$x, x$y, :
неизвестна ширина символа 0xe0 в кодировке CP1251
\end{verbatim}

\begin{verbatim}
Warning in grid.Call.graphics(C_text, as.graphicsAnnot(x$label), x$x, x$y, :
неизвестна ширина символа 0xeb в кодировке CP1251
\end{verbatim}

\begin{verbatim}
Warning in grid.Call.graphics(C_text, as.graphicsAnnot(x$label), x$x, x$y, :
неизвестна ширина символа 0xe8 в кодировке CP1251
\end{verbatim}

\begin{verbatim}
Warning in grid.Call.graphics(C_text, as.graphicsAnnot(x$label), x$x, x$y, :
неизвестна ширина символа 0xe7 в кодировке CP1251
\end{verbatim}

\begin{verbatim}
Warning in grid.Call.graphics(C_text, as.graphicsAnnot(x$label), x$x, x$y, :
неизвестна ширина символа 0xea в кодировке CP1251
\end{verbatim}

\begin{verbatim}
Warning in grid.Call.graphics(C_text, as.graphicsAnnot(x$label), x$x, x$y, :
неизвестна ширина символа 0xee в кодировке CP1251
\end{verbatim}

\begin{verbatim}
Warning in grid.Call.graphics(C_text, as.graphicsAnnot(x$label), x$x, x$y, :
неизвестна ширина символа 0xec в кодировке CP1251
\end{verbatim}

\begin{verbatim}
Warning in grid.Call.graphics(C_text, as.graphicsAnnot(x$label), x$x, x$y, :
неизвестна ширина символа 0xef в кодировке CP1251
\end{verbatim}

\begin{verbatim}
Warning in grid.Call.graphics(C_text, as.graphicsAnnot(x$label), x$x, x$y, :
неизвестна ширина символа 0xf0 в кодировке CP1251
\end{verbatim}

\begin{verbatim}
Warning in grid.Call.graphics(C_text, as.graphicsAnnot(x$label), x$x, x$y, :
неизвестна ширина символа 0xee в кодировке CP1251
\end{verbatim}

\begin{verbatim}
Warning in grid.Call.graphics(C_text, as.graphicsAnnot(x$label), x$x, x$y, :
неизвестна ширина символа 0xec в кодировке CP1251
\end{verbatim}

\begin{verbatim}
Warning in grid.Call.graphics(C_text, as.graphicsAnnot(x$label), x$x, x$y, :
неизвестна ширина символа 0xe8 в кодировке CP1251
\end{verbatim}

\begin{verbatim}
Warning in grid.Call.graphics(C_text, as.graphicsAnnot(x$label), x$x, x$y, :
неизвестна ширина символа 0xf1 в кодировке CP1251
Warning in grid.Call.graphics(C_text, as.graphicsAnnot(x$label), x$x, x$y, :
неизвестна ширина символа 0xf1 в кодировке CP1251
\end{verbatim}

\begin{verbatim}
Warning in grid.Call.graphics(C_text, as.graphicsAnnot(x$label), x$x, x$y, :
неизвестна ширина символа 0xee в кодировке CP1251
\end{verbatim}

\begin{verbatim}
Warning in grid.Call.graphics(C_text, as.graphicsAnnot(x$label), x$x, x$y, :
неизвестна ширина символа 0xe2 в кодировке CP1251
\end{verbatim}

\begin{verbatim}
Warning in grid.Call.graphics(C_text, as.graphicsAnnot(x$label), x$x, x$y, :
неизвестна ширина символа 0xec в кодировке CP1251
\end{verbatim}

\begin{verbatim}
Warning in grid.Call.graphics(C_text, as.graphicsAnnot(x$label), x$x, x$y, :
неизвестна ширина символа 0xe5 в кодировке CP1251
\end{verbatim}

\begin{verbatim}
Warning in grid.Call.graphics(C_text, as.graphicsAnnot(x$label), x$x, x$y, :
неизвестна ширина символа 0xe6 в кодировке CP1251
\end{verbatim}

\begin{verbatim}
Warning in grid.Call.graphics(C_text, as.graphicsAnnot(x$label), x$x, x$y, :
неизвестна ширина символа 0xe4 в кодировке CP1251
\end{verbatim}

\begin{verbatim}
Warning in grid.Call.graphics(C_text, as.graphicsAnnot(x$label), x$x, x$y, :
неизвестна ширина символа 0xf3 в кодировке CP1251
\end{verbatim}

\begin{verbatim}
Warning in grid.Call.graphics(C_text, as.graphicsAnnot(x$label), x$x, x$y, :
неизвестна ширина символа 0xf6 в кодировке CP1251
\end{verbatim}

\begin{verbatim}
Warning in grid.Call.graphics(C_text, as.graphicsAnnot(x$label), x$x, x$y, :
неизвестна ширина символа 0xe5 в кодировке CP1251
\end{verbatim}

\begin{verbatim}
Warning in grid.Call.graphics(C_text, as.graphicsAnnot(x$label), x$x, x$y, :
неизвестна ширина символа 0xeb в кодировке CP1251
\end{verbatim}

\begin{verbatim}
Warning in grid.Call.graphics(C_text, as.graphicsAnnot(x$label), x$x, x$y, :
неизвестна ширина символа 0xff в кодировке CP1251
\end{verbatim}

\begin{verbatim}
Warning in grid.Call.graphics(C_text, as.graphicsAnnot(x$label), x$x, x$y, :
неизвестна ширина символа 0xec в кодировке CP1251
\end{verbatim}

\begin{verbatim}
Warning in grid.Call.graphics(C_text, as.graphicsAnnot(x$label), x$x, x$y, :
неизвестна ширина символа 0xe8 в кодировке CP1251
\end{verbatim}

\begin{verbatim}
Warning in grid.Call.graphics(C_text, as.graphicsAnnot(x$label), x$x, x$y, :
неизвестна ширина символа 0xf3 в кодировке CP1251
\end{verbatim}

\begin{verbatim}
Warning in grid.Call.graphics(C_text, as.graphicsAnnot(x$label), x$x, x$y, :
неизвестна ширина символа 0xef в кодировке CP1251
\end{verbatim}

\begin{verbatim}
Warning in grid.Call.graphics(C_text, as.graphicsAnnot(x$label), x$x, x$y, :
неизвестна ширина символа 0xf0 в кодировке CP1251
\end{verbatim}

\begin{verbatim}
Warning in grid.Call.graphics(C_text, as.graphicsAnnot(x$label), x$x, x$y, :
неизвестна ширина символа 0xe0 в кодировке CP1251
\end{verbatim}

\begin{verbatim}
Warning in grid.Call.graphics(C_text, as.graphicsAnnot(x$label), x$x, x$y, :
неизвестна ширина символа 0xe2 в кодировке CP1251
\end{verbatim}

\begin{verbatim}
Warning in grid.Call.graphics(C_text, as.graphicsAnnot(x$label), x$x, x$y, :
неизвестна ширина символа 0xeb в кодировке CP1251
\end{verbatim}

\begin{verbatim}
Warning in grid.Call.graphics(C_text, as.graphicsAnnot(x$label), x$x, x$y, :
неизвестна ширина символа 0xe5 в кодировке CP1251
\end{verbatim}

\begin{verbatim}
Warning in grid.Call.graphics(C_text, as.graphicsAnnot(x$label), x$x, x$y, :
неизвестна ширина символа 0xed в кодировке CP1251
\end{verbatim}

\begin{verbatim}
Warning in grid.Call.graphics(C_text, as.graphicsAnnot(x$label), x$x, x$y, :
неизвестна ширина символа 0xe8 в кодировке CP1251
\end{verbatim}

\begin{verbatim}
Warning in grid.Call.graphics(C_text, as.graphicsAnnot(x$label), x$x, x$y, :
неизвестна ширина символа 0xff в кодировке CP1251
\end{verbatim}

\pandocbounded{\includegraphics[keepaspectratio]{chapter15_files/figure-pdf/unnamed-chunk-1-6.pdf}}

\begin{Shaded}
\begin{Highlighting}[]
\CommentTok{\# {-}{-}{-}{-}{-}{-}{-}{-}{-}{-}{-}{-}{-}{-}{-}{-}{-}{-}{-} 10. ДЕТАЛЬНЫЙ АНАЛИЗ {-}{-}{-}{-}{-}{-}{-}{-}{-}{-}{-}{-}{-}{-}{-}{-}{-}{-}{-}{-}}

\FunctionTok{cat}\NormalTok{(}\StringTok{"}\SpecialCharTok{\textbackslash{}n}\StringTok{========== ДЕТАЛЬНЫЙ АНАЛИЗ РЕЗУЛЬТАТОВ ==========}\SpecialCharTok{\textbackslash{}n}\StringTok{"}\NormalTok{)}
\end{Highlighting}
\end{Shaded}

\begin{verbatim}

========== ДЕТАЛЬНЫЙ АНАЛИЗ РЕЗУЛЬТАТОВ ==========
\end{verbatim}

\begin{Shaded}
\begin{Highlighting}[]
\DocumentationTok{\#\# 10.1 Анализ времени восстановления}
\FunctionTok{cat}\NormalTok{(}\StringTok{"}\SpecialCharTok{\textbackslash{}n}\StringTok{{-}{-}{-} Анализ восстановления запаса {-}{-}{-}}\SpecialCharTok{\textbackslash{}n}\StringTok{"}\NormalTok{)}
\end{Highlighting}
\end{Shaded}

\begin{verbatim}

--- Анализ восстановления запаса ---
\end{verbatim}

\begin{Shaded}
\begin{Highlighting}[]
\ControlFlowTok{if}\NormalTok{ (B\_current}\SpecialCharTok{/}\NormalTok{Bmsy\_true }\SpecialCharTok{\textless{}} \DecValTok{1}\NormalTok{) \{}
\NormalTok{  recovery\_analysis }\OtherTok{\textless{}{-}}\NormalTok{ mse\_results }\SpecialCharTok{\%\textgreater{}\%}
    \FunctionTok{group\_by}\NormalTok{(strategy, sim\_id) }\SpecialCharTok{\%\textgreater{}\%}
    \FunctionTok{summarise}\NormalTok{(}
      \AttributeTok{recovery\_year =} \FunctionTok{which}\NormalTok{(B\_Bmsy\_true }\SpecialCharTok{\textgreater{}} \DecValTok{1}\NormalTok{)[}\DecValTok{1}\NormalTok{],}
      \AttributeTok{recovered =} \SpecialCharTok{!}\FunctionTok{is.na}\NormalTok{(recovery\_year),}
      \AttributeTok{.groups =} \StringTok{"drop"}
\NormalTok{    ) }\SpecialCharTok{\%\textgreater{}\%}
    \FunctionTok{group\_by}\NormalTok{(strategy) }\SpecialCharTok{\%\textgreater{}\%}
    \FunctionTok{summarise}\NormalTok{(}
      \AttributeTok{prob\_recovery =} \FunctionTok{mean}\NormalTok{(recovered, }\AttributeTok{na.rm =} \ConstantTok{TRUE}\NormalTok{),}
      \AttributeTok{median\_recovery\_time =} \FunctionTok{median}\NormalTok{(recovery\_year, }\AttributeTok{na.rm =} \ConstantTok{TRUE}\NormalTok{),}
      \AttributeTok{q25\_recovery =} \FunctionTok{quantile}\NormalTok{(recovery\_year, }\FloatTok{0.25}\NormalTok{, }\AttributeTok{na.rm =} \ConstantTok{TRUE}\NormalTok{),}
      \AttributeTok{q75\_recovery =} \FunctionTok{quantile}\NormalTok{(recovery\_year, }\FloatTok{0.75}\NormalTok{, }\AttributeTok{na.rm =} \ConstantTok{TRUE}\NormalTok{),}
      \AttributeTok{.groups =} \StringTok{"drop"}
\NormalTok{    )}
  
  \ControlFlowTok{for}\NormalTok{ (i }\ControlFlowTok{in} \DecValTok{1}\SpecialCharTok{:}\FunctionTok{nrow}\NormalTok{(recovery\_analysis)) \{}
    \FunctionTok{cat}\NormalTok{(}\FunctionTok{sprintf}\NormalTok{(}\StringTok{"\%s:}\SpecialCharTok{\textbackslash{}n}\StringTok{"}\NormalTok{, recovery\_analysis}\SpecialCharTok{$}\NormalTok{strategy[i]))}
    \FunctionTok{cat}\NormalTok{(}\FunctionTok{sprintf}\NormalTok{(}\StringTok{"  Вероятность восстановления: \%.1f\%\%}\SpecialCharTok{\textbackslash{}n}\StringTok{"}\NormalTok{, }
\NormalTok{                recovery\_analysis}\SpecialCharTok{$}\NormalTok{prob\_recovery[i] }\SpecialCharTok{*} \DecValTok{100}\NormalTok{))}
    \ControlFlowTok{if}\NormalTok{ (}\SpecialCharTok{!}\FunctionTok{is.na}\NormalTok{(recovery\_analysis}\SpecialCharTok{$}\NormalTok{median\_recovery\_time[i])) \{}
      \FunctionTok{cat}\NormalTok{(}\FunctionTok{sprintf}\NormalTok{(}\StringTok{"  Медианное время восстановления: \%.0f лет [\%.0f{-}\%.0f]}\SpecialCharTok{\textbackslash{}n}\StringTok{"}\NormalTok{,}
\NormalTok{                  recovery\_analysis}\SpecialCharTok{$}\NormalTok{median\_recovery\_time[i],}
\NormalTok{                  recovery\_analysis}\SpecialCharTok{$}\NormalTok{q25\_recovery[i],}
\NormalTok{                  recovery\_analysis}\SpecialCharTok{$}\NormalTok{q75\_recovery[i]))}
\NormalTok{    \}}
\NormalTok{  \}}
\NormalTok{\} }\ControlFlowTok{else}\NormalTok{ \{}
  \FunctionTok{cat}\NormalTok{(}\StringTok{"Запас находится на уровне B/Bmsy \textgreater{}= 1. Восстановление не требуется.}\SpecialCharTok{\textbackslash{}n}\StringTok{"}\NormalTok{)}
  \FunctionTok{cat}\NormalTok{(}\FunctionTok{sprintf}\NormalTok{(}\StringTok{"Текущее состояние: B/Bmsy = \%.2f}\SpecialCharTok{\textbackslash{}n}\StringTok{"}\NormalTok{, B\_current }\SpecialCharTok{/}\NormalTok{ Bmsy\_true))}
\NormalTok{\}}
\end{Highlighting}
\end{Shaded}

\begin{verbatim}
Запас находится на уровне B/Bmsy >= 1. Восстановление не требуется.
Текущее состояние: B/Bmsy = 1.34
\end{verbatim}

\begin{Shaded}
\begin{Highlighting}[]
\DocumentationTok{\#\# 10.2 Анализ частоты закрытия промысла}
\NormalTok{closure\_analysis }\OtherTok{\textless{}{-}}\NormalTok{ mse\_results }\SpecialCharTok{\%\textgreater{}\%}
  \FunctionTok{group\_by}\NormalTok{(strategy) }\SpecialCharTok{\%\textgreater{}\%}
  \FunctionTok{summarise}\NormalTok{(}
    \AttributeTok{total\_closures =} \FunctionTok{sum}\NormalTok{(TAC }\SpecialCharTok{==} \DecValTok{0}\NormalTok{),}
    \AttributeTok{closure\_rate =} \FunctionTok{mean}\NormalTok{(TAC }\SpecialCharTok{==} \DecValTok{0}\NormalTok{),}
    \AttributeTok{avg\_closure\_duration =}\NormalTok{ \{}
\NormalTok{      rle\_results }\OtherTok{\textless{}{-}} \FunctionTok{rle}\NormalTok{(TAC }\SpecialCharTok{==} \DecValTok{0}\NormalTok{)}
      \ControlFlowTok{if}\NormalTok{ (}\FunctionTok{any}\NormalTok{(rle\_results}\SpecialCharTok{$}\NormalTok{values)) \{}
        \FunctionTok{mean}\NormalTok{(rle\_results}\SpecialCharTok{$}\NormalTok{lengths[rle\_results}\SpecialCharTok{$}\NormalTok{values])}
\NormalTok{      \} }\ControlFlowTok{else}\NormalTok{ \{}
        \DecValTok{0}
\NormalTok{      \}}
\NormalTok{    \},}
    \AttributeTok{.groups =} \StringTok{"drop"}
\NormalTok{  )}

\FunctionTok{cat}\NormalTok{(}\StringTok{"}\SpecialCharTok{\textbackslash{}n}\StringTok{{-}{-}{-} Анализ закрытия промысла {-}{-}{-}}\SpecialCharTok{\textbackslash{}n}\StringTok{"}\NormalTok{)}
\end{Highlighting}
\end{Shaded}

\begin{verbatim}

--- Анализ закрытия промысла ---
\end{verbatim}

\begin{Shaded}
\begin{Highlighting}[]
\ControlFlowTok{for}\NormalTok{ (i }\ControlFlowTok{in} \DecValTok{1}\SpecialCharTok{:}\FunctionTok{nrow}\NormalTok{(closure\_analysis)) \{}
  \FunctionTok{cat}\NormalTok{(}\FunctionTok{sprintf}\NormalTok{(}\StringTok{"\%s:}\SpecialCharTok{\textbackslash{}n}\StringTok{"}\NormalTok{, closure\_analysis}\SpecialCharTok{$}\NormalTok{strategy[i]))}
  \FunctionTok{cat}\NormalTok{(}\FunctionTok{sprintf}\NormalTok{(}\StringTok{"  Частота закрытия: \%.1f\%\%}\SpecialCharTok{\textbackslash{}n}\StringTok{"}\NormalTok{, closure\_analysis}\SpecialCharTok{$}\NormalTok{closure\_rate[i] }\SpecialCharTok{*} \DecValTok{100}\NormalTok{))}
  \ControlFlowTok{if}\NormalTok{ (closure\_analysis}\SpecialCharTok{$}\NormalTok{avg\_closure\_duration[i] }\SpecialCharTok{\textgreater{}} \DecValTok{0}\NormalTok{) \{}
    \FunctionTok{cat}\NormalTok{(}\FunctionTok{sprintf}\NormalTok{(}\StringTok{"  Средняя продолжительность закрытия: \%.1f лет}\SpecialCharTok{\textbackslash{}n}\StringTok{"}\NormalTok{,}
\NormalTok{                closure\_analysis}\SpecialCharTok{$}\NormalTok{avg\_closure\_duration[i]))}
\NormalTok{  \}}
\NormalTok{\}}
\end{Highlighting}
\end{Shaded}

\begin{verbatim}
Fish at Fmsy:
  Частота закрытия: 0.0%
ICES Advice Rule:
  Частота закрытия: 1.1%
  Средняя продолжительность закрытия: 2.3 лет
MSY Hockey-stick:
  Частота закрытия: 0.0%
  Средняя продолжительность закрытия: 2.0 лет
\end{verbatim}

\begin{Shaded}
\begin{Highlighting}[]
\CommentTok{\# {-}{-}{-}{-}{-}{-}{-}{-}{-}{-}{-}{-}{-}{-}{-}{-}{-}{-}{-} 11. СОХРАНЕНИЕ РЕЗУЛЬТАТОВ {-}{-}{-}{-}{-}{-}{-}{-}{-}{-}{-}{-}{-}{-}{-}{-}{-}{-}{-}{-}}

\FunctionTok{cat}\NormalTok{(}\StringTok{"}\SpecialCharTok{\textbackslash{}n}\StringTok{========== СОХРАНЕНИЕ РЕЗУЛЬТАТОВ ==========}\SpecialCharTok{\textbackslash{}n}\StringTok{"}\NormalTok{)}
\end{Highlighting}
\end{Shaded}

\begin{verbatim}

========== СОХРАНЕНИЕ РЕЗУЛЬТАТОВ ==========
\end{verbatim}

\begin{Shaded}
\begin{Highlighting}[]
\DocumentationTok{\#\# 11.2 Сохранение таблиц}
\FunctionTok{write.csv}\NormalTok{(performance\_metrics, }\StringTok{"MSE\_performance\_metrics.csv"}\NormalTok{, }\AttributeTok{row.names =} \ConstantTok{FALSE}\NormalTok{)}
\FunctionTok{write.csv}\NormalTok{(summary\_data, }\StringTok{"MSE\_summary\_by\_year.csv"}\NormalTok{, }\AttributeTok{row.names =} \ConstantTok{FALSE}\NormalTok{)}
\FunctionTok{cat}\NormalTok{(}\StringTok{"✓ Таблицы метрик сохранены}\SpecialCharTok{\textbackslash{}n}\StringTok{"}\NormalTok{)}
\end{Highlighting}
\end{Shaded}

\begin{verbatim}
<U+2713> Таблицы метрик сохранены
\end{verbatim}

\begin{Shaded}
\begin{Highlighting}[]
\DocumentationTok{\#\# 11.3 Сохранение полных результатов}
\FunctionTok{saveRDS}\NormalTok{(mse\_results, }\StringTok{"MSE\_full\_results.rds"}\NormalTok{)}
\FunctionTok{cat}\NormalTok{(}\StringTok{"✓ Полные результаты симуляций сохранены: \textquotesingle{}MSE\_full\_results.rds\textquotesingle{}}\SpecialCharTok{\textbackslash{}n}\StringTok{"}\NormalTok{)}
\end{Highlighting}
\end{Shaded}

\begin{verbatim}
<U+2713> Полные результаты симуляций сохранены: 'MSE_full_results.rds'
\end{verbatim}

\begin{Shaded}
\begin{Highlighting}[]
\DocumentationTok{\#\# 11.4 Создание текстового отчета}
\FunctionTok{sink}\NormalTok{(}\StringTok{"MSE\_report.txt"}\NormalTok{)}
\FunctionTok{cat}\NormalTok{(}\StringTok{"ОТЧЕТ ПО MANAGEMENT STRATEGY EVALUATION}\SpecialCharTok{\textbackslash{}n}\StringTok{"}\NormalTok{)}
\FunctionTok{cat}\NormalTok{(}\FunctionTok{strrep}\NormalTok{(}\StringTok{"="}\NormalTok{, }\DecValTok{60}\NormalTok{), }\StringTok{"}\SpecialCharTok{\textbackslash{}n\textbackslash{}n}\StringTok{"}\NormalTok{)}
\FunctionTok{cat}\NormalTok{(}\StringTok{"Дата анализа:"}\NormalTok{, }\FunctionTok{format}\NormalTok{(}\FunctionTok{Sys.Date}\NormalTok{(), }\StringTok{"\%d.\%m.\%Y"}\NormalTok{), }\StringTok{"}\SpecialCharTok{\textbackslash{}n}\StringTok{"}\NormalTok{)}
\FunctionTok{cat}\NormalTok{(}\StringTok{"Версия SPiCT:"}\NormalTok{, }\FunctionTok{as.character}\NormalTok{(}\FunctionTok{packageVersion}\NormalTok{(}\StringTok{"spict"}\NormalTok{)), }\StringTok{"}\SpecialCharTok{\textbackslash{}n\textbackslash{}n}\StringTok{"}\NormalTok{)}

\FunctionTok{cat}\NormalTok{(}\StringTok{"ПАРАМЕТРЫ СИМУЛЯЦИИ:}\SpecialCharTok{\textbackslash{}n}\StringTok{"}\NormalTok{)}
\FunctionTok{cat}\NormalTok{(}\FunctionTok{sprintf}\NormalTok{(}\StringTok{"  Количество симуляций: \%d}\SpecialCharTok{\textbackslash{}n}\StringTok{"}\NormalTok{, n\_sim))}
\FunctionTok{cat}\NormalTok{(}\FunctionTok{sprintf}\NormalTok{(}\StringTok{"  Период прогноза: \%d лет}\SpecialCharTok{\textbackslash{}n}\StringTok{"}\NormalTok{, n\_years))}
\FunctionTok{cat}\NormalTok{(}\FunctionTok{sprintf}\NormalTok{(}\StringTok{"  Интервал оценки: каждые \%d года}\SpecialCharTok{\textbackslash{}n}\StringTok{"}\NormalTok{, assessment\_interval))}
\FunctionTok{cat}\NormalTok{(}\FunctionTok{sprintf}\NormalTok{(}\StringTok{"  CV процессной ошибки: \%.0f\%\%}\SpecialCharTok{\textbackslash{}n}\StringTok{"}\NormalTok{, process\_error\_cv }\SpecialCharTok{*} \DecValTok{100}\NormalTok{))}
\FunctionTok{cat}\NormalTok{(}\FunctionTok{sprintf}\NormalTok{(}\StringTok{"  CV ошибки наблюдения: \%.0f\%\%}\SpecialCharTok{\textbackslash{}n}\StringTok{"}\NormalTok{, observation\_error\_cv }\SpecialCharTok{*} \DecValTok{100}\NormalTok{))}
\FunctionTok{cat}\NormalTok{(}\FunctionTok{sprintf}\NormalTok{(}\StringTok{"  CV смещения оценок: \%.0f\%\%}\SpecialCharTok{\textbackslash{}n}\StringTok{"}\NormalTok{, assessment\_bias\_cv }\SpecialCharTok{*} \DecValTok{100}\NormalTok{))}
\FunctionTok{cat}\NormalTok{(}\FunctionTok{sprintf}\NormalTok{(}\StringTok{"  CV ошибки реализации: \%.0f\%\%}\SpecialCharTok{\textbackslash{}n\textbackslash{}n}\StringTok{"}\NormalTok{, implementation\_error\_cv }\SpecialCharTok{*} \DecValTok{100}\NormalTok{))}

\FunctionTok{cat}\NormalTok{(}\StringTok{"ИСХОДНОЕ СОСТОЯНИЕ ЗАПАСА:}\SpecialCharTok{\textbackslash{}n}\StringTok{"}\NormalTok{)}
\FunctionTok{cat}\NormalTok{(}\FunctionTok{sprintf}\NormalTok{(}\StringTok{"  B/Bmsy: \%.2f}\SpecialCharTok{\textbackslash{}n}\StringTok{"}\NormalTok{, B\_current}\SpecialCharTok{/}\NormalTok{Bmsy\_true))}
\FunctionTok{cat}\NormalTok{(}\FunctionTok{sprintf}\NormalTok{(}\StringTok{"  F/Fmsy: \%.2f}\SpecialCharTok{\textbackslash{}n\textbackslash{}n}\StringTok{"}\NormalTok{, F\_current}\SpecialCharTok{/}\NormalTok{Fmsy\_true))}

\FunctionTok{cat}\NormalTok{(}\StringTok{"РЕЗУЛЬТАТЫ ПО СТРАТЕГИЯМ:}\SpecialCharTok{\textbackslash{}n\textbackslash{}n}\StringTok{"}\NormalTok{)}
\ControlFlowTok{for}\NormalTok{ (i }\ControlFlowTok{in} \DecValTok{1}\SpecialCharTok{:}\FunctionTok{nrow}\NormalTok{(performance\_metrics)) \{}
  \FunctionTok{cat}\NormalTok{(}\FunctionTok{sprintf}\NormalTok{(}\StringTok{"СТРАТЕГИЯ: \%s}\SpecialCharTok{\textbackslash{}n}\StringTok{"}\NormalTok{, performance\_metrics}\SpecialCharTok{$}\NormalTok{strategy[i]))}
  \FunctionTok{cat}\NormalTok{(}\FunctionTok{strrep}\NormalTok{(}\StringTok{"{-}"}\NormalTok{, }\DecValTok{40}\NormalTok{), }\StringTok{"}\SpecialCharTok{\textbackslash{}n}\StringTok{"}\NormalTok{)}
  \FunctionTok{cat}\NormalTok{(}\FunctionTok{sprintf}\NormalTok{(}\StringTok{"  Риск перелова: \%.1f\%\%}\SpecialCharTok{\textbackslash{}n}\StringTok{"}\NormalTok{, performance\_metrics}\SpecialCharTok{$}\NormalTok{prob\_overfishing[i] }\SpecialCharTok{*} \DecValTok{100}\NormalTok{))}
  \FunctionTok{cat}\NormalTok{(}\FunctionTok{sprintf}\NormalTok{(}\StringTok{"  Риск истощения: \%.1f\%\%}\SpecialCharTok{\textbackslash{}n}\StringTok{"}\NormalTok{, performance\_metrics}\SpecialCharTok{$}\NormalTok{prob\_overfished[i] }\SpecialCharTok{*} \DecValTok{100}\NormalTok{))}
  \FunctionTok{cat}\NormalTok{(}\FunctionTok{sprintf}\NormalTok{(}\StringTok{"  Вероятность устойчивого состояния: \%.1f\%\%}\SpecialCharTok{\textbackslash{}n}\StringTok{"}\NormalTok{, }
\NormalTok{              performance\_metrics}\SpecialCharTok{$}\NormalTok{prob\_green\_zone[i] }\SpecialCharTok{*} \DecValTok{100}\NormalTok{))}
  \FunctionTok{cat}\NormalTok{(}\FunctionTok{sprintf}\NormalTok{(}\StringTok{"  Средний вылов: \%.1f тыс. т}\SpecialCharTok{\textbackslash{}n}\StringTok{"}\NormalTok{, performance\_metrics}\SpecialCharTok{$}\NormalTok{mean\_catch[i]))}
  \FunctionTok{cat}\NormalTok{(}\FunctionTok{sprintf}\NormalTok{(}\StringTok{"  Стабильность вылова: \%.2f}\SpecialCharTok{\textbackslash{}n}\StringTok{"}\NormalTok{, performance\_metrics}\SpecialCharTok{$}\NormalTok{catch\_stability[i]))}
  \FunctionTok{cat}\NormalTok{(}\FunctionTok{sprintf}\NormalTok{(}\StringTok{"  Общий индекс производительности: \%.3f}\SpecialCharTok{\textbackslash{}n\textbackslash{}n}\StringTok{"}\NormalTok{, }
\NormalTok{              performance\_metrics}\SpecialCharTok{$}\NormalTok{overall\_score[i]))}
\NormalTok{\}}

\FunctionTok{cat}\NormalTok{(}\StringTok{"РЕКОМЕНДАЦИИ:}\SpecialCharTok{\textbackslash{}n}\StringTok{"}\NormalTok{)}
\NormalTok{best\_strategy }\OtherTok{\textless{}{-}}\NormalTok{ performance\_metrics}\SpecialCharTok{$}\NormalTok{strategy[}\DecValTok{1}\NormalTok{]}
\FunctionTok{cat}\NormalTok{(}\FunctionTok{sprintf}\NormalTok{(}\StringTok{"Оптимальная стратегия: \%s}\SpecialCharTok{\textbackslash{}n}\StringTok{"}\NormalTok{, best\_strategy))}
\FunctionTok{cat}\NormalTok{(}\StringTok{"}\SpecialCharTok{\textbackslash{}n}\StringTok{Обоснование:}\SpecialCharTok{\textbackslash{}n}\StringTok{"}\NormalTok{)}
\ControlFlowTok{if}\NormalTok{ (best\_strategy }\SpecialCharTok{==} \StringTok{"ICES Advice Rule"}\NormalTok{) \{}
  \FunctionTok{cat}\NormalTok{(}\StringTok{"{-} Предосторожный подход минимизирует биологические риски}\SpecialCharTok{\textbackslash{}n}\StringTok{"}\NormalTok{)}
  \FunctionTok{cat}\NormalTok{(}\StringTok{"{-} Ограничение межгодовых изменений обеспечивает стабильность для промышленности}\SpecialCharTok{\textbackslash{}n}\StringTok{"}\NormalTok{)}
  \FunctionTok{cat}\NormalTok{(}\StringTok{"{-} Баланс между сохранением запаса и экономической эффективностью}\SpecialCharTok{\textbackslash{}n}\StringTok{"}\NormalTok{)}
\NormalTok{\} }\ControlFlowTok{else} \ControlFlowTok{if}\NormalTok{ (best\_strategy }\SpecialCharTok{==} \StringTok{"MSY Hockey{-}stick"}\NormalTok{) \{}
  \FunctionTok{cat}\NormalTok{(}\StringTok{"{-} Адаптивное управление в зависимости от состояния запаса}\SpecialCharTok{\textbackslash{}n}\StringTok{"}\NormalTok{)}
  \FunctionTok{cat}\NormalTok{(}\StringTok{"{-} Защита запаса при низкой биомассе}\SpecialCharTok{\textbackslash{}n}\StringTok{"}\NormalTok{)}
  \FunctionTok{cat}\NormalTok{(}\StringTok{"{-} Возможность полного использования при хорошем состоянии}\SpecialCharTok{\textbackslash{}n}\StringTok{"}\NormalTok{)}
\NormalTok{\} }\ControlFlowTok{else}\NormalTok{ \{}
  \FunctionTok{cat}\NormalTok{(}\StringTok{"{-} Простота применения}\SpecialCharTok{\textbackslash{}n}\StringTok{"}\NormalTok{)}
  \FunctionTok{cat}\NormalTok{(}\StringTok{"{-} Максимизация долгосрочного вылова}\SpecialCharTok{\textbackslash{}n}\StringTok{"}\NormalTok{)}
  \FunctionTok{cat}\NormalTok{(}\StringTok{"{-} Требует точной оценки Fmsy}\SpecialCharTok{\textbackslash{}n}\StringTok{"}\NormalTok{)}
\NormalTok{\}}

\FunctionTok{sink}\NormalTok{()}
\FunctionTok{cat}\NormalTok{(}\StringTok{"✓ Текстовый отчет сохранен: \textquotesingle{}MSE\_report.txt\textquotesingle{}}\SpecialCharTok{\textbackslash{}n}\StringTok{"}\NormalTok{)}
\end{Highlighting}
\end{Shaded}

\begin{verbatim}
<U+2713> Текстовый отчет сохранен: 'MSE_report.txt'
\end{verbatim}

\begin{Shaded}
\begin{Highlighting}[]
\FunctionTok{cat}\NormalTok{(}\StringTok{"}\SpecialCharTok{\textbackslash{}n}\StringTok{"}\NormalTok{ , }\FunctionTok{strrep}\NormalTok{(}\StringTok{"="}\NormalTok{, }\DecValTok{60}\NormalTok{), }\StringTok{"}\SpecialCharTok{\textbackslash{}n}\StringTok{"}\NormalTok{)}
\end{Highlighting}
\end{Shaded}

\begin{verbatim}

 ============================================================ 
\end{verbatim}

\begin{Shaded}
\begin{Highlighting}[]
\FunctionTok{cat}\NormalTok{(}\StringTok{"         MSE АНАЛИЗ УСПЕШНО ЗАВЕРШЕН}\SpecialCharTok{\textbackslash{}n}\StringTok{"}\NormalTok{)}
\end{Highlighting}
\end{Shaded}

\begin{verbatim}
         MSE АНАЛИЗ УСПЕШНО ЗАВЕРШЕН
\end{verbatim}

\begin{Shaded}
\begin{Highlighting}[]
\FunctionTok{cat}\NormalTok{(}\FunctionTok{strrep}\NormalTok{(}\StringTok{"="}\NormalTok{, }\DecValTok{60}\NormalTok{), }\StringTok{"}\SpecialCharTok{\textbackslash{}n}\StringTok{"}\NormalTok{)}
\end{Highlighting}
\end{Shaded}

\begin{verbatim}
============================================================ 
\end{verbatim}

Результаты Management Strategy Evaluation: Сравнение стратегий
управления Результаты выполнения третьего скрипта представляют собой
комплексный анализ эффективности трех ключевых стратегий управления в
условиях неопределенности. После 500 симуляций для каждой стратегии на
100-летнем горизонте получены следующие результаты.

Стратегия ``Fish at Fmsy'' показала наихудшие биологические показатели.
Вероятность перелова составила 48.5\%, вероятность истощения запаса (B
\textless{} 0.5Bmsy) --- 4.7\%. Несмотря на наибольший средний вылов
(10.5 тыс. тонн), эта стратегия демонстрирует неприемлемо высокие риски.
Только 20\% времени запас находился в ``зеленой зоне'' --- состоянии,
когда биомасса превышает Bmsy при промысловой смертности ниже Fmsy.

Стратегия ``MSY Hockey-stick'' показала умеренное улучшение. Вероятность
перелова снизилась до 46.3\%, вероятность истощения --- до 3.7\%. Однако
показатель нахождения в зеленой зоне остался на низком уровне (20.2\%).
Средний вылов практически не изменился (10.5 тыс. тонн), что указывает
на ограниченную эффективность данного подхода в конкретных условиях
моделирования.

Стратегия ``ICES Advice Rule'' продемонстрировала наилучшие результаты.
Вероятность перелова составила лишь 8.3\%, вероятность истощения ---
0.2\%. Запас проводил 57.6\% времени в зеленой зоне, что свидетельствует
о высокой устойчивости. При этом средний вылов (10.3 тыс. тонн)
практически не уступил другим стратегиям.

Анализ динамики показал, что адаптивные стратегии обеспечивают более
стабильное состояние запаса в долгосрочной перспективе. К 20-му году
прогноза стратегия ICES стабилизировала запас на уровне 1.0-1.2 B/Bmsy,
в то время как другие стратегии демонстрировали значительные колебания.

Экономические показатели также свидетельствуют в пользу предосторожного
подхода. Стабильность вылова при стратегии ICES составила 0.95 против
0.92 у других стратегий, при этом межгодовая изменчивость вылова была
минимальной (1.3\%).

Общий индекс производительности, учитывающий как биологические, так и
экономические аспекты, составил 0.90 для ICES против 0.75 для обеих
других стратегий. Это указывает на явное преимущество предосторожного
подхода в условиях неопределенности.

Как мог бы заключить Довлатов, ``идеальное управление --- это когда и
рыба цела, и рыбаки довольны, но в реальности приходится выбирать между
риском и уверенностью''. Результаты MSE однозначно показывают, что
стратегия ICES Advice Rule обеспечивает оптимальный баланс между
сохранением ресурса и экономической эффективностью.

Важно отметить, что эти результаты справедливы для конкретных условий
моделирования --- начального состояния запаса (B/Bmsy = 1.34) и принятых
уровней неопределенности. Изменение исходных параметров может повлиять
на сравнительную эффективность стратегий, однако общее преимущество
предосторожного подхода сохраняется.

\bookmarksetup{startatroot}

\chapter{I. DLM: Введение в методы анализа при очень ограниченных
данных}\label{i.-dlm-ux432ux432ux435ux434ux435ux43dux438ux435-ux432-ux43cux435ux442ux43eux434ux44b-ux430ux43dux430ux43bux438ux437ux430-ux43fux440ux438-ux43eux447ux435ux43dux44c-ux43eux433ux440ux430ux43dux438ux447ux435ux43dux43dux44bux445-ux434ux430ux43dux43dux44bux445}

Это занятие открывает серию практик о так называемых ``немодельных'',
как написано в некоторых
\href{http://www.vniro.ru/files/publish/metod_rek_ocenka_zapasov_babayan.pdf}{методиках},
подходах, которые в англоязычной литературе принято называть DLM. Это
занятие вводное - возьмем cамый простой метод и прогоним весь цикл.
Data-Limited Methods (DLM) --- это не просто статистические подходы, а
скорее философия работы в условиях неопределенности. Если классические
методы оценки запасов требуют многолетних рядов данных, детальной
информации о возрасте, размерах, промысловом усилии и прочих параметрах,
то DLM предлагают прагматические решения для ситуаций, когда
единственное, что у вас есть --- это данные об уловах. И да, такое
бывает чаще, чем кажется.

Представьте себе типичную ситуацию: новый объект промысла, развивающийся
рыболовный регион, или просто вид, который никогда не был приоритетом
для исследований. Данные об уловах есть --- их собирают всегда. А вот
биологические параметры, индексы численности, данные научных съемок ---
это уже из области фантастики. Именно в таких условиях DLM становятся не
просто удобным инструментом, а единственным способом хоть как-то оценить
состояние запаса.

Среди множества DLM-методов особое место занимает Catch-MSY --- подход,
который позволяет оценить максимальный устойчивый вылов (\emph{MSY}) на
основе одних только данных об уловах. Метод основан на простой, но
гениальной идее: если мы знаем историю промысла, мы можем методом
подбора найти такие комбинации параметров роста популяции (\emph{r} и
\emph{K}), которые объясняют наблюдаемую динамику уловов и при этом
биологически реалистичны.

Конечно, у этого метода есть ограничения. Он предполагает, что популяция
следует логистической модели роста, что промысел был основным фактором
смертности, и что мы можем разумно ограничить диапазоны параметров *r**
и \emph{K}. Но как говорится, в условиях слепоты и одноглазый ---
король.

Что особенно важно --- Catch-MSY не просто дает точечную оценку
\emph{MSY}, но и обеспечивает распределение вероятностей различных
значений, что позволяет оценить неопределенность. Это важно для
управления рисками.

Переход от оценки к управлению --- это отдельная история. Полученные
оценки \emph{MSY} становятся основой для разработки правил управления
(Harvest Control Rules), которые определяют, как следует изменять
уровень изъятия в зависимости от состояния запаса. А уже эти правила
тестируются в рамках Management Strategy Evaluation (MSE) ---
своеобразного полигона, где мы проверяем, как различные стратегии
управления будут работать в условиях неопределенности.

Курьезность ситуации заключается в том, что несмотря на то, что мы
используем чрезвычайно простые методы для оценки, процесс управления
становится все более сложным. Но это именно тот случай, когда сложность
оправдана --- ведь на кону устойчивость промысла и сохранение ресурсов.

Ирония судьбы: методы, разработанные для условий недостатка данных,
часто оказываются более робастны (устойчивы), чем их сложные собратья,
требующие тонны информации. Просто потому что они ``осознают''
неопределенность, а не делают вид, что ее не существует.

В следующих лекциях мы перейдем от теории к практике и посмотрим, как
эти методы работают на реальных данных. Приготовьтесь к приятным
сюрпризам.

\begin{Shaded}
\begin{Highlighting}[]
\CommentTok{\# =============================================================================}
\CommentTok{\# ПРАКТИЧЕСКИЙ АНАЛИЗ ВОДНЫХ БИОРЕСУРСОВ МЕТОДОМ CATCH{-}MSY}
\CommentTok{\# =============================================================================}

\CommentTok{\# {-}{-}{-}{-}{-}{-}{-}{-}{-}{-}{-}{-}{-}{-}{-}{-}{-}{-}{-}{-}{-}{-}{-}{-}{-} 1. УСТАНОВКА И ЗАГРУЗКА ПАКЕТОВ {-}{-}{-}{-}{-}{-}{-}{-}{-}{-}{-}{-}{-}{-}{-}{-}{-}{-}{-}{-}{-}{-}{-}{-}{-}{-}{-}}

\CommentTok{\# Устанавливаем необходимые пакеты (если еще не установлены)}
\ControlFlowTok{if}\NormalTok{ (}\SpecialCharTok{!}\FunctionTok{require}\NormalTok{(}\StringTok{"ggplot2"}\NormalTok{)) }\FunctionTok{install.packages}\NormalTok{(}\StringTok{"ggplot2"}\NormalTok{)}
\end{Highlighting}
\end{Shaded}

\begin{verbatim}
Загрузка требуемого пакета: ggplot2
\end{verbatim}

\begin{Shaded}
\begin{Highlighting}[]
\ControlFlowTok{if}\NormalTok{ (}\SpecialCharTok{!}\FunctionTok{require}\NormalTok{(}\StringTok{"dplyr"}\NormalTok{)) }\FunctionTok{install.packages}\NormalTok{(}\StringTok{"dplyr"}\NormalTok{)}
\end{Highlighting}
\end{Shaded}

\begin{verbatim}
Загрузка требуемого пакета: dplyr
\end{verbatim}

\begin{verbatim}

Присоединяю пакет: 'dplyr'
\end{verbatim}

\begin{verbatim}
Следующие объекты скрыты от 'package:stats':

    filter, lag
\end{verbatim}

\begin{verbatim}
Следующие объекты скрыты от 'package:base':

    intersect, setdiff, setequal, union
\end{verbatim}

\begin{Shaded}
\begin{Highlighting}[]
\ControlFlowTok{if}\NormalTok{ (}\SpecialCharTok{!}\FunctionTok{require}\NormalTok{(}\StringTok{"tidyr"}\NormalTok{)) }\FunctionTok{install.packages}\NormalTok{(}\StringTok{"tidyr"}\NormalTok{)}
\end{Highlighting}
\end{Shaded}

\begin{verbatim}
Загрузка требуемого пакета: tidyr
\end{verbatim}

\begin{Shaded}
\begin{Highlighting}[]
\CommentTok{\# Загружаем пакеты}
\FunctionTok{library}\NormalTok{(ggplot2)}
\FunctionTok{library}\NormalTok{(dplyr)}
\FunctionTok{library}\NormalTok{(tidyr)}

\CommentTok{\# {-}{-}{-}{-}{-}{-}{-}{-}{-}{-}{-}{-}{-}{-}{-}{-}{-}{-}{-}{-}{-}{-}{-}{-}{-} 2. ЗАГРУЗКА ДАННЫХ {-}{-}{-}{-}{-}{-}{-}{-}{-}{-}{-}{-}{-}{-}{-}{-}{-}{-}{-}{-}{-}{-}{-}{-}{-}{-}{-}}

\CommentTok{\# Вектор лет наблюдений}
\NormalTok{Year }\OtherTok{\textless{}{-}} \DecValTok{2005}\SpecialCharTok{:}\DecValTok{2024}

\CommentTok{\# Данные по вылову (тыс. тонн)}
\NormalTok{Catch }\OtherTok{\textless{}{-}} \FunctionTok{c}\NormalTok{(}\DecValTok{5}\NormalTok{, }\DecValTok{7}\NormalTok{, }\DecValTok{6}\NormalTok{, }\DecValTok{10}\NormalTok{, }\DecValTok{14}\NormalTok{, }\DecValTok{25}\NormalTok{, }\DecValTok{28}\NormalTok{, }\DecValTok{30}\NormalTok{, }\DecValTok{32}\NormalTok{, }\DecValTok{35}\NormalTok{, }\DecValTok{25}\NormalTok{, }\DecValTok{20}\NormalTok{, }\DecValTok{15}\NormalTok{, }\DecValTok{12}\NormalTok{, }\DecValTok{10}\NormalTok{, }\DecValTok{12}\NormalTok{, }\DecValTok{10}\NormalTok{, }\DecValTok{13}\NormalTok{, }\DecValTok{11}\NormalTok{, }\DecValTok{12}\NormalTok{)}

\CommentTok{\# Создаем график динамики вылова}
\NormalTok{catch\_df }\OtherTok{\textless{}{-}} \FunctionTok{data.frame}\NormalTok{(}\AttributeTok{Year =}\NormalTok{ Year, }\AttributeTok{Catch =}\NormalTok{ Catch)}
\FunctionTok{ggplot}\NormalTok{(catch\_df, }\FunctionTok{aes}\NormalTok{(}\AttributeTok{x =}\NormalTok{ Year, }\AttributeTok{y =}\NormalTok{ Catch)) }\SpecialCharTok{+}
  \FunctionTok{geom\_bar}\NormalTok{(}\AttributeTok{stat =} \StringTok{"identity"}\NormalTok{, }\AttributeTok{fill =} \StringTok{"steelblue"}\NormalTok{, }\AttributeTok{alpha =} \FloatTok{0.7}\NormalTok{) }\SpecialCharTok{+}
  \FunctionTok{geom\_line}\NormalTok{(}\AttributeTok{color =} \StringTok{"red"}\NormalTok{, }\AttributeTok{linewidth =} \DecValTok{1}\NormalTok{) }\SpecialCharTok{+}
  \FunctionTok{labs}\NormalTok{(}\AttributeTok{title =} \StringTok{"Динамика вылова водных биоресурсов (2005{-}2024)"}\NormalTok{,}
       \AttributeTok{y =} \StringTok{"Вылов, тыс. тонн"}\NormalTok{, }\AttributeTok{x =} \StringTok{"Год"}\NormalTok{) }\SpecialCharTok{+}
  \FunctionTok{theme\_minimal}\NormalTok{() }\SpecialCharTok{+}
  \FunctionTok{theme}\NormalTok{(}\AttributeTok{plot.title =} \FunctionTok{element\_text}\NormalTok{(}\AttributeTok{hjust =} \FloatTok{0.5}\NormalTok{))}
\end{Highlighting}
\end{Shaded}

\begin{verbatim}
Warning in grid.Call(C_textBounds, as.graphicsAnnot(x$label), x$x, x$y, :
неизвестна ширина символа 0xc2 в кодировке CP1251
\end{verbatim}

\begin{verbatim}
Warning in grid.Call(C_textBounds, as.graphicsAnnot(x$label), x$x, x$y, :
неизвестна ширина символа 0xfb в кодировке CP1251
\end{verbatim}

\begin{verbatim}
Warning in grid.Call(C_textBounds, as.graphicsAnnot(x$label), x$x, x$y, :
неизвестна ширина символа 0xeb в кодировке CP1251
\end{verbatim}

\begin{verbatim}
Warning in grid.Call(C_textBounds, as.graphicsAnnot(x$label), x$x, x$y, :
неизвестна ширина символа 0xee в кодировке CP1251
\end{verbatim}

\begin{verbatim}
Warning in grid.Call(C_textBounds, as.graphicsAnnot(x$label), x$x, x$y, :
неизвестна ширина символа 0xe2 в кодировке CP1251
\end{verbatim}

\begin{verbatim}
Warning in grid.Call(C_textBounds, as.graphicsAnnot(x$label), x$x, x$y, :
неизвестна ширина символа 0xf2 в кодировке CP1251
\end{verbatim}

\begin{verbatim}
Warning in grid.Call(C_textBounds, as.graphicsAnnot(x$label), x$x, x$y, :
неизвестна ширина символа 0xfb в кодировке CP1251
\end{verbatim}

\begin{verbatim}
Warning in grid.Call(C_textBounds, as.graphicsAnnot(x$label), x$x, x$y, :
неизвестна ширина символа 0xf1 в кодировке CP1251
\end{verbatim}

\begin{verbatim}
Warning in grid.Call(C_textBounds, as.graphicsAnnot(x$label), x$x, x$y, :
неизвестна ширина символа 0xf2 в кодировке CP1251
\end{verbatim}

\begin{verbatim}
Warning in grid.Call(C_textBounds, as.graphicsAnnot(x$label), x$x, x$y, :
неизвестна ширина символа 0xee в кодировке CP1251
\end{verbatim}

\begin{verbatim}
Warning in grid.Call(C_textBounds, as.graphicsAnnot(x$label), x$x, x$y, :
неизвестна ширина символа 0xed в кодировке CP1251
Warning in grid.Call(C_textBounds, as.graphicsAnnot(x$label), x$x, x$y, :
неизвестна ширина символа 0xed в кодировке CP1251
\end{verbatim}

\begin{verbatim}
Warning in grid.Call(C_textBounds, as.graphicsAnnot(x$label), x$x, x$y, :
неизвестна ширина символа 0xc4 в кодировке CP1251
\end{verbatim}

\begin{verbatim}
Warning in grid.Call(C_textBounds, as.graphicsAnnot(x$label), x$x, x$y, :
неизвестна ширина символа 0xe8 в кодировке CP1251
\end{verbatim}

\begin{verbatim}
Warning in grid.Call(C_textBounds, as.graphicsAnnot(x$label), x$x, x$y, :
неизвестна ширина символа 0xed в кодировке CP1251
\end{verbatim}

\begin{verbatim}
Warning in grid.Call(C_textBounds, as.graphicsAnnot(x$label), x$x, x$y, :
неизвестна ширина символа 0xe0 в кодировке CP1251
\end{verbatim}

\begin{verbatim}
Warning in grid.Call(C_textBounds, as.graphicsAnnot(x$label), x$x, x$y, :
неизвестна ширина символа 0xec в кодировке CP1251
\end{verbatim}

\begin{verbatim}
Warning in grid.Call(C_textBounds, as.graphicsAnnot(x$label), x$x, x$y, :
неизвестна ширина символа 0xe8 в кодировке CP1251
\end{verbatim}

\begin{verbatim}
Warning in grid.Call(C_textBounds, as.graphicsAnnot(x$label), x$x, x$y, :
неизвестна ширина символа 0xea в кодировке CP1251
\end{verbatim}

\begin{verbatim}
Warning in grid.Call(C_textBounds, as.graphicsAnnot(x$label), x$x, x$y, :
неизвестна ширина символа 0xe0 в кодировке CP1251
\end{verbatim}

\begin{verbatim}
Warning in grid.Call(C_textBounds, as.graphicsAnnot(x$label), x$x, x$y, :
неизвестна ширина символа 0xe2 в кодировке CP1251
\end{verbatim}

\begin{verbatim}
Warning in grid.Call(C_textBounds, as.graphicsAnnot(x$label), x$x, x$y, :
неизвестна ширина символа 0xfb в кодировке CP1251
\end{verbatim}

\begin{verbatim}
Warning in grid.Call(C_textBounds, as.graphicsAnnot(x$label), x$x, x$y, :
неизвестна ширина символа 0xeb в кодировке CP1251
\end{verbatim}

\begin{verbatim}
Warning in grid.Call(C_textBounds, as.graphicsAnnot(x$label), x$x, x$y, :
неизвестна ширина символа 0xee в кодировке CP1251
\end{verbatim}

\begin{verbatim}
Warning in grid.Call(C_textBounds, as.graphicsAnnot(x$label), x$x, x$y, :
неизвестна ширина символа 0xe2 в кодировке CP1251
\end{verbatim}

\begin{verbatim}
Warning in grid.Call(C_textBounds, as.graphicsAnnot(x$label), x$x, x$y, :
неизвестна ширина символа 0xe0 в кодировке CP1251
\end{verbatim}

\begin{verbatim}
Warning in grid.Call(C_textBounds, as.graphicsAnnot(x$label), x$x, x$y, :
неизвестна ширина символа 0xe2 в кодировке CP1251
\end{verbatim}

\begin{verbatim}
Warning in grid.Call(C_textBounds, as.graphicsAnnot(x$label), x$x, x$y, :
неизвестна ширина символа 0xee в кодировке CP1251
\end{verbatim}

\begin{verbatim}
Warning in grid.Call(C_textBounds, as.graphicsAnnot(x$label), x$x, x$y, :
неизвестна ширина символа 0xe4 в кодировке CP1251
\end{verbatim}

\begin{verbatim}
Warning in grid.Call(C_textBounds, as.graphicsAnnot(x$label), x$x, x$y, :
неизвестна ширина символа 0xed в кодировке CP1251
\end{verbatim}

\begin{verbatim}
Warning in grid.Call(C_textBounds, as.graphicsAnnot(x$label), x$x, x$y, :
неизвестна ширина символа 0xfb в кодировке CP1251
\end{verbatim}

\begin{verbatim}
Warning in grid.Call(C_textBounds, as.graphicsAnnot(x$label), x$x, x$y, :
неизвестна ширина символа 0xf5 в кодировке CP1251
\end{verbatim}

\begin{verbatim}
Warning in grid.Call(C_textBounds, as.graphicsAnnot(x$label), x$x, x$y, :
неизвестна ширина символа 0xe1 в кодировке CP1251
\end{verbatim}

\begin{verbatim}
Warning in grid.Call(C_textBounds, as.graphicsAnnot(x$label), x$x, x$y, :
неизвестна ширина символа 0xe8 в кодировке CP1251
\end{verbatim}

\begin{verbatim}
Warning in grid.Call(C_textBounds, as.graphicsAnnot(x$label), x$x, x$y, :
неизвестна ширина символа 0xee в кодировке CP1251
\end{verbatim}

\begin{verbatim}
Warning in grid.Call(C_textBounds, as.graphicsAnnot(x$label), x$x, x$y, :
неизвестна ширина символа 0xf0 в кодировке CP1251
\end{verbatim}

\begin{verbatim}
Warning in grid.Call(C_textBounds, as.graphicsAnnot(x$label), x$x, x$y, :
неизвестна ширина символа 0xe5 в кодировке CP1251
\end{verbatim}

\begin{verbatim}
Warning in grid.Call(C_textBounds, as.graphicsAnnot(x$label), x$x, x$y, :
неизвестна ширина символа 0xf1 в кодировке CP1251
\end{verbatim}

\begin{verbatim}
Warning in grid.Call(C_textBounds, as.graphicsAnnot(x$label), x$x, x$y, :
неизвестна ширина символа 0xf3 в кодировке CP1251
\end{verbatim}

\begin{verbatim}
Warning in grid.Call(C_textBounds, as.graphicsAnnot(x$label), x$x, x$y, :
неизвестна ширина символа 0xf0 в кодировке CP1251
\end{verbatim}

\begin{verbatim}
Warning in grid.Call(C_textBounds, as.graphicsAnnot(x$label), x$x, x$y, :
неизвестна ширина символа 0xf1 в кодировке CP1251
\end{verbatim}

\begin{verbatim}
Warning in grid.Call(C_textBounds, as.graphicsAnnot(x$label), x$x, x$y, :
неизвестна ширина символа 0xee в кодировке CP1251
\end{verbatim}

\begin{verbatim}
Warning in grid.Call(C_textBounds, as.graphicsAnnot(x$label), x$x, x$y, :
неизвестна ширина символа 0xe2 в кодировке CP1251
\end{verbatim}

\begin{verbatim}
Warning in grid.Call(C_textBounds, as.graphicsAnnot(x$label), x$x, x$y, :
неизвестна ширина символа 0xc3 в кодировке CP1251
\end{verbatim}

\begin{verbatim}
Warning in grid.Call(C_textBounds, as.graphicsAnnot(x$label), x$x, x$y, :
неизвестна ширина символа 0xee в кодировке CP1251
\end{verbatim}

\begin{verbatim}
Warning in grid.Call(C_textBounds, as.graphicsAnnot(x$label), x$x, x$y, :
неизвестна ширина символа 0xe4 в кодировке CP1251
\end{verbatim}

\begin{verbatim}
Warning in grid.Call.graphics(C_text, as.graphicsAnnot(x$label), x$x, x$y, :
неизвестна ширина символа 0xc3 в кодировке CP1251
\end{verbatim}

\begin{verbatim}
Warning in grid.Call.graphics(C_text, as.graphicsAnnot(x$label), x$x, x$y, :
неизвестна ширина символа 0xee в кодировке CP1251
\end{verbatim}

\begin{verbatim}
Warning in grid.Call.graphics(C_text, as.graphicsAnnot(x$label), x$x, x$y, :
неизвестна ширина символа 0xe4 в кодировке CP1251
\end{verbatim}

\begin{verbatim}
Warning in grid.Call.graphics(C_text, as.graphicsAnnot(x$label), x$x, x$y, :
неизвестна ширина символа 0xc2 в кодировке CP1251
\end{verbatim}

\begin{verbatim}
Warning in grid.Call.graphics(C_text, as.graphicsAnnot(x$label), x$x, x$y, :
неизвестна ширина символа 0xfb в кодировке CP1251
\end{verbatim}

\begin{verbatim}
Warning in grid.Call.graphics(C_text, as.graphicsAnnot(x$label), x$x, x$y, :
неизвестна ширина символа 0xeb в кодировке CP1251
\end{verbatim}

\begin{verbatim}
Warning in grid.Call.graphics(C_text, as.graphicsAnnot(x$label), x$x, x$y, :
неизвестна ширина символа 0xee в кодировке CP1251
\end{verbatim}

\begin{verbatim}
Warning in grid.Call.graphics(C_text, as.graphicsAnnot(x$label), x$x, x$y, :
неизвестна ширина символа 0xe2 в кодировке CP1251
\end{verbatim}

\begin{verbatim}
Warning in grid.Call.graphics(C_text, as.graphicsAnnot(x$label), x$x, x$y, :
неизвестна ширина символа 0xf2 в кодировке CP1251
\end{verbatim}

\begin{verbatim}
Warning in grid.Call.graphics(C_text, as.graphicsAnnot(x$label), x$x, x$y, :
неизвестна ширина символа 0xfb в кодировке CP1251
\end{verbatim}

\begin{verbatim}
Warning in grid.Call.graphics(C_text, as.graphicsAnnot(x$label), x$x, x$y, :
неизвестна ширина символа 0xf1 в кодировке CP1251
\end{verbatim}

\begin{verbatim}
Warning in grid.Call.graphics(C_text, as.graphicsAnnot(x$label), x$x, x$y, :
неизвестна ширина символа 0xf2 в кодировке CP1251
\end{verbatim}

\begin{verbatim}
Warning in grid.Call.graphics(C_text, as.graphicsAnnot(x$label), x$x, x$y, :
неизвестна ширина символа 0xee в кодировке CP1251
\end{verbatim}

\begin{verbatim}
Warning in grid.Call.graphics(C_text, as.graphicsAnnot(x$label), x$x, x$y, :
неизвестна ширина символа 0xed в кодировке CP1251
Warning in grid.Call.graphics(C_text, as.graphicsAnnot(x$label), x$x, x$y, :
неизвестна ширина символа 0xed в кодировке CP1251
\end{verbatim}

\begin{verbatim}
Warning in grid.Call.graphics(C_text, as.graphicsAnnot(x$label), x$x, x$y, :
неизвестна ширина символа 0xc4 в кодировке CP1251
\end{verbatim}

\begin{verbatim}
Warning in grid.Call.graphics(C_text, as.graphicsAnnot(x$label), x$x, x$y, :
неизвестна ширина символа 0xe8 в кодировке CP1251
\end{verbatim}

\begin{verbatim}
Warning in grid.Call.graphics(C_text, as.graphicsAnnot(x$label), x$x, x$y, :
неизвестна ширина символа 0xed в кодировке CP1251
\end{verbatim}

\begin{verbatim}
Warning in grid.Call.graphics(C_text, as.graphicsAnnot(x$label), x$x, x$y, :
неизвестна ширина символа 0xe0 в кодировке CP1251
\end{verbatim}

\begin{verbatim}
Warning in grid.Call.graphics(C_text, as.graphicsAnnot(x$label), x$x, x$y, :
неизвестна ширина символа 0xec в кодировке CP1251
\end{verbatim}

\begin{verbatim}
Warning in grid.Call.graphics(C_text, as.graphicsAnnot(x$label), x$x, x$y, :
неизвестна ширина символа 0xe8 в кодировке CP1251
\end{verbatim}

\begin{verbatim}
Warning in grid.Call.graphics(C_text, as.graphicsAnnot(x$label), x$x, x$y, :
неизвестна ширина символа 0xea в кодировке CP1251
\end{verbatim}

\begin{verbatim}
Warning in grid.Call.graphics(C_text, as.graphicsAnnot(x$label), x$x, x$y, :
неизвестна ширина символа 0xe0 в кодировке CP1251
\end{verbatim}

\begin{verbatim}
Warning in grid.Call.graphics(C_text, as.graphicsAnnot(x$label), x$x, x$y, :
неизвестна ширина символа 0xe2 в кодировке CP1251
\end{verbatim}

\begin{verbatim}
Warning in grid.Call.graphics(C_text, as.graphicsAnnot(x$label), x$x, x$y, :
неизвестна ширина символа 0xfb в кодировке CP1251
\end{verbatim}

\begin{verbatim}
Warning in grid.Call.graphics(C_text, as.graphicsAnnot(x$label), x$x, x$y, :
неизвестна ширина символа 0xeb в кодировке CP1251
\end{verbatim}

\begin{verbatim}
Warning in grid.Call.graphics(C_text, as.graphicsAnnot(x$label), x$x, x$y, :
неизвестна ширина символа 0xee в кодировке CP1251
\end{verbatim}

\begin{verbatim}
Warning in grid.Call.graphics(C_text, as.graphicsAnnot(x$label), x$x, x$y, :
неизвестна ширина символа 0xe2 в кодировке CP1251
\end{verbatim}

\begin{verbatim}
Warning in grid.Call.graphics(C_text, as.graphicsAnnot(x$label), x$x, x$y, :
неизвестна ширина символа 0xe0 в кодировке CP1251
\end{verbatim}

\begin{verbatim}
Warning in grid.Call.graphics(C_text, as.graphicsAnnot(x$label), x$x, x$y, :
неизвестна ширина символа 0xe2 в кодировке CP1251
\end{verbatim}

\begin{verbatim}
Warning in grid.Call.graphics(C_text, as.graphicsAnnot(x$label), x$x, x$y, :
неизвестна ширина символа 0xee в кодировке CP1251
\end{verbatim}

\begin{verbatim}
Warning in grid.Call.graphics(C_text, as.graphicsAnnot(x$label), x$x, x$y, :
неизвестна ширина символа 0xe4 в кодировке CP1251
\end{verbatim}

\begin{verbatim}
Warning in grid.Call.graphics(C_text, as.graphicsAnnot(x$label), x$x, x$y, :
неизвестна ширина символа 0xed в кодировке CP1251
\end{verbatim}

\begin{verbatim}
Warning in grid.Call.graphics(C_text, as.graphicsAnnot(x$label), x$x, x$y, :
неизвестна ширина символа 0xfb в кодировке CP1251
\end{verbatim}

\begin{verbatim}
Warning in grid.Call.graphics(C_text, as.graphicsAnnot(x$label), x$x, x$y, :
неизвестна ширина символа 0xf5 в кодировке CP1251
\end{verbatim}

\begin{verbatim}
Warning in grid.Call.graphics(C_text, as.graphicsAnnot(x$label), x$x, x$y, :
неизвестна ширина символа 0xe1 в кодировке CP1251
\end{verbatim}

\begin{verbatim}
Warning in grid.Call.graphics(C_text, as.graphicsAnnot(x$label), x$x, x$y, :
неизвестна ширина символа 0xe8 в кодировке CP1251
\end{verbatim}

\begin{verbatim}
Warning in grid.Call.graphics(C_text, as.graphicsAnnot(x$label), x$x, x$y, :
неизвестна ширина символа 0xee в кодировке CP1251
\end{verbatim}

\begin{verbatim}
Warning in grid.Call.graphics(C_text, as.graphicsAnnot(x$label), x$x, x$y, :
неизвестна ширина символа 0xf0 в кодировке CP1251
\end{verbatim}

\begin{verbatim}
Warning in grid.Call.graphics(C_text, as.graphicsAnnot(x$label), x$x, x$y, :
неизвестна ширина символа 0xe5 в кодировке CP1251
\end{verbatim}

\begin{verbatim}
Warning in grid.Call.graphics(C_text, as.graphicsAnnot(x$label), x$x, x$y, :
неизвестна ширина символа 0xf1 в кодировке CP1251
\end{verbatim}

\begin{verbatim}
Warning in grid.Call.graphics(C_text, as.graphicsAnnot(x$label), x$x, x$y, :
неизвестна ширина символа 0xf3 в кодировке CP1251
\end{verbatim}

\begin{verbatim}
Warning in grid.Call.graphics(C_text, as.graphicsAnnot(x$label), x$x, x$y, :
неизвестна ширина символа 0xf0 в кодировке CP1251
\end{verbatim}

\begin{verbatim}
Warning in grid.Call.graphics(C_text, as.graphicsAnnot(x$label), x$x, x$y, :
неизвестна ширина символа 0xf1 в кодировке CP1251
\end{verbatim}

\begin{verbatim}
Warning in grid.Call.graphics(C_text, as.graphicsAnnot(x$label), x$x, x$y, :
неизвестна ширина символа 0xee в кодировке CP1251
\end{verbatim}

\begin{verbatim}
Warning in grid.Call.graphics(C_text, as.graphicsAnnot(x$label), x$x, x$y, :
неизвестна ширина символа 0xe2 в кодировке CP1251
\end{verbatim}

\pandocbounded{\includegraphics[keepaspectratio]{chapter16_files/figure-pdf/unnamed-chunk-1-1.pdf}}

\begin{Shaded}
\begin{Highlighting}[]
\CommentTok{\# {-}{-}{-}{-}{-}{-}{-}{-}{-}{-}{-}{-}{-}{-}{-}{-}{-}{-}{-}{-}{-}{-}{-}{-}{-} 3. РЕАЛИЗАЦИЯ МЕТОДА CATCH{-}MSY {-}{-}{-}{-}{-}{-}{-}{-}{-}{-}{-}{-}{-}{-}{-}{-}{-}{-}{-}{-}{-}{-}{-}{-}{-}{-}{-}}

\CommentTok{\# Функция для реализации метода Catch{-}MSY}
\NormalTok{catch\_msy\_analysis }\OtherTok{\textless{}{-}} \ControlFlowTok{function}\NormalTok{(catch, }\AttributeTok{n\_iter =} \DecValTok{100000}\NormalTok{, }\AttributeTok{r\_prior =} \FunctionTok{c}\NormalTok{(}\FloatTok{0.01}\NormalTok{, }\FloatTok{1.5}\NormalTok{)) \{}
\NormalTok{  n\_years }\OtherTok{\textless{}{-}} \FunctionTok{length}\NormalTok{(catch)}
  
  \CommentTok{\# Приоры для параметров}
\NormalTok{  r }\OtherTok{\textless{}{-}} \FunctionTok{runif}\NormalTok{(n\_iter, r\_prior[}\DecValTok{1}\NormalTok{], r\_prior[}\DecValTok{2}\NormalTok{])}
\NormalTok{  k }\OtherTok{\textless{}{-}} \FunctionTok{runif}\NormalTok{(n\_iter, }\FunctionTok{max}\NormalTok{(catch) }\SpecialCharTok{*} \DecValTok{3}\NormalTok{, }\FunctionTok{max}\NormalTok{(catch) }\SpecialCharTok{*} \DecValTok{50}\NormalTok{)}
\NormalTok{  msy }\OtherTok{\textless{}{-}}\NormalTok{ r }\SpecialCharTok{*}\NormalTok{ k }\SpecialCharTok{/} \DecValTok{4}  \CommentTok{\# MSY = r*K/4 для модели Шефера}
  
  \CommentTok{\# Массивы для хранения результатов}
\NormalTok{  viable }\OtherTok{\textless{}{-}} \FunctionTok{rep}\NormalTok{(}\ConstantTok{FALSE}\NormalTok{, n\_iter)}
\NormalTok{  b\_msy }\OtherTok{\textless{}{-}}\NormalTok{ k }\SpecialCharTok{/} \DecValTok{2}  \CommentTok{\# Bmsy = K/2 для модели Шефера}
  
  \CommentTok{\# Проверяем каждую комбинацию параметров}
  \ControlFlowTok{for}\NormalTok{ (i }\ControlFlowTok{in} \DecValTok{1}\SpecialCharTok{:}\NormalTok{n\_iter) \{}
\NormalTok{    biomass }\OtherTok{\textless{}{-}} \FunctionTok{numeric}\NormalTok{(n\_years }\SpecialCharTok{+} \DecValTok{1}\NormalTok{)}
\NormalTok{    biomass[}\DecValTok{1}\NormalTok{] }\OtherTok{\textless{}{-}}\NormalTok{ k[i]  }\CommentTok{\# Начальная биомасса = K}
    
    \CommentTok{\# Моделируем динамику биомассы}
    \ControlFlowTok{for}\NormalTok{ (t }\ControlFlowTok{in} \DecValTok{1}\SpecialCharTok{:}\NormalTok{n\_years) \{}
\NormalTok{      surplus\_production }\OtherTok{\textless{}{-}}\NormalTok{ r[i] }\SpecialCharTok{*}\NormalTok{ biomass[t] }\SpecialCharTok{*}\NormalTok{ (}\DecValTok{1} \SpecialCharTok{{-}}\NormalTok{ biomass[t] }\SpecialCharTok{/}\NormalTok{ k[i])}
\NormalTok{      biomass[t }\SpecialCharTok{+} \DecValTok{1}\NormalTok{] }\OtherTok{\textless{}{-}} \FunctionTok{max}\NormalTok{(}\FloatTok{0.001} \SpecialCharTok{*}\NormalTok{ k[i], biomass[t] }\SpecialCharTok{+}\NormalTok{ surplus\_production }\SpecialCharTok{{-}}\NormalTok{ catch[t])}
\NormalTok{    \}}
    
    \CommentTok{\# Проверяем условия жизнеспособности}
    \ControlFlowTok{if}\NormalTok{ (}\FunctionTok{all}\NormalTok{(biomass }\SpecialCharTok{\textgreater{}} \FloatTok{0.001} \SpecialCharTok{*}\NormalTok{ k[i]) }\SpecialCharTok{\&\&} \FunctionTok{all}\NormalTok{(biomass }\SpecialCharTok{\textless{}} \FloatTok{1.1} \SpecialCharTok{*}\NormalTok{ k[i]) }\SpecialCharTok{\&\&} 
\NormalTok{        biomass[n\_years }\SpecialCharTok{+} \DecValTok{1}\NormalTok{] }\SpecialCharTok{\textgreater{}} \FloatTok{0.2} \SpecialCharTok{*}\NormalTok{ k[i] }\SpecialCharTok{\&\&}\NormalTok{ biomass[n\_years }\SpecialCharTok{+} \DecValTok{1}\NormalTok{] }\SpecialCharTok{\textless{}} \FloatTok{0.8} \SpecialCharTok{*}\NormalTok{ k[i]) \{}
\NormalTok{      viable[i] }\OtherTok{\textless{}{-}} \ConstantTok{TRUE}
\NormalTok{    \}}
\NormalTok{  \}}
  
  \CommentTok{\# Возвращаем жизнеспособные комбинации параметров}
  \FunctionTok{list}\NormalTok{(}
    \AttributeTok{r =}\NormalTok{ r[viable],}
    \AttributeTok{k =}\NormalTok{ k[viable],}
    \AttributeTok{msy =}\NormalTok{ msy[viable],}
    \AttributeTok{b\_msy =}\NormalTok{ b\_msy[viable],}
    \AttributeTok{depletion =}\NormalTok{ biomass[n\_years }\SpecialCharTok{+} \DecValTok{1}\NormalTok{] }\SpecialCharTok{/}\NormalTok{ k[viable]}
\NormalTok{  )}
\NormalTok{\}}

\CommentTok{\# Применяем метод Catch{-}MSY}
\FunctionTok{set.seed}\NormalTok{(}\DecValTok{123}\NormalTok{)  }\CommentTok{\# Для воспроизводимости результатов}
\NormalTok{msy\_results }\OtherTok{\textless{}{-}} \FunctionTok{catch\_msy\_analysis}\NormalTok{(Catch)}

\CommentTok{\# {-}{-}{-}{-}{-}{-}{-}{-}{-}{-}{-}{-}{-}{-}{-}{-}{-}{-}{-}{-}{-}{-}{-}{-}{-} 4. АНАЛИЗ РЕЗУЛЬТАТОВ CATCH{-}MSY {-}{-}{-}{-}{-}{-}{-}{-}{-}{-}{-}{-}{-}{-}{-}{-}{-}{-}{-}{-}{-}{-}{-}{-}{-}{-}{-}}

\CommentTok{\# Основные статистики}
\NormalTok{msy\_summary }\OtherTok{\textless{}{-}} \FunctionTok{data.frame}\NormalTok{(}
  \AttributeTok{Parameter =} \FunctionTok{c}\NormalTok{(}\StringTok{"r"}\NormalTok{, }\StringTok{"K"}\NormalTok{, }\StringTok{"MSY"}\NormalTok{, }\StringTok{"Bmsy"}\NormalTok{),}
  \AttributeTok{Median =} \FunctionTok{c}\NormalTok{(}
    \FunctionTok{median}\NormalTok{(msy\_results}\SpecialCharTok{$}\NormalTok{r),}
    \FunctionTok{median}\NormalTok{(msy\_results}\SpecialCharTok{$}\NormalTok{k),}
    \FunctionTok{median}\NormalTok{(msy\_results}\SpecialCharTok{$}\NormalTok{msy),}
    \FunctionTok{median}\NormalTok{(msy\_results}\SpecialCharTok{$}\NormalTok{b\_msy)}
\NormalTok{  ),}
  \AttributeTok{Mean =} \FunctionTok{c}\NormalTok{(}
    \FunctionTok{mean}\NormalTok{(msy\_results}\SpecialCharTok{$}\NormalTok{r),}
    \FunctionTok{mean}\NormalTok{(msy\_results}\SpecialCharTok{$}\NormalTok{k),}
    \FunctionTok{mean}\NormalTok{(msy\_results}\SpecialCharTok{$}\NormalTok{msy),}
    \FunctionTok{mean}\NormalTok{(msy\_results}\SpecialCharTok{$}\NormalTok{b\_msy)}
\NormalTok{  ),}
  \AttributeTok{SD =} \FunctionTok{c}\NormalTok{(}
    \FunctionTok{sd}\NormalTok{(msy\_results}\SpecialCharTok{$}\NormalTok{r),}
    \FunctionTok{sd}\NormalTok{(msy\_results}\SpecialCharTok{$}\NormalTok{k),}
    \FunctionTok{sd}\NormalTok{(msy\_results}\SpecialCharTok{$}\NormalTok{msy),}
    \FunctionTok{sd}\NormalTok{(msy\_results}\SpecialCharTok{$}\NormalTok{b\_msy)}
\NormalTok{  )}
\NormalTok{)}

\FunctionTok{print}\NormalTok{(}\StringTok{"Результаты анализа Catch{-}MSY:"}\NormalTok{)}
\end{Highlighting}
\end{Shaded}

\begin{verbatim}
[1] "Результаты анализа Catch-MSY:"
\end{verbatim}

\begin{Shaded}
\begin{Highlighting}[]
\FunctionTok{print}\NormalTok{(msy\_summary)}
\end{Highlighting}
\end{Shaded}

\begin{verbatim}
  Parameter       Median        Mean          SD
1         r   0.09671814   0.1409528   0.1330245
2         K 487.67637851 555.3402905 292.9906278
3       MSY  13.08343781  12.7539980   5.7560545
4      Bmsy 243.83818925 277.6701453 146.4953139
\end{verbatim}

\begin{Shaded}
\begin{Highlighting}[]
\CommentTok{\# Визуализация распределения параметров}
\FunctionTok{par}\NormalTok{(}\AttributeTok{mfrow =} \FunctionTok{c}\NormalTok{(}\DecValTok{2}\NormalTok{, }\DecValTok{2}\NormalTok{))}
\FunctionTok{hist}\NormalTok{(msy\_results}\SpecialCharTok{$}\NormalTok{r, }\AttributeTok{main =} \StringTok{"Распределение r"}\NormalTok{, }\AttributeTok{xlab =} \StringTok{"r"}\NormalTok{, }\AttributeTok{col =} \StringTok{"lightblue"}\NormalTok{)}
\end{Highlighting}
\end{Shaded}

\begin{verbatim}
Warning in title(main = main, sub = sub, xlab = xlab, ylab = ylab, ...):
неизвестна ширина символа 0xd0 в кодировке CP1251
\end{verbatim}

\begin{verbatim}
Warning in title(main = main, sub = sub, xlab = xlab, ylab = ylab, ...):
неизвестна ширина символа 0xe0 в кодировке CP1251
\end{verbatim}

\begin{verbatim}
Warning in title(main = main, sub = sub, xlab = xlab, ylab = ylab, ...):
неизвестна ширина символа 0xf1 в кодировке CP1251
\end{verbatim}

\begin{verbatim}
Warning in title(main = main, sub = sub, xlab = xlab, ylab = ylab, ...):
неизвестна ширина символа 0xef в кодировке CP1251
\end{verbatim}

\begin{verbatim}
Warning in title(main = main, sub = sub, xlab = xlab, ylab = ylab, ...):
неизвестна ширина символа 0xf0 в кодировке CP1251
\end{verbatim}

\begin{verbatim}
Warning in title(main = main, sub = sub, xlab = xlab, ylab = ylab, ...):
неизвестна ширина символа 0xe5 в кодировке CP1251
\end{verbatim}

\begin{verbatim}
Warning in title(main = main, sub = sub, xlab = xlab, ylab = ylab, ...):
неизвестна ширина символа 0xe4 в кодировке CP1251
\end{verbatim}

\begin{verbatim}
Warning in title(main = main, sub = sub, xlab = xlab, ylab = ylab, ...):
неизвестна ширина символа 0xe5 в кодировке CP1251
\end{verbatim}

\begin{verbatim}
Warning in title(main = main, sub = sub, xlab = xlab, ylab = ylab, ...):
неизвестна ширина символа 0xeb в кодировке CP1251
\end{verbatim}

\begin{verbatim}
Warning in title(main = main, sub = sub, xlab = xlab, ylab = ylab, ...):
неизвестна ширина символа 0xe5 в кодировке CP1251
\end{verbatim}

\begin{verbatim}
Warning in title(main = main, sub = sub, xlab = xlab, ylab = ylab, ...):
неизвестна ширина символа 0xed в кодировке CP1251
\end{verbatim}

\begin{verbatim}
Warning in title(main = main, sub = sub, xlab = xlab, ylab = ylab, ...):
неизвестна ширина символа 0xe8 в кодировке CP1251
\end{verbatim}

\begin{verbatim}
Warning in title(main = main, sub = sub, xlab = xlab, ylab = ylab, ...):
неизвестна ширина символа 0xe5 в кодировке CP1251
\end{verbatim}

\begin{Shaded}
\begin{Highlighting}[]
\FunctionTok{hist}\NormalTok{(msy\_results}\SpecialCharTok{$}\NormalTok{k, }\AttributeTok{main =} \StringTok{"Распределение K"}\NormalTok{, }\AttributeTok{xlab =} \StringTok{"K"}\NormalTok{, }\AttributeTok{col =} \StringTok{"lightgreen"}\NormalTok{)}
\end{Highlighting}
\end{Shaded}

\begin{verbatim}
Warning in title(main = main, sub = sub, xlab = xlab, ylab = ylab, ...):
неизвестна ширина символа 0xd0 в кодировке CP1251
\end{verbatim}

\begin{verbatim}
Warning in title(main = main, sub = sub, xlab = xlab, ylab = ylab, ...):
неизвестна ширина символа 0xe0 в кодировке CP1251
\end{verbatim}

\begin{verbatim}
Warning in title(main = main, sub = sub, xlab = xlab, ylab = ylab, ...):
неизвестна ширина символа 0xf1 в кодировке CP1251
\end{verbatim}

\begin{verbatim}
Warning in title(main = main, sub = sub, xlab = xlab, ylab = ylab, ...):
неизвестна ширина символа 0xef в кодировке CP1251
\end{verbatim}

\begin{verbatim}
Warning in title(main = main, sub = sub, xlab = xlab, ylab = ylab, ...):
неизвестна ширина символа 0xf0 в кодировке CP1251
\end{verbatim}

\begin{verbatim}
Warning in title(main = main, sub = sub, xlab = xlab, ylab = ylab, ...):
неизвестна ширина символа 0xe5 в кодировке CP1251
\end{verbatim}

\begin{verbatim}
Warning in title(main = main, sub = sub, xlab = xlab, ylab = ylab, ...):
неизвестна ширина символа 0xe4 в кодировке CP1251
\end{verbatim}

\begin{verbatim}
Warning in title(main = main, sub = sub, xlab = xlab, ylab = ylab, ...):
неизвестна ширина символа 0xe5 в кодировке CP1251
\end{verbatim}

\begin{verbatim}
Warning in title(main = main, sub = sub, xlab = xlab, ylab = ylab, ...):
неизвестна ширина символа 0xeb в кодировке CP1251
\end{verbatim}

\begin{verbatim}
Warning in title(main = main, sub = sub, xlab = xlab, ylab = ylab, ...):
неизвестна ширина символа 0xe5 в кодировке CP1251
\end{verbatim}

\begin{verbatim}
Warning in title(main = main, sub = sub, xlab = xlab, ylab = ylab, ...):
неизвестна ширина символа 0xed в кодировке CP1251
\end{verbatim}

\begin{verbatim}
Warning in title(main = main, sub = sub, xlab = xlab, ylab = ylab, ...):
неизвестна ширина символа 0xe8 в кодировке CP1251
\end{verbatim}

\begin{verbatim}
Warning in title(main = main, sub = sub, xlab = xlab, ylab = ylab, ...):
неизвестна ширина символа 0xe5 в кодировке CP1251
\end{verbatim}

\begin{Shaded}
\begin{Highlighting}[]
\FunctionTok{hist}\NormalTok{(msy\_results}\SpecialCharTok{$}\NormalTok{msy, }\AttributeTok{main =} \StringTok{"Распределение MSY"}\NormalTok{, }\AttributeTok{xlab =} \StringTok{"MSY"}\NormalTok{, }\AttributeTok{col =} \StringTok{"lightcoral"}\NormalTok{)}
\end{Highlighting}
\end{Shaded}

\begin{verbatim}
Warning in title(main = main, sub = sub, xlab = xlab, ylab = ylab, ...):
неизвестна ширина символа 0xd0 в кодировке CP1251
\end{verbatim}

\begin{verbatim}
Warning in title(main = main, sub = sub, xlab = xlab, ylab = ylab, ...):
неизвестна ширина символа 0xe0 в кодировке CP1251
\end{verbatim}

\begin{verbatim}
Warning in title(main = main, sub = sub, xlab = xlab, ylab = ylab, ...):
неизвестна ширина символа 0xf1 в кодировке CP1251
\end{verbatim}

\begin{verbatim}
Warning in title(main = main, sub = sub, xlab = xlab, ylab = ylab, ...):
неизвестна ширина символа 0xef в кодировке CP1251
\end{verbatim}

\begin{verbatim}
Warning in title(main = main, sub = sub, xlab = xlab, ylab = ylab, ...):
неизвестна ширина символа 0xf0 в кодировке CP1251
\end{verbatim}

\begin{verbatim}
Warning in title(main = main, sub = sub, xlab = xlab, ylab = ylab, ...):
неизвестна ширина символа 0xe5 в кодировке CP1251
\end{verbatim}

\begin{verbatim}
Warning in title(main = main, sub = sub, xlab = xlab, ylab = ylab, ...):
неизвестна ширина символа 0xe4 в кодировке CP1251
\end{verbatim}

\begin{verbatim}
Warning in title(main = main, sub = sub, xlab = xlab, ylab = ylab, ...):
неизвестна ширина символа 0xe5 в кодировке CP1251
\end{verbatim}

\begin{verbatim}
Warning in title(main = main, sub = sub, xlab = xlab, ylab = ylab, ...):
неизвестна ширина символа 0xeb в кодировке CP1251
\end{verbatim}

\begin{verbatim}
Warning in title(main = main, sub = sub, xlab = xlab, ylab = ylab, ...):
неизвестна ширина символа 0xe5 в кодировке CP1251
\end{verbatim}

\begin{verbatim}
Warning in title(main = main, sub = sub, xlab = xlab, ylab = ylab, ...):
неизвестна ширина символа 0xed в кодировке CP1251
\end{verbatim}

\begin{verbatim}
Warning in title(main = main, sub = sub, xlab = xlab, ylab = ylab, ...):
неизвестна ширина символа 0xe8 в кодировке CP1251
\end{verbatim}

\begin{verbatim}
Warning in title(main = main, sub = sub, xlab = xlab, ylab = ylab, ...):
неизвестна ширина символа 0xe5 в кодировке CP1251
\end{verbatim}

\begin{Shaded}
\begin{Highlighting}[]
\FunctionTok{hist}\NormalTok{(msy\_results}\SpecialCharTok{$}\NormalTok{depletion, }\AttributeTok{main =} \StringTok{"Распределение деплетированности"}\NormalTok{, }
     \AttributeTok{xlab =} \StringTok{"Bcurrent/K"}\NormalTok{, }\AttributeTok{col =} \StringTok{"lightyellow"}\NormalTok{)}
\end{Highlighting}
\end{Shaded}

\begin{verbatim}
Warning in title(main = main, sub = sub, xlab = xlab, ylab = ylab, ...):
неизвестна ширина символа 0xd0 в кодировке CP1251
\end{verbatim}

\begin{verbatim}
Warning in title(main = main, sub = sub, xlab = xlab, ylab = ylab, ...):
неизвестна ширина символа 0xe0 в кодировке CP1251
\end{verbatim}

\begin{verbatim}
Warning in title(main = main, sub = sub, xlab = xlab, ylab = ylab, ...):
неизвестна ширина символа 0xf1 в кодировке CP1251
\end{verbatim}

\begin{verbatim}
Warning in title(main = main, sub = sub, xlab = xlab, ylab = ylab, ...):
неизвестна ширина символа 0xef в кодировке CP1251
\end{verbatim}

\begin{verbatim}
Warning in title(main = main, sub = sub, xlab = xlab, ylab = ylab, ...):
неизвестна ширина символа 0xf0 в кодировке CP1251
\end{verbatim}

\begin{verbatim}
Warning in title(main = main, sub = sub, xlab = xlab, ylab = ylab, ...):
неизвестна ширина символа 0xe5 в кодировке CP1251
\end{verbatim}

\begin{verbatim}
Warning in title(main = main, sub = sub, xlab = xlab, ylab = ylab, ...):
неизвестна ширина символа 0xe4 в кодировке CP1251
\end{verbatim}

\begin{verbatim}
Warning in title(main = main, sub = sub, xlab = xlab, ylab = ylab, ...):
неизвестна ширина символа 0xe5 в кодировке CP1251
\end{verbatim}

\begin{verbatim}
Warning in title(main = main, sub = sub, xlab = xlab, ylab = ylab, ...):
неизвестна ширина символа 0xeb в кодировке CP1251
\end{verbatim}

\begin{verbatim}
Warning in title(main = main, sub = sub, xlab = xlab, ylab = ylab, ...):
неизвестна ширина символа 0xe5 в кодировке CP1251
\end{verbatim}

\begin{verbatim}
Warning in title(main = main, sub = sub, xlab = xlab, ylab = ylab, ...):
неизвестна ширина символа 0xed в кодировке CP1251
\end{verbatim}

\begin{verbatim}
Warning in title(main = main, sub = sub, xlab = xlab, ylab = ylab, ...):
неизвестна ширина символа 0xe8 в кодировке CP1251
\end{verbatim}

\begin{verbatim}
Warning in title(main = main, sub = sub, xlab = xlab, ylab = ylab, ...):
неизвестна ширина символа 0xe5 в кодировке CP1251
\end{verbatim}

\begin{verbatim}
Warning in title(main = main, sub = sub, xlab = xlab, ylab = ylab, ...):
неизвестна ширина символа 0xe4 в кодировке CP1251
\end{verbatim}

\begin{verbatim}
Warning in title(main = main, sub = sub, xlab = xlab, ylab = ylab, ...):
неизвестна ширина символа 0xe5 в кодировке CP1251
\end{verbatim}

\begin{verbatim}
Warning in title(main = main, sub = sub, xlab = xlab, ylab = ylab, ...):
неизвестна ширина символа 0xef в кодировке CP1251
\end{verbatim}

\begin{verbatim}
Warning in title(main = main, sub = sub, xlab = xlab, ylab = ylab, ...):
неизвестна ширина символа 0xeb в кодировке CP1251
\end{verbatim}

\begin{verbatim}
Warning in title(main = main, sub = sub, xlab = xlab, ylab = ylab, ...):
неизвестна ширина символа 0xe5 в кодировке CP1251
\end{verbatim}

\begin{verbatim}
Warning in title(main = main, sub = sub, xlab = xlab, ylab = ylab, ...):
неизвестна ширина символа 0xf2 в кодировке CP1251
\end{verbatim}

\begin{verbatim}
Warning in title(main = main, sub = sub, xlab = xlab, ylab = ylab, ...):
неизвестна ширина символа 0xe8 в кодировке CP1251
\end{verbatim}

\begin{verbatim}
Warning in title(main = main, sub = sub, xlab = xlab, ylab = ylab, ...):
неизвестна ширина символа 0xf0 в кодировке CP1251
\end{verbatim}

\begin{verbatim}
Warning in title(main = main, sub = sub, xlab = xlab, ylab = ylab, ...):
неизвестна ширина символа 0xee в кодировке CP1251
\end{verbatim}

\begin{verbatim}
Warning in title(main = main, sub = sub, xlab = xlab, ylab = ylab, ...):
неизвестна ширина символа 0xe2 в кодировке CP1251
\end{verbatim}

\begin{verbatim}
Warning in title(main = main, sub = sub, xlab = xlab, ylab = ylab, ...):
неизвестна ширина символа 0xe0 в кодировке CP1251
\end{verbatim}

\begin{verbatim}
Warning in title(main = main, sub = sub, xlab = xlab, ylab = ylab, ...):
неизвестна ширина символа 0xed в кодировке CP1251
Warning in title(main = main, sub = sub, xlab = xlab, ylab = ylab, ...):
неизвестна ширина символа 0xed в кодировке CP1251
\end{verbatim}

\begin{verbatim}
Warning in title(main = main, sub = sub, xlab = xlab, ylab = ylab, ...):
неизвестна ширина символа 0xee в кодировке CP1251
\end{verbatim}

\begin{verbatim}
Warning in title(main = main, sub = sub, xlab = xlab, ylab = ylab, ...):
неизвестна ширина символа 0xf1 в кодировке CP1251
\end{verbatim}

\begin{verbatim}
Warning in title(main = main, sub = sub, xlab = xlab, ylab = ylab, ...):
неизвестна ширина символа 0xf2 в кодировке CP1251
\end{verbatim}

\begin{verbatim}
Warning in title(main = main, sub = sub, xlab = xlab, ylab = ylab, ...):
неизвестна ширина символа 0xe8 в кодировке CP1251
\end{verbatim}

\pandocbounded{\includegraphics[keepaspectratio]{chapter16_files/figure-pdf/unnamed-chunk-1-2.pdf}}

\begin{Shaded}
\begin{Highlighting}[]
\CommentTok{\# Текущий статус запаса}
\NormalTok{current\_status }\OtherTok{\textless{}{-}} \FunctionTok{data.frame}\NormalTok{(}
  \AttributeTok{Metric =} \FunctionTok{c}\NormalTok{(}\StringTok{"Текущий вылов"}\NormalTok{, }\StringTok{"MSY"}\NormalTok{, }\StringTok{"Отношение вылова к MSY"}\NormalTok{, }\StringTok{"Деплетированность (B/K)"}\NormalTok{),}
  \AttributeTok{Value =} \FunctionTok{c}\NormalTok{(}
    \FunctionTok{mean}\NormalTok{(Catch[}\FunctionTok{length}\NormalTok{(Catch)}\SpecialCharTok{{-}}\DecValTok{2}\SpecialCharTok{:}\DecValTok{0}\NormalTok{]),  }\CommentTok{\# Средний вылов за последние 3 года}
    \FunctionTok{median}\NormalTok{(msy\_results}\SpecialCharTok{$}\NormalTok{msy),}
    \FunctionTok{mean}\NormalTok{(Catch[}\FunctionTok{length}\NormalTok{(Catch)}\SpecialCharTok{{-}}\DecValTok{2}\SpecialCharTok{:}\DecValTok{0}\NormalTok{]) }\SpecialCharTok{/} \FunctionTok{median}\NormalTok{(msy\_results}\SpecialCharTok{$}\NormalTok{msy),}
    \FunctionTok{median}\NormalTok{(msy\_results}\SpecialCharTok{$}\NormalTok{depletion)}
\NormalTok{  )}
\NormalTok{)}

\FunctionTok{print}\NormalTok{(}\StringTok{"Текущий статус запаса:"}\NormalTok{)}
\end{Highlighting}
\end{Shaded}

\begin{verbatim}
[1] "Текущий статус запаса:"
\end{verbatim}

\begin{Shaded}
\begin{Highlighting}[]
\FunctionTok{print}\NormalTok{(current\_status)}
\end{Highlighting}
\end{Shaded}

\begin{verbatim}
                   Metric      Value
1           Текущий вылов 12.0000000
2                     MSY 13.0834378
3  Отношение вылова к MSY  0.9171901
4 Деплетированность (B/K)  3.1261895
\end{verbatim}

\begin{Shaded}
\begin{Highlighting}[]
\CommentTok{\# {-}{-}{-}{-}{-}{-}{-}{-}{-}{-}{-}{-}{-}{-}{-}{-}{-}{-}{-}{-}{-}{-}{-}{-}{-} 5. ОПРЕДЕЛЕНИЕ REFERENCE POINTS {-}{-}{-}{-}{-}{-}{-}{-}{-}{-}{-}{-}{-}{-}{-}{-}{-}{-}{-}{-}{-}{-}{-}{-}{-}{-}{-}}

\CommentTok{\# Reference points на основе Catch{-}MSY}
\NormalTok{reference\_points }\OtherTok{\textless{}{-}} \FunctionTok{data.frame}\NormalTok{(}
  \AttributeTok{Point =} \FunctionTok{c}\NormalTok{(}\StringTok{"MSY"}\NormalTok{, }\StringTok{"Bmsy"}\NormalTok{, }\StringTok{"Fmsy"}\NormalTok{),}
  \AttributeTok{Value =} \FunctionTok{c}\NormalTok{(}
    \FunctionTok{median}\NormalTok{(msy\_results}\SpecialCharTok{$}\NormalTok{msy),}
    \FunctionTok{median}\NormalTok{(msy\_results}\SpecialCharTok{$}\NormalTok{b\_msy),}
    \FunctionTok{median}\NormalTok{(msy\_results}\SpecialCharTok{$}\NormalTok{r) }\SpecialCharTok{/} \DecValTok{2}  \CommentTok{\# Fmsy = r/2 для модели Шефера}
\NormalTok{  )}
\NormalTok{)}

\FunctionTok{print}\NormalTok{(}\StringTok{"Reference Points:"}\NormalTok{)}
\end{Highlighting}
\end{Shaded}

\begin{verbatim}
[1] "Reference Points:"
\end{verbatim}

\begin{Shaded}
\begin{Highlighting}[]
\FunctionTok{print}\NormalTok{(reference\_points)}
\end{Highlighting}
\end{Shaded}

\begin{verbatim}
  Point        Value
1   MSY  13.08343781
2  Bmsy 243.83818925
3  Fmsy   0.04835907
\end{verbatim}

\begin{Shaded}
\begin{Highlighting}[]
\CommentTok{\# {-}{-}{-}{-}{-}{-}{-}{-}{-}{-}{-}{-}{-}{-}{-}{-}{-}{-}{-}{-}{-}{-}{-}{-}{-} 6. ПОСТРОЕНИЕ ПРАВИЛА УПРАВЛЕНИЯ (HCR) {-}{-}{-}{-}{-}{-}{-}{-}{-}{-}{-}{-}{-}{-}{-}{-}{-}{-}{-}{-}{-}{-}{-}{-}{-}{-}{-}}

\CommentTok{\# Функция для определения HCR на основе B/Bmsy}
\NormalTok{harvest\_control\_rule }\OtherTok{\textless{}{-}} \ControlFlowTok{function}\NormalTok{(b\_ratio, }
                                 \AttributeTok{blim =} \FloatTok{0.4}\NormalTok{, }
                                 \AttributeTok{btarget =} \FloatTok{0.8}\NormalTok{, }
                                 \AttributeTok{fmin =} \FloatTok{0.05}\NormalTok{, }
                                 \AttributeTok{fmax =} \FloatTok{1.0}\NormalTok{) \{}
  \ControlFlowTok{if}\NormalTok{ (b\_ratio }\SpecialCharTok{\textless{}=}\NormalTok{ blim) \{}
    \FunctionTok{return}\NormalTok{(fmin)  }\CommentTok{\# Минимальный уровень изъятия при низкой биомассе}
\NormalTok{  \} }\ControlFlowTok{else} \ControlFlowTok{if}\NormalTok{ (b\_ratio }\SpecialCharTok{\textgreater{}=}\NormalTok{ btarget) \{}
    \FunctionTok{return}\NormalTok{(fmax)  }\CommentTok{\# Максимальный уровень изъятия при высокой биомассе}
\NormalTok{  \} }\ControlFlowTok{else}\NormalTok{ \{}
    \CommentTok{\# Линейное увеличение от fmin до fmax}
    \FunctionTok{return}\NormalTok{(fmin }\SpecialCharTok{+}\NormalTok{ (fmax }\SpecialCharTok{{-}}\NormalTok{ fmin) }\SpecialCharTok{*}\NormalTok{ (b\_ratio }\SpecialCharTok{{-}}\NormalTok{ blim) }\SpecialCharTok{/}\NormalTok{ (btarget }\SpecialCharTok{{-}}\NormalTok{ blim))}
\NormalTok{  \}}
\NormalTok{\}}

\CommentTok{\# Пример применения HCR для различных уровней биомассы}
\NormalTok{b\_ratios }\OtherTok{\textless{}{-}} \FunctionTok{seq}\NormalTok{(}\FloatTok{0.2}\NormalTok{, }\FloatTok{1.2}\NormalTok{, }\AttributeTok{by =} \FloatTok{0.1}\NormalTok{)}
\NormalTok{hcr\_values }\OtherTok{\textless{}{-}} \FunctionTok{sapply}\NormalTok{(b\_ratios, harvest\_control\_rule)}

\NormalTok{hcr\_df }\OtherTok{\textless{}{-}} \FunctionTok{data.frame}\NormalTok{(}\AttributeTok{B\_Bmsy =}\NormalTok{ b\_ratios, }\AttributeTok{F\_Fmsy =}\NormalTok{ hcr\_values)}

\FunctionTok{ggplot}\NormalTok{(hcr\_df, }\FunctionTok{aes}\NormalTok{(}\AttributeTok{x =}\NormalTok{ B\_Bmsy, }\AttributeTok{y =}\NormalTok{ F\_Fmsy)) }\SpecialCharTok{+}
  \FunctionTok{geom\_line}\NormalTok{(}\AttributeTok{linewidth =} \FloatTok{1.5}\NormalTok{, }\AttributeTok{color =} \StringTok{"blue"}\NormalTok{) }\SpecialCharTok{+}
  \FunctionTok{geom\_vline}\NormalTok{(}\AttributeTok{xintercept =} \FloatTok{0.4}\NormalTok{, }\AttributeTok{linetype =} \StringTok{"dashed"}\NormalTok{, }\AttributeTok{color =} \StringTok{"red"}\NormalTok{) }\SpecialCharTok{+}
  \FunctionTok{geom\_vline}\NormalTok{(}\AttributeTok{xintercept =} \FloatTok{0.8}\NormalTok{, }\AttributeTok{linetype =} \StringTok{"dashed"}\NormalTok{, }\AttributeTok{color =} \StringTok{"green"}\NormalTok{) }\SpecialCharTok{+}
  \FunctionTok{labs}\NormalTok{(}\AttributeTok{title =} \StringTok{"Правило управления Harvest Control Rule (HCR)"}\NormalTok{,}
       \AttributeTok{x =} \StringTok{"Отношение биомассы к Bmsy (B/Bmsy)"}\NormalTok{,}
       \AttributeTok{y =} \StringTok{"Отношение fishing mortality к Fmsy (F/Fmsy)"}\NormalTok{) }\SpecialCharTok{+}
  \FunctionTok{theme\_minimal}\NormalTok{() }\SpecialCharTok{+}
  \FunctionTok{theme}\NormalTok{(}\AttributeTok{plot.title =} \FunctionTok{element\_text}\NormalTok{(}\AttributeTok{hjust =} \FloatTok{0.5}\NormalTok{))}
\end{Highlighting}
\end{Shaded}

\begin{verbatim}
Warning in grid.Call(C_textBounds, as.graphicsAnnot(x$label), x$x, x$y, :
неизвестна ширина символа 0xce в кодировке CP1251
\end{verbatim}

\begin{verbatim}
Warning in grid.Call(C_textBounds, as.graphicsAnnot(x$label), x$x, x$y, :
неизвестна ширина символа 0xf2 в кодировке CP1251
\end{verbatim}

\begin{verbatim}
Warning in grid.Call(C_textBounds, as.graphicsAnnot(x$label), x$x, x$y, :
неизвестна ширина символа 0xed в кодировке CP1251
\end{verbatim}

\begin{verbatim}
Warning in grid.Call(C_textBounds, as.graphicsAnnot(x$label), x$x, x$y, :
неизвестна ширина символа 0xee в кодировке CP1251
\end{verbatim}

\begin{verbatim}
Warning in grid.Call(C_textBounds, as.graphicsAnnot(x$label), x$x, x$y, :
неизвестна ширина символа 0xf8 в кодировке CP1251
\end{verbatim}

\begin{verbatim}
Warning in grid.Call(C_textBounds, as.graphicsAnnot(x$label), x$x, x$y, :
неизвестна ширина символа 0xe5 в кодировке CP1251
\end{verbatim}

\begin{verbatim}
Warning in grid.Call(C_textBounds, as.graphicsAnnot(x$label), x$x, x$y, :
неизвестна ширина символа 0xed в кодировке CP1251
\end{verbatim}

\begin{verbatim}
Warning in grid.Call(C_textBounds, as.graphicsAnnot(x$label), x$x, x$y, :
неизвестна ширина символа 0xe8 в кодировке CP1251
\end{verbatim}

\begin{verbatim}
Warning in grid.Call(C_textBounds, as.graphicsAnnot(x$label), x$x, x$y, :
неизвестна ширина символа 0xe5 в кодировке CP1251
\end{verbatim}

\begin{verbatim}
Warning in grid.Call(C_textBounds, as.graphicsAnnot(x$label), x$x, x$y, :
неизвестна ширина символа 0xea в кодировке CP1251
\end{verbatim}

\begin{verbatim}
Warning in grid.Call(C_textBounds, as.graphicsAnnot(x$label), x$x, x$y, :
неизвестна ширина символа 0xcf в кодировке CP1251
\end{verbatim}

\begin{verbatim}
Warning in grid.Call(C_textBounds, as.graphicsAnnot(x$label), x$x, x$y, :
неизвестна ширина символа 0xf0 в кодировке CP1251
\end{verbatim}

\begin{verbatim}
Warning in grid.Call(C_textBounds, as.graphicsAnnot(x$label), x$x, x$y, :
неизвестна ширина символа 0xe0 в кодировке CP1251
\end{verbatim}

\begin{verbatim}
Warning in grid.Call(C_textBounds, as.graphicsAnnot(x$label), x$x, x$y, :
неизвестна ширина символа 0xe2 в кодировке CP1251
\end{verbatim}

\begin{verbatim}
Warning in grid.Call(C_textBounds, as.graphicsAnnot(x$label), x$x, x$y, :
неизвестна ширина символа 0xe8 в кодировке CP1251
\end{verbatim}

\begin{verbatim}
Warning in grid.Call(C_textBounds, as.graphicsAnnot(x$label), x$x, x$y, :
неизвестна ширина символа 0xeb в кодировке CP1251
\end{verbatim}

\begin{verbatim}
Warning in grid.Call(C_textBounds, as.graphicsAnnot(x$label), x$x, x$y, :
неизвестна ширина символа 0xee в кодировке CP1251
\end{verbatim}

\begin{verbatim}
Warning in grid.Call(C_textBounds, as.graphicsAnnot(x$label), x$x, x$y, :
неизвестна ширина символа 0xf3 в кодировке CP1251
\end{verbatim}

\begin{verbatim}
Warning in grid.Call(C_textBounds, as.graphicsAnnot(x$label), x$x, x$y, :
неизвестна ширина символа 0xef в кодировке CP1251
\end{verbatim}

\begin{verbatim}
Warning in grid.Call(C_textBounds, as.graphicsAnnot(x$label), x$x, x$y, :
неизвестна ширина символа 0xf0 в кодировке CP1251
\end{verbatim}

\begin{verbatim}
Warning in grid.Call(C_textBounds, as.graphicsAnnot(x$label), x$x, x$y, :
неизвестна ширина символа 0xe0 в кодировке CP1251
\end{verbatim}

\begin{verbatim}
Warning in grid.Call(C_textBounds, as.graphicsAnnot(x$label), x$x, x$y, :
неизвестна ширина символа 0xe2 в кодировке CP1251
\end{verbatim}

\begin{verbatim}
Warning in grid.Call(C_textBounds, as.graphicsAnnot(x$label), x$x, x$y, :
неизвестна ширина символа 0xeb в кодировке CP1251
\end{verbatim}

\begin{verbatim}
Warning in grid.Call(C_textBounds, as.graphicsAnnot(x$label), x$x, x$y, :
неизвестна ширина символа 0xe5 в кодировке CP1251
\end{verbatim}

\begin{verbatim}
Warning in grid.Call(C_textBounds, as.graphicsAnnot(x$label), x$x, x$y, :
неизвестна ширина символа 0xed в кодировке CP1251
\end{verbatim}

\begin{verbatim}
Warning in grid.Call(C_textBounds, as.graphicsAnnot(x$label), x$x, x$y, :
неизвестна ширина символа 0xe8 в кодировке CP1251
\end{verbatim}

\begin{verbatim}
Warning in grid.Call(C_textBounds, as.graphicsAnnot(x$label), x$x, x$y, :
неизвестна ширина символа 0xff в кодировке CP1251
\end{verbatim}

\begin{verbatim}
Warning in grid.Call(C_textBounds, as.graphicsAnnot(x$label), x$x, x$y, :
неизвестна ширина символа 0xce в кодировке CP1251
\end{verbatim}

\begin{verbatim}
Warning in grid.Call(C_textBounds, as.graphicsAnnot(x$label), x$x, x$y, :
неизвестна ширина символа 0xf2 в кодировке CP1251
\end{verbatim}

\begin{verbatim}
Warning in grid.Call(C_textBounds, as.graphicsAnnot(x$label), x$x, x$y, :
неизвестна ширина символа 0xed в кодировке CP1251
\end{verbatim}

\begin{verbatim}
Warning in grid.Call(C_textBounds, as.graphicsAnnot(x$label), x$x, x$y, :
неизвестна ширина символа 0xee в кодировке CP1251
\end{verbatim}

\begin{verbatim}
Warning in grid.Call(C_textBounds, as.graphicsAnnot(x$label), x$x, x$y, :
неизвестна ширина символа 0xf8 в кодировке CP1251
\end{verbatim}

\begin{verbatim}
Warning in grid.Call(C_textBounds, as.graphicsAnnot(x$label), x$x, x$y, :
неизвестна ширина символа 0xe5 в кодировке CP1251
\end{verbatim}

\begin{verbatim}
Warning in grid.Call(C_textBounds, as.graphicsAnnot(x$label), x$x, x$y, :
неизвестна ширина символа 0xed в кодировке CP1251
\end{verbatim}

\begin{verbatim}
Warning in grid.Call(C_textBounds, as.graphicsAnnot(x$label), x$x, x$y, :
неизвестна ширина символа 0xe8 в кодировке CP1251
\end{verbatim}

\begin{verbatim}
Warning in grid.Call(C_textBounds, as.graphicsAnnot(x$label), x$x, x$y, :
неизвестна ширина символа 0xe5 в кодировке CP1251
\end{verbatim}

\begin{verbatim}
Warning in grid.Call(C_textBounds, as.graphicsAnnot(x$label), x$x, x$y, :
неизвестна ширина символа 0xe1 в кодировке CP1251
\end{verbatim}

\begin{verbatim}
Warning in grid.Call(C_textBounds, as.graphicsAnnot(x$label), x$x, x$y, :
неизвестна ширина символа 0xe8 в кодировке CP1251
\end{verbatim}

\begin{verbatim}
Warning in grid.Call(C_textBounds, as.graphicsAnnot(x$label), x$x, x$y, :
неизвестна ширина символа 0xee в кодировке CP1251
\end{verbatim}

\begin{verbatim}
Warning in grid.Call(C_textBounds, as.graphicsAnnot(x$label), x$x, x$y, :
неизвестна ширина символа 0xec в кодировке CP1251
\end{verbatim}

\begin{verbatim}
Warning in grid.Call(C_textBounds, as.graphicsAnnot(x$label), x$x, x$y, :
неизвестна ширина символа 0xe0 в кодировке CP1251
\end{verbatim}

\begin{verbatim}
Warning in grid.Call(C_textBounds, as.graphicsAnnot(x$label), x$x, x$y, :
неизвестна ширина символа 0xf1 в кодировке CP1251
Warning in grid.Call(C_textBounds, as.graphicsAnnot(x$label), x$x, x$y, :
неизвестна ширина символа 0xf1 в кодировке CP1251
\end{verbatim}

\begin{verbatim}
Warning in grid.Call(C_textBounds, as.graphicsAnnot(x$label), x$x, x$y, :
неизвестна ширина символа 0xfb в кодировке CP1251
\end{verbatim}

\begin{verbatim}
Warning in grid.Call(C_textBounds, as.graphicsAnnot(x$label), x$x, x$y, :
неизвестна ширина символа 0xea в кодировке CP1251
\end{verbatim}

\begin{verbatim}
Warning in grid.Call.graphics(C_text, as.graphicsAnnot(x$label), x$x, x$y, :
неизвестна ширина символа 0xce в кодировке CP1251
\end{verbatim}

\begin{verbatim}
Warning in grid.Call.graphics(C_text, as.graphicsAnnot(x$label), x$x, x$y, :
неизвестна ширина символа 0xf2 в кодировке CP1251
\end{verbatim}

\begin{verbatim}
Warning in grid.Call.graphics(C_text, as.graphicsAnnot(x$label), x$x, x$y, :
неизвестна ширина символа 0xed в кодировке CP1251
\end{verbatim}

\begin{verbatim}
Warning in grid.Call.graphics(C_text, as.graphicsAnnot(x$label), x$x, x$y, :
неизвестна ширина символа 0xee в кодировке CP1251
\end{verbatim}

\begin{verbatim}
Warning in grid.Call.graphics(C_text, as.graphicsAnnot(x$label), x$x, x$y, :
неизвестна ширина символа 0xf8 в кодировке CP1251
\end{verbatim}

\begin{verbatim}
Warning in grid.Call.graphics(C_text, as.graphicsAnnot(x$label), x$x, x$y, :
неизвестна ширина символа 0xe5 в кодировке CP1251
\end{verbatim}

\begin{verbatim}
Warning in grid.Call.graphics(C_text, as.graphicsAnnot(x$label), x$x, x$y, :
неизвестна ширина символа 0xed в кодировке CP1251
\end{verbatim}

\begin{verbatim}
Warning in grid.Call.graphics(C_text, as.graphicsAnnot(x$label), x$x, x$y, :
неизвестна ширина символа 0xe8 в кодировке CP1251
\end{verbatim}

\begin{verbatim}
Warning in grid.Call.graphics(C_text, as.graphicsAnnot(x$label), x$x, x$y, :
неизвестна ширина символа 0xe5 в кодировке CP1251
\end{verbatim}

\begin{verbatim}
Warning in grid.Call.graphics(C_text, as.graphicsAnnot(x$label), x$x, x$y, :
неизвестна ширина символа 0xe1 в кодировке CP1251
\end{verbatim}

\begin{verbatim}
Warning in grid.Call.graphics(C_text, as.graphicsAnnot(x$label), x$x, x$y, :
неизвестна ширина символа 0xe8 в кодировке CP1251
\end{verbatim}

\begin{verbatim}
Warning in grid.Call.graphics(C_text, as.graphicsAnnot(x$label), x$x, x$y, :
неизвестна ширина символа 0xee в кодировке CP1251
\end{verbatim}

\begin{verbatim}
Warning in grid.Call.graphics(C_text, as.graphicsAnnot(x$label), x$x, x$y, :
неизвестна ширина символа 0xec в кодировке CP1251
\end{verbatim}

\begin{verbatim}
Warning in grid.Call.graphics(C_text, as.graphicsAnnot(x$label), x$x, x$y, :
неизвестна ширина символа 0xe0 в кодировке CP1251
\end{verbatim}

\begin{verbatim}
Warning in grid.Call.graphics(C_text, as.graphicsAnnot(x$label), x$x, x$y, :
неизвестна ширина символа 0xf1 в кодировке CP1251
Warning in grid.Call.graphics(C_text, as.graphicsAnnot(x$label), x$x, x$y, :
неизвестна ширина символа 0xf1 в кодировке CP1251
\end{verbatim}

\begin{verbatim}
Warning in grid.Call.graphics(C_text, as.graphicsAnnot(x$label), x$x, x$y, :
неизвестна ширина символа 0xfb в кодировке CP1251
\end{verbatim}

\begin{verbatim}
Warning in grid.Call.graphics(C_text, as.graphicsAnnot(x$label), x$x, x$y, :
неизвестна ширина символа 0xea в кодировке CP1251
\end{verbatim}

\begin{verbatim}
Warning in grid.Call.graphics(C_text, as.graphicsAnnot(x$label), x$x, x$y, :
неизвестна ширина символа 0xce в кодировке CP1251
\end{verbatim}

\begin{verbatim}
Warning in grid.Call.graphics(C_text, as.graphicsAnnot(x$label), x$x, x$y, :
неизвестна ширина символа 0xf2 в кодировке CP1251
\end{verbatim}

\begin{verbatim}
Warning in grid.Call.graphics(C_text, as.graphicsAnnot(x$label), x$x, x$y, :
неизвестна ширина символа 0xed в кодировке CP1251
\end{verbatim}

\begin{verbatim}
Warning in grid.Call.graphics(C_text, as.graphicsAnnot(x$label), x$x, x$y, :
неизвестна ширина символа 0xee в кодировке CP1251
\end{verbatim}

\begin{verbatim}
Warning in grid.Call.graphics(C_text, as.graphicsAnnot(x$label), x$x, x$y, :
неизвестна ширина символа 0xf8 в кодировке CP1251
\end{verbatim}

\begin{verbatim}
Warning in grid.Call.graphics(C_text, as.graphicsAnnot(x$label), x$x, x$y, :
неизвестна ширина символа 0xe5 в кодировке CP1251
\end{verbatim}

\begin{verbatim}
Warning in grid.Call.graphics(C_text, as.graphicsAnnot(x$label), x$x, x$y, :
неизвестна ширина символа 0xed в кодировке CP1251
\end{verbatim}

\begin{verbatim}
Warning in grid.Call.graphics(C_text, as.graphicsAnnot(x$label), x$x, x$y, :
неизвестна ширина символа 0xe8 в кодировке CP1251
\end{verbatim}

\begin{verbatim}
Warning in grid.Call.graphics(C_text, as.graphicsAnnot(x$label), x$x, x$y, :
неизвестна ширина символа 0xe5 в кодировке CP1251
\end{verbatim}

\begin{verbatim}
Warning in grid.Call.graphics(C_text, as.graphicsAnnot(x$label), x$x, x$y, :
неизвестна ширина символа 0xea в кодировке CP1251
\end{verbatim}

\begin{verbatim}
Warning in grid.Call.graphics(C_text, as.graphicsAnnot(x$label), x$x, x$y, :
неизвестна ширина символа 0xcf в кодировке CP1251
\end{verbatim}

\begin{verbatim}
Warning in grid.Call.graphics(C_text, as.graphicsAnnot(x$label), x$x, x$y, :
неизвестна ширина символа 0xf0 в кодировке CP1251
\end{verbatim}

\begin{verbatim}
Warning in grid.Call.graphics(C_text, as.graphicsAnnot(x$label), x$x, x$y, :
неизвестна ширина символа 0xe0 в кодировке CP1251
\end{verbatim}

\begin{verbatim}
Warning in grid.Call.graphics(C_text, as.graphicsAnnot(x$label), x$x, x$y, :
неизвестна ширина символа 0xe2 в кодировке CP1251
\end{verbatim}

\begin{verbatim}
Warning in grid.Call.graphics(C_text, as.graphicsAnnot(x$label), x$x, x$y, :
неизвестна ширина символа 0xe8 в кодировке CP1251
\end{verbatim}

\begin{verbatim}
Warning in grid.Call.graphics(C_text, as.graphicsAnnot(x$label), x$x, x$y, :
неизвестна ширина символа 0xeb в кодировке CP1251
\end{verbatim}

\begin{verbatim}
Warning in grid.Call.graphics(C_text, as.graphicsAnnot(x$label), x$x, x$y, :
неизвестна ширина символа 0xee в кодировке CP1251
\end{verbatim}

\begin{verbatim}
Warning in grid.Call.graphics(C_text, as.graphicsAnnot(x$label), x$x, x$y, :
неизвестна ширина символа 0xf3 в кодировке CP1251
\end{verbatim}

\begin{verbatim}
Warning in grid.Call.graphics(C_text, as.graphicsAnnot(x$label), x$x, x$y, :
неизвестна ширина символа 0xef в кодировке CP1251
\end{verbatim}

\begin{verbatim}
Warning in grid.Call.graphics(C_text, as.graphicsAnnot(x$label), x$x, x$y, :
неизвестна ширина символа 0xf0 в кодировке CP1251
\end{verbatim}

\begin{verbatim}
Warning in grid.Call.graphics(C_text, as.graphicsAnnot(x$label), x$x, x$y, :
неизвестна ширина символа 0xe0 в кодировке CP1251
\end{verbatim}

\begin{verbatim}
Warning in grid.Call.graphics(C_text, as.graphicsAnnot(x$label), x$x, x$y, :
неизвестна ширина символа 0xe2 в кодировке CP1251
\end{verbatim}

\begin{verbatim}
Warning in grid.Call.graphics(C_text, as.graphicsAnnot(x$label), x$x, x$y, :
неизвестна ширина символа 0xeb в кодировке CP1251
\end{verbatim}

\begin{verbatim}
Warning in grid.Call.graphics(C_text, as.graphicsAnnot(x$label), x$x, x$y, :
неизвестна ширина символа 0xe5 в кодировке CP1251
\end{verbatim}

\begin{verbatim}
Warning in grid.Call.graphics(C_text, as.graphicsAnnot(x$label), x$x, x$y, :
неизвестна ширина символа 0xed в кодировке CP1251
\end{verbatim}

\begin{verbatim}
Warning in grid.Call.graphics(C_text, as.graphicsAnnot(x$label), x$x, x$y, :
неизвестна ширина символа 0xe8 в кодировке CP1251
\end{verbatim}

\begin{verbatim}
Warning in grid.Call.graphics(C_text, as.graphicsAnnot(x$label), x$x, x$y, :
неизвестна ширина символа 0xff в кодировке CP1251
\end{verbatim}

\pandocbounded{\includegraphics[keepaspectratio]{chapter16_files/figure-pdf/unnamed-chunk-1-3.pdf}}

\begin{Shaded}
\begin{Highlighting}[]
\CommentTok{\# {-}{-}{-}{-}{-}{-}{-}{-}{-}{-}{-}{-}{-}{-}{-}{-}{-}{-}{-}{-}{-}{-}{-}{-}{-} 7. МОДЕЛИРОВАНИЕ УПРАВЛЕНИЯ (MSE) {-}{-}{-}{-}{-}{-}{-}{-}{-}{-}{-}{-}{-}{-}{-}{-}{-}{-}{-}{-}{-}{-}{-}{-}{-}{-}{-}}

\CommentTok{\# Упрощенное моделирование управления}
\NormalTok{management\_simulation }\OtherTok{\textless{}{-}} \ControlFlowTok{function}\NormalTok{(init\_biomass, r, k, catch, }\AttributeTok{n\_years =} \DecValTok{20}\NormalTok{, hcr\_function) \{}
\NormalTok{  biomass }\OtherTok{\textless{}{-}} \FunctionTok{numeric}\NormalTok{(n\_years }\SpecialCharTok{+} \DecValTok{1}\NormalTok{)}
\NormalTok{  biomass[}\DecValTok{1}\NormalTok{] }\OtherTok{\textless{}{-}}\NormalTok{ init\_biomass}
\NormalTok{  catch\_rec }\OtherTok{\textless{}{-}} \FunctionTok{numeric}\NormalTok{(n\_years)}
\NormalTok{  f\_rec }\OtherTok{\textless{}{-}} \FunctionTok{numeric}\NormalTok{(n\_years)}
\NormalTok{  b\_ratio\_rec }\OtherTok{\textless{}{-}} \FunctionTok{numeric}\NormalTok{(n\_years)}
  
  \ControlFlowTok{for}\NormalTok{ (t }\ControlFlowTok{in} \DecValTok{1}\SpecialCharTok{:}\NormalTok{n\_years) \{}
    \CommentTok{\# Расчет текущего отношения B/Bmsy}
\NormalTok{    b\_ratio }\OtherTok{\textless{}{-}}\NormalTok{ biomass[t] }\SpecialCharTok{/}\NormalTok{ (k }\SpecialCharTok{/} \DecValTok{2}\NormalTok{)}
\NormalTok{    b\_ratio\_rec[t] }\OtherTok{\textless{}{-}}\NormalTok{ b\_ratio}
    
    \CommentTok{\# Применение HCR для определения уровня изъятия}
\NormalTok{    f\_multiplier }\OtherTok{\textless{}{-}} \FunctionTok{hcr\_function}\NormalTok{(b\_ratio)}
\NormalTok{    f\_rec[t] }\OtherTok{\textless{}{-}}\NormalTok{ f\_multiplier }\SpecialCharTok{*}\NormalTok{ (r }\SpecialCharTok{/} \DecValTok{2}\NormalTok{)  }\CommentTok{\# F = multiplier * Fmsy}
    
    \CommentTok{\# Расчет вылова на основе F}
\NormalTok{    catch\_rec[t] }\OtherTok{\textless{}{-}}\NormalTok{ f\_rec[t] }\SpecialCharTok{*}\NormalTok{ biomass[t]}
    
    \CommentTok{\# Обновление биомассы (модель Шефера)}
\NormalTok{    biomass[t }\SpecialCharTok{+} \DecValTok{1}\NormalTok{] }\OtherTok{\textless{}{-}} \FunctionTok{max}\NormalTok{(}\FloatTok{0.001} \SpecialCharTok{*}\NormalTok{ k, biomass[t] }\SpecialCharTok{+}\NormalTok{ r }\SpecialCharTok{*}\NormalTok{ biomass[t] }\SpecialCharTok{*}\NormalTok{ (}\DecValTok{1} \SpecialCharTok{{-}}\NormalTok{ biomass[t] }\SpecialCharTok{/}\NormalTok{ k) }\SpecialCharTok{{-}}\NormalTok{ catch\_rec[t])}
\NormalTok{  \}}
  
  \FunctionTok{list}\NormalTok{(}\AttributeTok{biomass =}\NormalTok{ biomass, }\AttributeTok{catch =}\NormalTok{ catch\_rec, }\AttributeTok{f =}\NormalTok{ f\_rec, }\AttributeTok{b\_ratio =}\NormalTok{ b\_ratio\_rec)}
\NormalTok{\}}

\CommentTok{\# Запуск моделирования с различными начальными условиями}
\FunctionTok{set.seed}\NormalTok{(}\DecValTok{123}\NormalTok{)}
\NormalTok{n\_sim }\OtherTok{\textless{}{-}} \DecValTok{10}
\NormalTok{init\_depletion }\OtherTok{\textless{}{-}} \FunctionTok{runif}\NormalTok{(n\_sim, }\FloatTok{0.3}\NormalTok{, }\FloatTok{0.7}\NormalTok{)  }\CommentTok{\# Различные начальные уровни деплетированности}

\NormalTok{sim\_results }\OtherTok{\textless{}{-}} \FunctionTok{list}\NormalTok{()}

\ControlFlowTok{for}\NormalTok{ (i }\ControlFlowTok{in} \DecValTok{1}\SpecialCharTok{:}\NormalTok{n\_sim) \{}
\NormalTok{  init\_biomass }\OtherTok{\textless{}{-}}\NormalTok{ init\_depletion[i] }\SpecialCharTok{*} \FunctionTok{median}\NormalTok{(msy\_results}\SpecialCharTok{$}\NormalTok{k)}
\NormalTok{  sim\_results[[i]] }\OtherTok{\textless{}{-}} \FunctionTok{management\_simulation}\NormalTok{(init\_biomass, }
                                           \FunctionTok{median}\NormalTok{(msy\_results}\SpecialCharTok{$}\NormalTok{r), }
                                           \FunctionTok{median}\NormalTok{(msy\_results}\SpecialCharTok{$}\NormalTok{k), }
\NormalTok{                                           Catch,}
                                           \AttributeTok{hcr\_function =}\NormalTok{ harvest\_control\_rule)}
\NormalTok{\}}

\CommentTok{\# {-}{-}{-}{-}{-}{-}{-}{-}{-}{-}{-}{-}{-}{-}{-}{-}{-}{-}{-}{-}{-}{-}{-}{-}{-} 8. АНАЛИЗ РЕЗУЛЬТАТОВ MSE {-}{-}{-}{-}{-}{-}{-}{-}{-}{-}{-}{-}{-}{-}{-}{-}{-}{-}{-}{-}{-}{-}{-}{-}{-}{-}{-}}

\CommentTok{\# Подготовка данных для визуализации}
\NormalTok{years\_proj }\OtherTok{\textless{}{-}} \DecValTok{1}\SpecialCharTok{:}\DecValTok{21}
\NormalTok{sim\_biomass }\OtherTok{\textless{}{-}} \FunctionTok{sapply}\NormalTok{(sim\_results, }\ControlFlowTok{function}\NormalTok{(x) x}\SpecialCharTok{$}\NormalTok{biomass)}
\NormalTok{sim\_catch }\OtherTok{\textless{}{-}} \FunctionTok{sapply}\NormalTok{(sim\_results, }\ControlFlowTok{function}\NormalTok{(x) x}\SpecialCharTok{$}\NormalTok{catch)}

\CommentTok{\# Визуализация траекторий биомассы}
\NormalTok{biomass\_df }\OtherTok{\textless{}{-}} \FunctionTok{data.frame}\NormalTok{(}\AttributeTok{Year =} \FunctionTok{rep}\NormalTok{(years\_proj, n\_sim),}
                         \AttributeTok{Biomass =} \FunctionTok{as.vector}\NormalTok{(sim\_biomass),}
                         \AttributeTok{Simulation =} \FunctionTok{rep}\NormalTok{(}\DecValTok{1}\SpecialCharTok{:}\NormalTok{n\_sim, }\AttributeTok{each =} \FunctionTok{length}\NormalTok{(years\_proj)))}

\FunctionTok{ggplot}\NormalTok{(biomass\_df, }\FunctionTok{aes}\NormalTok{(}\AttributeTok{x =}\NormalTok{ Year, }\AttributeTok{y =}\NormalTok{ Biomass, }\AttributeTok{group =}\NormalTok{ Simulation, }\AttributeTok{color =}\NormalTok{ Simulation)) }\SpecialCharTok{+}
  \FunctionTok{geom\_line}\NormalTok{(}\AttributeTok{linewidth =} \FloatTok{0.7}\NormalTok{) }\SpecialCharTok{+}
  \FunctionTok{geom\_hline}\NormalTok{(}\AttributeTok{yintercept =} \FunctionTok{median}\NormalTok{(msy\_results}\SpecialCharTok{$}\NormalTok{b\_msy), }\AttributeTok{linetype =} \StringTok{"dashed"}\NormalTok{, }\AttributeTok{color =} \StringTok{"red"}\NormalTok{) }\SpecialCharTok{+}
  \FunctionTok{labs}\NormalTok{(}\AttributeTok{title =} \StringTok{"Траектории биомассы в имитационном моделировании"}\NormalTok{,}
       \AttributeTok{y =} \StringTok{"Биомасса"}\NormalTok{) }\SpecialCharTok{+}
  \FunctionTok{theme\_minimal}\NormalTok{() }\SpecialCharTok{+}
  \FunctionTok{theme}\NormalTok{(}\AttributeTok{plot.title =} \FunctionTok{element\_text}\NormalTok{(}\AttributeTok{hjust =} \FloatTok{0.5}\NormalTok{))}
\end{Highlighting}
\end{Shaded}

\begin{verbatim}
Warning in grid.Call(C_textBounds, as.graphicsAnnot(x$label), x$x, x$y, :
неизвестна ширина символа 0xc1 в кодировке CP1251
\end{verbatim}

\begin{verbatim}
Warning in grid.Call(C_textBounds, as.graphicsAnnot(x$label), x$x, x$y, :
неизвестна ширина символа 0xe8 в кодировке CP1251
\end{verbatim}

\begin{verbatim}
Warning in grid.Call(C_textBounds, as.graphicsAnnot(x$label), x$x, x$y, :
неизвестна ширина символа 0xee в кодировке CP1251
\end{verbatim}

\begin{verbatim}
Warning in grid.Call(C_textBounds, as.graphicsAnnot(x$label), x$x, x$y, :
неизвестна ширина символа 0xec в кодировке CP1251
\end{verbatim}

\begin{verbatim}
Warning in grid.Call(C_textBounds, as.graphicsAnnot(x$label), x$x, x$y, :
неизвестна ширина символа 0xe0 в кодировке CP1251
\end{verbatim}

\begin{verbatim}
Warning in grid.Call(C_textBounds, as.graphicsAnnot(x$label), x$x, x$y, :
неизвестна ширина символа 0xf1 в кодировке CP1251
Warning in grid.Call(C_textBounds, as.graphicsAnnot(x$label), x$x, x$y, :
неизвестна ширина символа 0xf1 в кодировке CP1251
\end{verbatim}

\begin{verbatim}
Warning in grid.Call(C_textBounds, as.graphicsAnnot(x$label), x$x, x$y, :
неизвестна ширина символа 0xe0 в кодировке CP1251
\end{verbatim}

\begin{verbatim}
Warning in grid.Call(C_textBounds, as.graphicsAnnot(x$label), x$x, x$y, :
неизвестна ширина символа 0xd2 в кодировке CP1251
\end{verbatim}

\begin{verbatim}
Warning in grid.Call(C_textBounds, as.graphicsAnnot(x$label), x$x, x$y, :
неизвестна ширина символа 0xf0 в кодировке CP1251
\end{verbatim}

\begin{verbatim}
Warning in grid.Call(C_textBounds, as.graphicsAnnot(x$label), x$x, x$y, :
неизвестна ширина символа 0xe0 в кодировке CP1251
\end{verbatim}

\begin{verbatim}
Warning in grid.Call(C_textBounds, as.graphicsAnnot(x$label), x$x, x$y, :
неизвестна ширина символа 0xe5 в кодировке CP1251
\end{verbatim}

\begin{verbatim}
Warning in grid.Call(C_textBounds, as.graphicsAnnot(x$label), x$x, x$y, :
неизвестна ширина символа 0xea в кодировке CP1251
\end{verbatim}

\begin{verbatim}
Warning in grid.Call(C_textBounds, as.graphicsAnnot(x$label), x$x, x$y, :
неизвестна ширина символа 0xf2 в кодировке CP1251
\end{verbatim}

\begin{verbatim}
Warning in grid.Call(C_textBounds, as.graphicsAnnot(x$label), x$x, x$y, :
неизвестна ширина символа 0xee в кодировке CP1251
\end{verbatim}

\begin{verbatim}
Warning in grid.Call(C_textBounds, as.graphicsAnnot(x$label), x$x, x$y, :
неизвестна ширина символа 0xf0 в кодировке CP1251
\end{verbatim}

\begin{verbatim}
Warning in grid.Call(C_textBounds, as.graphicsAnnot(x$label), x$x, x$y, :
неизвестна ширина символа 0xe8 в кодировке CP1251
Warning in grid.Call(C_textBounds, as.graphicsAnnot(x$label), x$x, x$y, :
неизвестна ширина символа 0xe8 в кодировке CP1251
\end{verbatim}

\begin{verbatim}
Warning in grid.Call(C_textBounds, as.graphicsAnnot(x$label), x$x, x$y, :
неизвестна ширина символа 0xe1 в кодировке CP1251
\end{verbatim}

\begin{verbatim}
Warning in grid.Call(C_textBounds, as.graphicsAnnot(x$label), x$x, x$y, :
неизвестна ширина символа 0xe8 в кодировке CP1251
\end{verbatim}

\begin{verbatim}
Warning in grid.Call(C_textBounds, as.graphicsAnnot(x$label), x$x, x$y, :
неизвестна ширина символа 0xee в кодировке CP1251
\end{verbatim}

\begin{verbatim}
Warning in grid.Call(C_textBounds, as.graphicsAnnot(x$label), x$x, x$y, :
неизвестна ширина символа 0xec в кодировке CP1251
\end{verbatim}

\begin{verbatim}
Warning in grid.Call(C_textBounds, as.graphicsAnnot(x$label), x$x, x$y, :
неизвестна ширина символа 0xe0 в кодировке CP1251
\end{verbatim}

\begin{verbatim}
Warning in grid.Call(C_textBounds, as.graphicsAnnot(x$label), x$x, x$y, :
неизвестна ширина символа 0xf1 в кодировке CP1251
Warning in grid.Call(C_textBounds, as.graphicsAnnot(x$label), x$x, x$y, :
неизвестна ширина символа 0xf1 в кодировке CP1251
\end{verbatim}

\begin{verbatim}
Warning in grid.Call(C_textBounds, as.graphicsAnnot(x$label), x$x, x$y, :
неизвестна ширина символа 0xfb в кодировке CP1251
\end{verbatim}

\begin{verbatim}
Warning in grid.Call(C_textBounds, as.graphicsAnnot(x$label), x$x, x$y, :
неизвестна ширина символа 0xe2 в кодировке CP1251
\end{verbatim}

\begin{verbatim}
Warning in grid.Call(C_textBounds, as.graphicsAnnot(x$label), x$x, x$y, :
неизвестна ширина символа 0xe8 в кодировке CP1251
\end{verbatim}

\begin{verbatim}
Warning in grid.Call(C_textBounds, as.graphicsAnnot(x$label), x$x, x$y, :
неизвестна ширина символа 0xec в кодировке CP1251
\end{verbatim}

\begin{verbatim}
Warning in grid.Call(C_textBounds, as.graphicsAnnot(x$label), x$x, x$y, :
неизвестна ширина символа 0xe8 в кодировке CP1251
\end{verbatim}

\begin{verbatim}
Warning in grid.Call(C_textBounds, as.graphicsAnnot(x$label), x$x, x$y, :
неизвестна ширина символа 0xf2 в кодировке CP1251
\end{verbatim}

\begin{verbatim}
Warning in grid.Call(C_textBounds, as.graphicsAnnot(x$label), x$x, x$y, :
неизвестна ширина символа 0xe0 в кодировке CP1251
\end{verbatim}

\begin{verbatim}
Warning in grid.Call(C_textBounds, as.graphicsAnnot(x$label), x$x, x$y, :
неизвестна ширина символа 0xf6 в кодировке CP1251
\end{verbatim}

\begin{verbatim}
Warning in grid.Call(C_textBounds, as.graphicsAnnot(x$label), x$x, x$y, :
неизвестна ширина символа 0xe8 в кодировке CP1251
\end{verbatim}

\begin{verbatim}
Warning in grid.Call(C_textBounds, as.graphicsAnnot(x$label), x$x, x$y, :
неизвестна ширина символа 0xee в кодировке CP1251
\end{verbatim}

\begin{verbatim}
Warning in grid.Call(C_textBounds, as.graphicsAnnot(x$label), x$x, x$y, :
неизвестна ширина символа 0xed в кодировке CP1251
Warning in grid.Call(C_textBounds, as.graphicsAnnot(x$label), x$x, x$y, :
неизвестна ширина символа 0xed в кодировке CP1251
\end{verbatim}

\begin{verbatim}
Warning in grid.Call(C_textBounds, as.graphicsAnnot(x$label), x$x, x$y, :
неизвестна ширина символа 0xee в кодировке CP1251
\end{verbatim}

\begin{verbatim}
Warning in grid.Call(C_textBounds, as.graphicsAnnot(x$label), x$x, x$y, :
неизвестна ширина символа 0xec в кодировке CP1251
Warning in grid.Call(C_textBounds, as.graphicsAnnot(x$label), x$x, x$y, :
неизвестна ширина символа 0xec в кодировке CP1251
\end{verbatim}

\begin{verbatim}
Warning in grid.Call(C_textBounds, as.graphicsAnnot(x$label), x$x, x$y, :
неизвестна ширина символа 0xee в кодировке CP1251
\end{verbatim}

\begin{verbatim}
Warning in grid.Call(C_textBounds, as.graphicsAnnot(x$label), x$x, x$y, :
неизвестна ширина символа 0xe4 в кодировке CP1251
\end{verbatim}

\begin{verbatim}
Warning in grid.Call(C_textBounds, as.graphicsAnnot(x$label), x$x, x$y, :
неизвестна ширина символа 0xe5 в кодировке CP1251
\end{verbatim}

\begin{verbatim}
Warning in grid.Call(C_textBounds, as.graphicsAnnot(x$label), x$x, x$y, :
неизвестна ширина символа 0xeb в кодировке CP1251
\end{verbatim}

\begin{verbatim}
Warning in grid.Call(C_textBounds, as.graphicsAnnot(x$label), x$x, x$y, :
неизвестна ширина символа 0xe8 в кодировке CP1251
\end{verbatim}

\begin{verbatim}
Warning in grid.Call(C_textBounds, as.graphicsAnnot(x$label), x$x, x$y, :
неизвестна ширина символа 0xf0 в кодировке CP1251
\end{verbatim}

\begin{verbatim}
Warning in grid.Call(C_textBounds, as.graphicsAnnot(x$label), x$x, x$y, :
неизвестна ширина символа 0xee в кодировке CP1251
\end{verbatim}

\begin{verbatim}
Warning in grid.Call(C_textBounds, as.graphicsAnnot(x$label), x$x, x$y, :
неизвестна ширина символа 0xe2 в кодировке CP1251
\end{verbatim}

\begin{verbatim}
Warning in grid.Call(C_textBounds, as.graphicsAnnot(x$label), x$x, x$y, :
неизвестна ширина символа 0xe0 в кодировке CP1251
\end{verbatim}

\begin{verbatim}
Warning in grid.Call(C_textBounds, as.graphicsAnnot(x$label), x$x, x$y, :
неизвестна ширина символа 0xed в кодировке CP1251
\end{verbatim}

\begin{verbatim}
Warning in grid.Call(C_textBounds, as.graphicsAnnot(x$label), x$x, x$y, :
неизвестна ширина символа 0xe8 в кодировке CP1251
Warning in grid.Call(C_textBounds, as.graphicsAnnot(x$label), x$x, x$y, :
неизвестна ширина символа 0xe8 в кодировке CP1251
\end{verbatim}

\begin{verbatim}
Warning in grid.Call.graphics(C_text, as.graphicsAnnot(x$label), x$x, x$y, :
неизвестна ширина символа 0xc1 в кодировке CP1251
\end{verbatim}

\begin{verbatim}
Warning in grid.Call.graphics(C_text, as.graphicsAnnot(x$label), x$x, x$y, :
неизвестна ширина символа 0xe8 в кодировке CP1251
\end{verbatim}

\begin{verbatim}
Warning in grid.Call.graphics(C_text, as.graphicsAnnot(x$label), x$x, x$y, :
неизвестна ширина символа 0xee в кодировке CP1251
\end{verbatim}

\begin{verbatim}
Warning in grid.Call.graphics(C_text, as.graphicsAnnot(x$label), x$x, x$y, :
неизвестна ширина символа 0xec в кодировке CP1251
\end{verbatim}

\begin{verbatim}
Warning in grid.Call.graphics(C_text, as.graphicsAnnot(x$label), x$x, x$y, :
неизвестна ширина символа 0xe0 в кодировке CP1251
\end{verbatim}

\begin{verbatim}
Warning in grid.Call.graphics(C_text, as.graphicsAnnot(x$label), x$x, x$y, :
неизвестна ширина символа 0xf1 в кодировке CP1251
Warning in grid.Call.graphics(C_text, as.graphicsAnnot(x$label), x$x, x$y, :
неизвестна ширина символа 0xf1 в кодировке CP1251
\end{verbatim}

\begin{verbatim}
Warning in grid.Call.graphics(C_text, as.graphicsAnnot(x$label), x$x, x$y, :
неизвестна ширина символа 0xe0 в кодировке CP1251
\end{verbatim}

\begin{verbatim}
Warning in grid.Call.graphics(C_text, as.graphicsAnnot(x$label), x$x, x$y, :
неизвестна ширина символа 0xd2 в кодировке CP1251
\end{verbatim}

\begin{verbatim}
Warning in grid.Call.graphics(C_text, as.graphicsAnnot(x$label), x$x, x$y, :
неизвестна ширина символа 0xf0 в кодировке CP1251
\end{verbatim}

\begin{verbatim}
Warning in grid.Call.graphics(C_text, as.graphicsAnnot(x$label), x$x, x$y, :
неизвестна ширина символа 0xe0 в кодировке CP1251
\end{verbatim}

\begin{verbatim}
Warning in grid.Call.graphics(C_text, as.graphicsAnnot(x$label), x$x, x$y, :
неизвестна ширина символа 0xe5 в кодировке CP1251
\end{verbatim}

\begin{verbatim}
Warning in grid.Call.graphics(C_text, as.graphicsAnnot(x$label), x$x, x$y, :
неизвестна ширина символа 0xea в кодировке CP1251
\end{verbatim}

\begin{verbatim}
Warning in grid.Call.graphics(C_text, as.graphicsAnnot(x$label), x$x, x$y, :
неизвестна ширина символа 0xf2 в кодировке CP1251
\end{verbatim}

\begin{verbatim}
Warning in grid.Call.graphics(C_text, as.graphicsAnnot(x$label), x$x, x$y, :
неизвестна ширина символа 0xee в кодировке CP1251
\end{verbatim}

\begin{verbatim}
Warning in grid.Call.graphics(C_text, as.graphicsAnnot(x$label), x$x, x$y, :
неизвестна ширина символа 0xf0 в кодировке CP1251
\end{verbatim}

\begin{verbatim}
Warning in grid.Call.graphics(C_text, as.graphicsAnnot(x$label), x$x, x$y, :
неизвестна ширина символа 0xe8 в кодировке CP1251
Warning in grid.Call.graphics(C_text, as.graphicsAnnot(x$label), x$x, x$y, :
неизвестна ширина символа 0xe8 в кодировке CP1251
\end{verbatim}

\begin{verbatim}
Warning in grid.Call.graphics(C_text, as.graphicsAnnot(x$label), x$x, x$y, :
неизвестна ширина символа 0xe1 в кодировке CP1251
\end{verbatim}

\begin{verbatim}
Warning in grid.Call.graphics(C_text, as.graphicsAnnot(x$label), x$x, x$y, :
неизвестна ширина символа 0xe8 в кодировке CP1251
\end{verbatim}

\begin{verbatim}
Warning in grid.Call.graphics(C_text, as.graphicsAnnot(x$label), x$x, x$y, :
неизвестна ширина символа 0xee в кодировке CP1251
\end{verbatim}

\begin{verbatim}
Warning in grid.Call.graphics(C_text, as.graphicsAnnot(x$label), x$x, x$y, :
неизвестна ширина символа 0xec в кодировке CP1251
\end{verbatim}

\begin{verbatim}
Warning in grid.Call.graphics(C_text, as.graphicsAnnot(x$label), x$x, x$y, :
неизвестна ширина символа 0xe0 в кодировке CP1251
\end{verbatim}

\begin{verbatim}
Warning in grid.Call.graphics(C_text, as.graphicsAnnot(x$label), x$x, x$y, :
неизвестна ширина символа 0xf1 в кодировке CP1251
Warning in grid.Call.graphics(C_text, as.graphicsAnnot(x$label), x$x, x$y, :
неизвестна ширина символа 0xf1 в кодировке CP1251
\end{verbatim}

\begin{verbatim}
Warning in grid.Call.graphics(C_text, as.graphicsAnnot(x$label), x$x, x$y, :
неизвестна ширина символа 0xfb в кодировке CP1251
\end{verbatim}

\begin{verbatim}
Warning in grid.Call.graphics(C_text, as.graphicsAnnot(x$label), x$x, x$y, :
неизвестна ширина символа 0xe2 в кодировке CP1251
\end{verbatim}

\begin{verbatim}
Warning in grid.Call.graphics(C_text, as.graphicsAnnot(x$label), x$x, x$y, :
неизвестна ширина символа 0xe8 в кодировке CP1251
\end{verbatim}

\begin{verbatim}
Warning in grid.Call.graphics(C_text, as.graphicsAnnot(x$label), x$x, x$y, :
неизвестна ширина символа 0xec в кодировке CP1251
\end{verbatim}

\begin{verbatim}
Warning in grid.Call.graphics(C_text, as.graphicsAnnot(x$label), x$x, x$y, :
неизвестна ширина символа 0xe8 в кодировке CP1251
\end{verbatim}

\begin{verbatim}
Warning in grid.Call.graphics(C_text, as.graphicsAnnot(x$label), x$x, x$y, :
неизвестна ширина символа 0xf2 в кодировке CP1251
\end{verbatim}

\begin{verbatim}
Warning in grid.Call.graphics(C_text, as.graphicsAnnot(x$label), x$x, x$y, :
неизвестна ширина символа 0xe0 в кодировке CP1251
\end{verbatim}

\begin{verbatim}
Warning in grid.Call.graphics(C_text, as.graphicsAnnot(x$label), x$x, x$y, :
неизвестна ширина символа 0xf6 в кодировке CP1251
\end{verbatim}

\begin{verbatim}
Warning in grid.Call.graphics(C_text, as.graphicsAnnot(x$label), x$x, x$y, :
неизвестна ширина символа 0xe8 в кодировке CP1251
\end{verbatim}

\begin{verbatim}
Warning in grid.Call.graphics(C_text, as.graphicsAnnot(x$label), x$x, x$y, :
неизвестна ширина символа 0xee в кодировке CP1251
\end{verbatim}

\begin{verbatim}
Warning in grid.Call.graphics(C_text, as.graphicsAnnot(x$label), x$x, x$y, :
неизвестна ширина символа 0xed в кодировке CP1251
Warning in grid.Call.graphics(C_text, as.graphicsAnnot(x$label), x$x, x$y, :
неизвестна ширина символа 0xed в кодировке CP1251
\end{verbatim}

\begin{verbatim}
Warning in grid.Call.graphics(C_text, as.graphicsAnnot(x$label), x$x, x$y, :
неизвестна ширина символа 0xee в кодировке CP1251
\end{verbatim}

\begin{verbatim}
Warning in grid.Call.graphics(C_text, as.graphicsAnnot(x$label), x$x, x$y, :
неизвестна ширина символа 0xec в кодировке CP1251
Warning in grid.Call.graphics(C_text, as.graphicsAnnot(x$label), x$x, x$y, :
неизвестна ширина символа 0xec в кодировке CP1251
\end{verbatim}

\begin{verbatim}
Warning in grid.Call.graphics(C_text, as.graphicsAnnot(x$label), x$x, x$y, :
неизвестна ширина символа 0xee в кодировке CP1251
\end{verbatim}

\begin{verbatim}
Warning in grid.Call.graphics(C_text, as.graphicsAnnot(x$label), x$x, x$y, :
неизвестна ширина символа 0xe4 в кодировке CP1251
\end{verbatim}

\begin{verbatim}
Warning in grid.Call.graphics(C_text, as.graphicsAnnot(x$label), x$x, x$y, :
неизвестна ширина символа 0xe5 в кодировке CP1251
\end{verbatim}

\begin{verbatim}
Warning in grid.Call.graphics(C_text, as.graphicsAnnot(x$label), x$x, x$y, :
неизвестна ширина символа 0xeb в кодировке CP1251
\end{verbatim}

\begin{verbatim}
Warning in grid.Call.graphics(C_text, as.graphicsAnnot(x$label), x$x, x$y, :
неизвестна ширина символа 0xe8 в кодировке CP1251
\end{verbatim}

\begin{verbatim}
Warning in grid.Call.graphics(C_text, as.graphicsAnnot(x$label), x$x, x$y, :
неизвестна ширина символа 0xf0 в кодировке CP1251
\end{verbatim}

\begin{verbatim}
Warning in grid.Call.graphics(C_text, as.graphicsAnnot(x$label), x$x, x$y, :
неизвестна ширина символа 0xee в кодировке CP1251
\end{verbatim}

\begin{verbatim}
Warning in grid.Call.graphics(C_text, as.graphicsAnnot(x$label), x$x, x$y, :
неизвестна ширина символа 0xe2 в кодировке CP1251
\end{verbatim}

\begin{verbatim}
Warning in grid.Call.graphics(C_text, as.graphicsAnnot(x$label), x$x, x$y, :
неизвестна ширина символа 0xe0 в кодировке CP1251
\end{verbatim}

\begin{verbatim}
Warning in grid.Call.graphics(C_text, as.graphicsAnnot(x$label), x$x, x$y, :
неизвестна ширина символа 0xed в кодировке CP1251
\end{verbatim}

\begin{verbatim}
Warning in grid.Call.graphics(C_text, as.graphicsAnnot(x$label), x$x, x$y, :
неизвестна ширина символа 0xe8 в кодировке CP1251
Warning in grid.Call.graphics(C_text, as.graphicsAnnot(x$label), x$x, x$y, :
неизвестна ширина символа 0xe8 в кодировке CP1251
\end{verbatim}

\pandocbounded{\includegraphics[keepaspectratio]{chapter16_files/figure-pdf/unnamed-chunk-1-4.pdf}}

\begin{Shaded}
\begin{Highlighting}[]
\CommentTok{\# Визуализация траекторий вылова}
\NormalTok{catch\_df\_proj }\OtherTok{\textless{}{-}} \FunctionTok{data.frame}\NormalTok{(}\AttributeTok{Year =} \FunctionTok{rep}\NormalTok{(}\DecValTok{1}\SpecialCharTok{:}\DecValTok{20}\NormalTok{, n\_sim),}
                            \AttributeTok{Catch =} \FunctionTok{as.vector}\NormalTok{(sim\_catch),}
                            \AttributeTok{Simulation =} \FunctionTok{rep}\NormalTok{(}\DecValTok{1}\SpecialCharTok{:}\NormalTok{n\_sim, }\AttributeTok{each =} \DecValTok{20}\NormalTok{))}

\FunctionTok{ggplot}\NormalTok{(catch\_df\_proj, }\FunctionTok{aes}\NormalTok{(}\AttributeTok{x =}\NormalTok{ Year, }\AttributeTok{y =}\NormalTok{ Catch, }\AttributeTok{group =}\NormalTok{ Simulation, }\AttributeTok{color =}\NormalTok{ Simulation)) }\SpecialCharTok{+}
  \FunctionTok{geom\_line}\NormalTok{(}\AttributeTok{linewidth =} \FloatTok{0.7}\NormalTok{) }\SpecialCharTok{+}
  \FunctionTok{geom\_hline}\NormalTok{(}\AttributeTok{yintercept =} \FunctionTok{median}\NormalTok{(msy\_results}\SpecialCharTok{$}\NormalTok{msy), }\AttributeTok{linetype =} \StringTok{"dashed"}\NormalTok{, }\AttributeTok{color =} \StringTok{"red"}\NormalTok{) }\SpecialCharTok{+}
  \FunctionTok{labs}\NormalTok{(}\AttributeTok{title =} \StringTok{"Траектории вылова в имитационном моделировании"}\NormalTok{,}
       \AttributeTok{y =} \StringTok{"Вылов"}\NormalTok{) }\SpecialCharTok{+}
  \FunctionTok{theme\_minimal}\NormalTok{() }\SpecialCharTok{+}
  \FunctionTok{theme}\NormalTok{(}\AttributeTok{plot.title =} \FunctionTok{element\_text}\NormalTok{(}\AttributeTok{hjust =} \FloatTok{0.5}\NormalTok{))}
\end{Highlighting}
\end{Shaded}

\begin{verbatim}
Warning in grid.Call(C_textBounds, as.graphicsAnnot(x$label), x$x, x$y, :
неизвестна ширина символа 0xc2 в кодировке CP1251
\end{verbatim}

\begin{verbatim}
Warning in grid.Call(C_textBounds, as.graphicsAnnot(x$label), x$x, x$y, :
неизвестна ширина символа 0xfb в кодировке CP1251
\end{verbatim}

\begin{verbatim}
Warning in grid.Call(C_textBounds, as.graphicsAnnot(x$label), x$x, x$y, :
неизвестна ширина символа 0xeb в кодировке CP1251
\end{verbatim}

\begin{verbatim}
Warning in grid.Call(C_textBounds, as.graphicsAnnot(x$label), x$x, x$y, :
неизвестна ширина символа 0xee в кодировке CP1251
\end{verbatim}

\begin{verbatim}
Warning in grid.Call(C_textBounds, as.graphicsAnnot(x$label), x$x, x$y, :
неизвестна ширина символа 0xe2 в кодировке CP1251
\end{verbatim}

\begin{verbatim}
Warning in grid.Call(C_textBounds, as.graphicsAnnot(x$label), x$x, x$y, :
неизвестна ширина символа 0xd2 в кодировке CP1251
\end{verbatim}

\begin{verbatim}
Warning in grid.Call(C_textBounds, as.graphicsAnnot(x$label), x$x, x$y, :
неизвестна ширина символа 0xf0 в кодировке CP1251
\end{verbatim}

\begin{verbatim}
Warning in grid.Call(C_textBounds, as.graphicsAnnot(x$label), x$x, x$y, :
неизвестна ширина символа 0xe0 в кодировке CP1251
\end{verbatim}

\begin{verbatim}
Warning in grid.Call(C_textBounds, as.graphicsAnnot(x$label), x$x, x$y, :
неизвестна ширина символа 0xe5 в кодировке CP1251
\end{verbatim}

\begin{verbatim}
Warning in grid.Call(C_textBounds, as.graphicsAnnot(x$label), x$x, x$y, :
неизвестна ширина символа 0xea в кодировке CP1251
\end{verbatim}

\begin{verbatim}
Warning in grid.Call(C_textBounds, as.graphicsAnnot(x$label), x$x, x$y, :
неизвестна ширина символа 0xf2 в кодировке CP1251
\end{verbatim}

\begin{verbatim}
Warning in grid.Call(C_textBounds, as.graphicsAnnot(x$label), x$x, x$y, :
неизвестна ширина символа 0xee в кодировке CP1251
\end{verbatim}

\begin{verbatim}
Warning in grid.Call(C_textBounds, as.graphicsAnnot(x$label), x$x, x$y, :
неизвестна ширина символа 0xf0 в кодировке CP1251
\end{verbatim}

\begin{verbatim}
Warning in grid.Call(C_textBounds, as.graphicsAnnot(x$label), x$x, x$y, :
неизвестна ширина символа 0xe8 в кодировке CP1251
Warning in grid.Call(C_textBounds, as.graphicsAnnot(x$label), x$x, x$y, :
неизвестна ширина символа 0xe8 в кодировке CP1251
\end{verbatim}

\begin{verbatim}
Warning in grid.Call(C_textBounds, as.graphicsAnnot(x$label), x$x, x$y, :
неизвестна ширина символа 0xe2 в кодировке CP1251
\end{verbatim}

\begin{verbatim}
Warning in grid.Call(C_textBounds, as.graphicsAnnot(x$label), x$x, x$y, :
неизвестна ширина символа 0xfb в кодировке CP1251
\end{verbatim}

\begin{verbatim}
Warning in grid.Call(C_textBounds, as.graphicsAnnot(x$label), x$x, x$y, :
неизвестна ширина символа 0xeb в кодировке CP1251
\end{verbatim}

\begin{verbatim}
Warning in grid.Call(C_textBounds, as.graphicsAnnot(x$label), x$x, x$y, :
неизвестна ширина символа 0xee в кодировке CP1251
\end{verbatim}

\begin{verbatim}
Warning in grid.Call(C_textBounds, as.graphicsAnnot(x$label), x$x, x$y, :
неизвестна ширина символа 0xe2 в кодировке CP1251
\end{verbatim}

\begin{verbatim}
Warning in grid.Call(C_textBounds, as.graphicsAnnot(x$label), x$x, x$y, :
неизвестна ширина символа 0xe0 в кодировке CP1251
\end{verbatim}

\begin{verbatim}
Warning in grid.Call(C_textBounds, as.graphicsAnnot(x$label), x$x, x$y, :
неизвестна ширина символа 0xe2 в кодировке CP1251
\end{verbatim}

\begin{verbatim}
Warning in grid.Call(C_textBounds, as.graphicsAnnot(x$label), x$x, x$y, :
неизвестна ширина символа 0xe8 в кодировке CP1251
\end{verbatim}

\begin{verbatim}
Warning in grid.Call(C_textBounds, as.graphicsAnnot(x$label), x$x, x$y, :
неизвестна ширина символа 0xec в кодировке CP1251
\end{verbatim}

\begin{verbatim}
Warning in grid.Call(C_textBounds, as.graphicsAnnot(x$label), x$x, x$y, :
неизвестна ширина символа 0xe8 в кодировке CP1251
\end{verbatim}

\begin{verbatim}
Warning in grid.Call(C_textBounds, as.graphicsAnnot(x$label), x$x, x$y, :
неизвестна ширина символа 0xf2 в кодировке CP1251
\end{verbatim}

\begin{verbatim}
Warning in grid.Call(C_textBounds, as.graphicsAnnot(x$label), x$x, x$y, :
неизвестна ширина символа 0xe0 в кодировке CP1251
\end{verbatim}

\begin{verbatim}
Warning in grid.Call(C_textBounds, as.graphicsAnnot(x$label), x$x, x$y, :
неизвестна ширина символа 0xf6 в кодировке CP1251
\end{verbatim}

\begin{verbatim}
Warning in grid.Call(C_textBounds, as.graphicsAnnot(x$label), x$x, x$y, :
неизвестна ширина символа 0xe8 в кодировке CP1251
\end{verbatim}

\begin{verbatim}
Warning in grid.Call(C_textBounds, as.graphicsAnnot(x$label), x$x, x$y, :
неизвестна ширина символа 0xee в кодировке CP1251
\end{verbatim}

\begin{verbatim}
Warning in grid.Call(C_textBounds, as.graphicsAnnot(x$label), x$x, x$y, :
неизвестна ширина символа 0xed в кодировке CP1251
Warning in grid.Call(C_textBounds, as.graphicsAnnot(x$label), x$x, x$y, :
неизвестна ширина символа 0xed в кодировке CP1251
\end{verbatim}

\begin{verbatim}
Warning in grid.Call(C_textBounds, as.graphicsAnnot(x$label), x$x, x$y, :
неизвестна ширина символа 0xee в кодировке CP1251
\end{verbatim}

\begin{verbatim}
Warning in grid.Call(C_textBounds, as.graphicsAnnot(x$label), x$x, x$y, :
неизвестна ширина символа 0xec в кодировке CP1251
Warning in grid.Call(C_textBounds, as.graphicsAnnot(x$label), x$x, x$y, :
неизвестна ширина символа 0xec в кодировке CP1251
\end{verbatim}

\begin{verbatim}
Warning in grid.Call(C_textBounds, as.graphicsAnnot(x$label), x$x, x$y, :
неизвестна ширина символа 0xee в кодировке CP1251
\end{verbatim}

\begin{verbatim}
Warning in grid.Call(C_textBounds, as.graphicsAnnot(x$label), x$x, x$y, :
неизвестна ширина символа 0xe4 в кодировке CP1251
\end{verbatim}

\begin{verbatim}
Warning in grid.Call(C_textBounds, as.graphicsAnnot(x$label), x$x, x$y, :
неизвестна ширина символа 0xe5 в кодировке CP1251
\end{verbatim}

\begin{verbatim}
Warning in grid.Call(C_textBounds, as.graphicsAnnot(x$label), x$x, x$y, :
неизвестна ширина символа 0xeb в кодировке CP1251
\end{verbatim}

\begin{verbatim}
Warning in grid.Call(C_textBounds, as.graphicsAnnot(x$label), x$x, x$y, :
неизвестна ширина символа 0xe8 в кодировке CP1251
\end{verbatim}

\begin{verbatim}
Warning in grid.Call(C_textBounds, as.graphicsAnnot(x$label), x$x, x$y, :
неизвестна ширина символа 0xf0 в кодировке CP1251
\end{verbatim}

\begin{verbatim}
Warning in grid.Call(C_textBounds, as.graphicsAnnot(x$label), x$x, x$y, :
неизвестна ширина символа 0xee в кодировке CP1251
\end{verbatim}

\begin{verbatim}
Warning in grid.Call(C_textBounds, as.graphicsAnnot(x$label), x$x, x$y, :
неизвестна ширина символа 0xe2 в кодировке CP1251
\end{verbatim}

\begin{verbatim}
Warning in grid.Call(C_textBounds, as.graphicsAnnot(x$label), x$x, x$y, :
неизвестна ширина символа 0xe0 в кодировке CP1251
\end{verbatim}

\begin{verbatim}
Warning in grid.Call(C_textBounds, as.graphicsAnnot(x$label), x$x, x$y, :
неизвестна ширина символа 0xed в кодировке CP1251
\end{verbatim}

\begin{verbatim}
Warning in grid.Call(C_textBounds, as.graphicsAnnot(x$label), x$x, x$y, :
неизвестна ширина символа 0xe8 в кодировке CP1251
Warning in grid.Call(C_textBounds, as.graphicsAnnot(x$label), x$x, x$y, :
неизвестна ширина символа 0xe8 в кодировке CP1251
\end{verbatim}

\begin{verbatim}
Warning in grid.Call.graphics(C_text, as.graphicsAnnot(x$label), x$x, x$y, :
неизвестна ширина символа 0xc2 в кодировке CP1251
\end{verbatim}

\begin{verbatim}
Warning in grid.Call.graphics(C_text, as.graphicsAnnot(x$label), x$x, x$y, :
неизвестна ширина символа 0xfb в кодировке CP1251
\end{verbatim}

\begin{verbatim}
Warning in grid.Call.graphics(C_text, as.graphicsAnnot(x$label), x$x, x$y, :
неизвестна ширина символа 0xeb в кодировке CP1251
\end{verbatim}

\begin{verbatim}
Warning in grid.Call.graphics(C_text, as.graphicsAnnot(x$label), x$x, x$y, :
неизвестна ширина символа 0xee в кодировке CP1251
\end{verbatim}

\begin{verbatim}
Warning in grid.Call.graphics(C_text, as.graphicsAnnot(x$label), x$x, x$y, :
неизвестна ширина символа 0xe2 в кодировке CP1251
\end{verbatim}

\begin{verbatim}
Warning in grid.Call.graphics(C_text, as.graphicsAnnot(x$label), x$x, x$y, :
неизвестна ширина символа 0xd2 в кодировке CP1251
\end{verbatim}

\begin{verbatim}
Warning in grid.Call.graphics(C_text, as.graphicsAnnot(x$label), x$x, x$y, :
неизвестна ширина символа 0xf0 в кодировке CP1251
\end{verbatim}

\begin{verbatim}
Warning in grid.Call.graphics(C_text, as.graphicsAnnot(x$label), x$x, x$y, :
неизвестна ширина символа 0xe0 в кодировке CP1251
\end{verbatim}

\begin{verbatim}
Warning in grid.Call.graphics(C_text, as.graphicsAnnot(x$label), x$x, x$y, :
неизвестна ширина символа 0xe5 в кодировке CP1251
\end{verbatim}

\begin{verbatim}
Warning in grid.Call.graphics(C_text, as.graphicsAnnot(x$label), x$x, x$y, :
неизвестна ширина символа 0xea в кодировке CP1251
\end{verbatim}

\begin{verbatim}
Warning in grid.Call.graphics(C_text, as.graphicsAnnot(x$label), x$x, x$y, :
неизвестна ширина символа 0xf2 в кодировке CP1251
\end{verbatim}

\begin{verbatim}
Warning in grid.Call.graphics(C_text, as.graphicsAnnot(x$label), x$x, x$y, :
неизвестна ширина символа 0xee в кодировке CP1251
\end{verbatim}

\begin{verbatim}
Warning in grid.Call.graphics(C_text, as.graphicsAnnot(x$label), x$x, x$y, :
неизвестна ширина символа 0xf0 в кодировке CP1251
\end{verbatim}

\begin{verbatim}
Warning in grid.Call.graphics(C_text, as.graphicsAnnot(x$label), x$x, x$y, :
неизвестна ширина символа 0xe8 в кодировке CP1251
Warning in grid.Call.graphics(C_text, as.graphicsAnnot(x$label), x$x, x$y, :
неизвестна ширина символа 0xe8 в кодировке CP1251
\end{verbatim}

\begin{verbatim}
Warning in grid.Call.graphics(C_text, as.graphicsAnnot(x$label), x$x, x$y, :
неизвестна ширина символа 0xe2 в кодировке CP1251
\end{verbatim}

\begin{verbatim}
Warning in grid.Call.graphics(C_text, as.graphicsAnnot(x$label), x$x, x$y, :
неизвестна ширина символа 0xfb в кодировке CP1251
\end{verbatim}

\begin{verbatim}
Warning in grid.Call.graphics(C_text, as.graphicsAnnot(x$label), x$x, x$y, :
неизвестна ширина символа 0xeb в кодировке CP1251
\end{verbatim}

\begin{verbatim}
Warning in grid.Call.graphics(C_text, as.graphicsAnnot(x$label), x$x, x$y, :
неизвестна ширина символа 0xee в кодировке CP1251
\end{verbatim}

\begin{verbatim}
Warning in grid.Call.graphics(C_text, as.graphicsAnnot(x$label), x$x, x$y, :
неизвестна ширина символа 0xe2 в кодировке CP1251
\end{verbatim}

\begin{verbatim}
Warning in grid.Call.graphics(C_text, as.graphicsAnnot(x$label), x$x, x$y, :
неизвестна ширина символа 0xe0 в кодировке CP1251
\end{verbatim}

\begin{verbatim}
Warning in grid.Call.graphics(C_text, as.graphicsAnnot(x$label), x$x, x$y, :
неизвестна ширина символа 0xe2 в кодировке CP1251
\end{verbatim}

\begin{verbatim}
Warning in grid.Call.graphics(C_text, as.graphicsAnnot(x$label), x$x, x$y, :
неизвестна ширина символа 0xe8 в кодировке CP1251
\end{verbatim}

\begin{verbatim}
Warning in grid.Call.graphics(C_text, as.graphicsAnnot(x$label), x$x, x$y, :
неизвестна ширина символа 0xec в кодировке CP1251
\end{verbatim}

\begin{verbatim}
Warning in grid.Call.graphics(C_text, as.graphicsAnnot(x$label), x$x, x$y, :
неизвестна ширина символа 0xe8 в кодировке CP1251
\end{verbatim}

\begin{verbatim}
Warning in grid.Call.graphics(C_text, as.graphicsAnnot(x$label), x$x, x$y, :
неизвестна ширина символа 0xf2 в кодировке CP1251
\end{verbatim}

\begin{verbatim}
Warning in grid.Call.graphics(C_text, as.graphicsAnnot(x$label), x$x, x$y, :
неизвестна ширина символа 0xe0 в кодировке CP1251
\end{verbatim}

\begin{verbatim}
Warning in grid.Call.graphics(C_text, as.graphicsAnnot(x$label), x$x, x$y, :
неизвестна ширина символа 0xf6 в кодировке CP1251
\end{verbatim}

\begin{verbatim}
Warning in grid.Call.graphics(C_text, as.graphicsAnnot(x$label), x$x, x$y, :
неизвестна ширина символа 0xe8 в кодировке CP1251
\end{verbatim}

\begin{verbatim}
Warning in grid.Call.graphics(C_text, as.graphicsAnnot(x$label), x$x, x$y, :
неизвестна ширина символа 0xee в кодировке CP1251
\end{verbatim}

\begin{verbatim}
Warning in grid.Call.graphics(C_text, as.graphicsAnnot(x$label), x$x, x$y, :
неизвестна ширина символа 0xed в кодировке CP1251
Warning in grid.Call.graphics(C_text, as.graphicsAnnot(x$label), x$x, x$y, :
неизвестна ширина символа 0xed в кодировке CP1251
\end{verbatim}

\begin{verbatim}
Warning in grid.Call.graphics(C_text, as.graphicsAnnot(x$label), x$x, x$y, :
неизвестна ширина символа 0xee в кодировке CP1251
\end{verbatim}

\begin{verbatim}
Warning in grid.Call.graphics(C_text, as.graphicsAnnot(x$label), x$x, x$y, :
неизвестна ширина символа 0xec в кодировке CP1251
Warning in grid.Call.graphics(C_text, as.graphicsAnnot(x$label), x$x, x$y, :
неизвестна ширина символа 0xec в кодировке CP1251
\end{verbatim}

\begin{verbatim}
Warning in grid.Call.graphics(C_text, as.graphicsAnnot(x$label), x$x, x$y, :
неизвестна ширина символа 0xee в кодировке CP1251
\end{verbatim}

\begin{verbatim}
Warning in grid.Call.graphics(C_text, as.graphicsAnnot(x$label), x$x, x$y, :
неизвестна ширина символа 0xe4 в кодировке CP1251
\end{verbatim}

\begin{verbatim}
Warning in grid.Call.graphics(C_text, as.graphicsAnnot(x$label), x$x, x$y, :
неизвестна ширина символа 0xe5 в кодировке CP1251
\end{verbatim}

\begin{verbatim}
Warning in grid.Call.graphics(C_text, as.graphicsAnnot(x$label), x$x, x$y, :
неизвестна ширина символа 0xeb в кодировке CP1251
\end{verbatim}

\begin{verbatim}
Warning in grid.Call.graphics(C_text, as.graphicsAnnot(x$label), x$x, x$y, :
неизвестна ширина символа 0xe8 в кодировке CP1251
\end{verbatim}

\begin{verbatim}
Warning in grid.Call.graphics(C_text, as.graphicsAnnot(x$label), x$x, x$y, :
неизвестна ширина символа 0xf0 в кодировке CP1251
\end{verbatim}

\begin{verbatim}
Warning in grid.Call.graphics(C_text, as.graphicsAnnot(x$label), x$x, x$y, :
неизвестна ширина символа 0xee в кодировке CP1251
\end{verbatim}

\begin{verbatim}
Warning in grid.Call.graphics(C_text, as.graphicsAnnot(x$label), x$x, x$y, :
неизвестна ширина символа 0xe2 в кодировке CP1251
\end{verbatim}

\begin{verbatim}
Warning in grid.Call.graphics(C_text, as.graphicsAnnot(x$label), x$x, x$y, :
неизвестна ширина символа 0xe0 в кодировке CP1251
\end{verbatim}

\begin{verbatim}
Warning in grid.Call.graphics(C_text, as.graphicsAnnot(x$label), x$x, x$y, :
неизвестна ширина символа 0xed в кодировке CP1251
\end{verbatim}

\begin{verbatim}
Warning in grid.Call.graphics(C_text, as.graphicsAnnot(x$label), x$x, x$y, :
неизвестна ширина символа 0xe8 в кодировке CP1251
Warning in grid.Call.graphics(C_text, as.graphicsAnnot(x$label), x$x, x$y, :
неизвестна ширина символа 0xe8 в кодировке CP1251
\end{verbatim}

\pandocbounded{\includegraphics[keepaspectratio]{chapter16_files/figure-pdf/unnamed-chunk-1-5.pdf}}

\begin{Shaded}
\begin{Highlighting}[]
\CommentTok{\# Анализ производительности}
\NormalTok{performance\_metrics }\OtherTok{\textless{}{-}} \FunctionTok{data.frame}\NormalTok{(}
  \AttributeTok{Simulation =} \DecValTok{1}\SpecialCharTok{:}\NormalTok{n\_sim,}
  \AttributeTok{AvgCatch =} \FunctionTok{apply}\NormalTok{(sim\_catch, }\DecValTok{2}\NormalTok{, mean),}
  \AttributeTok{AvgBiomass =} \FunctionTok{apply}\NormalTok{(sim\_biomass[}\SpecialCharTok{{-}}\DecValTok{1}\NormalTok{,], }\DecValTok{2}\NormalTok{, mean),}
  \AttributeTok{MinBiomass =} \FunctionTok{apply}\NormalTok{(sim\_biomass, }\DecValTok{2}\NormalTok{, min),}
  \AttributeTok{FinalBiomass =}\NormalTok{ sim\_biomass[}\DecValTok{21}\NormalTok{,]}
\NormalTok{)}

\FunctionTok{print}\NormalTok{(}\StringTok{"Метрики производительности управления:"}\NormalTok{)}
\end{Highlighting}
\end{Shaded}

\begin{verbatim}
[1] "Метрики производительности управления:"
\end{verbatim}

\begin{Shaded}
\begin{Highlighting}[]
\FunctionTok{print}\NormalTok{(performance\_metrics)}
\end{Highlighting}
\end{Shaded}

\begin{verbatim}
   Simulation  AvgCatch AvgBiomass MinBiomass FinalBiomass
1           1 10.414521   216.5646   202.4008     226.5302
2           2 13.443319   276.0821   261.9283     261.9283
3           3 11.220346   232.5633   226.0823     236.9172
4           4 13.940186   285.6792   266.8545     266.8545
5           5 14.231859   291.2896   269.6423     269.6423
6           6  8.363161   191.6799   155.1896     210.7870
7           7 11.962992   247.2046   245.8428     245.8428
8           8 13.988406   286.6079   267.3206     267.3206
9           9 12.103308   249.9596   247.4627     247.4627
10         10 11.522725   238.5368   235.3750     240.6241
\end{verbatim}

\begin{Shaded}
\begin{Highlighting}[]
\CommentTok{\# {-}{-}{-}{-}{-}{-}{-}{-}{-}{-}{-}{-}{-}{-}{-}{-}{-}{-}{-}{-}{-}{-}{-}{-}{-} 9. ФОРМУЛИРОВАНИЕ РЕКОМЕНДАЦИЙ {-}{-}{-}{-}{-}{-}{-}{-}{-}{-}{-}{-}{-}{-}{-}{-}{-}{-}{-}{-}{-}{-}{-}{-}{-}{-}{-}}

\CommentTok{\# Формулируем рекомендации на основе анализа}
\NormalTok{median\_msy }\OtherTok{\textless{}{-}} \FunctionTok{median}\NormalTok{(msy\_results}\SpecialCharTok{$}\NormalTok{msy)}
\NormalTok{current\_catch }\OtherTok{\textless{}{-}} \FunctionTok{mean}\NormalTok{(Catch[(}\FunctionTok{length}\NormalTok{(Catch)}\SpecialCharTok{{-}}\DecValTok{2}\NormalTok{)}\SpecialCharTok{:}\FunctionTok{length}\NormalTok{(Catch)])}

\FunctionTok{cat}\NormalTok{(}\StringTok{"АНАЛИЗ РЕЗУЛЬТАТОВ И РЕКОМЕНДАЦИИ:}\SpecialCharTok{\textbackslash{}n}\StringTok{"}\NormalTok{)}
\end{Highlighting}
\end{Shaded}

\begin{verbatim}
АНАЛИЗ РЕЗУЛЬТАТОВ И РЕКОМЕНДАЦИИ:
\end{verbatim}

\begin{Shaded}
\begin{Highlighting}[]
\FunctionTok{cat}\NormalTok{(}\StringTok{"=============================================}\SpecialCharTok{\textbackslash{}n\textbackslash{}n}\StringTok{"}\NormalTok{)}
\end{Highlighting}
\end{Shaded}

\begin{verbatim}
=============================================
\end{verbatim}

\begin{Shaded}
\begin{Highlighting}[]
\FunctionTok{cat}\NormalTok{(}\StringTok{"1. Оценка MSY методом Catch{-}MSY: "}\NormalTok{, }\FunctionTok{round}\NormalTok{(median\_msy, }\DecValTok{2}\NormalTok{), }\StringTok{"тыс. тонн}\SpecialCharTok{\textbackslash{}n}\StringTok{"}\NormalTok{)}
\end{Highlighting}
\end{Shaded}

\begin{verbatim}
1. Оценка MSY методом Catch-MSY:  13.08 тыс. тонн
\end{verbatim}

\begin{Shaded}
\begin{Highlighting}[]
\FunctionTok{cat}\NormalTok{(}\StringTok{"2. Текущий уровень вылова: "}\NormalTok{, }\FunctionTok{round}\NormalTok{(current\_catch, }\DecValTok{2}\NormalTok{), }\StringTok{"тыс. тонн}\SpecialCharTok{\textbackslash{}n}\StringTok{"}\NormalTok{)}
\end{Highlighting}
\end{Shaded}

\begin{verbatim}
2. Текущий уровень вылова:  12 тыс. тонн
\end{verbatim}

\begin{Shaded}
\begin{Highlighting}[]
\FunctionTok{cat}\NormalTok{(}\StringTok{"3. Отношение вылова к MSY: "}\NormalTok{, }\FunctionTok{round}\NormalTok{(current\_catch }\SpecialCharTok{/}\NormalTok{ median\_msy, }\DecValTok{2}\NormalTok{), }\StringTok{"}\SpecialCharTok{\textbackslash{}n}\StringTok{"}\NormalTok{)}
\end{Highlighting}
\end{Shaded}

\begin{verbatim}
3. Отношение вылова к MSY:  0.92 
\end{verbatim}

\begin{Shaded}
\begin{Highlighting}[]
\FunctionTok{cat}\NormalTok{(}\StringTok{"4. Текущая деплетированность запаса (B/K): "}\NormalTok{, }\FunctionTok{round}\NormalTok{(}\FunctionTok{median}\NormalTok{(msy\_results}\SpecialCharTok{$}\NormalTok{depletion), }\DecValTok{2}\NormalTok{), }\StringTok{"}\SpecialCharTok{\textbackslash{}n\textbackslash{}n}\StringTok{"}\NormalTok{)}
\end{Highlighting}
\end{Shaded}

\begin{verbatim}
4. Текущая деплетированность запаса (B/K):  3.13 
\end{verbatim}

\begin{Shaded}
\begin{Highlighting}[]
\ControlFlowTok{if}\NormalTok{ (current\_catch }\SpecialCharTok{/}\NormalTok{ median\_msy }\SpecialCharTok{\textgreater{}} \DecValTok{1}\NormalTok{) \{}
  \FunctionTok{cat}\NormalTok{(}\StringTok{"ВЫВОД: Текущий уровень вылова превышает оценку MSY. Рекомендуется сокращение вылова.}\SpecialCharTok{\textbackslash{}n}\StringTok{"}\NormalTok{)}
\NormalTok{\} }\ControlFlowTok{else} \ControlFlowTok{if}\NormalTok{ (current\_catch }\SpecialCharTok{/}\NormalTok{ median\_msy }\SpecialCharTok{\textgreater{}} \FloatTok{0.8}\NormalTok{) \{}
  \FunctionTok{cat}\NormalTok{(}\StringTok{"ВЫВОД: Текущий уровень вылова близок к MSY. Рекомендуется осторожный подход.}\SpecialCharTok{\textbackslash{}n}\StringTok{"}\NormalTok{)}
\NormalTok{\} }\ControlFlowTok{else}\NormalTok{ \{}
  \FunctionTok{cat}\NormalTok{(}\StringTok{"ВЫВОД: Текущий уровень вылова ниже MSY. Возможно увеличение вылова.}\SpecialCharTok{\textbackslash{}n}\StringTok{"}\NormalTok{)}
\NormalTok{\}}
\end{Highlighting}
\end{Shaded}

\begin{verbatim}
ВЫВОД: Текущий уровень вылова близок к MSY. Рекомендуется осторожный подход.
\end{verbatim}

\begin{Shaded}
\begin{Highlighting}[]
\FunctionTok{cat}\NormalTok{(}\StringTok{"}\SpecialCharTok{\textbackslash{}n}\StringTok{РЕКОМЕНДАЦИИ ПО УПРАВЛЕНИЮ ЗАПАСОМ:}\SpecialCharTok{\textbackslash{}n}\StringTok{"}\NormalTok{)}
\end{Highlighting}
\end{Shaded}

\begin{verbatim}

РЕКОМЕНДАЦИИ ПО УПРАВЛЕНИЮ ЗАПАСОМ:
\end{verbatim}

\begin{Shaded}
\begin{Highlighting}[]
\FunctionTok{cat}\NormalTok{(}\StringTok{"{-} Установить целевой уровень вылова в диапазоне "}\NormalTok{, }\FunctionTok{round}\NormalTok{(median\_msy }\SpecialCharTok{*} \FloatTok{0.8}\NormalTok{, }\DecValTok{2}\NormalTok{), }\StringTok{"{-}"}\NormalTok{, }
    \FunctionTok{round}\NormalTok{(median\_msy, }\DecValTok{2}\NormalTok{), }\StringTok{"тыс. тонн}\SpecialCharTok{\textbackslash{}n}\StringTok{"}\NormalTok{)}
\end{Highlighting}
\end{Shaded}

\begin{verbatim}
- Установить целевой уровень вылова в диапазоне  10.47 - 13.08 тыс. тонн
\end{verbatim}

\begin{Shaded}
\begin{Highlighting}[]
\FunctionTok{cat}\NormalTok{(}\StringTok{"{-} Внедрить правило управления (HCR) для адаптивного регулирования вылова}\SpecialCharTok{\textbackslash{}n}\StringTok{"}\NormalTok{)}
\end{Highlighting}
\end{Shaded}

\begin{verbatim}
- Внедрить правило управления (HCR) для адаптивного регулирования вылова
\end{verbatim}

\begin{Shaded}
\begin{Highlighting}[]
\FunctionTok{cat}\NormalTok{(}\StringTok{"{-} Усилить мониторинг запаса для улучшения оценок}\SpecialCharTok{\textbackslash{}n}\StringTok{"}\NormalTok{)}
\end{Highlighting}
\end{Shaded}

\begin{verbatim}
- Усилить мониторинг запаса для улучшения оценок
\end{verbatim}

\begin{Shaded}
\begin{Highlighting}[]
\FunctionTok{cat}\NormalTok{(}\StringTok{"{-} Проводить регулярную оценку запаса с обновлением рекомендаций}\SpecialCharTok{\textbackslash{}n}\StringTok{"}\NormalTok{)}
\end{Highlighting}
\end{Shaded}

\begin{verbatim}
- Проводить регулярную оценку запаса с обновлением рекомендаций
\end{verbatim}

\begin{Shaded}
\begin{Highlighting}[]
\CommentTok{\# =============================================================================}
\CommentTok{\# КОНЕЦ СКРИПТА}
\CommentTok{\# =============================================================================}
\end{Highlighting}
\end{Shaded}

\section{Анализ результатов применения метода Catch-MSY: между надеждой
и статистической
реальностью}\label{ux430ux43dux430ux43bux438ux437-ux440ux435ux437ux443ux43bux44cux442ux430ux442ux43eux432-ux43fux440ux438ux43cux435ux43dux435ux43dux438ux44f-ux43cux435ux442ux43eux434ux430-catch-msy-ux43cux435ux436ux434ux443-ux43dux430ux434ux435ux436ux434ux43eux439-ux438-ux441ux442ux430ux442ux438ux441ux442ux438ux447ux435ux441ux43aux43eux439-ux440ux435ux430ux43bux44cux43dux43eux441ux442ux44cux44e}

Что ж, давайте посмотрим, что у нас получилось в результате выполнения
этого скрипта. Картина вырисовывается интересная, хотя и не без типичных
для DLM сюрпризов.

Начнем с того, что наш метод Catch-MSY отработал как швейцарские часы
--- если, конечно, считать нормальным тот факт, что из 100000 итераций у
нас осталось всего несколько тысяч жизнеспособных комбинаций параметров.
Но это как раз и есть философия метода: мы не пытаемся найти единственно
верное решение, а отбираем все биологически возможные сценарии.

Полученные оценки параметров говорят сами за себя. Коэффициент роста
\emph{r} со средним значением 0.14 --- это вполне типично для многих
промысловых видов. Хотя, если быть до конца честным, разброс значений от
0.01 до 1.5 и стандартное отклонение 0.13 намекают, что наша уверенность
в этом параметре несколько преувеличена.

Емкость среды \emph{K} оказалась в районе 555 тысяч тонн в среднем, что
при медианном \emph{MSY} в 13 тысяч тонн дает нам примерное
представление о потенциале запаса. Любопытно, что медианное значение
\emph{MSY} практически совпадает со средним --- 13.08 против 12.75 тысяч
тонн, что намекает на относительно симметричное распределение.

А теперь самое интересное --- текущий статус запаса. Средний вылов за
последние три года составляет 12 тысяч тонн против оцененного \emph{MSY}
в 13.08. Отношение 0.92 --- это тот самый момент, когда начинается
настоящая игра в рулетку управления. Формально мы еще не превысили
\emph{MSY}, но мы уже в опасной близости от границы.

Но настоящий сюрприз ждал нас в оценке ``деплетированности''. Медианное
значение 3.13 --- это, конечно, статистический нонсенс. Биомасса не
может превышать емкость среды, если только мы не открыли новый закон
популяционной динамики. Этот артефакт --- классический пример
ограничений метода Catch-MSY, когда при определенных комбинациях
параметров модель выдает биологически невозможные значения.

Ориентиры управления выглядят более-менее разумно: \emph{MSY} 13.08 тыс.
тонн, \emph{B\textsubscript{msy}} 243.8 тыс. тонн,
\emph{F\textsubscript{msy}} 0.048. Последнее значение особенно интересно
--- оно предполагает, что оптимальный уровень изъятия составляет около
5\% от биомассы.

Имитационное моделирование управления показало, что при использовании
предложенного правлила регулирования промысла (ПРП) средний вылов
колеблется от 8 до 14 тысяч тонн в зависимости от начальных условий. Что
характерно, даже в наихудших сценариях запас не коллапсирует --- система
управления оказывается достаточно устойчивой.

Графики траекторий биомассы и вылова --- это настоящая поэма в данных.
Мы видим, как различные сценарии сходятся к устойчивому состоянию около
\emph{B\textsubscript{msy}}, что собственно и является целью управления.
Хотя некоторые симуляции показывают излишне консервативные подходы с
выловом ниже потенциально возможного.

\section{Заключение: между статистикой и управленческой
реальностью}\label{ux437ux430ux43aux43bux44eux447ux435ux43dux438ux435-ux43cux435ux436ux434ux443-ux441ux442ux430ux442ux438ux441ux442ux438ux43aux43eux439-ux438-ux443ux43fux440ux430ux432ux43bux435ux43dux447ux435ux441ux43aux43eux439-ux440ux435ux430ux43bux44cux43dux43eux441ux442ux44cux44e}

По итогам анализа можно сказать, что мы получили ровно то, что ожидали
от метода Catch-MSY --- ориентировочные оценки с изрядной долей
неопределенности. \emph{MSY} в районе 13 тысяч тонн выглядит разумной
оценкой, учитывая историю промысла.

Текущий уровень вылова в 12 тысяч тонн --- это тот самый момент, когда
управляющим органам пора начинать тревожно задумываться. Мы еще не
превысили лимит, но мы уже близки к той грани, где любое дополнительное
давление может привести к перелову.

Рекомендация установить OДУ в диапазоне 10.5-13 тысяч тонн --- это не
проявление излишней осторожности, а единственно разумный подход в
условиях высокой неопределенности.

Метод Catch-MSY в очередной раз доказал свою полезность как инструмент
``первого приближения''. Он не дает нам точных оценок, но предоставляет
достаточно информации для принятия взвешенных управленческих решений.

Ирония ситуации заключается в том, что чем меньше у нас данных, тем
более сложные методы мы вынуждены использовать для их интерпретации. Но
как показывает практика, иногда простые решения, основанные на
ограниченной информации, оказываются более эффективными, чем сложные
модели, требующие тонны данных.

В конечном счете, управление рыболовством --- это не о поиске идеальных
решений, а о выборе наименее плохих из доступных вариантов. И в этом
смысле Catch-MSY предоставляет нам именно то, что нужно --- достаточно
информации, чтобы сделать этот выбор осознанно.

\bookmarksetup{startatroot}

\chapter{GMM кластеризация: Определение широкопалости самцов
краба-стригуна}\label{gmm-ux43aux43bux430ux441ux442ux435ux440ux438ux437ux430ux446ux438ux44f-ux43eux43fux440ux435ux434ux435ux43bux435ux43dux438ux435-ux448ux438ux440ux43eux43aux43eux43fux430ux43bux43eux441ux442ux438-ux441ux430ux43cux446ux43eux432-ux43aux440ux430ux431ux430-ux441ux442ux440ux438ux433ux443ux43dux430}

\section{Введение}\label{ux432ux432ux435ux434ux435ux43dux438ux435-12}

Определение функциональной зрелости у самцов крабов-стригунов --- одна
из тех задач, где биология встречается с статистикой лицом к лицу, и где
простое наблюдение часто уступает место изощренному анализу. У самцов
краба-стригуна переход к половозрелости маркируется драматическим
изменением морфометрии: узкие, почти изящные клешни неполовозрелого
«ювениала» (фенотип ``узкопалый'') сменяются массивным, брутальным
инструментом половозрелого самца (фенотип ``широкопалый''). Этот
аллометрический скачок --- не просто украшение, а ключевой адаптивный
признак, определяющий успех в боях за территорию и самок, а значит, и
репродуктивный вклад особи. Проблема в том, что в природе редко
встречаются учебные примеры; вместо этого она подсовывает нам сплошной
континуум, где самый тощий жених запросто может оказаться рядом с самым
упитанным холостяком, а шум измерений и индивидуальная изменчивость
довершают картину хаоса. Задача исследователя --- найти объективную
границу в этом континууме, разделив популяцию на две функциональные
группы, даже если сами крабы об этой границе не подозревают.

\begin{figure}[H]

{\centering \includegraphics[width=0.5\linewidth,height=\textheight,keepaspectratio]{images/SNOW1.jpg}

}

\caption{Рис. 1.: Краб-стригун опилио узкопалый (вверху) и широкопалый
(внизу)}

\end{figure}%

Метод «классификация без учителя» --- это по сути попытка найти скрытую
структуру в данных, когда у нас есть измерения, но нет заранее известных
меток. Мы как бы спрашиваем данные: ``На какие осмысленные группы вы
сами хотите распасться?''. Базовые методы вроде k-means пытаются сделать
это жестко, проставляя границы по принципу ближайшего центра, что часто
приводит к ошибкам на перекрывающихся хвостах распределений.
Иерархическая кластеризация видит структуру вложенно, но чувствительна к
шуму и опять-таки требует субъективного выбора уровня отсечения. Модели
гауссовских смесей (Gaussian mixture models; GMM) подходят к задаче
иначе --- они не просто группируют точки, а предполагают, что каждая
точка порождена одной из нескольких вероятностных моделей (компонент
смеси), каждая из которых является многомерным нормальным распределением
со своими параметрами --- вектором средних и ковариационной матрицей.
Это мощное допущение: оно позволяет каждой группе иметь свою собственную
форму, размер и ориентацию в пространстве признаков, что биологически
осмысленно --- ведь мы и ожидаем, что ``широкопалые'' особи будут не
только в среднем крупнее, но и иметь иную форму соотношения ширины
карапакса и клешни по сравнению с ``узкопалыми''. Преимущество GMM в
том, что это мягкая, вероятностная кластеризация; вместо того чтобы
насильно приписывать наблюдение к кластеру, модель оценивает вероятность
принадлежности, что особенно ценно для пограничных случаев, которые в
биологии встречаются сплошь и рядом. Однако и у классического GMM есть
ахиллесова пята --- он, как и большинство методов, основанных на
нормальном распределении, чувствителен к выбросам. Стоит появиться
нескольким аномально крупным или, наоборот, мелким особям, и оценки
параметров могут сместиться, а границы кластеров --- исказиться. Именно
здесь на сцену выходит робастная (robust) GMM, основанная на
t-распределении Стьюдента. t-распределение имеет более тяжелые хвосты,
чем гауссово, что позволяет модели терпимее относиться к выбросам ---
они перестают быть ``проблемными точками, которые нужно во что бы то ни
стало объяснить'', а становятся просто маловероятными, но допустимыми
событиями. Робастная GMM не пытается подогнать компоненты смеси под все
точки сразу, а более устойчиво оценивает параметры основных, центральных
групп, игнорируя экстремальные отклонения. Т.е. подобен человеку,
который на шумной вечеринке слушает не всех сразу, а только тех, кто
говорит внятно и по делу, мудро игнорируя пьяные дискуссии о работе. В
контексте нашей задачи это важно: полевые данные по крабам почти всегда
зашумлены --- возможны ошибки промеров, наличие больных или
травмированных особей, наконец, просто природные аномалии, которые не
отменяют общего правила. Использование робастной GMM позволяет нам
сосредоточиться на основной биологической сигнатуре --- различии между
двумя фенотипами, --- не позволяя шуму заглушить этот сигнал. Таким
образом, применение такой разновидности GMM это не просто технический
выбор, а методологическая необходимость, позволяющая извлечь из данных
истинную биологическую структуру, скрытую за завесой естественной
изменчивости и погрешности измерения.

\section{Данные и
скрипты}\label{ux434ux430ux43dux43dux44bux435-ux438-ux441ux43aux440ux438ux43fux442ux44b-1}

Исходные данные по ширине карапакса и высоте клешни находятся
\href{https://mombus.github.io/cRab/data/SNOW.xlsx}{здесь}. Скрипты
можно скачать целиком:
\href{https://mombus.github.io/cRab/data/SNOW_3_popular.R}{Первый} - три
разновидности метода классификации, включая обычную GMM.
\href{https://mombus.github.io/cRab/data/SNOW_GMM_robust.R}{Второй} --
робастная GMM.
\href{https://mombus.github.io/cRab/data/SNOW_GMM_robust_predict.R}{Третий}
-- прогноз с использованием робастной GMM.

\begin{figure}[H]

{\centering \includegraphics[width=0.6\linewidth,height=\textheight,keepaspectratio]{images/SNOW1.PNG}

}

\caption{Рис. 2.: Визуализация кластерного анализа половозрелости
краба-стригуна опилио тремя методами классификации без учителя}

\end{figure}%

\begin{figure}[H]

{\centering \includegraphics[width=0.8\linewidth,height=\textheight,keepaspectratio]{images/SNOW2.PNG}

}

\caption{Рис. 3.: Результаты использования робастной модели гауссовских
смесей (GMM)}

\end{figure}%

\bookmarksetup{startatroot}

\chapter{Модели истощения Лесли и
Делури}\label{ux43cux43eux434ux435ux43bux438-ux438ux441ux442ux43eux449ux435ux43dux438ux44f-ux43bux435ux441ux43bux438-ux438-ux434ux435ux43bux443ux440ux438}

\section{Введение}\label{ux432ux432ux435ux434ux435ux43dux438ux435-13}

Если вы когда-нибудь задумывались, как оценить, сколько всего крабов
было в море до того, как начался промысел, то сегодняшнее занятие ---
для вас. Мы погрузимся в причудливый и лапидарный мир моделей истощения
--- пожалуй, одних из самых интуитивно понятных и в то же время самых
капризных инструментов в арсенале рыбохозяйственной науки. Модели Лесли
и ДеЛури, рожденные в середине прошлого века, предлагают элегантное и на
первый взгляд простое решение: если мы видим, что с каждой проведенной
на промысле минутой улов на усилие (пресловутый CPUE или
производительность промысла) неумолимо падает, то, значит, мы
последовательно вылавливаем некую исходную популяцию. Проведя линию
через эти точки снижения, мы можем экстраполировать её назад, к моменту
начала лова, и получить оценку начальной численности. В теории всё
звучит прекрасно и логично. Лесли предлагает смотреть на зависимость
CPUE от кумулятивного (суммарного) усилия, а Делури --- логарифма CPUE
от кумулятивного улова; разница в подходе, но философия одна:
зафиксировать сам факт истощения и дать ему количественную оценку.

Проблема, однако, в том, что эти модели требуют идеальных, почти
лабораторных условий, которые в реальном мире встречаются реже, чем
искренний ответ пахаря голубой нивы на вопрос «Много ли поймал?». Нужно,
чтобы один и тот же флот, с одной и той же эффективностью, одним и тем
же орудием лова работал на ограниченной акватории, изымая запас без
всякой притравки извне. Нужно, чтобы за это время была устойчивая
погода, не изменилась температура воды, не подошла новая агрегация краба
с соседнего участка. Нужно, чтобы крабы вели себя как хорошие
статистические единицы --- предсказумо и пассивно. В общем, нужно чудо.

И такое чудо иногда случается. Данные, с которыми мы будем работать
сегодня --- тот самый редкий и почти уникальный случай. С 2007 по 2018
(рисунок немного устарел) год один промысловый флот вёл лов камчатского
краба на относительно локальном участке Баренцева моря по одной и той же
схеме. И в каждый промысловый сезон, длиной в два-три месяца, кривая
CPUE послушно и плавно ползла вниз, как будто краб специально подгадал и
выстроился в очередь по приказу статистика. Это был идеальный, просто
учебный сюжет, подарок для любого аналитика. Увы, всему хорошему
приходит конец. С приходом новых игроков на промысел идиллическая
картина разрушилась. Пространственная структура запаса перемешалась,
усилие стало прикладываться довольно неравномерно, и та самая чистая
сигнатура истощения, которую нам так нравилась, пропала, растворившись в
шуме конкуренции и возросшего пресса. Теперь эти данные --- скорее
исключение, музейный экспонат, наглядно демонстрирующий, как должны
выглядеть модели истощения в идеале, и почему мы не можем использовать
их так часто, как хотелось бы. Давайте же воспользуемся этой уникальной
возможностью и разберёмся, как работают эти капризные, но прекрасные в
своей простоте методы.

\begin{center}
\includegraphics[width=0.8\linewidth,height=\textheight,keepaspectratio]{images/KINGCRABdepletion.png}
\end{center}

\textbf{Для работы скрипта:}

\begin{enumerate}
\def\labelenumi{\arabic{enumi}.}
\item
  Скачайте файлы данных
  \href{https://mombus.github.io/cRab/data/DATAdep.csv}{DATAdep.csv}.
\item
  Установите рабочую директорию в setwd().
\item
  Установите необходимые пакеты :
  \textbf{\texttt{install.packages(c("FSA",\ "ggplot2",\ "dplyr","broom","tidyverse"\ ))}}.
\end{enumerate}

Файл скрипта находиться
\href{https://mombus.github.io/cRab/data/DEPLETION.R}{здесь}.

\begin{Shaded}
\begin{Highlighting}[]
\CommentTok{\# ========================================================================================================================}
\CommentTok{\# ПРАКТИЧЕСКОЕ ЗАНЯТИЕ: МОДЕЛИ ИСТОЩЕНИЯ ЛЕСЛИ И ДЕЛУРИ}
\CommentTok{\# Курс: "Оценка водных биоресурсов в среде R (для начинающих)"}
\CommentTok{\# Автор: Баканев С. В. Дата: 03.09.2025}
\CommentTok{\# ========================================================================================================================}

\CommentTok{\# 1. ПОДГОТОВКА СРЕДЫ ====================================================================================================}

\CommentTok{\# Установка и подключение необходимых пакетов}
\CommentTok{\# FSA {-} Fisheries Stock Analysis для моделей истощения}
\CommentTok{\# ggplot2 {-} для продвинутой визуализации}
\CommentTok{\# dplyr {-} для манипуляций с данными}
\CommentTok{\# tidyverse {-} современный подход к обработке данных}
\ControlFlowTok{if}\NormalTok{ (}\SpecialCharTok{!}\FunctionTok{require}\NormalTok{(}\StringTok{"FSA"}\NormalTok{)) }\FunctionTok{install.packages}\NormalTok{(}\StringTok{"FSA"}\NormalTok{)}
\end{Highlighting}
\end{Shaded}

\begin{verbatim}
Загрузка требуемого пакета: FSA
\end{verbatim}

\begin{verbatim}
## FSA v0.10.0. See citation('FSA') if used in publication.
## Run fishR() for related website and fishR('IFAR') for related book.
\end{verbatim}

\begin{Shaded}
\begin{Highlighting}[]
\ControlFlowTok{if}\NormalTok{ (}\SpecialCharTok{!}\FunctionTok{require}\NormalTok{(}\StringTok{"ggplot2"}\NormalTok{)) }\FunctionTok{install.packages}\NormalTok{(}\StringTok{"ggplot2"}\NormalTok{)}
\end{Highlighting}
\end{Shaded}

\begin{verbatim}
Загрузка требуемого пакета: ggplot2
\end{verbatim}

\begin{Shaded}
\begin{Highlighting}[]
\ControlFlowTok{if}\NormalTok{ (}\SpecialCharTok{!}\FunctionTok{require}\NormalTok{(}\StringTok{"dplyr"}\NormalTok{)) }\FunctionTok{install.packages}\NormalTok{(}\StringTok{"dplyr"}\NormalTok{)}
\end{Highlighting}
\end{Shaded}

\begin{verbatim}
Загрузка требуемого пакета: dplyr
\end{verbatim}

\begin{verbatim}

Присоединяю пакет: 'dplyr'
\end{verbatim}

\begin{verbatim}
Следующие объекты скрыты от 'package:stats':

    filter, lag
\end{verbatim}

\begin{verbatim}
Следующие объекты скрыты от 'package:base':

    intersect, setdiff, setequal, union
\end{verbatim}

\begin{Shaded}
\begin{Highlighting}[]
\ControlFlowTok{if}\NormalTok{ (}\SpecialCharTok{!}\FunctionTok{require}\NormalTok{(}\StringTok{"broom"}\NormalTok{)) }\FunctionTok{install.packages}\NormalTok{(}\StringTok{"broom"}\NormalTok{)}
\end{Highlighting}
\end{Shaded}

\begin{verbatim}
Загрузка требуемого пакета: broom
\end{verbatim}

\begin{Shaded}
\begin{Highlighting}[]
\ControlFlowTok{if}\NormalTok{ (}\SpecialCharTok{!}\FunctionTok{require}\NormalTok{(}\StringTok{"tidyverse"}\NormalTok{)) }\FunctionTok{install.packages}\NormalTok{(}\StringTok{"tidyverse"}\NormalTok{)}
\end{Highlighting}
\end{Shaded}

\begin{verbatim}
Загрузка требуемого пакета: tidyverse
\end{verbatim}

\begin{verbatim}
-- Attaching core tidyverse packages ------------------------ tidyverse 2.0.0 --
v forcats   1.0.0     v stringr   1.5.2
v lubridate 1.9.4     v tibble    3.2.1
v purrr     1.0.4     v tidyr     1.3.1
v readr     2.1.5     
-- Conflicts ------------------------------------------ tidyverse_conflicts() --
x dplyr::filter() masks stats::filter()
x dplyr::lag()    masks stats::lag()
i Use the conflicted package (<http://conflicted.r-lib.org/>) to force all conflicts to become errors
\end{verbatim}

\begin{Shaded}
\begin{Highlighting}[]
\FunctionTok{library}\NormalTok{(FSA)}
\FunctionTok{library}\NormalTok{(ggplot2)}
\FunctionTok{library}\NormalTok{(dplyr)}
\FunctionTok{library}\NormalTok{(broom)}
\FunctionTok{library}\NormalTok{(tidyverse)}

\CommentTok{\# Установка рабочей директории {-} укажите путь к папке с данными}
\FunctionTok{setwd}\NormalTok{(}\StringTok{"C:/DEPLETION/"}\NormalTok{)}

\CommentTok{\# 2. ЗАГРУЗКА И ПЕРВИЧНЫЙ АНАЛИЗ ДАННЫХ ==================================================================================}

\CommentTok{\# Чтение данных из CSV{-}файла}
\CommentTok{\# header = TRUE {-} первая строка содержит названия колонок}
\CommentTok{\# sep = ";" {-} разделитель точка с запятой (common для European CSV)}
\NormalTok{LESLIDATA }\OtherTok{\textless{}{-}} \FunctionTok{read.csv}\NormalTok{(}\StringTok{"DATAdep.csv"}\NormalTok{, }\AttributeTok{header =} \ConstantTok{TRUE}\NormalTok{, }\AttributeTok{sep =} \StringTok{";"}\NormalTok{)}

\CommentTok{\# Проверка структуры данных}
\CommentTok{\# Функция str() показывает:}
\CommentTok{\# {-} тип объекта (data.frame)}
\CommentTok{\# {-} количество наблюдений и переменных}
\CommentTok{\# {-} тип каждой переменной}
\FunctionTok{str}\NormalTok{(LESLIDATA)}
\end{Highlighting}
\end{Shaded}

\begin{verbatim}
'data.frame':   131 obs. of  5 variables:
 $ CPUE  : num  51 90 68 64 60 72 75 75 65 65 ...
 $ CATCH : num  262 826 596 545 564 ...
 $ YEAR  : int  2007 2007 2007 2007 2007 2007 2007 2007 2007 2007 ...
 $ WEEK  : int  37 38 39 40 41 42 43 44 45 46 ...
 $ EFFORT: num  5.14 9.18 8.77 8.52 9.39 ...
\end{verbatim}

\begin{Shaded}
\begin{Highlighting}[]
\CommentTok{\# 3. БАЗОВЫЙ АНАЛИЗ МОДЕЛИ ЛЕСЛИ ДЛЯ ОДНОГО ПЕРИОДА ======================================================================}

\CommentTok{\# Фильтрация данных для 2008 года (2007 \textless{} YEAR \textless{} 2009)}
\NormalTok{DATA }\OtherTok{\textless{}{-}}\NormalTok{ LESLIDATA[LESLIDATA}\SpecialCharTok{$}\NormalTok{YEAR }\SpecialCharTok{\textgreater{}} \DecValTok{2007} \SpecialCharTok{\&}\NormalTok{ LESLIDATA}\SpecialCharTok{$}\NormalTok{YEAR }\SpecialCharTok{\textless{}} \DecValTok{2009}\NormalTok{, ]}

\CommentTok{\# Построение модели истощения по методу Лесли}
\CommentTok{\# Модель Лесли: CPUE \textasciitilde{} кумулятивное усилие}
\CommentTok{\# Использует линейную регрессию для оценки начальной численности}
\NormalTok{lesli }\OtherTok{\textless{}{-}} \FunctionTok{depletion}\NormalTok{(DATA}\SpecialCharTok{$}\NormalTok{CATCH, DATA}\SpecialCharTok{$}\NormalTok{EFFORT, }\AttributeTok{method =} \StringTok{"Leslie"}\NormalTok{)}

\CommentTok{\# Визуализация модели}
\CommentTok{\# График показывает зависимость CPUE от кумулятивного усилия}
\FunctionTok{plot}\NormalTok{(lesli)}
\end{Highlighting}
\end{Shaded}

\pandocbounded{\includegraphics[keepaspectratio]{chapter18_files/figure-pdf/unnamed-chunk-1-1.pdf}}

\begin{Shaded}
\begin{Highlighting}[]
\CommentTok{\# Доверительные интервалы для параметров модели}
\FunctionTok{confint}\NormalTok{(lesli)}
\end{Highlighting}
\end{Shaded}

\begin{verbatim}
        95% LCI      95% UCI
No 1.041229e+04 1.836816e+04
q  3.005055e-03 6.548509e-03
\end{verbatim}

\begin{Shaded}
\begin{Highlighting}[]
\CommentTok{\# Сводная информация по модели}
\CommentTok{\# Включает оценки параметров и статистику качества}
\FunctionTok{summary}\NormalTok{(lesli)}
\end{Highlighting}
\end{Shaded}

\begin{verbatim}
       Estimate    Std. Err.
No 1.439023e+04 1.854700e+03
q  4.776782e-03 8.260623e-04
\end{verbatim}

\begin{Shaded}
\begin{Highlighting}[]
\CommentTok{\# 4. РАСШИРЕННЫЙ АНАЛИЗ ПО ГОДАМ (2007{-}2018) =============================================================================}

\CommentTok{\# Создаем функцию для расчета кумулятивных показателей}
\CommentTok{\# Кумулятивные показатели необходимы для построения моделей истощения}
\NormalTok{calculate\_cumulative }\OtherTok{\textless{}{-}} \ControlFlowTok{function}\NormalTok{(data) \{}
\NormalTok{  data }\SpecialCharTok{\%\textgreater{}\%}
    \FunctionTok{arrange}\NormalTok{(WEEK) }\SpecialCharTok{\%\textgreater{}\%}  \CommentTok{\# Сортировка по неделям}
    \FunctionTok{mutate}\NormalTok{(}
      \AttributeTok{cumulative\_effort =} \FunctionTok{cumsum}\NormalTok{(EFFORT),  }\CommentTok{\# Накопленное усилие}
      \AttributeTok{cumulative\_catch =} \FunctionTok{cumsum}\NormalTok{(CATCH)     }\CommentTok{\# Накопленный улов}
\NormalTok{    )}
\NormalTok{\}}

\CommentTok{\# Инициализация списка для хранения результатов}
\NormalTok{leslie\_models\_list }\OtherTok{\textless{}{-}} \FunctionTok{list}\NormalTok{()}

\CommentTok{\# Анализ для каждого года в диапазоне 2007{-}2018}
\ControlFlowTok{for}\NormalTok{ (year }\ControlFlowTok{in} \DecValTok{2007}\SpecialCharTok{:}\DecValTok{2018}\NormalTok{) \{}
  \CommentTok{\# Фильтрация данных по году}
\NormalTok{  year\_data }\OtherTok{\textless{}{-}}\NormalTok{ LESLIDATA }\SpecialCharTok{\%\textgreater{}\%} 
    \FunctionTok{filter}\NormalTok{(YEAR }\SpecialCharTok{==}\NormalTok{ year) }\SpecialCharTok{\%\textgreater{}\%}
    \FunctionTok{na.omit}\NormalTok{()  }\CommentTok{\# Удаление пропущенных значений}
  
  \CommentTok{\# Проверка достаточности данных (минимум 3 наблюдения)}
  \ControlFlowTok{if}\NormalTok{ (}\FunctionTok{nrow}\NormalTok{(year\_data) }\SpecialCharTok{\textless{}} \DecValTok{3}\NormalTok{) \{}
    \FunctionTok{message}\NormalTok{(}\FunctionTok{paste}\NormalTok{(}\StringTok{"Недостаточно данных для анализа в"}\NormalTok{, year))}
    \ControlFlowTok{next}  \CommentTok{\# Переход к следующему году}
\NormalTok{  \}}
  
  \CommentTok{\# Расчет кумулятивных показателей}
\NormalTok{  year\_data }\OtherTok{\textless{}{-}} \FunctionTok{calculate\_cumulative}\NormalTok{(year\_data)}
  
  \CommentTok{\# Добавление CPUE (улов на единицу усилия)}
\NormalTok{  year\_data}\SpecialCharTok{$}\NormalTok{CPUE }\OtherTok{\textless{}{-}}\NormalTok{ year\_data}\SpecialCharTok{$}\NormalTok{CATCH }\SpecialCharTok{/}\NormalTok{ year\_data}\SpecialCharTok{$}\NormalTok{EFFORT}
  
  \CommentTok{\# Построение модели Лесли через линейную регрессию}
\NormalTok{  leslie\_model }\OtherTok{\textless{}{-}} \FunctionTok{try}\NormalTok{(}\FunctionTok{lm}\NormalTok{(CPUE }\SpecialCharTok{\textasciitilde{}}\NormalTok{ cumulative\_effort, }\AttributeTok{data =}\NormalTok{ year\_data), }\AttributeTok{silent =} \ConstantTok{TRUE}\NormalTok{)}
  
  \ControlFlowTok{if}\NormalTok{ (}\FunctionTok{inherits}\NormalTok{(leslie\_model, }\StringTok{"try{-}error"}\NormalTok{)) \{}
    \CommentTok{\# Обработка ошибок моделирования}
\NormalTok{    year\_data}\SpecialCharTok{$}\NormalTok{leslie\_predicted }\OtherTok{\textless{}{-}} \ConstantTok{NA}
\NormalTok{    year\_data}\SpecialCharTok{$}\NormalTok{leslie\_lwr }\OtherTok{\textless{}{-}} \ConstantTok{NA}
\NormalTok{    year\_data}\SpecialCharTok{$}\NormalTok{leslie\_upr }\OtherTok{\textless{}{-}} \ConstantTok{NA}
    \FunctionTok{message}\NormalTok{(}\FunctionTok{paste}\NormalTok{(}\StringTok{"Ошибка в модели Лесли для"}\NormalTok{, year))}
\NormalTok{  \} }\ControlFlowTok{else}\NormalTok{ \{}
    \CommentTok{\# Получение предсказаний с доверительными интервалами}
\NormalTok{    predictions }\OtherTok{\textless{}{-}} \FunctionTok{predict}\NormalTok{(leslie\_model, }\AttributeTok{interval =} \StringTok{"confidence"}\NormalTok{, }\AttributeTok{level =} \FloatTok{0.95}\NormalTok{)}
\NormalTok{    year\_data}\SpecialCharTok{$}\NormalTok{leslie\_predicted }\OtherTok{\textless{}{-}}\NormalTok{ predictions[, }\StringTok{"fit"}\NormalTok{]}
\NormalTok{    year\_data}\SpecialCharTok{$}\NormalTok{leslie\_lwr }\OtherTok{\textless{}{-}}\NormalTok{ predictions[, }\StringTok{"lwr"}\NormalTok{]}
\NormalTok{    year\_data}\SpecialCharTok{$}\NormalTok{leslie\_upr }\OtherTok{\textless{}{-}}\NormalTok{ predictions[, }\StringTok{"upr"}\NormalTok{]}
    
    \CommentTok{\# Расчет начальной биомассы (No)}
    \CommentTok{\# No = {-}a/b, где a {-} интерсепт, b {-} коэффициент кумулятивного усилия}
\NormalTok{    a }\OtherTok{\textless{}{-}} \FunctionTok{coef}\NormalTok{(leslie\_model)[}\DecValTok{1}\NormalTok{]}
\NormalTok{    b }\OtherTok{\textless{}{-}} \FunctionTok{coef}\NormalTok{(leslie\_model)[}\DecValTok{2}\NormalTok{]}
\NormalTok{    No }\OtherTok{\textless{}{-}} \SpecialCharTok{{-}}\NormalTok{a }\SpecialCharTok{/}\NormalTok{ b}
\NormalTok{    year\_data}\SpecialCharTok{$}\NormalTok{No }\OtherTok{\textless{}{-}}\NormalTok{ No}
\NormalTok{  \}}
  
  \CommentTok{\# Сохранение результатов для года}
\NormalTok{  leslie\_models\_list[[}\FunctionTok{as.character}\NormalTok{(year)]] }\OtherTok{\textless{}{-}}\NormalTok{ year\_data}
\NormalTok{\}}

\CommentTok{\# Объединение данных всех лет}
\NormalTok{all\_years\_leslie }\OtherTok{\textless{}{-}} \FunctionTok{bind\_rows}\NormalTok{(leslie\_models\_list, }\AttributeTok{.id =} \StringTok{"Year"}\NormalTok{)}

\CommentTok{\# Преобразование Year в фактор с сохранением порядка}
\NormalTok{all\_years\_leslie}\SpecialCharTok{$}\NormalTok{Year }\OtherTok{\textless{}{-}} \FunctionTok{factor}\NormalTok{(all\_years\_leslie}\SpecialCharTok{$}\NormalTok{Year, }\AttributeTok{levels =} \FunctionTok{as.character}\NormalTok{(}\DecValTok{2007}\SpecialCharTok{:}\DecValTok{2018}\NormalTok{))}

\CommentTok{\# 5. ВИЗУАЛИЗАЦИЯ РЕЗУЛЬТАТОВ МОДЕЛИ ЛЕСЛИ ==============================================================================}

\CommentTok{\# Построение фасетного графика для всех лет}
\NormalTok{leslie\_facet\_plot }\OtherTok{\textless{}{-}} \FunctionTok{ggplot}\NormalTok{(all\_years\_leslie, }\FunctionTok{aes}\NormalTok{(}\AttributeTok{x =}\NormalTok{ cumulative\_effort)) }\SpecialCharTok{+}
  \FunctionTok{geom\_point}\NormalTok{(}\FunctionTok{aes}\NormalTok{(}\AttributeTok{y =}\NormalTok{ CPUE), }\AttributeTok{size =} \DecValTok{2}\NormalTok{, }\AttributeTok{color =} \StringTok{"darkblue"}\NormalTok{, }\AttributeTok{alpha =} \FloatTok{0.7}\NormalTok{) }\SpecialCharTok{+}
  \FunctionTok{geom\_ribbon}\NormalTok{(}\FunctionTok{aes}\NormalTok{(}\AttributeTok{ymin =}\NormalTok{ leslie\_lwr, }\AttributeTok{ymax =}\NormalTok{ leslie\_upr), }
              \AttributeTok{fill =} \StringTok{"red"}\NormalTok{, }\AttributeTok{alpha =} \FloatTok{0.2}\NormalTok{) }\SpecialCharTok{+}  \CommentTok{\# Доверительный интервал}
  \FunctionTok{geom\_line}\NormalTok{(}\FunctionTok{aes}\NormalTok{(}\AttributeTok{y =}\NormalTok{ leslie\_predicted), }\AttributeTok{color =} \StringTok{"red"}\NormalTok{, }\AttributeTok{linewidth =} \DecValTok{1}\NormalTok{) }\SpecialCharTok{+}
  \FunctionTok{facet\_wrap}\NormalTok{(}\SpecialCharTok{\textasciitilde{}}\NormalTok{ Year, }\AttributeTok{scales =} \StringTok{"free"}\NormalTok{, }\AttributeTok{ncol =} \DecValTok{3}\NormalTok{) }\SpecialCharTok{+}  \CommentTok{\# Свободные масштабы для каждого года}
  \FunctionTok{labs}\NormalTok{(}
    \AttributeTok{title =} \StringTok{"Модель Лесли: зависимость CPUE от кумулятивного усилия"}\NormalTok{,}
    \AttributeTok{subtitle =} \StringTok{"С доверительными интервалами (95\%)"}\NormalTok{,}
    \AttributeTok{x =} \StringTok{"Кумулятивное усилие лова"}\NormalTok{,}
    \AttributeTok{y =} \StringTok{"CPUE (улов на единицу усилия)"}
\NormalTok{  ) }\SpecialCharTok{+}
  \FunctionTok{theme\_minimal}\NormalTok{() }\SpecialCharTok{+}
  \FunctionTok{theme}\NormalTok{(}
    \AttributeTok{plot.title =} \FunctionTok{element\_text}\NormalTok{(}\AttributeTok{hjust =} \FloatTok{0.5}\NormalTok{, }\AttributeTok{face =} \StringTok{"bold"}\NormalTok{, }\AttributeTok{size =} \DecValTok{16}\NormalTok{),}
    \AttributeTok{plot.subtitle =} \FunctionTok{element\_text}\NormalTok{(}\AttributeTok{hjust =} \FloatTok{0.5}\NormalTok{, }\AttributeTok{size =} \DecValTok{12}\NormalTok{),}
    \AttributeTok{strip.text =} \FunctionTok{element\_text}\NormalTok{(}\AttributeTok{face =} \StringTok{"bold"}\NormalTok{, }\AttributeTok{size =} \DecValTok{10}\NormalTok{)}
\NormalTok{  )}

\CommentTok{\# Вывод графика}
\FunctionTok{print}\NormalTok{(leslie\_facet\_plot)}
\end{Highlighting}
\end{Shaded}

\begin{verbatim}
Warning in grid.Call(C_textBounds, as.graphicsAnnot(x$label), x$x, x$y, :
неизвестна ширина символа 0xf3 в кодировке CP1251
\end{verbatim}

\begin{verbatim}
Warning in grid.Call(C_textBounds, as.graphicsAnnot(x$label), x$x, x$y, :
неизвестна ширина символа 0xeb в кодировке CP1251
\end{verbatim}

\begin{verbatim}
Warning in grid.Call(C_textBounds, as.graphicsAnnot(x$label), x$x, x$y, :
неизвестна ширина символа 0xee в кодировке CP1251
\end{verbatim}

\begin{verbatim}
Warning in grid.Call(C_textBounds, as.graphicsAnnot(x$label), x$x, x$y, :
неизвестна ширина символа 0xe2 в кодировке CP1251
\end{verbatim}

\begin{verbatim}
Warning in grid.Call(C_textBounds, as.graphicsAnnot(x$label), x$x, x$y, :
неизвестна ширина символа 0xed в кодировке CP1251
\end{verbatim}

\begin{verbatim}
Warning in grid.Call(C_textBounds, as.graphicsAnnot(x$label), x$x, x$y, :
неизвестна ширина символа 0xe0 в кодировке CP1251
\end{verbatim}

\begin{verbatim}
Warning in grid.Call(C_textBounds, as.graphicsAnnot(x$label), x$x, x$y, :
неизвестна ширина символа 0xe5 в кодировке CP1251
\end{verbatim}

\begin{verbatim}
Warning in grid.Call(C_textBounds, as.graphicsAnnot(x$label), x$x, x$y, :
неизвестна ширина символа 0xe4 в кодировке CP1251
\end{verbatim}

\begin{verbatim}
Warning in grid.Call(C_textBounds, as.graphicsAnnot(x$label), x$x, x$y, :
неизвестна ширина символа 0xe8 в кодировке CP1251
\end{verbatim}

\begin{verbatim}
Warning in grid.Call(C_textBounds, as.graphicsAnnot(x$label), x$x, x$y, :
неизвестна ширина символа 0xed в кодировке CP1251
\end{verbatim}

\begin{verbatim}
Warning in grid.Call(C_textBounds, as.graphicsAnnot(x$label), x$x, x$y, :
неизвестна ширина символа 0xe8 в кодировке CP1251
\end{verbatim}

\begin{verbatim}
Warning in grid.Call(C_textBounds, as.graphicsAnnot(x$label), x$x, x$y, :
неизвестна ширина символа 0xf6 в кодировке CP1251
\end{verbatim}

\begin{verbatim}
Warning in grid.Call(C_textBounds, as.graphicsAnnot(x$label), x$x, x$y, :
неизвестна ширина символа 0xf3 в кодировке CP1251
Warning in grid.Call(C_textBounds, as.graphicsAnnot(x$label), x$x, x$y, :
неизвестна ширина символа 0xf3 в кодировке CP1251
\end{verbatim}

\begin{verbatim}
Warning in grid.Call(C_textBounds, as.graphicsAnnot(x$label), x$x, x$y, :
неизвестна ширина символа 0xf1 в кодировке CP1251
\end{verbatim}

\begin{verbatim}
Warning in grid.Call(C_textBounds, as.graphicsAnnot(x$label), x$x, x$y, :
неизвестна ширина символа 0xe8 в кодировке CP1251
\end{verbatim}

\begin{verbatim}
Warning in grid.Call(C_textBounds, as.graphicsAnnot(x$label), x$x, x$y, :
неизвестна ширина символа 0xeb в кодировке CP1251
\end{verbatim}

\begin{verbatim}
Warning in grid.Call(C_textBounds, as.graphicsAnnot(x$label), x$x, x$y, :
неизвестна ширина символа 0xe8 в кодировке CP1251
\end{verbatim}

\begin{verbatim}
Warning in grid.Call(C_textBounds, as.graphicsAnnot(x$label), x$x, x$y, :
неизвестна ширина символа 0xff в кодировке CP1251
\end{verbatim}

\begin{verbatim}
Warning in grid.Call(C_textBounds, as.graphicsAnnot(x$label), x$x, x$y, :
неизвестна ширина символа 0xcc в кодировке CP1251
\end{verbatim}

\begin{verbatim}
Warning in grid.Call(C_textBounds, as.graphicsAnnot(x$label), x$x, x$y, :
неизвестна ширина символа 0xee в кодировке CP1251
\end{verbatim}

\begin{verbatim}
Warning in grid.Call(C_textBounds, as.graphicsAnnot(x$label), x$x, x$y, :
неизвестна ширина символа 0xe4 в кодировке CP1251
\end{verbatim}

\begin{verbatim}
Warning in grid.Call(C_textBounds, as.graphicsAnnot(x$label), x$x, x$y, :
неизвестна ширина символа 0xe5 в кодировке CP1251
\end{verbatim}

\begin{verbatim}
Warning in grid.Call(C_textBounds, as.graphicsAnnot(x$label), x$x, x$y, :
неизвестна ширина символа 0xeb в кодировке CP1251
\end{verbatim}

\begin{verbatim}
Warning in grid.Call(C_textBounds, as.graphicsAnnot(x$label), x$x, x$y, :
неизвестна ширина символа 0xfc в кодировке CP1251
\end{verbatim}

\begin{verbatim}
Warning in grid.Call(C_textBounds, as.graphicsAnnot(x$label), x$x, x$y, :
неизвестна ширина символа 0xcb в кодировке CP1251
\end{verbatim}

\begin{verbatim}
Warning in grid.Call(C_textBounds, as.graphicsAnnot(x$label), x$x, x$y, :
неизвестна ширина символа 0xe5 в кодировке CP1251
\end{verbatim}

\begin{verbatim}
Warning in grid.Call(C_textBounds, as.graphicsAnnot(x$label), x$x, x$y, :
неизвестна ширина символа 0xf1 в кодировке CP1251
\end{verbatim}

\begin{verbatim}
Warning in grid.Call(C_textBounds, as.graphicsAnnot(x$label), x$x, x$y, :
неизвестна ширина символа 0xeb в кодировке CP1251
\end{verbatim}

\begin{verbatim}
Warning in grid.Call(C_textBounds, as.graphicsAnnot(x$label), x$x, x$y, :
неизвестна ширина символа 0xe8 в кодировке CP1251
\end{verbatim}

\begin{verbatim}
Warning in grid.Call(C_textBounds, as.graphicsAnnot(x$label), x$x, x$y, :
неизвестна ширина символа 0xe7 в кодировке CP1251
\end{verbatim}

\begin{verbatim}
Warning in grid.Call(C_textBounds, as.graphicsAnnot(x$label), x$x, x$y, :
неизвестна ширина символа 0xe0 в кодировке CP1251
\end{verbatim}

\begin{verbatim}
Warning in grid.Call(C_textBounds, as.graphicsAnnot(x$label), x$x, x$y, :
неизвестна ширина символа 0xe2 в кодировке CP1251
\end{verbatim}

\begin{verbatim}
Warning in grid.Call(C_textBounds, as.graphicsAnnot(x$label), x$x, x$y, :
неизвестна ширина символа 0xe8 в кодировке CP1251
\end{verbatim}

\begin{verbatim}
Warning in grid.Call(C_textBounds, as.graphicsAnnot(x$label), x$x, x$y, :
неизвестна ширина символа 0xf1 в кодировке CP1251
\end{verbatim}

\begin{verbatim}
Warning in grid.Call(C_textBounds, as.graphicsAnnot(x$label), x$x, x$y, :
неизвестна ширина символа 0xe8 в кодировке CP1251
\end{verbatim}

\begin{verbatim}
Warning in grid.Call(C_textBounds, as.graphicsAnnot(x$label), x$x, x$y, :
неизвестна ширина символа 0xec в кодировке CP1251
\end{verbatim}

\begin{verbatim}
Warning in grid.Call(C_textBounds, as.graphicsAnnot(x$label), x$x, x$y, :
неизвестна ширина символа 0xee в кодировке CP1251
\end{verbatim}

\begin{verbatim}
Warning in grid.Call(C_textBounds, as.graphicsAnnot(x$label), x$x, x$y, :
неизвестна ширина символа 0xf1 в кодировке CP1251
\end{verbatim}

\begin{verbatim}
Warning in grid.Call(C_textBounds, as.graphicsAnnot(x$label), x$x, x$y, :
неизвестна ширина символа 0xf2 в кодировке CP1251
\end{verbatim}

\begin{verbatim}
Warning in grid.Call(C_textBounds, as.graphicsAnnot(x$label), x$x, x$y, :
неизвестна ширина символа 0xfc в кодировке CP1251
\end{verbatim}

\begin{verbatim}
Warning in grid.Call(C_textBounds, as.graphicsAnnot(x$label), x$x, x$y, :
неизвестна ширина символа 0xee в кодировке CP1251
\end{verbatim}

\begin{verbatim}
Warning in grid.Call(C_textBounds, as.graphicsAnnot(x$label), x$x, x$y, :
неизвестна ширина символа 0xf2 в кодировке CP1251
\end{verbatim}

\begin{verbatim}
Warning in grid.Call(C_textBounds, as.graphicsAnnot(x$label), x$x, x$y, :
неизвестна ширина символа 0xea в кодировке CP1251
\end{verbatim}

\begin{verbatim}
Warning in grid.Call(C_textBounds, as.graphicsAnnot(x$label), x$x, x$y, :
неизвестна ширина символа 0xf3 в кодировке CP1251
\end{verbatim}

\begin{verbatim}
Warning in grid.Call(C_textBounds, as.graphicsAnnot(x$label), x$x, x$y, :
неизвестна ширина символа 0xec в кодировке CP1251
\end{verbatim}

\begin{verbatim}
Warning in grid.Call(C_textBounds, as.graphicsAnnot(x$label), x$x, x$y, :
неизвестна ширина символа 0xf3 в кодировке CP1251
\end{verbatim}

\begin{verbatim}
Warning in grid.Call(C_textBounds, as.graphicsAnnot(x$label), x$x, x$y, :
неизвестна ширина символа 0xeb в кодировке CP1251
\end{verbatim}

\begin{verbatim}
Warning in grid.Call(C_textBounds, as.graphicsAnnot(x$label), x$x, x$y, :
неизвестна ширина символа 0xff в кодировке CP1251
\end{verbatim}

\begin{verbatim}
Warning in grid.Call(C_textBounds, as.graphicsAnnot(x$label), x$x, x$y, :
неизвестна ширина символа 0xf2 в кодировке CP1251
\end{verbatim}

\begin{verbatim}
Warning in grid.Call(C_textBounds, as.graphicsAnnot(x$label), x$x, x$y, :
неизвестна ширина символа 0xe8 в кодировке CP1251
\end{verbatim}

\begin{verbatim}
Warning in grid.Call(C_textBounds, as.graphicsAnnot(x$label), x$x, x$y, :
неизвестна ширина символа 0xe2 в кодировке CP1251
\end{verbatim}

\begin{verbatim}
Warning in grid.Call(C_textBounds, as.graphicsAnnot(x$label), x$x, x$y, :
неизвестна ширина символа 0xed в кодировке CP1251
\end{verbatim}

\begin{verbatim}
Warning in grid.Call(C_textBounds, as.graphicsAnnot(x$label), x$x, x$y, :
неизвестна ширина символа 0xee в кодировке CP1251
\end{verbatim}

\begin{verbatim}
Warning in grid.Call(C_textBounds, as.graphicsAnnot(x$label), x$x, x$y, :
неизвестна ширина символа 0xe3 в кодировке CP1251
\end{verbatim}

\begin{verbatim}
Warning in grid.Call(C_textBounds, as.graphicsAnnot(x$label), x$x, x$y, :
неизвестна ширина символа 0xee в кодировке CP1251
\end{verbatim}

\begin{verbatim}
Warning in grid.Call(C_textBounds, as.graphicsAnnot(x$label), x$x, x$y, :
неизвестна ширина символа 0xf3 в кодировке CP1251
\end{verbatim}

\begin{verbatim}
Warning in grid.Call(C_textBounds, as.graphicsAnnot(x$label), x$x, x$y, :
неизвестна ширина символа 0xf1 в кодировке CP1251
\end{verbatim}

\begin{verbatim}
Warning in grid.Call(C_textBounds, as.graphicsAnnot(x$label), x$x, x$y, :
неизвестна ширина символа 0xe8 в кодировке CP1251
\end{verbatim}

\begin{verbatim}
Warning in grid.Call(C_textBounds, as.graphicsAnnot(x$label), x$x, x$y, :
неизвестна ширина символа 0xeb в кодировке CP1251
\end{verbatim}

\begin{verbatim}
Warning in grid.Call(C_textBounds, as.graphicsAnnot(x$label), x$x, x$y, :
неизвестна ширина символа 0xe8 в кодировке CP1251
\end{verbatim}

\begin{verbatim}
Warning in grid.Call(C_textBounds, as.graphicsAnnot(x$label), x$x, x$y, :
неизвестна ширина символа 0xff в кодировке CP1251
\end{verbatim}

\begin{verbatim}
Warning in grid.Call(C_textBounds, as.graphicsAnnot(x$label), x$x, x$y, :
неизвестна ширина символа 0xd1 в кодировке CP1251
\end{verbatim}

\begin{verbatim}
Warning in grid.Call(C_textBounds, as.graphicsAnnot(x$label), x$x, x$y, :
неизвестна ширина символа 0xe4 в кодировке CP1251
\end{verbatim}

\begin{verbatim}
Warning in grid.Call(C_textBounds, as.graphicsAnnot(x$label), x$x, x$y, :
неизвестна ширина символа 0xee в кодировке CP1251
\end{verbatim}

\begin{verbatim}
Warning in grid.Call(C_textBounds, as.graphicsAnnot(x$label), x$x, x$y, :
неизвестна ширина символа 0xe2 в кодировке CP1251
\end{verbatim}

\begin{verbatim}
Warning in grid.Call(C_textBounds, as.graphicsAnnot(x$label), x$x, x$y, :
неизвестна ширина символа 0xe5 в кодировке CP1251
\end{verbatim}

\begin{verbatim}
Warning in grid.Call(C_textBounds, as.graphicsAnnot(x$label), x$x, x$y, :
неизвестна ширина символа 0xf0 в кодировке CP1251
\end{verbatim}

\begin{verbatim}
Warning in grid.Call(C_textBounds, as.graphicsAnnot(x$label), x$x, x$y, :
неизвестна ширина символа 0xe8 в кодировке CP1251
\end{verbatim}

\begin{verbatim}
Warning in grid.Call(C_textBounds, as.graphicsAnnot(x$label), x$x, x$y, :
неизвестна ширина символа 0xf2 в кодировке CP1251
\end{verbatim}

\begin{verbatim}
Warning in grid.Call(C_textBounds, as.graphicsAnnot(x$label), x$x, x$y, :
неизвестна ширина символа 0xe5 в кодировке CP1251
\end{verbatim}

\begin{verbatim}
Warning in grid.Call(C_textBounds, as.graphicsAnnot(x$label), x$x, x$y, :
неизвестна ширина символа 0xeb в кодировке CP1251
\end{verbatim}

\begin{verbatim}
Warning in grid.Call(C_textBounds, as.graphicsAnnot(x$label), x$x, x$y, :
неизвестна ширина символа 0xfc в кодировке CP1251
\end{verbatim}

\begin{verbatim}
Warning in grid.Call(C_textBounds, as.graphicsAnnot(x$label), x$x, x$y, :
неизвестна ширина символа 0xed в кодировке CP1251
\end{verbatim}

\begin{verbatim}
Warning in grid.Call(C_textBounds, as.graphicsAnnot(x$label), x$x, x$y, :
неизвестна ширина символа 0xfb в кодировке CP1251
\end{verbatim}

\begin{verbatim}
Warning in grid.Call(C_textBounds, as.graphicsAnnot(x$label), x$x, x$y, :
неизвестна ширина символа 0xec в кодировке CP1251
\end{verbatim}

\begin{verbatim}
Warning in grid.Call(C_textBounds, as.graphicsAnnot(x$label), x$x, x$y, :
неизвестна ширина символа 0xe8 в кодировке CP1251
Warning in grid.Call(C_textBounds, as.graphicsAnnot(x$label), x$x, x$y, :
неизвестна ширина символа 0xe8 в кодировке CP1251
\end{verbatim}

\begin{verbatim}
Warning in grid.Call(C_textBounds, as.graphicsAnnot(x$label), x$x, x$y, :
неизвестна ширина символа 0xed в кодировке CP1251
\end{verbatim}

\begin{verbatim}
Warning in grid.Call(C_textBounds, as.graphicsAnnot(x$label), x$x, x$y, :
неизвестна ширина символа 0xf2 в кодировке CP1251
\end{verbatim}

\begin{verbatim}
Warning in grid.Call(C_textBounds, as.graphicsAnnot(x$label), x$x, x$y, :
неизвестна ширина символа 0xe5 в кодировке CP1251
\end{verbatim}

\begin{verbatim}
Warning in grid.Call(C_textBounds, as.graphicsAnnot(x$label), x$x, x$y, :
неизвестна ширина символа 0xf0 в кодировке CP1251
\end{verbatim}

\begin{verbatim}
Warning in grid.Call(C_textBounds, as.graphicsAnnot(x$label), x$x, x$y, :
неизвестна ширина символа 0xe2 в кодировке CP1251
\end{verbatim}

\begin{verbatim}
Warning in grid.Call(C_textBounds, as.graphicsAnnot(x$label), x$x, x$y, :
неизвестна ширина символа 0xe0 в кодировке CP1251
\end{verbatim}

\begin{verbatim}
Warning in grid.Call(C_textBounds, as.graphicsAnnot(x$label), x$x, x$y, :
неизвестна ширина символа 0xeb в кодировке CP1251
\end{verbatim}

\begin{verbatim}
Warning in grid.Call(C_textBounds, as.graphicsAnnot(x$label), x$x, x$y, :
неизвестна ширина символа 0xe0 в кодировке CP1251
\end{verbatim}

\begin{verbatim}
Warning in grid.Call(C_textBounds, as.graphicsAnnot(x$label), x$x, x$y, :
неизвестна ширина символа 0xec в кодировке CP1251
\end{verbatim}

\begin{verbatim}
Warning in grid.Call(C_textBounds, as.graphicsAnnot(x$label), x$x, x$y, :
неизвестна ширина символа 0xe8 в кодировке CP1251
\end{verbatim}

\begin{verbatim}
Warning in grid.Call(C_textBounds, as.graphicsAnnot(x$label), x$x, x$y, :
неизвестна ширина символа 0xca в кодировке CP1251
\end{verbatim}

\begin{verbatim}
Warning in grid.Call(C_textBounds, as.graphicsAnnot(x$label), x$x, x$y, :
неизвестна ширина символа 0xf3 в кодировке CP1251
\end{verbatim}

\begin{verbatim}
Warning in grid.Call(C_textBounds, as.graphicsAnnot(x$label), x$x, x$y, :
неизвестна ширина символа 0xec в кодировке CP1251
\end{verbatim}

\begin{verbatim}
Warning in grid.Call(C_textBounds, as.graphicsAnnot(x$label), x$x, x$y, :
неизвестна ширина символа 0xf3 в кодировке CP1251
\end{verbatim}

\begin{verbatim}
Warning in grid.Call(C_textBounds, as.graphicsAnnot(x$label), x$x, x$y, :
неизвестна ширина символа 0xeb в кодировке CP1251
\end{verbatim}

\begin{verbatim}
Warning in grid.Call(C_textBounds, as.graphicsAnnot(x$label), x$x, x$y, :
неизвестна ширина символа 0xff в кодировке CP1251
\end{verbatim}

\begin{verbatim}
Warning in grid.Call(C_textBounds, as.graphicsAnnot(x$label), x$x, x$y, :
неизвестна ширина символа 0xf2 в кодировке CP1251
\end{verbatim}

\begin{verbatim}
Warning in grid.Call(C_textBounds, as.graphicsAnnot(x$label), x$x, x$y, :
неизвестна ширина символа 0xe8 в кодировке CP1251
\end{verbatim}

\begin{verbatim}
Warning in grid.Call(C_textBounds, as.graphicsAnnot(x$label), x$x, x$y, :
неизвестна ширина символа 0xe2 в кодировке CP1251
\end{verbatim}

\begin{verbatim}
Warning in grid.Call(C_textBounds, as.graphicsAnnot(x$label), x$x, x$y, :
неизвестна ширина символа 0xed в кодировке CP1251
\end{verbatim}

\begin{verbatim}
Warning in grid.Call(C_textBounds, as.graphicsAnnot(x$label), x$x, x$y, :
неизвестна ширина символа 0xee в кодировке CP1251
\end{verbatim}

\begin{verbatim}
Warning in grid.Call(C_textBounds, as.graphicsAnnot(x$label), x$x, x$y, :
неизвестна ширина символа 0xe5 в кодировке CP1251
\end{verbatim}

\begin{verbatim}
Warning in grid.Call(C_textBounds, as.graphicsAnnot(x$label), x$x, x$y, :
неизвестна ширина символа 0xf3 в кодировке CP1251
\end{verbatim}

\begin{verbatim}
Warning in grid.Call(C_textBounds, as.graphicsAnnot(x$label), x$x, x$y, :
неизвестна ширина символа 0xf1 в кодировке CP1251
\end{verbatim}

\begin{verbatim}
Warning in grid.Call(C_textBounds, as.graphicsAnnot(x$label), x$x, x$y, :
неизвестна ширина символа 0xe8 в кодировке CP1251
\end{verbatim}

\begin{verbatim}
Warning in grid.Call(C_textBounds, as.graphicsAnnot(x$label), x$x, x$y, :
неизвестна ширина символа 0xeb в кодировке CP1251
\end{verbatim}

\begin{verbatim}
Warning in grid.Call(C_textBounds, as.graphicsAnnot(x$label), x$x, x$y, :
неизвестна ширина символа 0xe8 в кодировке CP1251
\end{verbatim}

\begin{verbatim}
Warning in grid.Call(C_textBounds, as.graphicsAnnot(x$label), x$x, x$y, :
неизвестна ширина символа 0xe5 в кодировке CP1251
\end{verbatim}

\begin{verbatim}
Warning in grid.Call(C_textBounds, as.graphicsAnnot(x$label), x$x, x$y, :
неизвестна ширина символа 0xeb в кодировке CP1251
\end{verbatim}

\begin{verbatim}
Warning in grid.Call(C_textBounds, as.graphicsAnnot(x$label), x$x, x$y, :
неизвестна ширина символа 0xee в кодировке CP1251
\end{verbatim}

\begin{verbatim}
Warning in grid.Call(C_textBounds, as.graphicsAnnot(x$label), x$x, x$y, :
неизвестна ширина символа 0xe2 в кодировке CP1251
\end{verbatim}

\begin{verbatim}
Warning in grid.Call(C_textBounds, as.graphicsAnnot(x$label), x$x, x$y, :
неизвестна ширина символа 0xe0 в кодировке CP1251
\end{verbatim}

\begin{verbatim}
Warning in grid.Call.graphics(C_text, as.graphicsAnnot(x$label), x$x, x$y, :
неизвестна ширина символа 0xca в кодировке CP1251
\end{verbatim}

\begin{verbatim}
Warning in grid.Call.graphics(C_text, as.graphicsAnnot(x$label), x$x, x$y, :
неизвестна ширина символа 0xf3 в кодировке CP1251
\end{verbatim}

\begin{verbatim}
Warning in grid.Call.graphics(C_text, as.graphicsAnnot(x$label), x$x, x$y, :
неизвестна ширина символа 0xec в кодировке CP1251
\end{verbatim}

\begin{verbatim}
Warning in grid.Call.graphics(C_text, as.graphicsAnnot(x$label), x$x, x$y, :
неизвестна ширина символа 0xf3 в кодировке CP1251
\end{verbatim}

\begin{verbatim}
Warning in grid.Call.graphics(C_text, as.graphicsAnnot(x$label), x$x, x$y, :
неизвестна ширина символа 0xeb в кодировке CP1251
\end{verbatim}

\begin{verbatim}
Warning in grid.Call.graphics(C_text, as.graphicsAnnot(x$label), x$x, x$y, :
неизвестна ширина символа 0xff в кодировке CP1251
\end{verbatim}

\begin{verbatim}
Warning in grid.Call.graphics(C_text, as.graphicsAnnot(x$label), x$x, x$y, :
неизвестна ширина символа 0xf2 в кодировке CP1251
\end{verbatim}

\begin{verbatim}
Warning in grid.Call.graphics(C_text, as.graphicsAnnot(x$label), x$x, x$y, :
неизвестна ширина символа 0xe8 в кодировке CP1251
\end{verbatim}

\begin{verbatim}
Warning in grid.Call.graphics(C_text, as.graphicsAnnot(x$label), x$x, x$y, :
неизвестна ширина символа 0xe2 в кодировке CP1251
\end{verbatim}

\begin{verbatim}
Warning in grid.Call.graphics(C_text, as.graphicsAnnot(x$label), x$x, x$y, :
неизвестна ширина символа 0xed в кодировке CP1251
\end{verbatim}

\begin{verbatim}
Warning in grid.Call.graphics(C_text, as.graphicsAnnot(x$label), x$x, x$y, :
неизвестна ширина символа 0xee в кодировке CP1251
\end{verbatim}

\begin{verbatim}
Warning in grid.Call.graphics(C_text, as.graphicsAnnot(x$label), x$x, x$y, :
неизвестна ширина символа 0xe5 в кодировке CP1251
\end{verbatim}

\begin{verbatim}
Warning in grid.Call.graphics(C_text, as.graphicsAnnot(x$label), x$x, x$y, :
неизвестна ширина символа 0xf3 в кодировке CP1251
\end{verbatim}

\begin{verbatim}
Warning in grid.Call.graphics(C_text, as.graphicsAnnot(x$label), x$x, x$y, :
неизвестна ширина символа 0xf1 в кодировке CP1251
\end{verbatim}

\begin{verbatim}
Warning in grid.Call.graphics(C_text, as.graphicsAnnot(x$label), x$x, x$y, :
неизвестна ширина символа 0xe8 в кодировке CP1251
\end{verbatim}

\begin{verbatim}
Warning in grid.Call.graphics(C_text, as.graphicsAnnot(x$label), x$x, x$y, :
неизвестна ширина символа 0xeb в кодировке CP1251
\end{verbatim}

\begin{verbatim}
Warning in grid.Call.graphics(C_text, as.graphicsAnnot(x$label), x$x, x$y, :
неизвестна ширина символа 0xe8 в кодировке CP1251
\end{verbatim}

\begin{verbatim}
Warning in grid.Call.graphics(C_text, as.graphicsAnnot(x$label), x$x, x$y, :
неизвестна ширина символа 0xe5 в кодировке CP1251
\end{verbatim}

\begin{verbatim}
Warning in grid.Call.graphics(C_text, as.graphicsAnnot(x$label), x$x, x$y, :
неизвестна ширина символа 0xeb в кодировке CP1251
\end{verbatim}

\begin{verbatim}
Warning in grid.Call.graphics(C_text, as.graphicsAnnot(x$label), x$x, x$y, :
неизвестна ширина символа 0xee в кодировке CP1251
\end{verbatim}

\begin{verbatim}
Warning in grid.Call.graphics(C_text, as.graphicsAnnot(x$label), x$x, x$y, :
неизвестна ширина символа 0xe2 в кодировке CP1251
\end{verbatim}

\begin{verbatim}
Warning in grid.Call.graphics(C_text, as.graphicsAnnot(x$label), x$x, x$y, :
неизвестна ширина символа 0xe0 в кодировке CP1251
\end{verbatim}

\begin{verbatim}
Warning in grid.Call.graphics(C_text, as.graphicsAnnot(x$label), x$x, x$y, :
неизвестна ширина символа 0xf3 в кодировке CP1251
\end{verbatim}

\begin{verbatim}
Warning in grid.Call.graphics(C_text, as.graphicsAnnot(x$label), x$x, x$y, :
неизвестна ширина символа 0xeb в кодировке CP1251
\end{verbatim}

\begin{verbatim}
Warning in grid.Call.graphics(C_text, as.graphicsAnnot(x$label), x$x, x$y, :
неизвестна ширина символа 0xee в кодировке CP1251
\end{verbatim}

\begin{verbatim}
Warning in grid.Call.graphics(C_text, as.graphicsAnnot(x$label), x$x, x$y, :
неизвестна ширина символа 0xe2 в кодировке CP1251
\end{verbatim}

\begin{verbatim}
Warning in grid.Call.graphics(C_text, as.graphicsAnnot(x$label), x$x, x$y, :
неизвестна ширина символа 0xed в кодировке CP1251
\end{verbatim}

\begin{verbatim}
Warning in grid.Call.graphics(C_text, as.graphicsAnnot(x$label), x$x, x$y, :
неизвестна ширина символа 0xe0 в кодировке CP1251
\end{verbatim}

\begin{verbatim}
Warning in grid.Call.graphics(C_text, as.graphicsAnnot(x$label), x$x, x$y, :
неизвестна ширина символа 0xe5 в кодировке CP1251
\end{verbatim}

\begin{verbatim}
Warning in grid.Call.graphics(C_text, as.graphicsAnnot(x$label), x$x, x$y, :
неизвестна ширина символа 0xe4 в кодировке CP1251
\end{verbatim}

\begin{verbatim}
Warning in grid.Call.graphics(C_text, as.graphicsAnnot(x$label), x$x, x$y, :
неизвестна ширина символа 0xe8 в кодировке CP1251
\end{verbatim}

\begin{verbatim}
Warning in grid.Call.graphics(C_text, as.graphicsAnnot(x$label), x$x, x$y, :
неизвестна ширина символа 0xed в кодировке CP1251
\end{verbatim}

\begin{verbatim}
Warning in grid.Call.graphics(C_text, as.graphicsAnnot(x$label), x$x, x$y, :
неизвестна ширина символа 0xe8 в кодировке CP1251
\end{verbatim}

\begin{verbatim}
Warning in grid.Call.graphics(C_text, as.graphicsAnnot(x$label), x$x, x$y, :
неизвестна ширина символа 0xf6 в кодировке CP1251
\end{verbatim}

\begin{verbatim}
Warning in grid.Call.graphics(C_text, as.graphicsAnnot(x$label), x$x, x$y, :
неизвестна ширина символа 0xf3 в кодировке CP1251
Warning in grid.Call.graphics(C_text, as.graphicsAnnot(x$label), x$x, x$y, :
неизвестна ширина символа 0xf3 в кодировке CP1251
\end{verbatim}

\begin{verbatim}
Warning in grid.Call.graphics(C_text, as.graphicsAnnot(x$label), x$x, x$y, :
неизвестна ширина символа 0xf1 в кодировке CP1251
\end{verbatim}

\begin{verbatim}
Warning in grid.Call.graphics(C_text, as.graphicsAnnot(x$label), x$x, x$y, :
неизвестна ширина символа 0xe8 в кодировке CP1251
\end{verbatim}

\begin{verbatim}
Warning in grid.Call.graphics(C_text, as.graphicsAnnot(x$label), x$x, x$y, :
неизвестна ширина символа 0xeb в кодировке CP1251
\end{verbatim}

\begin{verbatim}
Warning in grid.Call.graphics(C_text, as.graphicsAnnot(x$label), x$x, x$y, :
неизвестна ширина символа 0xe8 в кодировке CP1251
\end{verbatim}

\begin{verbatim}
Warning in grid.Call.graphics(C_text, as.graphicsAnnot(x$label), x$x, x$y, :
неизвестна ширина символа 0xff в кодировке CP1251
\end{verbatim}

\begin{verbatim}
Warning in grid.Call.graphics(C_text, as.graphicsAnnot(x$label), x$x, x$y, :
неизвестна ширина символа 0xd1 в кодировке CP1251
\end{verbatim}

\begin{verbatim}
Warning in grid.Call.graphics(C_text, as.graphicsAnnot(x$label), x$x, x$y, :
неизвестна ширина символа 0xe4 в кодировке CP1251
\end{verbatim}

\begin{verbatim}
Warning in grid.Call.graphics(C_text, as.graphicsAnnot(x$label), x$x, x$y, :
неизвестна ширина символа 0xee в кодировке CP1251
\end{verbatim}

\begin{verbatim}
Warning in grid.Call.graphics(C_text, as.graphicsAnnot(x$label), x$x, x$y, :
неизвестна ширина символа 0xe2 в кодировке CP1251
\end{verbatim}

\begin{verbatim}
Warning in grid.Call.graphics(C_text, as.graphicsAnnot(x$label), x$x, x$y, :
неизвестна ширина символа 0xe5 в кодировке CP1251
\end{verbatim}

\begin{verbatim}
Warning in grid.Call.graphics(C_text, as.graphicsAnnot(x$label), x$x, x$y, :
неизвестна ширина символа 0xf0 в кодировке CP1251
\end{verbatim}

\begin{verbatim}
Warning in grid.Call.graphics(C_text, as.graphicsAnnot(x$label), x$x, x$y, :
неизвестна ширина символа 0xe8 в кодировке CP1251
\end{verbatim}

\begin{verbatim}
Warning in grid.Call.graphics(C_text, as.graphicsAnnot(x$label), x$x, x$y, :
неизвестна ширина символа 0xf2 в кодировке CP1251
\end{verbatim}

\begin{verbatim}
Warning in grid.Call.graphics(C_text, as.graphicsAnnot(x$label), x$x, x$y, :
неизвестна ширина символа 0xe5 в кодировке CP1251
\end{verbatim}

\begin{verbatim}
Warning in grid.Call.graphics(C_text, as.graphicsAnnot(x$label), x$x, x$y, :
неизвестна ширина символа 0xeb в кодировке CP1251
\end{verbatim}

\begin{verbatim}
Warning in grid.Call.graphics(C_text, as.graphicsAnnot(x$label), x$x, x$y, :
неизвестна ширина символа 0xfc в кодировке CP1251
\end{verbatim}

\begin{verbatim}
Warning in grid.Call.graphics(C_text, as.graphicsAnnot(x$label), x$x, x$y, :
неизвестна ширина символа 0xed в кодировке CP1251
\end{verbatim}

\begin{verbatim}
Warning in grid.Call.graphics(C_text, as.graphicsAnnot(x$label), x$x, x$y, :
неизвестна ширина символа 0xfb в кодировке CP1251
\end{verbatim}

\begin{verbatim}
Warning in grid.Call.graphics(C_text, as.graphicsAnnot(x$label), x$x, x$y, :
неизвестна ширина символа 0xec в кодировке CP1251
\end{verbatim}

\begin{verbatim}
Warning in grid.Call.graphics(C_text, as.graphicsAnnot(x$label), x$x, x$y, :
неизвестна ширина символа 0xe8 в кодировке CP1251
Warning in grid.Call.graphics(C_text, as.graphicsAnnot(x$label), x$x, x$y, :
неизвестна ширина символа 0xe8 в кодировке CP1251
\end{verbatim}

\begin{verbatim}
Warning in grid.Call.graphics(C_text, as.graphicsAnnot(x$label), x$x, x$y, :
неизвестна ширина символа 0xed в кодировке CP1251
\end{verbatim}

\begin{verbatim}
Warning in grid.Call.graphics(C_text, as.graphicsAnnot(x$label), x$x, x$y, :
неизвестна ширина символа 0xf2 в кодировке CP1251
\end{verbatim}

\begin{verbatim}
Warning in grid.Call.graphics(C_text, as.graphicsAnnot(x$label), x$x, x$y, :
неизвестна ширина символа 0xe5 в кодировке CP1251
\end{verbatim}

\begin{verbatim}
Warning in grid.Call.graphics(C_text, as.graphicsAnnot(x$label), x$x, x$y, :
неизвестна ширина символа 0xf0 в кодировке CP1251
\end{verbatim}

\begin{verbatim}
Warning in grid.Call.graphics(C_text, as.graphicsAnnot(x$label), x$x, x$y, :
неизвестна ширина символа 0xe2 в кодировке CP1251
\end{verbatim}

\begin{verbatim}
Warning in grid.Call.graphics(C_text, as.graphicsAnnot(x$label), x$x, x$y, :
неизвестна ширина символа 0xe0 в кодировке CP1251
\end{verbatim}

\begin{verbatim}
Warning in grid.Call.graphics(C_text, as.graphicsAnnot(x$label), x$x, x$y, :
неизвестна ширина символа 0xeb в кодировке CP1251
\end{verbatim}

\begin{verbatim}
Warning in grid.Call.graphics(C_text, as.graphicsAnnot(x$label), x$x, x$y, :
неизвестна ширина символа 0xe0 в кодировке CP1251
\end{verbatim}

\begin{verbatim}
Warning in grid.Call.graphics(C_text, as.graphicsAnnot(x$label), x$x, x$y, :
неизвестна ширина символа 0xec в кодировке CP1251
\end{verbatim}

\begin{verbatim}
Warning in grid.Call.graphics(C_text, as.graphicsAnnot(x$label), x$x, x$y, :
неизвестна ширина символа 0xe8 в кодировке CP1251
\end{verbatim}

\begin{verbatim}
Warning in grid.Call.graphics(C_text, as.graphicsAnnot(x$label), x$x, x$y, :
неизвестна ширина символа 0xcc в кодировке CP1251
\end{verbatim}

\begin{verbatim}
Warning in grid.Call.graphics(C_text, as.graphicsAnnot(x$label), x$x, x$y, :
неизвестна ширина символа 0xee в кодировке CP1251
\end{verbatim}

\begin{verbatim}
Warning in grid.Call.graphics(C_text, as.graphicsAnnot(x$label), x$x, x$y, :
неизвестна ширина символа 0xe4 в кодировке CP1251
\end{verbatim}

\begin{verbatim}
Warning in grid.Call.graphics(C_text, as.graphicsAnnot(x$label), x$x, x$y, :
неизвестна ширина символа 0xe5 в кодировке CP1251
\end{verbatim}

\begin{verbatim}
Warning in grid.Call.graphics(C_text, as.graphicsAnnot(x$label), x$x, x$y, :
неизвестна ширина символа 0xeb в кодировке CP1251
\end{verbatim}

\begin{verbatim}
Warning in grid.Call.graphics(C_text, as.graphicsAnnot(x$label), x$x, x$y, :
неизвестна ширина символа 0xfc в кодировке CP1251
\end{verbatim}

\begin{verbatim}
Warning in grid.Call.graphics(C_text, as.graphicsAnnot(x$label), x$x, x$y, :
неизвестна ширина символа 0xcb в кодировке CP1251
\end{verbatim}

\begin{verbatim}
Warning in grid.Call.graphics(C_text, as.graphicsAnnot(x$label), x$x, x$y, :
неизвестна ширина символа 0xe5 в кодировке CP1251
\end{verbatim}

\begin{verbatim}
Warning in grid.Call.graphics(C_text, as.graphicsAnnot(x$label), x$x, x$y, :
неизвестна ширина символа 0xf1 в кодировке CP1251
\end{verbatim}

\begin{verbatim}
Warning in grid.Call.graphics(C_text, as.graphicsAnnot(x$label), x$x, x$y, :
неизвестна ширина символа 0xeb в кодировке CP1251
\end{verbatim}

\begin{verbatim}
Warning in grid.Call.graphics(C_text, as.graphicsAnnot(x$label), x$x, x$y, :
неизвестна ширина символа 0xe8 в кодировке CP1251
\end{verbatim}

\begin{verbatim}
Warning in grid.Call.graphics(C_text, as.graphicsAnnot(x$label), x$x, x$y, :
неизвестна ширина символа 0xe7 в кодировке CP1251
\end{verbatim}

\begin{verbatim}
Warning in grid.Call.graphics(C_text, as.graphicsAnnot(x$label), x$x, x$y, :
неизвестна ширина символа 0xe0 в кодировке CP1251
\end{verbatim}

\begin{verbatim}
Warning in grid.Call.graphics(C_text, as.graphicsAnnot(x$label), x$x, x$y, :
неизвестна ширина символа 0xe2 в кодировке CP1251
\end{verbatim}

\begin{verbatim}
Warning in grid.Call.graphics(C_text, as.graphicsAnnot(x$label), x$x, x$y, :
неизвестна ширина символа 0xe8 в кодировке CP1251
\end{verbatim}

\begin{verbatim}
Warning in grid.Call.graphics(C_text, as.graphicsAnnot(x$label), x$x, x$y, :
неизвестна ширина символа 0xf1 в кодировке CP1251
\end{verbatim}

\begin{verbatim}
Warning in grid.Call.graphics(C_text, as.graphicsAnnot(x$label), x$x, x$y, :
неизвестна ширина символа 0xe8 в кодировке CP1251
\end{verbatim}

\begin{verbatim}
Warning in grid.Call.graphics(C_text, as.graphicsAnnot(x$label), x$x, x$y, :
неизвестна ширина символа 0xec в кодировке CP1251
\end{verbatim}

\begin{verbatim}
Warning in grid.Call.graphics(C_text, as.graphicsAnnot(x$label), x$x, x$y, :
неизвестна ширина символа 0xee в кодировке CP1251
\end{verbatim}

\begin{verbatim}
Warning in grid.Call.graphics(C_text, as.graphicsAnnot(x$label), x$x, x$y, :
неизвестна ширина символа 0xf1 в кодировке CP1251
\end{verbatim}

\begin{verbatim}
Warning in grid.Call.graphics(C_text, as.graphicsAnnot(x$label), x$x, x$y, :
неизвестна ширина символа 0xf2 в кодировке CP1251
\end{verbatim}

\begin{verbatim}
Warning in grid.Call.graphics(C_text, as.graphicsAnnot(x$label), x$x, x$y, :
неизвестна ширина символа 0xfc в кодировке CP1251
\end{verbatim}

\begin{verbatim}
Warning in grid.Call.graphics(C_text, as.graphicsAnnot(x$label), x$x, x$y, :
неизвестна ширина символа 0xee в кодировке CP1251
\end{verbatim}

\begin{verbatim}
Warning in grid.Call.graphics(C_text, as.graphicsAnnot(x$label), x$x, x$y, :
неизвестна ширина символа 0xf2 в кодировке CP1251
\end{verbatim}

\begin{verbatim}
Warning in grid.Call.graphics(C_text, as.graphicsAnnot(x$label), x$x, x$y, :
неизвестна ширина символа 0xea в кодировке CP1251
\end{verbatim}

\begin{verbatim}
Warning in grid.Call.graphics(C_text, as.graphicsAnnot(x$label), x$x, x$y, :
неизвестна ширина символа 0xf3 в кодировке CP1251
\end{verbatim}

\begin{verbatim}
Warning in grid.Call.graphics(C_text, as.graphicsAnnot(x$label), x$x, x$y, :
неизвестна ширина символа 0xec в кодировке CP1251
\end{verbatim}

\begin{verbatim}
Warning in grid.Call.graphics(C_text, as.graphicsAnnot(x$label), x$x, x$y, :
неизвестна ширина символа 0xf3 в кодировке CP1251
\end{verbatim}

\begin{verbatim}
Warning in grid.Call.graphics(C_text, as.graphicsAnnot(x$label), x$x, x$y, :
неизвестна ширина символа 0xeb в кодировке CP1251
\end{verbatim}

\begin{verbatim}
Warning in grid.Call.graphics(C_text, as.graphicsAnnot(x$label), x$x, x$y, :
неизвестна ширина символа 0xff в кодировке CP1251
\end{verbatim}

\begin{verbatim}
Warning in grid.Call.graphics(C_text, as.graphicsAnnot(x$label), x$x, x$y, :
неизвестна ширина символа 0xf2 в кодировке CP1251
\end{verbatim}

\begin{verbatim}
Warning in grid.Call.graphics(C_text, as.graphicsAnnot(x$label), x$x, x$y, :
неизвестна ширина символа 0xe8 в кодировке CP1251
\end{verbatim}

\begin{verbatim}
Warning in grid.Call.graphics(C_text, as.graphicsAnnot(x$label), x$x, x$y, :
неизвестна ширина символа 0xe2 в кодировке CP1251
\end{verbatim}

\begin{verbatim}
Warning in grid.Call.graphics(C_text, as.graphicsAnnot(x$label), x$x, x$y, :
неизвестна ширина символа 0xed в кодировке CP1251
\end{verbatim}

\begin{verbatim}
Warning in grid.Call.graphics(C_text, as.graphicsAnnot(x$label), x$x, x$y, :
неизвестна ширина символа 0xee в кодировке CP1251
\end{verbatim}

\begin{verbatim}
Warning in grid.Call.graphics(C_text, as.graphicsAnnot(x$label), x$x, x$y, :
неизвестна ширина символа 0xe3 в кодировке CP1251
\end{verbatim}

\begin{verbatim}
Warning in grid.Call.graphics(C_text, as.graphicsAnnot(x$label), x$x, x$y, :
неизвестна ширина символа 0xee в кодировке CP1251
\end{verbatim}

\begin{verbatim}
Warning in grid.Call.graphics(C_text, as.graphicsAnnot(x$label), x$x, x$y, :
неизвестна ширина символа 0xf3 в кодировке CP1251
\end{verbatim}

\begin{verbatim}
Warning in grid.Call.graphics(C_text, as.graphicsAnnot(x$label), x$x, x$y, :
неизвестна ширина символа 0xf1 в кодировке CP1251
\end{verbatim}

\begin{verbatim}
Warning in grid.Call.graphics(C_text, as.graphicsAnnot(x$label), x$x, x$y, :
неизвестна ширина символа 0xe8 в кодировке CP1251
\end{verbatim}

\begin{verbatim}
Warning in grid.Call.graphics(C_text, as.graphicsAnnot(x$label), x$x, x$y, :
неизвестна ширина символа 0xeb в кодировке CP1251
\end{verbatim}

\begin{verbatim}
Warning in grid.Call.graphics(C_text, as.graphicsAnnot(x$label), x$x, x$y, :
неизвестна ширина символа 0xe8 в кодировке CP1251
\end{verbatim}

\begin{verbatim}
Warning in grid.Call.graphics(C_text, as.graphicsAnnot(x$label), x$x, x$y, :
неизвестна ширина символа 0xff в кодировке CP1251
\end{verbatim}

\pandocbounded{\includegraphics[keepaspectratio]{chapter18_files/figure-pdf/unnamed-chunk-1-2.pdf}}

\begin{Shaded}
\begin{Highlighting}[]
\CommentTok{\# Сохранение графика (раскомментируйте для использования)}
\CommentTok{\# ggsave("Leslie\_Model\_Facets\_2007\_2018\_with\_CI.png", plot = leslie\_facet\_plot, width = 14, height = 10, dpi = 300)}

\CommentTok{\# 6. СРАВНИТЕЛЬНЫЙ АНАЛИЗ МОДЕЛЕЙ ЛЕСЛИ И ДЕЛУРИ ========================================================================}

\CommentTok{\# Инициализация списка для хранения результатов}
\NormalTok{results\_list }\OtherTok{\textless{}{-}} \FunctionTok{list}\NormalTok{()}

\CommentTok{\# Анализ для каждого года}
\ControlFlowTok{for}\NormalTok{ (year }\ControlFlowTok{in} \DecValTok{2007}\SpecialCharTok{:}\DecValTok{2018}\NormalTok{) \{}
\NormalTok{  year\_data }\OtherTok{\textless{}{-}}\NormalTok{ LESLIDATA }\SpecialCharTok{\%\textgreater{}\%} 
    \FunctionTok{filter}\NormalTok{(YEAR }\SpecialCharTok{==}\NormalTok{ year) }\SpecialCharTok{\%\textgreater{}\%}
    \FunctionTok{na.omit}\NormalTok{()}
  
  \ControlFlowTok{if}\NormalTok{ (}\FunctionTok{nrow}\NormalTok{(year\_data) }\SpecialCharTok{\textless{}} \DecValTok{3}\NormalTok{) }\ControlFlowTok{next}
  
  \CommentTok{\# Модель Лесли через FSA}
\NormalTok{  leslie\_model }\OtherTok{\textless{}{-}} \FunctionTok{try}\NormalTok{(}\FunctionTok{depletion}\NormalTok{(year\_data}\SpecialCharTok{$}\NormalTok{CATCH, year\_data}\SpecialCharTok{$}\NormalTok{EFFORT, }\AttributeTok{method =} \StringTok{"Leslie"}\NormalTok{), }\AttributeTok{silent =} \ConstantTok{TRUE}\NormalTok{)}
  \ControlFlowTok{if}\NormalTok{ (}\FunctionTok{inherits}\NormalTok{(leslie\_model, }\StringTok{"try{-}error"}\NormalTok{)) \{}
\NormalTok{    leslie\_no }\OtherTok{\textless{}{-}}\NormalTok{ leslie\_lci }\OtherTok{\textless{}{-}}\NormalTok{ leslie\_uci }\OtherTok{\textless{}{-}} \ConstantTok{NA}
\NormalTok{  \} }\ControlFlowTok{else}\NormalTok{ \{}
\NormalTok{    leslie\_ci }\OtherTok{\textless{}{-}} \FunctionTok{confint}\NormalTok{(leslie\_model)}
\NormalTok{    leslie\_no }\OtherTok{\textless{}{-}} \FunctionTok{coef}\NormalTok{(leslie\_model)[}\StringTok{"No"}\NormalTok{]}
\NormalTok{    leslie\_lci }\OtherTok{\textless{}{-}}\NormalTok{ leslie\_ci[}\StringTok{"No"}\NormalTok{, }\StringTok{"95\% LCI"}\NormalTok{]}
\NormalTok{    leslie\_uci }\OtherTok{\textless{}{-}}\NormalTok{ leslie\_ci[}\StringTok{"No"}\NormalTok{, }\StringTok{"95\% UCI"}\NormalTok{]}
\NormalTok{  \}}
  
  \CommentTok{\# Модель Делури через FSA}
\NormalTok{  delury\_model }\OtherTok{\textless{}{-}} \FunctionTok{try}\NormalTok{(}\FunctionTok{depletion}\NormalTok{(year\_data}\SpecialCharTok{$}\NormalTok{CATCH, year\_data}\SpecialCharTok{$}\NormalTok{EFFORT, }\AttributeTok{method =} \StringTok{"Delury"}\NormalTok{), }\AttributeTok{silent =} \ConstantTok{TRUE}\NormalTok{)}
  \ControlFlowTok{if}\NormalTok{ (}\FunctionTok{inherits}\NormalTok{(delury\_model, }\StringTok{"try{-}error"}\NormalTok{)) \{}
\NormalTok{    delury\_no }\OtherTok{\textless{}{-}}\NormalTok{ delury\_lci }\OtherTok{\textless{}{-}}\NormalTok{ delury\_uci }\OtherTok{\textless{}{-}} \ConstantTok{NA}
\NormalTok{  \} }\ControlFlowTok{else}\NormalTok{ \{}
\NormalTok{    delury\_ci }\OtherTok{\textless{}{-}} \FunctionTok{confint}\NormalTok{(delury\_model)}
\NormalTok{    delury\_no }\OtherTok{\textless{}{-}} \FunctionTok{coef}\NormalTok{(delury\_model)[}\StringTok{"No"}\NormalTok{]}
\NormalTok{    delury\_lci }\OtherTok{\textless{}{-}}\NormalTok{ delury\_ci[}\StringTok{"No"}\NormalTok{, }\StringTok{"95\% LCI"}\NormalTok{]}
\NormalTok{    delury\_uci }\OtherTok{\textless{}{-}}\NormalTok{ delury\_ci[}\StringTok{"No"}\NormalTok{, }\StringTok{"95\% UCI"}\NormalTok{]}
\NormalTok{  \}}
  
  \CommentTok{\# Сохранение результатов}
\NormalTok{  results\_list[[}\FunctionTok{as.character}\NormalTok{(year)]] }\OtherTok{\textless{}{-}} \FunctionTok{data.frame}\NormalTok{(}
    \AttributeTok{Year =}\NormalTok{ year,}
    \AttributeTok{Model =} \FunctionTok{c}\NormalTok{(}\StringTok{"Лесли"}\NormalTok{, }\StringTok{"Делури"}\NormalTok{),}
    \AttributeTok{Initial\_Biomass =} \FunctionTok{c}\NormalTok{(leslie\_no, delury\_no),}
    \AttributeTok{LCI =} \FunctionTok{c}\NormalTok{(leslie\_lci, delury\_lci),}
    \AttributeTok{UCI =} \FunctionTok{c}\NormalTok{(leslie\_uci, delury\_uci)}
\NormalTok{  )}
\NormalTok{\}}
\end{Highlighting}
\end{Shaded}

\begin{verbatim}
Warning: Estimates are suspect as model did not exhibit a significantly (p>0.05)
  negative slope.
\end{verbatim}

\begin{verbatim}
Warning: Estimates are suspect as model did not exhibit a significantly (p>0.05)
  negative slope.
Warning: Estimates are suspect as model did not exhibit a significantly (p>0.05)
  negative slope.
Warning: Estimates are suspect as model did not exhibit a significantly (p>0.05)
  negative slope.
Warning: Estimates are suspect as model did not exhibit a significantly (p>0.05)
  negative slope.
Warning: Estimates are suspect as model did not exhibit a significantly (p>0.05)
  negative slope.
Warning: Estimates are suspect as model did not exhibit a significantly (p>0.05)
  negative slope.
Warning: Estimates are suspect as model did not exhibit a significantly (p>0.05)
  negative slope.
\end{verbatim}

\begin{Shaded}
\begin{Highlighting}[]
\CommentTok{\# Преобразование списка в dataframe}
\NormalTok{results\_df }\OtherTok{\textless{}{-}} \FunctionTok{bind\_rows}\NormalTok{(results\_list)}

\CommentTok{\# 7. ВИЗУАЛИЗАЦИЯ СРАВНЕНИЯ МОДЕЛЕЙ =====================================================================================}

\CommentTok{\# График динамики начальной биомассы}
\NormalTok{biomass\_plot }\OtherTok{\textless{}{-}} \FunctionTok{ggplot}\NormalTok{(results\_df, }\FunctionTok{aes}\NormalTok{(}\AttributeTok{x =}\NormalTok{ Year, }\AttributeTok{y =}\NormalTok{ Initial\_Biomass}\SpecialCharTok{/}\DecValTok{1000}\NormalTok{, }\AttributeTok{color =}\NormalTok{ Model)) }\SpecialCharTok{+}
  \FunctionTok{geom\_point}\NormalTok{(}\AttributeTok{size =} \FloatTok{2.5}\NormalTok{) }\SpecialCharTok{+}
  \FunctionTok{geom\_errorbar}\NormalTok{(}\FunctionTok{aes}\NormalTok{(}\AttributeTok{ymin =}\NormalTok{ LCI}\SpecialCharTok{/}\DecValTok{1000}\NormalTok{, }\AttributeTok{ymax =}\NormalTok{ UCI}\SpecialCharTok{/}\DecValTok{1000}\NormalTok{), }\AttributeTok{width =} \FloatTok{0.2}\NormalTok{) }\SpecialCharTok{+}
  \FunctionTok{geom\_line}\NormalTok{(}\FunctionTok{aes}\NormalTok{(}\AttributeTok{group =}\NormalTok{ Model), }\AttributeTok{linetype =} \StringTok{"dashed"}\NormalTok{) }\SpecialCharTok{+}
  \FunctionTok{labs}\NormalTok{(}
    \AttributeTok{title =} \StringTok{"Динамика начальной биомассы (2007{-}2018)"}\NormalTok{,}
    \AttributeTok{subtitle =} \StringTok{"Модели Лесли и Делури с 95\% доверительными интервалами"}\NormalTok{,}
    \AttributeTok{x =} \StringTok{"Год"}\NormalTok{,}
    \AttributeTok{y =} \StringTok{"Начальная биомасса, тыс. т"}\NormalTok{,}
    \AttributeTok{color =} \StringTok{"Модель"}
\NormalTok{  ) }\SpecialCharTok{+}
  \FunctionTok{theme\_minimal}\NormalTok{() }\SpecialCharTok{+}
  \FunctionTok{theme}\NormalTok{(}
    \AttributeTok{plot.title =} \FunctionTok{element\_text}\NormalTok{(}\AttributeTok{hjust =} \FloatTok{0.5}\NormalTok{, }\AttributeTok{face =} \StringTok{"bold"}\NormalTok{),}
    \AttributeTok{plot.subtitle =} \FunctionTok{element\_text}\NormalTok{(}\AttributeTok{hjust =} \FloatTok{0.5}\NormalTok{),}
    \AttributeTok{legend.position =} \StringTok{"top"}
\NormalTok{  ) }\SpecialCharTok{+}
  \FunctionTok{scale\_color\_manual}\NormalTok{(}\AttributeTok{values =} \FunctionTok{c}\NormalTok{(}\StringTok{"Лесли"} \OtherTok{=} \StringTok{"blue"}\NormalTok{, }\StringTok{"Делури"} \OtherTok{=} \StringTok{"red"}\NormalTok{))}

\FunctionTok{print}\NormalTok{(biomass\_plot)}
\end{Highlighting}
\end{Shaded}

\begin{verbatim}
Warning in grid.Call(C_textBounds, as.graphicsAnnot(x$label), x$x, x$y, :
неизвестна ширина символа 0xc4 в кодировке CP1251
\end{verbatim}

\begin{verbatim}
Warning in grid.Call(C_textBounds, as.graphicsAnnot(x$label), x$x, x$y, :
неизвестна ширина символа 0xe5 в кодировке CP1251
\end{verbatim}

\begin{verbatim}
Warning in grid.Call(C_textBounds, as.graphicsAnnot(x$label), x$x, x$y, :
неизвестна ширина символа 0xeb в кодировке CP1251
\end{verbatim}

\begin{verbatim}
Warning in grid.Call(C_textBounds, as.graphicsAnnot(x$label), x$x, x$y, :
неизвестна ширина символа 0xf3 в кодировке CP1251
\end{verbatim}

\begin{verbatim}
Warning in grid.Call(C_textBounds, as.graphicsAnnot(x$label), x$x, x$y, :
неизвестна ширина символа 0xf0 в кодировке CP1251
\end{verbatim}

\begin{verbatim}
Warning in grid.Call(C_textBounds, as.graphicsAnnot(x$label), x$x, x$y, :
неизвестна ширина символа 0xe8 в кодировке CP1251
\end{verbatim}

\begin{verbatim}
Warning in grid.Call(C_textBounds, as.graphicsAnnot(x$label), x$x, x$y, :
неизвестна ширина символа 0xcb в кодировке CP1251
\end{verbatim}

\begin{verbatim}
Warning in grid.Call(C_textBounds, as.graphicsAnnot(x$label), x$x, x$y, :
неизвестна ширина символа 0xe5 в кодировке CP1251
\end{verbatim}

\begin{verbatim}
Warning in grid.Call(C_textBounds, as.graphicsAnnot(x$label), x$x, x$y, :
неизвестна ширина символа 0xf1 в кодировке CP1251
\end{verbatim}

\begin{verbatim}
Warning in grid.Call(C_textBounds, as.graphicsAnnot(x$label), x$x, x$y, :
неизвестна ширина символа 0xeb в кодировке CP1251
\end{verbatim}

\begin{verbatim}
Warning in grid.Call(C_textBounds, as.graphicsAnnot(x$label), x$x, x$y, :
неизвестна ширина символа 0xe8 в кодировке CP1251
\end{verbatim}

\begin{verbatim}
Warning in grid.Call(C_textBounds, as.graphicsAnnot(x$label), x$x, x$y, :
неизвестна ширина символа 0xc4 в кодировке CP1251
\end{verbatim}

\begin{verbatim}
Warning in grid.Call(C_textBounds, as.graphicsAnnot(x$label), x$x, x$y, :
неизвестна ширина символа 0xe5 в кодировке CP1251
\end{verbatim}

\begin{verbatim}
Warning in grid.Call(C_textBounds, as.graphicsAnnot(x$label), x$x, x$y, :
неизвестна ширина символа 0xeb в кодировке CP1251
\end{verbatim}

\begin{verbatim}
Warning in grid.Call(C_textBounds, as.graphicsAnnot(x$label), x$x, x$y, :
неизвестна ширина символа 0xf3 в кодировке CP1251
\end{verbatim}

\begin{verbatim}
Warning in grid.Call(C_textBounds, as.graphicsAnnot(x$label), x$x, x$y, :
неизвестна ширина символа 0xf0 в кодировке CP1251
\end{verbatim}

\begin{verbatim}
Warning in grid.Call(C_textBounds, as.graphicsAnnot(x$label), x$x, x$y, :
неизвестна ширина символа 0xe8 в кодировке CP1251
\end{verbatim}

\begin{verbatim}
Warning in grid.Call(C_textBounds, as.graphicsAnnot(x$label), x$x, x$y, :
неизвестна ширина символа 0xcb в кодировке CP1251
\end{verbatim}

\begin{verbatim}
Warning in grid.Call(C_textBounds, as.graphicsAnnot(x$label), x$x, x$y, :
неизвестна ширина символа 0xe5 в кодировке CP1251
\end{verbatim}

\begin{verbatim}
Warning in grid.Call(C_textBounds, as.graphicsAnnot(x$label), x$x, x$y, :
неизвестна ширина символа 0xf1 в кодировке CP1251
\end{verbatim}

\begin{verbatim}
Warning in grid.Call(C_textBounds, as.graphicsAnnot(x$label), x$x, x$y, :
неизвестна ширина символа 0xeb в кодировке CP1251
\end{verbatim}

\begin{verbatim}
Warning in grid.Call(C_textBounds, as.graphicsAnnot(x$label), x$x, x$y, :
неизвестна ширина символа 0xe8 в кодировке CP1251
\end{verbatim}

\begin{verbatim}
Warning in grid.Call(C_textBounds, as.graphicsAnnot(x$label), x$x, x$y, :
неизвестна ширина символа 0xcc в кодировке CP1251
\end{verbatim}

\begin{verbatim}
Warning in grid.Call(C_textBounds, as.graphicsAnnot(x$label), x$x, x$y, :
неизвестна ширина символа 0xee в кодировке CP1251
\end{verbatim}

\begin{verbatim}
Warning in grid.Call(C_textBounds, as.graphicsAnnot(x$label), x$x, x$y, :
неизвестна ширина символа 0xe4 в кодировке CP1251
\end{verbatim}

\begin{verbatim}
Warning in grid.Call(C_textBounds, as.graphicsAnnot(x$label), x$x, x$y, :
неизвестна ширина символа 0xe5 в кодировке CP1251
\end{verbatim}

\begin{verbatim}
Warning in grid.Call(C_textBounds, as.graphicsAnnot(x$label), x$x, x$y, :
неизвестна ширина символа 0xeb в кодировке CP1251
\end{verbatim}

\begin{verbatim}
Warning in grid.Call(C_textBounds, as.graphicsAnnot(x$label), x$x, x$y, :
неизвестна ширина символа 0xfc в кодировке CP1251
\end{verbatim}

\begin{verbatim}
Warning in grid.Call(C_textBounds, as.graphicsAnnot(x$label), x$x, x$y, :
неизвестна ширина символа 0xcc в кодировке CP1251
\end{verbatim}

\begin{verbatim}
Warning in grid.Call(C_textBounds, as.graphicsAnnot(x$label), x$x, x$y, :
неизвестна ширина символа 0xee в кодировке CP1251
\end{verbatim}

\begin{verbatim}
Warning in grid.Call(C_textBounds, as.graphicsAnnot(x$label), x$x, x$y, :
неизвестна ширина символа 0xe4 в кодировке CP1251
\end{verbatim}

\begin{verbatim}
Warning in grid.Call(C_textBounds, as.graphicsAnnot(x$label), x$x, x$y, :
неизвестна ширина символа 0xe5 в кодировке CP1251
\end{verbatim}

\begin{verbatim}
Warning in grid.Call(C_textBounds, as.graphicsAnnot(x$label), x$x, x$y, :
неизвестна ширина символа 0xeb в кодировке CP1251
\end{verbatim}

\begin{verbatim}
Warning in grid.Call(C_textBounds, as.graphicsAnnot(x$label), x$x, x$y, :
неизвестна ширина символа 0xfc в кодировке CP1251
\end{verbatim}

\begin{verbatim}
Warning in grid.Call(C_textBounds, as.graphicsAnnot(x$label), x$x, x$y, :
неизвестна ширина символа 0xcd в кодировке CP1251
\end{verbatim}

\begin{verbatim}
Warning in grid.Call(C_textBounds, as.graphicsAnnot(x$label), x$x, x$y, :
неизвестна ширина символа 0xe0 в кодировке CP1251
\end{verbatim}

\begin{verbatim}
Warning in grid.Call(C_textBounds, as.graphicsAnnot(x$label), x$x, x$y, :
неизвестна ширина символа 0xf7 в кодировке CP1251
\end{verbatim}

\begin{verbatim}
Warning in grid.Call(C_textBounds, as.graphicsAnnot(x$label), x$x, x$y, :
неизвестна ширина символа 0xe0 в кодировке CP1251
\end{verbatim}

\begin{verbatim}
Warning in grid.Call(C_textBounds, as.graphicsAnnot(x$label), x$x, x$y, :
неизвестна ширина символа 0xeb в кодировке CP1251
\end{verbatim}

\begin{verbatim}
Warning in grid.Call(C_textBounds, as.graphicsAnnot(x$label), x$x, x$y, :
неизвестна ширина символа 0xfc в кодировке CP1251
\end{verbatim}

\begin{verbatim}
Warning in grid.Call(C_textBounds, as.graphicsAnnot(x$label), x$x, x$y, :
неизвестна ширина символа 0xed в кодировке CP1251
\end{verbatim}

\begin{verbatim}
Warning in grid.Call(C_textBounds, as.graphicsAnnot(x$label), x$x, x$y, :
неизвестна ширина символа 0xe0 в кодировке CP1251
\end{verbatim}

\begin{verbatim}
Warning in grid.Call(C_textBounds, as.graphicsAnnot(x$label), x$x, x$y, :
неизвестна ширина символа 0xff в кодировке CP1251
\end{verbatim}

\begin{verbatim}
Warning in grid.Call(C_textBounds, as.graphicsAnnot(x$label), x$x, x$y, :
неизвестна ширина символа 0xe1 в кодировке CP1251
\end{verbatim}

\begin{verbatim}
Warning in grid.Call(C_textBounds, as.graphicsAnnot(x$label), x$x, x$y, :
неизвестна ширина символа 0xe8 в кодировке CP1251
\end{verbatim}

\begin{verbatim}
Warning in grid.Call(C_textBounds, as.graphicsAnnot(x$label), x$x, x$y, :
неизвестна ширина символа 0xee в кодировке CP1251
\end{verbatim}

\begin{verbatim}
Warning in grid.Call(C_textBounds, as.graphicsAnnot(x$label), x$x, x$y, :
неизвестна ширина символа 0xec в кодировке CP1251
\end{verbatim}

\begin{verbatim}
Warning in grid.Call(C_textBounds, as.graphicsAnnot(x$label), x$x, x$y, :
неизвестна ширина символа 0xe0 в кодировке CP1251
\end{verbatim}

\begin{verbatim}
Warning in grid.Call(C_textBounds, as.graphicsAnnot(x$label), x$x, x$y, :
неизвестна ширина символа 0xf1 в кодировке CP1251
Warning in grid.Call(C_textBounds, as.graphicsAnnot(x$label), x$x, x$y, :
неизвестна ширина символа 0xf1 в кодировке CP1251
\end{verbatim}

\begin{verbatim}
Warning in grid.Call(C_textBounds, as.graphicsAnnot(x$label), x$x, x$y, :
неизвестна ширина символа 0xe0 в кодировке CP1251
\end{verbatim}

\begin{verbatim}
Warning in grid.Call(C_textBounds, as.graphicsAnnot(x$label), x$x, x$y, :
неизвестна ширина символа 0xf2 в кодировке CP1251
\end{verbatim}

\begin{verbatim}
Warning in grid.Call(C_textBounds, as.graphicsAnnot(x$label), x$x, x$y, :
неизвестна ширина символа 0xfb в кодировке CP1251
\end{verbatim}

\begin{verbatim}
Warning in grid.Call(C_textBounds, as.graphicsAnnot(x$label), x$x, x$y, :
неизвестна ширина символа 0xf1 в кодировке CP1251
\end{verbatim}

\begin{verbatim}
Warning in grid.Call(C_textBounds, as.graphicsAnnot(x$label), x$x, x$y, :
неизвестна ширина символа 0xf2 в кодировке CP1251
\end{verbatim}

\begin{verbatim}
Warning in grid.Call(C_textBounds, as.graphicsAnnot(x$label), x$x, x$y, :
неизвестна ширина символа 0xc4 в кодировке CP1251
\end{verbatim}

\begin{verbatim}
Warning in grid.Call(C_textBounds, as.graphicsAnnot(x$label), x$x, x$y, :
неизвестна ширина символа 0xe8 в кодировке CP1251
\end{verbatim}

\begin{verbatim}
Warning in grid.Call(C_textBounds, as.graphicsAnnot(x$label), x$x, x$y, :
неизвестна ширина символа 0xed в кодировке CP1251
\end{verbatim}

\begin{verbatim}
Warning in grid.Call(C_textBounds, as.graphicsAnnot(x$label), x$x, x$y, :
неизвестна ширина символа 0xe0 в кодировке CP1251
\end{verbatim}

\begin{verbatim}
Warning in grid.Call(C_textBounds, as.graphicsAnnot(x$label), x$x, x$y, :
неизвестна ширина символа 0xec в кодировке CP1251
\end{verbatim}

\begin{verbatim}
Warning in grid.Call(C_textBounds, as.graphicsAnnot(x$label), x$x, x$y, :
неизвестна ширина символа 0xe8 в кодировке CP1251
\end{verbatim}

\begin{verbatim}
Warning in grid.Call(C_textBounds, as.graphicsAnnot(x$label), x$x, x$y, :
неизвестна ширина символа 0xea в кодировке CP1251
\end{verbatim}

\begin{verbatim}
Warning in grid.Call(C_textBounds, as.graphicsAnnot(x$label), x$x, x$y, :
неизвестна ширина символа 0xe0 в кодировке CP1251
\end{verbatim}

\begin{verbatim}
Warning in grid.Call(C_textBounds, as.graphicsAnnot(x$label), x$x, x$y, :
неизвестна ширина символа 0xed в кодировке CP1251
\end{verbatim}

\begin{verbatim}
Warning in grid.Call(C_textBounds, as.graphicsAnnot(x$label), x$x, x$y, :
неизвестна ширина символа 0xe0 в кодировке CP1251
\end{verbatim}

\begin{verbatim}
Warning in grid.Call(C_textBounds, as.graphicsAnnot(x$label), x$x, x$y, :
неизвестна ширина символа 0xf7 в кодировке CP1251
\end{verbatim}

\begin{verbatim}
Warning in grid.Call(C_textBounds, as.graphicsAnnot(x$label), x$x, x$y, :
неизвестна ширина символа 0xe0 в кодировке CP1251
\end{verbatim}

\begin{verbatim}
Warning in grid.Call(C_textBounds, as.graphicsAnnot(x$label), x$x, x$y, :
неизвестна ширина символа 0xeb в кодировке CP1251
\end{verbatim}

\begin{verbatim}
Warning in grid.Call(C_textBounds, as.graphicsAnnot(x$label), x$x, x$y, :
неизвестна ширина символа 0xfc в кодировке CP1251
\end{verbatim}

\begin{verbatim}
Warning in grid.Call(C_textBounds, as.graphicsAnnot(x$label), x$x, x$y, :
неизвестна ширина символа 0xed в кодировке CP1251
\end{verbatim}

\begin{verbatim}
Warning in grid.Call(C_textBounds, as.graphicsAnnot(x$label), x$x, x$y, :
неизвестна ширина символа 0xee в кодировке CP1251
\end{verbatim}

\begin{verbatim}
Warning in grid.Call(C_textBounds, as.graphicsAnnot(x$label), x$x, x$y, :
неизвестна ширина символа 0xe9 в кодировке CP1251
\end{verbatim}

\begin{verbatim}
Warning in grid.Call(C_textBounds, as.graphicsAnnot(x$label), x$x, x$y, :
неизвестна ширина символа 0xe1 в кодировке CP1251
\end{verbatim}

\begin{verbatim}
Warning in grid.Call(C_textBounds, as.graphicsAnnot(x$label), x$x, x$y, :
неизвестна ширина символа 0xe8 в кодировке CP1251
\end{verbatim}

\begin{verbatim}
Warning in grid.Call(C_textBounds, as.graphicsAnnot(x$label), x$x, x$y, :
неизвестна ширина символа 0xee в кодировке CP1251
\end{verbatim}

\begin{verbatim}
Warning in grid.Call(C_textBounds, as.graphicsAnnot(x$label), x$x, x$y, :
неизвестна ширина символа 0xec в кодировке CP1251
\end{verbatim}

\begin{verbatim}
Warning in grid.Call(C_textBounds, as.graphicsAnnot(x$label), x$x, x$y, :
неизвестна ширина символа 0xe0 в кодировке CP1251
\end{verbatim}

\begin{verbatim}
Warning in grid.Call(C_textBounds, as.graphicsAnnot(x$label), x$x, x$y, :
неизвестна ширина символа 0xf1 в кодировке CP1251
Warning in grid.Call(C_textBounds, as.graphicsAnnot(x$label), x$x, x$y, :
неизвестна ширина символа 0xf1 в кодировке CP1251
\end{verbatim}

\begin{verbatim}
Warning in grid.Call(C_textBounds, as.graphicsAnnot(x$label), x$x, x$y, :
неизвестна ширина символа 0xfb в кодировке CP1251
\end{verbatim}

\begin{verbatim}
Warning in grid.Call(C_textBounds, as.graphicsAnnot(x$label), x$x, x$y, :
неизвестна ширина символа 0xcc в кодировке CP1251
\end{verbatim}

\begin{verbatim}
Warning in grid.Call(C_textBounds, as.graphicsAnnot(x$label), x$x, x$y, :
неизвестна ширина символа 0xee в кодировке CP1251
\end{verbatim}

\begin{verbatim}
Warning in grid.Call(C_textBounds, as.graphicsAnnot(x$label), x$x, x$y, :
неизвестна ширина символа 0xe4 в кодировке CP1251
\end{verbatim}

\begin{verbatim}
Warning in grid.Call(C_textBounds, as.graphicsAnnot(x$label), x$x, x$y, :
неизвестна ширина символа 0xe5 в кодировке CP1251
\end{verbatim}

\begin{verbatim}
Warning in grid.Call(C_textBounds, as.graphicsAnnot(x$label), x$x, x$y, :
неизвестна ширина символа 0xeb в кодировке CP1251
\end{verbatim}

\begin{verbatim}
Warning in grid.Call(C_textBounds, as.graphicsAnnot(x$label), x$x, x$y, :
неизвестна ширина символа 0xe8 в кодировке CP1251
\end{verbatim}

\begin{verbatim}
Warning in grid.Call(C_textBounds, as.graphicsAnnot(x$label), x$x, x$y, :
неизвестна ширина символа 0xcb в кодировке CP1251
\end{verbatim}

\begin{verbatim}
Warning in grid.Call(C_textBounds, as.graphicsAnnot(x$label), x$x, x$y, :
неизвестна ширина символа 0xe5 в кодировке CP1251
\end{verbatim}

\begin{verbatim}
Warning in grid.Call(C_textBounds, as.graphicsAnnot(x$label), x$x, x$y, :
неизвестна ширина символа 0xf1 в кодировке CP1251
\end{verbatim}

\begin{verbatim}
Warning in grid.Call(C_textBounds, as.graphicsAnnot(x$label), x$x, x$y, :
неизвестна ширина символа 0xeb в кодировке CP1251
\end{verbatim}

\begin{verbatim}
Warning in grid.Call(C_textBounds, as.graphicsAnnot(x$label), x$x, x$y, :
неизвестна ширина символа 0xe8 в кодировке CP1251
Warning in grid.Call(C_textBounds, as.graphicsAnnot(x$label), x$x, x$y, :
неизвестна ширина символа 0xe8 в кодировке CP1251
\end{verbatim}

\begin{verbatim}
Warning in grid.Call(C_textBounds, as.graphicsAnnot(x$label), x$x, x$y, :
неизвестна ширина символа 0xc4 в кодировке CP1251
\end{verbatim}

\begin{verbatim}
Warning in grid.Call(C_textBounds, as.graphicsAnnot(x$label), x$x, x$y, :
неизвестна ширина символа 0xe5 в кодировке CP1251
\end{verbatim}

\begin{verbatim}
Warning in grid.Call(C_textBounds, as.graphicsAnnot(x$label), x$x, x$y, :
неизвестна ширина символа 0xeb в кодировке CP1251
\end{verbatim}

\begin{verbatim}
Warning in grid.Call(C_textBounds, as.graphicsAnnot(x$label), x$x, x$y, :
неизвестна ширина символа 0xf3 в кодировке CP1251
\end{verbatim}

\begin{verbatim}
Warning in grid.Call(C_textBounds, as.graphicsAnnot(x$label), x$x, x$y, :
неизвестна ширина символа 0xf0 в кодировке CP1251
\end{verbatim}

\begin{verbatim}
Warning in grid.Call(C_textBounds, as.graphicsAnnot(x$label), x$x, x$y, :
неизвестна ширина символа 0xe8 в кодировке CP1251
\end{verbatim}

\begin{verbatim}
Warning in grid.Call(C_textBounds, as.graphicsAnnot(x$label), x$x, x$y, :
неизвестна ширина символа 0xf1 в кодировке CP1251
\end{verbatim}

\begin{verbatim}
Warning in grid.Call(C_textBounds, as.graphicsAnnot(x$label), x$x, x$y, :
неизвестна ширина символа 0xe4 в кодировке CP1251
\end{verbatim}

\begin{verbatim}
Warning in grid.Call(C_textBounds, as.graphicsAnnot(x$label), x$x, x$y, :
неизвестна ширина символа 0xee в кодировке CP1251
\end{verbatim}

\begin{verbatim}
Warning in grid.Call(C_textBounds, as.graphicsAnnot(x$label), x$x, x$y, :
неизвестна ширина символа 0xe2 в кодировке CP1251
\end{verbatim}

\begin{verbatim}
Warning in grid.Call(C_textBounds, as.graphicsAnnot(x$label), x$x, x$y, :
неизвестна ширина символа 0xe5 в кодировке CP1251
\end{verbatim}

\begin{verbatim}
Warning in grid.Call(C_textBounds, as.graphicsAnnot(x$label), x$x, x$y, :
неизвестна ширина символа 0xf0 в кодировке CP1251
\end{verbatim}

\begin{verbatim}
Warning in grid.Call(C_textBounds, as.graphicsAnnot(x$label), x$x, x$y, :
неизвестна ширина символа 0xe8 в кодировке CP1251
\end{verbatim}

\begin{verbatim}
Warning in grid.Call(C_textBounds, as.graphicsAnnot(x$label), x$x, x$y, :
неизвестна ширина символа 0xf2 в кодировке CP1251
\end{verbatim}

\begin{verbatim}
Warning in grid.Call(C_textBounds, as.graphicsAnnot(x$label), x$x, x$y, :
неизвестна ширина символа 0xe5 в кодировке CP1251
\end{verbatim}

\begin{verbatim}
Warning in grid.Call(C_textBounds, as.graphicsAnnot(x$label), x$x, x$y, :
неизвестна ширина символа 0xeb в кодировке CP1251
\end{verbatim}

\begin{verbatim}
Warning in grid.Call(C_textBounds, as.graphicsAnnot(x$label), x$x, x$y, :
неизвестна ширина символа 0xfc в кодировке CP1251
\end{verbatim}

\begin{verbatim}
Warning in grid.Call(C_textBounds, as.graphicsAnnot(x$label), x$x, x$y, :
неизвестна ширина символа 0xed в кодировке CP1251
\end{verbatim}

\begin{verbatim}
Warning in grid.Call(C_textBounds, as.graphicsAnnot(x$label), x$x, x$y, :
неизвестна ширина символа 0xfb в кодировке CP1251
\end{verbatim}

\begin{verbatim}
Warning in grid.Call(C_textBounds, as.graphicsAnnot(x$label), x$x, x$y, :
неизвестна ширина символа 0xec в кодировке CP1251
\end{verbatim}

\begin{verbatim}
Warning in grid.Call(C_textBounds, as.graphicsAnnot(x$label), x$x, x$y, :
неизвестна ширина символа 0xe8 в кодировке CP1251
Warning in grid.Call(C_textBounds, as.graphicsAnnot(x$label), x$x, x$y, :
неизвестна ширина символа 0xe8 в кодировке CP1251
\end{verbatim}

\begin{verbatim}
Warning in grid.Call(C_textBounds, as.graphicsAnnot(x$label), x$x, x$y, :
неизвестна ширина символа 0xed в кодировке CP1251
\end{verbatim}

\begin{verbatim}
Warning in grid.Call(C_textBounds, as.graphicsAnnot(x$label), x$x, x$y, :
неизвестна ширина символа 0xf2 в кодировке CP1251
\end{verbatim}

\begin{verbatim}
Warning in grid.Call(C_textBounds, as.graphicsAnnot(x$label), x$x, x$y, :
неизвестна ширина символа 0xe5 в кодировке CP1251
\end{verbatim}

\begin{verbatim}
Warning in grid.Call(C_textBounds, as.graphicsAnnot(x$label), x$x, x$y, :
неизвестна ширина символа 0xf0 в кодировке CP1251
\end{verbatim}

\begin{verbatim}
Warning in grid.Call(C_textBounds, as.graphicsAnnot(x$label), x$x, x$y, :
неизвестна ширина символа 0xe2 в кодировке CP1251
\end{verbatim}

\begin{verbatim}
Warning in grid.Call(C_textBounds, as.graphicsAnnot(x$label), x$x, x$y, :
неизвестна ширина символа 0xe0 в кодировке CP1251
\end{verbatim}

\begin{verbatim}
Warning in grid.Call(C_textBounds, as.graphicsAnnot(x$label), x$x, x$y, :
неизвестна ширина символа 0xeb в кодировке CP1251
\end{verbatim}

\begin{verbatim}
Warning in grid.Call(C_textBounds, as.graphicsAnnot(x$label), x$x, x$y, :
неизвестна ширина символа 0xe0 в кодировке CP1251
\end{verbatim}

\begin{verbatim}
Warning in grid.Call(C_textBounds, as.graphicsAnnot(x$label), x$x, x$y, :
неизвестна ширина символа 0xec в кодировке CP1251
\end{verbatim}

\begin{verbatim}
Warning in grid.Call(C_textBounds, as.graphicsAnnot(x$label), x$x, x$y, :
неизвестна ширина символа 0xe8 в кодировке CP1251
\end{verbatim}

\begin{verbatim}
Warning in grid.Call(C_textBounds, as.graphicsAnnot(x$label), x$x, x$y, :
неизвестна ширина символа 0xc3 в кодировке CP1251
\end{verbatim}

\begin{verbatim}
Warning in grid.Call(C_textBounds, as.graphicsAnnot(x$label), x$x, x$y, :
неизвестна ширина символа 0xee в кодировке CP1251
\end{verbatim}

\begin{verbatim}
Warning in grid.Call(C_textBounds, as.graphicsAnnot(x$label), x$x, x$y, :
неизвестна ширина символа 0xe4 в кодировке CP1251
\end{verbatim}

\begin{verbatim}
Warning in grid.Call.graphics(C_text, as.graphicsAnnot(x$label), x$x, x$y, :
неизвестна ширина символа 0xc3 в кодировке CP1251
\end{verbatim}

\begin{verbatim}
Warning in grid.Call.graphics(C_text, as.graphicsAnnot(x$label), x$x, x$y, :
неизвестна ширина символа 0xee в кодировке CP1251
\end{verbatim}

\begin{verbatim}
Warning in grid.Call.graphics(C_text, as.graphicsAnnot(x$label), x$x, x$y, :
неизвестна ширина символа 0xe4 в кодировке CP1251
\end{verbatim}

\begin{verbatim}
Warning in grid.Call.graphics(C_text, as.graphicsAnnot(x$label), x$x, x$y, :
неизвестна ширина символа 0xcd в кодировке CP1251
\end{verbatim}

\begin{verbatim}
Warning in grid.Call.graphics(C_text, as.graphicsAnnot(x$label), x$x, x$y, :
неизвестна ширина символа 0xe0 в кодировке CP1251
\end{verbatim}

\begin{verbatim}
Warning in grid.Call.graphics(C_text, as.graphicsAnnot(x$label), x$x, x$y, :
неизвестна ширина символа 0xf7 в кодировке CP1251
\end{verbatim}

\begin{verbatim}
Warning in grid.Call.graphics(C_text, as.graphicsAnnot(x$label), x$x, x$y, :
неизвестна ширина символа 0xe0 в кодировке CP1251
\end{verbatim}

\begin{verbatim}
Warning in grid.Call.graphics(C_text, as.graphicsAnnot(x$label), x$x, x$y, :
неизвестна ширина символа 0xeb в кодировке CP1251
\end{verbatim}

\begin{verbatim}
Warning in grid.Call.graphics(C_text, as.graphicsAnnot(x$label), x$x, x$y, :
неизвестна ширина символа 0xfc в кодировке CP1251
\end{verbatim}

\begin{verbatim}
Warning in grid.Call.graphics(C_text, as.graphicsAnnot(x$label), x$x, x$y, :
неизвестна ширина символа 0xed в кодировке CP1251
\end{verbatim}

\begin{verbatim}
Warning in grid.Call.graphics(C_text, as.graphicsAnnot(x$label), x$x, x$y, :
неизвестна ширина символа 0xe0 в кодировке CP1251
\end{verbatim}

\begin{verbatim}
Warning in grid.Call.graphics(C_text, as.graphicsAnnot(x$label), x$x, x$y, :
неизвестна ширина символа 0xff в кодировке CP1251
\end{verbatim}

\begin{verbatim}
Warning in grid.Call.graphics(C_text, as.graphicsAnnot(x$label), x$x, x$y, :
неизвестна ширина символа 0xe1 в кодировке CP1251
\end{verbatim}

\begin{verbatim}
Warning in grid.Call.graphics(C_text, as.graphicsAnnot(x$label), x$x, x$y, :
неизвестна ширина символа 0xe8 в кодировке CP1251
\end{verbatim}

\begin{verbatim}
Warning in grid.Call.graphics(C_text, as.graphicsAnnot(x$label), x$x, x$y, :
неизвестна ширина символа 0xee в кодировке CP1251
\end{verbatim}

\begin{verbatim}
Warning in grid.Call.graphics(C_text, as.graphicsAnnot(x$label), x$x, x$y, :
неизвестна ширина символа 0xec в кодировке CP1251
\end{verbatim}

\begin{verbatim}
Warning in grid.Call.graphics(C_text, as.graphicsAnnot(x$label), x$x, x$y, :
неизвестна ширина символа 0xe0 в кодировке CP1251
\end{verbatim}

\begin{verbatim}
Warning in grid.Call.graphics(C_text, as.graphicsAnnot(x$label), x$x, x$y, :
неизвестна ширина символа 0xf1 в кодировке CP1251
Warning in grid.Call.graphics(C_text, as.graphicsAnnot(x$label), x$x, x$y, :
неизвестна ширина символа 0xf1 в кодировке CP1251
\end{verbatim}

\begin{verbatim}
Warning in grid.Call.graphics(C_text, as.graphicsAnnot(x$label), x$x, x$y, :
неизвестна ширина символа 0xe0 в кодировке CP1251
\end{verbatim}

\begin{verbatim}
Warning in grid.Call.graphics(C_text, as.graphicsAnnot(x$label), x$x, x$y, :
неизвестна ширина символа 0xf2 в кодировке CP1251
\end{verbatim}

\begin{verbatim}
Warning in grid.Call.graphics(C_text, as.graphicsAnnot(x$label), x$x, x$y, :
неизвестна ширина символа 0xfb в кодировке CP1251
\end{verbatim}

\begin{verbatim}
Warning in grid.Call.graphics(C_text, as.graphicsAnnot(x$label), x$x, x$y, :
неизвестна ширина символа 0xf1 в кодировке CP1251
\end{verbatim}

\begin{verbatim}
Warning in grid.Call.graphics(C_text, as.graphicsAnnot(x$label), x$x, x$y, :
неизвестна ширина символа 0xf2 в кодировке CP1251
\end{verbatim}

\begin{verbatim}
Warning in grid.Call.graphics(C_text, as.graphicsAnnot(x$label), x$x, x$y, :
неизвестна ширина символа 0xcc в кодировке CP1251
\end{verbatim}

\begin{verbatim}
Warning in grid.Call.graphics(C_text, as.graphicsAnnot(x$label), x$x, x$y, :
неизвестна ширина символа 0xee в кодировке CP1251
\end{verbatim}

\begin{verbatim}
Warning in grid.Call.graphics(C_text, as.graphicsAnnot(x$label), x$x, x$y, :
неизвестна ширина символа 0xe4 в кодировке CP1251
\end{verbatim}

\begin{verbatim}
Warning in grid.Call.graphics(C_text, as.graphicsAnnot(x$label), x$x, x$y, :
неизвестна ширина символа 0xe5 в кодировке CP1251
\end{verbatim}

\begin{verbatim}
Warning in grid.Call.graphics(C_text, as.graphicsAnnot(x$label), x$x, x$y, :
неизвестна ширина символа 0xeb в кодировке CP1251
\end{verbatim}

\begin{verbatim}
Warning in grid.Call.graphics(C_text, as.graphicsAnnot(x$label), x$x, x$y, :
неизвестна ширина символа 0xfc в кодировке CP1251
\end{verbatim}

\begin{verbatim}
Warning in grid.Call.graphics(C_text, as.graphicsAnnot(x$label), x$x, x$y, :
неизвестна ширина символа 0xc4 в кодировке CP1251
\end{verbatim}

\begin{verbatim}
Warning in grid.Call.graphics(C_text, as.graphicsAnnot(x$label), x$x, x$y, :
неизвестна ширина символа 0xe5 в кодировке CP1251
\end{verbatim}

\begin{verbatim}
Warning in grid.Call.graphics(C_text, as.graphicsAnnot(x$label), x$x, x$y, :
неизвестна ширина символа 0xeb в кодировке CP1251
\end{verbatim}

\begin{verbatim}
Warning in grid.Call.graphics(C_text, as.graphicsAnnot(x$label), x$x, x$y, :
неизвестна ширина символа 0xf3 в кодировке CP1251
\end{verbatim}

\begin{verbatim}
Warning in grid.Call.graphics(C_text, as.graphicsAnnot(x$label), x$x, x$y, :
неизвестна ширина символа 0xf0 в кодировке CP1251
\end{verbatim}

\begin{verbatim}
Warning in grid.Call.graphics(C_text, as.graphicsAnnot(x$label), x$x, x$y, :
неизвестна ширина символа 0xe8 в кодировке CP1251
\end{verbatim}

\begin{verbatim}
Warning in grid.Call.graphics(C_text, as.graphicsAnnot(x$label), x$x, x$y, :
неизвестна ширина символа 0xcb в кодировке CP1251
\end{verbatim}

\begin{verbatim}
Warning in grid.Call.graphics(C_text, as.graphicsAnnot(x$label), x$x, x$y, :
неизвестна ширина символа 0xe5 в кодировке CP1251
\end{verbatim}

\begin{verbatim}
Warning in grid.Call.graphics(C_text, as.graphicsAnnot(x$label), x$x, x$y, :
неизвестна ширина символа 0xf1 в кодировке CP1251
\end{verbatim}

\begin{verbatim}
Warning in grid.Call.graphics(C_text, as.graphicsAnnot(x$label), x$x, x$y, :
неизвестна ширина символа 0xeb в кодировке CP1251
\end{verbatim}

\begin{verbatim}
Warning in grid.Call.graphics(C_text, as.graphicsAnnot(x$label), x$x, x$y, :
неизвестна ширина символа 0xe8 в кодировке CP1251
\end{verbatim}

\begin{verbatim}
Warning in grid.Call.graphics(C_text, as.graphicsAnnot(x$label), x$x, x$y, :
неизвестна ширина символа 0xcc в кодировке CP1251
\end{verbatim}

\begin{verbatim}
Warning in grid.Call.graphics(C_text, as.graphicsAnnot(x$label), x$x, x$y, :
неизвестна ширина символа 0xee в кодировке CP1251
\end{verbatim}

\begin{verbatim}
Warning in grid.Call.graphics(C_text, as.graphicsAnnot(x$label), x$x, x$y, :
неизвестна ширина символа 0xe4 в кодировке CP1251
\end{verbatim}

\begin{verbatim}
Warning in grid.Call.graphics(C_text, as.graphicsAnnot(x$label), x$x, x$y, :
неизвестна ширина символа 0xe5 в кодировке CP1251
\end{verbatim}

\begin{verbatim}
Warning in grid.Call.graphics(C_text, as.graphicsAnnot(x$label), x$x, x$y, :
неизвестна ширина символа 0xeb в кодировке CP1251
\end{verbatim}

\begin{verbatim}
Warning in grid.Call.graphics(C_text, as.graphicsAnnot(x$label), x$x, x$y, :
неизвестна ширина символа 0xe8 в кодировке CP1251
\end{verbatim}

\begin{verbatim}
Warning in grid.Call.graphics(C_text, as.graphicsAnnot(x$label), x$x, x$y, :
неизвестна ширина символа 0xcb в кодировке CP1251
\end{verbatim}

\begin{verbatim}
Warning in grid.Call.graphics(C_text, as.graphicsAnnot(x$label), x$x, x$y, :
неизвестна ширина символа 0xe5 в кодировке CP1251
\end{verbatim}

\begin{verbatim}
Warning in grid.Call.graphics(C_text, as.graphicsAnnot(x$label), x$x, x$y, :
неизвестна ширина символа 0xf1 в кодировке CP1251
\end{verbatim}

\begin{verbatim}
Warning in grid.Call.graphics(C_text, as.graphicsAnnot(x$label), x$x, x$y, :
неизвестна ширина символа 0xeb в кодировке CP1251
\end{verbatim}

\begin{verbatim}
Warning in grid.Call.graphics(C_text, as.graphicsAnnot(x$label), x$x, x$y, :
неизвестна ширина символа 0xe8 в кодировке CP1251
Warning in grid.Call.graphics(C_text, as.graphicsAnnot(x$label), x$x, x$y, :
неизвестна ширина символа 0xe8 в кодировке CP1251
\end{verbatim}

\begin{verbatim}
Warning in grid.Call.graphics(C_text, as.graphicsAnnot(x$label), x$x, x$y, :
неизвестна ширина символа 0xc4 в кодировке CP1251
\end{verbatim}

\begin{verbatim}
Warning in grid.Call.graphics(C_text, as.graphicsAnnot(x$label), x$x, x$y, :
неизвестна ширина символа 0xe5 в кодировке CP1251
\end{verbatim}

\begin{verbatim}
Warning in grid.Call.graphics(C_text, as.graphicsAnnot(x$label), x$x, x$y, :
неизвестна ширина символа 0xeb в кодировке CP1251
\end{verbatim}

\begin{verbatim}
Warning in grid.Call.graphics(C_text, as.graphicsAnnot(x$label), x$x, x$y, :
неизвестна ширина символа 0xf3 в кодировке CP1251
\end{verbatim}

\begin{verbatim}
Warning in grid.Call.graphics(C_text, as.graphicsAnnot(x$label), x$x, x$y, :
неизвестна ширина символа 0xf0 в кодировке CP1251
\end{verbatim}

\begin{verbatim}
Warning in grid.Call.graphics(C_text, as.graphicsAnnot(x$label), x$x, x$y, :
неизвестна ширина символа 0xe8 в кодировке CP1251
\end{verbatim}

\begin{verbatim}
Warning in grid.Call.graphics(C_text, as.graphicsAnnot(x$label), x$x, x$y, :
неизвестна ширина символа 0xf1 в кодировке CP1251
\end{verbatim}

\begin{verbatim}
Warning in grid.Call.graphics(C_text, as.graphicsAnnot(x$label), x$x, x$y, :
неизвестна ширина символа 0xe4 в кодировке CP1251
\end{verbatim}

\begin{verbatim}
Warning in grid.Call.graphics(C_text, as.graphicsAnnot(x$label), x$x, x$y, :
неизвестна ширина символа 0xee в кодировке CP1251
\end{verbatim}

\begin{verbatim}
Warning in grid.Call.graphics(C_text, as.graphicsAnnot(x$label), x$x, x$y, :
неизвестна ширина символа 0xe2 в кодировке CP1251
\end{verbatim}

\begin{verbatim}
Warning in grid.Call.graphics(C_text, as.graphicsAnnot(x$label), x$x, x$y, :
неизвестна ширина символа 0xe5 в кодировке CP1251
\end{verbatim}

\begin{verbatim}
Warning in grid.Call.graphics(C_text, as.graphicsAnnot(x$label), x$x, x$y, :
неизвестна ширина символа 0xf0 в кодировке CP1251
\end{verbatim}

\begin{verbatim}
Warning in grid.Call.graphics(C_text, as.graphicsAnnot(x$label), x$x, x$y, :
неизвестна ширина символа 0xe8 в кодировке CP1251
\end{verbatim}

\begin{verbatim}
Warning in grid.Call.graphics(C_text, as.graphicsAnnot(x$label), x$x, x$y, :
неизвестна ширина символа 0xf2 в кодировке CP1251
\end{verbatim}

\begin{verbatim}
Warning in grid.Call.graphics(C_text, as.graphicsAnnot(x$label), x$x, x$y, :
неизвестна ширина символа 0xe5 в кодировке CP1251
\end{verbatim}

\begin{verbatim}
Warning in grid.Call.graphics(C_text, as.graphicsAnnot(x$label), x$x, x$y, :
неизвестна ширина символа 0xeb в кодировке CP1251
\end{verbatim}

\begin{verbatim}
Warning in grid.Call.graphics(C_text, as.graphicsAnnot(x$label), x$x, x$y, :
неизвестна ширина символа 0xfc в кодировке CP1251
\end{verbatim}

\begin{verbatim}
Warning in grid.Call.graphics(C_text, as.graphicsAnnot(x$label), x$x, x$y, :
неизвестна ширина символа 0xed в кодировке CP1251
\end{verbatim}

\begin{verbatim}
Warning in grid.Call.graphics(C_text, as.graphicsAnnot(x$label), x$x, x$y, :
неизвестна ширина символа 0xfb в кодировке CP1251
\end{verbatim}

\begin{verbatim}
Warning in grid.Call.graphics(C_text, as.graphicsAnnot(x$label), x$x, x$y, :
неизвестна ширина символа 0xec в кодировке CP1251
\end{verbatim}

\begin{verbatim}
Warning in grid.Call.graphics(C_text, as.graphicsAnnot(x$label), x$x, x$y, :
неизвестна ширина символа 0xe8 в кодировке CP1251
Warning in grid.Call.graphics(C_text, as.graphicsAnnot(x$label), x$x, x$y, :
неизвестна ширина символа 0xe8 в кодировке CP1251
\end{verbatim}

\begin{verbatim}
Warning in grid.Call.graphics(C_text, as.graphicsAnnot(x$label), x$x, x$y, :
неизвестна ширина символа 0xed в кодировке CP1251
\end{verbatim}

\begin{verbatim}
Warning in grid.Call.graphics(C_text, as.graphicsAnnot(x$label), x$x, x$y, :
неизвестна ширина символа 0xf2 в кодировке CP1251
\end{verbatim}

\begin{verbatim}
Warning in grid.Call.graphics(C_text, as.graphicsAnnot(x$label), x$x, x$y, :
неизвестна ширина символа 0xe5 в кодировке CP1251
\end{verbatim}

\begin{verbatim}
Warning in grid.Call.graphics(C_text, as.graphicsAnnot(x$label), x$x, x$y, :
неизвестна ширина символа 0xf0 в кодировке CP1251
\end{verbatim}

\begin{verbatim}
Warning in grid.Call.graphics(C_text, as.graphicsAnnot(x$label), x$x, x$y, :
неизвестна ширина символа 0xe2 в кодировке CP1251
\end{verbatim}

\begin{verbatim}
Warning in grid.Call.graphics(C_text, as.graphicsAnnot(x$label), x$x, x$y, :
неизвестна ширина символа 0xe0 в кодировке CP1251
\end{verbatim}

\begin{verbatim}
Warning in grid.Call.graphics(C_text, as.graphicsAnnot(x$label), x$x, x$y, :
неизвестна ширина символа 0xeb в кодировке CP1251
\end{verbatim}

\begin{verbatim}
Warning in grid.Call.graphics(C_text, as.graphicsAnnot(x$label), x$x, x$y, :
неизвестна ширина символа 0xe0 в кодировке CP1251
\end{verbatim}

\begin{verbatim}
Warning in grid.Call.graphics(C_text, as.graphicsAnnot(x$label), x$x, x$y, :
неизвестна ширина символа 0xec в кодировке CP1251
\end{verbatim}

\begin{verbatim}
Warning in grid.Call.graphics(C_text, as.graphicsAnnot(x$label), x$x, x$y, :
неизвестна ширина символа 0xe8 в кодировке CP1251
\end{verbatim}

\begin{verbatim}
Warning in grid.Call.graphics(C_text, as.graphicsAnnot(x$label), x$x, x$y, :
неизвестна ширина символа 0xc4 в кодировке CP1251
\end{verbatim}

\begin{verbatim}
Warning in grid.Call.graphics(C_text, as.graphicsAnnot(x$label), x$x, x$y, :
неизвестна ширина символа 0xe8 в кодировке CP1251
\end{verbatim}

\begin{verbatim}
Warning in grid.Call.graphics(C_text, as.graphicsAnnot(x$label), x$x, x$y, :
неизвестна ширина символа 0xed в кодировке CP1251
\end{verbatim}

\begin{verbatim}
Warning in grid.Call.graphics(C_text, as.graphicsAnnot(x$label), x$x, x$y, :
неизвестна ширина символа 0xe0 в кодировке CP1251
\end{verbatim}

\begin{verbatim}
Warning in grid.Call.graphics(C_text, as.graphicsAnnot(x$label), x$x, x$y, :
неизвестна ширина символа 0xec в кодировке CP1251
\end{verbatim}

\begin{verbatim}
Warning in grid.Call.graphics(C_text, as.graphicsAnnot(x$label), x$x, x$y, :
неизвестна ширина символа 0xe8 в кодировке CP1251
\end{verbatim}

\begin{verbatim}
Warning in grid.Call.graphics(C_text, as.graphicsAnnot(x$label), x$x, x$y, :
неизвестна ширина символа 0xea в кодировке CP1251
\end{verbatim}

\begin{verbatim}
Warning in grid.Call.graphics(C_text, as.graphicsAnnot(x$label), x$x, x$y, :
неизвестна ширина символа 0xe0 в кодировке CP1251
\end{verbatim}

\begin{verbatim}
Warning in grid.Call.graphics(C_text, as.graphicsAnnot(x$label), x$x, x$y, :
неизвестна ширина символа 0xed в кодировке CP1251
\end{verbatim}

\begin{verbatim}
Warning in grid.Call.graphics(C_text, as.graphicsAnnot(x$label), x$x, x$y, :
неизвестна ширина символа 0xe0 в кодировке CP1251
\end{verbatim}

\begin{verbatim}
Warning in grid.Call.graphics(C_text, as.graphicsAnnot(x$label), x$x, x$y, :
неизвестна ширина символа 0xf7 в кодировке CP1251
\end{verbatim}

\begin{verbatim}
Warning in grid.Call.graphics(C_text, as.graphicsAnnot(x$label), x$x, x$y, :
неизвестна ширина символа 0xe0 в кодировке CP1251
\end{verbatim}

\begin{verbatim}
Warning in grid.Call.graphics(C_text, as.graphicsAnnot(x$label), x$x, x$y, :
неизвестна ширина символа 0xeb в кодировке CP1251
\end{verbatim}

\begin{verbatim}
Warning in grid.Call.graphics(C_text, as.graphicsAnnot(x$label), x$x, x$y, :
неизвестна ширина символа 0xfc в кодировке CP1251
\end{verbatim}

\begin{verbatim}
Warning in grid.Call.graphics(C_text, as.graphicsAnnot(x$label), x$x, x$y, :
неизвестна ширина символа 0xed в кодировке CP1251
\end{verbatim}

\begin{verbatim}
Warning in grid.Call.graphics(C_text, as.graphicsAnnot(x$label), x$x, x$y, :
неизвестна ширина символа 0xee в кодировке CP1251
\end{verbatim}

\begin{verbatim}
Warning in grid.Call.graphics(C_text, as.graphicsAnnot(x$label), x$x, x$y, :
неизвестна ширина символа 0xe9 в кодировке CP1251
\end{verbatim}

\begin{verbatim}
Warning in grid.Call.graphics(C_text, as.graphicsAnnot(x$label), x$x, x$y, :
неизвестна ширина символа 0xe1 в кодировке CP1251
\end{verbatim}

\begin{verbatim}
Warning in grid.Call.graphics(C_text, as.graphicsAnnot(x$label), x$x, x$y, :
неизвестна ширина символа 0xe8 в кодировке CP1251
\end{verbatim}

\begin{verbatim}
Warning in grid.Call.graphics(C_text, as.graphicsAnnot(x$label), x$x, x$y, :
неизвестна ширина символа 0xee в кодировке CP1251
\end{verbatim}

\begin{verbatim}
Warning in grid.Call.graphics(C_text, as.graphicsAnnot(x$label), x$x, x$y, :
неизвестна ширина символа 0xec в кодировке CP1251
\end{verbatim}

\begin{verbatim}
Warning in grid.Call.graphics(C_text, as.graphicsAnnot(x$label), x$x, x$y, :
неизвестна ширина символа 0xe0 в кодировке CP1251
\end{verbatim}

\begin{verbatim}
Warning in grid.Call.graphics(C_text, as.graphicsAnnot(x$label), x$x, x$y, :
неизвестна ширина символа 0xf1 в кодировке CP1251
Warning in grid.Call.graphics(C_text, as.graphicsAnnot(x$label), x$x, x$y, :
неизвестна ширина символа 0xf1 в кодировке CP1251
\end{verbatim}

\begin{verbatim}
Warning in grid.Call.graphics(C_text, as.graphicsAnnot(x$label), x$x, x$y, :
неизвестна ширина символа 0xfb в кодировке CP1251
\end{verbatim}

\pandocbounded{\includegraphics[keepaspectratio]{chapter18_files/figure-pdf/unnamed-chunk-1-3.pdf}}

\begin{Shaded}
\begin{Highlighting}[]
\CommentTok{\# 8. ДЕТАЛИЗИРОВАННЫЙ АНАЛИЗ ПАРАМЕТРОВ =================================================================================}

\CommentTok{\# Создание таблиц с параметрами моделей}
\NormalTok{leslie\_all\_years }\OtherTok{\textless{}{-}} \FunctionTok{data.frame}\NormalTok{()}
\NormalTok{delury\_all\_years }\OtherTok{\textless{}{-}} \FunctionTok{data.frame}\NormalTok{()}

\ControlFlowTok{for}\NormalTok{ (year }\ControlFlowTok{in} \DecValTok{2007}\SpecialCharTok{:}\DecValTok{2018}\NormalTok{) \{}
\NormalTok{  year\_data }\OtherTok{\textless{}{-}}\NormalTok{ LESLIDATA }\SpecialCharTok{\%\textgreater{}\%} \FunctionTok{filter}\NormalTok{(YEAR }\SpecialCharTok{==}\NormalTok{ year) }\SpecialCharTok{\%\textgreater{}\%} \FunctionTok{na.omit}\NormalTok{()}
  \ControlFlowTok{if}\NormalTok{ (}\FunctionTok{nrow}\NormalTok{(year\_data) }\SpecialCharTok{\textless{}} \DecValTok{3}\NormalTok{) }\ControlFlowTok{next}
  
  \CommentTok{\# Анализ для модели Лесли}
  \FunctionTok{tryCatch}\NormalTok{(\{}
\NormalTok{    leslie\_model }\OtherTok{\textless{}{-}} \FunctionTok{depletion}\NormalTok{(year\_data}\SpecialCharTok{$}\NormalTok{CATCH, year\_data}\SpecialCharTok{$}\NormalTok{EFFORT, }\AttributeTok{method =} \StringTok{"Leslie"}\NormalTok{)}
\NormalTok{    leslie\_ci }\OtherTok{\textless{}{-}} \FunctionTok{confint}\NormalTok{(leslie\_model)}
\NormalTok{    leslie\_all\_years }\OtherTok{\textless{}{-}} \FunctionTok{rbind}\NormalTok{(leslie\_all\_years, }\FunctionTok{data.frame}\NormalTok{(}
\NormalTok{      Модель }\OtherTok{=} \StringTok{"Лесли"}\NormalTok{,}
\NormalTok{      Год }\OtherTok{=}\NormalTok{ year,}
      \AttributeTok{B0 =} \FunctionTok{round}\NormalTok{(leslie\_model}\SpecialCharTok{$}\NormalTok{est[}\StringTok{"No"}\NormalTok{, }\StringTok{"Estimate"}\NormalTok{], }\DecValTok{2}\NormalTok{),}
      \AttributeTok{B0\_LCI =} \FunctionTok{round}\NormalTok{(leslie\_ci[}\StringTok{"No"}\NormalTok{, }\StringTok{"95\% LCI"}\NormalTok{], }\DecValTok{2}\NormalTok{),}
      \AttributeTok{B0\_UCI =} \FunctionTok{round}\NormalTok{(leslie\_ci[}\StringTok{"No"}\NormalTok{, }\StringTok{"95\% UCI"}\NormalTok{], }\DecValTok{2}\NormalTok{),}
      \AttributeTok{q =} \FunctionTok{round}\NormalTok{(leslie\_model}\SpecialCharTok{$}\NormalTok{est[}\StringTok{"q"}\NormalTok{, }\StringTok{"Estimate"}\NormalTok{], }\DecValTok{6}\NormalTok{),}
      \AttributeTok{q\_LCI =} \FunctionTok{round}\NormalTok{(leslie\_ci[}\StringTok{"q"}\NormalTok{, }\StringTok{"95\% LCI"}\NormalTok{], }\DecValTok{6}\NormalTok{),}
      \AttributeTok{q\_UCI =} \FunctionTok{round}\NormalTok{(leslie\_ci[}\StringTok{"q"}\NormalTok{, }\StringTok{"95\% UCI"}\NormalTok{], }\DecValTok{6}\NormalTok{),}
      \AttributeTok{R2 =} \FunctionTok{round}\NormalTok{(}\FunctionTok{summary}\NormalTok{(leslie\_model}\SpecialCharTok{$}\NormalTok{lm)}\SpecialCharTok{$}\NormalTok{r.squared, }\DecValTok{4}\NormalTok{)}
\NormalTok{    ))}
\NormalTok{  \}, }\AttributeTok{error =} \ControlFlowTok{function}\NormalTok{(e) \{}
    \FunctionTok{message}\NormalTok{(}\FunctionTok{paste}\NormalTok{(}\StringTok{"Ошибка в модели Лесли для"}\NormalTok{, year, }\StringTok{":"}\NormalTok{, e}\SpecialCharTok{$}\NormalTok{message))}
\NormalTok{  \})}
  
  \CommentTok{\# Анализ для модели Делури}
  \FunctionTok{tryCatch}\NormalTok{(\{}
\NormalTok{    delury\_model }\OtherTok{\textless{}{-}} \FunctionTok{depletion}\NormalTok{(year\_data}\SpecialCharTok{$}\NormalTok{CATCH, year\_data}\SpecialCharTok{$}\NormalTok{EFFORT, }\AttributeTok{method =} \StringTok{"DeLury"}\NormalTok{)}
\NormalTok{    delury\_ci }\OtherTok{\textless{}{-}} \FunctionTok{confint}\NormalTok{(delury\_model)}
\NormalTok{    delury\_all\_years }\OtherTok{\textless{}{-}} \FunctionTok{rbind}\NormalTok{(delury\_all\_years, }\FunctionTok{data.frame}\NormalTok{(}
\NormalTok{      Модель }\OtherTok{=} \StringTok{"Делури"}\NormalTok{,}
\NormalTok{      Год }\OtherTok{=}\NormalTok{ year,}
      \AttributeTok{B0 =} \FunctionTok{round}\NormalTok{(delury\_model}\SpecialCharTok{$}\NormalTok{est[}\StringTok{"No"}\NormalTok{, }\StringTok{"Estimate"}\NormalTok{], }\DecValTok{2}\NormalTok{),}
      \AttributeTok{B0\_LCI =} \FunctionTok{round}\NormalTok{(delury\_ci[}\StringTok{"No"}\NormalTok{, }\StringTok{"95\% LCI"}\NormalTok{], }\DecValTok{2}\NormalTok{),}
      \AttributeTok{B0\_UCI =} \FunctionTok{round}\NormalTok{(delury\_ci[}\StringTok{"No"}\NormalTok{, }\StringTok{"95\% UCI"}\NormalTok{], }\DecValTok{2}\NormalTok{),}
      \AttributeTok{q =} \FunctionTok{round}\NormalTok{(delury\_model}\SpecialCharTok{$}\NormalTok{est[}\StringTok{"q"}\NormalTok{, }\StringTok{"Estimate"}\NormalTok{], }\DecValTok{6}\NormalTok{),}
      \AttributeTok{q\_LCI =} \FunctionTok{round}\NormalTok{(delury\_ci[}\StringTok{"q"}\NormalTok{, }\StringTok{"95\% LCI"}\NormalTok{], }\DecValTok{6}\NormalTok{),}
      \AttributeTok{q\_UCI =} \FunctionTok{round}\NormalTok{(delury\_ci[}\StringTok{"q"}\NormalTok{, }\StringTok{"95\% UCI"}\NormalTok{], }\DecValTok{6}\NormalTok{),}
      \AttributeTok{R2 =} \FunctionTok{round}\NormalTok{(}\FunctionTok{summary}\NormalTok{(delury\_model}\SpecialCharTok{$}\NormalTok{lm)}\SpecialCharTok{$}\NormalTok{r.squared, }\DecValTok{4}\NormalTok{)}
\NormalTok{    ))}
\NormalTok{  \}, }\AttributeTok{error =} \ControlFlowTok{function}\NormalTok{(e) \{}
    \FunctionTok{message}\NormalTok{(}\FunctionTok{paste}\NormalTok{(}\StringTok{"Ошибка в модели Делури для"}\NormalTok{, year, }\StringTok{":"}\NormalTok{, e}\SpecialCharTok{$}\NormalTok{message))}
\NormalTok{  \})}
\NormalTok{\}}
\end{Highlighting}
\end{Shaded}

\begin{verbatim}
Warning: Estimates are suspect as model did not exhibit a significantly (p>0.05)
  negative slope.
\end{verbatim}

\begin{verbatim}
Warning: Estimates are suspect as model did not exhibit a significantly (p>0.05)
  negative slope.
Warning: Estimates are suspect as model did not exhibit a significantly (p>0.05)
  negative slope.
Warning: Estimates are suspect as model did not exhibit a significantly (p>0.05)
  negative slope.
Warning: Estimates are suspect as model did not exhibit a significantly (p>0.05)
  negative slope.
Warning: Estimates are suspect as model did not exhibit a significantly (p>0.05)
  negative slope.
Warning: Estimates are suspect as model did not exhibit a significantly (p>0.05)
  negative slope.
Warning: Estimates are suspect as model did not exhibit a significantly (p>0.05)
  negative slope.
\end{verbatim}

\begin{Shaded}
\begin{Highlighting}[]
\CommentTok{\# Вывод результатов}
\FunctionTok{print}\NormalTok{(}\StringTok{"Таблица параметров модели Лесли по годам:"}\NormalTok{)}
\end{Highlighting}
\end{Shaded}

\begin{verbatim}
[1] "Таблица параметров модели Лесли по годам:"
\end{verbatim}

\begin{Shaded}
\begin{Highlighting}[]
\FunctionTok{print}\NormalTok{(leslie\_all\_years)}
\end{Highlighting}
\end{Shaded}

\begin{verbatim}
   Модель  Год       B0    B0_LCI    B0_UCI        q     q_LCI    q_UCI     R2
1   Лесли 2007 31498.37   2705.68  60291.05 0.002323 -0.000097 0.004743 0.2672
2   Лесли 2008 14390.23  10412.29  18368.16 0.004777  0.003005 0.006549 0.7049
3   Лесли 2009 11715.54   7860.93  15570.16 0.004168  0.002409 0.005927 0.6684
4   Лесли 2010  5817.66   4369.39   7265.92 0.010571  0.006406 0.014736 0.6791
5   Лесли 2011  7968.21   4793.42  11143.00 0.011168  0.005162 0.017174 0.7342
6   Лесли 2012 18913.61   -844.81  38672.03 0.006095 -0.001239 0.013430 0.4080
7   Лесли 2013 27185.11  -3514.76  57884.98 0.006883 -0.001716 0.015481 0.3900
8   Лесли 2014 37864.24 -41069.72 116798.19 0.006719 -0.008193 0.021630 0.2812
9   Лесли 2015 26234.94  11351.15  41118.72 0.009624  0.003571 0.015677 0.7161
10  Лесли 2016 22168.22   8133.31  36203.14 0.009094  0.002560 0.015628 0.7191
11  Лесли 2017 15614.16  12692.12  18536.19 0.012778  0.009719 0.015838 0.9330
12  Лесли 2018 19825.00  14882.60  24767.40 0.012953  0.008866 0.017039 0.8330
\end{verbatim}

\begin{Shaded}
\begin{Highlighting}[]
\FunctionTok{print}\NormalTok{(}\StringTok{"Таблица параметров модели Делури по годам:"}\NormalTok{)}
\end{Highlighting}
\end{Shaded}

\begin{verbatim}
[1] "Таблица параметров модели Делури по годам:"
\end{verbatim}

\begin{Shaded}
\begin{Highlighting}[]
\FunctionTok{print}\NormalTok{(delury\_all\_years)}
\end{Highlighting}
\end{Shaded}

\begin{verbatim}
   Модель  Год       B0    B0_LCI    B0_UCI        q     q_LCI    q_UCI     R2
1  Делури 2007 28236.12   3911.96  52560.28 0.002584 -0.000016 0.005184 0.2810
2  Делури 2008 14297.12  11422.52  17171.73 0.004778  0.003382 0.006174 0.7938
3  Делури 2009 12725.30   8827.49  16623.12 0.003720  0.002219 0.005220 0.6881
4  Делури 2010  4393.17   3603.17   5183.16 0.016319  0.010545 0.022094 0.7241
5  Делури 2011  6583.20   4041.91   9124.50 0.014016  0.005647 0.022386 0.6914
6  Делури 2012 18610.09  -3184.94  40405.13 0.006133 -0.002249 0.014516 0.3482
7  Делури 2013 29170.84  -5308.38  63650.06 0.006347 -0.001969 0.014662 0.3676
8  Делури 2014 38177.41 -41839.92 118194.74 0.006637 -0.008199 0.021473 0.2783
9  Делури 2015 25898.41  11522.67  40274.15 0.009716  0.003691 0.015740 0.7218
10 Делури 2016 23437.65   6629.31  40245.99 0.008496  0.001564 0.015428 0.6650
11 Делури 2017 15112.98  12490.27  17735.70 0.013178  0.010103 0.016254 0.9362
12 Делури 2018 17743.61  13271.76  22215.47 0.014745  0.009658 0.019833 0.8066
\end{verbatim}

\begin{Shaded}
\begin{Highlighting}[]
\CommentTok{\# Сохранение результатов}
\FunctionTok{write.csv}\NormalTok{(leslie\_all\_years, }\StringTok{"Leslie\_parameters\_all\_years.csv"}\NormalTok{, }\AttributeTok{row.names =} \ConstantTok{FALSE}\NormalTok{)}
\FunctionTok{write.csv}\NormalTok{(delury\_all\_years, }\StringTok{"Delury\_parameters\_all\_years.csv"}\NormalTok{, }\AttributeTok{row.names =} \ConstantTok{FALSE}\NormalTok{)}

\CommentTok{\# ========================================================================================================================}
\CommentTok{\# ИНТЕРПРЕТАЦИЯ РЕЗУЛЬТАТОВ:}
\CommentTok{\# 1. Модель Лесли: CPUE = a + b * (кумулятивное усилие)}
\CommentTok{\# 2. Модель Делури: ln(CPUE) = a + b * (кумулятивное усилие)}
\CommentTok{\# Параметры:}
\CommentTok{\# {-} B0 (No): начальная биомасса/численность}
\CommentTok{\# {-} q: коэффициент уловистости}
\CommentTok{\# {-} R2: показатель качества модели (0{-}1)}
\CommentTok{\# Доверительные интервалы показывают точность оценок}
\CommentTok{\# ========================================================================================================================}
\end{Highlighting}
\end{Shaded}

\section{Результаты применения моделей
истощения}\label{ux440ux435ux437ux443ux43bux44cux442ux430ux442ux44b-ux43fux440ux438ux43cux435ux43dux435ux43dux438ux44f-ux43cux43eux434ux435ux43bux435ux439-ux438ux441ux442ux43eux449ux435ux43dux438ux44f}

Анализ данных промысла камчатского краба в Баренцевом море с 2007 по
2018 год выявил крайне неоднородную эффективность моделей истощения. В
течение 2008-2011 и 2017-2018 годов обе модели демонстрировали
статистически значимые результаты с высокими коэффициентами детерминации
(R² \textgreater{} 0.65), что указывало на четко выраженный эффект
истощения запаса в эти периоды. Наиболее надежные оценки начальной
биомассы (\emph{B\textsubscript{0}}) были получены для 2008 года: модель
Лесли показала 14.4 тыс. тонн, а модель Делури --- 14.3 тыс. тонн с
относительно узкими доверительными интервалами.

Однако в другие годы (2012-2014) результаты оказались статистически
несостоятельными --- модели не выявили значимого отрицательного наклона,
а доверительные интервалы включали биологически нереалистичные значения,
включая отрицательную биомассу. Особенно показательными был 2014 год,
где верхняя граница доверительного интервала превышала нижнюю более чем
на 150 тыс. тонн, что свидетельствует о крайне ненадежных оценках.

Анализ коэффициента уловистости (\emph{q}) выявил интересную динамику:
значения демонстрировали выраженную изменчивость от 0.0023 в 2007 году
до 0.0163 в 2010 году по модели Делури. Начиная с 2011 года значения
\emph{q} стабилизировались на более высоком уровне (0.009-0.015), что
может указывать на изменение эффективности промысла или поведенческих
характеристик краба.

Сравнительный анализ двух моделей показал их принципиальную
согласованность: расхождения в оценках \emph{B\textsubscript{0}} редко
превышали 10-15\%, что является приемлемым для данного типа моделей.
Однако модель ДеЛури в большинстве случаев демонстрировала несколько
более высокую точность оценок, выражающуюся в немного более узких
доверительных интервалах.

Ключевым выводом исследования стало подтверждение ограниченной
применимости моделей истощения --- они работают адекватно только в
условиях контролируемого промысла с изолированной популяцией. После 2018
года, с появлением новых промысловых игроков, модели потеряли
практическую ценность, поскольку нарушилось основное предположение о
закрытости системы и стабильности технологии добычи. Таким образом,
несмотря на теоретическую элегантность, модели Лесли и Делури остаются
нишевым инструментом, применимым лишь в редких условиях идеального
промыслового сценария.

\bookmarksetup{startatroot}

\chapter{II. DLM: CATCH-ONLY METHODS
(COM)}\label{ii.-dlm-catch-only-methods-com}

\section{Введение}\label{ux432ux432ux435ux434ux435ux43dux438ux435-14}

«В условиях неопределенности лучше быть приблизительно правым, чем точно
ошибаться» --- Джон Мейнард Кейнс

Настоящее занятие посвящено методам оценки запасов водных биоресурсов
при ограниченных данных Data-Limited Methods (DLM) подраздел Catch-only
(есть только уловы) --- инструментам, которые позволяют принимать
управленческие решения даже тогда, когда почти вся информация о
популяции недоступна. Если предыдущие занятия фокусировались на анализе
полных данных съемок и промысла, то здесь мы обращаемся к реальности
многих регионов и видов: у нас есть лишь временные ряды уловов,
отрывочные сведения о биологии и, возможно, экспертные оценки.
DLM-методы --- это не «упрощенные версии» сложных моделей, а
самостоятельный класс подходов, основанных на робастных принципах,
предосторожности и явных допущениях.

Как отмечал Нассим Талеб, уязвимость системы часто скрывается в её
зависимости от точных данных там, где возможна лишь устойчивая оценка.
DLM-методы предлагают философию «антихрупкости»: они признают
неопределенность,оценивают её и встраивают в процесс принятия решений. В
этом их ключевое отличие от «полноформатных» моделей вроде SS3 или SAM,
которые требуют детальных данных, но могут давать иллюзию точности там,
где её нет.

Познакомимся с классификации DLM-методов по системе Tier* --- иерархии,
основанной на объеме доступных данных. На нижней ступени (Tier 0)
находятся методы, опирающиеся лишь на аналогии и экспертные оценки; на
верхней (Tier 4) --- комбинированные подходы, интегрирующие данные
уловов, индексы численности и размерную структуру. Посередине ---
ключевые для практики методы, которые мы разберем детально: Catch-MSY
(CMSY), Depletion-Based SRA (DB-SRA), DCAC и другие. Каждый из них
решает конкретный вопрос: как оценить MSY и текущее состояние запаса,
если у нас есть только история уловов? Как учесть истощение? Как связать
темп роста популяции с её устойчивостью?

Практическая часть занятия построена вокруг скрипта, который шаг за
шагом проводит анализ по данным демонстрационного запаса. Вы увидите,
как:

\begin{itemize}
\tightlist
\item
  подготовить данные уловов для DLM-анализа;
\item
  задать априорные параметры на основе биологических характеристик вида;
\item
  запустить и интерпретировать результаты методов CMSY, DB-SRA, DCAC;
\item
  визуализировать траектории \emph{B/B\textsubscript{msy}} и
  \emph{F/F\textsubscript{msy}} в фазовой плоскости Кобе;
\item
  сравнить рекомендации по OДУ от разных методов и выработать
  консенсусную оценку.
\end{itemize}

Особое внимание уделим взгляду на допущения: почему выбор априорных
диапазонов для \emph{r} (внутреннего темпа роста) так важен? Как тренды
в уловах влияют на оценку истощения? Что делать, если методы дают
противоречивые результаты? Мы будем использовать байесовский подход там,
где он уместен (CMSY), и частотный там, где он прозрачнее (DB-SRA),
всегда оговаривая ограничения.

В результате вы получите не просто набор кода, а framework для работы с
данными в условиях их недостатка: от первичной диагностики временного
ряда уловов до принятия решений о допустимом изъятии. Эти навыки
критически важны для работы с новыми объектами промысла, восстановлением
запасов или в регионах с ограниченным мониторингом.

Как и в предыдущих занятиях, мы будем сочетать статистическую строгость
с биологической интерпретацией. Помните: DLM-методы --- это не
«костыли», а «ходунки»: они позволяют сделать первые шаги к устойчивому
управлению даже там, где данных мало, но решения принимать необходимо.

Файл скрипта находиться
\href{https://mombus.github.io/cRab/data/DLM_COM.R}{здесь}.

* \_Система Tier (уровней данных):

Tier 0: Нет количественных данных → Экспертные оценки, аналогии с
другими запасами

Tier 1: Только уловы (Catch-Only) → CMSY, OCOM, DB-SRA, SSS

Tier 2: Уловы + индекс биомассы (CPUE/Survey) → DCAC, SPiCT, Простые
продукционные модели

Tier 3: Уловы + размерная/возрастная структура → LBB, LBSPR, LIME,
Mean-length методы

Tier 4: Комбинация данных (уловы + индексы + структура) → SS-CL, LIME с
индексом, a4a

Tier 5: Полные данные → Stock Synthesis, VPA, SAM (не DLM)

\begin{Shaded}
\begin{Highlighting}[]
\CommentTok{\# ===============================================================}
\CommentTok{\#     ЗАНЯТИЕ 3: CATCH{-}ONLY METHODS (COM)}
\CommentTok{\#     Методы оценки запаса только по данным уловов}
\CommentTok{\#     Курс: Оценка водных биоресурсов при недостатке данных в R}
\CommentTok{\#     Обновлено: использование DLMtool вместо datalimited2}
\CommentTok{\# ===============================================================}

\CommentTok{\# ======================= ПОДГОТОВКА ==========================}

\CommentTok{\# Очистка рабочей среды}
\FunctionTok{rm}\NormalTok{(}\AttributeTok{list =} \FunctionTok{ls}\NormalTok{())}

\CommentTok{\# Установка и загрузка пакетов}
\CommentTok{\# Функция для установки если отсутствует}
\NormalTok{install\_if\_missing }\OtherTok{\textless{}{-}} \ControlFlowTok{function}\NormalTok{(pkg) \{}
  \ControlFlowTok{if}\NormalTok{ (}\SpecialCharTok{!}\FunctionTok{require}\NormalTok{(pkg, }\AttributeTok{character.only =} \ConstantTok{TRUE}\NormalTok{)) \{}
    \FunctionTok{install.packages}\NormalTok{(pkg)}
    \FunctionTok{library}\NormalTok{(pkg, }\AttributeTok{character.only =} \ConstantTok{TRUE}\NormalTok{)}
\NormalTok{  \}}
\NormalTok{\}}

\CommentTok{\# Установка основных пакетов}
\FunctionTok{cat}\NormalTok{(}\StringTok{"}\SpecialCharTok{\textbackslash{}n}\StringTok{========== УСТАНОВКА И ЗАГРУЗКА ПАКЕТОВ ==========}\SpecialCharTok{\textbackslash{}n}\StringTok{"}\NormalTok{)}

\CommentTok{\# DLMtool {-} основной пакет для DLM}
\CommentTok{\# install\_if\_missing("DLMtool")  \# Раскомментируйте при необходимости}

\CommentTok{\# Для CMSY метода устанавливаем отдельно}
\CommentTok{\# install.packages("remotes")}
\CommentTok{\# remotes::install\_github("SISTA16/cmsy")  \# Раскомментируйте для установки}

\CommentTok{\# Загрузка библиотек}
\CommentTok{\# Загрузка библиотек}
\FunctionTok{library}\NormalTok{(DLMtool)       }\CommentTok{\# Основной пакет для DLM методов}
\FunctionTok{library}\NormalTok{(ggplot2)       }\CommentTok{\# Визуализация}
\FunctionTok{library}\NormalTok{(tidyverse)     }\CommentTok{\# Обработка данных}
\FunctionTok{library}\NormalTok{(gridExtra)     }\CommentTok{\# Компоновка графиков}
\FunctionTok{library}\NormalTok{(viridis)       }\CommentTok{\# Цветовые схемы}

\CommentTok{\# Установка seed для воспроизводимости}
\FunctionTok{set.seed}\NormalTok{(}\DecValTok{42}\NormalTok{)}

\CommentTok{\# Отключение предупреждений DLMtool (опционально)}
\FunctionTok{options}\NormalTok{(}\AttributeTok{DLMtool.silent =} \ConstantTok{TRUE}\NormalTok{)}

\CommentTok{\# ======================= ИСХОДНЫЕ ДАННЫЕ =======================}

\FunctionTok{cat}\NormalTok{(}\StringTok{"}\SpecialCharTok{\textbackslash{}n}\StringTok{========== ИСХОДНЫЕ ДАННЫЕ ==========}\SpecialCharTok{\textbackslash{}n}\StringTok{"}\NormalTok{)}

\CommentTok{\# Вектор лет наблюдений}
\NormalTok{Year }\OtherTok{\textless{}{-}} \DecValTok{2005}\SpecialCharTok{:}\DecValTok{2024}
\NormalTok{nyears }\OtherTok{\textless{}{-}} \FunctionTok{length}\NormalTok{(Year)}

\CommentTok{\# Данные по вылову (тыс. тонн)}
\NormalTok{Catch }\OtherTok{\textless{}{-}} \FunctionTok{c}\NormalTok{(}\DecValTok{5}\NormalTok{, }\DecValTok{7}\NormalTok{, }\DecValTok{6}\NormalTok{, }\DecValTok{10}\NormalTok{, }\DecValTok{14}\NormalTok{, }\DecValTok{25}\NormalTok{, }\DecValTok{28}\NormalTok{, }\DecValTok{30}\NormalTok{, }\DecValTok{32}\NormalTok{, }\DecValTok{35}\NormalTok{, }
          \DecValTok{25}\NormalTok{, }\DecValTok{20}\NormalTok{, }\DecValTok{15}\NormalTok{, }\DecValTok{12}\NormalTok{, }\DecValTok{10}\NormalTok{, }\DecValTok{12}\NormalTok{, }\DecValTok{10}\NormalTok{, }\DecValTok{13}\NormalTok{, }\DecValTok{11}\NormalTok{, }\DecValTok{12}\NormalTok{)}

\CommentTok{\# Создание датафрейма для удобства}
\NormalTok{catch\_df }\OtherTok{\textless{}{-}} \FunctionTok{data.frame}\NormalTok{(}
  \AttributeTok{Year =}\NormalTok{ Year,}
  \AttributeTok{Catch =}\NormalTok{ Catch}
\NormalTok{)}

\CommentTok{\# Базовая статистика}
\FunctionTok{cat}\NormalTok{(}\StringTok{"}\SpecialCharTok{\textbackslash{}n}\StringTok{Основная статистика уловов:}\SpecialCharTok{\textbackslash{}n}\StringTok{"}\NormalTok{)}
\FunctionTok{cat}\NormalTok{(}\FunctionTok{sprintf}\NormalTok{(}\StringTok{"Период: \%d {-} \%d (\%d лет)}\SpecialCharTok{\textbackslash{}n}\StringTok{"}\NormalTok{, }\FunctionTok{min}\NormalTok{(Year), }\FunctionTok{max}\NormalTok{(Year), nyears))}
\FunctionTok{cat}\NormalTok{(}\FunctionTok{sprintf}\NormalTok{(}\StringTok{"Средний улов: \%.1f тыс. т}\SpecialCharTok{\textbackslash{}n}\StringTok{"}\NormalTok{, }\FunctionTok{mean}\NormalTok{(Catch)))}
\FunctionTok{cat}\NormalTok{(}\FunctionTok{sprintf}\NormalTok{(}\StringTok{"Максимальный улов: \%.1f тыс. т (\%d год)}\SpecialCharTok{\textbackslash{}n}\StringTok{"}\NormalTok{, }
            \FunctionTok{max}\NormalTok{(Catch), Year[}\FunctionTok{which.max}\NormalTok{(Catch)]))}
\FunctionTok{cat}\NormalTok{(}\FunctionTok{sprintf}\NormalTok{(}\StringTok{"Минимальный улов: \%.1f тыс. т (\%d год)}\SpecialCharTok{\textbackslash{}n}\StringTok{"}\NormalTok{, }
            \FunctionTok{min}\NormalTok{(Catch), Year[}\FunctionTok{which.min}\NormalTok{(Catch)]))}
\FunctionTok{cat}\NormalTok{(}\FunctionTok{sprintf}\NormalTok{(}\StringTok{"Коэффициент вариации: \%.2f}\SpecialCharTok{\textbackslash{}n}\StringTok{"}\NormalTok{, }\FunctionTok{sd}\NormalTok{(Catch)}\SpecialCharTok{/}\FunctionTok{mean}\NormalTok{(Catch)))}

\CommentTok{\# Анализ тренда}
\NormalTok{recent\_trend }\OtherTok{\textless{}{-}} \FunctionTok{mean}\NormalTok{(}\FunctionTok{tail}\NormalTok{(Catch, }\DecValTok{5}\NormalTok{)) }\SpecialCharTok{/} \FunctionTok{mean}\NormalTok{(}\FunctionTok{head}\NormalTok{(Catch, }\DecValTok{5}\NormalTok{))}
\FunctionTok{cat}\NormalTok{(}\FunctionTok{sprintf}\NormalTok{(}\StringTok{"Изменение за период: \%.0f\%\%}\SpecialCharTok{\textbackslash{}n}\StringTok{"}\NormalTok{, (recent\_trend }\SpecialCharTok{{-}} \DecValTok{1}\NormalTok{) }\SpecialCharTok{*} \DecValTok{100}\NormalTok{))}

\CommentTok{\# ======================= ВИЗУАЛИЗАЦИЯ ИСХОДНЫХ ДАННЫХ =======================}

\CommentTok{\# График временного ряда с анализом}
\NormalTok{p1 }\OtherTok{\textless{}{-}} \FunctionTok{ggplot}\NormalTok{(catch\_df, }\FunctionTok{aes}\NormalTok{(}\AttributeTok{x =}\NormalTok{ Year, }\AttributeTok{y =}\NormalTok{ Catch)) }\SpecialCharTok{+}
  \CommentTok{\# Основные данные}
  \FunctionTok{geom\_line}\NormalTok{(}\AttributeTok{linewidth =} \FloatTok{1.2}\NormalTok{, }\AttributeTok{color =} \StringTok{"darkblue"}\NormalTok{) }\SpecialCharTok{+}
  \FunctionTok{geom\_point}\NormalTok{(}\AttributeTok{size =} \DecValTok{3}\NormalTok{, }\AttributeTok{color =} \StringTok{"darkblue"}\NormalTok{) }\SpecialCharTok{+}
  
  \CommentTok{\# Скользящее среднее (3 года)}
  \FunctionTok{geom\_smooth}\NormalTok{(}\AttributeTok{method =} \StringTok{"loess"}\NormalTok{, }\AttributeTok{span =} \FloatTok{0.3}\NormalTok{, }\AttributeTok{se =} \ConstantTok{TRUE}\NormalTok{, }
              \AttributeTok{alpha =} \FloatTok{0.2}\NormalTok{, }\AttributeTok{linewidth =} \DecValTok{1}\NormalTok{, }\AttributeTok{color =} \StringTok{"red"}\NormalTok{) }\SpecialCharTok{+}
  
  \CommentTok{\# Средний уровень}
  \FunctionTok{geom\_hline}\NormalTok{(}\AttributeTok{yintercept =} \FunctionTok{mean}\NormalTok{(Catch), }
             \AttributeTok{linetype =} \StringTok{"dashed"}\NormalTok{, }\AttributeTok{color =} \StringTok{"gray50"}\NormalTok{) }\SpecialCharTok{+}
  
  \CommentTok{\# Аннотации}
  \FunctionTok{annotate}\NormalTok{(}\StringTok{"text"}\NormalTok{, }\AttributeTok{x =} \FunctionTok{min}\NormalTok{(Year), }\AttributeTok{y =} \FunctionTok{mean}\NormalTok{(Catch), }
           \AttributeTok{label =} \StringTok{"Средний улов"}\NormalTok{, }\AttributeTok{vjust =} \SpecialCharTok{{-}}\FloatTok{0.5}\NormalTok{, }\AttributeTok{hjust =} \DecValTok{0}\NormalTok{, }\AttributeTok{size =} \DecValTok{3}\NormalTok{) }\SpecialCharTok{+}
  
  \CommentTok{\# Оформление}
  \FunctionTok{labs}\NormalTok{(}\AttributeTok{title =} \StringTok{"Временной ряд уловов"}\NormalTok{,}
       \AttributeTok{subtitle =} \StringTok{"Исходные данные для catch{-}only анализа"}\NormalTok{,}
       \AttributeTok{x =} \StringTok{"Год"}\NormalTok{, }\AttributeTok{y =} \StringTok{"Улов (тыс. т)"}\NormalTok{) }\SpecialCharTok{+}
  \FunctionTok{theme\_minimal}\NormalTok{() }\SpecialCharTok{+}
  \FunctionTok{theme}\NormalTok{(}\AttributeTok{plot.title =} \FunctionTok{element\_text}\NormalTok{(}\AttributeTok{size =} \DecValTok{14}\NormalTok{, }\AttributeTok{face =} \StringTok{"bold"}\NormalTok{))}

\FunctionTok{print}\NormalTok{(p1)}
\end{Highlighting}
\end{Shaded}

\begin{center}
\includegraphics[width=0.8\linewidth,height=\textheight,keepaspectratio]{images/DLMII1.PNG}
\end{center}

\begin{Shaded}
\begin{Highlighting}[]
\CommentTok{\# ==================}
\CommentTok{\# ======================= СОЗДАНИЕ ОБЪЕКТОВ DLMtool =======================}

\FunctionTok{cat}\NormalTok{(}\StringTok{"}\SpecialCharTok{\textbackslash{}n}\StringTok{========== СОЗДАНИЕ ОБЪЕКТОВ DLMtool ==========}\SpecialCharTok{\textbackslash{}n}\StringTok{"}\NormalTok{)}

\CommentTok{\# DLMtool требует специальную структуру данных}
\CommentTok{\# Создаем объект Data\_object с минимальной информацией}

\DocumentationTok{\#\# 1. Создание объекта Data}
\NormalTok{CatchOnly\_Data }\OtherTok{\textless{}{-}} \FunctionTok{new}\NormalTok{(}\StringTok{"Data"}\NormalTok{)}

\CommentTok{\# Основные параметры}
\NormalTok{CatchOnly\_Data}\SpecialCharTok{@}\NormalTok{Name }\OtherTok{\textless{}{-}} \StringTok{"Demo Stock"}
\NormalTok{CatchOnly\_Data}\SpecialCharTok{@}\NormalTok{Common\_Name }\OtherTok{\textless{}{-}} \StringTok{"Демерсальная рыба"}
\NormalTok{CatchOnly\_Data}\SpecialCharTok{@}\NormalTok{Year }\OtherTok{\textless{}{-}} \FunctionTok{as.numeric}\NormalTok{(Year)  }\CommentTok{\# Важно: числовой вектор}
\NormalTok{CatchOnly\_Data}\SpecialCharTok{@}\NormalTok{Cat }\OtherTok{\textless{}{-}} \FunctionTok{matrix}\NormalTok{(Catch, }\AttributeTok{nrow =} \DecValTok{1}\NormalTok{)  }\CommentTok{\# Матрица уловов}
\NormalTok{CatchOnly\_Data}\SpecialCharTok{@}\NormalTok{Units }\OtherTok{\textless{}{-}} \StringTok{"тыс. тонн"}
\NormalTok{CatchOnly\_Data}\SpecialCharTok{@}\NormalTok{nareas }\OtherTok{\textless{}{-}} \DecValTok{1}  \CommentTok{\# Количество районов}

\CommentTok{\# Добавляем минимальную биологическую информацию}
\CommentTok{\# Эти параметры типичны для демерсальной рыбы средней продолжительности жизни}
\NormalTok{CatchOnly\_Data}\SpecialCharTok{@}\NormalTok{Mort }\OtherTok{\textless{}{-}} \FloatTok{0.2}         \CommentTok{\# Естественная смертность (M)}
\NormalTok{CatchOnly\_Data}\SpecialCharTok{@}\NormalTok{CV\_Mort }\OtherTok{\textless{}{-}} \FloatTok{0.2}      \CommentTok{\# CV для M}
\NormalTok{CatchOnly\_Data}\SpecialCharTok{@}\NormalTok{vbK }\OtherTok{\textless{}{-}} \FloatTok{0.15}         \CommentTok{\# Параметр роста фон Берталанфи K}
\NormalTok{CatchOnly\_Data}\SpecialCharTok{@}\NormalTok{CV\_vbK }\OtherTok{\textless{}{-}} \FloatTok{0.2}       \CommentTok{\# CV для K}
\NormalTok{CatchOnly\_Data}\SpecialCharTok{@}\NormalTok{vbLinf }\OtherTok{\textless{}{-}} \DecValTok{100}       \CommentTok{\# Асимптотическая длина (см)}
\NormalTok{CatchOnly\_Data}\SpecialCharTok{@}\NormalTok{CV\_vbLinf }\OtherTok{\textless{}{-}} \FloatTok{0.1}    \CommentTok{\# CV для Linf}
\NormalTok{CatchOnly\_Data}\SpecialCharTok{@}\NormalTok{vbt0 }\OtherTok{\textless{}{-}} \SpecialCharTok{{-}}\FloatTok{0.5}        \CommentTok{\# t0 в уравнении роста}
\NormalTok{CatchOnly\_Data}\SpecialCharTok{@}\NormalTok{CV\_vbt0 }\OtherTok{\textless{}{-}} \FloatTok{0.2}      \CommentTok{\# CV для t0}
\NormalTok{CatchOnly\_Data}\SpecialCharTok{@}\NormalTok{wla }\OtherTok{\textless{}{-}} \FloatTok{0.00001}      \CommentTok{\# Параметр a в соотношении длина{-}вес}
\NormalTok{CatchOnly\_Data}\SpecialCharTok{@}\NormalTok{wlb }\OtherTok{\textless{}{-}} \FloatTok{3.0}          \CommentTok{\# Параметр b в соотношении длина{-}вес}
\NormalTok{CatchOnly\_Data}\SpecialCharTok{@}\NormalTok{MaxAge }\OtherTok{\textless{}{-}} \DecValTok{20}        \CommentTok{\# Максимальный возраст}
\NormalTok{CatchOnly\_Data}\SpecialCharTok{@}\NormalTok{BMSY\_B0 }\OtherTok{\textless{}{-}} \FloatTok{0.5}      \CommentTok{\# Отношение BMSY/B0}

\CommentTok{\# Априорная информация о состоянии запаса}
\NormalTok{CatchOnly\_Data}\SpecialCharTok{@}\NormalTok{Dep }\OtherTok{\textless{}{-}} \FloatTok{0.3}          \CommentTok{\# Текущее истощение (B/B0) {-} экспертная оценка}
\NormalTok{CatchOnly\_Data}\SpecialCharTok{@}\NormalTok{CV\_Dep }\OtherTok{\textless{}{-}} \FloatTok{0.5}       \CommentTok{\# Высокая неопределенность}

\CommentTok{\# Информация о промысле}
\NormalTok{CatchOnly\_Data}\SpecialCharTok{@}\NormalTok{AvC }\OtherTok{\textless{}{-}} \FunctionTok{mean}\NormalTok{(Catch)  }\CommentTok{\# Средний исторический улов}
\NormalTok{CatchOnly\_Data}\SpecialCharTok{@}\NormalTok{CV\_Cat }\OtherTok{\textless{}{-}} \FunctionTok{matrix}\NormalTok{(}\FloatTok{0.1}\NormalTok{, }\AttributeTok{nrow =} \DecValTok{1}\NormalTok{, }\AttributeTok{ncol =} \FunctionTok{length}\NormalTok{(Catch))  }\CommentTok{\# CV уловов}

\CommentTok{\# Важные дополнительные параметры для DCAC}
\NormalTok{CatchOnly\_Data}\SpecialCharTok{@}\NormalTok{LHYear }\OtherTok{\textless{}{-}} \FunctionTok{max}\NormalTok{(Year)  }\CommentTok{\# Год, к которому относятся биологические параметры}
\NormalTok{CatchOnly\_Data}\SpecialCharTok{@}\NormalTok{FMSY\_M }\OtherTok{\textless{}{-}} \FloatTok{0.8}        \CommentTok{\# Отношение FMSY/M (типичное значение)}
\NormalTok{CatchOnly\_Data}\SpecialCharTok{@}\NormalTok{CV\_FMSY\_M }\OtherTok{\textless{}{-}} \FloatTok{0.2}     \CommentTok{\# CV для FMSY/M}

\FunctionTok{cat}\NormalTok{(}\StringTok{"✓ Объект Data создан успешно}\SpecialCharTok{\textbackslash{}n}\StringTok{"}\NormalTok{)}
\FunctionTok{cat}\NormalTok{(}\FunctionTok{sprintf}\NormalTok{(}\StringTok{"  Название запаса: \%s}\SpecialCharTok{\textbackslash{}n}\StringTok{"}\NormalTok{, CatchOnly\_Data}\SpecialCharTok{@}\NormalTok{Name))}
\FunctionTok{cat}\NormalTok{(}\FunctionTok{sprintf}\NormalTok{(}\StringTok{"  Период данных: \%d {-} \%d}\SpecialCharTok{\textbackslash{}n}\StringTok{"}\NormalTok{, }\FunctionTok{min}\NormalTok{(Year), }\FunctionTok{max}\NormalTok{(Year)))}
\FunctionTok{cat}\NormalTok{(}\FunctionTok{sprintf}\NormalTok{(}\StringTok{"  Средний улов: \%.1f \%s}\SpecialCharTok{\textbackslash{}n}\StringTok{"}\NormalTok{, CatchOnly\_Data}\SpecialCharTok{@}\NormalTok{AvC, CatchOnly\_Data}\SpecialCharTok{@}\NormalTok{Units))}

\CommentTok{\# ======================= МЕТОД 1: DCAC (Depletion{-}Corrected Average Catch) =======================}

\FunctionTok{cat}\NormalTok{(}\StringTok{"}\SpecialCharTok{\textbackslash{}n}\StringTok{========== МЕТОД DCAC ==========}\SpecialCharTok{\textbackslash{}n}\StringTok{"}\NormalTok{)}
\FunctionTok{cat}\NormalTok{(}\StringTok{"Depletion{-}Corrected Average Catch}\SpecialCharTok{\textbackslash{}n}\StringTok{"}\NormalTok{)}
\FunctionTok{cat}\NormalTok{(}\StringTok{"Простейший catch{-}only метод с коррекцией на истощение}\SpecialCharTok{\textbackslash{}n\textbackslash{}n}\StringTok{"}\NormalTok{)}

\CommentTok{\# DCAC {-} встроенный метод в DLMtool}
\NormalTok{DCAC\_results }\OtherTok{\textless{}{-}} \FunctionTok{DCAC}\NormalTok{(}
  \AttributeTok{x =} \DecValTok{1}\NormalTok{,                          }\CommentTok{\# Индекс симуляции}
  \AttributeTok{Data =}\NormalTok{ CatchOnly\_Data,          }\CommentTok{\# Объект данных}
  \AttributeTok{reps =} \DecValTok{1000}                     \CommentTok{\# Количество репликаций}
\NormalTok{)}


\CommentTok{\# Детальный расчет DCAC для понимания}
\NormalTok{DCAC\_manual }\OtherTok{\textless{}{-}} \ControlFlowTok{function}\NormalTok{(catches, depletion, }\AttributeTok{M =} \FloatTok{0.2}\NormalTok{) \{}
  \CommentTok{\# DCAC = средний улов * коррекция на истощение}
\NormalTok{  avg\_catch }\OtherTok{\textless{}{-}} \FunctionTok{mean}\NormalTok{(catches)}
  
  \CommentTok{\# Коррекция зависит от истощения и M}
  \ControlFlowTok{if}\NormalTok{ (depletion }\SpecialCharTok{\textless{}} \FloatTok{0.5}\NormalTok{) \{}
\NormalTok{    correction }\OtherTok{\textless{}{-}}\NormalTok{ depletion }\SpecialCharTok{*}\NormalTok{ (}\DecValTok{1} \SpecialCharTok{+}\NormalTok{ M)}
\NormalTok{  \} }\ControlFlowTok{else}\NormalTok{ \{}
\NormalTok{    correction }\OtherTok{\textless{}{-}} \FloatTok{1.0}
\NormalTok{  \}}
  
\NormalTok{  dcac\_tac }\OtherTok{\textless{}{-}}\NormalTok{ avg\_catch }\SpecialCharTok{*}\NormalTok{ correction}
  
  \FunctionTok{return}\NormalTok{(}\FunctionTok{list}\NormalTok{(}
    \AttributeTok{avg\_catch =}\NormalTok{ avg\_catch,}
    \AttributeTok{depletion =}\NormalTok{ depletion,}
    \AttributeTok{correction =}\NormalTok{ correction,}
    \AttributeTok{tac =}\NormalTok{ dcac\_tac}
\NormalTok{  ))}
\NormalTok{\}}

\NormalTok{dcac\_manual\_result }\OtherTok{\textless{}{-}} \FunctionTok{DCAC\_manual}\NormalTok{(Catch, }\AttributeTok{depletion =} \FloatTok{0.3}\NormalTok{, }\AttributeTok{M =} \FloatTok{0.2}\NormalTok{)}

\FunctionTok{cat}\NormalTok{(}\StringTok{"}\SpecialCharTok{\textbackslash{}n}\StringTok{Ручной расчет DCAC:}\SpecialCharTok{\textbackslash{}n}\StringTok{"}\NormalTok{)}
\FunctionTok{cat}\NormalTok{(}\FunctionTok{sprintf}\NormalTok{(}\StringTok{"  Средний улов: \%.1f тыс. т}\SpecialCharTok{\textbackslash{}n}\StringTok{"}\NormalTok{, dcac\_manual\_result}\SpecialCharTok{$}\NormalTok{avg\_catch))}
\FunctionTok{cat}\NormalTok{(}\FunctionTok{sprintf}\NormalTok{(}\StringTok{"  Истощение: \%.0f\%\%}\SpecialCharTok{\textbackslash{}n}\StringTok{"}\NormalTok{, dcac\_manual\_result}\SpecialCharTok{$}\NormalTok{depletion }\SpecialCharTok{*} \DecValTok{100}\NormalTok{))}
\FunctionTok{cat}\NormalTok{(}\FunctionTok{sprintf}\NormalTok{(}\StringTok{"  Коэффициент коррекции: \%.2f}\SpecialCharTok{\textbackslash{}n}\StringTok{"}\NormalTok{, dcac\_manual\_result}\SpecialCharTok{$}\NormalTok{correction))}
\FunctionTok{cat}\NormalTok{(}\FunctionTok{sprintf}\NormalTok{(}\StringTok{"  TAC = \%.1f × \%.2f = \%.1f тыс. т}\SpecialCharTok{\textbackslash{}n}\StringTok{"}\NormalTok{, }
\NormalTok{            dcac\_manual\_result}\SpecialCharTok{$}\NormalTok{avg\_catch, }
\NormalTok{            dcac\_manual\_result}\SpecialCharTok{$}\NormalTok{correction, }
\NormalTok{            dcac\_manual\_result}\SpecialCharTok{$}\NormalTok{tac))}

\CommentTok{\# ======================= МЕТОД 2: DB{-}SRA (Depletion{-}Based Stock Reduction Analysis) =======================}

\FunctionTok{cat}\NormalTok{(}\StringTok{"}\SpecialCharTok{\textbackslash{}n}\StringTok{========== МЕТОД DB{-}SRA ==========}\SpecialCharTok{\textbackslash{}n}\StringTok{"}\NormalTok{)}
\FunctionTok{cat}\NormalTok{(}\StringTok{"Стохастический анализ сокращения запаса}\SpecialCharTok{\textbackslash{}n\textbackslash{}n}\StringTok{"}\NormalTok{)}

\CommentTok{\# Функция для DB{-}SRA}
\NormalTok{DBSRA }\OtherTok{\textless{}{-}} \ControlFlowTok{function}\NormalTok{(catch\_data, }\AttributeTok{depletion\_prior =} \FunctionTok{c}\NormalTok{(}\FloatTok{0.2}\NormalTok{, }\FloatTok{0.5}\NormalTok{), }
                  \AttributeTok{r\_prior =} \FunctionTok{c}\NormalTok{(}\FloatTok{0.1}\NormalTok{, }\FloatTok{0.6}\NormalTok{), }\AttributeTok{nsim =} \DecValTok{1000}\NormalTok{) \{}
  
\NormalTok{  nyears }\OtherTok{\textless{}{-}} \FunctionTok{length}\NormalTok{(catch\_data)}
  
  \CommentTok{\# Массивы для хранения результатов}
\NormalTok{  r\_vals }\OtherTok{\textless{}{-}} \FunctionTok{numeric}\NormalTok{(nsim)}
\NormalTok{  k\_vals }\OtherTok{\textless{}{-}} \FunctionTok{numeric}\NormalTok{(nsim)}
\NormalTok{  msy\_vals }\OtherTok{\textless{}{-}} \FunctionTok{numeric}\NormalTok{(nsim)}
\NormalTok{  b\_final }\OtherTok{\textless{}{-}} \FunctionTok{numeric}\NormalTok{(nsim)}
\NormalTok{  b\_bmsy\_final }\OtherTok{\textless{}{-}} \FunctionTok{numeric}\NormalTok{(nsim)}
  
  \CommentTok{\# Счетчик успешных симуляций}
\NormalTok{  success }\OtherTok{\textless{}{-}} \DecValTok{0}
  
  \FunctionTok{cat}\NormalTok{(}\StringTok{"Запуск DB{-}SRA симуляций...}\SpecialCharTok{\textbackslash{}n}\StringTok{"}\NormalTok{)}
\NormalTok{  pb }\OtherTok{\textless{}{-}} \FunctionTok{txtProgressBar}\NormalTok{(}\AttributeTok{min =} \DecValTok{0}\NormalTok{, }\AttributeTok{max =}\NormalTok{ nsim, }\AttributeTok{style =} \DecValTok{3}\NormalTok{)}
  
  \ControlFlowTok{for}\NormalTok{ (sim }\ControlFlowTok{in} \DecValTok{1}\SpecialCharTok{:}\NormalTok{nsim) \{}
    
    \CommentTok{\# Генерация случайных параметров из приоров}
\NormalTok{    r }\OtherTok{\textless{}{-}} \FunctionTok{runif}\NormalTok{(}\DecValTok{1}\NormalTok{, r\_prior[}\DecValTok{1}\NormalTok{], r\_prior[}\DecValTok{2}\NormalTok{])}
\NormalTok{    init\_depl }\OtherTok{\textless{}{-}} \FunctionTok{runif}\NormalTok{(}\DecValTok{1}\NormalTok{, depletion\_prior[}\DecValTok{1}\NormalTok{], depletion\_prior[}\DecValTok{2}\NormalTok{])}
    
    \CommentTok{\# Начальная биомасса как доля от K}
    \CommentTok{\# K оценивается из максимального улова}
\NormalTok{    k\_guess }\OtherTok{\textless{}{-}} \FunctionTok{max}\NormalTok{(catch\_data) }\SpecialCharTok{*} \FunctionTok{runif}\NormalTok{(}\DecValTok{1}\NormalTok{, }\DecValTok{4}\NormalTok{, }\DecValTok{12}\NormalTok{)}
    
    \CommentTok{\# Инициализация биомассы}
\NormalTok{    B }\OtherTok{\textless{}{-}} \FunctionTok{numeric}\NormalTok{(nyears }\SpecialCharTok{+} \DecValTok{1}\NormalTok{)}
\NormalTok{    B[}\DecValTok{1}\NormalTok{] }\OtherTok{\textless{}{-}}\NormalTok{ k\_guess }\SpecialCharTok{*}\NormalTok{ init\_depl}
    
    \CommentTok{\# Проекция популяции}
    \ControlFlowTok{for}\NormalTok{ (t }\ControlFlowTok{in} \DecValTok{1}\SpecialCharTok{:}\NormalTok{nyears) \{}
      \CommentTok{\# Продукционная модель Шефера}
\NormalTok{      surplus }\OtherTok{\textless{}{-}}\NormalTok{ r }\SpecialCharTok{*}\NormalTok{ B[t] }\SpecialCharTok{*}\NormalTok{ (}\DecValTok{1} \SpecialCharTok{{-}}\NormalTok{ B[t]}\SpecialCharTok{/}\NormalTok{k\_guess)}
      
      \CommentTok{\# Проверка, что улов не превышает доступную биомассу}
      \ControlFlowTok{if}\NormalTok{ (catch\_data[t] }\SpecialCharTok{\textgreater{}}\NormalTok{ (B[t] }\SpecialCharTok{+}\NormalTok{ surplus) }\SpecialCharTok{*} \FloatTok{0.95}\NormalTok{) \{}
        \CommentTok{\# Неудачная симуляция}
        \ControlFlowTok{break}
\NormalTok{      \}}
      
      \CommentTok{\# Обновление биомассы}
\NormalTok{      B[t}\SpecialCharTok{+}\DecValTok{1}\NormalTok{] }\OtherTok{\textless{}{-}}\NormalTok{ B[t] }\SpecialCharTok{+}\NormalTok{ surplus }\SpecialCharTok{{-}}\NormalTok{ catch\_data[t]}
      
      \CommentTok{\# Проверка на отрицательную биомассу}
      \ControlFlowTok{if}\NormalTok{ (B[t}\SpecialCharTok{+}\DecValTok{1}\NormalTok{] }\SpecialCharTok{\textless{}=} \DecValTok{0}\NormalTok{) \{}
        \ControlFlowTok{break}
\NormalTok{      \}}
\NormalTok{    \}}
    
    \CommentTok{\# Если симуляция успешна}
    \ControlFlowTok{if}\NormalTok{ (t }\SpecialCharTok{==}\NormalTok{ nyears }\SpecialCharTok{\&\&}\NormalTok{ B[nyears}\SpecialCharTok{+}\DecValTok{1}\NormalTok{] }\SpecialCharTok{\textgreater{}} \DecValTok{0}\NormalTok{) \{}
\NormalTok{      success }\OtherTok{\textless{}{-}}\NormalTok{ success }\SpecialCharTok{+} \DecValTok{1}
      
      \CommentTok{\# Сохранение результатов}
\NormalTok{      r\_vals[success] }\OtherTok{\textless{}{-}}\NormalTok{ r}
\NormalTok{      k\_vals[success] }\OtherTok{\textless{}{-}}\NormalTok{ k\_guess}
\NormalTok{      msy\_vals[success] }\OtherTok{\textless{}{-}}\NormalTok{ r }\SpecialCharTok{*}\NormalTok{ k\_guess }\SpecialCharTok{/} \DecValTok{4}  \CommentTok{\# MSY для модели Шефера}
\NormalTok{      b\_final[success] }\OtherTok{\textless{}{-}}\NormalTok{ B[nyears}\SpecialCharTok{+}\DecValTok{1}\NormalTok{]}
\NormalTok{      b\_bmsy\_final[success] }\OtherTok{\textless{}{-}}\NormalTok{ B[nyears}\SpecialCharTok{+}\DecValTok{1}\NormalTok{] }\SpecialCharTok{/}\NormalTok{ (k\_guess}\SpecialCharTok{/}\DecValTok{2}\NormalTok{)}
\NormalTok{    \}}
    
    \FunctionTok{setTxtProgressBar}\NormalTok{(pb, sim)}
\NormalTok{  \}}
  \FunctionTok{close}\NormalTok{(pb)}
  
  \CommentTok{\# Обрезаем массивы до количества успешных симуляций}
\NormalTok{  r\_vals }\OtherTok{\textless{}{-}}\NormalTok{ r\_vals[}\DecValTok{1}\SpecialCharTok{:}\NormalTok{success]}
\NormalTok{  k\_vals }\OtherTok{\textless{}{-}}\NormalTok{ k\_vals[}\DecValTok{1}\SpecialCharTok{:}\NormalTok{success]}
\NormalTok{  msy\_vals }\OtherTok{\textless{}{-}}\NormalTok{ msy\_vals[}\DecValTok{1}\SpecialCharTok{:}\NormalTok{success]}
\NormalTok{  b\_final }\OtherTok{\textless{}{-}}\NormalTok{ b\_final[}\DecValTok{1}\SpecialCharTok{:}\NormalTok{success]}
\NormalTok{  b\_bmsy\_final }\OtherTok{\textless{}{-}}\NormalTok{ b\_bmsy\_final[}\DecValTok{1}\SpecialCharTok{:}\NormalTok{success]}
  
  \FunctionTok{cat}\NormalTok{(}\FunctionTok{sprintf}\NormalTok{(}\StringTok{"}\SpecialCharTok{\textbackslash{}n}\StringTok{✓ Успешных симуляций: \%d из \%d (\%.1f\%\%)}\SpecialCharTok{\textbackslash{}n}\StringTok{"}\NormalTok{, }
\NormalTok{              success, nsim, success}\SpecialCharTok{/}\NormalTok{nsim}\SpecialCharTok{*}\DecValTok{100}\NormalTok{))}
  
  \FunctionTok{return}\NormalTok{(}\FunctionTok{list}\NormalTok{(}
    \AttributeTok{r =}\NormalTok{ r\_vals,}
    \AttributeTok{k =}\NormalTok{ k\_vals,}
    \AttributeTok{msy =}\NormalTok{ msy\_vals,}
    \AttributeTok{b\_final =}\NormalTok{ b\_final,}
    \AttributeTok{b\_bmsy =}\NormalTok{ b\_bmsy\_final,}
    \AttributeTok{n\_success =}\NormalTok{ success}
\NormalTok{  ))}
\NormalTok{\}}

\CommentTok{\# Запуск DB{-}SRA}
\NormalTok{dbsra\_results }\OtherTok{\textless{}{-}} \FunctionTok{DBSRA}\NormalTok{(}
  \AttributeTok{catch\_data =}\NormalTok{ Catch,}
  \AttributeTok{depletion\_prior =} \FunctionTok{c}\NormalTok{(}\FloatTok{0.2}\NormalTok{, }\FloatTok{0.5}\NormalTok{),}
  \AttributeTok{r\_prior =} \FunctionTok{c}\NormalTok{(}\FloatTok{0.1}\NormalTok{, }\FloatTok{0.6}\NormalTok{),}
  \AttributeTok{nsim =} \DecValTok{2000}
\NormalTok{)}

\CommentTok{\# Вывод результатов DB{-}SRA}
\FunctionTok{cat}\NormalTok{(}\StringTok{"}\SpecialCharTok{\textbackslash{}n}\StringTok{{-}{-}{-} Результаты DB{-}SRA {-}{-}{-}}\SpecialCharTok{\textbackslash{}n}\StringTok{"}\NormalTok{)}
\FunctionTok{cat}\NormalTok{(}\FunctionTok{sprintf}\NormalTok{(}\StringTok{"r (медиана): \%.3f [\%.3f {-} \%.3f]}\SpecialCharTok{\textbackslash{}n}\StringTok{"}\NormalTok{, }
            \FunctionTok{median}\NormalTok{(dbsra\_results}\SpecialCharTok{$}\NormalTok{r),}
            \FunctionTok{quantile}\NormalTok{(dbsra\_results}\SpecialCharTok{$}\NormalTok{r, }\FloatTok{0.25}\NormalTok{),}
            \FunctionTok{quantile}\NormalTok{(dbsra\_results}\SpecialCharTok{$}\NormalTok{r, }\FloatTok{0.75}\NormalTok{)))}
\FunctionTok{cat}\NormalTok{(}\FunctionTok{sprintf}\NormalTok{(}\StringTok{"K (медиана): \%.1f [\%.1f {-} \%.1f] тыс. т}\SpecialCharTok{\textbackslash{}n}\StringTok{"}\NormalTok{,}
            \FunctionTok{median}\NormalTok{(dbsra\_results}\SpecialCharTok{$}\NormalTok{k),}
            \FunctionTok{quantile}\NormalTok{(dbsra\_results}\SpecialCharTok{$}\NormalTok{k, }\FloatTok{0.25}\NormalTok{),}
            \FunctionTok{quantile}\NormalTok{(dbsra\_results}\SpecialCharTok{$}\NormalTok{k, }\FloatTok{0.75}\NormalTok{)))}
\FunctionTok{cat}\NormalTok{(}\FunctionTok{sprintf}\NormalTok{(}\StringTok{"MSY (медиана): \%.1f [\%.1f {-} \%.1f] тыс. т}\SpecialCharTok{\textbackslash{}n}\StringTok{"}\NormalTok{,}
            \FunctionTok{median}\NormalTok{(dbsra\_results}\SpecialCharTok{$}\NormalTok{msy),}
            \FunctionTok{quantile}\NormalTok{(dbsra\_results}\SpecialCharTok{$}\NormalTok{msy, }\FloatTok{0.25}\NormalTok{),}
            \FunctionTok{quantile}\NormalTok{(dbsra\_results}\SpecialCharTok{$}\NormalTok{msy, }\FloatTok{0.75}\NormalTok{)))}
\FunctionTok{cat}\NormalTok{(}\FunctionTok{sprintf}\NormalTok{(}\StringTok{"B/Bmsy текущее (медиана): \%.2f [\%.2f {-} \%.2f]}\SpecialCharTok{\textbackslash{}n}\StringTok{"}\NormalTok{,}
            \FunctionTok{median}\NormalTok{(dbsra\_results}\SpecialCharTok{$}\NormalTok{b\_bmsy),}
            \FunctionTok{quantile}\NormalTok{(dbsra\_results}\SpecialCharTok{$}\NormalTok{b\_bmsy, }\FloatTok{0.25}\NormalTok{),}
            \FunctionTok{quantile}\NormalTok{(dbsra\_results}\SpecialCharTok{$}\NormalTok{b\_bmsy, }\FloatTok{0.75}\NormalTok{)))}

\CommentTok{\# ======================= МЕТОД 3: CMSY (Catch{-}MSY) =======================}

\FunctionTok{cat}\NormalTok{(}\StringTok{"}\SpecialCharTok{\textbackslash{}n}\StringTok{========== МЕТОД CMSY ==========}\SpecialCharTok{\textbackslash{}n}\StringTok{"}\NormalTok{)}
\FunctionTok{cat}\NormalTok{(}\StringTok{"Catch{-}MSY метод (Froese et al. 2017)}\SpecialCharTok{\textbackslash{}n}\StringTok{"}\NormalTok{)}
\FunctionTok{cat}\NormalTok{(}\StringTok{"Байесовская оценка MSY из временного ряда уловов}\SpecialCharTok{\textbackslash{}n\textbackslash{}n}\StringTok{"}\NormalTok{)}

\CommentTok{\# Упрощенная реализация CMSY}
\NormalTok{CMSY\_simple }\OtherTok{\textless{}{-}} \ControlFlowTok{function}\NormalTok{(catch\_data, }\AttributeTok{resilience =} \StringTok{"Medium"}\NormalTok{, }\AttributeTok{nsim =} \DecValTok{10000}\NormalTok{) \{}
  
\NormalTok{  nyears }\OtherTok{\textless{}{-}} \FunctionTok{length}\NormalTok{(catch\_data)}
  
  \CommentTok{\# Приоры для r на основе устойчивости (resilience)}
\NormalTok{  r\_priors }\OtherTok{\textless{}{-}} \FunctionTok{list}\NormalTok{(}
    \StringTok{"Very low"} \OtherTok{=} \FunctionTok{c}\NormalTok{(}\FloatTok{0.015}\NormalTok{, }\FloatTok{0.1}\NormalTok{),}
    \StringTok{"Low"} \OtherTok{=} \FunctionTok{c}\NormalTok{(}\FloatTok{0.05}\NormalTok{, }\FloatTok{0.5}\NormalTok{),}
    \StringTok{"Medium"} \OtherTok{=} \FunctionTok{c}\NormalTok{(}\FloatTok{0.2}\NormalTok{, }\FloatTok{1.0}\NormalTok{),}
    \StringTok{"High"} \OtherTok{=} \FunctionTok{c}\NormalTok{(}\FloatTok{0.6}\NormalTok{, }\FloatTok{1.5}\NormalTok{)}
\NormalTok{  )}
  
\NormalTok{  r\_range }\OtherTok{\textless{}{-}}\NormalTok{ r\_priors[[resilience]]}
  
  \CommentTok{\# Приоры для K (2{-}25 раз больше максимального улова)}
\NormalTok{  k\_range }\OtherTok{\textless{}{-}} \FunctionTok{c}\NormalTok{(}\FunctionTok{max}\NormalTok{(catch\_data) }\SpecialCharTok{*} \DecValTok{2}\NormalTok{, }\FunctionTok{max}\NormalTok{(catch\_data) }\SpecialCharTok{*} \DecValTok{25}\NormalTok{)}
  
  \CommentTok{\# Приоры для начального и конечного истощения}
  \CommentTok{\# Основаны на трендах в уловах}
\NormalTok{  catch\_trend }\OtherTok{\textless{}{-}} \FunctionTok{mean}\NormalTok{(}\FunctionTok{tail}\NormalTok{(catch\_data, }\DecValTok{3}\NormalTok{)) }\SpecialCharTok{/} \FunctionTok{mean}\NormalTok{(}\FunctionTok{head}\NormalTok{(catch\_data, }\DecValTok{3}\NormalTok{))}
  
  \ControlFlowTok{if}\NormalTok{ (catch\_trend }\SpecialCharTok{\textgreater{}} \FloatTok{1.5}\NormalTok{) \{}
    \CommentTok{\# Растущий тренд {-} запас был слабо эксплуатируемым}
\NormalTok{    start\_depl }\OtherTok{\textless{}{-}} \FunctionTok{c}\NormalTok{(}\FloatTok{0.5}\NormalTok{, }\FloatTok{0.9}\NormalTok{)}
\NormalTok{    end\_depl }\OtherTok{\textless{}{-}} \FunctionTok{c}\NormalTok{(}\FloatTok{0.3}\NormalTok{, }\FloatTok{0.7}\NormalTok{)}
\NormalTok{  \} }\ControlFlowTok{else} \ControlFlowTok{if}\NormalTok{ (catch\_trend }\SpecialCharTok{\textless{}} \FloatTok{0.5}\NormalTok{) \{}
    \CommentTok{\# Снижающийся тренд {-} запас истощается}
\NormalTok{    start\_depl }\OtherTok{\textless{}{-}} \FunctionTok{c}\NormalTok{(}\FloatTok{0.3}\NormalTok{, }\FloatTok{0.7}\NormalTok{)}
\NormalTok{    end\_depl }\OtherTok{\textless{}{-}} \FunctionTok{c}\NormalTok{(}\FloatTok{0.01}\NormalTok{, }\FloatTok{0.4}\NormalTok{)}
\NormalTok{  \} }\ControlFlowTok{else}\NormalTok{ \{}
    \CommentTok{\# Стабильный тренд}
\NormalTok{    start\_depl }\OtherTok{\textless{}{-}} \FunctionTok{c}\NormalTok{(}\FloatTok{0.3}\NormalTok{, }\FloatTok{0.7}\NormalTok{)}
\NormalTok{    end\_depl }\OtherTok{\textless{}{-}} \FunctionTok{c}\NormalTok{(}\FloatTok{0.2}\NormalTok{, }\FloatTok{0.6}\NormalTok{)}
\NormalTok{  \}}
  
  \CommentTok{\# Массивы для результатов}
\NormalTok{  viable\_r }\OtherTok{\textless{}{-}} \FunctionTok{numeric}\NormalTok{()}
\NormalTok{  viable\_k }\OtherTok{\textless{}{-}} \FunctionTok{numeric}\NormalTok{()}
\NormalTok{  viable\_msy }\OtherTok{\textless{}{-}} \FunctionTok{numeric}\NormalTok{()}
\NormalTok{  viable\_b }\OtherTok{\textless{}{-}} \FunctionTok{matrix}\NormalTok{(}\AttributeTok{nrow =} \DecValTok{0}\NormalTok{, }\AttributeTok{ncol =}\NormalTok{ nyears }\SpecialCharTok{+} \DecValTok{1}\NormalTok{)}
  
  \FunctionTok{cat}\NormalTok{(}\StringTok{"Запуск CMSY с"}\NormalTok{, nsim, }\StringTok{"итерациями...}\SpecialCharTok{\textbackslash{}n}\StringTok{"}\NormalTok{)}
\NormalTok{  pb }\OtherTok{\textless{}{-}} \FunctionTok{txtProgressBar}\NormalTok{(}\AttributeTok{min =} \DecValTok{0}\NormalTok{, }\AttributeTok{max =}\NormalTok{ nsim, }\AttributeTok{style =} \DecValTok{3}\NormalTok{)}
  
  \ControlFlowTok{for}\NormalTok{ (sim }\ControlFlowTok{in} \DecValTok{1}\SpecialCharTok{:}\NormalTok{nsim) \{}
    
    \CommentTok{\# Случайные параметры}
\NormalTok{    r }\OtherTok{\textless{}{-}} \FunctionTok{runif}\NormalTok{(}\DecValTok{1}\NormalTok{, r\_range[}\DecValTok{1}\NormalTok{], r\_range[}\DecValTok{2}\NormalTok{])}
\NormalTok{    k }\OtherTok{\textless{}{-}} \FunctionTok{runif}\NormalTok{(}\DecValTok{1}\NormalTok{, k\_range[}\DecValTok{1}\NormalTok{], k\_range[}\DecValTok{2}\NormalTok{])}
\NormalTok{    start\_b }\OtherTok{\textless{}{-}} \FunctionTok{runif}\NormalTok{(}\DecValTok{1}\NormalTok{, start\_depl[}\DecValTok{1}\NormalTok{], start\_depl[}\DecValTok{2}\NormalTok{])}
    
    \CommentTok{\# Траектория биомассы}
\NormalTok{    B }\OtherTok{\textless{}{-}} \FunctionTok{numeric}\NormalTok{(nyears }\SpecialCharTok{+} \DecValTok{1}\NormalTok{)}
\NormalTok{    B[}\DecValTok{1}\NormalTok{] }\OtherTok{\textless{}{-}}\NormalTok{ k }\SpecialCharTok{*}\NormalTok{ start\_b}
    
\NormalTok{    viable }\OtherTok{\textless{}{-}} \ConstantTok{TRUE}
    
    \ControlFlowTok{for}\NormalTok{ (t }\ControlFlowTok{in} \DecValTok{1}\SpecialCharTok{:}\NormalTok{nyears) \{}
      \CommentTok{\# Продукция}
\NormalTok{      surplus }\OtherTok{\textless{}{-}}\NormalTok{ r }\SpecialCharTok{*}\NormalTok{ B[t] }\SpecialCharTok{*}\NormalTok{ (}\DecValTok{1} \SpecialCharTok{{-}}\NormalTok{ B[t]}\SpecialCharTok{/}\NormalTok{k)}
      
      \CommentTok{\# Проверка реалистичности}
      \ControlFlowTok{if}\NormalTok{ (catch\_data[t] }\SpecialCharTok{\textgreater{}}\NormalTok{ B[t] }\SpecialCharTok{+}\NormalTok{ surplus) \{}
\NormalTok{        viable }\OtherTok{\textless{}{-}} \ConstantTok{FALSE}
        \ControlFlowTok{break}
\NormalTok{      \}}
      
      \CommentTok{\# Обновление}
\NormalTok{      B[t}\SpecialCharTok{+}\DecValTok{1}\NormalTok{] }\OtherTok{\textless{}{-}}\NormalTok{ B[t] }\SpecialCharTok{+}\NormalTok{ surplus }\SpecialCharTok{{-}}\NormalTok{ catch\_data[t]}
      
      \ControlFlowTok{if}\NormalTok{ (B[t}\SpecialCharTok{+}\DecValTok{1}\NormalTok{] }\SpecialCharTok{\textless{}=} \DecValTok{0} \SpecialCharTok{||}\NormalTok{ B[t}\SpecialCharTok{+}\DecValTok{1}\NormalTok{] }\SpecialCharTok{\textgreater{}}\NormalTok{ k }\SpecialCharTok{*} \FloatTok{1.1}\NormalTok{) \{}
\NormalTok{        viable }\OtherTok{\textless{}{-}} \ConstantTok{FALSE}
        \ControlFlowTok{break}
\NormalTok{      \}}
\NormalTok{    \}}
    
    \CommentTok{\# Проверка конечного истощения}
\NormalTok{    final\_depl }\OtherTok{\textless{}{-}}\NormalTok{ B[nyears}\SpecialCharTok{+}\DecValTok{1}\NormalTok{] }\SpecialCharTok{/}\NormalTok{ k}
    \ControlFlowTok{if}\NormalTok{ (viable }\SpecialCharTok{\&\&}\NormalTok{ final\_depl }\SpecialCharTok{\textgreater{}=}\NormalTok{ end\_depl[}\DecValTok{1}\NormalTok{] }\SpecialCharTok{\&\&}\NormalTok{ final\_depl }\SpecialCharTok{\textless{}=}\NormalTok{ end\_depl[}\DecValTok{2}\NormalTok{]) \{}
\NormalTok{      viable\_r }\OtherTok{\textless{}{-}} \FunctionTok{c}\NormalTok{(viable\_r, r)}
\NormalTok{      viable\_k }\OtherTok{\textless{}{-}} \FunctionTok{c}\NormalTok{(viable\_k, k)}
\NormalTok{      viable\_msy }\OtherTok{\textless{}{-}} \FunctionTok{c}\NormalTok{(viable\_msy, r }\SpecialCharTok{*}\NormalTok{ k }\SpecialCharTok{/} \DecValTok{4}\NormalTok{)}
\NormalTok{      viable\_b }\OtherTok{\textless{}{-}} \FunctionTok{rbind}\NormalTok{(viable\_b, B)}
\NormalTok{    \}}
    
    \FunctionTok{setTxtProgressBar}\NormalTok{(pb, sim)}
\NormalTok{  \}}
  \FunctionTok{close}\NormalTok{(pb)}
  
\NormalTok{  n\_viable }\OtherTok{\textless{}{-}} \FunctionTok{length}\NormalTok{(viable\_r)}
  \FunctionTok{cat}\NormalTok{(}\FunctionTok{sprintf}\NormalTok{(}\StringTok{"}\SpecialCharTok{\textbackslash{}n}\StringTok{✓ Жизнеспособных комбинаций: \%d из \%d (\%.1f\%\%)}\SpecialCharTok{\textbackslash{}n}\StringTok{"}\NormalTok{, }
\NormalTok{              n\_viable, nsim, n\_viable}\SpecialCharTok{/}\NormalTok{nsim}\SpecialCharTok{*}\DecValTok{100}\NormalTok{))}
  
  \ControlFlowTok{if}\NormalTok{ (n\_viable }\SpecialCharTok{\textless{}} \DecValTok{10}\NormalTok{) \{}
    \FunctionTok{cat}\NormalTok{(}\StringTok{"⚠ Слишком мало жизнеспособных результатов! Попробуйте другие настройки.}\SpecialCharTok{\textbackslash{}n}\StringTok{"}\NormalTok{)}
    \FunctionTok{return}\NormalTok{(}\ConstantTok{NULL}\NormalTok{)}
\NormalTok{  \}}
  
  \CommentTok{\# Расчет траекторий B/Bmsy и F/Fmsy}
\NormalTok{  bmsy\_mat }\OtherTok{\textless{}{-}}\NormalTok{ viable\_k }\SpecialCharTok{/} \DecValTok{2}  \CommentTok{\# Bmsy = K/2 для модели Шефера}
\NormalTok{  bbmsy\_trajectories }\OtherTok{\textless{}{-}}\NormalTok{ viable\_b }\SpecialCharTok{/}\NormalTok{ bmsy\_mat}
  
  \CommentTok{\# F/Fmsy}
\NormalTok{  ffmsy\_trajectories }\OtherTok{\textless{}{-}} \FunctionTok{matrix}\NormalTok{(}\AttributeTok{nrow =}\NormalTok{ n\_viable, }\AttributeTok{ncol =}\NormalTok{ nyears)}
  \ControlFlowTok{for}\NormalTok{ (i }\ControlFlowTok{in} \DecValTok{1}\SpecialCharTok{:}\NormalTok{n\_viable) \{}
    \ControlFlowTok{for}\NormalTok{ (t }\ControlFlowTok{in} \DecValTok{1}\SpecialCharTok{:}\NormalTok{nyears) \{}
\NormalTok{      f\_t }\OtherTok{\textless{}{-}}\NormalTok{ catch\_data[t] }\SpecialCharTok{/}\NormalTok{ viable\_b[i, t]}
\NormalTok{      fmsy }\OtherTok{\textless{}{-}}\NormalTok{ viable\_r[i] }\SpecialCharTok{/} \DecValTok{2}
\NormalTok{      ffmsy\_trajectories[i, t] }\OtherTok{\textless{}{-}}\NormalTok{ f\_t }\SpecialCharTok{/}\NormalTok{ fmsy}
\NormalTok{    \}}
\NormalTok{  \}}
  
  \FunctionTok{return}\NormalTok{(}\FunctionTok{list}\NormalTok{(}
    \AttributeTok{r =}\NormalTok{ viable\_r,}
    \AttributeTok{k =}\NormalTok{ viable\_k,}
    \AttributeTok{msy =}\NormalTok{ viable\_msy,}
    \AttributeTok{biomass =}\NormalTok{ viable\_b,}
    \AttributeTok{bbmsy =}\NormalTok{ bbmsy\_trajectories,}
    \AttributeTok{ffmsy =}\NormalTok{ ffmsy\_trajectories,}
    \AttributeTok{n\_viable =}\NormalTok{ n\_viable}
\NormalTok{  ))}
\NormalTok{\}}

\CommentTok{\# Запуск CMSY}
\NormalTok{cmsy\_results }\OtherTok{\textless{}{-}} \FunctionTok{CMSY\_simple}\NormalTok{(}
  \AttributeTok{catch\_data =}\NormalTok{ Catch,}
  \AttributeTok{resilience =} \StringTok{"Medium"}\NormalTok{,}
  \AttributeTok{nsim =} \DecValTok{10000}
\NormalTok{)}

\ControlFlowTok{if}\NormalTok{ (}\SpecialCharTok{!}\FunctionTok{is.null}\NormalTok{(cmsy\_results)) \{}
  \CommentTok{\# Вывод результатов CMSY}
  \FunctionTok{cat}\NormalTok{(}\StringTok{"}\SpecialCharTok{\textbackslash{}n}\StringTok{{-}{-}{-} Результаты CMSY {-}{-}{-}}\SpecialCharTok{\textbackslash{}n}\StringTok{"}\NormalTok{)}
  \FunctionTok{cat}\NormalTok{(}\FunctionTok{sprintf}\NormalTok{(}\StringTok{"r: \%.3f [\%.3f {-} \%.3f]}\SpecialCharTok{\textbackslash{}n}\StringTok{"}\NormalTok{,}
              \FunctionTok{median}\NormalTok{(cmsy\_results}\SpecialCharTok{$}\NormalTok{r),}
              \FunctionTok{quantile}\NormalTok{(cmsy\_results}\SpecialCharTok{$}\NormalTok{r, }\FloatTok{0.025}\NormalTok{),}
              \FunctionTok{quantile}\NormalTok{(cmsy\_results}\SpecialCharTok{$}\NormalTok{r, }\FloatTok{0.975}\NormalTok{)))}
  \FunctionTok{cat}\NormalTok{(}\FunctionTok{sprintf}\NormalTok{(}\StringTok{"K: \%.1f [\%.1f {-} \%.1f] тыс. т}\SpecialCharTok{\textbackslash{}n}\StringTok{"}\NormalTok{,}
              \FunctionTok{median}\NormalTok{(cmsy\_results}\SpecialCharTok{$}\NormalTok{k),}
              \FunctionTok{quantile}\NormalTok{(cmsy\_results}\SpecialCharTok{$}\NormalTok{k, }\FloatTok{0.025}\NormalTok{),}
              \FunctionTok{quantile}\NormalTok{(cmsy\_results}\SpecialCharTok{$}\NormalTok{k, }\FloatTok{0.975}\NormalTok{)))}
  \FunctionTok{cat}\NormalTok{(}\FunctionTok{sprintf}\NormalTok{(}\StringTok{"MSY: \%.1f [\%.1f {-} \%.1f] тыс. т}\SpecialCharTok{\textbackslash{}n}\StringTok{"}\NormalTok{,}
              \FunctionTok{median}\NormalTok{(cmsy\_results}\SpecialCharTok{$}\NormalTok{msy),}
              \FunctionTok{quantile}\NormalTok{(cmsy\_results}\SpecialCharTok{$}\NormalTok{msy, }\FloatTok{0.025}\NormalTok{),}
              \FunctionTok{quantile}\NormalTok{(cmsy\_results}\SpecialCharTok{$}\NormalTok{msy, }\FloatTok{0.975}\NormalTok{)))}
  
  \CommentTok{\# Текущее состояние}
\NormalTok{  current\_bbmsy }\OtherTok{\textless{}{-}}\NormalTok{ cmsy\_results}\SpecialCharTok{$}\NormalTok{bbmsy[, nyears]}
  \FunctionTok{cat}\NormalTok{(}\FunctionTok{sprintf}\NormalTok{(}\StringTok{"B/Bmsy (2024): \%.2f [\%.2f {-} \%.2f]}\SpecialCharTok{\textbackslash{}n}\StringTok{"}\NormalTok{,}
              \FunctionTok{median}\NormalTok{(current\_bbmsy),}
              \FunctionTok{quantile}\NormalTok{(current\_bbmsy, }\FloatTok{0.025}\NormalTok{),}
              \FunctionTok{quantile}\NormalTok{(current\_bbmsy, }\FloatTok{0.975}\NormalTok{)))}
  
\NormalTok{  current\_ffmsy }\OtherTok{\textless{}{-}}\NormalTok{ cmsy\_results}\SpecialCharTok{$}\NormalTok{ffmsy[, nyears]}
  \FunctionTok{cat}\NormalTok{(}\FunctionTok{sprintf}\NormalTok{(}\StringTok{"F/Fmsy (2024): \%.2f [\%.2f {-} \%.2f]}\SpecialCharTok{\textbackslash{}n}\StringTok{"}\NormalTok{,}
              \FunctionTok{median}\NormalTok{(current\_ffmsy),}
              \FunctionTok{quantile}\NormalTok{(current\_ffmsy, }\FloatTok{0.025}\NormalTok{),}
              \FunctionTok{quantile}\NormalTok{(current\_ffmsy, }\FloatTok{0.975}\NormalTok{)))}
\NormalTok{\}}


\CommentTok{\# ======================= МЕТОД 4: CC (Constant Catch) =======================}

\FunctionTok{cat}\NormalTok{(}\StringTok{"}\SpecialCharTok{\textbackslash{}n}\StringTok{========== МЕТОД CC (Constant Catch) ==========}\SpecialCharTok{\textbackslash{}n}\StringTok{"}\NormalTok{)}
\FunctionTok{cat}\NormalTok{(}\StringTok{"Простейший метод {-} постоянный улов на уровне исторического среднего}\SpecialCharTok{\textbackslash{}n\textbackslash{}n}\StringTok{"}\NormalTok{)}

\CommentTok{\# CC метод из DLMtool}
\NormalTok{CC\_results }\OtherTok{\textless{}{-}} \FunctionTok{CC1}\NormalTok{(}
  \AttributeTok{x =} \DecValTok{1}\NormalTok{,}
  \AttributeTok{Data =}\NormalTok{ CatchOnly\_Data,}
  \AttributeTok{reps =} \DecValTok{1}
\NormalTok{)}

\CommentTok{\# Извлекаем результат из объекта Rec}
\ControlFlowTok{if}\NormalTok{(}\FunctionTok{class}\NormalTok{(CC\_results)[}\DecValTok{1}\NormalTok{] }\SpecialCharTok{==} \StringTok{"Rec"}\NormalTok{) \{}
\NormalTok{  cc\_tac }\OtherTok{\textless{}{-}}\NormalTok{ CC\_results}\SpecialCharTok{@}\NormalTok{TAC}
  
  \FunctionTok{cat}\NormalTok{(}\StringTok{"{-}{-}{-} Результаты CC {-}{-}{-}}\SpecialCharTok{\textbackslash{}n}\StringTok{"}\NormalTok{)}
  \FunctionTok{cat}\NormalTok{(}\FunctionTok{sprintf}\NormalTok{(}\StringTok{"Рекомендуемый TAC: \%.1f тыс. т}\SpecialCharTok{\textbackslash{}n}\StringTok{"}\NormalTok{, cc\_tac))}
  \FunctionTok{cat}\NormalTok{(}\StringTok{"(Средний исторический улов за последние 5 лет)}\SpecialCharTok{\textbackslash{}n}\StringTok{"}\NormalTok{)}
  \FunctionTok{cat}\NormalTok{(}\FunctionTok{sprintf}\NormalTok{(}\StringTok{"Это составляет \%.0f\%\% от текущего улова}\SpecialCharTok{\textbackslash{}n}\StringTok{"}\NormalTok{, }
\NormalTok{              cc\_tac}\SpecialCharTok{/}\FunctionTok{tail}\NormalTok{(Catch, }\DecValTok{1}\NormalTok{) }\SpecialCharTok{*} \DecValTok{100}\NormalTok{))}
\NormalTok{\} }\ControlFlowTok{else}\NormalTok{ \{}
\NormalTok{  cc\_tac }\OtherTok{\textless{}{-}} \ConstantTok{NA}
  \FunctionTok{cat}\NormalTok{(}\StringTok{"Ошибка: Неожиданный тип результата CC}\SpecialCharTok{\textbackslash{}n}\StringTok{"}\NormalTok{)}
\NormalTok{\}}

\CommentTok{\# ======================= ВИЗУАЛИЗАЦИЯ РЕЗУЛЬТАТОВ =======================}

\FunctionTok{cat}\NormalTok{(}\StringTok{"}\SpecialCharTok{\textbackslash{}n}\StringTok{========== СОЗДАНИЕ ГРАФИКОВ ==========}\SpecialCharTok{\textbackslash{}n}\StringTok{"}\NormalTok{)}

\CommentTok{\# 1. Сравнение оценок MSY}
\ControlFlowTok{if}\NormalTok{ (}\SpecialCharTok{!}\FunctionTok{is.null}\NormalTok{(cmsy\_results)) \{}
\NormalTok{  msy\_comparison }\OtherTok{\textless{}{-}} \FunctionTok{data.frame}\NormalTok{(}
    \AttributeTok{Method =} \FunctionTok{c}\NormalTok{(}\FunctionTok{rep}\NormalTok{(}\StringTok{"DB{-}SRA"}\NormalTok{, }\FunctionTok{length}\NormalTok{(dbsra\_results}\SpecialCharTok{$}\NormalTok{msy)),}
               \FunctionTok{rep}\NormalTok{(}\StringTok{"CMSY"}\NormalTok{, }\FunctionTok{length}\NormalTok{(cmsy\_results}\SpecialCharTok{$}\NormalTok{msy))),}
    \AttributeTok{MSY =} \FunctionTok{c}\NormalTok{(dbsra\_results}\SpecialCharTok{$}\NormalTok{msy, cmsy\_results}\SpecialCharTok{$}\NormalTok{msy)}
\NormalTok{  )}
  
\NormalTok{  p2 }\OtherTok{\textless{}{-}} \FunctionTok{ggplot}\NormalTok{(msy\_comparison, }\FunctionTok{aes}\NormalTok{(}\AttributeTok{x =}\NormalTok{ Method, }\AttributeTok{y =}\NormalTok{ MSY, }\AttributeTok{fill =}\NormalTok{ Method)) }\SpecialCharTok{+}
    \FunctionTok{geom\_violin}\NormalTok{(}\AttributeTok{alpha =} \FloatTok{0.7}\NormalTok{, }\AttributeTok{draw\_quantiles =} \FunctionTok{c}\NormalTok{(}\FloatTok{0.25}\NormalTok{, }\FloatTok{0.5}\NormalTok{, }\FloatTok{0.75}\NormalTok{)) }\SpecialCharTok{+}
    \FunctionTok{geom\_boxplot}\NormalTok{(}\AttributeTok{width =} \FloatTok{0.2}\NormalTok{, }\AttributeTok{alpha =} \FloatTok{0.9}\NormalTok{, }\AttributeTok{outlier.shape =} \ConstantTok{NA}\NormalTok{) }\SpecialCharTok{+}
    \FunctionTok{scale\_fill\_viridis\_d}\NormalTok{() }\SpecialCharTok{+}
    \FunctionTok{labs}\NormalTok{(}\AttributeTok{title =} \StringTok{"Сравнение оценок MSY"}\NormalTok{,}
         \AttributeTok{subtitle =} \StringTok{"Распределения из байесовских методов"}\NormalTok{,}
         \AttributeTok{x =} \StringTok{""}\NormalTok{, }\AttributeTok{y =} \StringTok{"MSY (тыс. т)"}\NormalTok{) }\SpecialCharTok{+}
    \FunctionTok{theme\_minimal}\NormalTok{() }\SpecialCharTok{+}
    \FunctionTok{theme}\NormalTok{(}\AttributeTok{legend.position =} \StringTok{"none"}\NormalTok{)}
  
  \FunctionTok{print}\NormalTok{(p2)}
\NormalTok{\}}
\end{Highlighting}
\end{Shaded}

\begin{center}
\includegraphics[width=0.8\linewidth,height=\textheight,keepaspectratio]{images/DLMII2.PNG}
\end{center}

\begin{Shaded}
\begin{Highlighting}[]
\CommentTok{\# 2. Траектории B/Bmsy (для CMSY)}
\ControlFlowTok{if}\NormalTok{ (}\SpecialCharTok{!}\FunctionTok{is.null}\NormalTok{(cmsy\_results)) \{}
  \CommentTok{\# Подготовка данных}
\NormalTok{  bbmsy\_median }\OtherTok{\textless{}{-}} \FunctionTok{apply}\NormalTok{(cmsy\_results}\SpecialCharTok{$}\NormalTok{bbmsy, }\DecValTok{2}\NormalTok{, median)}
\NormalTok{  bbmsy\_q25 }\OtherTok{\textless{}{-}} \FunctionTok{apply}\NormalTok{(cmsy\_results}\SpecialCharTok{$}\NormalTok{bbmsy, }\DecValTok{2}\NormalTok{, quantile, }\FloatTok{0.25}\NormalTok{)}
\NormalTok{  bbmsy\_q75 }\OtherTok{\textless{}{-}} \FunctionTok{apply}\NormalTok{(cmsy\_results}\SpecialCharTok{$}\NormalTok{bbmsy, }\DecValTok{2}\NormalTok{, quantile, }\FloatTok{0.75}\NormalTok{)}
\NormalTok{  bbmsy\_q05 }\OtherTok{\textless{}{-}} \FunctionTok{apply}\NormalTok{(cmsy\_results}\SpecialCharTok{$}\NormalTok{bbmsy, }\DecValTok{2}\NormalTok{, quantile, }\FloatTok{0.05}\NormalTok{)}
\NormalTok{  bbmsy\_q95 }\OtherTok{\textless{}{-}} \FunctionTok{apply}\NormalTok{(cmsy\_results}\SpecialCharTok{$}\NormalTok{bbmsy, }\DecValTok{2}\NormalTok{, quantile, }\FloatTok{0.95}\NormalTok{)}
  
\NormalTok{  bbmsy\_df }\OtherTok{\textless{}{-}} \FunctionTok{data.frame}\NormalTok{(}
    \AttributeTok{Year =} \FunctionTok{c}\NormalTok{(Year, }\FunctionTok{max}\NormalTok{(Year) }\SpecialCharTok{+} \DecValTok{1}\NormalTok{),}
    \AttributeTok{Median =}\NormalTok{ bbmsy\_median,}
    \AttributeTok{Q25 =}\NormalTok{ bbmsy\_q25,}
    \AttributeTok{Q75 =}\NormalTok{ bbmsy\_q75,}
    \AttributeTok{Q05 =}\NormalTok{ bbmsy\_q05,}
    \AttributeTok{Q95 =}\NormalTok{ bbmsy\_q95}
\NormalTok{  )}
  
\NormalTok{  p3 }\OtherTok{\textless{}{-}} \FunctionTok{ggplot}\NormalTok{(bbmsy\_df, }\FunctionTok{aes}\NormalTok{(}\AttributeTok{x =}\NormalTok{ Year)) }\SpecialCharTok{+}
    \CommentTok{\# 90\% интервал}
    \FunctionTok{geom\_ribbon}\NormalTok{(}\FunctionTok{aes}\NormalTok{(}\AttributeTok{ymin =}\NormalTok{ Q05, }\AttributeTok{ymax =}\NormalTok{ Q95), }\AttributeTok{alpha =} \FloatTok{0.2}\NormalTok{, }\AttributeTok{fill =} \StringTok{"blue"}\NormalTok{) }\SpecialCharTok{+}
    \CommentTok{\# 50\% интервал}
    \FunctionTok{geom\_ribbon}\NormalTok{(}\FunctionTok{aes}\NormalTok{(}\AttributeTok{ymin =}\NormalTok{ Q25, }\AttributeTok{ymax =}\NormalTok{ Q75), }\AttributeTok{alpha =} \FloatTok{0.4}\NormalTok{, }\AttributeTok{fill =} \StringTok{"blue"}\NormalTok{) }\SpecialCharTok{+}
    \CommentTok{\# Медиана}
    \FunctionTok{geom\_line}\NormalTok{(}\FunctionTok{aes}\NormalTok{(}\AttributeTok{y =}\NormalTok{ Median), }\AttributeTok{size =} \FloatTok{1.5}\NormalTok{, }\AttributeTok{color =} \StringTok{"darkblue"}\NormalTok{) }\SpecialCharTok{+}
    \CommentTok{\# Референсные линии}
    \FunctionTok{geom\_hline}\NormalTok{(}\AttributeTok{yintercept =} \DecValTok{1}\NormalTok{, }\AttributeTok{linetype =} \StringTok{"dashed"}\NormalTok{, }\AttributeTok{color =} \StringTok{"black"}\NormalTok{) }\SpecialCharTok{+}
    \FunctionTok{geom\_hline}\NormalTok{(}\AttributeTok{yintercept =} \FloatTok{0.5}\NormalTok{, }\AttributeTok{linetype =} \StringTok{"dotted"}\NormalTok{, }\AttributeTok{color =} \StringTok{"red"}\NormalTok{) }\SpecialCharTok{+}
    \CommentTok{\# Аннотации}
    \FunctionTok{annotate}\NormalTok{(}\StringTok{"text"}\NormalTok{, }\AttributeTok{x =} \FunctionTok{min}\NormalTok{(Year), }\AttributeTok{y =} \FloatTok{1.05}\NormalTok{, }\AttributeTok{label =} \StringTok{"Bmsy"}\NormalTok{, }\AttributeTok{hjust =} \DecValTok{0}\NormalTok{) }\SpecialCharTok{+}
    \FunctionTok{annotate}\NormalTok{(}\StringTok{"text"}\NormalTok{, }\AttributeTok{x =} \FunctionTok{min}\NormalTok{(Year), }\AttributeTok{y =} \FloatTok{0.55}\NormalTok{, }\AttributeTok{label =} \StringTok{"0.5 Bmsy"}\NormalTok{, }\AttributeTok{hjust =} \DecValTok{0}\NormalTok{, }\AttributeTok{color =} \StringTok{"red"}\NormalTok{) }\SpecialCharTok{+}
    \CommentTok{\# Оформление}
    \FunctionTok{labs}\NormalTok{(}\AttributeTok{title =} \StringTok{"Траектория B/Bmsy (CMSY)"}\NormalTok{,}
         \AttributeTok{subtitle =} \StringTok{"Медиана с 50\% и 90\% доверительными интервалами"}\NormalTok{,}
         \AttributeTok{x =} \StringTok{"Год"}\NormalTok{, }\AttributeTok{y =} \StringTok{"B/Bmsy"}\NormalTok{) }\SpecialCharTok{+}
    \FunctionTok{theme\_minimal}\NormalTok{() }\SpecialCharTok{+}
    \FunctionTok{coord\_cartesian}\NormalTok{(}\AttributeTok{ylim =} \FunctionTok{c}\NormalTok{(}\DecValTok{0}\NormalTok{, }\DecValTok{2}\NormalTok{))}
  
  \FunctionTok{print}\NormalTok{(p3)}
\NormalTok{\}}
\end{Highlighting}
\end{Shaded}

\begin{center}
\includegraphics[width=0.8\linewidth,height=\textheight,keepaspectratio]{images/DLMII3.PNG}
\end{center}

\begin{Shaded}
\begin{Highlighting}[]
\CommentTok{\# 3. Фазовая диаграмма Кобе}
\ControlFlowTok{if}\NormalTok{ (}\SpecialCharTok{!}\FunctionTok{is.null}\NormalTok{(cmsy\_results)) \{}
  \CommentTok{\# Проверяем размерности}
  \FunctionTok{cat}\NormalTok{(}\StringTok{"Размерность bbmsy:"}\NormalTok{, }\FunctionTok{dim}\NormalTok{(cmsy\_results}\SpecialCharTok{$}\NormalTok{bbmsy), }\StringTok{"}\SpecialCharTok{\textbackslash{}n}\StringTok{"}\NormalTok{)}
  \FunctionTok{cat}\NormalTok{(}\StringTok{"Размерность ffmsy:"}\NormalTok{, }\FunctionTok{dim}\NormalTok{(cmsy\_results}\SpecialCharTok{$}\NormalTok{ffmsy), }\StringTok{"}\SpecialCharTok{\textbackslash{}n}\StringTok{"}\NormalTok{)}
\NormalTok{\}  }
  \CommentTok{\# Для фазовой диаграммы нам нужны только годы с данными по уловам (не включая начальный год биомассы)}
  \CommentTok{\# bbmsy имеет 21 столбец (годы 0{-}20), ffmsy имеет 20 столбцов (годы 1{-}20)}
  \CommentTok{\# Используем столбцы 2:21 из bbmsy, чтобы соответствовать ffmsy}
  
  \CommentTok{\# Медианные траектории}
\NormalTok{  kobe\_median }\OtherTok{\textless{}{-}} \FunctionTok{data.frame}\NormalTok{(}
    \AttributeTok{Year =}\NormalTok{ Year,}
    \AttributeTok{BBmsy =} \FunctionTok{apply}\NormalTok{(cmsy\_results}\SpecialCharTok{$}\NormalTok{bbmsy[, }\DecValTok{2}\SpecialCharTok{:}\NormalTok{(nyears}\SpecialCharTok{+}\DecValTok{1}\NormalTok{)], }\DecValTok{2}\NormalTok{, median),  }\CommentTok{\# Столбцы 2:21}
    \AttributeTok{FFmsy =} \FunctionTok{apply}\NormalTok{(cmsy\_results}\SpecialCharTok{$}\NormalTok{ffmsy, }\DecValTok{2}\NormalTok{, median)  }\CommentTok{\# Все столбцы}
\NormalTok{  )}
  
  \CommentTok{\# Альтернативный вариант {-} использовать последние значения для точечной диаграммы}
\NormalTok{  kobe\_current }\OtherTok{\textless{}{-}} \FunctionTok{data.frame}\NormalTok{(}
    \AttributeTok{BBmsy =}\NormalTok{ cmsy\_results}\SpecialCharTok{$}\NormalTok{bbmsy[, nyears}\SpecialCharTok{+}\DecValTok{1}\NormalTok{],  }\CommentTok{\# Последний столбец bbmsy}
    \AttributeTok{FFmsy =}\NormalTok{ cmsy\_results}\SpecialCharTok{$}\NormalTok{ffmsy[, nyears]      }\CommentTok{\# Последний столбец ffmsy}
\NormalTok{  )}
  
  \CommentTok{\# График 1: Траектория во времени}
\NormalTok{  p4 }\OtherTok{\textless{}{-}} \FunctionTok{ggplot}\NormalTok{() }\SpecialCharTok{+}
    \CommentTok{\# Зоны Кобе}
    \FunctionTok{annotate}\NormalTok{(}\StringTok{"rect"}\NormalTok{, }\AttributeTok{xmin =} \DecValTok{0}\NormalTok{, }\AttributeTok{xmax =} \DecValTok{1}\NormalTok{, }\AttributeTok{ymin =} \DecValTok{1}\NormalTok{, }\AttributeTok{ymax =} \DecValTok{4}\NormalTok{,}
             \AttributeTok{fill =} \StringTok{"red"}\NormalTok{, }\AttributeTok{alpha =} \FloatTok{0.2}\NormalTok{) }\SpecialCharTok{+}
    \FunctionTok{annotate}\NormalTok{(}\StringTok{"rect"}\NormalTok{, }\AttributeTok{xmin =} \DecValTok{1}\NormalTok{, }\AttributeTok{xmax =} \DecValTok{4}\NormalTok{, }\AttributeTok{ymin =} \DecValTok{1}\NormalTok{, }\AttributeTok{ymax =} \DecValTok{4}\NormalTok{,}
             \AttributeTok{fill =} \StringTok{"\#FFA500"}\NormalTok{, }\AttributeTok{alpha =} \FloatTok{0.2}\NormalTok{) }\SpecialCharTok{+}
    \FunctionTok{annotate}\NormalTok{(}\StringTok{"rect"}\NormalTok{, }\AttributeTok{xmin =} \DecValTok{0}\NormalTok{, }\AttributeTok{xmax =} \DecValTok{1}\NormalTok{, }\AttributeTok{ymin =} \DecValTok{0}\NormalTok{, }\AttributeTok{ymax =} \DecValTok{1}\NormalTok{,}
             \AttributeTok{fill =} \StringTok{"\#FFFF00"}\NormalTok{, }\AttributeTok{alpha =} \FloatTok{0.2}\NormalTok{) }\SpecialCharTok{+}
    \FunctionTok{annotate}\NormalTok{(}\StringTok{"rect"}\NormalTok{, }\AttributeTok{xmin =} \DecValTok{1}\NormalTok{, }\AttributeTok{xmax =} \DecValTok{4}\NormalTok{, }\AttributeTok{ymin =} \DecValTok{0}\NormalTok{, }\AttributeTok{ymax =} \DecValTok{1}\NormalTok{,}
             \AttributeTok{fill =} \StringTok{"green"}\NormalTok{, }\AttributeTok{alpha =} \FloatTok{0.2}\NormalTok{) }\SpecialCharTok{+}
    
    \CommentTok{\# Медианная траектория}
    \FunctionTok{geom\_path}\NormalTok{(}\AttributeTok{data =}\NormalTok{ kobe\_median, }
              \FunctionTok{aes}\NormalTok{(}\AttributeTok{x =}\NormalTok{ BBmsy, }\AttributeTok{y =}\NormalTok{ FFmsy),}
              \AttributeTok{linewidth =} \FloatTok{1.5}\NormalTok{, }\AttributeTok{color =} \StringTok{"black"}\NormalTok{,}
              \AttributeTok{arrow =} \FunctionTok{arrow}\NormalTok{(}\AttributeTok{length =} \FunctionTok{unit}\NormalTok{(}\FloatTok{0.3}\NormalTok{, }\StringTok{"cm"}\NormalTok{))) }\SpecialCharTok{+}
    
    \CommentTok{\# Точки по годам}
    \FunctionTok{geom\_point}\NormalTok{(}\AttributeTok{data =}\NormalTok{ kobe\_median,}
               \FunctionTok{aes}\NormalTok{(}\AttributeTok{x =}\NormalTok{ BBmsy, }\AttributeTok{y =}\NormalTok{ FFmsy, }\AttributeTok{color =}\NormalTok{ Year),}
               \AttributeTok{size =} \DecValTok{3}\NormalTok{) }\SpecialCharTok{+}
    
    \CommentTok{\# Начало и конец}
    \FunctionTok{geom\_point}\NormalTok{(}\AttributeTok{data =}\NormalTok{ kobe\_median[}\DecValTok{1}\NormalTok{, ],}
               \FunctionTok{aes}\NormalTok{(}\AttributeTok{x =}\NormalTok{ BBmsy, }\AttributeTok{y =}\NormalTok{ FFmsy),}
               \AttributeTok{size =} \DecValTok{5}\NormalTok{, }\AttributeTok{shape =} \DecValTok{17}\NormalTok{, }\AttributeTok{color =} \StringTok{"blue"}\NormalTok{) }\SpecialCharTok{+}
    \FunctionTok{geom\_point}\NormalTok{(}\AttributeTok{data =}\NormalTok{ kobe\_median[}\FunctionTok{nrow}\NormalTok{(kobe\_median), ],}
               \FunctionTok{aes}\NormalTok{(}\AttributeTok{x =}\NormalTok{ BBmsy, }\AttributeTok{y =}\NormalTok{ FFmsy),}
               \AttributeTok{size =} \DecValTok{5}\NormalTok{, }\AttributeTok{shape =} \DecValTok{15}\NormalTok{, }\AttributeTok{color =} \StringTok{"red"}\NormalTok{) }\SpecialCharTok{+}
    
    \CommentTok{\# Референсные линии}
    \FunctionTok{geom\_vline}\NormalTok{(}\AttributeTok{xintercept =} \DecValTok{1}\NormalTok{, }\AttributeTok{linetype =} \StringTok{"solid"}\NormalTok{, }\AttributeTok{linewidth =} \FloatTok{0.8}\NormalTok{) }\SpecialCharTok{+}
    \FunctionTok{geom\_hline}\NormalTok{(}\AttributeTok{yintercept =} \DecValTok{1}\NormalTok{, }\AttributeTok{linetype =} \StringTok{"solid"}\NormalTok{, }\AttributeTok{linewidth =} \FloatTok{0.8}\NormalTok{) }\SpecialCharTok{+}
    
    \CommentTok{\# Оформление}
    \FunctionTok{scale\_color\_viridis\_c}\NormalTok{() }\SpecialCharTok{+}
    \FunctionTok{labs}\NormalTok{(}\AttributeTok{title =} \StringTok{"Фазовая диаграмма Кобе (CMSY)"}\NormalTok{,}
         \AttributeTok{subtitle =} \StringTok{"Медианная траектория состояния запаса"}\NormalTok{,}
         \AttributeTok{x =} \StringTok{"B/Bmsy"}\NormalTok{, }\AttributeTok{y =} \StringTok{"F/Fmsy"}\NormalTok{,}
         \AttributeTok{color =} \StringTok{"Год"}\NormalTok{) }\SpecialCharTok{+}
    \FunctionTok{theme\_minimal}\NormalTok{() }\SpecialCharTok{+}
    \FunctionTok{coord\_cartesian}\NormalTok{(}\AttributeTok{xlim =} \FunctionTok{c}\NormalTok{(}\DecValTok{0}\NormalTok{, }\DecValTok{2}\NormalTok{), }\AttributeTok{ylim =} \FunctionTok{c}\NormalTok{(}\DecValTok{0}\NormalTok{, }\DecValTok{3}\NormalTok{)) }\SpecialCharTok{+}
    
    \CommentTok{\# Подписи зон}
    \FunctionTok{annotate}\NormalTok{(}\StringTok{"text"}\NormalTok{, }\AttributeTok{x =} \FloatTok{0.5}\NormalTok{, }\AttributeTok{y =} \DecValTok{2}\NormalTok{, }\AttributeTok{label =} \StringTok{"Перелов +}\SpecialCharTok{\textbackslash{}n}\StringTok{Истощение"}\NormalTok{, }
             \AttributeTok{size =} \DecValTok{3}\NormalTok{, }\AttributeTok{fontface =} \StringTok{"bold"}\NormalTok{) }\SpecialCharTok{+}
    \FunctionTok{annotate}\NormalTok{(}\StringTok{"text"}\NormalTok{, }\AttributeTok{x =} \FloatTok{1.5}\NormalTok{, }\AttributeTok{y =} \DecValTok{2}\NormalTok{, }\AttributeTok{label =} \StringTok{"Перелов"}\NormalTok{, }
             \AttributeTok{size =} \DecValTok{3}\NormalTok{, }\AttributeTok{fontface =} \StringTok{"bold"}\NormalTok{) }\SpecialCharTok{+}
    \FunctionTok{annotate}\NormalTok{(}\StringTok{"text"}\NormalTok{, }\AttributeTok{x =} \FloatTok{0.5}\NormalTok{, }\AttributeTok{y =} \FloatTok{0.5}\NormalTok{, }\AttributeTok{label =} \StringTok{"Истощение"}\NormalTok{, }
             \AttributeTok{size =} \DecValTok{3}\NormalTok{, }\AttributeTok{fontface =} \StringTok{"bold"}\NormalTok{) }\SpecialCharTok{+}
    \FunctionTok{annotate}\NormalTok{(}\StringTok{"text"}\NormalTok{, }\AttributeTok{x =} \FloatTok{1.5}\NormalTok{, }\AttributeTok{y =} \FloatTok{0.5}\NormalTok{, }\AttributeTok{label =} \StringTok{"Устойчивое}\SpecialCharTok{\textbackslash{}n}\StringTok{состояние"}\NormalTok{, }
             \AttributeTok{size =} \DecValTok{3}\NormalTok{, }\AttributeTok{fontface =} \StringTok{"bold"}\NormalTok{, }\AttributeTok{color =} \StringTok{"darkgreen"}\NormalTok{)}
  
  \FunctionTok{print}\NormalTok{(p4)}
\end{Highlighting}
\end{Shaded}

\begin{center}
\includegraphics[width=0.8\linewidth,height=\textheight,keepaspectratio]{images/DLMII4.PNG}
\end{center}

\begin{Shaded}
\begin{Highlighting}[]
\CommentTok{\# График 2: Распределение текущего состояния с неопределенностью}
\NormalTok{  p5 }\OtherTok{\textless{}{-}} \FunctionTok{ggplot}\NormalTok{(kobe\_current, }\FunctionTok{aes}\NormalTok{(}\AttributeTok{x =}\NormalTok{ BBmsy, }\AttributeTok{y =}\NormalTok{ FFmsy)) }\SpecialCharTok{+}
    \CommentTok{\# Зоны Кобе}
    \FunctionTok{annotate}\NormalTok{(}\StringTok{"rect"}\NormalTok{, }\AttributeTok{xmin =} \DecValTok{0}\NormalTok{, }\AttributeTok{xmax =} \DecValTok{1}\NormalTok{, }\AttributeTok{ymin =} \DecValTok{1}\NormalTok{, }\AttributeTok{ymax =} \DecValTok{4}\NormalTok{,}
             \AttributeTok{fill =} \StringTok{"red"}\NormalTok{, }\AttributeTok{alpha =} \FloatTok{0.2}\NormalTok{) }\SpecialCharTok{+}
    \FunctionTok{annotate}\NormalTok{(}\StringTok{"rect"}\NormalTok{, }\AttributeTok{xmin =} \DecValTok{1}\NormalTok{, }\AttributeTok{xmax =} \DecValTok{4}\NormalTok{, }\AttributeTok{ymin =} \DecValTok{1}\NormalTok{, }\AttributeTok{ymax =} \DecValTok{4}\NormalTok{,}
             \AttributeTok{fill =} \StringTok{"\#FFA500"}\NormalTok{, }\AttributeTok{alpha =} \FloatTok{0.2}\NormalTok{) }\SpecialCharTok{+}
    \FunctionTok{annotate}\NormalTok{(}\StringTok{"rect"}\NormalTok{, }\AttributeTok{xmin =} \DecValTok{0}\NormalTok{, }\AttributeTok{xmax =} \DecValTok{1}\NormalTok{, }\AttributeTok{ymin =} \DecValTok{0}\NormalTok{, }\AttributeTok{ymax =} \DecValTok{1}\NormalTok{,}
             \AttributeTok{fill =} \StringTok{"\#FFFF00"}\NormalTok{, }\AttributeTok{alpha =} \FloatTok{0.2}\NormalTok{) }\SpecialCharTok{+}
    \FunctionTok{annotate}\NormalTok{(}\StringTok{"rect"}\NormalTok{, }\AttributeTok{xmin =} \DecValTok{1}\NormalTok{, }\AttributeTok{xmax =} \DecValTok{4}\NormalTok{, }\AttributeTok{ymin =} \DecValTok{0}\NormalTok{, }\AttributeTok{ymax =} \DecValTok{1}\NormalTok{,}
             \AttributeTok{fill =} \StringTok{"green"}\NormalTok{, }\AttributeTok{alpha =} \FloatTok{0.2}\NormalTok{) }\SpecialCharTok{+}
    
    \CommentTok{\# Точки всех симуляций}
    \FunctionTok{geom\_point}\NormalTok{(}\AttributeTok{alpha =} \FloatTok{0.3}\NormalTok{, }\AttributeTok{size =} \DecValTok{2}\NormalTok{, }\AttributeTok{color =} \StringTok{"darkblue"}\NormalTok{) }\SpecialCharTok{+}
    
    \CommentTok{\# Контуры плотности}
    \FunctionTok{stat\_density\_2d}\NormalTok{(}\AttributeTok{color =} \StringTok{"darkblue"}\NormalTok{, }\AttributeTok{alpha =} \FloatTok{0.7}\NormalTok{) }\SpecialCharTok{+}
    
    \CommentTok{\# Медиана}
    \FunctionTok{geom\_point}\NormalTok{(}\FunctionTok{aes}\NormalTok{(}\AttributeTok{x =} \FunctionTok{median}\NormalTok{(BBmsy), }\AttributeTok{y =} \FunctionTok{median}\NormalTok{(FFmsy)),}
               \AttributeTok{size =} \DecValTok{5}\NormalTok{, }\AttributeTok{color =} \StringTok{"red"}\NormalTok{, }\AttributeTok{shape =} \DecValTok{17}\NormalTok{) }\SpecialCharTok{+}
    
    \CommentTok{\# Референсные линии}
    \FunctionTok{geom\_vline}\NormalTok{(}\AttributeTok{xintercept =} \DecValTok{1}\NormalTok{, }\AttributeTok{linetype =} \StringTok{"solid"}\NormalTok{, }\AttributeTok{linewidth =} \FloatTok{0.8}\NormalTok{) }\SpecialCharTok{+}
    \FunctionTok{geom\_hline}\NormalTok{(}\AttributeTok{yintercept =} \DecValTok{1}\NormalTok{, }\AttributeTok{linetype =} \StringTok{"solid"}\NormalTok{, }\AttributeTok{linewidth =} \FloatTok{0.8}\NormalTok{) }\SpecialCharTok{+}
    
    \CommentTok{\# Оформление}
    \FunctionTok{labs}\NormalTok{(}\AttributeTok{title =} \StringTok{"Текущее состояние запаса (2024)"}\NormalTok{,}
         \AttributeTok{subtitle =} \FunctionTok{sprintf}\NormalTok{(}\StringTok{"CMSY оценка с неопределенностью (\%d симуляций)"}\NormalTok{, cmsy\_results}\SpecialCharTok{$}\NormalTok{n\_viable),}
         \AttributeTok{x =} \StringTok{"B/Bmsy"}\NormalTok{, }\AttributeTok{y =} \StringTok{"F/Fmsy"}\NormalTok{) }\SpecialCharTok{+}
    \FunctionTok{theme\_minimal}\NormalTok{() }\SpecialCharTok{+}
    \FunctionTok{coord\_cartesian}\NormalTok{(}\AttributeTok{xlim =} \FunctionTok{c}\NormalTok{(}\DecValTok{0}\NormalTok{, }\FloatTok{2.5}\NormalTok{), }\AttributeTok{ylim =} \FunctionTok{c}\NormalTok{(}\DecValTok{0}\NormalTok{, }\DecValTok{3}\NormalTok{)) }\SpecialCharTok{+}
    
    \CommentTok{\# Подписи зон}
    \FunctionTok{annotate}\NormalTok{(}\StringTok{"text"}\NormalTok{, }\AttributeTok{x =} \FloatTok{0.5}\NormalTok{, }\AttributeTok{y =} \FloatTok{2.5}\NormalTok{, }\AttributeTok{label =} \StringTok{"Критическое"}\NormalTok{, }
             \AttributeTok{size =} \DecValTok{3}\NormalTok{, }\AttributeTok{fontface =} \StringTok{"bold"}\NormalTok{, }\AttributeTok{color =} \StringTok{"darkred"}\NormalTok{) }\SpecialCharTok{+}
    \FunctionTok{annotate}\NormalTok{(}\StringTok{"text"}\NormalTok{, }\AttributeTok{x =} \FloatTok{1.75}\NormalTok{, }\AttributeTok{y =} \FloatTok{2.5}\NormalTok{, }\AttributeTok{label =} \StringTok{"Переэксплуатация"}\NormalTok{, }
             \AttributeTok{size =} \DecValTok{3}\NormalTok{, }\AttributeTok{fontface =} \StringTok{"bold"}\NormalTok{, }\AttributeTok{color =} \StringTok{"darkorange"}\NormalTok{) }\SpecialCharTok{+}
    \FunctionTok{annotate}\NormalTok{(}\StringTok{"text"}\NormalTok{, }\AttributeTok{x =} \FloatTok{0.5}\NormalTok{, }\AttributeTok{y =} \FloatTok{0.2}\NormalTok{, }\AttributeTok{label =} \StringTok{"Восстановление"}\NormalTok{, }
             \AttributeTok{size =} \DecValTok{3}\NormalTok{, }\AttributeTok{fontface =} \StringTok{"bold"}\NormalTok{, }\AttributeTok{color =} \StringTok{"goldenrod"}\NormalTok{) }\SpecialCharTok{+}
    \FunctionTok{annotate}\NormalTok{(}\StringTok{"text"}\NormalTok{, }\AttributeTok{x =} \FloatTok{1.75}\NormalTok{, }\AttributeTok{y =} \FloatTok{0.2}\NormalTok{, }\AttributeTok{label =} \StringTok{"Устойчивое"}\NormalTok{, }
             \AttributeTok{size =} \DecValTok{3}\NormalTok{, }\AttributeTok{fontface =} \StringTok{"bold"}\NormalTok{, }\AttributeTok{color =} \StringTok{"darkgreen"}\NormalTok{)}
  
  \FunctionTok{print}\NormalTok{(p5)}
\end{Highlighting}
\end{Shaded}

\begin{center}
\includegraphics[width=0.8\linewidth,height=\textheight,keepaspectratio]{images/DLMII5.PNG}
\end{center}

\begin{Shaded}
\begin{Highlighting}[]
\CommentTok{\# ======================= СВОДНАЯ ТАБЛИЦА РЕЗУЛЬТАТОВ =======================}
\CommentTok{\# Создание сводной таблицы с надежной обработкой NA}
\CommentTok{\# Сначала обрабатываем DCAC результат}
\NormalTok{dcac\_tac\_value }\OtherTok{\textless{}{-}} \FloatTok{6.0}  \CommentTok{\# Используем значение из ручного расчета как основу}
\ControlFlowTok{if}\NormalTok{ (}\FunctionTok{exists}\NormalTok{(}\StringTok{"DCAC\_results"}\NormalTok{) }\SpecialCharTok{\&\&} \FunctionTok{inherits}\NormalTok{(DCAC\_results, }\StringTok{"Rec"}\NormalTok{) }\SpecialCharTok{\&\&} \SpecialCharTok{!}\FunctionTok{is.null}\NormalTok{(DCAC\_results}\SpecialCharTok{@}\NormalTok{TAC)) \{}
\NormalTok{  valid\_tac }\OtherTok{\textless{}{-}}\NormalTok{ DCAC\_results}\SpecialCharTok{@}\NormalTok{TAC[}\SpecialCharTok{!}\FunctionTok{is.na}\NormalTok{(DCAC\_results}\SpecialCharTok{@}\NormalTok{TAC)]}
  \ControlFlowTok{if}\NormalTok{ (}\FunctionTok{length}\NormalTok{(valid\_tac) }\SpecialCharTok{\textgreater{}} \DecValTok{0}\NormalTok{) \{}
\NormalTok{    dcac\_tac\_value }\OtherTok{\textless{}{-}} \FunctionTok{median}\NormalTok{(valid\_tac)}
\NormalTok{  \}}
\NormalTok{\}}

\CommentTok{\# Для DB{-}SRA и CMSY добавляем проверку на NA}
\NormalTok{dbsra\_msy }\OtherTok{\textless{}{-}} \ControlFlowTok{if}\NormalTok{ (}\FunctionTok{exists}\NormalTok{(}\StringTok{"dbsra\_results"}\NormalTok{) }\SpecialCharTok{\&\&} \SpecialCharTok{!}\FunctionTok{is.null}\NormalTok{(dbsra\_results}\SpecialCharTok{$}\NormalTok{msy)) }\FunctionTok{median}\NormalTok{(dbsra\_results}\SpecialCharTok{$}\NormalTok{msy, }\AttributeTok{na.rm =} \ConstantTok{TRUE}\NormalTok{) }\ControlFlowTok{else} \ConstantTok{NA}
\NormalTok{cmsy\_msy }\OtherTok{\textless{}{-}} \ControlFlowTok{if}\NormalTok{ (}\SpecialCharTok{!}\FunctionTok{is.null}\NormalTok{(cmsy\_results) }\SpecialCharTok{\&\&} \SpecialCharTok{!}\FunctionTok{is.null}\NormalTok{(cmsy\_results}\SpecialCharTok{$}\NormalTok{msy)) }\FunctionTok{median}\NormalTok{(cmsy\_results}\SpecialCharTok{$}\NormalTok{msy, }\AttributeTok{na.rm =} \ConstantTok{TRUE}\NormalTok{) }\ControlFlowTok{else} \ConstantTok{NA}

\NormalTok{summary\_table }\OtherTok{\textless{}{-}} \FunctionTok{data.frame}\NormalTok{(}
\NormalTok{  Метод }\OtherTok{=} \FunctionTok{c}\NormalTok{(}\StringTok{"DCAC"}\NormalTok{, }\StringTok{"DB{-}SRA"}\NormalTok{, }\StringTok{"CMSY"}\NormalTok{, }\StringTok{"CC"}\NormalTok{),}
  
  \AttributeTok{MSY =} \FunctionTok{c}\NormalTok{(}
    \ConstantTok{NA}\NormalTok{,}
    \ControlFlowTok{if}\NormalTok{ (}\SpecialCharTok{!}\FunctionTok{is.na}\NormalTok{(dbsra\_msy)) }\FunctionTok{sprintf}\NormalTok{(}\StringTok{"\%.1f [\%.1f{-}\%.1f]"}\NormalTok{, }
\NormalTok{                                 dbsra\_msy,}
                                 \FunctionTok{quantile}\NormalTok{(dbsra\_results}\SpecialCharTok{$}\NormalTok{msy, }\FloatTok{0.25}\NormalTok{, }\AttributeTok{na.rm =} \ConstantTok{TRUE}\NormalTok{),}
                                 \FunctionTok{quantile}\NormalTok{(dbsra\_results}\SpecialCharTok{$}\NormalTok{msy, }\FloatTok{0.75}\NormalTok{, }\AttributeTok{na.rm =} \ConstantTok{TRUE}\NormalTok{))}
    \ControlFlowTok{else} \ConstantTok{NA}\NormalTok{,}
    \ControlFlowTok{if}\NormalTok{ (}\SpecialCharTok{!}\FunctionTok{is.null}\NormalTok{(cmsy\_results) }\SpecialCharTok{\&\&} \SpecialCharTok{!}\FunctionTok{is.na}\NormalTok{(cmsy\_msy)) }\FunctionTok{sprintf}\NormalTok{(}\StringTok{"\%.1f [\%.1f{-}\%.1f]"}\NormalTok{,}
\NormalTok{                                                         cmsy\_msy,}
                                                         \FunctionTok{quantile}\NormalTok{(cmsy\_results}\SpecialCharTok{$}\NormalTok{msy, }\FloatTok{0.025}\NormalTok{, }\AttributeTok{na.rm =} \ConstantTok{TRUE}\NormalTok{),}
                                                         \FunctionTok{quantile}\NormalTok{(cmsy\_results}\SpecialCharTok{$}\NormalTok{msy, }\FloatTok{0.975}\NormalTok{, }\AttributeTok{na.rm =} \ConstantTok{TRUE}\NormalTok{))}
    \ControlFlowTok{else} \ConstantTok{NA}\NormalTok{,}
    \ConstantTok{NA}
\NormalTok{  ),}
  
  \StringTok{\textasciigrave{}}\AttributeTok{B/Bmsy}\StringTok{\textasciigrave{}} \OtherTok{=} \FunctionTok{c}\NormalTok{(}
    \ConstantTok{NA}\NormalTok{,}
    \ControlFlowTok{if}\NormalTok{ (}\FunctionTok{exists}\NormalTok{(}\StringTok{"dbsra\_results"}\NormalTok{) }\SpecialCharTok{\&\&} \SpecialCharTok{!}\FunctionTok{is.null}\NormalTok{(dbsra\_results}\SpecialCharTok{$}\NormalTok{b\_bmsy)) }\FunctionTok{sprintf}\NormalTok{(}\StringTok{"\%.2f [\%.2f{-}\%.2f]"}\NormalTok{,}
                                                                       \FunctionTok{median}\NormalTok{(dbsra\_results}\SpecialCharTok{$}\NormalTok{b\_bmsy, }\AttributeTok{na.rm =} \ConstantTok{TRUE}\NormalTok{),}
                                                                       \FunctionTok{quantile}\NormalTok{(dbsra\_results}\SpecialCharTok{$}\NormalTok{b\_bmsy, }\FloatTok{0.25}\NormalTok{, }\AttributeTok{na.rm =} \ConstantTok{TRUE}\NormalTok{),}
                                                                       \FunctionTok{quantile}\NormalTok{(dbsra\_results}\SpecialCharTok{$}\NormalTok{b\_bmsy, }\FloatTok{0.75}\NormalTok{, }\AttributeTok{na.rm =} \ConstantTok{TRUE}\NormalTok{))}
    \ControlFlowTok{else} \ConstantTok{NA}\NormalTok{,}
    \ControlFlowTok{if}\NormalTok{ (}\SpecialCharTok{!}\FunctionTok{is.null}\NormalTok{(cmsy\_results)) }\FunctionTok{sprintf}\NormalTok{(}\StringTok{"\%.2f [\%.2f{-}\%.2f]"}\NormalTok{,}
                                      \FunctionTok{median}\NormalTok{(cmsy\_results}\SpecialCharTok{$}\NormalTok{bbmsy[, nyears], }\AttributeTok{na.rm =} \ConstantTok{TRUE}\NormalTok{),}
                                      \FunctionTok{quantile}\NormalTok{(cmsy\_results}\SpecialCharTok{$}\NormalTok{bbmsy[, nyears], }\FloatTok{0.025}\NormalTok{, }\AttributeTok{na.rm =} \ConstantTok{TRUE}\NormalTok{),}
                                      \FunctionTok{quantile}\NormalTok{(cmsy\_results}\SpecialCharTok{$}\NormalTok{bbmsy[, nyears], }\FloatTok{0.975}\NormalTok{, }\AttributeTok{na.rm =} \ConstantTok{TRUE}\NormalTok{))}
    \ControlFlowTok{else} \ConstantTok{NA}\NormalTok{,}
    \ConstantTok{NA}
\NormalTok{  ),}
  
  \StringTok{\textasciigrave{}}\AttributeTok{TAC рекомендация}\StringTok{\textasciigrave{}} \OtherTok{=} \FunctionTok{c}\NormalTok{(}
    \FunctionTok{sprintf}\NormalTok{(}\StringTok{"\%.1f"}\NormalTok{, dcac\_tac\_value),  }\CommentTok{\# Теперь точно будет 6.0}
    \ControlFlowTok{if}\NormalTok{ (}\SpecialCharTok{!}\FunctionTok{is.na}\NormalTok{(dbsra\_msy)) }\FunctionTok{sprintf}\NormalTok{(}\StringTok{"\%.1f"}\NormalTok{, dbsra\_msy }\SpecialCharTok{*} \FloatTok{0.8}\NormalTok{) }\ControlFlowTok{else} \ConstantTok{NA}\NormalTok{,}
    \ControlFlowTok{if}\NormalTok{ (}\SpecialCharTok{!}\FunctionTok{is.null}\NormalTok{(cmsy\_results) }\SpecialCharTok{\&\&} \SpecialCharTok{!}\FunctionTok{is.na}\NormalTok{(cmsy\_msy)) }\FunctionTok{sprintf}\NormalTok{(}\StringTok{"\%.1f"}\NormalTok{, cmsy\_msy }\SpecialCharTok{*} \FloatTok{0.8}\NormalTok{) }\ControlFlowTok{else} \ConstantTok{NA}\NormalTok{,}
    \FunctionTok{sprintf}\NormalTok{(}\StringTok{"\%.1f"}\NormalTok{, cc\_tac)}
\NormalTok{  ),}
  
\NormalTok{  Примечание }\OtherTok{=} \FunctionTok{c}\NormalTok{(}
    \StringTok{"Простая коррекция среднего улова"}\NormalTok{,}
    \StringTok{"80\% от MSY (предосторожность)"}\NormalTok{,}
    \StringTok{"80\% от MSY (предосторожность)"}\NormalTok{,}
    \StringTok{"Средний исторический улов за последние 5 лет"}
\NormalTok{  ),}
  
  \AttributeTok{stringsAsFactors =} \ConstantTok{FALSE}
\NormalTok{)}

\FunctionTok{print}\NormalTok{(summary\_table, }\AttributeTok{row.names =} \ConstantTok{FALSE}\NormalTok{)}


\CommentTok{\# ======================= РЕКОМЕНДАЦИИ =======================}
\FunctionTok{cat}\NormalTok{(}\StringTok{"}\SpecialCharTok{\textbackslash{}n}\StringTok{========== ИТОГОВЫЕ РЕКОМЕНДАЦИИ ==========}\SpecialCharTok{\textbackslash{}n\textbackslash{}n}\StringTok{"}\NormalTok{)}

\CommentTok{\# Анализ консенсуса методов}
\NormalTok{all\_tacs }\OtherTok{\textless{}{-}} \FunctionTok{c}\NormalTok{(}
  \FunctionTok{median}\NormalTok{(DCAC\_results}\SpecialCharTok{@}\NormalTok{TAC, }\AttributeTok{na.rm =} \ConstantTok{TRUE}\NormalTok{),  }\CommentTok{\# Правильное извлечение TAC из S4 объекта}
  \FunctionTok{median}\NormalTok{(dbsra\_results}\SpecialCharTok{$}\NormalTok{msy, }\AttributeTok{na.rm =} \ConstantTok{TRUE}\NormalTok{) }\SpecialCharTok{*} \FloatTok{0.8}\NormalTok{,}
  \FunctionTok{ifelse}\NormalTok{(}\SpecialCharTok{!}\FunctionTok{is.null}\NormalTok{(cmsy\_results), }\FunctionTok{median}\NormalTok{(cmsy\_results}\SpecialCharTok{$}\NormalTok{msy, }\AttributeTok{na.rm =} \ConstantTok{TRUE}\NormalTok{) }\SpecialCharTok{*} \FloatTok{0.8}\NormalTok{, }\ConstantTok{NA}\NormalTok{)}
\NormalTok{)}

\CommentTok{\# Удаляем NA значения}
\NormalTok{all\_tacs }\OtherTok{\textless{}{-}} \FunctionTok{na.omit}\NormalTok{(all\_tacs)}

\NormalTok{consensus\_tac }\OtherTok{\textless{}{-}} \FunctionTok{median}\NormalTok{(all\_tacs)}
\NormalTok{tac\_range }\OtherTok{\textless{}{-}} \FunctionTok{range}\NormalTok{(all\_tacs)}

\FunctionTok{cat}\NormalTok{(}\StringTok{"АНАЛИЗ РЕЗУЛЬТАТОВ:}\SpecialCharTok{\textbackslash{}n}\StringTok{"}\NormalTok{)}
\FunctionTok{cat}\NormalTok{(}\FunctionTok{strrep}\NormalTok{(}\StringTok{"{-}"}\NormalTok{, }\DecValTok{50}\NormalTok{), }\StringTok{"}\SpecialCharTok{\textbackslash{}n}\StringTok{"}\NormalTok{)}
\FunctionTok{cat}\NormalTok{(}\FunctionTok{sprintf}\NormalTok{(}\StringTok{"Консенсус TAC (медиана): \%.1f тыс. т}\SpecialCharTok{\textbackslash{}n}\StringTok{"}\NormalTok{, consensus\_tac))}
\FunctionTok{cat}\NormalTok{(}\FunctionTok{sprintf}\NormalTok{(}\StringTok{"Диапазон рекомендаций: \%.1f {-} \%.1f тыс. т}\SpecialCharTok{\textbackslash{}n}\StringTok{"}\NormalTok{, tac\_range[}\DecValTok{1}\NormalTok{], tac\_range[}\DecValTok{2}\NormalTok{]))}
\FunctionTok{cat}\NormalTok{(}\FunctionTok{sprintf}\NormalTok{(}\StringTok{"Текущий улов (2024): \%.1f тыс. т}\SpecialCharTok{\textbackslash{}n}\StringTok{"}\NormalTok{, }\FunctionTok{tail}\NormalTok{(Catch, }\DecValTok{1}\NormalTok{)))}

\ControlFlowTok{if}\NormalTok{ (consensus\_tac }\SpecialCharTok{\textless{}} \FunctionTok{tail}\NormalTok{(Catch, }\DecValTok{1}\NormalTok{) }\SpecialCharTok{*} \FloatTok{0.8}\NormalTok{) \{}
  \FunctionTok{cat}\NormalTok{(}\StringTok{"}\SpecialCharTok{\textbackslash{}n}\StringTok{⚠️ РЕКОМЕНДАЦИЯ: Существенно СНИЗИТЬ промысловое усилие}\SpecialCharTok{\textbackslash{}n}\StringTok{"}\NormalTok{)}
\NormalTok{\} }\ControlFlowTok{else} \ControlFlowTok{if}\NormalTok{ (consensus\_tac }\SpecialCharTok{\textless{}} \FunctionTok{tail}\NormalTok{(Catch, }\DecValTok{1}\NormalTok{)) \{}
  \FunctionTok{cat}\NormalTok{(}\StringTok{"}\SpecialCharTok{\textbackslash{}n}\StringTok{⚠️ РЕКОМЕНДАЦИЯ: Умеренно СНИЗИТЬ промысловое усилие}\SpecialCharTok{\textbackslash{}n}\StringTok{"}\NormalTok{)}
\NormalTok{\} }\ControlFlowTok{else}\NormalTok{ \{}
  \FunctionTok{cat}\NormalTok{(}\StringTok{"}\SpecialCharTok{\textbackslash{}n}\StringTok{✓ РЕКОМЕНДАЦИЯ: Текущий уровень промысла приемлем}\SpecialCharTok{\textbackslash{}n}\StringTok{"}\NormalTok{)}
\NormalTok{\}}

\CommentTok{\# Оценка состояния запаса}
\ControlFlowTok{if}\NormalTok{ (}\SpecialCharTok{!}\FunctionTok{is.null}\NormalTok{(cmsy\_results)) \{}
\NormalTok{  bbmsy\_current }\OtherTok{\textless{}{-}} \FunctionTok{median}\NormalTok{(cmsy\_results}\SpecialCharTok{$}\NormalTok{bbmsy[, nyears])}
  \ControlFlowTok{if}\NormalTok{ (bbmsy\_current }\SpecialCharTok{\textless{}} \FloatTok{0.5}\NormalTok{) \{}
    \FunctionTok{cat}\NormalTok{(}\StringTok{"}\SpecialCharTok{\textbackslash{}n}\StringTok{🔴 СОСТОЯНИЕ ЗАПАСА: Критическое (B/Bmsy \textless{} 0.5)}\SpecialCharTok{\textbackslash{}n}\StringTok{"}\NormalTok{)}
    \FunctionTok{cat}\NormalTok{(}\StringTok{"   Необходимы срочные меры по восстановлению}\SpecialCharTok{\textbackslash{}n}\StringTok{"}\NormalTok{)}
\NormalTok{  \} }\ControlFlowTok{else} \ControlFlowTok{if}\NormalTok{ (bbmsy\_current }\SpecialCharTok{\textless{}} \DecValTok{1}\NormalTok{) \{}
    \FunctionTok{cat}\NormalTok{(}\StringTok{"}\SpecialCharTok{\textbackslash{}n}\StringTok{🟡 СОСТОЯНИЕ ЗАПАСА: Ниже оптимального (0.5 \textless{} B/Bmsy \textless{} 1)}\SpecialCharTok{\textbackslash{}n}\StringTok{"}\NormalTok{)}
    \FunctionTok{cat}\NormalTok{(}\StringTok{"   Рекомендуется осторожный подход}\SpecialCharTok{\textbackslash{}n}\StringTok{"}\NormalTok{)}
\NormalTok{  \} }\ControlFlowTok{else}\NormalTok{ \{}
    \FunctionTok{cat}\NormalTok{(}\StringTok{"}\SpecialCharTok{\textbackslash{}n}\StringTok{🟢 СОСТОЯНИЕ ЗАПАСА: Хорошее (B/Bmsy \textgreater{} 1)}\SpecialCharTok{\textbackslash{}n}\StringTok{"}\NormalTok{)}
    \FunctionTok{cat}\NormalTok{(}\StringTok{"   Возможен устойчивый промысел}\SpecialCharTok{\textbackslash{}n}\StringTok{"}\NormalTok{)}
\NormalTok{  \}}
\NormalTok{\}}

\FunctionTok{cat}\NormalTok{(}\StringTok{"}\SpecialCharTok{\textbackslash{}n}\StringTok{ПРЕДОСТОРОЖНЫЙ ПОДХОД:}\SpecialCharTok{\textbackslash{}n}\StringTok{"}\NormalTok{)}
\FunctionTok{cat}\NormalTok{(}\FunctionTok{strrep}\NormalTok{(}\StringTok{"{-}"}\NormalTok{, }\DecValTok{50}\NormalTok{), }\StringTok{"}\SpecialCharTok{\textbackslash{}n}\StringTok{"}\NormalTok{)}
\NormalTok{precautionary\_tac }\OtherTok{\textless{}{-}}\NormalTok{ consensus\_tac }\SpecialCharTok{*} \FloatTok{0.9}
\FunctionTok{cat}\NormalTok{(}\FunctionTok{sprintf}\NormalTok{(}\StringTok{"Рекомендуемый TAC с учетом неопределенности: \%.1f тыс. т}\SpecialCharTok{\textbackslash{}n}\StringTok{"}\NormalTok{, }
\NormalTok{            precautionary\_tac))}
\FunctionTok{cat}\NormalTok{(}\StringTok{"(90\% от консенсусной оценки)}\SpecialCharTok{\textbackslash{}n}\StringTok{"}\NormalTok{)}

\CommentTok{\# ======================= СОХРАНЕНИЕ РЕЗУЛЬТАТОВ =======================}

\FunctionTok{cat}\NormalTok{(}\StringTok{"}\SpecialCharTok{\textbackslash{}n}\StringTok{========== СОХРАНЕНИЕ РЕЗУЛЬТАТОВ ==========}\SpecialCharTok{\textbackslash{}n}\StringTok{"}\NormalTok{)}

\CommentTok{\# Сохранение всех результатов}
\NormalTok{catch\_only\_results }\OtherTok{\textless{}{-}} \FunctionTok{list}\NormalTok{(}
  \AttributeTok{data =}\NormalTok{ catch\_df,}
  \AttributeTok{dcac =}\NormalTok{ DCAC\_results,}
  \AttributeTok{dbsra =}\NormalTok{ dbsra\_results,}
  \AttributeTok{cmsy =}\NormalTok{ cmsy\_results,}
  \AttributeTok{cc =}\NormalTok{ CC\_variants,}
  \AttributeTok{summary\_table =}\NormalTok{ summary\_table,}
  \AttributeTok{recommendations =} \FunctionTok{list}\NormalTok{(}
    \AttributeTok{consensus\_tac =}\NormalTok{ consensus\_tac,}
    \AttributeTok{precautionary\_tac =}\NormalTok{ precautionary\_tac,}
    \AttributeTok{current\_catch =} \FunctionTok{tail}\NormalTok{(Catch, }\DecValTok{1}\NormalTok{)}
\NormalTok{  )}
\NormalTok{)}

\FunctionTok{saveRDS}\NormalTok{(catch\_only\_results, }\StringTok{"catch\_only\_analysis.rds"}\NormalTok{)}
\FunctionTok{cat}\NormalTok{(}\StringTok{"✓ Результаты сохранены в \textquotesingle{}catch\_only\_analysis.rds\textquotesingle{}}\SpecialCharTok{\textbackslash{}n}\StringTok{"}\NormalTok{)}

\CommentTok{\# Сохранение графиков}
\FunctionTok{pdf}\NormalTok{(}\StringTok{"catch\_only\_plots.pdf"}\NormalTok{, }\AttributeTok{width =} \DecValTok{12}\NormalTok{, }\AttributeTok{height =} \DecValTok{10}\NormalTok{)}
\FunctionTok{print}\NormalTok{(p1)  }\CommentTok{\# Временной ряд уловов}
\ControlFlowTok{if}\NormalTok{ (}\FunctionTok{exists}\NormalTok{(}\StringTok{"p2"}\NormalTok{)) }\FunctionTok{print}\NormalTok{(p2)  }\CommentTok{\# Сравнение MSY}
\ControlFlowTok{if}\NormalTok{ (}\FunctionTok{exists}\NormalTok{(}\StringTok{"p3"}\NormalTok{)) }\FunctionTok{print}\NormalTok{(p3)  }\CommentTok{\# Траектории B/Bmsy}
\ControlFlowTok{if}\NormalTok{ (}\FunctionTok{exists}\NormalTok{(}\StringTok{"p4"}\NormalTok{)) }\FunctionTok{print}\NormalTok{(p4)  }\CommentTok{\# Диаграмма Кобе}
\FunctionTok{print}\NormalTok{(p5)  }\CommentTok{\# Сравнение TAC}
\FunctionTok{dev.off}\NormalTok{()}
\FunctionTok{cat}\NormalTok{(}\StringTok{"✓ Графики сохранены в \textquotesingle{}catch\_only\_plots.pdf\textquotesingle{}}\SpecialCharTok{\textbackslash{}n}\StringTok{"}\NormalTok{)}

\CommentTok{\# Экспорт таблицы}
\FunctionTok{write.csv}\NormalTok{(summary\_table, }\StringTok{"catch\_only\_summary.csv"}\NormalTok{, }\AttributeTok{row.names =} \ConstantTok{FALSE}\NormalTok{)}
\FunctionTok{cat}\NormalTok{(}\StringTok{"✓ Таблица сохранена в \textquotesingle{}catch\_only\_summary.csv\textquotesingle{}}\SpecialCharTok{\textbackslash{}n}\StringTok{"}\NormalTok{)}

\FunctionTok{cat}\NormalTok{(}\StringTok{"}\SpecialCharTok{\textbackslash{}n}\StringTok{=============== АНАЛИЗ ЗАВЕРШЕН ===============}\SpecialCharTok{\textbackslash{}n}\StringTok{"}\NormalTok{)}
\end{Highlighting}
\end{Shaded}

\begin{Shaded}
\begin{Highlighting}[]
\SpecialCharTok{\textgreater{}} \FunctionTok{print}\NormalTok{(summary\_table, }\AttributeTok{row.names =} \ConstantTok{FALSE}\NormalTok{)}
\end{Highlighting}
\end{Shaded}

\begin{longtable}[]{@{}
  >{\raggedright\arraybackslash}p{(\linewidth - 8\tabcolsep) * \real{0.1351}}
  >{\raggedright\arraybackslash}p{(\linewidth - 8\tabcolsep) * \real{0.1622}}
  >{\raggedright\arraybackslash}p{(\linewidth - 8\tabcolsep) * \real{0.1757}}
  >{\raggedright\arraybackslash}p{(\linewidth - 8\tabcolsep) * \real{0.2027}}
  >{\raggedright\arraybackslash}p{(\linewidth - 8\tabcolsep) * \real{0.3243}}@{}}
\toprule\noalign{}
\begin{minipage}[b]{\linewidth}\raggedright
Метод
\end{minipage} & \begin{minipage}[b]{\linewidth}\raggedright
MSY
\end{minipage} & \begin{minipage}[b]{\linewidth}\raggedright
B.Bmsy
\end{minipage} & \begin{minipage}[b]{\linewidth}\raggedright
TAC
\end{minipage} & \begin{minipage}[b]{\linewidth}\raggedright
Примечание
\end{minipage} \\
\midrule\noalign{}
\endhead
\bottomrule\noalign{}
\endlastfoot
DCAC & & & 6.0 & Простая коррекция среднего улова \\
DB-SRA & 29.4 {[}23.0-38.1{]} & 1.73 {[}1.48-1.82{]} & 23.58\% от MSY &
Предосторожность \\
CMSY & 17.7 {[}14.1-20.8{]} & 1.11 {[}0.62-1.35{]} & 14.28\% от MSY &
Предосторожность \\
CC & & & 11.6 & Средний исторический улов за последние 5 лет \\
\end{longtable}

\section{Результаты применения
моделей}\label{ux440ux435ux437ux443ux43bux44cux442ux430ux442ux44b-ux43fux440ux438ux43cux435ux43dux435ux43dux438ux44f-ux43cux43eux434ux435ux43bux435ux439}

На основе данных временного ряда уловов (2005-2024 гг.) с пиком в 35
тыс. т и текущим уровнем 12 тыс. т были получены следующие результаты:

\textbf{1. DCAC (Depletion-Corrected Average Catch)}

\begin{itemize}
\item
  \textbf{Результат:} Рекомендуемый TAC = \textbf{6.0 тыс. т}
\item
  \textbf{Суть метода:} Самый простой и консервативный метод. Берет
  средний исторический улов (16.6 тыс. т) и применяет к нему поправочный
  коэффициент, основанный на предполагаемом уровне истощения запаса (был
  задан на уровне 30\%) и естественной смертности (M=0.2). Формула:
  \texttt{TAC\ =\ Avg\_Catch\ *\ Depletion\ *\ (1\ +\ M)}.
\item
  \textbf{Плюсы:}

  \begin{itemize}
  \item
    Чрезвычайно прост для понимания и расчета.
  \item
    Крайне предосторожен. В ситуации высокой неопределенности
    предотвращает дальнейшее истощение.
  \end{itemize}
\item
  \textbf{Минусы:}

  \begin{itemize}
  \item
    Сильно зависит от \textbf{априорного предположения о depletion
    (\emph{B/B₀})}. Если оценка истощения неточна, результат будет
    сильно смещен.
  \item
    Не дает оценок ключевых параметров популяции (\emph{r}, \emph{K},
    \emph{MSY}).
  \item
    Игнорирует динамику уловов (тренды), учитывая лишь среднее значение.
  \end{itemize}
\end{itemize}

\textbf{2. DB-SRA (Depletion-Based Stock Reduction Analysis)}

\begin{itemize}
\item
  \textbf{Результаты:}

  \begin{itemize}
  \item
    \emph{MSY}= 29.4 {[}23.0 - 38.1{]} тыс. т
  \item
    \emph{B}/\emph{B\textsubscript{msy}} (2024) = 1.73 (запас выше
    целевого уровня)
  \item
    TAC (80\% от \emph{MSY}) = 23.5 тыс. т
  \end{itemize}
\item
  \textbf{Суть метода:} Стохастический метод, который итеративно
  подбирает параметры \emph{\texttt{r}} (темп роста) и \emph{\texttt{K}}
  (емкость среды), чтобы смоделировать траекторию популяции под
  воздействием известных уловов. Исходит из того, что запас не мог быть
  полностью истощен за период промысла.
\item
  \textbf{Плюсы:}

  \begin{itemize}
  \item
    Учитывает весь временной ряд уловов, а не только среднее значение.
  \item
    Прямо моделирует траекторию биомассы.
  \item
    Дает полное распределение вероятностей для параметров (\emph{r},
    \emph{K}, \emph{MSY}) и текущего состояния запаса
    (\emph{B}/\emph{B\textsubscript{msy}}).
  \end{itemize}
\item
  \textbf{Минусы:}

  \begin{itemize}
  \item
    Дает самые оптимистичные оценки среди всех методов. Это часто
    происходит из-за широких априорных допущений на \emph{r} и \emph{K}.
  \item
    Может быть вычислительно затратным.
  \item
    Сильно зависит от корректности априорных диапазонов для \emph{r} и
    начального истощения.
  \end{itemize}
\end{itemize}

\textbf{3. CMSY (Catch-MSY)}

\begin{itemize}
\item
  \textbf{Результаты:}

  \begin{itemize}
  \item
    \emph{MSY} = 17.7 {[}14.1 - 20.8{]} тыс. т
  \item
    \emph{B}/\emph{B\textsubscript{msy}} (2024) = 1.11 (запас близок к
    целевому уровню)
  \item
    \emph{F}/\emph{F\textsubscript{msy}} (2024) = 0.61 (промысловое
    усилие ниже переловного)
  \item
    TAC (80\% от \emph{MSY}) = 14.2 тыс. т
  \end{itemize}
\item
  \textbf{Суть метода:} Байесовский метод, который перебирает тысячи пар
  \emph{r}-\emph{K} и отсеивает те, которые биологически невозможны
  (например, когда улов превышает возможную продуктивность). Оставшиеся
  ``жизнеспособные'' пары образуют распределение вероятностей для
  \emph{MSY} и текущего состояния запаса.
\item
  \textbf{Плюсы:}

  \begin{itemize}
  \item
    Надежный и популярный метод с хорошей теоретической базой.
  \item
    Учитывает тренды в уловах для определения правдоподобных диапазонов
    начального и конечного истощения.
  \item
    Дает оценки как состояния запаса (\emph{B}/\emph{Bmsy}), так и
    промысла (\emph{F}/\emph{Fmsy}), что позволяет построить диаграмму
    Кобе.
  \item
    Менее чувствителен к точному значению depletion, чем DCAC, но более
    чувствителен, чем DB-SRA.
  \end{itemize}
\item
  \textbf{Минусы:}

  \begin{itemize}
  \item
    Может иметь низкую ``проходимость'' (в данном случае лишь 2.1\%
    симуляций были признаны жизнеспособными), что требует большого числа
    итераций.
  \item
    Результаты могут быть чувствительны к выбору априорных диапазонов
    для \emph{r} (на основе resilience).
  \end{itemize}
\end{itemize}

\textbf{4. CC (Constant Catch)}

\begin{itemize}
\item
  \textbf{Результат:} Рекомендуемый TAC = \textbf{11.6 тыс. т}
\item
  \textbf{Суть метода:} Рекомендует постоянный улов на уровне среднего
  улова за последние 5 лет.
\item
  \textbf{Плюсы:} Простота.
\item
  \textbf{Минусы:} Слепое использование недавней истории может привести
  к рекомендациям, не связанным с реальным состоянием запаса (например,
  если последние 5 лет были периодом перелова).
\end{itemize}

\subsection{Сравнительная таблица
результатов}\label{ux441ux440ux430ux432ux43dux438ux442ux435ux43bux44cux43dux430ux44f-ux442ux430ux431ux43bux438ux446ux430-ux440ux435ux437ux443ux43bux44cux442ux430ux442ux43eux432}

\begin{longtable}[]{@{}
  >{\raggedright\arraybackslash}p{(\linewidth - 10\tabcolsep) * \real{0.1667}}
  >{\raggedright\arraybackslash}p{(\linewidth - 10\tabcolsep) * \real{0.1667}}
  >{\raggedright\arraybackslash}p{(\linewidth - 10\tabcolsep) * \real{0.1667}}
  >{\raggedright\arraybackslash}p{(\linewidth - 10\tabcolsep) * \real{0.1667}}
  >{\raggedright\arraybackslash}p{(\linewidth - 10\tabcolsep) * \real{0.1667}}
  >{\raggedright\arraybackslash}p{(\linewidth - 10\tabcolsep) * \real{0.1667}}@{}}
\toprule\noalign{}
\begin{minipage}[b]{\linewidth}\raggedright
Метод
\end{minipage} & \begin{minipage}[b]{\linewidth}\raggedright
Рекомендуемый TAC (тыс. т)
\end{minipage} & \begin{minipage}[b]{\linewidth}\raggedright
Оценка MSY (тыс. т)
\end{minipage} & \begin{minipage}[b]{\linewidth}\raggedright
Оценка B/Bmsy
\end{minipage} & \begin{minipage}[b]{\linewidth}\raggedright
Преимущества
\end{minipage} & \begin{minipage}[b]{\linewidth}\raggedright
Недостатки
\end{minipage} \\
\midrule\noalign{}
\endhead
\bottomrule\noalign{}
\endlastfoot
\textbf{DCAC} & \textbf{6.0} & - & - & Крайняя предосторожность,
простота & Сильная зависимость от априорного depletion, не дает MSY \\
\textbf{DB-SRA} & \textbf{23.5} & 29.4 & 1.73 & Учет динамики уловов,
моделирование биомассы & Часто излишне оптимистичный, широкие априоры \\
\textbf{CMSY} & \textbf{14.2} & 17.7 & 1.11 & Баланс предосторожности и
оптимизма, оценка B/Bmsy и F/Fmsy & Низкая ``проходимость'',
чувствительность к априорам на r \\
\textbf{CC} & \textbf{11.6} & - & - & Простота & Слепое использование
данных, не отражает состояние запаса \\
\end{longtable}

\subsection{Причина различий в
оценках}\label{ux43fux440ux438ux447ux438ux43dux430-ux440ux430ux437ux43bux438ux447ux438ux439-ux432-ux43eux446ux435ux43dux43aux430ux445}

Разброс в оценках TAC от \textbf{6.0} до \textbf{23.5} тыс. т является
\textbf{прямым следствием различных математических допущений и априорной
информации}, заложенной в каждый метод.

\begin{enumerate}
\def\labelenumi{\arabic{enumi}.}
\item
  \textbf{Разные фундаментальные подходы:}

  \begin{itemize}
  \item
    \textbf{DCAC} исходит из текущего \emph{состояния истощения}.
  \item
    \textbf{DB-SRA} и \textbf{CMSY} исходят из \emph{продуктивности}
    популяции (параметры \emph{r} и \emph{K}), чтобы найти \emph{MSY}, а
    затем определяют состояние запаса.
  \end{itemize}
\item
  \textbf{Чувствительность к разным априорным предположениям:}

  \begin{itemize}
  \item
    \textbf{DCAC} чувствителен к \texttt{depletion}.
  \item
    \textbf{CMSY} чувствителен к \texttt{resilience} (что задает широкий
    априорный диапазон для \emph{r}).
  \item
    \textbf{DB-SRA} чувствителен к априорным диапазонам как для
    \emph{r}, так и для \texttt{depletion}.
  \end{itemize}
\item
  \textbf{Интерпретация истории уловов:}\\
  Резкий рост уловов с последующим спадом может интерпретироваться
  по-разному:

  \begin{itemize}
  \item
    \textbf{CMSY} видит в этом признаки перелова и оценивает запас как
    близкий к Bmsy.
  \item
    \textbf{DB-SRA}, с его более широкими априорами, может счесть этот
    спад не столь критичным и оценить запас как значительно превышающий
    Bmsy.
  \end{itemize}
\item
  \textbf{Уровень предосторожности:}\\
  DCAC по дизайну самый предосторожный. DB-SRA часто оказывается самым
  оптимистичным. CMSY занимает промежуточное положение.
\end{enumerate}

\subsection{Шаги по последующему улучшению
оценок}\label{ux448ux430ux433ux438-ux43fux43e-ux43fux43eux441ux43bux435ux434ux443ux44eux449ux435ux43cux443-ux443ux43bux443ux447ux448ux435ux43dux438ux44e-ux43eux446ux435ux43dux43eux43a}

Чтобы уменьшить неопределенность и выбрать наиболее обоснованное
управленческое решение, необходимо:

\begin{enumerate}
\def\labelenumi{\arabic{enumi}.}
\item
  \textbf{Решить вопрос с априорными допущениями (ключевой шаг!):}

  \begin{itemize}
  \item
    \textbf{Провести анализ чувствительности.} Запустить все модели,
    систематически меняя ключевые априорные распределения (например,
    \texttt{depletion} для DCAC, диапазоны \texttt{r} для CMSY и
    DB-SRA).
  \item
    \textbf{Обосновать априорные распределения параметров на основе
    биологии вида.} Использовать данные по близким видам-аналогам
    (например, из базы данных FishBase) для задания более реалистичных и
    узких диапазонов для \emph{r}, \emph{M}, \emph{L\textsubscript{inf}}
    и т.д. Это самое слабое место DLM-анализа.
  \end{itemize}
\item
  \textbf{Интегрировать дополнительные данные (переход на Tier 2-3):}

  \begin{itemize}
  \item
    \textbf{Данные CPUE:} Даже единственная точка данных об
    относительной численности (например, от индекса стандартизированного
    CPUE) может быть использована для калибровки моделей (например, в
    методе \texttt{BSM} или \texttt{SPiCT}) и резкого сужения
    неопределенности.
  \item
    \textbf{Данные о размере/возрасте:} Если есть выборочные данные о
    длине или возрасте, можно применить методы \textbf{LBB}
    (Length-Based Bayesian) или \textbf{LBSPR} для независимой оценки
    \texttt{B/Bmsy}, \texttt{F/Fmsy} и \texttt{M}. Эти оценки затем
    можно использовать для калибровки catch-only методов.
  \end{itemize}
\item
  \textbf{Провести взвешивание моделей (Model Averaging):}

  \begin{itemize}
  \tightlist
  \item
    После анализа чувствительности и, если возможно, калибровки, не
    обязательно выбирать одну модель. Можно использовать
    \textbf{средневзвешенное значение} рекомендаций TAC, где вес модели
    определяется ее правдоподобием (например, на основе AIC) или
    соответствием независимым данным.
  \end{itemize}
\item
  \textbf{Принять предосторожный подход:}

  \begin{itemize}
  \tightlist
  \item
    В условиях сохраняющейся неопределенности разумно принять
    \textbf{консенсусную, но предосторожную оценку}. В данном случае
    консенсус медианных оценок моделей (DCAC, DB-SRA, CMSY) составляет
    \textasciitilde18.8 тыс. т. Однако, учитывая, что две модели (DCAC и
    CMSY) указывают на необходимость более осторожного подхода, а
    текущий улов составляет 12 тыс. т, \textbf{рекомендация TAC на
    уровне 12-14 тыс. т} (с последующим мониторингом) выглядит наиболее
    обоснованной и соответствующей принципу предосторожности.
  \end{itemize}
\end{enumerate}

\textbf{Вывод:} Разброс в оценках --- это не ошибка, а отражение
неопределенности. Задача аналитика --- не просто запустить скрипт, а
сузить эту неопределенность за счет обоснования априорных распределений
параметров и интеграции всех доступных данных, а затем сформулировать
управленческую рекомендацию, явно учитывающую оставшуюся
неопределенность.

\bookmarksetup{startatroot}

\chapter{Оценка параметров роста гидробионтов методом
ELEFAN}\label{ux43eux446ux435ux43dux43aux430-ux43fux430ux440ux430ux43cux435ux442ux440ux43eux432-ux440ux43eux441ux442ux430-ux433ux438ux434ux440ux43eux431ux438ux43eux43dux442ux43eux432-ux43cux435ux442ux43eux434ux43eux43c-elefan}

\section{Введение}\label{ux432ux432ux435ux434ux435ux43dux438ux435-15}

Это практическое занятие --- про то, как извлекать параметры роста
(\emph{L∞}, \emph{K}) из рядов распределений по длине, когда возраст
недоступен. Мы используем ELEFAN (Electronic Length Frequency Analysis)
--- алгоритм, который по последовательности размерных срезов находит
модальные «гребни» когорт и подбирает сезонную кривую фон Берталанфи,
проходящую через них. Метод исторически разработан для тропических рыб,
но применим и к холодноводным видам, если помнить об ограничениях: рост
у ракообразных дискретный (линьки), а терминальная линька нарушает
гладкую асимптотику VBGF. Поэтому \emph{K} интерпретируем как
«эффективный» темп роста, а \emph{L∞} верифицируем правым хвостом
промысла.

Как отмечает Паули (1987), традиционные методы ихтиологии, разработанные
в Северной Атлантике (например, для трески и сельди), часто опирались на
определение возраста по кольцам на чешуе или отолитах. Однако в
тропиках, где сезонные колебания температуры и других факторов среды
выражены слабее, годовые кольца формируются нечётко или отсутствуют
вовсе. Кроме того, многие важные промысловые объекты --- например, крабы
и креветки --- вообще не имеют твёрдых структур, пригодных для
определения возраста. В таких условиях распределения по длине становятся
основным, а зачастую и единственным источником информации о росте,
смертности и пополнении популяций. Более того, как подчёркивает Паули,
многие биологические и промысловые процессы (хищничество, селективность
орудий лова, товарная стоимость) зависят именно от размера, а не от
возраста. Это делает основанный на длине (length-based) подход не просто
вынужденной мерой, а биологически обоснованной альтернативой основанным
на возрасте (age-based методам).

ELEFAN --- это не одна программа, а целая система взаимосвязанных
программ (ELEFAN 0--IV), предназначенных для:

ELEFAN I: Оценка параметров роста (\emph{L∞}, \emph{K}, \emph{C},
\emph{ts}) по сезонной версии уравнения Берталанфи.

ELEFAN II: Оценка смертности (\emph{Z}, \emph{M}, \emph{F}),
вероятностей поимки и сезонности пополнения.

ELEFAN III: Виртуальный анализ популяций (VPA) на основе данных по
длине.

ELEFAN IV: Оценка естественной смертности (\emph{M}) с учётом
селективности орудий лова.

Сердце системы --- ELEFAN I. В отличие от классических графических
методов, он использует алгоритм ``высокочастотной фильтрации'' для
автоматического выявления пиков (мод) в распределениях по длине, а затем
подбирает такую кривую роста, которая ``объясняет'' максимальное
количество этих пиков во времени. Ключевой показатель --- ESP/ASP
(Explained Sum of Peaks / Available Sum of Peaks) --- позволяет
объективно выбрать наилучший набор параметров.

Особенно важно, что ELEFAN I не требует предварительного знания возраста
или числа поколений в выборке --- он выводит их автоматически, что
делает его мощным инструментом для анализа плохо изученных видов.

Что делаем на занятии: Сбор и визуализация данных: - Строим
ridgeline-графики (ggridges) распределений ширины карапакса по годам
отдельно для съёмки и промысла (самцы). - Формируем LFQ-объект (бининги
по 2 мм; стабильные месяцы съёмки).

Оценка параметров ростовой кривой: - Запускаем ELEFAN\_SA (seasonal VBGF
Somers: \emph{L∞}, \emph{K}, \emph{C}, \emph{ts}, \emph{t\_anchor}) с
реалистичными ограничениями для опилио. - Инициализацию \emph{L∞} берём
из правого хвоста промысла (Q99/0.95) и «прижимаем» к 155--165 мм.

Сдерживаем сезонность (\emph{C}) и не допускаем слишком малых \emph{K}.

Диагностика и валидация: - Смотрим «реструктурированные» rcounts ---
кривая должна проходить через зелёные гребни. - Сравниваем \emph{L∞} с
промысловым хвостом по плотностям. - Проверяем здравый смысл по
\emph{φ′} и по времени достижения 95\% от \emph{L∞}: \emph{t95} ≈
−ln(1−0.95)/\emph{K} (для опилио целевой ориентир ≈ 12--15 лет).

Выходы занятия: - Оценённые параметры роста (CSV), графики ridges,
плотностей и сезонной VBGF, rcounts.

Минимальные требования к данным - Поособные длины (ширина карапакса,
мм), дата (год/месяц), пол, источник (съёмка/промысел). - Стабильные
месяцы съёмки по годам, достаточный объём выборки в годовых срезах. -
Для сопоставимости с промыслом --- анализируем самцов; бины 2 мм
(≈1--2\% от ожидаемого \emph{L∞}).

\textbf{Для работы скрипта
(}\href{https://mombus.github.io/cRab/data/Bertalanffy_ELEFAN.R}{полный
скрипт}\textbf{):}

\begin{enumerate}
\def\labelenumi{\arabic{enumi}.}
\tightlist
\item
  Скачайте файлы данных
  (``\href{https://mombus.github.io/cRab/data/SURVEYDATA.csv}{SURVEYDATA.csv}''
  и
  ``\href{https://mombus.github.io/cRab/data/FISHERYDATA.csv}{FISHERYDATA.csv}'')
\item
  Установите рабочую директорию в setwd() 3.Установите необходимые
  пакеты (tidyverse, lubridate, ggridges, TropFishR).
\end{enumerate}

\begin{Shaded}
\begin{Highlighting}[]
\CommentTok{\# =============================================================================}
\CommentTok{\# НАСТРОЙКА СРЕДЫ И ЗАГРУЗКА ПАКЕТОВ}
\CommentTok{\# =============================================================================}

\CommentTok{\# Установка рабочей директории {-} указываем путь к папке с данными}
\FunctionTok{setwd}\NormalTok{(}\StringTok{"C:/LBM/"}\NormalTok{)}

\CommentTok{\# Тихая загрузка пакетов (без вывода сообщений)}
\FunctionTok{suppressPackageStartupMessages}\NormalTok{(\{}
  \FunctionTok{library}\NormalTok{(tidyverse)    }\CommentTok{\# Набор пакетов для обработки и визуализации данных}
  \FunctionTok{library}\NormalTok{(lubridate)    }\CommentTok{\# Работа с датами}
  \FunctionTok{library}\NormalTok{(ggridges)     }\CommentTok{\# Риджлайн{-}графики (горные хребты)}
  \FunctionTok{library}\NormalTok{(TropFishR)    }\CommentTok{\# Пакет для анализа данных рыболовства}
\NormalTok{\})}

\CommentTok{\# =============================================================================}
\CommentTok{\# 1. ЗАГРУЗКА И ПРЕДВАРИТЕЛЬНАЯ ОБРАБОТКА ДАННЫХ}
\CommentTok{\# =============================================================================}

\CommentTok{\# Загрузка данных из CSV{-}файлов с разделителем ";"}
\NormalTok{SURVEY  }\OtherTok{\textless{}{-}} \FunctionTok{read.csv}\NormalTok{(}\StringTok{"SURVEYDATA.csv"}\NormalTok{,  }\AttributeTok{sep =} \StringTok{";"}\NormalTok{, }\AttributeTok{stringsAsFactors =} \ConstantTok{FALSE}\NormalTok{)  }\CommentTok{\# Данные съемки}
\NormalTok{FISHERY }\OtherTok{\textless{}{-}} \FunctionTok{read.csv}\NormalTok{(}\StringTok{"FISHERYDATA.csv"}\NormalTok{, }\AttributeTok{sep =} \StringTok{";"}\NormalTok{, }\AttributeTok{stringsAsFactors =} \ConstantTok{FALSE}\NormalTok{)  }\CommentTok{\# Данные промысла}

\CommentTok{\# ОБРАБОТКА ДАННЫХ СЪЕМКИ:}
\CommentTok{\# 1. Добавляем столбец TYPE для идентификации типа данных}
\CommentTok{\# 2. Стандартизируем обозначения пола (M/F)}
\CommentTok{\# 3. Создаем дату (середина месяца)}
\CommentTok{\# 4. Преобразуем размер карапакса в числовой формат}
\CommentTok{\# 5. Фильтруем некорректные значения (отсутствующие, отрицательные, слишком большие)}
\NormalTok{survey }\OtherTok{\textless{}{-}}\NormalTok{ SURVEY }\SpecialCharTok{\%\textgreater{}\%}
  \FunctionTok{mutate}\NormalTok{(}
    \AttributeTok{TYPE     =} \StringTok{"Съёмка"}\NormalTok{,}
    \AttributeTok{SEX      =} \FunctionTok{if\_else}\NormalTok{(SEX }\SpecialCharTok{\%in\%} \FunctionTok{c}\NormalTok{(}\StringTok{"M"}\NormalTok{,}\StringTok{"male"}\NormalTok{,}\StringTok{"m"}\NormalTok{), }\StringTok{"M"}\NormalTok{,}
                       \FunctionTok{if\_else}\NormalTok{(SEX }\SpecialCharTok{\%in\%} \FunctionTok{c}\NormalTok{(}\StringTok{"F"}\NormalTok{,}\StringTok{"female"}\NormalTok{,}\StringTok{"f"}\NormalTok{), }\StringTok{"F"}\NormalTok{, }\ConstantTok{NA\_character\_}\NormalTok{)),}
    \AttributeTok{date     =} \FunctionTok{ymd}\NormalTok{(}\FunctionTok{paste}\NormalTok{(YEAR, MONTH, }\DecValTok{15}\NormalTok{, }\AttributeTok{sep =} \StringTok{"{-}"}\NormalTok{)),}
    \AttributeTok{CARAPACE =} \FunctionTok{as.numeric}\NormalTok{(CARAPACE)}
\NormalTok{  ) }\SpecialCharTok{\%\textgreater{}\%}
  \FunctionTok{filter}\NormalTok{(}\SpecialCharTok{!}\FunctionTok{is.na}\NormalTok{(CARAPACE), CARAPACE }\SpecialCharTok{\textgreater{}} \DecValTok{0}\NormalTok{, CARAPACE }\SpecialCharTok{\textless{}} \DecValTok{220}\NormalTok{)}

\CommentTok{\# Аналогичная обработка данных промысла}
\NormalTok{fishery }\OtherTok{\textless{}{-}}\NormalTok{ FISHERY }\SpecialCharTok{\%\textgreater{}\%}
  \FunctionTok{mutate}\NormalTok{(}
    \AttributeTok{TYPE     =} \StringTok{"Промысел"}\NormalTok{,}
    \AttributeTok{SEX      =} \FunctionTok{if\_else}\NormalTok{(SEX }\SpecialCharTok{\%in\%} \FunctionTok{c}\NormalTok{(}\StringTok{"M"}\NormalTok{,}\StringTok{"male"}\NormalTok{,}\StringTok{"m"}\NormalTok{), }\StringTok{"M"}\NormalTok{,}
                       \FunctionTok{if\_else}\NormalTok{(SEX }\SpecialCharTok{\%in\%} \FunctionTok{c}\NormalTok{(}\StringTok{"F"}\NormalTok{,}\StringTok{"female"}\NormalTok{,}\StringTok{"f"}\NormalTok{), }\StringTok{"F"}\NormalTok{, }\ConstantTok{NA\_character\_}\NormalTok{)),}
    \AttributeTok{date     =} \FunctionTok{ymd}\NormalTok{(}\FunctionTok{paste}\NormalTok{(YEAR, MONTH, }\DecValTok{15}\NormalTok{, }\AttributeTok{sep =} \StringTok{"{-}"}\NormalTok{)),}
    \AttributeTok{CARAPACE =} \FunctionTok{as.numeric}\NormalTok{(CARAPACE)}
\NormalTok{  ) }\SpecialCharTok{\%\textgreater{}\%}
  \FunctionTok{filter}\NormalTok{(}\SpecialCharTok{!}\FunctionTok{is.na}\NormalTok{(CARAPACE), CARAPACE }\SpecialCharTok{\textgreater{}} \DecValTok{0}\NormalTok{, CARAPACE }\SpecialCharTok{\textless{}} \DecValTok{240}\NormalTok{)}

\CommentTok{\# =============================================================================}
\CommentTok{\# 2. ВИЗУАЛИЗАЦИЯ РАЗМЕРНЫХ РАСПРЕДЕЛЕНИЙ (РИДЖЛАЙН{-}ГРАФИКИ)}
\CommentTok{\# =============================================================================}

\CommentTok{\# Объединяем данные съемки и промысла и фильтруем только самцов}
\NormalTok{df }\OtherTok{\textless{}{-}} \FunctionTok{bind\_rows}\NormalTok{(survey, fishery) }\SpecialCharTok{\%\textgreater{}\%}
  \FunctionTok{filter}\NormalTok{(}\SpecialCharTok{!}\FunctionTok{is.na}\NormalTok{(YEAR), }\SpecialCharTok{!}\FunctionTok{is.na}\NormalTok{(TYPE), SEX }\SpecialCharTok{==} \StringTok{"M"}\NormalTok{) }\SpecialCharTok{\%\textgreater{}\%}
  \FunctionTok{mutate}\NormalTok{(}
    \AttributeTok{YEAR   =} \FunctionTok{as.integer}\NormalTok{(YEAR),}
    \AttributeTok{YEAR\_F =} \FunctionTok{factor}\NormalTok{(YEAR, }\AttributeTok{levels =} \FunctionTok{sort}\NormalTok{(}\FunctionTok{unique}\NormalTok{(YEAR)))  }\CommentTok{\# Преобразуем год в фактор}
\NormalTok{  )}

\CommentTok{\# Определяем границы для оси X (округленные до десятков)}
\NormalTok{xmin }\OtherTok{\textless{}{-}} \FunctionTok{floor}\NormalTok{(}\FunctionTok{min}\NormalTok{(df}\SpecialCharTok{$}\NormalTok{CARAPACE, }\AttributeTok{na.rm =} \ConstantTok{TRUE}\NormalTok{) }\SpecialCharTok{/} \DecValTok{10}\NormalTok{) }\SpecialCharTok{*} \DecValTok{10}
\NormalTok{xmax }\OtherTok{\textless{}{-}} \FunctionTok{ceiling}\NormalTok{(}\FunctionTok{max}\NormalTok{(df}\SpecialCharTok{$}\NormalTok{CARAPACE, }\AttributeTok{na.rm =} \ConstantTok{TRUE}\NormalTok{) }\SpecialCharTok{/} \DecValTok{10}\NormalTok{) }\SpecialCharTok{*} \DecValTok{10}

\CommentTok{\# Создаем риджлайн{-}график (горный хребет)}
\CommentTok{\# Риджлайны показывают распределение размеров по годам}
\NormalTok{p }\OtherTok{\textless{}{-}} \FunctionTok{ggplot}\NormalTok{(df, }\FunctionTok{aes}\NormalTok{(}\AttributeTok{x =}\NormalTok{ CARAPACE, }\AttributeTok{y =}\NormalTok{ YEAR\_F, }\AttributeTok{fill =} \FunctionTok{after\_stat}\NormalTok{(x))) }\SpecialCharTok{+}
  \FunctionTok{stat\_density\_ridges}\NormalTok{(}
    \AttributeTok{geom =} \StringTok{"density\_ridges\_gradient"}\NormalTok{,}
    \AttributeTok{calc\_ecdf =} \ConstantTok{FALSE}\NormalTok{,}
    \AttributeTok{rel\_min\_height =} \FloatTok{0.001}\NormalTok{,}
    \AttributeTok{bandwidth =} \DecValTok{3}  \CommentTok{\# Параметр сглаживания}
\NormalTok{  ) }\SpecialCharTok{+}
  \FunctionTok{scale\_fill\_viridis\_c}\NormalTok{(}\AttributeTok{name =} \StringTok{"Размер, мм"}\NormalTok{, }\AttributeTok{option =} \StringTok{"C"}\NormalTok{) }\SpecialCharTok{+}  \CommentTok{\# Цветовая схема}
  \FunctionTok{facet\_wrap}\NormalTok{(}\SpecialCharTok{\textasciitilde{}}\NormalTok{ TYPE, }\AttributeTok{ncol =} \DecValTok{1}\NormalTok{, }\AttributeTok{scales =} \StringTok{"free\_y"}\NormalTok{) }\SpecialCharTok{+}  \CommentTok{\# Разделяем по типам данных}
  \FunctionTok{coord\_cartesian}\NormalTok{(}\AttributeTok{xlim =} \FunctionTok{c}\NormalTok{(xmin, xmax)) }\SpecialCharTok{+}  \CommentTok{\# Устанавливаем границы оси X}
  \FunctionTok{labs}\NormalTok{(}\AttributeTok{x =} \StringTok{"Ширина карапакса, мм"}\NormalTok{, }\AttributeTok{y =} \StringTok{"Год"}\NormalTok{,}
       \AttributeTok{title =} \StringTok{"Размерные распределения по годам: съёмка и промысел (самцы)"}\NormalTok{) }\SpecialCharTok{+}
  \FunctionTok{theme\_minimal}\NormalTok{(}\AttributeTok{base\_size =} \DecValTok{12}\NormalTok{) }\SpecialCharTok{+}
  \FunctionTok{theme}\NormalTok{(}\AttributeTok{legend.position =} \StringTok{"right"}\NormalTok{, }\AttributeTok{panel.grid.minor =} \FunctionTok{element\_blank}\NormalTok{())}

\NormalTok{p  }\CommentTok{\# Отображаем график}

\CommentTok{\# Сохранение графика (раскомментируйте при необходимости)}
\CommentTok{\# ggsave("size\_ridges\_by\_year\_type\_M.png", p, width = 10, height = 12, dpi = 200)}
\end{Highlighting}
\end{Shaded}

\begin{center}
\includegraphics[width=0.8\linewidth,height=\textheight,keepaspectratio]{images/BERTA1.png}
\end{center}

\begin{Shaded}
\begin{Highlighting}[]
\CommentTok{\# =============================================================================}
\CommentTok{\# 3. ОЦЕНКА ПАРАМЕТРОВ РОСТА МЕТОДОМ ELEFAN\_SA}
\CommentTok{\# =============================================================================}

\CommentTok{\# Фильтруем данные только по самцам}
\NormalTok{survey\_m }\OtherTok{\textless{}{-}}\NormalTok{ survey }\SpecialCharTok{\%\textgreater{}\%} \FunctionTok{filter}\NormalTok{(SEX }\SpecialCharTok{==} \StringTok{"M"}\NormalTok{)}
\NormalTok{fish\_m   }\OtherTok{\textless{}{-}}\NormalTok{ fishery }\SpecialCharTok{\%\textgreater{}\%} \FunctionTok{filter}\NormalTok{(SEX }\SpecialCharTok{==} \StringTok{"M"}\NormalTok{)}

\CommentTok{\# Не ограничиваем анализ определенными годами для устойчивости результатов}
\NormalTok{survey\_m }\OtherTok{\textless{}{-}}\NormalTok{ survey\_m }\SpecialCharTok{\%\textgreater{}\%} \FunctionTok{filter}\NormalTok{(}\SpecialCharTok{!}\FunctionTok{is.na}\NormalTok{(YEAR))  }\CommentTok{\# Все доступные данные}

 \CommentTok{\#years\_use \textless{}{-} 2010:2019  \# 10 лет вместо неограниченного периода}
 \CommentTok{\#survey\_m \textless{}{-} survey\_m \%\textgreater{}\% filter(YEAR \%in\% years\_use)}


\CommentTok{\# Создаем объект LFQ (Length{-}Frequency Data) для анализа}
\NormalTok{binw }\OtherTok{\textless{}{-}} \DecValTok{2}  \CommentTok{\# Ширина класса размеров (2 мм)}
\NormalTok{lfq }\OtherTok{\textless{}{-}}\NormalTok{ TropFishR}\SpecialCharTok{::}\FunctionTok{lfqCreate}\NormalTok{(}
  \AttributeTok{data     =}\NormalTok{ survey\_m }\SpecialCharTok{\%\textgreater{}\%} \FunctionTok{mutate}\NormalTok{(}\AttributeTok{date =} \FunctionTok{as.Date}\NormalTok{(date)),}
  \AttributeTok{Lname    =} \StringTok{"CARAPACE"}\NormalTok{,  }\CommentTok{\# Столбец с размерами}
  \AttributeTok{Dname    =} \StringTok{"date"}\NormalTok{,      }\CommentTok{\# Столбец с датами}
  \AttributeTok{bin\_size =}\NormalTok{ binw         }\CommentTok{\# Ширина класса}
\NormalTok{)}

\CommentTok{\# Инициализация параметра L∞ (максимальный теоретический размер)}
\CommentTok{\# Основана на 99{-}м процентиле размеров промысла с поправочным коэффициентом}
\NormalTok{init\_Linf2 }\OtherTok{\textless{}{-}} \FunctionTok{as.numeric}\NormalTok{(}\FunctionTok{quantile}\NormalTok{(fish\_m}\SpecialCharTok{$}\NormalTok{CARAPACE, }\FloatTok{0.99}\NormalTok{, }\AttributeTok{na.rm =} \ConstantTok{TRUE}\NormalTok{)) }\SpecialCharTok{/} \FloatTok{0.95}
\NormalTok{init\_Linf2 }\OtherTok{\textless{}{-}} \FunctionTok{min}\NormalTok{(}\FunctionTok{max}\NormalTok{(init\_Linf2, }\DecValTok{155}\NormalTok{), }\DecValTok{165}\NormalTok{)  }\CommentTok{\# Ограничиваем разумными пределами}

\CommentTok{\# Оценка параметров роста методом имитации отжига (ELEFAN\_SA)}
\FunctionTok{set.seed}\NormalTok{(}\DecValTok{1}\NormalTok{)  }\CommentTok{\# Для воспроизводимости результатов}
\NormalTok{res\_sa2 }\OtherTok{\textless{}{-}}\NormalTok{ TropFishR}\SpecialCharTok{::}\FunctionTok{ELEFAN\_SA}\NormalTok{(}
\NormalTok{  lfq,}
  \AttributeTok{seasonalised =} \ConstantTok{TRUE}\NormalTok{,  }\CommentTok{\# Используем сезонную модель}
  \CommentTok{\# Начальные значения параметров:}
  \AttributeTok{init\_par =} \FunctionTok{list}\NormalTok{(}\AttributeTok{Linf =}\NormalTok{ init\_Linf2, }\AttributeTok{K =} \FloatTok{0.2}\NormalTok{, }\AttributeTok{t\_anchor =} \FloatTok{0.1}\NormalTok{, }\AttributeTok{C =} \FloatTok{0.2}\NormalTok{, }\AttributeTok{ts =} \FloatTok{0.5}\NormalTok{),}
  \CommentTok{\# Нижние границы параметров:}
  \AttributeTok{low\_par  =} \FunctionTok{list}\NormalTok{(}\AttributeTok{Linf =} \DecValTok{155}\NormalTok{,        }\AttributeTok{K =} \FloatTok{0.20}\NormalTok{, }\AttributeTok{t\_anchor =} \FloatTok{0.0}\NormalTok{, }\AttributeTok{C =} \FloatTok{0.0}\NormalTok{, }\AttributeTok{ts =} \FloatTok{0.0}\NormalTok{),}
  \CommentTok{\# Верхние границы параметров:}
  \AttributeTok{up\_par   =} \FunctionTok{list}\NormalTok{(}\AttributeTok{Linf =} \DecValTok{165}\NormalTok{,        }\AttributeTok{K =} \FloatTok{0.35}\NormalTok{, }\AttributeTok{t\_anchor =} \FloatTok{1.0}\NormalTok{, }\AttributeTok{C =} \FloatTok{0.4}\NormalTok{, }\AttributeTok{ts =} \FloatTok{1.0}\NormalTok{),}
  \AttributeTok{SA\_time  =} \DecValTok{60}  \CommentTok{\# Время работы алгоритма (секунды)}
\NormalTok{)}


\CommentTok{\# Извлекаем оцененные параметры}
\NormalTok{pars2 }\OtherTok{\textless{}{-}}\NormalTok{ res\_sa2}\SpecialCharTok{$}\NormalTok{par}

\CommentTok{\# Рассчитываем дополнительные показатели:}
\CommentTok{\# φ\textquotesingle{} (фи{-}prime) {-} индекс производительности роста}
\NormalTok{phi\_prime }\OtherTok{\textless{}{-}} \FunctionTok{log10}\NormalTok{(pars2}\SpecialCharTok{$}\NormalTok{K) }\SpecialCharTok{+} \DecValTok{2}\SpecialCharTok{*}\FunctionTok{log10}\NormalTok{(pars2}\SpecialCharTok{$}\NormalTok{Linf)}
\CommentTok{\# Время достижения 95\% от L∞}
\NormalTok{t95 }\OtherTok{\textless{}{-}} \ControlFlowTok{function}\NormalTok{(K) }\SpecialCharTok{{-}}\FunctionTok{log}\NormalTok{(}\DecValTok{1} \SpecialCharTok{{-}} \FloatTok{0.95}\NormalTok{)}\SpecialCharTok{/}\NormalTok{K}

\CommentTok{\# Вывод результатов}
\FunctionTok{cat}\NormalTok{(}\FunctionTok{sprintf}\NormalTok{(}\StringTok{"ELEFAN\_SA: L∞=\%.1f мм, K=\%.3f, C=\%.2f, ts=\%.2f, phi\textquotesingle{} = \%.3f, t95 = \%.1f лет}\SpecialCharTok{\textbackslash{}n}\StringTok{"}\NormalTok{,}
\NormalTok{            pars2}\SpecialCharTok{$}\NormalTok{Linf, pars2}\SpecialCharTok{$}\NormalTok{K, pars2}\SpecialCharTok{$}\NormalTok{C, pars2}\SpecialCharTok{$}\NormalTok{ts, phi\_prime, }\FunctionTok{t95}\NormalTok{(pars2}\SpecialCharTok{$}\NormalTok{K)))}
\end{Highlighting}
\end{Shaded}

\section{Вывод
результатов}\label{ux432ux44bux432ux43eux434-ux440ux435ux437ux443ux43bux44cux442ux430ux442ux43eux432}

ELEFAN\_SA : \emph{L∞}=163.4 мм, \emph{K}=0.211, \emph{C}=0.07,
\emph{ts}=0.17, \emph{phi'} = 3.750, \emph{t95} = 14.2 лет

Интерпретация параметров:

\begin{enumerate}
\def\labelenumi{\arabic{enumi}.}
\item
  \emph{L∞} = 163.4 мм • Асимптотическая длина --- максимальный
  теоретический размер, которого могут достигать самцы краба-опилио. •
  Реалистично: Для самцов опилио L∞ обычно лежит в диапазоне 150--170 мм
  (по ширине карапакса). Ваш результат попадает точно в этот диапазон.
\item
  \emph{K} = 0.211 год⁻¹ • Коэффициент роста --- показывает, насколько
  быстро особь приближается к \emph{L∞}. • Реалистично: Для опилио K
  обычно находится в диапазоне 0.20--0.30 год⁻¹. Значение 0.211 ---
  идеально подходит.
\item
  \emph{C} = 0.07 • Амплитуда сезонных колебаний роста. • Реалистично:
  Значение очень низкое, что говорит о слабо выраженной сезонности
  роста. • Это типично для демерсальных ракообразных в холодных водах,
  где температура дна меняется незначительно в течение года. Рост
  замедляется, но не останавливается. • Согласуется с данными Паули
  (1987), где показано, что \emph{C} коррелирует с амплитудой колебаний
  температуры среды. Для холодных вод --- \emph{C} близок к 0.
\item
  \emph{ts} = 0.17 • Сдвиг сезонных колебаний. Показывает, когда
  начинается период замедленного роста. • \emph{ts} = 0.17 ≈ февраль
  (0.17 * 12 ≈ 2 месяца). • Реалистично: Для северных популяций
  замедление роста часто приходится на зимние месяцы (январь-февраль),
  когда температура воды минимальна и/или пищи меньше.
\item
  \emph{φ'} = 3.750 • Индекс производительности роста. Комбинирует
  \emph{L∞} и \emph{K}. • Для ракообразных \emph{φ'} обычно лежит в
  диапазоне 3.5--4.0. Наш результат 3.750 --- идеален. • Показывает, что
  вид имеет умеренно высокую скорость роста при крупном максимальном
  размере --- типично для промысловых крабов.
\item
  \emph{t95} = 14.2 года • Возраст, в котором особь достигает 95\% от
  \emph{L∞}. • Реалистично: Для самцов опилио t95 обычно составляет
  12--15 лет. Наш результат 14.2 года --- биологически обоснован. • Это
  не ошибка, а реальность для этого вида. Краб-стригун опилио ---
  долгожитель по меркам ракообразных. Максимальный зарегистрированный
  возраст --- до 18 лет. • Важно: \emph{t95} --- это не
  продолжительность жизни, а возраст, к которому рост практически
  прекращается. После этого особь может жить еще несколько лет, не
  увеличиваясь в размере.
\end{enumerate}

\begin{Shaded}
\begin{Highlighting}[]
\CommentTok{\# =============================================================================}
\CommentTok{\# 4. СРАВНЕНИЕ РАСПРЕДЕЛЕНИЙ РАЗМЕРОВ СЪЕМКИ И ПРОМЫСЛА}
\CommentTok{\# =============================================================================}

\CommentTok{\# График плотности распределений размеров}
\NormalTok{p\_dens }\OtherTok{\textless{}{-}} \FunctionTok{ggplot}\NormalTok{() }\SpecialCharTok{+}
  \CommentTok{\# Плотность распределения для данных съемки}
  \FunctionTok{geom\_density}\NormalTok{(}\AttributeTok{data =}\NormalTok{ survey\_m, }\FunctionTok{aes}\NormalTok{(CARAPACE, }\AttributeTok{colour =} \StringTok{"Съёмка"}\NormalTok{), }\AttributeTok{bw =} \DecValTok{3}\NormalTok{) }\SpecialCharTok{+}
  \CommentTok{\# Плотность распределения для данных промысла}
  \FunctionTok{geom\_density}\NormalTok{(}\AttributeTok{data =}\NormalTok{ fish\_m,   }\FunctionTok{aes}\NormalTok{(CARAPACE, }\AttributeTok{colour =} \StringTok{"Промысел"}\NormalTok{), }\AttributeTok{bw =} \DecValTok{3}\NormalTok{) }\SpecialCharTok{+}
  \CommentTok{\# Вертикальная линия, показывающая оценку L∞}
  \FunctionTok{geom\_vline}\NormalTok{(}\AttributeTok{xintercept =}\NormalTok{ pars2}\SpecialCharTok{$}\NormalTok{Linf, }\AttributeTok{linetype =} \DecValTok{2}\NormalTok{) }\SpecialCharTok{+}
  \CommentTok{\# Цветовая схема}
  \FunctionTok{scale\_colour\_manual}\NormalTok{(}\AttributeTok{values =} \FunctionTok{c}\NormalTok{(}\StringTok{"Съёмка"} \OtherTok{=} \StringTok{"steelblue"}\NormalTok{, }\StringTok{"Промысел"} \OtherTok{=} \StringTok{"tomato"}\NormalTok{)) }\SpecialCharTok{+}
  \FunctionTok{labs}\NormalTok{(}\AttributeTok{x =} \StringTok{"Ширина карапакса, мм"}\NormalTok{, }\AttributeTok{y =} \StringTok{"Плотность"}\NormalTok{, }\AttributeTok{colour =} \ConstantTok{NULL}\NormalTok{,}
       \AttributeTok{title =} \StringTok{"Распределения по размерам; штрих — оценка L∞ (ELEFAN)"}\NormalTok{) }\SpecialCharTok{+}
  \FunctionTok{theme\_minimal}\NormalTok{(}\AttributeTok{base\_size =} \DecValTok{12}\NormalTok{)}

\NormalTok{p\_dens }\CommentTok{\# Отображаем график}

\CommentTok{\# Сохранение графика}
\FunctionTok{ggsave}\NormalTok{(}\StringTok{"densities\_survey\_fishery\_Linf.png"}\NormalTok{, p\_dens, }\AttributeTok{width =} \DecValTok{8}\NormalTok{, }\AttributeTok{height =} \DecValTok{5}\NormalTok{, }\AttributeTok{dpi =} \DecValTok{200}\NormalTok{)}
\end{Highlighting}
\end{Shaded}

\begin{center}
\includegraphics[width=0.8\linewidth,height=\textheight,keepaspectratio]{images/BERTA2.png}
\end{center}

\begin{Shaded}
\begin{Highlighting}[]
\CommentTok{\# =============================================================================}
\CommentTok{\# 4.1 Визуализация "реструктурированных" распределений по длине (restructured counts)}
\CommentTok{\# с наложенной оптимальной кривой роста, подобранной методом ELEFAN\_SA.}
\CommentTok{\# =============================================================================}

\NormalTok{restr }\OtherTok{\textless{}{-}} \FunctionTok{plot}\NormalTok{(res\_sa2, }\AttributeTok{Fname =} \StringTok{"rcounts"}\NormalTok{, }\AttributeTok{image.col =} \FunctionTok{colorRampPalette}\NormalTok{(}\FunctionTok{c}\NormalTok{(}\StringTok{"red"}\NormalTok{,}\StringTok{"grey100"}\NormalTok{,}\StringTok{"green"}\NormalTok{))(}\DecValTok{21}\NormalTok{))}

\NormalTok{restr  }\CommentTok{\# Отображаем график}

\CommentTok{\# Сохранение графика}
\FunctionTok{ggsave}\NormalTok{(}\StringTok{"restr.png"}\NormalTok{, restr, }\AttributeTok{width =} \DecValTok{8}\NormalTok{, }\AttributeTok{height =} \DecValTok{5}\NormalTok{, }\AttributeTok{dpi =} \DecValTok{200}\NormalTok{)}
\end{Highlighting}
\end{Shaded}

\begin{center}
\includegraphics[width=1\linewidth,height=\textheight,keepaspectratio]{images/BERTA3.png}
\end{center}

Визуализация ``реструктурированных'' распределений по длине
(restructured counts) с наложенной оптимальной кривой роста, подобранной
методом ELEFAN\_SA.

\begin{enumerate}
\def\labelenumi{\arabic{enumi}.}
\tightlist
\item
  restructured counts (rcounts) --- это ключевое понятие в методе
  ELEFAN, подробно описанное в статье Паули (стр. 14--16, Fig. 4). • Это
  не исходные гистограммы, а преобразованные данные, где: • Пики (моды)
  выделены положительными значениями (зелёный цвет). • Впадины между
  пиками --- отрицательными (красный цвет). • Фон (серый) ---
  нейтральные зоны. • Цель --- ``очистить'' данные от шума и выделить
  структуру, которую должна объяснить кривая роста.
\item
  Оптимальные кривые роста (пунктиры) --- это результат работы
  ELEFAN\_SA, который максимизирует ESP/ASP (Explained Sum of Peaks /
  Available Sum of Peaks) --- главный критерий качества подгонки в
  ELEFAN (Pauly, 1987, стр. 15--16).
\end{enumerate}

\begin{Shaded}
\begin{Highlighting}[]
\CommentTok{\# =============================================================================}
\CommentTok{\# 5. ПОСТРОЕНИЕ КРИВОЙ РОСТА С ДОВЕРИТЕЛЬНЫМИ ИНТЕРВАЛАЛАМИ}
\CommentTok{\# =============================================================================}

\CommentTok{\# Функция сезонного уравнения роста Берталанфи}
\CommentTok{\# t {-} возраст, Linf {-} асимптотическая длина, K {-} коэффициент роста,}
\CommentTok{\# t0 {-} теоретический возраст при нулевой длине, C {-} амплитуда сезонных колебаний,}
\CommentTok{\# ts {-} параметр сдвига сезонных колебаний}
\NormalTok{seasonal\_vbgf }\OtherTok{\textless{}{-}} \ControlFlowTok{function}\NormalTok{(t, Linf, K, t0, C, ts) \{}
\NormalTok{  term }\OtherTok{\textless{}{-}} \SpecialCharTok{{-}}\NormalTok{K}\SpecialCharTok{*}\NormalTok{(t }\SpecialCharTok{{-}}\NormalTok{ t0) }\SpecialCharTok{+}\NormalTok{ (C}\SpecialCharTok{*}\NormalTok{K}\SpecialCharTok{/}\NormalTok{(}\DecValTok{2}\SpecialCharTok{*}\NormalTok{pi)) }\SpecialCharTok{*} \FunctionTok{sin}\NormalTok{(}\DecValTok{2}\SpecialCharTok{*}\NormalTok{pi}\SpecialCharTok{*}\NormalTok{(t }\SpecialCharTok{{-}}\NormalTok{ ts)) }\SpecialCharTok{{-}}\NormalTok{ (C}\SpecialCharTok{*}\NormalTok{K}\SpecialCharTok{/}\NormalTok{(}\DecValTok{2}\SpecialCharTok{*}\NormalTok{pi)) }\SpecialCharTok{*} \FunctionTok{sin}\NormalTok{(}\DecValTok{2}\SpecialCharTok{*}\NormalTok{pi}\SpecialCharTok{*}\NormalTok{(t0 }\SpecialCharTok{{-}}\NormalTok{ ts))}
\NormalTok{  Lt }\OtherTok{\textless{}{-}}\NormalTok{ Linf }\SpecialCharTok{*}\NormalTok{ (}\DecValTok{1} \SpecialCharTok{{-}} \FunctionTok{exp}\NormalTok{(term))}
  \FunctionTok{return}\NormalTok{(Lt)}
\NormalTok{\}}

\CommentTok{\# Функция для предсказания роста с учетом неопределенности параметров}
\NormalTok{predict\_growth }\OtherTok{\textless{}{-}} \ControlFlowTok{function}\NormalTok{(ages, params, }\AttributeTok{n\_sim =} \DecValTok{1000}\NormalTok{, }\AttributeTok{alpha =} \FloatTok{0.05}\NormalTok{) \{}
  \CommentTok{\# Извлекаем точечные оценки параметров}
\NormalTok{  Linf\_est }\OtherTok{\textless{}{-}}\NormalTok{ params}\SpecialCharTok{$}\NormalTok{Linf}
\NormalTok{  K\_est }\OtherTok{\textless{}{-}}\NormalTok{ params}\SpecialCharTok{$}\NormalTok{K}
\NormalTok{  t0\_est }\OtherTok{\textless{}{-}}\NormalTok{ params}\SpecialCharTok{$}\NormalTok{t\_anchor}
\NormalTok{  C\_est }\OtherTok{\textless{}{-}}\NormalTok{ params}\SpecialCharTok{$}\NormalTok{C}
\NormalTok{  ts\_est }\OtherTok{\textless{}{-}}\NormalTok{ params}\SpecialCharTok{$}\NormalTok{ts}
  
  \CommentTok{\# Матрица для хранения симуляций}
\NormalTok{  simulations }\OtherTok{\textless{}{-}} \FunctionTok{matrix}\NormalTok{(}\ConstantTok{NA}\NormalTok{, }\AttributeTok{nrow =}\NormalTok{ n\_sim, }\AttributeTok{ncol =} \FunctionTok{length}\NormalTok{(ages))}
  
  \CommentTok{\# Симуляция возможных значений параметров на основе их неопределенности}
  \FunctionTok{set.seed}\NormalTok{(}\DecValTok{123}\NormalTok{)}
\NormalTok{  Linf\_sim }\OtherTok{\textless{}{-}} \FunctionTok{rnorm}\NormalTok{(n\_sim, }\AttributeTok{mean =}\NormalTok{ Linf\_est, }\AttributeTok{sd =}\NormalTok{ Linf\_est }\SpecialCharTok{*} \FloatTok{0.05}\NormalTok{)  }\CommentTok{\# CV 5\%}
\NormalTok{  K\_sim }\OtherTok{\textless{}{-}} \FunctionTok{rnorm}\NormalTok{(n\_sim, }\AttributeTok{mean =}\NormalTok{ K\_est, }\AttributeTok{sd =}\NormalTok{ K\_est }\SpecialCharTok{*} \FloatTok{0.1}\NormalTok{)           }\CommentTok{\# CV 10\%}
\NormalTok{  t0\_sim }\OtherTok{\textless{}{-}} \FunctionTok{rnorm}\NormalTok{(n\_sim, }\AttributeTok{mean =}\NormalTok{ t0\_est, }\AttributeTok{sd =} \FloatTok{0.1}\NormalTok{)                 }\CommentTok{\# SD 0.1}
\NormalTok{  C\_sim }\OtherTok{\textless{}{-}} \FunctionTok{rnorm}\NormalTok{(n\_sim, }\AttributeTok{mean =}\NormalTok{ C\_est, }\AttributeTok{sd =} \FloatTok{0.05}\NormalTok{)                  }\CommentTok{\# SD 0.05}
\NormalTok{  ts\_sim }\OtherTok{\textless{}{-}} \FunctionTok{rnorm}\NormalTok{(n\_sim, }\AttributeTok{mean =}\NormalTok{ ts\_est, }\AttributeTok{sd =} \FloatTok{0.05}\NormalTok{)                }\CommentTok{\# SD 0.05}
  
  \CommentTok{\# Ограничение параметров физиологически реалистичными значениями}
\NormalTok{  Linf\_sim }\OtherTok{\textless{}{-}} \FunctionTok{pmax}\NormalTok{(}\FunctionTok{pmin}\NormalTok{(Linf\_sim, }\DecValTok{180}\NormalTok{), }\DecValTok{140}\NormalTok{)}
\NormalTok{  K\_sim }\OtherTok{\textless{}{-}} \FunctionTok{pmax}\NormalTok{(}\FunctionTok{pmin}\NormalTok{(K\_sim, }\FloatTok{0.5}\NormalTok{), }\FloatTok{0.05}\NormalTok{)}
\NormalTok{  C\_sim }\OtherTok{\textless{}{-}} \FunctionTok{pmax}\NormalTok{(}\FunctionTok{pmin}\NormalTok{(C\_sim, }\DecValTok{1}\NormalTok{), }\DecValTok{0}\NormalTok{)}
\NormalTok{  ts\_sim }\OtherTok{\textless{}{-}} \FunctionTok{pmax}\NormalTok{(}\FunctionTok{pmin}\NormalTok{(ts\_sim, }\DecValTok{1}\NormalTok{), }\DecValTok{0}\NormalTok{)}
  
  \CommentTok{\# Расчет кривых роста для каждой симуляции}
  \ControlFlowTok{for}\NormalTok{ (i }\ControlFlowTok{in} \DecValTok{1}\SpecialCharTok{:}\NormalTok{n\_sim) \{}
\NormalTok{    simulations[i, ] }\OtherTok{\textless{}{-}} \FunctionTok{seasonal\_vbgf}\NormalTok{(ages, Linf\_sim[i], K\_sim[i], }
\NormalTok{                                      t0\_sim[i], C\_sim[i], ts\_sim[i])}
\NormalTok{  \}}
  
  \CommentTok{\# Расчет доверительных интервалов}
\NormalTok{  mean\_pred }\OtherTok{\textless{}{-}} \FunctionTok{apply}\NormalTok{(simulations, }\DecValTok{2}\NormalTok{, mean)}
\NormalTok{  lower\_ci }\OtherTok{\textless{}{-}} \FunctionTok{apply}\NormalTok{(simulations, }\DecValTok{2}\NormalTok{, quantile, }\AttributeTok{probs =}\NormalTok{ alpha}\SpecialCharTok{/}\DecValTok{2}\NormalTok{)}
\NormalTok{  upper\_ci }\OtherTok{\textless{}{-}} \FunctionTok{apply}\NormalTok{(simulations, }\DecValTok{2}\NormalTok{, quantile, }\AttributeTok{probs =} \DecValTok{1} \SpecialCharTok{{-}}\NormalTok{ alpha}\SpecialCharTok{/}\DecValTok{2}\NormalTok{)}
  
  \FunctionTok{return}\NormalTok{(}\FunctionTok{data.frame}\NormalTok{(}
    \AttributeTok{Age =}\NormalTok{ ages,}
    \AttributeTok{Length =}\NormalTok{ mean\_pred,}
    \AttributeTok{Lower =}\NormalTok{ lower\_ci,}
    \AttributeTok{Upper =}\NormalTok{ upper\_ci}
\NormalTok{  ))}
\NormalTok{\}}

\CommentTok{\# Создаем данные для графика с доверительными интервалами}
\NormalTok{ages }\OtherTok{\textless{}{-}} \FunctionTok{seq}\NormalTok{(}\DecValTok{0}\NormalTok{, }\DecValTok{20}\NormalTok{, }\AttributeTok{by =} \FloatTok{0.1}\NormalTok{)  }\CommentTok{\# Возраст от 0 до 20 лет с шагом 0.1}
\NormalTok{growth\_df }\OtherTok{\textless{}{-}} \FunctionTok{predict\_growth}\NormalTok{(ages, pars2, }\AttributeTok{n\_sim =} \DecValTok{1000}\NormalTok{, }\AttributeTok{alpha =} \FloatTok{0.05}\NormalTok{)  }\CommentTok{\# 95\% ДИ}

\CommentTok{\# Построение графика кривой роста}
\NormalTok{p\_growth }\OtherTok{\textless{}{-}} \FunctionTok{ggplot}\NormalTok{(growth\_df, }\FunctionTok{aes}\NormalTok{(}\AttributeTok{x =}\NormalTok{ Age, }\AttributeTok{y =}\NormalTok{ Length)) }\SpecialCharTok{+}
  \CommentTok{\# Доверительные интервалы (заполненная область)}
  \FunctionTok{geom\_ribbon}\NormalTok{(}\FunctionTok{aes}\NormalTok{(}\AttributeTok{ymin =}\NormalTok{ Lower, }\AttributeTok{ymax =}\NormalTok{ Upper), }\AttributeTok{alpha =} \FloatTok{0.2}\NormalTok{, }\AttributeTok{fill =} \StringTok{"blue"}\NormalTok{) }\SpecialCharTok{+}
  \CommentTok{\# Средняя кривая роста}
  \FunctionTok{geom\_line}\NormalTok{(}\AttributeTok{color =} \StringTok{"blue"}\NormalTok{, }\AttributeTok{linewidth =} \DecValTok{1}\NormalTok{) }\SpecialCharTok{+}
  \CommentTok{\# Горизонтальная линия, показывающая L∞}
  \FunctionTok{geom\_hline}\NormalTok{(}\AttributeTok{yintercept =}\NormalTok{ pars2}\SpecialCharTok{$}\NormalTok{Linf, }\AttributeTok{linetype =} \StringTok{"dashed"}\NormalTok{, }\AttributeTok{color =} \StringTok{"red"}\NormalTok{) }\SpecialCharTok{+}
  \FunctionTok{labs}\NormalTok{(}\AttributeTok{x =} \StringTok{"Возраст (лет)"}\NormalTok{, }\AttributeTok{y =} \StringTok{"Ширина карапакса (мм)"}\NormalTok{,}
       \AttributeTok{title =} \StringTok{"Кривая роста (сезонный ВБГФ) с доверительными интервалами"}\NormalTok{,}
       \AttributeTok{subtitle =} \FunctionTok{sprintf}\NormalTok{(}\StringTok{"L∞=\%.1f мм, K=\%.3f, C=\%.2f, ts=\%.2f, t95≈\%.1f лет"}\NormalTok{,}
\NormalTok{                          pars2}\SpecialCharTok{$}\NormalTok{Linf, pars2}\SpecialCharTok{$}\NormalTok{K, pars2}\SpecialCharTok{$}\NormalTok{C, pars2}\SpecialCharTok{$}\NormalTok{ts, }\FunctionTok{t95}\NormalTok{(pars2}\SpecialCharTok{$}\NormalTok{K))) }\SpecialCharTok{+}
  \FunctionTok{theme\_minimal}\NormalTok{(}\AttributeTok{base\_size =} \DecValTok{13}\NormalTok{) }\SpecialCharTok{+}
  \FunctionTok{theme}\NormalTok{(}\AttributeTok{plot.title =} \FunctionTok{element\_text}\NormalTok{(}\AttributeTok{hjust =} \FloatTok{0.5}\NormalTok{))}

\FunctionTok{print}\NormalTok{(p\_growth)}
\CommentTok{\# Сохранение графика}
\FunctionTok{ggsave}\NormalTok{(}\StringTok{"seasonal\_VBGF\_with\_CI.png"}\NormalTok{, p\_growth, }\AttributeTok{width =} \DecValTok{8}\NormalTok{, }\AttributeTok{height =} \DecValTok{5}\NormalTok{, }\AttributeTok{dpi =} \DecValTok{200}\NormalTok{)}
\end{Highlighting}
\end{Shaded}

\begin{center}
\includegraphics[width=0.8\linewidth,height=\textheight,keepaspectratio]{images/BERTA4.png}
\end{center}

\begin{Shaded}
\begin{Highlighting}[]
\CommentTok{\# =============================================================================}
\CommentTok{\# 6. СОХРАНЕНИЕ РЕЗУЛЬТАТОВ}
\CommentTok{\# =============================================================================}

\CommentTok{\# Сохранение оцененных параметров роста в CSV{-}файл}
\FunctionTok{write.csv}\NormalTok{(}\FunctionTok{tibble}\NormalTok{(}
  \AttributeTok{Linf =}\NormalTok{ pars2}\SpecialCharTok{$}\NormalTok{Linf, }
  \AttributeTok{K =}\NormalTok{ pars2}\SpecialCharTok{$}\NormalTok{K, }
  \AttributeTok{C =}\NormalTok{ pars2}\SpecialCharTok{$}\NormalTok{C, }
  \AttributeTok{ts =}\NormalTok{ pars2}\SpecialCharTok{$}\NormalTok{ts,}
  \AttributeTok{t\_anchor =}\NormalTok{ pars2}\SpecialCharTok{$}\NormalTok{t\_anchor, }
  \AttributeTok{phi\_prime =}\NormalTok{ phi\_prime, }
  \AttributeTok{t95\_years =} \FunctionTok{t95}\NormalTok{(pars2}\SpecialCharTok{$}\NormalTok{K)}
\NormalTok{), }\StringTok{"ELEFAN\_params\_constrained.csv"}\NormalTok{, }\AttributeTok{row.names =} \ConstantTok{FALSE}\NormalTok{)}

\CommentTok{\# Сообщение о завершении анализа}
\FunctionTok{cat}\NormalTok{(}\StringTok{"Готово. Файлы: size\_ridges\_by\_year\_type\_M.png, densities\_survey\_fishery\_Linf.png, seasonal\_VBGF\_to15y.png, ELEFAN\_params\_constrained.csv}\SpecialCharTok{\textbackslash{}n}\StringTok{"}\NormalTok{)}
\end{Highlighting}
\end{Shaded}

\section{Заключение}\label{ux437ux430ux43aux43bux44eux447ux435ux43dux438ux435-1}

• По демонстрационному прогону для наших данных ELEFAN под ограничениями
даёт реалистичный набор параметров для самцов опилио: \emph{L∞} в
интервале 155--165 мм, \emph{K} порядка 0.20--0.25 год⁻¹, слабая
сезонность (\emph{C} близка к 0), \emph{t95} ≈ 12--15 лет. Это
согласуется с биологией вида и правым хвостом промысловых размеров. •
Ключевые проверки качества: o Соответствие \emph{L∞} правому хвосту
промысла по годам (\emph{Q99}, \emph{Lmax}). o \emph{t95} не выходит за
пределы жизненного горизонта (для опилио --- порядка ≤15 лет). o Кривая
в rcounts проходит через «зелёные» гребни мод, без систематических
сдвигов. o Индекс \emph{φ′} лежит в ожидаемом для ракообразных диапазоне
(ориентир 3.5--4.0). • Ограничения метода: o ELEFAN работает с модальной
прогрессией длин, а не с истинным возрастом. Для ракообразных дискретные
линьки и терминальная линька могут смещать \emph{K} вниз и \emph{L∞}
вверх, если не ограничивать поиск. Интерпретируйте K как эффективный
параметр, пригодный для length-based процедур, а не как «истинную»
физиологическую константу. o Неполное внутригодовое покрытие (у вас ---
август--ноябрь) усложняет оценку сезонности (**C\emph{, }ts*); при
необходимости фиксируйте **C* ≈ 0. • Рекомендации по устойчивости: o
Повторите оценку на поднаборах лет с хорошей выборкой; проверьте
чувствительность к ширине бина (2--5 мм). o Проведите бутстреп по
столбцам LFQ, чтобы получить доверительные интервалы для L∞ и K. o
Разделите анализ по районам/стратам, если селективность и размерная
структура различаются. • Что дальше в курсе: o Использовать получённые
\emph{L∞}, *K** (и при необходимости \emph{M}/\emph{k}) как
приоры/входные параметры в length-based методах для data limited (LBSPR,
LBB, LIME). o Построить крабо специфичную модель линьки: вероятность
линьки MP(L) и прирост MI(L) по межлиночным категориям; сверить
«эффективный» рост из матрицы переходов с оценками ELEFAN. o Рассчитать
индикаторы по длинам (Lopt, Pmat, Popt, Pmega) и сформировать набор
управленческих ориентиров с явным учётом неопределённости. Итог: при
аккуратной настройке и валидации ELEFAN даёт воспроизводимые и
биологически согласованные оценки роста, достаточные для length based
оценки состояния запаса и первичных управленческих выводов. Для
ракообразных результаты ELEFAN следует трактовать совместно с моделью
линьки и длиновыми индикаторами --- это повышает надёжность при принятии
решений в условиях неполных данных.

\bookmarksetup{startatroot}

\chapter{Оценка естественной
смертности}\label{ux43eux446ux435ux43dux43aux430-ux435ux441ux442ux435ux441ux442ux432ux435ux43dux43dux43eux439-ux441ux43cux435ux440ux442ux43dux43eux441ux442ux438}

\section{Введение}\label{ux432ux432ux435ux434ux435ux43dux438ux435-16}

Это практическое занятие посвящено оценке естественной смертности (M) и
отношения естественной смертности к коэффициенту роста M/k~ (или M/K)
для последующего применения length-based методов в условиях ограниченных
данных (LBSPR, LBB, LIME). Из одних только распределений по длине M и K
по отдельности определяются слабо: разные комбинации ``быстроты роста''
и ``интенсивности фоновой смертности'' могут производить схожие
размерно-частотные профили. Поэтому мы целенаправленно:

1) получаем независимые оценки M из разных источников (эмпирики и
«встроенные» в наши данные методы); 2) объединяем их в единый
согласованный приор на M; 3) переводим его в приор на M/k, используя K
из уравнения роста (например, из
\href{https://mombus.github.io/cRab/chapter20.html}{ELEFAN}).

Почему так: LBSPR и LBB работают именно с M/k, а в LIME M обычно
фиксируют или задают информативный прайер (приор). Для краба-стригуна
опилио (Баренцево море) возрастные признаки недоступны, а съёмочные
индексы имеют вариативный коэффициент уловистости (q). Комбинация
нескольких независимых методов позволяет снизить смещение каждого по
отдельности и явным образом учесть неопределённость.

\section{Определения и биологический
смысл}\label{ux43eux43fux440ux435ux434ux435ux43bux435ux43dux438ux44f-ux438-ux431ux438ux43eux43bux43eux433ux438ux447ux435ux441ux43aux438ux439-ux441ux43cux44bux441ux43b}

- Естественная смертность M (год−1): мгновенная скорость выбывания из
популяции по всем причинам, не связанным с промыслом (хищники, болезни,
старение, экстремальные условия). Годовая выживаемость S ≈ e\^{}(−M).
Общая смертность: Z = M + F, где F -- промысловая смертность, Z --
общая.

- Коэффициент роста уравнения Берталанфи (ВБГФ) \emph{K} (год−1; часто
пишут \emph{k}): задаёт временной масштаб приближения к \emph{L∞}. Чем
больше \emph{K}, тем быстрее достигается близость к асимптоте. Ориентир:
t95 ≈ 3/K --- время до 95\% L∞ (для несезонной формы).

- Отношение M/k (безразмерно): сравнительная «быстрота» фоновой
смертности относительно темпа роста. Большое M/k --- ``быстрый''
жизненный цикл (высокие потери относительно роста), малое ---
``медленный'' (рост доминирует над смертностью). Это ключевой инвариант
в length-based моделях.

\section{Какие методы мы используем и
почему}\label{ux43aux430ux43aux438ux435-ux43cux435ux442ux43eux434ux44b-ux43cux44b-ux438ux441ux43fux43eux43bux44cux437ux443ux435ux43c-ux438-ux43fux43eux447ux435ux43cux443}

1) Эмпирика Pauly* (M как функция L∞, K и температуры T)

- Биологический смысл. Основана на метаболической и
жизненно-исторической масштабируемости: у видов/популяций с меньшим L∞ и
большим K (быстрый рост) фоновые потери обычно выше; в тёплой воде
метаболизм быстрее и M выше. Формула агрегирует большую базу запасов,
давая устойчивый ориентир.

- Зачем в нашем случае. Это «внешняя» информация о M, не зависящая от
специфики наших выборок и q. Для холодноводных бенто-пелагических
сообществ Баренцева моря используем реалистичную сетку придонных
температур (например, 1--3 °C), чтобы получить диапазон M, а не единую
точку.

- Пример (на наших оценках роста): L∞ ≈ 163 мм (16.3 см), K ≈ 0.211
год−1. При T = 1; 2; 3 °C формула Паули даёт M приблизительно 0.16;
0.23; 0.27 год−1. Это разумный ``якорь'' для холодных вод.

- Сильные стороны. Просто, прозрачно, независимая опора на
жизненно-исторические закономерности.

- Ограничения. Формула выведена на рыбах и на длине тела (см), а у нас
ширина карапакса (мм): возможна структурная погрешность шкалы.
Используем как ориентир, а не истину в последней инстанции.

*Pauly, D. (1987). A review of the ELEFAN system for analysis of
length-frequency data in fish and aquatic invertebrates. In: D. Pauly \&
G. R. Morgan (Eds.), Length-based methods in fisheries research
(pp.~7--34). ICLARM Conference Proceedings 13. International Center for
Living Aquatic Resources Management (ICLARM), Manila.

2) Эмпирика Then** по максимальному возрасту tmax

- Биологический смысл. По сути --- обратное от Pauly: вместо температуры
и роста используем ``сколько живут''. Чем меньше tmax, тем выше M, и
наоборот. Это один из самых информативных эмпирических предикторов.

- Проблема для крабов. Истинный tmax редко известен. Выходим так:
аппроксимируем tmax из наблюдаемых максимальных размеров (Lmax или
высоких квантилей, например Q97.5--Q99) через обратный ВБГФ: t(L) = t0 −
(1/K) ln(1 − L/L∞). Это даёт оценку возраста у самых крупных особей.
Она, скорее всего, занижает истинный максимум (из-за выборки,
терминальной линьки, границ по селективности) --- значит, получаем
верхнюю границу M (консервативно вверх).

- Пример. Для Q99 ≈ 150 мм при L∞ ≈ 163 мм, K ≈ 0.211, t0 ≈ 0.8 получаем
tmax ≈ 12.5--13 лет. Тогда по Then (M = 4.899·tmax\^{}−0.916), т.е. ~M ≈
0.4--0.5 год−1. Это выше, чем по Паули, и логично как верхняя граница
(мы взяли ``слишком короткую'' жизнь, а значит M «перестрахованно»
высока).

- Сильные стороны. Задействует независимую информацию о ``долголетии''.

- Ограничения. Очень чувствителен к tmax; оценка tmax из Lq может быть
смещена вниз; даёт верхнюю границу M. Корректнее использовать Q97.5--Q99
и/или прибавлять 1--3 года к tmax по экспертной информации.

**Then, A. Y., Hoenig, J. M., Hall, N. G., \& Hewitt, D. A. (2015).
Evaluating the predictive performance of empirical estimators of natural
mortality rate using information on life history and their application
to fish stocks. ICES Journal of Marine Science, 72(1), 82--92.
https://doi.org/10.1093/icesjms/fsu136

3) Lorenzen *** (M как функция массы: M = c·W\^{}−d)

- Биологический смысл. Множество видов показывает метрическое снижение
смертности с ростом массы: мелкие особи уязвимее к хищникам и стрессам;
крупные --- устойчивее. Это «механистический» взгляд на смертность,
заданный через массу (W = a·L\^{}b).

- Как применяем к опилио. Сначала подгоняем длина--масса W = a·CW\^{}b
по вашим данным (отдельно по полу или суммарно). Затем получаем M(L) =
c·{[}a·L\^{}b{]}\^{}−d и усредняем по диапазону «полной улавливаемости»
промысла (например, L ≥ 110 мм) --- это даёт ``эффективное'' M для
взрослых, релевантное length-based моделям.

- Пример. По вашим коэффициентам для самцов (a ≈ 4.54e−4; b ≈ 2.96)
крабы 120--160 мм весят примерно 0.6--1.5 кг. При ``рыбьих'' константах
c ≈ 3, d ≈ 0.288 типично получается M \textasciitilde{} 0.25--0.45 год−1
по взрослым размерам. Но это важно: c и d получены для рыб; для
ракообразных лучше калибровать хотя бы c так, чтобы ``среднее'' M по
Lorenzen не конфликтовало с Паули/Then и с независимыми оценками. В
наших целях Lorenzen --- источник формы M(L) и ориентиров по взрослой
части размерного спектра.

- Сильные стороны. Даёт длино-специфичную смертность и биологически
реалистичную зависимость ``крупнее --- живут дольше''.

- Ограничения. Требует корректных a,b (длина--масса) и видоспецифичных
c,d; чувствителен к единицам массы.

*** Lorenzen, K. (2000). Allometry of natural mortality as a basis for
assessing optimal release size in fish-stocking programmes. Canadian
Journal of Fisheries and Aquatic Sciences, 57(12), 2374--2381.
https://doi.org/10.1139/f00-215

4) Length-converted catch curve: Z из нисходящей ветви + M = Z − F
(через C/B при неопределённом q)

- Биологический смысл. На стационарном массиве с постоянной
селективностью и пополнении нисходящая ветвь размерной частоты (правее
моды) убывает экспоненциально: наклон в шкале ln N --- это −Z. Мы
«переводим» длину в возраст через ВБГФ и регрессируем ln(n/Δt) по t в
правой части. Получаем Z = M + F.

- Как отделяем M от F. Используем ежегодные выловы C и индекс съёмки I =
q·B. Если q был бы известен, F ≈ −ln(1 − C/B). Но q переменный, поэтому
задаём логнормальный приор q \textasciitilde{} Lognormal(median, CV),
делаем Монте‑Карло: B = I/q, F = −ln(1 − C/B), M = Z − F. На выходе ---
распределение M, отражающее неопределённость в q.

- Пример. Если по съёмке получаем Z ≈ 0.55 год−1, а при
q\textasciitilde Lognormal(медиана 0.3, CV 0.5) типичные F оказываются
0.20--0.35 год−1, то M = Z − F ``ложится'' в 0.20--0.35 год−1, что
хорошо согласуется с диапазоном Паули и ниже ``верхней границы'' Then.

- Сильные стороны. Прямо использует ваши эмпирические ряды по длине,
выловы и индексы --- то есть укоренена в данных конкретной популяции.

- Ограничения. Требует стационарности и стабильной селективности на
правой ветви; чувствительна к выбору диапазона длин; зависит от
предположений о q (мы честно вносим их в неопределённость).

\section{Как мы объединяем источники и выходим на
M/k}\label{ux43aux430ux43a-ux43cux44b-ux43eux431ux44aux435ux434ux438ux43dux44fux435ux43c-ux438ux441ux442ux43eux447ux43dux438ux43aux438-ux438-ux432ux44bux445ux43eux434ux438ux43c-ux43dux430-mk}

- Мы собираем все M-оценки: Pauly (сетка T), Then (через tmax из Lq),
Lorenzen (при наличии корректных a,b; усреднённое M по взрослым длинам),
Z−F (MC по q).

- Режем экстремальные хвосты (например, 99-й перцентиль) и
аппроксимируем приор на M логнормальным распределением (лог‑среднее и
лог‑SD).

- Получаем сводную оценку M (медиана и 95\% интервал) и переводим её в
M/k, деля на K из роста (ELEFAN). Именно M/k нужен напрямую LBSPR/LBB, а
в LIME его удобно использовать как проверку/настройку фиксированного M.

\section{Зачем нам сразу несколько
методов}\label{ux437ux430ux447ux435ux43c-ux43dux430ux43c-ux441ux440ux430ux437ux443-ux43dux435ux441ux43aux43eux43bux44cux43aux43e-ux43cux435ux442ux43eux434ux43eux432}

- Они опираются на разные «оси» биологии: температура и темп жизни
(Pauly), долгожительство (Then), размер и плотность энергии (Lorenzen),
фактический профиль размерной убыли и эксплуатация (catch curve + C/B).

- Смещения у них разнонаправленные: Then с Lq даёт верхние границы M;
Pauly в холодных водах --- умеренные; Lorenzen --- структуру по размеру;
Z−F --- якорение в ваших данных, но с неопределённым q. Комбинация
стабилизирует итог и делает приор на M (а затем и M/k) ``честно
широким'', но информативным.

\section{Практические ориентиры и проверка здравого
смысла}\label{ux43fux440ux430ux43aux442ux438ux447ux435ux441ux43aux438ux435-ux43eux440ux438ux435ux43dux442ux438ux440ux44b-ux438-ux43fux440ux43eux432ux435ux440ux43aux430-ux437ux434ux440ux430ux432ux43eux433ux43e-ux441ux43cux44bux441ux43bux430}

- Согласованность с ростом: t95 ≈ 3/K. Если K ≈ 0.21, то t95 ≈ 14 лет
--- M «очевидно» не может быть чрезмерно высоким, иначе структура длин
бы «ломалась».

- Согласованность источников: разумно ожидать, что медиана M попадёт в
«общий» коридор Pauly и Z−F, а Then задаст верхнюю границу.

- Согласованность с литературой по опилио: взрослые самцы в холодных
морях --- M обычно низко‑средний (около 0.15--0.35 год−1), с вариациями
по районам и периодам.

\section{Итог}\label{ux438ux442ux43eux433}

- Pauly даёт внешний температурно‑жизненно‑исторический ``якорь'' M.

- Then даёт «потолок» M через tmax, особенно полезен как проверка --- не
завышаем ли долгожительство.

- Lorenzen добавляет длино‑специфичность и биологическую
правдоподобность ``крупнее --- живут дольше'' (важно для взрослых самцов
опилио).

- Length‑converted catch curve и C/B с неопределённым q якорят оценку M
в ваших данных (съёмка+промысел) и честно вносят q в интервал
неопределённости.

- Комбинируя их, мы получаем прозрачный и согласованный приор на M и
производный приор на M/k --- именно то, что нужно для дальнейших
length‑based оценок (LBSPR, LBB, LIME) в условиях ограниченных данных и
меняющегося q в Баренцевом море.

\section{Скрипт и входные
данные}\label{ux441ux43aux440ux438ux43fux442-ux438-ux432ux445ux43eux434ux43dux44bux435-ux434ux430ux43dux43dux44bux435}

\href{https://mombus.github.io/cRab/data/M_all_methods.R}{Скрипт}.

ВХОДНЫЕ ДАННЫЕ (должны быть в рабочей директории):

\begin{itemize}
\tightlist
\item
  \href{https://mombus.github.io/cRab/data/SURVEYDATA.csv}{SURVEYDATA.csv}
  - данные съемки (длины особей)
\item
  \href{https://mombus.github.io/cRab/data/FISHERYDATA.csv}{FISHERYDATA.csv}
  - данные промысла (длины особей)
\item
  \href{https://mombus.github.io/cRab/data/CATCH.csv}{CATCH.csv} -
  годовые уловы (тонны)
\item
  \href{https://mombus.github.io/cRab/data/SURVEY_INDEX.csv}{SURVEY\_INDEX.csv}
  - индексы биомассы по годам
\item
  \href{https://mombus.github.io/cRab/data/ELEFAN_params_constrained.csv}{ELEFAN\_params\_constrained.csv}
  - параметры роста (Linf, K, t\_anchor)
\item
  \href{https://mombus.github.io/cRab/data/LW_coeffs.csv}{LW\_coeffs.csv}
  - коэффициенты длина-масса
\end{itemize}

\bookmarksetup{startatroot}

\chapter{Оценка селективности
промысла}\label{ux43eux446ux435ux43dux43aux430-ux441ux435ux43bux435ux43aux442ux438ux432ux43dux43eux441ux442ux438-ux43fux440ux43eux43cux44bux441ux43bux430}

\section{Введение}\label{ux432ux432ux435ux434ux435ux43dux438ux435-17}

Это практическое занятие посвящено оценке селктивности промысла (SL50 и
SL95) для последующего применения length-based методов в условиях
ограниченных данных (LBSPR, LBB, LIME). В оценке водных биоресурсов,
особенно при работе с неполными данными, существенным фактором, который
часто упускают из виду или прячут в ``черный ящик'' модели, является
селективность промысла. Это не абстрактная статистическая величина, а
физический фильтр, через который проходит реальная популяция при
взаимодействии с орудием лова. Представьте себе рыболовную сеть: она не
ловит все особи одинаково, как идеальный сортировщик. Ячейки сетного
полотна пропускают мелких особей, задерживают средних, а крупные особи
либо избегают и уходят из орудия лова, либо попадают в нее с разной
вероятностью. Если мы не учтем этот фильтр, любые оценки биомассы, MSY,
или динамики запаса будут систематически искажены, как если бы мы
пытались измерить температуру комнаты термометром, который не
откалиброван и показывает значения на 5 градусов выше реальных. В
контексте length-based моделей --- таких как LBSPR, LBB, LIME ---
селективность становится важной, потому что эти модели оперируют
распределением по длине, а не по возрасту, что особенно актуально для
беспозвоночных и рыб, где возрастная структура труднодоступна для
оценки. Без точного знания того, как орудие лова ``фильтрует'' особей по
длине (размеру), мы не сможем правильно оценить истинную размерную
структуру популяции, а значит, не сможем адекватно оценить параметры
роста, смертности и воспроизводства. Это не просто техническая деталь
--- это своего рода фундамент, на котором строится вся последующая
оценка. В реальности селективность не постоянна: она меняется в
зависимости от размера особей, пола, возраста, поведения, характеристик
орудия лова, условий промысла, сезона и региона. В наших данных всегда
есть шум, пропуски, аномалии, и наша задача --- не прятать эту
неопределенность, а честно ее оценить и визуализировать. В этом
практическом задании мы оцениваем параметры логистической селективности
по длине: SL50 (длина, при которой 50\% особей улавливаются орудием
лова) и SL95 (длина, при которой 95\% особей улавливаются). Эти
параметры не абстрактные числа --- это биологически значимые величины,
которые напрямую влияют на понимание того, как промысел воздействует на
популяцию. SL50 --- это длина, при которой снасть становится
эффективной, а SL95 --- длина, при которой снасть практически не
пропускает особей. Эти значения используются в length-based моделях для
коррекции наблюдаемого распределения по длине и получения истинной
структуры популяции. Скрипт начинается с загрузки данных съёмки, которая
отражает доступность особей в популяции, и данных промысла, которая
отражает выборку, прошедшую через селективный фильтр снасти. Мы
фильтруем данные по полу, годам, месяцам, чтобы учесть сезонные и
региональные особенности. Затем создаем бины по длине, что позволяет
работать с дискретными интервалами, а не с непрерывными величинами, что
особенно важно для данных с ограниченной точностью измерений.
Селективность моделируется логистической кривой: S(L) = 1 / {[}1 +
exp(-k(L - SL50)){]}, где k = ln(19)/SR, а SR = SL95 - SL50. Это не
случайный выбор --- логистическая кривая хорошо описывает процесс
улавливания, где вероятность попадания в снасть плавно растет с
увеличением длины, с четким порогом в SL50. Мы подбираем параметры SL50
и SR так, чтобы вероятность наблюдать промысловые длины была
максимальна, используя мультиномиальное правдоподобие. Для устойчивости
оценки мы сглаживаем нули в распределении съёмки добавкой 0.5, что
предотвращает деление на ноль и обеспечивает корректное масштабирование.
Мы также проводим бутстреп-анализ для оценки доверительных интервалов,
потому что в реальном мире нет идеальных данных, и наша оценка должна
отражать эту неопределенность. Результаты скрипта --- это не просто
числа в таблице, а биологически интерпретируемая информация: SL50 и SL95
с доверительными интервалами, которые показывают, насколько уверенно мы
можем говорить о параметрах селективности. Первый график показывает
эмпирическое отношение промыслового и съёмочного распределений в
сравнении с теоретической логистической кривой. Если кривая не совпадает
с данными --- это сигнал о том, что модель неверна или данные нужно
перепроверить. Второй график визуализирует саму селективную кривую с
отметками SL50 и SL95, что позволяет сразу увидеть, как снасть
``фильтрует'' особей по длине. Эти графики --- прозрачный инструмент
коммуникации, который позволяет биологам, менеджерам и промысловикам
сразу понять, как работает снасть, без необходимости разбираться в
статистических деталях. Важно помнить, что результаты этого анализа ---
только первый шаг. Нужно проверить их на биологическую правдоподобность:
например, SL50 для краба не должен быть меньше размера маленьких особей
или больше размера взрослых. Нужно сопоставить с другими источниками
данных, учесть сезонные изменения, проверить чувствительность к выбору
бинов. Это часть дисциплины --- всегда проверять свои допущения, не
верить результатам слепо, учитывать неопределенность. В оценке запасов,
как и в любой научной работе, нет места иллюзиям контроля. Мы не можем
знать истинную селективность с абсолютной точностью, но можем честно
оценить ее с учетом доступных данных и неопределенности. Это не
слабость, а сила: понимание того, что мы не знаем, позволяет принимать
более устойчивые решения, которые не рухнут при изменении условий. В
этом и состоит суть научного подхода --- не в поиске идеальных ответов,
а в честном признании неопределенности и построении решений, которые
остаются работоспособными даже в условиях неполных данных.

\section{Скрипт и входные
данные}\label{ux441ux43aux440ux438ux43fux442-ux438-ux432ux445ux43eux434ux43dux44bux435-ux434ux430ux43dux43dux44bux435-1}

\href{https://mombus.github.io/cRab/data/estimate_selectivity_from_survey_vs_fishery.R}{Скрипт}.

ВХОДНЫЕ ДАННЫЕ (должны быть в рабочей директории):

\begin{itemize}
\tightlist
\item
  \href{https://mombus.github.io/cRab/data/SURVEYDATA.csv}{SURVEYDATA.csv}
  - данные съемки (длины особей)
\item
  \href{https://mombus.github.io/cRab/data/FISHERYDATA.csv}{FISHERYDATA.csv}
  - данные промысла (длины особей)
\end{itemize}

\begin{center}
\includegraphics[width=0.95\linewidth,height=\textheight,keepaspectratio]{images/Selectivity.PNG}
\end{center}

\bookmarksetup{startatroot}

\chapter{Определение параметров зрелости краба-стригуна опилио: L50 и
L95}\label{ux43eux43fux440ux435ux434ux435ux43bux435ux43dux438ux435-ux43fux430ux440ux430ux43cux435ux442ux440ux43eux432-ux437ux440ux435ux43bux43eux441ux442ux438-ux43aux440ux430ux431ux430-ux441ux442ux440ux438ux433ux443ux43dux430-ux43eux43fux438ux43bux438ux43e-l50-ux438-l95}

\section{Введение}\label{ux432ux432ux435ux434ux435ux43dux438ux435-18}

Это практическое занятие посвящено оценке параметров зрелости
краба-стригуна опилио: L50 и L95 для последующего применения
length-based методов в условиях ограниченных данных (LBSPR, LBB, LIME).
Задача проста: понять, при каких размерах эти членистоногие начинают
активно размножаться, чтобы мы могли грамотно их вылавливать, оставляя
достаточное количество производителей для поддержания популяции.

Методика анализа напоминает попытку разделить подростков на ``еще
детей'' и ``уже взрослых'' по размеру одежды --- вроде и работает, но
всегда есть пара особей, которые ни в какие рамки не лезут. Мы
используем кластерный анализ на основе t-распределения, который как раз
терпимо относится к таким статистическим аутсайдерам.

После того как машинное обучение более-менее уверенно разделит крабов на
``зеленую молодежь'' и ``зрелых товарищей'', мы построим логистическую
кривую. Она нам и покажет, при каких размерах большинство особей
начинает интересоваться вопросами продолжения рода вместо простого
наращивания биомассы.

L50 --- длина карапакса, при которой 50\% особей в популяции достигают
половой зрелости --- служит биологической основой, например, для
установления минимального промыслового размера. Этот параметр
обеспечивает возможность хотя бы одного нереста до вылова, что иногда
является ключевым условием устойчивого рыболовства. Например, для
камчатского краба L50 составляет около 120-130 мм, что и определяет
соответствующие промысловые нормативы. 150 мм по ширине карапкса -
начало промыслового размера, позволяющего особи учавствовать в нересте
плюс-минус два сезона.

L95 --- размер, при котором 95\% особей становятся половозрелыми ---
используется для более тонкой настройки регуляторных мер. Этот
показатель особенно важен при внедрении селективного промысла, когда
необходимо максимально точно определить размерные группы, подлежащие
изъятию.

Практическое применение этих параметров многогранно. При установлении
минимальной ячейки тралов сетное полотно калибруется таким образом,
чтобы избежать вылова неполовозрелых особей. В случае с краболовными
ловушками конструктивные особенности должны обеспечивать свободный выход
молоди.

Эти же показатели могут использоваться при расчете промыслового усилия и
общего допустимого улова. Зная размерно-возрастную структуру популяции и
параметры зрелости, можно более точно прогнозировать репродуктивный
потенциал и устанавливать лимиты, обеспечивающие воспроизводство.

Интересный аспект --- пространственное распределение параметров
зрелости. В разных частях ареала краба-стригуна опилио значения L50
могут значительно варьировать из-за различий в температурном режиме,
кормовой базе и плотности популяции. Это может потребовать разработки
дифференцированных мер регулирования для различных промысловых районов.

Особую важность эти параметры приобретают в условиях меняющегося
климата. Наблюдаемое смещение сроков и размеров полового созревания
требует регулярного пересмотра установленных нормативов. Мониторинг
динамики L50 и L95 становится индикатором адаптационных процессов в
популяции.

Таким образом, за сухими статистическими показателями скрывается сложная
система биологических взаимосвязей, понимание которых позволяет
балансировать между экономической эффективностью промысла и
экологической устойчивостью популяции. Определение этих параметров ---
не академическое упражнение, а необходимое условие рационального
использования морских биоресурсов.

Что ж, приступим к этому увлекательному занятию --- подсчету крабьих
карапаксов и построению кривых зрелости. Наука редко бывает романтичной,
но от этого не менее необходимой.

\section{Скрипт и входные
данные}\label{ux441ux43aux440ux438ux43fux442-ux438-ux432ux445ux43eux434ux43dux44bux435-ux434ux430ux43dux43dux44bux435-2}

\href{https://mombus.github.io/cRab/data/SnowMaturity.R}{Скрипт}.

ВХОДНЫЕ ДАННЫЕ (должны быть в рабочей директории):

\begin{itemize}
\tightlist
\item
  \href{https://mombus.github.io/cRab/data/SNOW.xlsx}{SNOW.xlsx} -
  данные промеров ширины карапаксов и высоты клешни
\end{itemize}

\begin{center}
\includegraphics[width=0.75\linewidth,height=\textheight,keepaspectratio]{images/SnowMaturity.PNG}
\end{center}

\bookmarksetup{startatroot}

\chapter{III. DLM: LBSPR}\label{iii.-dlm-lbspr}

\section{Введение}\label{ux432ux432ux435ux434ux435ux43dux438ux435-19}

В рыбохозяйственных исследованиях часто возникает ситуация, когда данных
катастрофически не хватает, но оценку запасов проводить необходимо. Вот
здесь и приходят на помощь основанные на размерах (length-based) методы,
которые извлекают максимум информации из минимального набора данных ---
по сути, пытаются восстановить скелет по нескольким костям. Эти методы
работают с размерной структурой уловов, позволяя оценивать состояние
запасов там, где традиционные подходы просто неприменимы.

Возьмем LBSPR --- метод, основанный на анализе репродуктивного
потенциала через размерные показатели. Он требует знания биологических
параметров вида, но зато дает оценку промысловой смертности и
соотношения половозрелой части популяции. Это как пытаться определить
возраст дерева по толщине ствола --- приблизительно, но лучше чем
ничего. Метод особенно полезен для видов с выраженным размерным
диморфизмом, где размер четко коррелирует с половой зрелостью.

LBB представляет собой еще более элегантное решение --- он работает
вообще без каких-либо предварительных данных о виде. Чистая байесовская
(Bayesian) статистика, которая из одного лишь размерного состава улова
извлекает оценки и промысловой смертности, и естественной смертности, и
даже оптимального размера ячеи орудий лова. Это напоминает гадание на
кофейной гуще, но статистически обоснованное --- метод строит
вероятностные распределения и показывает, какие параметры наиболее
вероятны при имеющихся данных.

LIME занимает промежуточное положение, позволяя интегрировать
разрозненные данные за несколько лет и учитывать пространственную
неоднородность. Он особенно полезен при мониторинге видов с выраженной
межгодовой динамикой, где простое усреднение показателей может дать
искаженную картину.

Практическое применение этих методов сталкивается с неизбежными
ограничениями. Все они предполагают, что размерный состав улова
репрезентативен для всей популяции, что далеко не всегда соответствует
действительности. Селективность орудий лова, пространственная
неоднородность распределения, сезонные миграции --- все это вносит
искажения в данные. Для ракообразных, таких как краб-стригун,
добавляются дополнительные сложности: дискретный рост во время линьки,
сильно выраженный половой диморфизм, разные темпы роста самцов и самок.

Тем не менее, эти методы остаются незаменимым инструментом для
data-limited fisheries --- ситуаций, когда данные ограничены, но
управлять промыслом нужно. Они позволяют проводить оперативный
мониторинг состояния запасов, оценивать эффективность мер регулирования
и, что особенно важно, определять направления для более детальных
исследований. Как говорят специалисты, лучше приблизительный ответ на
правильный вопрос, чем точный ответ на неправильный.

На практике рекомендуется использовать несколько методов параллельно ---
их сравнение позволяет оценить устойчивость результатов и выявить
потенциальные проблемы с данными. LBSPR хорош для оценки репродуктивного
потенциала, LBB --- для оперативного мониторинга, LIME --- для анализа
многолетней динамики. Важно помнить, что все эти методы дают
относительные оценки, а не абсолютные значения, и их сила в сравнении, а
не в абсолютных цифрах.

В конечном счете, искусство оценки запасов заключается не в применении
самых сложных моделей, а в понимании того, какая информация
действительно содержится в имеющихся данных --- и length-based методы
предоставляют для этого мощный и гибкий инструментарий.

\section{Что
делаем}\label{ux447ux442ux43e-ux434ux435ux43bux430ux435ux43c}

LBSPR (Length-Based Spawning Potential Ratio) - это современный метод
оценки состояния рыбных запасов, разработанный Hordyk et al.~(2015,
2016). Метод позволяет оценить интенсивность промысла и состояние
запаса, используя только данные о размерной структуре уловов и базовые
биологические параметры.

\textbf{Основная концепция SPR (Spawning Potential Ratio):} - SPR
показывает, какая доля репродуктивного потенциала популяции сохраняется
при текущем уровне промысла по сравнению с неэксплуатируемой популяцией
- SPR = 1.0 означает девственную популяцию (нет промысла) - SPR = 0.4
считается целевым уровнем для устойчивого промысла - SPR \textless{} 0.2
указывает на критический перелов

\textbf{Как работает LBSPR:} 1. \textbf{Моделирование равновесной
популяции}: Метод предполагает, что популяция находится в равновесном
состоянии при текущем уровне промысловой смертности 2. \textbf{Анализ
формы размерного распределения}: По соотношению мелких, средних и
крупных особей в улове метод определяет: - Насколько интенсивен промысел
(F/M - отношение промысловой к естественной смертности) - Какая часть
репродуктивного потенциала сохраняется (SPR) 3. \textbf{Использование
модели GTG (Growth-Type-Group)}: Учитывает индивидуальную изменчивость в
росте особей, что делает оценки более реалистичными

\section{Описание работы
скрипта}\label{ux43eux43fux438ux441ux430ux43dux438ux435-ux440ux430ux431ux43eux442ux44b-ux441ux43aux440ux438ux43fux442ux430}

\begin{enumerate}
\def\labelenumi{\arabic{enumi}.}
\tightlist
\item
  Подготовка данных (строки 1-100) \textbf{Загрузка и обработка данных:}
\end{enumerate}

\begin{itemize}
\tightlist
\item
  Загружаются 8 CSV файлов с данными о промысле, съемках, уловах,
  параметрах роста
\item
  Данные фильтруются по полу (только самцы - ``M'')
\item
  Длины карапакса приводятся к числовому формату
\item
  Определяются параметры:

  \begin{itemize}
  \tightlist
  \item
    Linf = 163.4 мм (асимптотическая длина)
  \item
    M/K = 1.085 (отношение естественной смертности к коэффициенту роста)
  \item
    L50 зрелости = 73 мм
  \item
    L50 селективности = 96 мм
  \end{itemize}
\end{itemize}

\begin{enumerate}
\def\labelenumi{\arabic{enumi}.}
\setcounter{enumi}{1}
\tightlist
\item
  Анализ методом LBSPR (строки 107-175) \textbf{Процесс анализа:} а.
  \textbf{Создание размерных классов}: Данные группируются в бины по 2
  мм (от 40 до 220 мм) б. \textbf{Формирование матрицы длинных
  композиций}: Для каждого года (2013-2024) подсчитывается количество
  особей в каждом размерном классе в. \textbf{Настройка параметров
  модели}:

  \begin{itemize}
  \tightlist
  \item
    Параметры роста (Linf, M/K)
  \item
    Параметры зрелости (L50=73, L95=126 мм)
  \item
    Параметры селективности промысла (SL50=96, SL95=112 мм)
  \item
    Коэффициент вариации длины (CV=0.1) г. \textbf{Запуск модели LBSPR}:
    Итеративная подгонка модели к наблюдаемым данным для оценки SPR и
    F/M
  \end{itemize}
\item
  Расчет дополнительных индикаторов (строки 177-295)
\end{enumerate}

\textbf{Индикаторы по Froese (2004):} - \textbf{Pmat} - доля
половозрелых особей в улове - \textbf{Popt} - доля особей оптимального
размера (\textgreater0.9×Linf) - \textbf{Pmega} - доля мегаспаунеров
(\textgreater Linf)

\textbf{Индикаторы размерной структуры:} - \textbf{Lmean/Linf} - средняя
длина относительно асимптотической - \textbf{L95/Linf} - 95-й процентиль
длины - Анализ трендов за последние 5 лет

\section{Интерпретация
результатов}\label{ux438ux43dux442ux435ux440ux43fux440ux435ux442ux430ux446ux438ux44f-ux440ux435ux437ux443ux43bux44cux442ux430ux442ux43eux432-1}

Результаты LBSPR

\textbf{Ключевые показатели:} - \textbf{Средний SPR = 0.423} (за весь
период) - \textbf{SPR в 2024 году = 0.353} - \textbf{F/M в 2024 году =
2.03} \textbf{Что это означает:} - SPR = 0.353 \textless{} 0.4 указывает
на \textbf{приближение к перелову} - Только 35\% репродуктивного
потенциала сохраняется - F/M = 2.03 означает, что промысловая смертность
в 2 раза выше естественной - это \textbf{высокая интенсивность промысла}

Анализ размерной структуры \textbf{Индикаторы за 2022-2024 годы:} -
\textbf{Lmean/Linf = 0.661} - средний размер составляет только 66\% от
максимального - \textbf{99.5\% особей половозрелые} - хорошая
селективность, молодь защищена - \textbf{0.1\% крупных особей} -
практически нет особей оптимального размера - \textbf{0\% мегаспаунеров}
- полное отсутствие самых крупных производителей Тренды - Средняя длина
\textbf{снижается на 0.6 мм/год} за последние 5 лет - Это негативный
тренд, указывающий на усиление пресса промысла

Сводная таблица результатов:

\begin{longtable}[]{@{}llll@{}}
\toprule\noalign{}
Индикатор & Значение & Целевой уровень & Статус \\
\midrule\noalign{}
\endhead
\bottomrule\noalign{}
\endlastfoot
SPR & 0.353 & ≥0.40 & \textbf{Перелов} \\
F/M & 2.03 & ≤1.0 & \textbf{Перелов} \\
Lmean/Linf & 0.661 & ≥0.70 & \textbf{Перелов} \\
\% зрелых & 99.5\% & ≥90\% & OK \\
\% оптимальных & 0.1\% & ≥30\% & \textbf{Критически низкая} \\
\end{longtable}

Заключение

\textbf{Состояние запаса: УМЕРЕННЫЙ ПЕРЕЛОВ с признаками ухудшения}
\textbf{Основные проблемы:}

\begin{enumerate}
\def\labelenumi{\arabic{enumi}.}
\tightlist
\item
  \textbf{Перелов крупных особей} - в популяции практически отсутствуют
  крупные высокопродуктивные производители
\item
  \textbf{Высокая промысловая смертность} - F/M в 2 раза превышает
  рекомендуемый уровень
\item
  \textbf{Снижение SPR ниже целевого уровня} - репродуктивный потенциал
  под угрозой
\item
  \textbf{Негативный тренд} - средние размеры продолжают снижаться
\end{enumerate}

\textbf{Положительные моменты:} - Хорошая селективность - молодь
защищена от промысла - Большинство особей успевает достичь половой
зрелости

Рекомендации для управления

\begin{enumerate}
\def\labelenumi{\arabic{enumi}.}
\tightlist
\item
  \textbf{Снизить интенсивность промысла} для восстановления SPR до
  уровня ≥0.4
\item
  \textbf{Защитить крупных производителей} - рассмотреть введение
  максимального размера
\item
  \textbf{Мониторинг тренда} - если снижение средних размеров
  продолжится, необходимы срочные меры
\item
  \textbf{Целевой ориентир} - довести долю особей оптимального размера
  до 30\%
\end{enumerate}

Данный анализ показывает, что запас находится в состоянии умеренного
перелова с тенденцией к ухудшению, требются корректирующие меры
управления для обеспечения устойчивости промысла.

\section{Скрипт и входные
данные}\label{ux441ux43aux440ux438ux43fux442-ux438-ux432ux445ux43eux434ux43dux44bux435-ux434ux430ux43dux43dux44bux435-3}

\href{https://mombus.github.io/cRab/data/LBSPR.R}{Скрипт}.

ВХОДНЫЕ ДАННЫЕ (должны быть в рабочей директории):

\begin{enumerate}
\def\labelenumi{\arabic{enumi}.}
\tightlist
\item
  SURVEY \textless-
  \href{https://mombus.github.io/cRab/data/SURVEYDATA.csv}{SURVEYDATA.csv}
\item
  FISHERY \textless-
  \href{https://mombus.github.io/cRab/data/FISHERYDATA.csv}{FISHERYDATA.csv}
\item
  CATCH \textless-
  \href{https://mombus.github.io/cRab/data/CATCH.csv}{CATCH.csv}
\item
  SINDEX \textless-
  \href{https://mombus.github.io/cRab/data/SURVEY_INDEX.csv}{SURVEY\_INDEX.csv}
\item
  GROWTH \textless-
  \href{https://mombus.github.io/cRab/data/ELEFAN_params_constrained.csv}{ELEFAN\_params\_constrained.csv}
\item
  MK\_PRI \textless-
  \href{https://mombus.github.io/cRab/data/MK_prior_summary.csv}{MK\_prior\_summary.csv}
\item
  M\_PRI \textless-
  \href{https://mombus.github.io/cRab/data/M_prior_summary.csv}{M\_prior\_summary.csv}
\item
  LWCOEF \textless-
  \href{https://mombus.github.io/cRab/data/LW_coeffs.csv}{LW\_coeffs.csv}
\end{enumerate}

\bookmarksetup{startatroot}

\chapter{Моделирование дрейфа
личинок}\label{ux43cux43eux434ux435ux43bux438ux440ux43eux432ux430ux43dux438ux435-ux434ux440ux435ux439ux444ux430-ux43bux438ux447ux438ux43dux43eux43a}

\begin{center}
\includegraphics[width=0.75\linewidth,height=\textheight,keepaspectratio]{images/LarvaeDrifting.PNG}
\end{center}

\section{Введение}\label{ux432ux432ux435ux434ux435ux43dux438ux435-20}

В фокусе нашего сегодняшнего рассмотрения находится вероятностное
моделирование Лагранжева переноса --- методология, занимающая
центральное место в современной физической экологии и являющаяся
основным инструментом для исследования процессов распространения морских
организмов на ранних стадиях онтогенеза. Давайте разберемся, что
скрывается за этим сложным термином. В основе его лежит фундаментальное
разделение способов описания движения. Представьте себе океан. Вы можете
изучать его, стоя на капитанском мосту и измеряя скорость и направление
течения в фиксированных точках пространства --- это Эйлеров подход,
описывающий поле скоростей в координатах x, y, z. Но если вы хотите
понять судьбу конкретной частицы, например, личинки краба, вы должны
сесть на нее верхом и отправиться в путешествие, описывая изменение ее
координат во времени --- это и есть Лагранжев подход. Таким образом,
Лагранжев перенос --- это метод траекторного моделирования, который
позволяет нам реконструировать путь, проходимый условной частицей в уже
известном поле течений.

Однако океан --- среда турбулентная и вероятностная по своей природе.
Если бы мы просто перемещали частицу со средней скоростью течения, мы бы
получили идеализированную, прямолинейную траекторию, которая имеет очень
мало общего с реальностью. Реальная личинка подвергается воздействию
целого спектра процессов, которые не отражены в осредненных полях
скоростей: это и турбулентная диффузия, и ветровое дрейфовое течение, и
мелкомасштабные вихри, и даже собственное поведение организма. Именно
поэтому чисто детерминистическая Лагранжева модель была бы серьезным
упрощением. Чтобы приблизиться к реальности, мы вводим вероятностную,
или стохастическую, составляющую. По сути, мы признаем, что наше знание
о поле течений неполно, и эта неполнота моделируется путем добавления
случайных возмущений к основным параметрам --- скорости и направлению
движения.

В контексте данной модели эти возмущения вносятся двумя основными
способами. Во-первых, это добавление случайной компоненты к направлению
переноса. Вместо того чтобы строго следовать интерполированному
направлению течения, частица на каждом шаге получает случайное
отклонение, обычно распределенное по нормальному закону с нулевым
средним, но с стандартным отклонением, которое задается на основе данных
о естественной изменчивости течений. Во-вторых, аналогичным образом
случайно модулируется и скорость движения частицы. Это позволяет учесть
тот факт, что мгновенная скорость потока может отклоняться от
осредненного по климатологии или по, например, декаде (10 дней)
значения. Таким образом, каждая частица из одного и того же пункта
выпуска проходит уникальный путь, и ансамбль из тысяч таких
смоделированных траекторий образует так называемое «облако вероятности»,
которое визуализируется в виде карт плотности распределения.

Интерпретация результатов такой модели имеет принципиально вероятностный
характер. Мы не можем сказать точно, куда приплывет личинка из точки
\emph{А}; мы можем сказать, что с вероятностью, скажем, 70\% она
окажется в радиусе 50 км от точки \emph{B} через 30 дней, или что лишь
5\% личинок достигнут отдаленного шельфового полигона. Этот вывод важен
для решения целого ряда прикладных задач. Например, для оценки связности
между разными участками популяции: модель показывает, возможен ли обмен
личинками между двумя удаленными друг от друга банками или отмелями с
зарослями водорослей, или же эти субпопуляции изолированы. Для
планирования морских охраняемых районов (МРА) --- чтобы понять,
насколько эффективно тот или иной МРА может действовать как источник
рекрутов для окружающих акваторий. Наконец, в условиях меняющегося
климата такие модели позволяют строить прогнозы смещения ареалов видов и
перестройки трофических цепей из-за изменения переноса на ранних стадиях
жизни. В итоге, стохастическое Лагранжево моделирование превращается из
чисто математического упражнения в мощный инструмент экологического
прогнозирования, позволяющий заглянуть в будущее морских экосистем и
дать количественную оценку тем процессам, которые мы не можем
непосредственно наблюдать в полном объеме.

\section{Скрипт и входные
данные}\label{ux441ux43aux440ux438ux43fux442-ux438-ux432ux445ux43eux434ux43dux44bux435-ux434ux430ux43dux43dux44bux435-4}

Давайте теперь подробно разберем, как именно работает этот сложный
вычислительный организм --- наша модель. Представьте, что мы проводим
грандиозный виртуальный эксперимент, и я буду вашим гидом по его
основным этапам. Все начинается с подготовки лаборатории: мы очищаем
рабочее пространство от старых данных и загружаем весь необходимый
инструментарий --- специальные библиотеки для чтения сеточных данных в
формате NetCDF, для их обработки, анализа и, что крайне важно, для
визуализации. Без этого наш дальнейший анализ был бы просто невозможен.
Затем мы подгружаем сами данные (4 файла
\href{https://www.bio-oracle.org/downloads-to-email.php}{Bio-ORACLE}), а
это, напомню, массивные многомерные массивы, содержащие информацию о
средних значениях и размахе скорости и направления течений по всему
мировому океану. Но поскольку наша цель --- Баренцево море, мы совершаем
первую важную операцию --- обрезаем эти глобальные данные до размеров
нашего региона интереса, чтобы не тратить вычислительные ресурсы на
обработку лишней информации. Это как взять подробную карту мира и
вырезать из нее нужный нам квадрат. Но и эти данные неидеальны --- в них
могут встречаться пропуски или артефакты, поэтому мы проводим их
очистку, заменяя аномальные значения на физически осмысленные
консервативные оценки.

\begin{center}
\includegraphics[width=0.6\linewidth,height=\textheight,keepaspectratio]{images/CurrBar.png}
\end{center}

Теперь, когда у нас есть готовое к работе поле течений, мы определяем
правила игры --- набор функций, которые и являются математическим
сердцем модели. Ключевая из них --- функция билинейной интерполяции. Ее
задача --- быть переводчиком между дискретной сеткой наших данных и
непрерывным миром, в котором движется личинка. Частица может находиться
в любой точке, а не обязательно в узле сетки, где нам известны параметры
течения. Эта функция берет четыре ближайших узла и вычисляет взвешенное
среднее, плавно определяя скорость и направление течения именно в той
точке, где находится наша виртуальная личинка в данный момент времени.
Следующий важный блок --- это функции, рассчитывающие перемещение. Одна
из них, основанная на формуле гаверсинуса, умеет точно вычислять
расстояние между двумя точками на сфере, что принципиально важно для
корректности моделирования на планете Земля. А вторая --- это уже
непосредственно закон движения, который, зная текущую позицию, скорость,
направление течения и временной шаг, вычисляет новую координату. И здесь
в игру вступает та самая стохастичность: к интерполированным значениям
скорости и направления мы добавляем случайные возмущения, параметры
которых заданы на основе данных о естественной изменчивости течений.
Именно это позволяет каждой частице из одной и той же начальной точки
пойти по своему уникальному пути, имитируя воздействие турбулентности и
других неучтенных процессов.

Наконец, мы запускаем сам эксперимент. Мы задаем координаты точек
выпуска личинок --- тех мест, где, как мы полагаем, происходит нерест. И
затем для каждой точки мы выпускаем целое облако из сотни или тысячи
виртуальных частиц-трассеров. Для каждой частицы мы в цикле, шаг за
шагом, моделируем ее судьбу на протяжении девяноста дней. На каждом
шестичасовом шаге мы интерполируем для нее поле течений, добавляем
случайность, рассчитываем новое положение и сохраняем его. И так тысячи
частиц, тысячи шагов --- рождается огромный массив данных, трехмерный
куб траекторий, где одно измерение --- это точка выпуска, второе ---
номер частицы, а третье --- временной шаг.

Но сырые координаты сами по себе не являются знанием. Следующая фаза ---
это пост-обработка и анализ, где мы превращаем этот массив чисел в
понятные человеку образы и выводы. Мы преобразуем данные в удобный
табличный формат и начинаем задавать им вопросы. Года находится основная
масса личинок из каждой точки на 30-й день? Мы строим карты плотности
вероятности, подсчитывая, в какие ячейки сетки попало больше всего
частиц. Как далеко они уплывают и как рассеиваются? Мы считаем
статистики --- средние, максимальные дистанции, стандартные отклонения и
площадь занимаемого ими облака. И наконец, мы визуализируем все эти
результаты, создавая серию карт, которые наглядно показывают эволюцию
облака рассеивания во времени, и графиков, показывающих динамику
ключевых параметров. Весь этот путь --- от чтения сырых данных до
итоговых карт вероятности --- и представляет собой замкнутый контур
вероятностного моделирования Лагранжева переноса, который дает нам не
точный прогноз, а наиболее вероятную картину развития событий в океане.

\href{https://mombus.github.io/cRab/data/LarvaeDriting.R}{Скрипт}.

ВХОДНЫЕ ДАННЫЕ (должны быть в рабочей директории), скачанные с сайта
\href{https://www.bio-oracle.org/downloads-to-email.php}{Bio-ORACLE}:

\begin{enumerate}
\def\labelenumi{\arabic{enumi}.}
\tightlist
\item
  Среднее направление течений (градусы) - Current direction
  {[}mean{]}.nc
\item
  Диапазон направления течений (градусы) - Current direction
  {[}range{]}.nc
\item
  Средняя скорость течений (метры в секунду) - Current velocity
  {[}mean{]}.nc
\item
  Диапазон скорости течений (метры в секунду) - Current velocity
  {[}range{]}.nc
\end{enumerate}

\begin{center}
\includegraphics[width=0.7\linewidth,height=\textheight,keepaspectratio]{images/LarvaeRose.PNG}
\end{center}

\bookmarksetup{startatroot}

\chapter{Картография: круговые
диаграммы}\label{ux43aux430ux440ux442ux43eux433ux440ux430ux444ux438ux44f-ux43aux440ux443ux433ux43eux432ux44bux435-ux434ux438ux430ux433ux440ux430ux43cux43cux44b}

\begin{center}
\includegraphics[width=0.75\linewidth,height=\textheight,keepaspectratio]{images/PieMap.PNG}
\end{center}

\section{Введение}\label{ux432ux432ux435ux434ux435ux43dux438ux435-21}

Картография --- это не просто рисование границ и точек, это язык
визуального мышления, который позволяет превращать сырые координаты и
цифры в осмысленные пространственные истории. Когда мы накладываем на
карту круговые диаграммы, мы совершаем важный переход от одномерной
визуализации к многомерной: каждая точка перестаёт быть просто меткой и
становится окном в сложную структуру --- распределение типов, пропорции,
соотношения. Это особенно ценно в гидробиологии и рыбохозяйственной
науке, где важно не только где что поймано, но и в каком соотношении
представлены возрастные группы, половые компоненты или другие категории
улова. Круговая диаграмма на карте --- это компактный, интуитивно
понятный способ показать состав, не загромождая пространство легендами и
таблицами. Она говорит сама за себя: большой сектор --- доминирующая
группа, маленький --- редкий компонент, а размер круга --- общая
численность или интенсивность явления. Это делает такие карты мощным
инструментом для сравнения регионов, выявления пространственных
паттернов и принятия решений --- будь то планирование съёмки, анализ
промысла или оценка состояния запаса.

На этом практическом занятии будет представлен путь от наивного
наложения кругов в градусах до географически корректной, метрически
точной визуализации. Сначала мы столкнулись с тем, что кажется багом, но
на самом деле --- фундаментальное свойство нашего мира: в системе
координат WGS84, где широта и долгота выражены в градусах, один градус
по вертикали и один градус по горизонтали --- это совершенно разные
расстояния, особенно на высоких широтах. В районе Баренцева моря, где
проходят наши исследования, один градус долготы короче градуса широты
почти в три раза --- и потому любой круг, нарисованный в этих единицах,
превращается в вытянутый по вертикали овал, искажая реальную картину.
Это не просто эстетическая проблема --- это искажение восприятия:
пропорции секторов визуально лгут, а размеры объектов теряют смысл. Мы
не можем строить серьёзные выводы на основе визуализаций, которые
геометрически неверны.

Поэтому ключевым шагом стало введение проекции --- не просто технической
трансформации, а смены системы отсчёта. Мы перешли от угловой системы
WGS84 к метрической проекции Lambert Azimuthal Equal Area (EPSG:3575),
специально разработанной для Арктики. Эта проекция сохраняет площади и,
что важно для нас, обеспечивает равенство единиц измерения по осям X и Y
--- один метр есть один метр в любом направлении. Именно это свойство
позволило нам построить настоящие, идеально круглые диаграммы, пропорции
которых отражают реальные соотношения в данных, а не артефакты проекции.
Мы не просто «поправили» форму --- мы перевели всю карту, все точки, все
границы в систему, где геометрия работает честно. Мы научились
переводить не только координаты точек, но и радиусы кругов --- сначала
измеряя эталонный отрезок в градусах, затем переводя его в метры в новой
проекции, чтобы масштаб оставался постоянным. Мы связали размер каждой
круговой диаграммы с общей численностью самцов в точке, сделав
визуализацию не только качественной, но и количественной: большой круг
--- много особей, маленький --- мало. И, наконец,оформили всё это как
настоящую научную карту --- с легендой, разбитой на две строки для
удобства чтения, с чёткой рамкой по краю панели, с подписями осей, чтобы
любой, кто посмотрит на неё, понимал масштаб и контекст. Это занятие ---
про дисциплину восприятия: хорошая карта не та, которая красивая, а та,
в которой каждая линия, каждый цвет, каждый размер несёт проверяемую,
географически обоснованную информацию.

\section{Скрипт и входные
данные}\label{ux441ux43aux440ux438ux43fux442-ux438-ux432ux445ux43eux434ux43dux44bux435-ux434ux430ux43dux43dux44bux435-5}

\href{https://mombus.github.io/cRab/data/15.pie_chart.R}{Скрипт}.

\href{https://mombus.github.io/cRab/data/KARTOGRAPHICpie.xlsx}{Входные
данные} (должны быть в рабочей директории).

\bookmarksetup{startatroot}

\chapter*{References}\label{references}
\addcontentsline{toc}{chapter}{References}

\markboth{References}{References}

\phantomsection\label{refs}




\end{document}
